\part{Calculus II}

\apexchapter{Differentiation Concluded}{chapter:diff_conc}
\section{Derivatives of Inverse Functions}\label{sec:deriv_inverse_function}

%Recall that a function $y=f(x)$ is said to be \textit{one-to-one} if it passes the horizontal line test; that is, for two different $x$ values $x_1$ and $x_2$, we do \textit{not} have $f(x_1)=f(x_2)$. In some cases the domain of $f$ must be restricted so that it is one-to-one. For instance, consider $f(x)=x^2$. Clearly, $f(-1)= f(1)$, so $f$ is not one-to-one on its regular domain, but by restricting $f$ to $(0,\infty)$, $f$ is one-to-one.\index{derivative!inverse function}
%
%Now recall that one-to-one functions have \textit{inverses}. That is, if $f$ is one-to-one, it has an inverse function, denoted by $f\primeskip^{-1}$, such that if $f(a)=b$, then $f\primeskip^{-1}(b) = a$. The domain of $f\primeskip^{-1}$ is the range of $f$, and vice-versa. For ease of notation, we set $g=f\primeskip^{-1}$ and treat $g$ as a function of $x$.
%
%Since $f(a)=b$ implies $g(b)=a$, when we compose $f$ and $g$ we get a nice result: $$f\big(g(b)\big) = f(a) = b.$$ In general, $f\big(g(x)\big) =x$ and $g\big(f(x)\big) = x$. This gives us a convenient way to check if two functions are inverses of each other: compose them and if the result is $x$, then they are inverses (on the appropriate domains.)
%
%When the point $(a,b)$ lies on the graph of $f$, the point $(b,a)$ lies on the graph of $g$. This leads us to discover that the graph of $g$ is the reflection of $f$ across the line $y=x$. In \autoref{fig:inverse1} we see a function graphed along with its inverse. See how the point $(1,1.5)$ lies on one graph, whereas $(1.5,1)$ lies on the other. Because of this relationship, whatever we know about $f$ can quickly be transferred into knowledge about $g$.
%
%\mfigure{-1in}{A function $f$ along with its inverse $f\primeskip^{-1}$. (Note how it does not matter which function we refer to as $f$; the other is $f\primeskip^{-1}$.)}{fig:inverse1}{figures/figinverse1}
%
%For example, consider \autoref{fig:inverse2} where the tangent line to $f$ at the point $(a,b)$ is drawn. That line has slope $\fp(a)$. Through reflection across $y=x$, we can see that the tangent line to $g$ at the point $(b,a)$ should have slope $\ds \frac{1}{\fp(a)}$. This then tells us that $\ds g\primeskip'(b) = \frac{1}{\fp(a)}.$
%
%\mfigure{0in}{Corresponding tangent lines drawn to $f$ and $f\primeskip^{-1}$.}{fig:inverse2}{figures/figinverse2}
%
%Consider:
%\begin{center}
%	\begin{tabular}{ccc}
%	Information about $f$ & & Information about $g=f\primeskip^{-1}$ \\ \hline
%	\parbox{100pt}{\centering $(-0.5,0.375)$ lies on $f$}\rule{0pt}{12pt} & \h skip 40pt & \parbox{100pt}{\centering $(0.375,-0.5)$ lies on $g$}\\
%	\rule{0pt}{20pt}\parbox{100pt}{\centering Slope of tangent line to $f$ at $x=-0.5$ is $3/4$} & & \parbox{100pt}{\centering Slope of tangent line to $g$ at $x=0.375$ is $4/3$}\rule{0pt}{17pt} \\
%	\rule{0pt}{15pt}$\fp(-0.5) = 3/4$ & & $g\primeskip'(0.375) = 4/3$\rule{0pt}{12pt}
%	\end{tabular}
%\end{center}
%
%We have discovered a relationship between $\fp$ and $g\primeskip'$ in a mostly graphical way. We can realize this relationship analytically as well. Let $y = g(x)$, where again $g = f\primeskip^{-1}$. We want to find $\ds y\primeskip'$. Since $y = g(x)$, we know that $f(y) = x$. Using the Chain Rule and Implicit Differentiation, take the derivative of both sides of this last equality.
%		\begin{align*}
%			\frac{d}{dx}\Big(f(y)\Big) &= \frac{d}{dx}\Big(x\Big) \\
%			\fp(y)\cdot y\primeskip' &= 1\\
%			y\primeskip' &= \frac{1}{\fp(y)}\\
%			y\primeskip' &= \frac{1}{\fp(g(x))}
%		\end{align*}
%		
%This leads us to the following theorem.

In this section we will figure out how to differentiate the inverse of a function. To do so, we recall that if $f$ and $g$ are inverses, then $f(g(x))=x$ for all $x$ in the domain of $f$. Differentiating and simplifying yields:
\begin{align*}
f(g(x))&=x\\
\fp(g(x))g\primeskip'(x)&=1\\
g\primeskip'(x)&=\frac 1{\fp(g(x))} \quad\text{assuming $\fp(x)$ is nonzero}
\end{align*}
Note that the derivation above assumes that the function $g$ is differentiable. It is possible to prove that $g$ must be differentiable if $\fp$ is nonzero, but the proof is beyond the scope of this text. However, assuming this fact we have shown the following:

\theorem{thm:deriv_inverse_functions}{Derivatives of Inverse Functions}
{Let $f$ be differentiable and one-to-one on an open interval $I$, where $\fp(x) \neq 0$ for all $x$ in $I$, let $J$ be the range of $f$ on $I$, let $g$ be the inverse function of $f$, and let $f(a) = b$ for some $a$ in $I$. Then $g$ is a differentiable function on $J$, and in particular,
\begin{align*}
 \left(f\primeskip^{-1}\right)'(b) &=g\primeskip'(b) = \frac{1}{\fp(a)} \\
 \left(f\primeskip^{-1}\right)'(x) &=g\primeskip'(x) = \frac{1}{\fp(g(x))}
\end{align*}}

The results of \autoref{thm:deriv_inverse_functions} are not trivial; the notation may seem confusing at first. Careful consideration, along with examples, should earn understanding.

\youtubeVideo{RKfGMX0pn2k}{Derivative of an Inverse Function, Ex 2}

In the next example we apply \autoref{thm:deriv_inverse_functions} to the arcsine function.

\example{ex_deriv_arcsin}{Finding the derivative of an inverse trigonometric function}{Let $y = \sin^{-1} x$. Find $y\primeskip'$ using \autoref{thm:deriv_inverse_functions}.}
{Adopting our previously defined notation, let $g(x) = \sin^{-1} x$ and $f(x) = \sin x$. Thus $\fp(x) = \cos x$. Applying \autoref{thm:deriv_inverse_functions}, we have 
\begin{align*}
	g\primeskip'(x) &= \frac{1}{\fp(g(x))} \\
	&= \frac{1}{\cos(\sin^{-1}x)}.
\end{align*}
			
\mfigure{0in}{A right triangle defined by $y=\sin ^{-1}(x/1)$ with the length of the third leg found using the Pythagorean Theorem.}{fig:inverse3}{figures/figinverse3}

This last expression is not immediately illuminating. Drawing a figure will help, as shown in \autoref{fig:inverse3}. Recall that the sine function can be viewed as taking in an angle and returning a ratio of sides of a right triangle, specifically, the ratio ``opposite over hypotenuse.'' This means that the arcsine function takes as input a ratio of sides and returns an angle. The equation $y=\sin^{-1} x$ can be rewritten as $y=\sin^{-1}(x/1)$; that is, consider a right triangle where the hypotenuse has length 1 and the side opposite of the angle with measure $y$ has length $x$. This means the final side has length $\sqrt{1-x^2}$, using the Pythagorean Theorem.

Therefore $\cos (\sin^{-1} x) = \cos y = \sqrt{1-x^2}/1 = \sqrt{1-x^2}$, resulting in $$\frac d{dx}\big(\sin^{-1}x\big)=g\primeskip'(x)=\frac1{\sqrt{1-x^2}}.\eoehere$$}

Remember that the input $x$ of the arcsine function is a ratio of a side of a right triangle to its hypotenuse; the absolute value of this ratio will be less than 1. Therefore $1-x^2$ will be positive.

\mfigure{0in}{Graphs of $y=\sin x$ and $y=\sin^{-1}x$ along with corresponding tangent lines.}{fig:inverse4}{figures/figinverse4}

In order to make $y=\sin x$ one-to-one, we restrict its domain to $[-\pi/2,\pi/2]$; on this domain, the range is $[-1,1]$. Therefore the domain of $y=\sin^{-1}x$ is $[-1,1]$ and the range is $[-\pi/2,\pi/2]$. When $x=\pm 1$, note how the derivative of the arcsine function is undefined; this corresponds to the fact that as $x\to \pm1$, the tangent lines to arcsine approach vertical lines with undefined slopes.

In \autoref{fig:inverse4} we see $f(x) = \sin x$ and $f\primeskip^{-1}(x)=\sin^{-1} x$ graphed on their respective domains. The line tangent to $\sin x$ at the point $(\pi/3, \sqrt{3}/2)$ has slope $\cos \pi/3 = 1/2$. The slope of the corresponding point on $\sin^{-1}x$, the point $(\sqrt{3}/2,\pi/3)$, is $$\frac{1}{\sqrt{1-(\sqrt{3}/2)^2}} = \frac{1}{\sqrt{1-3/4}} = \frac{1}{\sqrt{1/4}} = \frac{1}{1/2}=2,$$ verifying \autoref{thm:deriv_inverse_functions} yet again: at corresponding points, a function and its inverse have reciprocal slopes.\bigskip

Using similar techniques, we can find the derivatives of all the inverse trig\-o\-no\-metric functions after first restricting their domains according to \autoref{fig:domain_trig} to allow them to be invertible.

\theorem{thm:deriv_inverse_trig}{Derivatives of Inverse Trigonometric Functions}
{The inverse trigonometric functions are differentiable on all open sets contained in their domains (as listed in \autoref{fig:domain_trig}) and their derivatives are as follows:\\
\begin{minipage}{.5\specialboxlength}\small
	\begin{enumerate}
		\item	$\ds\frac d{dx}\big(\sin^{-1}x\big)= \frac1{\sqrt{1-x^2}}$ 
		\item	$\ds\frac d{dx}\big(\sec^{-1}x\big)= \frac1{\abs{x}\sqrt{x^2-1}}$
		\item	$\ds\frac d{dx}\big(\tan^{-1}x\big)= \frac1{1+x^2}$
	\end{enumerate}
	\end{minipage}%
	\begin{minipage}{.5\specialboxlength}\small
	\begin{enumerate}\addtocounter{enumi}{3}
		\item	$\ds\frac d{dx}\big(\cos^{-1}x\big)=-\frac1{\sqrt{1-x^2}}$ 
		\item	$\ds\frac d{dx}\big(\csc^{-1}x\big)=-\frac1{\abs{x}\sqrt{x^2-1}}$
		\item	$\ds\frac d{dx}\big(\cot^{-1}x\big)=-\frac1{1+x^2}$
	\end{enumerate}\index{derivative!inverse trig.}
	\normalsize
\end{minipage}}			

Note how the last three derivatives are merely the negatives of the first three, respectively. Because of this, the first three are used almost exclusively throughout this text.

%In \autoref{sec:basic_diff_rules}, we stated without proof or explanation that $\ds \frac{d}{dx}\bigl(\ln x\bigr) = \frac1x$. We can justify that now using \autoref{thm:deriv_inverse_functions}, as shown in the example.
%
%\example{ex_deriv_lnx}{Finding the derivative of $y=\ln x$}{Use \autoref{thm:deriv_inverse_functions} to compute $\ds \frac{d}{dx}\big(\ln x\big)$.}
%{View $y= \ln x$ as the inverse of $y = e^x$. Therefore, using our standard notation, let $f(x) = e^x$ and $g(x) = \ln x$. We wish to find $g\primeskip'(x)$. \autoref{thm:deriv_inverse_functions} gives:
%\begin{align*}
%	g\primeskip'(x)
%	&= \frac1{\fp(g(x))} \\
%	&= \frac1{e^{\ln x}} \\
%	&= \frac1x.\eoehere
%\end{align*}}

\example{eg_inv_derivs}{Finding derivatives of inverse functions}{Find the derivatives of the following functions:
\[
 \text{1.}\quad f(x)=\cos^{-1}(x^2)\qquad
 \text{2.}\quad g(x)=\frac{\sin^{-1}x}{\sqrt{1-x^2}}\qquad
 \text{3.}\quad f(x)=\sin^{-1}(\cos x)
\]}{\begin{enumerate}
\item We use \autoref{thm:deriv_inverse_trig} and the Chain Rule to find:
% todo Tim show this
\[\fp(x)=-\frac{1}{\sqrt{1-(x^2)^2}}(2x)=- \frac{2x}{\sqrt{1-x^4}}\]
\item We use \autoref{thm:deriv_inverse_trig} and the Quotient Rule to compute: 
\begin{align*}
 g\primeskip'(x)
 &=\frac{\left(\frac 1{\sqrt{1-x^2}}\right)\sqrt{1-x^2} - (\sin^{-1} x)\left( \frac 1{2\sqrt{1-x^2}} (-2x)\right)}{\left(\sqrt{1-x^2}\right)^2}\\
 &=\frac{\sqrt{1-x^2}+x\sin^{-1}x}{\left(\sqrt{1-x^2}\right)^3}
\end{align*}
\item We apply \autoref{thm:deriv_inverse_trig} and the Chain Rule again to compute:
\begin{align*}
 \fp(x)&=\frac{1}{\sqrt{1-\cos^2x}}(-\sin x)\\
 &=\frac{-\sin x}{\sqrt{\sin^2x}}\\
 &=\frac{-\sin x}{\sin x}\\
 &=-1.\eoehere
\end{align*}
\end{enumerate}}

\autoref{thm:deriv_inverse_trig} allows us to integrate some functions that we could not integrate before. For example,
\[\int\frac{dx}{\sqrt{1-x^2}}=\sin^{-1}x+C.\]
Combining these formulas with $u$-substitution yields the following:

\theorem{thm:int_inverse_trig}{Integrals Involving Inverse Trigonometric Functions}
{Let $a>0$.
\begin{enumerate}
	\item	$\ds\int\frac1{a^2+x^2}\ dx=\frac1a\tan^{-1}\left(\frac xa\right) + C$
	\item	$\ds\int\frac1{\sqrt{a^2-x^2}}\ dx=\sin^{-1}\left(\frac xa\right)+C$
	\item	$\ds\int\frac1{x\sqrt{x^2-a^2}}\ dx=\frac1a\sec^{-1}\left(\frac{\abs x}a\right)+C$
\end{enumerate}}

We will look at the second part of this theorem. The other parts are similar and are left as exercises.

First we note that the integrand involves the number $a^2$, but does not explicitly involve $a$. We make the assumption that $a>0$ in order to simplify what follows. We can rewrite the integral as follows:
\[\int\frac{dx}{\sqrt{a^2-x^2}}=\int\frac{dx}{\sqrt{a^2(1-(x/a)^2)}}=\int\frac{dx}{a\sqrt{1-(x/a)^2}}\]
We next use the substitution $u=x/a$ and $du=dx/a$ to find: 
\begin{align*}
\int\frac{dx}{a\sqrt{1-(x/a)^2}}
&=\int\frac{a}{a\sqrt{1-u^2}}\ du\\
&=\int \frac{du}{\sqrt{1-u^2}}\\
&=\sin^{-1}u+C\\
&=\sin^{-1}(x/a)+C
\end{align*}

We conclude this section with several examples.

\example{eg_inv_deriv_harder}{Finding antiderivatives involving inverse functions}{Find the following integrals.
\[
 \text{1.}\quad\int\frac{dx}{100+x^2}\qquad
 \text{2.}\quad\int\frac{\sin^{-1} x}{\sqrt{1-x^2}}\ dx\qquad
 \text{3.}\int\frac{dx}{x^2+2x+5}
\]}{\begin{enumerate}
\item $\ds\int\frac{dx}{100+x^2}=\int\frac{dx}{10^2+x^2}=\frac1{10}\tan^{-1}(x/10)+C$
\item We use the substitution $u=\sin^{-1}x$ and $du=\frac{dx}{\sqrt{1-x^2}}$ to find:
\[\int\frac{\sin^{-1}x}{\sqrt{1-x^2}}=\int u\ du=\frac12 u^2+C=\frac 12\left(\sin^{-1}x\right)^2+C\]
\item This does not immediately look like one of the forms in \autoref{thm:int_inverse_trig}, but we can complete the square in the denominator to see that
\[\int\frac{dx}{x^2+2x+5} =\int\frac{dx}{(x^2+2x+1)+4}=\int\frac{dx}{4+(x+1)^2}\]
We now use the substitution $u=x+1$ and $du=dx$ to find:
\[\int\frac{dx}{4+(x+1)^2} =\int\frac{du}{4+u^2}=\frac12 \tan^{-1}(u/2)+C =\frac 12\tan^{-1}\left(\frac{x+1}2\right)+C.\eoehere\]
\end{enumerate}}

% moved to end of implicit differentiation.  but do we want to give it again?
%In this chapter we have defined the derivative, given rules to facilitate its computation, and given the derivatives of a number of standard functions. We restate the most important of these in the following theorem, intended to be a reference for further work.
%
%\theorem{thm:deriv_glossary}{Glossary of Derivatives of Elementary Functions}
%{Let $u$ and $v$ be differentiable functions, and let $a$, $c$ and $n$ be real numbers, $a>0$, $n\neq 0$. \\
%
%\noindent%
%	\begin{minipage}{.5\specialboxlength}
%	\begin{enumerate}
%	\item		$\frac{d}{dx}\big(cu\big) = cu'$\addtocounter{enumi}{1}
%	\item		$\frac{d}{dx}\big(u\cdot v\big) = uv'+u'v$\addtocounter{enumi}{1}
%	\item		$\frac{d}{dx}\big(u(v)\big) = u'(v)v'$\addtocounter{enumi}{1}
%	\item		$\frac{d}{dx}\big(x\big) = 1$\addtocounter{enumi}{1}
%	\item		$\frac{d}{dx}\big(e^x\big) = e^x$\addtocounter{enumi}{1}
%	\item		$\frac{d}{dx}\big(\ln x\big) = \frac{1}{x}$\addtocounter{enumi}{1}
%	\item		$\frac{d}{dx}\big(\sin x\big) = \cos x$\addtocounter{enumi}{1}
%	\item		$\frac{d}{dx}\big(\csc x\big) = -\csc x\cot x$\addtocounter{enumi}{1}
%	\item		$\frac{d}{dx}\big(\tan x\big) = \sec^2x$\addtocounter{enumi}{1}
%	\item		$\frac{d}{dx}\big(\sin^{-1}x\big) = \frac{1}{\sqrt{1-x^2}}$\addtocounter{enumi}{1}
%	\item		$\frac{d}{dx}\big(\csc^{-1}x\big) = -\frac{1}{\abs{x}\sqrt{x^2-1}}$\addtocounter{enumi}{1}
%	\item		$\frac{d}{dx}\big(\tan^{-1}x\big) = \frac{1}{1+x^2}$\addtocounter{enumi}{1}
%	\end{enumerate}
%\normalsize
%\end{minipage}%
%\begin{minipage}{.5\specialboxlength}
%	\begin{enumerate}\addtocounter{enumi}{1}
%	\item		$\frac{d}{dx}\big(u\pm v\big) = u'\pm v'$\addtocounter{enumi}{1}
%	\item		$\frac{d}{dx}\big(\frac uv\big) = \frac{u'v-uv'}{v^2}$\addtocounter{enumi}{1}
%	\item		$\frac{d}{dx}\big(c\big) = 0$\addtocounter{enumi}{1}
%	\item		$\frac{d}{dx}\big(x^n\big) = nx^{n-1}$\addtocounter{enumi}{1}
%	\item		$\frac{d}{dx}\big(a^x\big) = \ln a\cdot a^x$\addtocounter{enumi}{1}
%	\item		$\frac{d}{dx}\big(\log_a x\big) = \frac{1}{\ln a}\cdot\frac{1}{x}$\addtocounter{enumi}{1}
%	\item		$\frac{d}{dx}\big(\cos x\big) = -\sin x$\addtocounter{enumi}{1}
%	\item		$\frac{d}{dx}\big(\sec x\big) = \sec x\tan x$\addtocounter{enumi}{1}
%	\item		$\frac{d}{dx}\big(\cot x\big) = -\csc^2x$\addtocounter{enumi}{1}
%	\item		$\frac{d}{dx}\big(\cos^{-1}x\big) = -\frac{1}{\sqrt{1-x^2}}$\addtocounter{enumi}{1}
%	\item		$\frac{d}{dx}\big(\sec^{-1}x\big) = \frac{1}{\abs{x}\sqrt{x^2-1}}$\addtocounter{enumi}{1}
%	\item		$\frac{d}{dx}\big(\cot^{-1}x\big) = -\frac{1}{1+x^2}$
%	\end{enumerate}
%\normalsize
%\end{minipage}
%}

\printexercises{exercises/02_07_exercises}

%\section{Exponential Functions and Their Derivatives}

Exponentials functions are functions of the form $f(x)=b^x$. Here
the positive constant $b$ is called the \textbf{base}. We restrict
ourselves to positive bases since if the base where negative, the
definition of some powers of the base, such as $(-1)^{\frac12}$, would involve
imaginary numbers, and we want to stick with real numbers in Calculus. 

%\begin{wrapfigure}{r}{0pt}
%\includegraphics{images/7_1_exponential}
%\end{wrapfigure}
You should recognize the characteristic shape of the graph of
exponential functions $f(x)=b^x$. They all have a \emph{swoopy} sort of
look that you probably picture when you hear someone use the phrase \emph{such and such is growing exponentially} (for example, population is
growing exponentially, or the national debt is growing exponentially).

Make sure that you are familiar with the laws of exponents presented here:
\begin{align*}
b^0&=1 & (b^x)^y&=b^{xy} & b^{1/n}&=\sqrt[n]{b}\\
b^x b^y&=b^{x+y} & \frac{b^x}{b^y}&=b^{x-y} & b^{-x}&=\frac1{b^x}.
\end{align*}
If you can't work comfortably with these rules, this chapter will be quite
difficult.
The same can be said of logarithmic function, which we will see shortly.

Actually, the \emph{swoopy up} picture is correct only if the constant $b$
is greater than $1$. Of course, if $b=1$, then $b^x=1$ for all $x$, and
so the graph is just a horizontal line, and if $0<b<1$, then the graph
swoops down, approaching the positive $x$-axis as a horizontal
asymptote.

Now for some calculus of the exponential functions. What is the
slope of the tangent line to the graph of the exponential function
$f(x)=b^x$ at some point $(x,b^x)$ on the curve? The answer, of course, using the
methods we learned in Calculus I, 
is
\[f'(x)=\lim_{h\to0}\frac{f(x+h)-f(x)}h=\lim_{h\to0}\frac{b^{x+h}-b^x}h.\]

Just to practice a bit, let's compute the slope of the tangent line at
$x=0$. That would be given by
\[f'(0)=\lim_{h\to0}\frac{f(0+h)-f(0)}h=\lim_{h\to0}\frac{b^h-1}h.\]

Well, that certainly looks like a tough limit to compute. So it seems hard
even to find the slope of the tangent line at $x=0$. How about the derivative in
general? In other words, suppose we wanted to compute
\[f'(x)=\lim_{h\to0}\frac{f(x+h)-f(x)}h=\lim_{h\to0}\frac{b^{x+h}-b^x}h.\]

To make a little more progress with this limit, we will make use of
a property of exponents: $b^{t+u}=b^tb^u$. This, and other of the
familiar properties you are used to for manipulating exponents can all
be proved as theorems based on the definition of $b^x$ as a limit. In
any case, using the property mentioned above, we find
\[
 f'(x)=\lim_{h\to0}\frac{b^{x+h}-b^x}h
 =\lim_{h\to0}\frac{b^xb^h-b^x}h
 =\lim_{h\to0}b^x\frac{b^h-1}h
 =b^x\lim_{h\to0}\frac{b^h-1}h.
\]
Since the limit on the right is the formula for $f'(0)$, 
we can write that, for every $x$, $f'(x)=f'(0)b^x$, and
so to compute $f'(x)$, all we really need to do is determine the
value of $f'(0)$.

Unfortunately, for most choices of $b$, the value of the limit
\[\lim_{h\to0}\frac{b^h-1}h\]
is a number that is difficult to determine, and not very neat anyway. For example,
when $b=10$ the value of the limit turns out to be (after a lot of work)
$2.302585093\cdots$. Not a very nice number, and certainly a number that you don't
want to have to memorize to use in computations. That's why the exponential
function $y=10^x$ is not very popular in mathematics courses.
(We'll find out in the next section that the limit is $\ln b$.)

However, as demonstrated in the text on page 346, it seems reasonable that
there is a 
choice for the base $a$ somewhere between $2$ and $3$ for which
$f'(0)=1$. This number is denoted by $e$ and it is called the
base for the \textbf{natural exponential function}. It is an irrational
number; its approximate value is $2.718281828459\cdots$, but just call it $e$ for
short, and remember that it is about $2.7$. If the phrase
\emph{exponential function} is used with no further explanation, it is
safe to assume that $f(x)=e^x$ is the one that is meant.

The important thing to keep in mind is that since $f'(0) = 1$ when
$f(x)=e^x$, the computations above 
show
\[f'(x)=\left(e^x\right)'=e^xf'(0)=e^x\cdot1=e^x.\]
In other words, we get the important formula
\[\left(e^x\right)'=e^x.\]
If someone refers to ``the'' exponential function, this is the one that they
mean (and this is the reason why this base is the best choice).
If someone only refers to some exponential function, they mean $b^x$ for some
$b$ (although we can use logarithms to rewrite this as $e^{(\ln b)x}$).

Every time a new differentiation formula is derived, a free integration formula comes along for the 
ride:
\[\int e^{x} dx=e^x+C.\]
And of course, we can immediately write down more general versions of those formulas:
\begin{align}
\left(e^{mx+b}\right)&=m e^{mx+b}\\
\int e^{mx+b}dx&=\frac1m e^{mx+b}+C.
\end{align}
%\equDesc{Derivative and integral of the natural exponential function}

Exponential functions grow extremely quickly.
The function $1.1^x$ is eventually bigger than $x^{100}$, once we take $x$
large enough (we need $x\ge9624$ in this case).

\subsection*{An Optional Side Note}

The value of $b^x$ when $x$ is a rational number (fraction) is
reasonably obvious. For example $3^{\frac35}$, means cube $3$ to get
$27$, and now compute the $5$-th root of $27$, which we can do as
accurately as we wish by a variety of algebraic techniques. However, it
is not at all clear how we would ever compute $3^{\sqrt2}$. We
\textbf{define} $b^x$, when $x$ is irrational, by the following rule:
\[b^x=\lim_{r\to x}b^x,\qquad r\text{ rational.}\]
So we base (pun intended!) the definition of $b^x$ for irrational $x$ on the sorts
of powers we understand, namely \textbf{rational} powers of $b$.

%\section{Logarithmic Functions}

The \textbf{logarithm base b}, written as $\log_b$, is defined to
be the function inverse of $f(x)=b^x$. In other words, if you are asked
to compute $f(x) = b^x$, you need answer the question \emph{what do I get
when I raise the number $b$ to the power $x$}. On the other hand, if
you are asked to compute $g(x)=\log_b x$, you need to answer the
question \emph{to what power do I need to raise $b$ in order to get the result $x$}.
You can see this is just the inverse question of the first one. For
example, $\log_3 81 = 4$, since we need to raise $3$ to the $4^\text{th}$
power to get the result $81$.

The most important thing for you to keep in mind is that the
equations
\[a=b^c\qquad\text{and}\qquad c=\log_b a\]
say exactly the same thing. When you see one of these two equations, you
can replace it with the other. Remember that; it is a very useful
tool.

Just as $e$ is used as the standard base for the exponential
function, logarithms base $e$ are the most frequently seen logarithms.
Logarithms base $e$ are called \textbf{natural logarithms}, and instead of
writing $\log_e x$, the usual symbol is $\ln x$.  That  is pronounced 
\emph{ell-en of $x$}

Historically, logarithms were important tools for carrying out
complicated calculations. They are not used for that purpose much any
more, but they have many other important applications in science and
mathematics today. There are a number of properties of logarithms that
made them useful for computational problems. Essentially, logarithms
convert a messy multiplication problem to a less messy addition problem.
As an equation, this is expressed as
\[\log(st) = \log s + \log t.\]
In other words, \emph{the log of a product is the sum of the logs}.
The particular base used makes no difference, so I left it unspecified.
Two related properties of the logarithm functions are
\[
 \log\left(\frac st\right) = \log s - \log t,
 \qquad\text{and}\qquad
 \log \left(s^t\right) = t\log s.
\]
All three of these facts are easy to prove using the corresponding
properties of the exponential functions mentioned in Section 7.1. You
should see if you can work out the verifications.

You'll notice that your calculator can directly compute only
logarithms base $10$ (usually a key labeled $\log$), and base $e$
(usually a key labeled $\ln$). You can convert logarithms in any base to
natural logarithms, and, since calculus is much neater using the natural
base, you should always do so. The conversion is easy. Suppose we
have to deal with the number
\[a = \log_b x.\]
Then we know that $b^a=x$.
Taking natural logs of both sides of that last equation, we find
$\ln b^a=\ln x$,
and making use of one of the properties of logs, we
get $a\ln b = \ln x$,
or $a=\dfrac{\ln x}{\ln b}.$
In other words, we
can replace 
\[\log_b x\qquad\text{with}\qquad\frac{\ln x}{\ln b}.\]

%\section{Derivatives of Logarithmic Functions}

The derivation of the formula for the derivative of $y=\ln x$
%given on the bottom of page 360
is a quick exercise in implicit
differentiation. The result is $(\ln x)'=\frac1x$.
When that is teamed up with the chain rule,
you can now differentiate some pretty
spectacular functions without much effort. Imagine trying to find the
derivative of $f(x) = \ln(\sec^2x)$ by taking a limit of a difference
quotient! But, using the chain rule, we quickly find
\[
 \left(\ln(\sec^2x)\right)'=\frac1{\sec^2x}\cdot2(\sec x)\cdot\sec x\tan x
 =2\tan x.
\]
Now that was pretty easy!

Recall that $\ln x$ is defined only for positive values of $x$.
The function $f(x)=\ln\lvert x\rvert$ on the other hand, is defined for \textbf{all}
non-zero values of $x$. The graph of this function is, of course,
identical to the graph of $y=\ln x$ on the right side of the $y$-axis,
and is the mirror image of that on the left side of the $y$-axis.
Another way to express the function $f$ is as
\[
 f(x)=\begin{cases}
  \ln x, &\text{ if }x>0,\\
  \ln (-x), &\text{ if }x<0.
 \end{cases}
\]
Differentiating that, and remembering to use the chain rule in the second
line, we see
\[
 f'(x)=\begin{cases}
  \frac1x, &\text{ if }x>0,\\
  \frac1{-x}\cdot(-1)=\frac1x, &\text{ if }x<0.
 \end{cases}
\]
Thus, for all non-zero $x$,
\begin{equation}
 (\ln\lvert x\rvert)'=\frac1x\qquad\text{and}\qquad\int\frac1x dx=\ln\lvert x\rvert+C.
\end{equation}
%\equDesc{Derivative and integral with the natural logarithmic function}

Finding derivatives of logarithms to bases besides $e$ (that is,
besides the natural logarithm) is best done by converting all logs to
natural logs. Recall that
$\log_b x=\dfrac{\ln x}{\ln b}=\dfrac1{\ln b}\cdot\ln x$.
Thus
\begin{equation}(\log_b x)'=\frac1{\ln b}\frac1x=\frac1{x\ln b}.\end{equation}
%\equDesc{Derivative of logarithmic function in other bases}

Another important differentiation formula we can now derive is for
the general exponential functions. Of course we already know
$(e^x)'=e^x.$ But what is the derivative of $2^x$ for example?
Well, for any positive value of $b$ notice that
\[e^{x\ln b}=\left(e^{\ln b}\right)^x = b^x,\]
where we used $e^{\ln b}=b$, which is a consequence of the fact that
the natural exponential function and the natural logarithm are
functional inverses for each other.
Thus we see the useful formula
\[b^x=e^{x\ln b}.\]
In other words, there is really no reason to use bases in exponential
functions other than base $e$ since all other bases can be replaced by
$e$ (by changing the exponent appropriately of course). This idea makes
it a snap to compute the derivative for exponentials using bases
besides $e$. Thus
\begin{equation}
 (b^x)'=\left(e^{x\ln b}\right)'=e^{x\ln b}\ln b=b^x\ln b,
\end{equation}
%\equDesc{Derivative of exponential function with other bases}
where we had to remember to use the chain rule.

If you encounter a derivative of complicated products, quotients or powers,
you may want to try the algebra saving technique of logarithmic differentiation.
Since $(\ln f(x))'=\frac{f'(x)}{f(x)}$, we find that
$f'(x)=f(x)(\ln f(x))'$.  For example, if we wanted to find the derivative of
$x^x$, we could take its logarithm to get $x\ln x$, differentiate this to get
$1+\ln x$, and then multiply by the original $x^x$ to get that the derivative is
$x^x(1+\ln x)$.
If we didn't want to use logarithmic differentiation, we would have to
write $x^x=e^{x\ln x}$, and then take the derivative to get
$e^{x\ln x}(x\ln x)'=e^{x\ln x}(1+\ln x)=x^x(1+\ln x)$.
This technique works best when the logarithm is much simpler than the original
function.

\subsection*{An Optional Approach}
There are two basic approaches to the previous three sections. Our approach was
\begin{enumerate}
 \item define $a^x$ for constant $a$,
 \item decide that $a=e$ is the best constant,
 \item define $\log_a x$ to be the inverse of $a^x$, and
 \item find the derivative of $\log_a x$ using the theorem from
  Section \ref{sec:deriv_inverse_function}.
\end{enumerate}
A different approach is:
\begin{enumerate}
 \item define $\ln x=\int_1^x\frac1x dx$,
 \item define $e^x$ to be the inverse of $\ln x$,
 \item find the derivative of $e^x$ using the theorem from
  Section \ref{sec:deriv_inverse_function}
 \item define $a^x=e^{x\ln a}$, and
 \item define $\log_a x$ to be the inverse of $a^x$.
\end{enumerate}
Both approaches have their merits.  The first approach forces us to define $a^x$ as a limiting process.  The second approach forces us to define a commonly used function as a definite integral.

\section{Hyperbolic Functions}\label{sec:hyperbolic}

The \textbf{hyperbolic functions} are functions that have many applications to mathematics, physics, and engineering. Among many other applications, they are used to describe the formation of satellite rings around planets, to describe the shape of a rope hanging from two points, and have application to the theory of special relativity. This section defines the hyperbolic functions and describes many of their properties, especially their usefulness to calculus.

\mtable{Using trigonometric functions to define points on a circle and hyperbolic functions to define points on a hyperbola.}{fig:hfcircle}{\begin{tikzpicture}
 \begin{axis}[height=\marginparwidth,width=\marginparwidth,
   tick label style={font=\scriptsize},
   axis y line=middle,axis x line=middle,name=myplot,axis on top,
   xtick={-1,1},ytick={-1,1},ymin=-1.1,ymax=1.1,xmin=-1.1,xmax=1.1]
  \addplot [draw={\coloronefill},fill={\coloronefill},domain=0:45] ({cos(x)},{sin(x)})
   -- (axis cs:0,0)--cycle;
  \addplot [draw={\colorone},domain=0:360,thick,smooth,samples=40] ({cos(x)},{sin(x)});
  \filldraw (axis cs:.707,.707) circle (1pt) node [left]%[shift={(4pt,11pt)}]
   {\scriptsize ($\cos \theta$,$\sin \theta$)};
  \draw (axis cs:.6,.25) node {\scriptsize $A=\dfrac{\theta}{2}$};
  \draw (axis cs:-.5,.2) node {\scriptsize $x^2+y^2=1$};
 \end{axis}
% \draw[draw={\colorone}](0,0)circle(1);
 \node [right] at (myplot.right of origin) {\scriptsize $x$};
 \node [above] at (myplot.above origin) {\scriptsize $y$};
\end{tikzpicture}
\vspace{10pt}
\begin{tikzpicture}
 \begin{axis}[height=\marginparwidth,width=\marginparwidth,
   tick label style={font=\scriptsize},
   axis y line=middle,axis x line=middle,name=myplot,axis on top,
   ymin=-3.1,ymax=3.1,xmin=-3.1,xmax=3.1]
  \addplot [draw={\coloronefill},fill={\coloronefill},domain=0:1.6] ({cosh(x)},{sinh(x)})
   -- (axis cs:0,0)--cycle;
  \addplot [draw={\colorone},domain=-2:2,thick,smooth] ({cosh(x)},{sinh(x)});
  \addplot [draw={\colorone},domain=-2:2,thick,smooth] ({-cosh(x)},{sinh(x)});
  \filldraw (axis cs:2.577,2.376) circle (1pt) node [left]
   {\scriptsize ($\cosh \theta$,$\sinh \theta$)};
  \draw (axis cs:2,.6) node {\scriptsize $A=\dfrac{\theta}{2}$};
  \draw (axis cs:-1.75,2.75) node {\scriptsize $x^2-y^2=1$};
 \end{axis}
 \node [right] at (myplot.right of origin) {\scriptsize $x$};
 \node [above] at (myplot.above origin) {\scriptsize $y$};
\end{tikzpicture}}

These functions are sometimes referred to as the ``hyperbolic trigonometric functions'' as there are many connections between them and the standard trigonometric functions. \autoref{fig:hfcircle} demonstrates one such connection. Just as cosine and sine are used to define points on the circle defined by $x^2+y^2=1$, the functions \textbf{hyperbolic cosine} and \textbf{hyperbolic sine} are used to define points on the hyperbola $x^2-y^2=1$.

We begin with their definitions.

\definition{def:hyperbolic_functions}{Hyperbolic Functions}
{\noindent%
\begin{minipage}{.5\specialboxlength}\index{hyperbolic function!definition}%
\begin{enumerate}
\item		$\ds \cosh x = \frac{e^x+e^{-x}}2$
\item		$\ds \sinh x = \frac{e^x-e^{-x}}2$
\item		$\ds \tanh x = \frac{\sinh x}{\cosh x}$
\end{enumerate}
\end{minipage}%
\begin{minipage}{.5\specialboxlength}
\begin{enumerate}\addtocounter{enumi}{3}
\item		$\ds \sech x = \frac{1}{\cosh x}$
\item		$\ds \csch x = \frac{1}{\sinh x}$
\item		$\ds \coth x = \frac{\cosh x}{\sinh x}$
\end{enumerate}
\end{minipage}}

The hyperbolic functions are graphed in \autoref{fig:hyperbolic}. In the graphs of $\cosh x$ and $\sinh x$, graphs of $e^x/2$ and $e^{-x}/2$ are included with dashed lines. As $x$ gets ``large,'' $\cosh x$ and $\sinh x$ each act like $e^x/2$; when $x$ is a large negative number, $\cosh x$ acts like $e^{-x}/2$ whereas $\sinh x$ acts like $-e^{-x}/2$.

\mnote{\textbf{Pronunciation Note:} \\
``cosh'' rhymes with ``gosh,'' \\
``sinh'' rhymes with ``pinch,'' and\\
``tanh'' rhymes with ``ranch,'' \\
%``sech'' rhymes with ``fetch,'' and \\
%``coth'' rhymes with ``moth.''
}

\begin{figure}[!ht]
\centering
\addtolength{\tabcolsep}{6pt}
\begin{tabular}{cc}
\begin{tikzpicture}
\begin{axis}[width=1.16\marginparwidth,tick label style={font=\scriptsize},
axis y line=middle,axis x line=middle,name=myplot,axis on top,ymin=-10.9,ymax=10.9,
xmin=-3.5,xmax=3.5,scaled ticks=false]
\addplot [draw={\colorone},thick,smooth,domain=-3:3] {cosh(x)};
\draw (axis cs:2,-6) node {\scriptsize $f(x)=\cosh x$};
\addplot [draw={\colortwo},smooth,thick,dashed,domain=-3:3] {exp(x)/2};
\addplot [draw={\colortwo},smooth,thick,dashed,domain=-3:3] {exp(-x)/2};
\end{axis}
\node [right] at (myplot.right of origin) {\scriptsize $x$};
\node [above] at (myplot.above origin) {\scriptsize $y$};
\end{tikzpicture}
&
\begin{tikzpicture}
\begin{axis}[width=1.16\marginparwidth,tick label style={font=\scriptsize},
axis y line=middle,axis x line=middle,name=myplot,axis on top,ymin=-10.9,ymax=10.9,
xmin=-3.5,xmax=3.5,scaled ticks=false]
\addplot [draw={\colorone},thick,smooth,domain=-3:3] {sinh(x)};
\draw (axis cs:2,-6) node {\scriptsize $f(x)=\sinh x$};
\addplot [draw={\colortwo},smooth,thick,dashed,domain=-3:3] {exp(x)/2};
\addplot [draw={\colortwo},smooth,thick,dashed,domain=-3:3] {exp(-x)/2};
\end{axis}
\node [right] at (myplot.right of origin) {\scriptsize $x$};
\node [above] at (myplot.above origin) {\scriptsize $y$};
\end{tikzpicture}
\\[20pt]
\begin{tikzpicture}
\begin{axis}[width=1.16\marginparwidth,tick label style={font=\scriptsize},
axis y line=middle,axis x line=middle,name=myplot,axis on top,ytick={-2,2},
ymin=-3.5,ymax=3.5,xmin=-3.5,xmax=3.5,scaled ticks=false]
\addplot [draw={\colorone},thick,smooth,domain=-3:3] {tanh(x)};
\draw (axis cs:1.7,-1.5) node {\scriptsize $f(x)=\tanh x$};
\draw (axis cs:2.2,2) node {\scriptsize $f(x)=\coth x$};
\draw [->,>=stealth] (axis cs:1,-1) -- (axis cs:.6,.45);
\addplot [draw={\colortwo},smooth,thick,domain=-3:-.1] {1/tanh(x)};
\addplot [draw={\colortwo},smooth,thick,domain=.1:3] {1/tanh(x)};
\draw [loosely dashed] (axis cs:-3,1) -- (axis cs:3,1);
\draw [loosely dashed] (axis cs:-3,-1) -- (axis cs:3,-1);
\end{axis}
\node [right] at (myplot.right of origin) {\scriptsize $x$};
\node [above] at (myplot.above origin) {\scriptsize $y$};
\end{tikzpicture}
&
\begin{tikzpicture}
\begin{axis}[width=1.16\marginparwidth,tick label style={font=\scriptsize},
axis y line=middle,axis x line=middle,name=myplot,axis on top,ytick={-3,-2,-1,1,2,3},
ymin=-3.5,ymax=3.5,xmin=-3.5,xmax=3.5,scaled ticks=false]
\addplot [draw={\colorone},thick,smooth,domain=-3:3] {1/cosh(x)};
\draw (axis cs:-2,1.5) node {\scriptsize $f(x)=\sech x$};
\draw (axis cs:2.2,2) node {\scriptsize $f(x)=\csch x$};
\addplot [draw={\colortwo},smooth,thick,domain=-3:-.1] {1/sinh(x)};
\addplot [draw={\colortwo},smooth,thick,domain=.1:3] {1/sinh(x)};
\end{axis}
\node [right] at (myplot.right of origin) {\scriptsize $x$};
\node [above] at (myplot.above origin) {\scriptsize $y$};
\end{tikzpicture}
\end{tabular}
\captionsetup{type=figure}
\caption{Graphs of the hyperbolic functions.}
\label{fig:hyperbolic}
\end{figure}
\bigskip

Notice the domains of $\tanh x$ and $\sech x$ are $(-\infty,\infty)$, whereas both $\coth x$ and $\csch x$ have vertical asymptotes at $x=0$. Also note the ranges of these functions, especially $\tanh x$: as $x\to\infty$, both $\sinh x$ and $\cosh x$ approach $e^{-x}/2$, hence $\tanh x$ approaches $1$.

\youtubeVideo{G1C1Z5aTZSQ}{Hyperbolic Functions --- The Basics}

%It is no coincidence that these functions share a name similar to the trigonometric functions. 
The following example explores some of the properties of these functions that bear remarkable resemblance to the properties of their trigonometric counterparts.\\

\example{ex_hf1}{Exploring properties of hyperbolic functions}{Use \autoref{def:hyperbolic_functions} to rewrite the following expressions.

\noindent\begin{minipage}{.5\linewidth}
\begin{enumerate}
\item		$\cosh^2 x-\sinh^2x$
\item		$\tanh^2 x+\sech^2 x$
\item		$2\cosh x\sinh x$
\end{enumerate}
\end{minipage}%
\begin{minipage}{.5\linewidth}
\begin{enumerate}\addtocounter{enumi}{3}
\item		$\frac{d}{dx}\big(\cosh x\big)$
\item		$\frac{d}{dx}\big(\sinh x\big)$
\item		$\frac{d}{dx}\big(\tanh x\big)$
\end{enumerate}
\end{minipage}
}
{\begin{enumerate}
\item \mbox{}\\[-3\baselineskip]
\begin{align*}
 \cosh^2x-\sinh^2x
 &= \left(\frac{e^x+e^{-x}}2\right)^2 -\left(\frac{e^x-e^{-x}}2\right)^2\\
 &= \frac{e^{2x}+2e^xe^{-x} + e^{-2x}}4 - \frac{e^{2x}-2e^xe^{-x} + e^{-2x}}4\\
 &= \frac44=1.
\end{align*}
So $\cosh^2 x-\sinh^2x=1$.

\item	\mbox{}\\[-3\baselineskip]
\begin{align*}
 \tanh^2 x+\sech^2 x
 &=\frac{\sinh^2x}{\cosh^2 x} + \frac{1}{\cosh^2 x} \\
 &= \frac{\sinh^2x+1}{\cosh^2 x}\qquad \text{\small Now use identity from \#1.}\\
 &= \frac{\cosh^2 x}{\cosh^2 x} = 1.
\end{align*}
So $\tanh^2 x+\sech^2 x=1$.

\item \mbox{}\\[-3\baselineskip]
\begin{align*}
	2\cosh x\sinh x
	&= 2\left(\frac{e^x+e^{-x}}2\right)\left(\frac{e^x-e^{-x}}2\right) \\
	&= 2 \cdot\frac{e^{2x} - e^{-2x}}4\\
	&= \frac{e^{2x} - e^{-2x}}2 = \sinh (2x).\\
\end{align*}
Thus $2\cosh x\sinh x = \sinh (2x)$.

\item \mbox{}\\[-3\baselineskip]
\begin{align*}
	\frac{d}{dx}\big(\cosh x\big)
	&= \frac{d}{dx}\left(\frac{e^x+e^{-x}}2\right) \\
	&= \frac{e^x-e^{-x}}2\\
	&= \sinh x.
\end{align*}
So $\frac{d}{dx}\big(\cosh x\big) = \sinh x.$
	
\item  \mbox{}\\[-3\baselineskip]
\begin{align*}
	\frac{d}{dx}\big(\sinh x\big)
	&= \frac{d}{dx}\left(\frac{e^x-e^{-x}}2\right) \\
	&= \frac{e^x+e^{-x}}2\\
	&= \cosh x.
\end{align*}
So $\frac{d}{dx}\big(\sinh x\big) = \cosh x.$

\item  \mbox{}\\[-3\baselineskip]
\begin{align*}
	\frac{d}{dx}\big(\tanh x\big)
	&= \frac{d}{dx}\left(\frac{\sinh x}{\cosh x}\right) \\
	&= \frac{\cosh x \cosh x - \sinh x \sinh x}{\cosh^2 x}\\
	&= \frac{1}{\cosh^2 x}\\
	&= \sech^2 x.
\end{align*}
So $\frac{d}{dx}\big(\tanh x\big) = \sech^2 x.$\eoehere
\end{enumerate}}

The following Key Idea summarizes many of the important identities relating to hyperbolic functions. Each can be verified by referring back to \autoref{def:hyperbolic_functions}.

\setboxwidth{160pt}
\keyidea{idea:hyperbolic_identities}{Useful Hyperbolic Function Properties}
{\begin{minipage}[t]{.33\specialboxlength}
\textbf{Basic Identities}\par
\index{hyperbolic function!identities}\index{hyperbolic function!derivatives}\index{hyperbolic function!integrals}\index{derivative!hyperbolic funct.}\index{integration!hyperbolic funct.}%
\begin{enumerate}
\item $\cosh^2x-\sinh^2x=1$
\item	$\tanh^2x+\sech^2x=1$
\item	$\coth^2x-\csch^2x = 1$
\item	$\cosh 2x=\cosh^2x+\sinh^2x$
\item	$\sinh 2x = 2\sinh x\cosh x$
\item	$\ds\cosh^2x = \frac{\cosh 2x+1}{2}$
\item $\ds \sinh^2x=\frac{\cosh 2x-1}{2}$
\end{enumerate}
\end{minipage}%
\begin{minipage}[t]{.33\specialboxlength}
\textbf{Derivatives}
\begin{enumerate}
\item $\frac{d}{dx}\big(\cosh x\big) = \sinh x$
\item $\frac{d}{dx}\big(\sinh x\big) = \cosh x$
\item $\frac{d}{dx}\big(\tanh x\big) = \sech^2 x$
\item $\frac{d}{dx}\big(\sech x\big) = -\sech x\tanh x$
\item $\frac{d}{dx}\big(\csch x\big) = -\csch x\coth x$
\item $\frac{d}{dx}\big(\coth x\big) = -\csch^2x$
\end{enumerate}
\end{minipage}%
\begin{minipage}[t]{.33\specialboxlength}
\textbf{Integrals}
\begin{enumerate}
\item $\ds\int \cosh x\ dx = \sinh x+C$
\item $\ds\int \sinh x\ dx = \cosh x+C$
\item $\ds\int \tanh x\ dx = \ln(\cosh x) +C$
\item $\ds\int \coth x\ dx = \ln\abs{\sinh x}+C$
\end{enumerate}
\end{minipage}}

We practice using \autoref{idea:hyperbolic_identities}.

\example{ex_hf2}{Derivatives and integrals of hyperbolic functions}{Evaluate the following derivatives and integrals.\\
\begin{minipage}[t]{.5\linewidth}
\begin{enumerate}
\item		$\ds\frac{d}{dx}\big(\cosh 2x\big)$
\item		$\ds\int \sech^2(7t-3)\ dt$
\end{enumerate}
\end{minipage}%
\begin{minipage}[t]{.5\linewidth}
\begin{enumerate}\addtocounter{enumi}{2}
\item		$\ds \int_0^{\ln 2} \cosh x\ dx$
\end{enumerate}
\end{minipage}}
{\begin{enumerate}
\item		Using the Chain Rule directly, we have $\frac{d}{dx} \big(\cosh 2x\big) = 2\sinh 2x$.

Just to demonstrate that it works, let's also use the Basic Identity found in \autoref{idea:hyperbolic_identities}: $\cosh 2x = \cosh^2x+\sinh^2x$.
\begin{align*}\frac{d}{dx}\big(\cosh 2x\big) = \frac{d}{dx}\big(\cosh^2x+\sinh^2x\big) &= 2\cosh x\sinh x+ 2\sinh x\cosh x\\ &= 4\cosh x\sinh x.
\end{align*}
Using another Basic Identity, we can see that $4\cosh x\sinh x = 2\sinh 2x$. We get the same answer either way.

\item	  We employ substitution, with $u = 7t-3$ and $du = 7dt$. Applying
\autoref{idea:hyperbolic_identities} we have:
\[\int \sech^2 (7t-3)\ dt = \frac17\tanh (7t-3) + C.\]

\item		\[\int_0^{\ln 2} \cosh x\ dx = \sinh x\Big|_0^{\ln 2} = \sinh (\ln 2) - \sinh 0 = \sinh(\ln 2).\]

We can simplify this last expression as $\sinh x$ is based on exponentials:
\[\sinh(\ln 2) = \frac{e^{\ln 2}-e^{-\ln 2}}2 = \frac{2-1/2}{2} = \frac34.\eoehere\]
\end{enumerate}}

\subsection{Inverse Hyperbolic Functions}

Just as the inverse trigonometric functions are useful in certain integrations, the inverse hyperbolic functions are useful with others. \autoref{fig:hfinverse2} shows the restrictions on the domains to make each function one-to-one and the resulting domains and ranges of their inverse functions. Their graphs are shown in \autoref{fig:hfinverse1}.\index{hyperbolic function!inverse}

Because the hyperbolic functions are defined in terms of exponential functions, their inverses can be expressed in terms of logarithms as shown in \autoref{idea:hyperbolic_log}. It is often more convenient to refer to $\sinh^{-1}x$ than to $\ln\big(x+\sqrt{x^2+1}\big)$, especially when one is working on theory and does not need to compute actual values. On the other hand, when computations are needed, technology is often helpful but many hand-held calculators lack a \textit{convenient} $\sinh^{-1}x$ button. (Often it can be accessed under a menu system, but not conveniently.) In such a situation, the logarithmic representation is useful. The reader is not encouraged to memorize these, but rather know they exist and know how to use them when needed.

\enlargethispage{2\baselineskip}

\begin{lxfigure}
\flushinner{%
\small
\begin{tabular}{lcc @{\hspace{2em}} ccc}
Function & Domain & Range & Function & Domain & Range \\ \cmidrule(r{2em}){1-3} \cmidrule(l{-1em}){4-6}
$\cosh x$ & $[0,\infty)$ & $[1,\infty)$ &
 $\cosh^{-1} x$ & $[1,\infty)$ & $[0,\infty)$ \\
$\sinh x$ & $(-\infty,\infty)$ & $(-\infty,\infty)$ &
 $\sinh^{-1} x$ & $(-\infty,\infty)$ & $(-\infty,\infty)$\\
$\tanh x$ & $(-\infty,\infty)$ & $(-1,1)$ &
 $\tanh^{-1} x$ & $(-1,1)$ & $(-\infty,\infty)$\\
$\sech x$ & $[0,\infty)$ & $(0,1]$ & $\sech^{-1} x$ & $(0,1]$ & $[0,\infty)$\\
$\csch x$ & $(-\infty,0) \cup (0,\infty)$ & $(-\infty,0) \cup (0,\infty)$ &
 $\csch^{-1} x$ & $(-\infty,0) \cup (0,\infty)$ & $(-\infty,0) \cup (0,\infty)$\\
$\coth x$ & $(-\infty,0) \cup (0,\infty)$ & $(-\infty,-1) \cup (1,\infty)$ &
 $\coth^{-1} x$ & $(-\infty,-1) \cup (1,\infty)$ & $(-\infty,0) \cup (0,\infty)$
\end{tabular}}
\caption{Domains and ranges of the hyperbolic and inverse hyperbolic functions.}
\label{fig:hfinverse2}
\end{lxfigure}

\begin{lxfigure}
\flushinner{%
\addtolength{\tabcolsep}{6pt}
\begin{tabular}{cc}
\begin{tikzpicture}
\begin{axis}[width=1.16\marginparwidth,tick label style={font=\scriptsize},
axis y line=middle,axis x line=middle,name=myplot,axis on top,axis equal,
ymin=-.9,ymax=10.9,xmin=-.9,xmax=10.9,scaled ticks=false]
\addplot [draw={\colorone},thick,smooth,domain=0:3] {cosh(x)};
\draw (axis cs:8,4) node {\scriptsize $y=\cosh^{-1} x$};
\draw (axis cs:5.6,10) node {\scriptsize $y=\cosh x$};
\addplot [draw={\colortwo},smooth,thick,domain=0:3] ({cosh(x)},x);
\addplot [dashed,domain=-.5:10] {x};
\end{axis}
\node [right] at (myplot.right of origin) {\scriptsize $x$};
\node [above] at (myplot.above origin) {\scriptsize $y$};
\end{tikzpicture}
&
\begin{tikzpicture}
\begin{axis}[width=1.16\marginparwidth,tick label style={font=\scriptsize},
axis y line=middle,axis x line=middle,name=myplot,axis on top,axis equal,
ymin=-10.9,ymax=10.9,xmin=-10.9,xmax=10.9,scaled ticks=false]
\addplot [draw={\colorone},thick,smooth,domain=-3:3] {sinh(x)};
\draw (axis cs:-6,7) node {\scriptsize $y=\sinh x$};
\draw (axis cs:6,-5) node {\scriptsize $y=\sinh^{-1} x$};
\draw[->,>=stealth] (axis cs:-1,7) -- (axis cs:2,7);
\draw[->,>=stealth] (axis cs:7,-3) -- (axis cs:7,2);
\addplot [draw={\colortwo},smooth,thick,domain=-3:3] ({sinh(x)},x);
\addplot [dashed,domain=-10:10] {x};
\end{axis}
\node [right] at (myplot.right of origin) {\scriptsize $x$};
\node [above] at (myplot.above origin) {\scriptsize $y$};
\end{tikzpicture}
\\[15pt]
\begin{tikzpicture}
\begin{axis}[width=1.16\marginparwidth,tick label style={font=\scriptsize},
axis y line=middle,axis x line=middle,name=myplot,axis on top,xtick={-2,2},
ytick={-2,2},ymin=-3.2,ymax=3.2,xmin=-3.2,xmax=3.2,scaled ticks=false,axis equal]
\addplot [draw={\colortwo},thick,smooth,domain=.348:3] ({1/tanh(x)},x);
\addplot [draw={\colortwo},thick,smooth,domain=-3:-.348] ({1/tanh(x)},x);
\draw (axis cs:2.2,1.5) node {\tiny $y=\coth^{-1} x$};
\draw (axis cs:2.2,-1.5) node {\tiny $y=\tanh^{-1} x$};
\draw [->,>=stealth] (axis cs:1.8,-1.3) -- (axis cs:.2,-.2);
\draw [loosely dashed] (axis cs:-1,-3)--(axis cs:-1,3);
\draw [loosely dashed] (axis cs:1,-3)--(axis cs:1,3);
\addplot [draw={\colorone},smooth,thick,domain=-3.8:3.8] ({tanh(x)},x);
\end{axis}
\node [right] at (myplot.right of origin) {\scriptsize $x$};
\node [above] at (myplot.above origin) {\scriptsize $y$};
\end{tikzpicture}
&
\begin{tikzpicture}
\begin{axis}[width=1.16\marginparwidth,tick label style={font=\scriptsize},
axis y line=middle,axis x line=middle,name=myplot,axis on top,
xtick={-3,-2,-1,1,2,3},ytick={-3,-2,-1,1,2,3},
ymin=-3.5,ymax=3.5,xmin=-3.5,xmax=3.5,scaled ticks=false,axis equal]
\draw (axis cs:1.5,-1.5) node {\tiny $y=\sech^{-1} x$};
\draw (axis cs:-2.2,-1.6) node {\tiny $y=\csch^{-1} x$};
\draw [->,>=latex] (axis cs:1,-1.2) -- (axis cs:1,-.2);
\addplot [draw={\colortwo},smooth,thick,domain=-3:-.3275] ({1/sinh(x)},x);
\addplot [draw={\colortwo},smooth,thick,domain=.3275:3] ({1/sinh(x)},x);
\addplot [draw={\colorone},thick,smooth,domain=0:3] ({1/cosh(x)},x);
\end{axis}
\node [right] at (myplot.right of origin) {\scriptsize $x$};
\node [above] at (myplot.above origin) {\scriptsize $y$};
\end{tikzpicture}
\end{tabular}}
\caption{Graphs of the hyperbolic functions and their inverses.}\label{fig:hfinverse1}
\end{lxfigure}

Now let's consider the inverses of the hyperbolic functions. We begin with the function $f(x)=\sinh x$. Since $\fp(x)=\cosh x>0$ for all real $x$, $f$ is increasing and must be one-to-one. We proceed as in \autoref{sec:inv_funcs}:
\begin{align*}\allowdisplaybreaks
y&=\frac{e^x-e^{-x}}2\\
2y&=e^x-e^{-x} \qquad\text{(now multiply by $e^x$)}\\
2ye^x&=e^{2x}-1 \qquad\text{(a quadratic form )}\\
\left(e^x\right)^2-2ye^x-1&=0 \qquad\text{(use the quadratic formula)}\\
e^x&=\frac{2y\pm\sqrt{4y^2-4}}2\\
e^x&=y\pm\sqrt{y^2+1} \qquad\text{(use the fact that $e^x>0$)}\\
e^x&=y+\sqrt{y^2+1}\\
x&=\ln(y+\sqrt{y^2+1})\\
\end{align*}
Finally, interchange the variable to find that
\[\sinh^{-1} x=\ln(x+\sqrt{x^2+1}).\]
In a similar manner we find that the inverses of the other hyperbolic functions are given by:

\setboxwidth{120pt}
\keyidea{idea:hyperbolic_log}{Logarithmic definitions of Inverse Hyperbolic Functions}
{\noindent%
\begin{minipage}[t]{.5\specialboxlength}
\index{hyperbolic function!inverse!logarithmic def.}
\begin{enumerate}
\item $\ds\cosh^{-1}x=\ln\big(x+\sqrt{x^2-1}\big);\ x\geq1$\rule[-10pt]{0pt}{20pt}
\item $\ds\tanh^{-1}x = \frac12\ln\left(\frac{1+x}{1-x}\right);\ \abs x<1$\rule[-10pt]{0pt}{20pt}
\item $\ds \sech^{-1}x = \ln\left(\frac{1+\sqrt{1-x^2}}x\right);\ 0<x\leq1$\rule[-10pt]{0pt}{20pt}
\end{enumerate}
\end{minipage}%
\begin{minipage}[t]{.5\specialboxlength}
\begin{enumerate}\addtocounter{enumi}{3}
\item $\ds\sinh^{-1}x = \ln\big(x+\sqrt{x^2+1}\big)$\rule[-10pt]{0pt}{20pt}
\item	 $\ds\coth^{-1}x = \frac12\ln\left(\frac{x+1}{x-1}\right);\ \abs x>1$\rule[-10pt]{0pt}{20pt}
\item $\ds\csch^{-1}x = \ln\left(\frac1x+\frac{\sqrt{1+x^2}}{\abs x}\right);\ x\neq0$\rule[-10pt]{0pt}{20pt}
\end{enumerate}
\end{minipage}}

The following Key Ideas give the derivatives and integrals relating to the inverse hyperbolic functions. In \autoref{idea:hyperbolic_inverse_integrals}, both the inverse hyperbolic and logarithmic function representations of the antiderivative are given, based on \autoref{idea:hyperbolic_log}. Again, these latter functions are often more useful than the former.
%Note how inverse hyperbolic functions can be used to solve integrals we used Trigonometric Substitution to solve in \autoref{sec:trig_sub}.



%\mtable{Logarithmic definitions of the inverse hyperbolic functions.}{fig:hfinverse5}{%
%\begin{align*}
%\cosh^{-1}x&=\ln\big(x+\sqrt{x^2-1}\big);\ x\geq1\\
%\sinh^{-1}x &= \ln\big(x+\sqrt{x^2+1}\big)\\
%\tanh^{-1}x &= \frac12\ln\left(\frac{1+x}{1-x}\right);\ \abs x<1\\
%\sech^{-1}x &= \ln\left(\frac{1+\sqrt{1-x^2}}x\right);\ 0<x\leq1\\
%\csch^{-1}x &= \ln\left(\frac1x+\frac{\sqrt{1+x^2}}{\abs x}\right);\ x\neq0\\
%\coth^{-1}x &= \frac12\ln\left(\frac{x+1}{x-1}\right);\ \abs x>1
%\end{align*}
%}

\setboxwidth{120pt}
\keyidea{idea:hyperbolic_inverse_derivatives}{Derivatives Involving Inverse Hyperbolic Functions}
{\index{hyperbolic function!inverse!derivative}\index{derivative!inverse hyper.}%
\begin{minipage}[t]{.45\specialboxlength}
\begin{enumerate}
\item $\ds\frac{d}{dx}\big(\cosh^{-1} x\big) = \frac{1}{\sqrt{x^2-1}};\ x>1$
\item $\ds\frac{d}{dx}\big(\sinh^{-1} x\big) = \frac{1}{\sqrt{x^2+1}}$
\item $\ds\frac{d}{dx}\big(\tanh^{-1} x\big) = \frac{1}{1-x^2};\ \abs x<1$
\end{enumerate}
\end{minipage}%
\begin{minipage}[t]{.55\specialboxlength}
\begin{enumerate}\addtocounter{enumi}{3}
\item $\ds\frac{d}{dx}\big(\sech^{-1} x\big) = \frac{-1}{x\sqrt{1-x^2}}; 0<x<1$
\item $\ds\frac{d}{dx}\big(\csch^{-1} x\big) = \frac{-1}{\abs x\sqrt{1+x^2}};\ x\neq0$
\item $\ds\frac{d}{dx}\big(\coth^{-1} x\big) = \frac{1}{1-x^2};\ \abs x>1$
\end{enumerate}
\end{minipage}}

\setboxwidth{120pt}
\keyidea{idea:hyperbolic_inverse_integrals}{Integrals Involving Inverse Hyperbolic Functions}
{\index{integration!inverse hyper.}\index{hyperbolic function!inverse!integration}%
\begin{enumerate}
\item \parbox{70pt}{$\ds\int \frac{1}{\sqrt{x^2-a^2}}\ dx$} \parbox{180pt}{$\ds=\qquad \cosh^{-1}\left(\frac xa\right)+C;\ 0<a<x$} $\ds=\ln\abs{x+\sqrt{x^2-a^2}}+C$

\item \parbox{70pt}{$\ds\int \frac{1}{\sqrt{x^2+a^2}}\ dx$} \parbox{180pt}{$\ds=\qquad \sinh^{-1}\left(\frac xa\right)+C;\ a>0$} $\ds=\ln\abs{x+\sqrt{x^2+a^2}}+C$

\item \parbox{70pt}{$\ds\int \frac{1}{a^2-x^2}\ dx$} \parbox{180pt}{$\ds=\qquad \begin{cases}\frac1a\tanh^{-1}\left(\frac xa\right)+C & \abs x<\abs a \\
\frac1a\coth^{-1}\left(\frac xa\right)+C & \abs a<\abs x \end{cases}$} $\ds=\frac1{2a}\ln\abs{\frac{a+x}{a-x}}+C$

\item \parbox{70pt}{$\ds\int \frac{1}{x\sqrt{a^2-x^2}}\ dx $} \parbox{180pt}{$\ds=\qquad -\frac1a\sech^{-1}\left(\frac xa\right)+C;\ 0<x<a$} $\ds= \frac1a \ln\left(\frac{x}{a+\sqrt{a^2-x^2}}\right)+C $

\item	\parbox{70pt}{$\ds\int \frac{1}{x\sqrt{x^2+a^2}}\ dx $} \parbox{180pt}{$\ds=\qquad -\frac1a\csch^{-1}\abs{\frac xa} + C;\ x\neq 0,\ a>0$}$\ds= \frac1a \ln\abs{\frac{x}{a+\sqrt{a^2+x^2}}}+C $
\end{enumerate}}

We practice using the derivative and integral formulas in the following example.\bigskip

\example{ex_hf3}{Derivatives and integrals involving inverse hyperbolic functions}
{Evaluate the following.\\
\begin{minipage}[t]{.5\textwidth}
\begin{enumerate}
\item	$\ds \frac{d}{dx}\left[\cosh^{-1}\left(\frac{3x-2}{5}\right)\right]$
\item	$\ds \int\frac{1}{x^2-1}\ dx$
\end{enumerate}
\end{minipage}%
\begin{minipage}[t]{.5\textwidth}
\begin{enumerate}\addtocounter{enumi}{2}
\item	$\ds \int \frac{1}{\sqrt{9x^2+10}}\ dx$
\end{enumerate}
\end{minipage}
}
{\begin{enumerate}
\item	Applying \autoref{idea:hyperbolic_inverse_derivatives} with the Chain Rule gives:
		\[\frac{d}{dx}\left[\cosh^{-1}\left(\frac{3x-2}5\right)\right] = \frac{1}{\sqrt{\left(\frac{3x-2}5\right)^2-1}}\cdot\frac35.\]

\item		Multiplying the numerator and denominator by $(-1)$ gives: $\ds \int \frac{1}{x^2-1}\ dx = \int \frac{-1}{1-x^2}\ dx$. The second integral can be solved with a direct application of item \#3 from \autoref{idea:hyperbolic_inverse_integrals}, with $a=1$. Thus
\begin{align}
\int \frac{1}{x^2-1}\ dx &= -\int \frac{1}{1-x^2}\ dx \notag \\
		&= \begin{cases}-\tanh^{-1}\left(x\right)+C & x^2<1 \\
-\coth^{-1}\left(x\right)+C & 1<x^2 \end{cases} \notag\\
     &=-\frac12\ln\abs{\frac{x+1}{x-1}}+C\notag\\
     &=\frac12\ln\abs{\frac{x-1}{x+1}}+C.\label{eq:hf3}
     \end{align}

%We should note that this exact problem was solved at the beginning of \autoref{sec:partial_fraction}. In that example the answer was given as $\frac12\ln\abs{x-1}-\frac12\ln\abs{x+1}+C.$ Note that this is equivalent to the answer given in \autoeqref{eq:hf3}, as $\ln(a/b) = \ln a - \ln b$.

%The key to linking the two seemingly different answers together is \autoref{fig:hfinverse5}, where the logarithmic definitions of the inverse hyperbolic functions are given. Note that the definitions of $\tanh^{-1}x$ and $\coth^{-1}x$ are very similar; the conditions placed on $\abs x$ ensure that the argument of $\ln$ is always positive. Thus one could say 
%\[\frac12\ln\abs{\frac{x+1}{x-1}} = \begin{cases}\tanh^{-1}x+C & \abs x<1 \\ \\
%\coth^{-1}x+C & \abs x>1 \end{cases}.\]
%
%We reconcile the two answers by returning to \autoeqref{eq:hf3} and continuing:
%\begin{align*}
%\int \frac{1}{x^2-1}\ dx &= \int \frac{-1}{1-x^2}\ dx \\
%			&= \begin{cases}-\frac1a\tanh^{-1}\left(\frac xa\right)+C & x^2<a^2 \\ \\
%-\frac1a\coth^{-1}\left(\frac xa\right)+C & a^2<x^2 \end{cases}\\
%			&= -\frac12\ln\abs{\frac{x+1}{x-1}}+C \\
%			&= -\frac12\ln\abs{x+1} + \frac12\ln\abs{x-1} +C,
%\end{align*}
%matching the answer previously obtained.

\item		This requires a substitution, then item \#2 of \autoref{idea:hyperbolic_inverse_integrals} can be applied.

Let $u = 3x$, hence $du = 3dx$. We have 
\begin{align*}
	\int \frac{1}{\sqrt{9x^2+10}}\ dx
	&= \frac13\int\frac{1}{\sqrt{u^2+10}}\ du. \\
	\intertext{Note $a^2=10$, hence $a = \sqrt{10}.$ Now apply the integral rule.}\\
	 &= \frac13 \sinh^{-1}\left(\frac{3x}{\sqrt{10}}\right) + C \\
	 &= \frac13 \ln \abs{3x+\sqrt{9x^2+10}}+C.\eoehere
\end{align*}
\end{enumerate}}

This section covers a lot of ground. New functions were introduced, along with some of their fundamental identities, their derivatives and antiderivatives, their inverses, and the derivatives and antiderivatives of these inverses. Four Key Ideas were presented, each including quite a bit of information.

Do not view this section as containing a source of information to be memorized, but rather as a reference for future problem solving. \autoref{idea:hyperbolic_inverse_integrals} contains perhaps the most useful information. Know the integration forms it helps evaluate and understand how to use the inverse hyperbolic answer and the logarithmic answer.

The next section takes a brief break from demonstrating new integration techniques. It instead demonstrates a technique of evaluating limits that return indeterminate forms. This technique will be useful in \autoref{sec:improper_integration}, where limits will arise in the evaluation of certain definite integrals.


\printexercises{exercises/06_05_exercises}

\section{L'H\^opital's Rule}\label{sec:lhopitals_rule}

This section is concerned with a technique for evaluating certain limits that will be useful in later chapters.

Our treatment of limits exposed us to ``0/0'', an indeterminate form. If $\ds \lim_{x\to c}f(x)=0$ and $\ds \lim_{x\to c} g(x) =0$, we do not conclude that $\ds \lim_{x\to c} f(x)/g(x)$ is $0/0$; rather, we use $0/0$ as notation to describe the fact that both the numerator and denominator approach 0. The expression 0/0 has no numeric value; other work must be done to evaluate the limit.

Other indeterminate forms exist; they are: $\infty/\infty$, $0\cdot\infty$, $\infty-\infty$, $0^0$, $1^\infty$ and $\infty^0$. Just as ``0/0'' does not mean ``divide 0 by 0,'' the expression ``$\infty/\infty$'' does not mean ``divide infinity by infinity.'' Instead, it means ``a quantity is growing without bound and is being divided by another quantity that is growing without bound.'' We cannot determine from such a statement what value, if any, results in the limit. Likewise, ``$0\cdot \infty$'' does not mean ``multiply zero by infinity.'' Instead, it means ``one quantity is shrinking to zero, and is being multiplied by a quantity that is growing without bound.'' We cannot determine from such a description what the result of such a limit will be.

This section introduces L'H\^opital's Rule, a method of resolving limits that produce the indeterminate forms 0/0 and $\infty/\infty$. We'll also show how algebraic manipulation can be used to convert other indeterminate expressions into one of these two forms so that our new rule can be applied.

\theorem{thm:LHR_1}{L'H\^opital's Rule, Part 1}
{Let $f$ and $g$ be differentiable functions on an open interval $I$ containing $a$.
\begin{enumerate}
\item If $\ds\lim_{x\to a}f(x)=0$, $\ds\lim_{x\to a}g(x)=0$, and $g\primeskip'(x)\neq 0$ except possibly at $x=a$, then \[\lim_{x\to a}\frac{f(x)}{g(x)}=\lim_{x\to a} \frac{\fp(x)}{g\primeskip'(x)},\]
assuming that the limit on the right exists.
\item If  $\ds\lim_{x\to a}f(x)=\pm\infty$ and $\ds\lim_{x\to a}g(x)=\pm\infty$, then \[\lim_{x\to a}\frac{f(x)}{g(x)}=\lim_{x\to a} \frac{\fp(x)}{g\primeskip'(x)},\]
assuming that the limit on the right exists.
\index{LHopitals Rule@L'H\^opital's Rule}
\end{enumerate}}

A similar statement holds if we just look at the one sided limits $\ds\lim_{x\to a^-}$ and $\ds\lim_{x\to a^+}$.

\theorem{thm:LHR_2}{L'H\^opital's Rule, Part 2}
{Let $f$ and $g$ be differentiable functions on the open interval $(c,\infty)$ for some value $c$ and $g\primeskip'(x)\neq0$ on $(c,\infty)$.
\begin{enumerate}
\item If $\ds\lim_{x\to\infty}f(x)=0$ and $\ds\lim_{x\to \infty}g(x)=0$, then
\[\lim_{x\to\infty}\frac{f(x)}{g(x)}=\lim_{x\to\infty}\frac{\fp(x)}{g\primeskip'(x)},\]
assuming that the limit on the right exists.
\item If  $\ds\lim_{x\to \infty}f(x)=\pm\infty$ and $\ds\lim_{x\to \infty}g(x)=\pm\infty$, then
\[\lim_{x\to\infty}\frac{f(x)}{g(x)}=\lim_{x\to\infty}\frac{\fp(x)}{g\primeskip'(x)},\]
assuming that the limit on the right exists.
\end{enumerate}
Similar statements can be made where $x$ approaches $-\infty$.
\index{LHopitals Rule@L'H\^opital's Rule}}

We demonstrate the use of L'H\^opital's Rule in the following examples; we will often use ``LHR'' as an abbreviation of ``L'H\^opital's Rule.''

\example{ex_lhr1}{Using L'H\^opital's Rule}{Evaluate the following limits, using L'H\^opital's Rule as needed.\\
\begin{minipage}[t]{.5\textwidth}
\begin{enumerate}
\item		$\ds \lim_{x\to0}\frac{\sin x}x$
\item		$\ds \lim_{x\to 1}\frac{\sqrt{x+3}-2}{1-x}$
\item		$\ds \lim_{x\to0}\frac{x^2}{1-\cos x}$
\end{enumerate}
\end{minipage}%
\begin{minipage}[t]{.5\textwidth}
\begin{enumerate}\addtocounter{enumi}{3}
	\item	$\ds\lim_{x\to-3}\frac{x^3+27}{x^2+9}$
	\item	$\ds\lim_{x\to\infty}\frac{3x^2-100x+2}{4x^2+5x-1000}$
	\item	$\ds\lim_{x\to\infty}\frac{e^x}{x^3}$
\end{enumerate}
\end{minipage}}
{\begin{enumerate}
	\item	This has the indeterminate form $0/0$. We proved this limit is 1 in \autoref{ex_limit_sinx_prove} using the Squeeze Theorem. Here we use L'H\^opital's Rule to show its power.
\[\lim_{x\to0}\frac{\sin x}x \LHequals \lim_{x\to0} \frac{\cos x}{1}=1.\]
While this seems easier than using the Squeeze Theorem to find this limit, we note that applying L'H\^opital's Rule here requires us to know the derivative of $\sin x$. We originally encountered this limit when we were trying to find that derivative.

	\item	This has the indeterminate form $0/0$.
\[\lim_{x\to 1}\frac{\sqrt{x+3}-2}{1-x} 	 \LHequals \lim_{x \to 1} \frac{\frac12(x+3)^{-1/2}}{-1} =-\frac 14.\]

	\item	This has the indeterminate form $0/0$.
\[\lim_{x\to 0}\frac{x^2}{1-\cos x}  \LHequals  \lim_{x\to 0} \frac{2x}{\sin x}.\]
This latter limit also evaluates to the $0/0$ indeterminate form. To evaluate it, we apply L'H\^opital's Rule again.
\[
 \lim_{x\to 0} \frac{2x}{\sin x}
 \LHequals \frac{2}{\cos x} = 2 .
\]
Thus $\ds \lim_{x\to0}\frac{x^2}{1-\cos x}=2.$

	\item \mbox{}\\[-2\baselineskip]
\[\lim_{x\to-3}\frac{x^3+27}{x^2+9} =\frac 0{18}=0\]
We cannot use L'H\^opital's Rule in this case because the original limit does not return an indeterminate form, so L'H\^opital's Rule does not apply. In fact, the inappropriate use of L'H\^opital's Rule here would result in the incorrect limit $-\frac92$.
% was \frac32

	\item	We can evaluate this limit already using \autoref{thm:lim_rational_fn_at_infty}; the answer is 3/4. We apply L'H\^opital's Rule to demonstrate its applicability.
\[
 \lim_{x\to\infty} \frac{3x^2-100x+2}{4x^2+5x-1000}
 \LHequals \lim_{x\to\infty} \frac{6x-100}{8x+5}
 \LHequals \lim_{x\to\infty} \frac68 = \frac34.
\]

	\item	$\ds\lim_{x\to \infty}\frac{e^x}{x^3} \LHequals \lim_{x\to\infty} \frac{e^x}{3x^2} \LHequals \lim_{x\to\infty} \frac{e^x}{6x} \LHequals \lim_{x\to\infty} \frac{e^x}{6} = \infty.$

Recall that this means that the limit does not exist; as $x$ approaches $\infty$, the expression $e^x/x^3$ grows without bound. We can infer from this that $e^x$ grows ``faster'' than $x^3$; as $x$ gets large, $e^x$ is far larger than $x^3$. (This has important implications in computing when considering efficiency of algorithms.)\eoehere
\end{enumerate}}

\subsection{Indeterminate Forms \texorpdfstring{$0\cdot\infty$ and $\infty-\infty$}{0·∞ and ∞-∞}}

L'H\^opital's Rule can only be applied to ratios of functions. When faced with an indeterminate form such as $0\cdot\infty$ or $\infty-\infty$, we can sometimes apply algebra to rewrite the limit so that L'H\^opital's Rule can be applied. We demonstrate the general idea in the next example.
\index{limit!indeterminate form}\index{indeterminate form}

\youtubeVideo{kEnwac_9lyg}{L'Ho\^pital's Rule --- Indeterminate Powers}

\example{ex_LHR3}{Applying L'H\^opital's Rule to other indeterminate forms}{Evaluate the following limits.\\
\begin{minipage}[t]{.5\textwidth}
\begin{enumerate}
\item		$\ds \lim_{x\to0^+} x\cdot e^{1/x}$
\item		$\ds \lim_{x\to0^-} x\cdot e^{1/x}$
\end{enumerate}
\end{minipage}%
\begin{minipage}[t]{.5\textwidth}
\begin{enumerate}\addtocounter{enumi}{2}
\item		$\ds \lim_{x\to\infty} \left(\ln(x+1)-\ln x\right)$
\item		$\ds \lim_{x\to\infty} \left(x^2-e^x\right)$
\end{enumerate}
\end{minipage}}
{\begin{enumerate}
	\item	As $x\to 0^+$, note that $x\to 0$ and $e^{1/x}\to \infty$. Thus we have the indeterminate form $0\cdot\infty$. We rewrite the expression $x\cdot e^{1/x}$ as $\ds\frac{e^{1/x}}{1/x}$; now, as $x\to 0^+$, we get the indeterminate form $\infty/\infty$ to which L'H\^opital's Rule can be applied. 
\[
\lim_{x\to0^+} x\cdot e^{1/x} = \lim_{x\to 0^+} \frac{e^{1/x}}{1/x} \LHequals \lim_{x\to 0^+}\frac{(-1/x^2)e^{1/x}}{-1/x^2} =\lim_{x\to 0^+}e^{1/x} =\infty.
\]

Interpretation: $e^{1/x}$ grows ``faster'' than $x$ shrinks to zero, meaning their product grows without bound.

	\item	As $x\to 0^-$, note that $x\to 0$ and $e^{1/x}\to e^{-\infty}\to 0$. The the limit evaluates to $0\cdot 0$ which is not an indeterminate form. We conclude then that
	\[\lim_{x\to 0^-}x\cdot e^{1/x} = 0.\]

	\item	This limit initially evaluates to the indeterminate form $\infty-\infty$. By applying a logarithmic rule, we can rewrite the limit as 
\[
\lim_{x\to\infty}\left(\ln(x+1)-\ln x\right) = \lim_{x\to \infty} \ln \left(\frac{x+1}x\right).
\]

As $x\to \infty$, the argument of the natural logarithm approaches $\infty/\infty$, to which we can apply L'H\^opital's Rule.
\[\lim_{x\to\infty} \frac{x+1}x \LHequals \lim_{x\to\infty}\frac11=1.
\]

Since $x\to\infty$ implies $\ds\frac{x+1}x\to 1$, it follows that 
\[x\to\infty \quad \text{ implies }\quad \ln\left(\frac{x+1}x\right)\to\ln 1=0.\]

Thus
\[
 \lim_{x\to\infty} \left(\ln(x+1)-\ln x\right)
 = \lim_{x\to \infty} \ln \left(\frac{x+1}x\right)=0.
\]
Interpretation: since this limit evaluates to 0, it means that for large $x$, there is essentially no difference between $\ln (x+1)$ and $\ln x$; their difference is essentially 0.

	\item	The limit $\ds \lim_{x\to\infty} \left(x^2-e^x\right)$ initially returns the indeterminate form $\infty-\infty$. We can rewrite the expression by factoring out $x^2$; $\ds x^2-e^x = x^2\left(1-\frac{e^x}{x^2}\right).$ We need to evaluate how $e^x/x^2$ behaves as $x\to\infty$:
\[
\lim_{x\to\infty}\frac{e^x}{x^2} \LHequals \lim_{x\to\infty} \frac{e^x}{2x}
\LHequals \lim_{x\to\infty} \frac{e^x}{2} = \infty.
\]

Thus $\lim_{x\to\infty}x^2(1-e^x/x^2)$ evaluates to $\infty\cdot(-\infty)$, which is not an indeterminate form; rather, $\infty\cdot(-\infty)$ evaluates to $-\infty$. We conclude that 
$\ds \lim_{x\to\infty} \left(x^2-e^x\right) = -\infty.$

Interpretation: as $x$ gets large, the difference between $x^2$ and $e^x$ grows very large.\eoehere
\end{enumerate}}

\subsection{Indeterminate Forms\ \ \texorpdfstring{$0^0$, $1^\infty$, and $\infty^0$}{0\^{}0, 1\^{}∞, and ∞\^{}0}}

When faced with a limit that returns one of the indeterminate forms $0^0$, $1^\infty$, or $\infty^0$, it is often useful to use the natural logarithm to convert to an indeterminate form we already know how to find the limit of, then use the natural exponential function find the original limit. This is possible because the natural logarithm and natural exponential functions are inverses and because they are both continuous. The following Key Idea expresses the concept, which is followed by an example that demonstrates its use.

\keyidea{idea:LHR_power}{\parbox[t]{200pt}{Evaluating Limits Involving Indeterminate Forms $0^0$, $1^\infty$ and $\infty^0$}}
{If $\ds \lim_{x\to c} \ln\big(f(x)\big) = L$,\quad then 
$\ds \lim_{x\to c} f(x) = \lim_{x\to c} e^{\ln(f(x))} = e\,^L.$ \index{limit!indeterminate form}\index{indeterminate form}}

\example{ex_LHR4}{Using L'H\^opital's Rule with indeterminate forms involving exponents}
{Evaluate the following limits.
\[
 \text{1.}\quad\lim_{x\to\infty} \left(1+\frac1x\right)^x \qquad\qquad
 \text{2.}\quad\lim_{x\to0^+} x^x.
\]}
{\begin{enumerate}
\item		This is equivalent to a special limit given in \autoref{thm:special_limits}; these limits have important applications in mathematics and finance. Note that the exponent approaches $\infty$ while the base approaches 1, leading to the indeterminate form $1^\infty$. Let $f(x) = (1+1/x)^x$; the problem asks to evaluate $\ds\lim_{x\to\infty}f(x)$. Let's first evaluate $\ds \lim_{x\to\infty}\ln\big(f(x)\big)$.
\begin{align*}
\lim_{x\to\infty}\ln\big(f(x)\big)
			&= \lim_{x\to\infty} \ln \left(1+\frac1x\right)^x \\
			&= \lim_{x\to\infty} x\ln\left(1+\frac1x\right)\\
			&= \lim_{x\to\infty} \frac{\ln\left(1+\frac1x\right)}{1/x}\\
			\intertext{This produces the indeterminate form 0/0, so we apply L'H\^opital's Rule.}
			&=	\lim_{x\to\infty} \frac{\frac{1}{1+1/x}\cdot(-1/x^2)}{(-1/x^2)} \\
			&= \lim_{x\to\infty}\frac{1}{1+1/x}\\
			&= 1.
\end{align*}
Thus $\ds\lim_{x\to\infty} \ln \big(f(x)\big) = 1.$ We return to the original limit and apply \autoref{idea:LHR_power}.
\[\lim_{x\to\infty}\left(1+\frac1x\right)^x = \lim_{x\to\infty} f(x) =  \lim_{x\to\infty}e^{\ln (f(x))} = e^1 = e.\]
This is another way to determine the value of the number $e$.

\item		This limit leads to the indeterminate form $0^0$. Let $f(x) = x^x$ and consider first $\ds\lim_{x\to0^+} \ln\big(f(x)\big)$. 
%
\mtable{A graph of $f(x)=x^x$ supporting the fact that as $x\to 0^+$, $f(x)\to 1$.}{fig:LHR4}{\begin{tikzpicture}
\begin{axis}[width=1.16\marginparwidth,tick label style={font=\scriptsize},
axis y line=middle,axis x line=middle,name=myplot,axis on top,ytick={1,2,3,4},
ymin=-.4,ymax=4.5,xmin=-.1,xmax=2.2,scaled ticks=false]
\addplot[draw={\colorone},thick,smooth,domain=.01:2]{exp(x*ln(x))};
\draw (axis cs:1,1) node [below right] {\scriptsize $f(x)=x^x$};
\end{axis}
\node [right] at (myplot.right of origin) {\scriptsize $x$};
\node [above] at (myplot.above origin) {\scriptsize $y$};
\end{tikzpicture}}%
%
\begin{align*}
\lim_{x\to0^+} \ln\big(f(x)\big) &= \lim_{x\to0^+} \ln\left(x^x\right) \\
			&= \lim_{x\to0^+} x\ln x \\
			\intertext{This produces the indeterminate form $0(-\infty)$, so we rewrite it in order to apply L'H\^opital's Rule.}
			&= \lim_{x\to0^+} \frac{\ln x}{1/x}.\\
			\intertext{This produces the indeterminate form $-\infty/\infty$ so we apply L'H\^opital's Rule.}
			&=	\lim_{x\to0^+} \frac{1/x}{-1/x^2} \\
			&= \lim_{x\to0^+} -x \\
			&= 0.
\end{align*}%
Thus $\ds\lim_{x\to0^+} \ln\big(f(x)\big) =0$. We return to the original limit and apply \autoref{idea:LHR_power}.
\[
\lim_{x\to0^+} x^x = \lim_{x\to0^+} f(x) = \lim_{x\to0^+} e^{\ln(f(x))} = e^0 = 1.
\]
This result is supported by the graph of $f(x)=x^x$ given in \autoref{fig:LHR4}.\eoehere
\end{enumerate}}

% todo Tim do we want to add a hierarchy of function growth to the end of LH section?

%Our brief revisit of limits will be rewarded in the next section where we consider \textit{improper integration.} So far, we have only considered definite integrals where the bounds are finite numbers, such as $\ds \int_0^1 f(x)\ dx$. Improper integration considers integrals where one, or both, of the bounds are ``infinity.'' Such integrals have many uses and applications, in addition to generating ideas that are enlightening.

\printexercises{exercises/06_06_exercises}


\apexchapter{Techniques of Antidifferentiation}{chapter:anti_tech}
Chapter \ref{chapter:integration} introduced the antiderivative and connected it to signed areas under a curve through the Fundamental Theorem of Calculus. The next chapter explores more applications of definite integrals than just area. As evaluating definite integrals will become important, we will want to find antiderivatives of a variety of functions.

This chapter is devoted to exploring techniques of antidifferentiation. While not every function has an antiderivative in terms of elementary functions (a concept introduced in the section on Numerical Integration), we can still find antiderivatives of a wide variety of functions.
\section{Integration by Parts}\label{sec:IBP}

Here's a simple integral that we can't yet evaluate:
\[\int x\cos x\dd x.\]
It's a simple matter to take the derivative of the integrand using the Product Rule, but there is no Product Rule for integrals.  However, this section introduces \emph{Integration by Parts}, a method of integration that is based on the Product Rule for derivatives. It will enable us to evaluate this integral.

The Product Rule says that if $u$ and $v$ are functions of $x$, then  $(uv)' = u\primeskip'v + uv\primeskip'$.  For simplicity, we've written $u$ for $u(x)$ and $v$ for $v(x)$.  Suppose we integrate both sides with respect to $x$.  This gives
\[\int (uv)'\dd x = \int (u\primeskip'v+uv\primeskip')\dd x.\]
By the Fundamental Theorem of Calculus, the left side integrates to $uv$.  The right side can be broken up into two integrals, and we have
\[uv = \int u\primeskip'v\dd x + \int uv\primeskip'\dd x.\]
Solving for the second integral we have
\[\int uv\primeskip'\dd x = uv - \int u\primeskip'v\dd x.\]
Using differential notation, we can write
\[
\begin{aligned}
 u\primeskip'&=\frac{\dd u}{\dd x} \\
 v\primeskip'&=\frac{\dd v}{\dd x}
\end{aligned}
\qquad\Rightarrow\qquad
\begin{aligned}
 \dd u&=u\primeskip'\dd x \\
 \dd v&=v\primeskip'\dd x.
\end{aligned}
\]
Thus, the equation above can be written as follows:
\[\int u\dd v = uv - \int v\dd u.\]
This is the Integration by Parts formula. For reference purposes, we state this in a theorem.

\begin{theorem}[Integration by Parts]\label{thm:IBP}
Let $u$ and $v$ be differentiable functions of $x$ on an interval $I$ containing $a$ and $b$. Then 
\[\int u\dd v = uv - \int v\dd u,\]
and applying FTC part 2 we have \index{integration!by parts}
\[\int_{x=a}^{x=b} u\dd v = uv\Big|_a^b - \int_{x=a}^{x=b}v\dd u.\]
\end{theorem}

\youtubeVideo{zGGI4PkHzhI}{Integration by Parts --- Definite Integral}

Let's try an example to understand our new technique.

\begin{example}[Integrating using Integration by Parts]\label{ex_ibp1}
Evaluate $\ds\int x\cos{x}\dd x$.
\solution
The key to Integration by Parts is to identify part of the integrand as ``$u$'' and part as ``$\dd v$.'' Regular practice will help one make good identifications, and later we will introduce some principles that help. For now, let  $u=x$ and $\dd v=\cos x\dd x$.

It is generally useful to make a small table of these values.
\[
\begin{aligned}
u&= x & \dd v&=\cos x\dd x\\
\dd u&= \text{?} & v&=\text{?}
\end{aligned}
\qquad\Rightarrow\qquad
\begin{aligned}
u&= x & \dd v&=\cos x\dd x\\
\dd u&= \dd x & v&=\sin x
\end{aligned}
\]
Right now we only know $u$ and $dv$ as shown on the left; on the right we fill in the rest of what we need. If $u = x$, then $\dd u = \dd x$. Since $\dd v = \cos x\dd x$, $v$ is an antiderivative of $\cos x$, so $v = \sin x$.

Now substitute all of this into the Integration by Parts formula, giving
\[\int x\cos x\dd x = x\sin x - \int \sin x\dd x.\]
We can then integrate $\sin x$ to get $-\cos x + C$ and overall our answer is
\[\int x\cos x\dd x = x\sin x + \cos x + C.\]
We have two important notes here: (1) notice how the antiderivative contains the product, $x\sin x$. This product is what makes integration by parts necessary. And (2) antidifferentiating $\dd v$ does result in $v+C$. The intermediate $+C$s are all added together and represented by one $+C$ in the final answer.
\end{example}

The example above demonstrates how Integration by Parts works in general.  We try to identify $u$ and $\dd v$ in the integral we are given, and the key is that we usually want to choose $u$ and $\dd v$ so that $\dd u$ is simpler than $u$ and $v$ is hopefully not too much more complicated than $\dd v$.  This will mean that the integral on the right side of the Integration by Parts formula, $\int v\dd u$ will be simpler to integrate than the original integral $\int u\dd v$.

In the example above, we chose $u=x$ and $\dd v=\cos x\dd x$.  Then $\dd u=\dd x$ was simpler than $u$ and $v=\sin x$ is no more complicated than $\dd v$.  Therefore, instead of integrating $x\cos x\dd x$, we could integrate $\sin x\dd x$, which we knew how to do.

If we had chosen $u=\cos x$ and $\dd v=x\dd x$, so that $\dd u=-\sin x\dd x$ and $v=\frac12x^2$, then
\[\int x\cos x\dd x=\frac12x^2\cos x-\left(-\frac12\right)\int x^2\sin x\dd x.\]
We then need to integrate $x^2\sin x$, which is more complicated than our original integral, making this an unproductive choice.

%A useful mnemonic for helping to determine $u$ is ``LIATE,'' where 
%\begin{center}L = \textbf{L}ogarithmic, I = \textbf{I}nverse Trig., A = \textbf{A}lgebraic (polynomials), 
%T = \textbf{T}rigonometric, and E = \textbf{E}xponential.
%\end{center}

%If the integrand contains both a logarithmic and an algebraic term, in general letting $u$ be the logarithmic term works best, as indicated by L coming before A in LIATE.

We now consider another example.

\begin{example}[Integrating using Integration by Parts]\label{ex_ibp2}
Evaluate $\ds\int x e^x\dd x$.
\solution
Notice that $x$ becomes simpler when differentiated and $e^x$ is unchanged by differentiation or integration. This suggests that we should let $u=x$ and $\dd v=e^x\dd x$:
%The integrand contains an \textbf{A}lgebraic term ($x$) and an \textbf{E}xponential term ($e^x$). Our mnemonic suggests letting $u$ be the algebraic term, so we choose $u=x$ and $\dd v=e^x\dd x$.  Then $\dd u=\dd x$ and $v=e^x$ as indicated by the tables below.\\
\[
\begin{aligned}
u&= x & \dd v&=e^x\dd x\\
\dd u&= \text{?} & v&=\text{?}
\end{aligned}
\qquad\Rightarrow\qquad
\begin{aligned}
u&= x & \dd v&=e^x\dd x\\
\dd u&= \dd x & v&=e^x
\end{aligned}
\]

%We see $du$ is simpler than $u$, while there is no change in going from $dv$ to $v$.  This is good.
The Integration by Parts formula gives
\[\int x e^x\dd x = xe^x - \int e^x\dd x.\]
The integral on the right is simple; our final answer is
\[\int xe^x\dd x = xe^x - e^x + C.\]
Note again how the antiderivatives contain a product term.
\end{example}

\begin{example}[Integrating using Integration by Parts]\label{ex_ibp3}
Evaluate $\ds\int x^2\cos x\dd x$.
\solution
Let $u=x^2$ instead of the trigonometric function, hence $\dd v=\cos x\dd x$.  Then $\dd u=2x\dd x$ and $v=\sin x$ as shown below.
\[
\begin{aligned}
u&= x^2 & \dd v&=\cos x\dd x\\
\dd u&= \text{?} & v&=\text{?}
\end{aligned}
\qquad\Rightarrow\qquad
\begin{aligned}
u&= x^2 & \dd v&=\cos x\dd x\\
\dd u&= 2x\dd x & v&=\sin x
\end{aligned}
\]

The Integration by Parts formula gives
\[\int x^2\cos x\dd x = x^2\sin x - \int 2x\sin x\dd x.\]
At this point, the integral on the right is indeed simpler than the one we started with, but to evaluate it, we need to do Integration by Parts again. Here we choose $u=2x$ and $\dd v=\sin x\dd x$ and fill in the rest below.
\[
\begin{aligned}
u&= 2x & \dd v&=\sin x\dd x\\
\dd u&= \text{?} & v&=\text{?}
\end{aligned}
\qquad\Rightarrow\qquad
\begin{aligned}
u&= 2x & \dd v&=\sin x\dd x\\
\dd u&= 2\dd x & v&=-\cos x
\end{aligned}
\]

This means that
\[\int x^2\cos x\dd x = x^2\sin x - \left(-2x\cos x - \int -2\cos x\dd x\right).\]
The integral all the way on the right is now something we can evaluate.  It evaluates to $-2\sin x$.  Then going through and simplifying, being careful to keep all the signs straight, our answer is
\[\int x^2\cos x\dd x = x^2\sin x  + 2x\cos x - 2\sin x + C.\]
\end{example}

\begin{example}[Integrating using Integration by Parts]\label{ex_ibp4}
Evaluate $\ds\int e^x\cos x\dd x$.
\solution
This is a classic problem.
%  Our mnemonic suggests letting $u$ be the trigonometric function instead of the exponential.
In this particular example, one can let $u$ be either $\cos x$ or $e^x$;
%to demonstrate that we do not have to follow LIATE,
we choose $u=e^x$ and hence $\dd v = \cos x\dd x$.  Then $\dd u=e^x\dd x$ and $v=\sin x$ as shown below.
\[
\begin{aligned}
u&= e^x & \dd v&=\cos x\dd x\\
\dd u&= \text{?} & v&=\text{?}
\end{aligned}
\qquad\Rightarrow\qquad
\begin{aligned}
u&= e^x& \dd v&=\cos x\dd x\\
\dd u&= e^x\dd x & v&=\sin x
\end{aligned}
\]

Notice that $\dd u$ is no simpler than $u$, going against our general rule (but bear with us). The Integration by Parts formula yields
\[\int e^x\cos x\dd x = e^x\sin x - \int e^x\sin x\dd x.\]
The integral on the right is not much different from the one we started with, so it seems like we have gotten nowhere. Let's keep working and apply Integration by Parts to the new integral. So what should we use for $u$ and $\dd v$ this time? We may feel like letting the trigonometric function be $\dd v$ and the exponential be $u$ was a bad choice last time since we still can't integrate the new integral. However, if we let $u=\sin x$ and $\dd v=e^x\dd x$ this time we will reverse what we just did, taking us back to the beginning. So, we let $u=e^x$ and $\dd v = \sin x\dd x$. This leads us to the following:
\[
\begin{aligned}
u&= e^x & \dd v&=\sin x\dd x\\
\dd u&= \text{?} & v&=\text{?}
\end{aligned}
\qquad\Rightarrow\qquad
\begin{aligned}
u&= e^x& \dd v&=\sin x\dd x\\
\dd u&= e^x\dd x & v&=-\cos x
\end{aligned}
\]

The Integration by Parts formula then gives:
\begin{align*}
 \int e^x\cos x\dd x
 &= e^x\sin x - \left(-e^x\cos x - \int -e^x\cos x\dd x\right)\\
 &= e^x\sin x+ e^x\cos x - \int e^x\cos x\dd x.
\end{align*}
It seems we are back right where we started, as the right hand side contains $\int e^x\cos x\dd x$.  But this is actually a good thing.  

Add $\ds\int e^x\cos x\dd x$ to both sides. This gives 
\begin{align*}
2\int e^x\cos x\dd x & = e^x\sin x + e^x\cos x \\
\intertext{Now divide both sides by 2:}
\int e^x\cos x\dd x & = \frac{1}{2}\bigl(e^x\sin x + e^x\cos x\bigr).
\end{align*}

Simplifying a little and adding the constant of integration, our answer is thus
\[\int e^x\cos x\dd x = \frac12e^x\left(\sin x + \cos x\right)+C.\]
\end{example}

% see Example 2.4.4.
\begin{example}[Using Integration by Parts: antiderivative of $\ln x$]\label{ex_ibp5}
Evaluate $\ds\int \ln x\dd x$.
\solution
One may have noticed that we have rules for integrating the familiar trigonometric functions and $e^x$, but we have not yet given a rule for integrating $\ln x$.  That is because $\ln x$ can't easily be integrated with any of the rules we have learned up to this point.  But we can find its antiderivative by a clever application of Integration by Parts.  Set $u=\ln x$ and $\dd v=\dd x$.  This is a good strategy to learn as it can help in other situations. This determines $\dd u=(1/x)\dd x$ and $v=x$ as shown below.
\[
\begin{aligned}
u&= \ln x & \dd v&=\dd x\\
\dd u&= \text{?} & v&=\text{?}
\end{aligned}
\qquad\Rightarrow\qquad
\begin{aligned}
u&= \ln x& \dd v&=\dd x\\
\dd u&= 1/x\dd x & v&=x
\end{aligned}
\]
Putting this all together in the Integration by Parts formula, things work out very nicely:
\begin{align*}
 \int \ln x\dd x
 &= x\ln x - \int x\,\frac1x\dd x \\
 &= x\ln x - \int 1\dd x \\
 &= x\ln x - x + C.
\end{align*}
\end{example}

\begin{example}[Using Integration by Parts: antiderivative of $\tan^{-1} x$]\label{ex_ibp6}
Evaluate $\displaystyle \int \tan^{-1} x\dd x$.
\solution
The same strategy of $\dd v=\dd x$ that we used above works here.  Let $u=\tan^{-1} x$ and $\dd v=\dd x$.  Then $\dd u=1/(1+x^2)\dd x$ and $v=x$.  The Integration by Parts formula gives
\[\int \tan^{-1} x\dd x = x\tan^{-1} x - \int \frac x{1+x^2}\dd x.\]
The integral on the right can be solved by substitution.  Taking $t=1+x^2$, we get $\dd t=2x\dd x$.  The integral then becomes
\[\int \tan^{-1} x\dd x = x\tan^{-1} x - \frac12\int \frac 1{t}\dd t.\]
The integral on the right evaluates to $\ln\abs t+C$, which becomes $\ln(1+x^2)+C$.  Therefore, the answer is
\[\int \tan^{-1} x\dd x = x\tan^{-1} x - \frac12\ln(1+x^2) + C.\]
Since $1+x^2>0$, we do not need to include the absolute value in the $\ln(1+x^2)$ term.
\end{example}

\subsection{Substitution Before Integration}

When taking derivatives, it was common to employ multiple rules (such as using both the Quotient and the Chain Rules). It should then come as no surprise that some integrals are best evaluated by combining integration techniques. In particular, here we illustrate making an ``unusual'' substitution first before using Integration by Parts.

\begin{example}[Integration by Parts after substitution]\label{ex_ibp8}
Evaluate $\ds \int \cos(\ln x)\dd x$.
\solution
The integrand contains a composition of functions, leading us to think Substitution would be beneficial. Letting $u=\ln x$, we have $\dd u = 1/x\dd x$. This seems problematic, as we do not have a $1/x$ in the integrand. But consider:
\[\dd u = \frac 1x\dd x \Rightarrow x\cdot\dd u = \dd x.\]
Since $u = \ln x$, we can use inverse functions to solve for $x = e^u$. Therefore we have that
\begin{align*}
\dd x &= x\cdot \dd u \\
		&= e^u\dd u.
\end{align*}
We can thus replace $\ln x$ with $u$ and $\dd x$ with $e^u\dd u$. Thus we rewrite our integral as 
\[\int \cos(\ln x)\dd x = \int e^u\cos u\dd u.\]
We evaluated this integral in \autoref{ex_ibp4}. Using the result there, we have:
\begin{align*}
\int \cos(\ln x)\dd x &= \int e^u\cos u\dd u \\
				&= \frac12e^u\bigl(\sin u + \cos u\bigr) + C \\
				&= \frac12e^{\ln x} \bigl(\sin(\ln x) + \cos (\ln x)\bigr)+C\\
				&= \frac12x \bigl(\sin(\ln x) + \cos (\ln x)\bigr)+C.
\end{align*}
\end{example}

\subsection{Definite Integrals and Integration By Parts}

So far we have focused only on evaluating indefinite integrals. Of course, we can use Integration by Parts to evaluate definite integrals as well, as \autoref{thm:IBP} states. We do so in the next example.

\begin{example}[Definite integration using Integration by Parts]\label{ex_ibp7}
Evaluate $\displaystyle \int_1^2 x^2 \ln x\dd x$.
\solution
%Once again, our mnemonic suggests we let $u=\ln x$.  %(We could let $u = x^2$ and $\dd v = \ln x\dd x$, as we now know the antiderivatives of $\ln x$. However, letting $u = \ln x$ makes our next integral much simpler as it removes the logarithm from the integral entirely.)
To simplify the integral we let $u=\ln x$ and $\dd v =x^2\dd x$. 
%So we have $u=\ln x$ and $\dd v=x^2\dd x$.
We then get $\dd u = (1/x)\dd x$ and $v=x^3/3$ as shown below.
\[
\begin{aligned}
u&= \ln x & \dd v&=x^2\dd x\\
\dd u&= \text{?} & v&=\text{?}
\end{aligned}
\qquad\Rightarrow\qquad
\begin{aligned}
u&= \ln x& \dd v&=x^2\dd x\\
\dd u&= 1/x\dd x & v&=x^3/3
\end{aligned}
\]

This may seem counterintuitive since the power on the algebraic factor has increased ($v=x^3/3$), but as we see this is a wise choice:
%The Integration by Parts formula then gives
\begin{align*}
	\int_1^2 x^2 \ln x\dd x
	&= \frac{x^3}3\ln x\bigg|_1^2 - \int_1^2 \frac{x^3}{3}\,\frac 1x\dd x \\
	&=  \frac{x^3}3\ln x\bigg|_1^2 - \int_1^2 \frac{x^2}{3}\dd x \\
	&=  \frac{x^3}3\ln x\bigg|_1^2 - \frac{x^3}{9}\bigg|_1^2\\
	&=  \left(\frac{x^3}3\ln x - \frac{x^3}{9}\right)\bigg|_1^2\\
	&=	\left(\frac83\ln 2 - \frac89\right)-\left(\frac13\ln 1 - \frac19\right) \\
	&= \frac83\ln 2 - \frac79. % \approx 1.07.
\end{align*}
\end{example}

In general, Integration by Parts is useful for integrating certain products of functions, like $\ds\int x e^x\dd x$ or $\ds\int x^3\sin x\dd x$.   It is also useful for integrals involving logarithms and inverse trigonometric functions.

As stated before, integration is generally more difficult than differentiation. We are developing tools for handling a large array of integrals, and experience will tell us when one tool is preferable/necessary over another. For instance, consider the three similar-looking integrals 
\[
\int xe^x\dd x, \qquad  \int x e^{x^2}\dd x \qquad \text{and} \qquad \int xe^{x^3}\dd x.
\]

While the first is calculated easily with Integration by Parts, the second is best approached with Substitution.  Taking things one step further, the third integral has no answer in terms of elementary functions, so none of the methods we learn in calculus will get us the exact answer. We will learn how to approximate this integral in \autoref{chapter:sequences_series}

Integration by Parts is a very useful method, second only to substitution. In the following sections of this chapter, we continue to learn other integration techniques. The next section focuses on handling integrals containing trigonometric functions.

\printexercises{exercises/06-02-exercises}

%maybe $\int \sin(\sqrt x )\dd x $
% If you're looking for a trickier example....

%Maybe an example where Integration by Parts is useful in a theoretical context....



% this section gets 06_01_ex_27 ?

\section{Trigonometric Integrals}\label{sec:trigint}

Trigonometric functions are useful for describing periodic behavior. This section describes several techniques for finding antiderivatives of certain combinations of trigonometric functions.

\subsection{Integrals of the form \texorpdfstring{$\ds \int \sin^m x\cos^n x\dd x$}{∫(sin x)\^{}m (cos x)\^{}n dx}}

In learning the technique of Substitution, we saw the integral $\int \sin x\cos x\dd x$ in \autoref{ex_sub10}. The integration was not difficult, and one could easily evaluate the indefinite integral by letting $u=\sin x$ or by letting $u = \cos x$. This integral is easy since the power of both sine and cosine is 1.

We generalize this and consider integrals of the form $\int \sin^mx\cos^nx\dd x$, where $m,n$ are nonnegative integers. Our strategy for evaluating these integrals is to use the identity $\cos^2x+\sin^2x=1$ to convert high powers of one trigonometric function into the other, leaving a single sine or cosine term in the integrand. We summarize the general technique in the following Key Idea.

\youtubeVideo{zyg9k1je7Fg}{Trigonometric Integrals --- Part 2 of 6}


{%
\tcbset{grow to right by=4em} % 4 sufficient
\begin{keyidea}[Integrals Involving Powers of Sine and Cosine]\label{idea:trig_int_1}%
Consider $\ds \int \sin^mx\cos^nx\dd x$, where $m,n$ are nonnegative integers.\index{integration!of trig.\ powers}
\begin{enumerate}
	\item	If $m$ is odd, then $m=2k+1$ for some integer $k$. Rewrite \small
		\[
		\sin^mx = \sin^{2k+1}x = \sin^{2k}x\sin x = (\sin^2x)^k\sin x = (1-\cos^2x)^k\sin x.
		\]
		\normalsize Then \small
		\[
		\int \sin^mx\cos^nx\dd x = \int (1-\cos^2x)^k\sin x\cos^nx\dd x = -\int (1-u^2)^ku^n\dd u,
		\]
		\normalsize where $u = \cos x$ and $\dd u = -\sin x\dd x$. 
	\item	If $n$ is odd, then using substitutions similar to that outlined above we have \small
		\[\int \sin^mx\cos^nx\dd x = \int u^m(1-u^2)^k\dd u,\]
		\normalsize where $u = \sin x$ and $\dd u = \cos x\dd x$.
	\item	If both $m$ and $n$ are even, use the half-angle identities \small
		\[
		\cos^2x = \frac{1+\cos (2x)}{2} \quad \text{and}\quad \sin^2x = \frac{1-\cos(2x)}2
		\]
		\normalsize to reduce the degree of the integrand. Expand the result and apply the principles of this Key Idea again.
	\end{enumerate}
\end{keyidea}%
}

We practice applying \autoref{idea:trig_int_1} in the next examples.

\begin{example}[Integrating powers of sine and cosine]\label{ex_trigint1}%
Evaluate $\ds\int\sin^5x\cos^8x\dd x$.
\solution
The power of the sine factor is odd, so we rewrite $\sin^5x$ as
\[\sin^5x = \sin^4x\sin x = (\sin^2x)^2\sin x = (1-\cos^2x)^2\sin x.\]

Our integral is now $\ds \int (1-\cos^2x)^2\cos^8x\sin x\dd x$. Let $u = \cos x$, hence $\dd u = -\sin x\dd x$. Making the substitution and expanding the integrand gives
\begin{align*}
 \int (1-\cos^2x)^2\cos^8x\sin x\dd x
 &= -\int (1-u^2)^2u^8\dd u \\
 &= -\int \bigl(1-2u^2+u^4\bigr)u^8\dd u \\
 &= -\int \bigl(u^8-2u^{10}+u^{12}\bigr)\dd u \\
 &= -\frac19u^9 + \frac2{11}u^{11} - \frac1{13}u^{13} + C \\
 &=-\frac19\cos^9 x + \frac2{11}\cos^{11} x - \frac1{13}\cos^{13} x + C.
\end{align*}
\end{example}

\begin{example}[Integrating powers of sine and cosine]\label{ex_trigint2}%
Evaluate $\ds \int\sin^5x\cos^9x\dd x$.
\solution
Because the powers of both the sine and cosine factors are odd, we can apply the techniques of \autoref{idea:trig_int_1} to either power.
We choose to work with the power of the sine factor since that has a smaller exponent.
% We choose to work with the power of the cosine factor since the previous example used the sine factor's power.

We rewrite $\sin^5x$ as
\begin{align*}
 \sin^5x&=\sin^4x\sin x\\
 &=(1-\cos^2x)^2\sin x\\
 &=(1-2\cos^2x+\cos^4x)\sin x.
\end{align*}
This lets us rewrite the integral as
\[
\int\sin^5x\cos^9x\dd x=\int\bigl(1-2\cos^2x+\cos^4x\bigr)\sin x\cos^9x\dd x.
\]

Substituting and integrating with $u=\cos x$ and $\dd u=-\sin x\dd x$, we have
\begin{align*}
\int\bigl(1-2\cos^2x+\cos^4x\bigr)&\sin x\cos^9x\dd x\\
&=-\int\bigl(1-2u^2+u^4\bigr)u^9\dd u\\
&=-\int u^9-2u^{11}+u^{13}\dd u\\
&=-\frac1{10}u^{10}+\frac16u^{12}-\frac1{14}u^{14}+C\\
&=-\frac1{10}\cos^{10}x+\frac16\cos^{12}x-\frac1{14}\cos^{14}x+C.
\end{align*}

Instead, another approach would be to rewrite $\cos^9x$ as
\begin{align*} \cos^9 x &= \cos^8x\cos x \\
				&= (\cos^2x)^4\cos x \\
				&= (1-\sin^2x)^4\cos x \\
				&= (1-4\sin^2x+6\sin^4x-4\sin^6x+\sin^8x)\cos x.
\end{align*}

We rewrite the integral as 
\[\int\sin^5x\cos^9x\dd x = \int\bigl(\sin^5x\bigr)\bigl(1-4\sin^2x+6\sin^4x-4\sin^6x+\sin^8x\bigr)\cos x\dd x.\]

Now substitute and integrate, using $u = \sin x $ and $\dd u = \cos x\dd x$.
\begin{align*}
 \int & \bigl(\sin^5x\bigr)\bigl(1-4\sin^2x+6\sin^4x-4\sin^6x+\sin^8x\bigr)\cos x\dd x \\
 &=\int u^5(1-4u^2+6u^4-4u^6+u^8)\dd u \\
 &= \int\bigl(u^5-4u^7+6u^9-4u^{11}+u^{13}\bigr)\dd u \\
 &= \frac16u^6-\frac12u^8+\frac35u^{10}-\frac13u^{12}+\frac{1}{14}u^{14}+C\\
 &= \frac16\sin^6 x-\frac12\sin^8 x+\frac35\sin^{10} x-\frac13\sin^{12} x+\frac{1}{14}\sin^{14} x+C.
\end{align*}
%
\end{example}

\paragraph{Technology Note:} The work we are doing here can be a bit tedious, but the skills developed (problem solving, algebraic manipulation, etc.) are important. Nowadays problems of this sort are often solved using a computer algebra system. The powerful program \textit{Mathematica}\textsuperscript{\textregistered} integrates $\int \sin^5x\cos^9x\dd x$ as

{\small
\begin{multline*}
 f(x)=\\
 -\frac{45 \cos (2 x)}{16384}-\frac{5 \cos (4 x)}{8192}+\frac{19 \cos (6
   x)}{49152}+\frac{\cos (8 x)}{4096}-\frac{\cos (10 x)}{81920}-\frac{\cos (12
   x)}{24576}-\frac{\cos (14 x)}{114688},
\end{multline*}}
which clearly has a different form than our second answer in \autoref{ex_trigint2}, which is
%
\mtable{A plot of $f(x)$ and $g(x)$ from \autoref{ex_trigint2} and the Technology Note.}{fig:trigint2}{\pdftooltip{\begin{tikzpicture}
\begin{axis}[width=\marginparwidth,tick label style={font=\scriptsize},
axis y line=middle,axis x line=middle,name=myplot,axis on top,
ytick={-.002,.002,.004},yticklabels={$-0.002$,$0.002$,$0.004$},
ymin=-.003,ymax=0.005,xmin=-.1,xmax=3.15,scaled ticks=false]
% should we actually do this instead of coordinates?
\addplot [draw={\colortwo},thick,smooth] coordinates {(0,-0.0027879) (0.15708,-0.0027856) (0.31416,-0.0026798) (0.47124,-0.0020322) (0.62832,-0.00060997) (0.7854,0.00089518)(0.94248,0.0017233) (1.0996,0.0019485) (1.2566,0.0019734) (1.4137,0.0019741) (1.5708,0.0019741) (1.7279,0.0019741) (1.885,0.0019734) (2.042,0.0019485) (2.1991,0.0017233) (2.3562,0.00089518) (2.5133,-0.00060997) (2.6704,-0.0020322) (2.8274,-0.0026798) (2.9845,-0.0027856) (3.1416,-0.0027879)};
\draw (axis cs:2.6,0.004) node {\scriptsize $g(x)$};
\addplot [draw={\colorone},thick,smooth] coordinates {(0,0) (0.15708,0) (0.31416,0.00010807) (0.47124,0.0007557) (0.62832,0.0021779) (0.7854,0.003683) (0.94248,0.0045111)
(1.0996,0.0047364) (1.2566,0.0047612) (1.4137,0.0047619)
(1.5708,0.0047619) (1.7279,0.0047619) (1.885,0.0047612) (2.042,0.0047364) (2.1991,0.0045111) (2.3562,0.003683) (2.5133,0.0021779) (2.6704,0.0007557) (2.8274,0.00010807) (2.9845,0) (3.1416,0)};
\draw (axis cs:2.4,-0.002) node {\scriptsize $f(x)$};
\end{axis}
\node [right] at (myplot.right of origin) {\scriptsize $x$};
\node [above] at (myplot.above origin) {\scriptsize $y$};
\end{tikzpicture}}{ALT-TEXT-TO-BE-DETERMINED}}
\[
 g(x)=\frac16\sin^6 x-\frac12\sin^8 x+\frac35\sin^{10} x-\frac13\sin^{12} x+\frac{1}{14}\sin^{14} x.
\]
\autoref{fig:trigint2} shows a graph of $f$ and $g$; they are clearly not equal, but they differ \emph{only by a constant}: $g(x) = f(x) + C$ for some constant $C$. We have two different antiderivatives of the same function, meaning both answers are correct.\bigskip

%This is a common trigonometric integral.\bigskip

\begin{example}[Integrating powers of sine and cosine]\label{ex_sub8}%
Evaluate $\ds \int \sin^2x\dd x$.
\solution
The power of sine is even so we employ a half-angle identity, algebra and a u- substitution as follows:
\begin{align*}
	\int \sin^2x\dd x
	&= \int \frac{1-\cos(2x)}2\dd x \\
	&= \frac12\int 1-\cos(2x)\dd x \\
	&= \frac12\left(x - \frac12\sin(2x)\right)+C \\
	&= \frac12x - \frac14\sin(2x) + C.
\end{align*}
\end{example}

\begin{example}[Integrating powers of sine and cosine]\label{ex_trigint3}%
Evaluate $\ds\int\cos^4x\sin^2x\dd x$.
\solution
The powers of sine and cosine are both even, so we employ the half-angle formulas and algebra as follows.
\begin{align*}
	\int \cos^4x\sin^2x\dd x
	&= \int\left(\frac{1+\cos(2x)}{2}\right)^2\left(\frac{1-\cos(2x)}2\right)\dd x \\
	&= \int\frac{1+2\cos(2x)+\cos^2(2x)}4\cdot\frac{1-\cos(2x)}2\dd x\\
	&=	\int \frac18\bigl(1+\cos(2x)-\cos^2(2x)-\cos^3(2x)\bigr)\dd x
\end{align*}
The $\cos(2x)$ term is easy to integrate.
%, especially with \autoref{idea:linearsub}.
The $\cos^2(2x)$ term is another trigonometric integral with an even power, requiring the half-angle formula again. The $\cos^3(2x)$ term is a cosine function with an odd power, requiring a substitution as done before. We integrate each in turn below.

\begin{gather*}
\int\cos(2x)\dd x = \frac12\sin(2x)+C.\\
\int\cos^2(2x)\dd x = \int \frac{1+\cos(4x)}2\dd x
= \frac12\bigl(x+\frac14\sin(4x)\bigr)+C.
\end{gather*}

Finally, we rewrite $\cos^3(2x)$ as
\[\cos^3(2x) = \cos^2(2x)\cos(2x) = \bigl(1-\sin^2(2x)\bigr)\cos(2x).\]
Letting $u=\sin(2x)$, we have $\dd u = 2\cos(2x)\dd x$, hence
\begin{align*}
\int \cos^3(2x)\dd x &= \int\bigl(1-\sin^2(2x)\bigr)\cos(2x)\dd x\\
							&= \int \frac12(1-u^2)\dd u\\
							&= \frac12\Bigl(u-\frac13u^3\Bigr)+C\\
							&=	\frac12\Bigl(\sin(2x)-\frac13\sin^3(2x)\Bigr)+C
\end{align*}

Putting all the pieces together, we have
\begin{align*}
	\int &\cos^4x\sin^2x\dd x \\
	&=\int \frac18\bigl(1+\cos(2x)-\cos^2(2x)-\cos^3(2x)\bigr)\dd x \\
	&= \frac18
	\Bigl[x+\frac12\sin(2x)-\frac12\bigl(x+\frac14\sin(4x)\bigr)
	-\frac12\Bigl(\sin(2x)-\frac13\sin^3(2x)\Bigr)\Bigr]
	+C \\
	&=\frac18\Bigl[\frac12x-\frac18\sin(4x)+\frac16\sin^3(2x)\Bigr]+C.
\end{align*}
\end{example}

The process above was a bit long and tedious, but being able to work a problem such as this from start to finish is important.

\subsection{Integrals of the form \texorpdfstring{$\ds\int\tan^mx\sec^nx\dd x$}{∫(tan x)\^{}m (sec x)\^{}n dx}}

When evaluating integrals of the form $\int \sin^mx\cos^nx\dd x$, the Pythagorean Theorem allowed us to convert even powers of sine into even powers of cosine, and vice versa. If, for instance, the power of sine was odd, we pulled out one $\sin x$ and converted the remaining even power of $\sin x$ into a function using powers of $\cos x$, leading to an easy substitution.

The same basic strategy applies to integrals of the form $\int \tan^mx\sec^n x\dd x$, albeit a bit more nuanced. The following three facts will prove useful:
\begin{itemize}
\item $\frac{\dd}{\dd x}(\tan x) = \sec^2x$, 
\item $\frac{\dd}{\dd x}(\sec x) = \sec x\tan x$ , and 
\item	$1+\tan^2x = \sec^2x$ (the Pythagorean Theorem).
\end{itemize}

If the integrand can be manipulated to separate a $\sec^2x$ term with the remaining secant power even, or if a $\sec x\tan x$ term can be separated with the remaining $\tan x$ power even, the Pythagorean Theorem can be employed, leading to a simple substitution. This strategy is outlined in the following Key Idea.

{
\tcbset{grow to right by=13em}
\begin{keyidea}[Integrals Involving Powers of Tangent and Secant]\label{idea:trig_int_2}%
Consider $\ds\int\tan^mx\sec^nx\dd x$, where $m$ and $n$ are nonnegative integers.\index{integration!of trig. powers}
\begin{enumerate}
\item		If $n$ is even, then $n=2k$ for some integer $k$. Rewrite $\sec^nx$ as 
\[\sec^nx = \sec^{2k}x = \sec^{2k-2}x\sec^2x = (1+\tan^2x)^{k-1}\sec^2x.\]
Then
\[
\int\tan^mx\sec^nx\dd x=\int\tan^mx(1+\tan^2x)^{k-1}\sec^2x\dd x = \int u^m(1+u^2)^{k-1}\dd u,
\]
where $u = \tan x$ and $\dd u = \sec^2x\dd x$.

\item		If $m$ is odd and $n>0$, then $m=2k+1$ for some integer $k$. Rewrite $\tan^mx\sec^nx$ as
\[
\tan^mx\sec^nx = \tan^{2k+1}x\sec^nx = \tan^{2k}x\sec^{n-1}x\sec x\tan x = (\sec^2x-1)^k\sec^{n-1}x\sec x\tan x.
\]
Then
\[
\int\tan^mx\sec^nx\dd x=\int(\sec^2x-1)^k\sec^{n-1}x\sec x\tan x\dd x = \int(u^2-1)^ku^{n-1}\dd u,
\]
where $u = \sec x$ and $\dd u = \sec x\tan x\dd x$.

\item If $n$ is odd and $m$ is even, then $m=2k$ for some integer $k$. Convert $\tan^mx $ to $(\sec^2x-1)^k$. Expand the new integrand and use Integration By Parts, with $\dd v = \sec^2x\dd x$.

\item		If $m$ is even and $n=0$, rewrite $\tan^mx$ as
\[
\tan^mx = \tan^{m-2}x\tan^2x = \tan^{m-2}x(\sec^2x-1) = \tan^{m-2}\sec^2x-\tan^{m-2}x.
\]
So
\[
\int\tan^mx\dd x = \underbrace{\int\tan^{m-2}x\sec^2x\dd x}_{\text{\small apply rule \#1}}\quad - \underbrace{\int\tan^{m-2}x\dd x}_{\text{\small apply rule \#4 again}}.
\]

\end{enumerate}
\end{keyidea}
}

The techniques described in items 1 and 2 of \autoref{idea:trig_int_2} are relatively straightforward, but the techniques in items 3 and 4 can be rather tedious. A few examples will help with these methods.

\begin{example}[Integrating powers of tangent and secant]\label{ex_trigint5}%
Evaluate $\ds\int \tan^2x\sec^6x\dd x$.
\solution
Since the power of secant is even, we use rule \#1 from \autoref{idea:trig_int_2} and pull out a $\sec^2x$ in the integrand. We convert the remaining powers of secant into powers of tangent.
\begin{align*}
\int \tan^2x\sec^6x\dd x &= \int\tan^2x\sec^4x\sec^2x\dd x \\
		&= \int \tan^2x\bigl(1+\tan^2x\bigr)^2\sec^2x\dd x \\
\intertext{Now substitute, with $u=\tan x$, with $\dd u = \sec^2x\dd x$.}
		&=\int u^2\bigl(1+u^2\bigr)^2\dd u\\
\intertext{We leave the integration and subsequent substitution to the reader. The final answer is}
		&=\frac13\tan^3x+\frac25\tan^5x+\frac17\tan^7x+C.
\end{align*}
\end{example}

We derived integrals for tangent and secant in \autoref{sec:substitution} and will regularly use them when evaluating integrals of the form $\tan^m x \sec^n x \dd x$.  As a reminder:
\begin{align*}
 \int\tan x\dd x &=\ln\abs{\sec x}+C \\
 \int\sec x\dd x &=\ln\abs{\sec x+\tan x}+C
\end{align*}

\begin{example}[Integrating powers of tangent and secant]\label{ex_trigint6}%
Evaluate $\ds\int \sec^3x\dd x$.
\solution
We apply rule \#3 from \autoref{idea:trig_int_2} as the power of secant is odd and the power of tangent is even (0 is an even number). We use Integration by Parts; the rule suggests letting $\dd v = \sec^2x\dd x$, meaning that $u = \sec x$. \\
\noindent\begin{minipage}[t]{\linewidth}\noindent%
\captionsetup{type=figure}%
\[
\begin{aligned}
u&= \sec x & \dd v&=\sec^2 x\dd x\\
\dd u&= \text{?} & v&=\text{?}
\end{aligned}
\qquad\Rightarrow\qquad
\begin{aligned}
u&= \sec x & \dd v&=\sec^2 x\dd x\\
\dd u&= \sec x\tan x\dd x & v&=\tan x
\end{aligned}
\]
\caption{Setting up Integration by Parts.}\label{fig:trigint1}
\end{minipage}

Employing Integration by Parts, we have
\begin{align*}
\int \sec^3x\dd x
 	&=	\int \underbrace{\sec x}_u\cdot\underbrace{\sec^2 x\dd x}_{\dd v}\\
	&=	\sec x\tan x - \int \sec x\tan^2x\dd x. \\
\intertext{This new integral also requires applying rule \#3 of \autoref{idea:trig_int_2}:}
	&= \sec x\tan x - \int \sec x \bigl(\sec^2 x-1\bigr)\dd x\\
	&=	\sec x\tan x - \int \sec^3x\dd x + \int \sec x\dd x \\
	&= \sec x\tan x -\int \sec^3x\dd x + \ln\abs{\sec x+\tan x}
\end{align*}
%
\mnote{\textbf{Note:} Remember that in \autoref{ex_sub7}, we found that $\int\sec x\dd x=\ln\abs{\sec x+\tan x}+C$}
%
In previous applications of Integration by Parts, we have seen where the original integral has reappeared in our work. We resolve this by adding $\int \sec^3x\dd x$ to both sides, giving:
\begin{align*}
2\int \sec^3x\dd x &= \sec x\tan x + \ln\abs{\sec x+\tan x} \\
\int \sec^3x\dd x &= \frac12\Bigl(\sec x\tan x + \ln\abs{\sec x+\tan x}\Bigr)+C.
\end{align*}
\end{example}

We give one more example.

\begin{example}[Integrating powers of tangent and secant]\label{ex_trigint7}%
Evaluate $\ds\int\tan^6x\dd x$.
\solution
We employ rule \#4 of \autoref{idea:trig_int_2}. 
\begin{align*}
	\int \tan^6x\dd x
	&= \int \tan^4x\tan^2x\dd x \\
	&= \int\tan^4x\bigl(\sec^2x-1\bigr)\dd x\\
	&= \int\tan^4x\sec^2x\dd x - \int\tan^4x\dd x \\
\intertext{We integrate the first integral with substitution, $u=\tan x$ and $\dd u=\sec^2x\dd x$; and the second by employing rule \#4 again.}
	&= \int u^4\dd u-\int\tan^2 x\tan^2 x\dd x \\
	&=	\frac15\tan^5x-\int\tan^2x\bigl(\sec^2x-1\bigr)\dd x \\
	&= \frac15\tan^5x -\int\tan^2x\sec^2x\dd x + \int\tan^2x\dd x\\
\intertext{Again, use substitution for the first integral and rule \#4 for the second.}
	&= \frac15\tan^5x-\frac13\tan^3x+\int\bigl(\sec^2x-1\bigr)\dd x \\
	&=	 \frac15\tan^5x-\frac13\tan^3x+\tan x - x+C.
\end{align*}
\end{example}

\subsection{Integrals of the form \texorpdfstring{$\ds\int\cot^mx\csc^nx\dd x$}{∫(cot x)\^{}m (csc x)\^{}n dx}}

Not surprisingly, evaluating integrals of the form $\int\cot^mx\csc^nx\dd x$ is similar to evaluating $\int\tan^mx\sec^nx\dd x$. The guidelines from \autoref{idea:trig_int_2} and the following three facts will be useful:
\begin{align*}
 \frac{\dd}{\dd x}(\cot x) &= -\csc^2x \\
 \frac{\dd}{\dd x}(\csc x) &= -\csc x\cot x,\qquad\text{and} \\
 \csc^2 x &= \cot^2x+1
\end{align*}

\begin{example}[Integrating powers of cotangent and cosecant]\label{ex_int_cot_csc}%
Evaluate $\ds\int\cot^2x\csc^4x\dd x$
\solution
Since the power of cosecant is even we will let $u=\cot x$ and save a $\csc^2x$ for the resulting $\dd u=-\csc^2x\dd x$.
\begin{align*}
 \int\cot^2x\csc^4x\dd x
 &=\int\cot^2x\csc^2x\csc^2x\dd x \\
 &=\int\cot^2x(1+\cot^2x)\csc^2x\dd x \\
 &=-\int u^2(1+u^2)\dd u.
\end{align*}
The integration and substitution required to finish this example are similar to that of previous examples in this section. The result is
\[-\frac13\cot^3x-\frac15\cot^5x+C.\]
\end{example}

\subsection{Integrals of the form \texorpdfstring{$\ds \int\sin(mx)\sin(nx)\dd x,$ $\ds\int \cos(mx)\cos(nx)\dd x$, and $\ds\int \sin(mx)\cos(nx)\dd x$.}{∫sin(mx)sin(nx)dx, ∫cos(mx)cos(nx)dx, and ∫sin(mx)cos(nx)dx}}

Functions that contain products of sines and cosines of differing periods are important in many applications including the analysis of sound waves. Integrals of the form 
\[
\int\sin(mx)\sin(nx)\dd x,\quad \int \cos(mx)\cos(nx)\dd x \quad \text{and}\quad\int \sin(mx)\cos(nx)\dd x
\]
are best approached by first applying the Product to Sum Formulas of Trigonometry found in the back cover of this text, namely
\begin{align*}
\sin(mx)\sin(nx) &= \frac12\Bigl[\cos\bigl((m-n)x\bigr)-\cos\bigl((m+n)x\bigr)\Bigr] \\
\cos(mx)\cos(nx) &= \frac12\Bigl[\cos\bigl((m-n)x\bigr)+\cos\bigl((m+n)x\bigr)\Bigr] \\
\sin(mx)\cos(nx) &=	\frac12\Bigl[\sin\bigl((m-n)x\bigl)+\sin\bigl((m+n)x\bigr)\Bigr]
\end{align*}

\begin{example}[Integrating products of $\sin(mx)$ and $\cos(nx)$]\label{ex_trigint4}%
Evaluate $\ds\int\sin(5x)\cos(2x)\dd x$.
\solution
The application of the formula and subsequent integration are straightforward:
\begin{align*}
	\int\sin(5x)\cos(2x)\dd x
	&= \int \frac12\Bigl[\sin(3x)+\sin(7x)\Bigr]\dd x \\
	&= -\frac16\cos(3x) - \frac1{14}\cos(7x) + C.
\end{align*}
\end{example}

\subsection{Integrating other combinations of trigonometric functions}

Combinations of trigonometric functions that we have not discussed in this chapter are evaluated by applying algebra, trigonometric identities and other integration strategies to create an equivalent integrand that we can evaluate. To evaluate ``crazy'' combinations, those not readily manipulated into a familiar form, one should use integral tables. A table of ``common crazy'' combinations can be found at the end of this text.

These latter examples were admittedly long, with repeated applications of the same rule. Try to not be overwhelmed by the length of the problem, but rather admire how robust this solution method is. A trigonometric function of a high power can be systematically reduced to trigonometric functions of lower powers until all antiderivatives can be computed. 

The next section introduces an integration technique known as Trigonometric Substitution, a clever combination of Substitution and the Pythagorean Theorem.

\printexercises{exercises/06-03-exercises}

% the following was taken from u-substitution

%\subsection*{Substitution and Inverse Trigonometric Functions}
%
%%In \autoref{sec:deriv_inverse_function} 
%When studying derivatives of inverse functions, we learned that $$\frac{d}{dx}\big(\tan^{-1}x\big) = \frac{1}{1+x^2}.$$ Applying the Chain Rule to this is not difficult; for instance, $$\frac{d}{dx}\big(\tan^{-1}5x\big) = \frac{5}{1+25x^2}.$$ We now explore how Substitution can be used to ``undo'' certain derivatives that are the result of the Chain Rule applied to Inverse Trigonometric functions. We begin with an example.\\
%
%\example{ex_subst14}{Integrating by substitution: inverse trigonometric functions}{
%Evaluate $\ds \int \frac{1}{25+x^2}\ dx$.}
%{The integrand looks similar to the derivative of the arctangent function. Note:
%\begin{align*}
%\frac{1}{25+x^2} &= \frac{1}{25(1+\frac{x^2}{25})}\\
%							&= \frac{1}{25(1+\left(\frac{x}{5}\right)^2)} \\
%							&= \frac{1}{25}\frac{1}{1+\left(\frac{x}{5}\right)^2}\ .
%\end{align*}
%Thus $$\int\frac{1}{25+x^2}\ dx = \frac{1}{25}\int \frac{1}{1+\left(\frac{x}{5}\right)^2}\ dx.$$ This can be integrated using Substitution. Set $u = x/5$, hence $du = dx/5$ or $dx=5du$. Thus
%\begin{align*}
%	\int\frac{1}{25+x^2}\ dx
%	&= \frac{1}{25}\int \frac{1}{1+\left(\frac{x}{5}\right)^2}\ dx \\
%	&= \frac15\int \frac{1}{1+u^2}\ du \\
%	&= \frac15\tan^{-1}u + C \\
%	&= \frac15\tan^{-1}\left(\frac x5\right)+C
%\end{align*}}
%
%\autoref{ex_subst14} demonstrates a general technique that can be applied to other integrands that result in inverse trigonometric functions. The results are summarized here.
%
%\theorem{thm:int_inverse_trig}{Integrals Involving Inverse Trigonomentric Functions}
%{Let $a>0$.
%\begin{enumerate}
%	\item	$\ds \int \frac{1}{a^2+x^2}\ dx = \frac1a\tan^{-1}\left(\frac{x}{a}\right) + C$
%	\item	$\ds \int \frac{1}{\sqrt{a^2-x^2}}\ dx = \sin^{-1}\left(\frac{x}{a}\right)+C$
%	\item	$\ds \int \frac{1}{x\sqrt{x^2-a^2}}\ dx = \frac1a\sec^{-1}\left(\frac{|x|}{a}\right)+C$
%\end{enumerate}}
%
%Let's practice using \autoref{thm:int_inverse_trig}.\bigskip
%
%\example{ex_subst15}{Integrating by substitution: inverse trigonometric functions}{Evaluate the given indefinite integrals.
%$$\int \frac{1}{9+x^2}\ dx,\quad \int \frac{1}{x\sqrt{x^2-\frac{1}{100}}}\ dx\quad \text{ and }\quad  \int \frac{1}{\sqrt{5-x^2}}\ dx.$$}
%{Each can be answered using a straightforward application of \autoref{thm:int_inverse_trig}.
%\begin{gather*}
%\int \frac{1}{9+x^2}\ dx = \frac13\tan^{-1} \frac x3 + C,\text{ as }a=3.
%\int\frac1{x\sqrt{x^2-\frac1{100}}}\ dx=10\sec^{-1}10x+C,\text{ as }a=\frac1{10}.
%\int \frac{1}{\sqrt{5-x^2}} = \sin^{-1}\frac{x}{\sqrt{5}}+C,\text{ as }a = \sqrt{5}.}
%
%Most applications of \autoref{thm:int_inverse_trig} are not as straightforward. The next examples show some common integrals that can still be approached with this theorem.\\
%
%\example{ex_subst16}{Integrating by substitution: completing the square}{Evaluate $\ds \int\frac{1}{x^2-4x+13}\ dx$.}
%{Initially, this integral seems to have nothing in common with the integrals in \autoref{thm:int_inverse_trig}. As it lacks a square root, it almost certainly is not related to arcsine or arcsecant. It is, however, related to the arctangent function.
%
%We see this by \textit{completing the square} in the denominator. We give a brief reminder of the process here. 
%
%Start with a quadratic with a leading coefficient of 1. It will have the form of $x^2 + bx + c$. Take 1/2 of $b$, square it, and add/subtract it back into the expression. I.e., 
%\begin{align*}
%	x^2+bx+ c
%	&= \underbrace{x^2 + bx + \frac{b^2}4}_{(x+b/2)^2} - \frac{b^2}4 + c\\
%	&= \left(x+\frac b2\right)^2 + c-\frac{b^2}4
%\end{align*}
%In our example, we take half of $-4$ and square it, getting $4$. We add/subtract it into the denominator as follows:
%
%\begin{align*}
%	\frac{1}{x^2-4x+13}
%	&= \frac{1}{\underbrace{x^2-4x+4}_{(x-2)^2}-4+13}\\
%	&=\frac{1}{(x-2)^2 + 9}
%\end{align*}
%We can now integrate this using the arctangent rule. Technically, we need to substitute first with $u=x-2$, but by now we can do this easily. Thus we have 
%\[
%\int \frac{1}{x^2-4x+13}\ dx
%= \int \frac{1}{(x-2)^2+9}\ dx
%= \frac13\tan^{-1}\frac{x-2}{3}+C.
%\]}
%
%\example{ex_subst17}{Integrals requiring multiple methods}{
%Evaluate $\ds \int \frac{4-x}{\sqrt{16-x^2}}\ dx$.}
%{This integral requires two different methods to evaluate it. We get to those methods by splitting up the integral: 
%\[
%\int \frac{4-x}{\sqrt{16-x^2}}\ dx
%= \int \frac{4}{\sqrt{16-x^2}}\ dx - \int \frac{x}{\sqrt{16-x^2}}\ dx.
%\]
%The first integral is handled using a straightforward application of \autoref{thm:int_inverse_trig}; the second integral is handled by substitution, with $u = 16-x^2$. We handle each separately.
%\[\int \frac{4}{\sqrt{16-x^2}}\ dx = 4\sin^{-1}\frac{x}{4} + C.\]
%For $\ds \int\frac{x}{\sqrt{16-x^2}}\ dx$, we set $u = 16-x^2$, so $du = -2xdx$ and $xdx = -du/2$. We have
%\begin{align*}
%	\int\frac{x}{\sqrt{16-x^2}}\ dx
%	&= \int\frac{-du/2}{\sqrt{u}}\\
%	&= -\frac12\int \frac{1}{\sqrt{u}}\ du \\
%	&= - \sqrt{u} + C\\
%	&= -\sqrt{16-x^2} + C.
%\end{align*}
%Combining these together, we have 
%\[\int \frac{4-x}{\sqrt{16-x^2}}\ dx = 4\sin^{-1}\frac x4 + \sqrt{16-x^2}+C.\]}


\section{Trigonometric Substitution}\label{sec:trig_sub}

In \autoref{sec:def_int} we defined the definite integral as the ``signed area under the curve.'' In that section we had not yet learned the Fundamental Theorem of Calculus, so we evaluated special definite integrals which described nice, geometric shapes. For instance, we were able to evaluate
\begin{equation}
\int_{-3}^3\sqrt{9-x^2}\ dx = \frac{9\pi}{2}\label{eq:trigsub1}
\end{equation}
 as we recognized that $f(x) = \sqrt{9-x^2}$ described the upper half of a circle with radius 3. 

We have since learned a number of integration techniques, including Substitution and Integration by Parts, yet we are still unable to evaluate the above integral without resorting to a geometric interpretation. This section introduces Trigonometric Substitution, a method of integration that fills this gap in our integration skill. This technique works on the same principle as Substitution as found in \autoref{sec:substitution}, though it can feel ``backward.'' In \autoref{sec:substitution}, we set $u=f(x)$, for some function $f$, and replaced $f(x)$ with $u$. In this section, we will set $x=f(\theta)$, where $f$ is a trigonometric function, then replace $x$ with $f(\theta)$. 

\youtubeVideo{yW6Odu0YHL0}{Trigonometric Substitution --- Example 3 / Part 1}

We start by demonstrating this method in evaluating the integral in \autoeqref{eq:trigsub1}. After the example, we will generalize the method and give more examples.

\example{ex_trigsub1}{Using Trigonometric Substitution}{Evaluate $\ds \int_{-3}^3\sqrt{9-x^2}\ dx$.}
{We begin by noting that $9\sin^2\theta + 9\cos^2\theta = 9$, and hence $9\cos^2\theta = 9-9\sin^2\theta$. If we let $x=3\sin\theta$, then $9-x^2 = 9-9\sin^2\theta = 9\cos^2\theta$. 

Setting $x=3\sin \theta$ gives  $dx = 3\cos\theta\ d\theta$. We are almost ready to substitute. We also change our bounds of integration. The bound $x=-3$ corresponds to $\theta = -\pi/2$ (for when $\theta = -\pi/2$, $x=3\sin \theta = -3$). Likewise, the bound of $x=3$ is replaced by the bound $\theta = \pi/2$. Thus
\begin{align*}
	\int_{-3}^3\sqrt{9-x^2}\ dx
	&= \int_{-\pi/2}^{\pi/2} \sqrt{9-9\sin^2\theta} (3\cos\theta)\ d\theta \\
	&= \int_{-\pi/2}^{\pi/2} 3\sqrt{9\cos^2\theta} \cos\theta\ d\theta \\
	&=\int_{-\pi/2}^{\pi/2} 3\abs{3\cos \theta} \cos\theta\ d\theta.
	\intertext{On $[-\pi/2,\pi/2]$, $\cos \theta$ is always positive, so we can drop the absolute value bars, then employ a half--angle formula:}
	&= \int_{-\pi/2}^{\pi/2} 9\cos^2 \theta\ d\theta\\
	&= \int_{-\pi/2}^{\pi/2} \frac{9}{2}\big(1+\cos(2\theta)\big)\ d\theta\\
	& = \frac92 \left.\left(\theta +\frac12\sin(2\theta)\right)\right|_{-\pi/2}^{\pi/2}= \frac92\pi.
\end{align*}
This matches our answer from before.}

We now describe in detail Trigonometric Substitution. This method excels when dealing with integrands that contain $\sqrt{a^2-x^2}$, $\sqrt{x^2-a^2}$ and $\sqrt{x^2+a^2}$. The following Key Idea outlines the procedure for each case, followed by more examples.
% Each right triangle acts as a reference to help us understand the relationships between $x$ and $\theta$.

\keyidea{idea:trigsub}{Trigonometric Substitution}
{\mbox{}\\[-2\baselineskip]\begin{enumerate}
	\item[(a)] \noindent%
		For integrands containing $\sqrt{a^2-x^2}$:\index{integration!trig. subst.}\smallskip\\
		Let $x=a\sin\theta$, \quad for $-\pi/2\leq \theta\leq \pi/2$. \smallskip\\
	On this interval, $\cos\theta\geq 0$, so	$\sqrt{a^2-x^2} = a\cos\theta$
		
	\item[(b)] \noindent
		For integrands containing $\sqrt{x^2+a^2}$:\smallskip\\
		Let $x=a\tan\theta$, \quad for $-\pi/2 < \theta < \pi/2$. \smallskip\\
	On this interval, $\sec\theta> 0$, so $\sqrt{x^2+a^2} = a\sec\theta$
		
	\item[(c)] \noindent
		For integrands containing $\sqrt{x^2-a^2}$:\smallskip\\
		Let $x=a\sec\theta$, \quad restricting our work to where $x\geq a$,\\
		so $x/a\geq 1$, and $0\leq\theta<\pi/2$. \smallskip\\
	On this interval, $\tan\theta\geq 0$, so	$\sqrt{x^2-a^2} = a\tan\theta$
\end{enumerate}}

\example{ex_trigsub3}{Using Trigonometric Substitution}{Evaluate $\ds \int \frac{1}{\sqrt{5+x^2}}\ dx.$}
{Using \autoref{idea:trigsub}(b), we recognize $a=\sqrt{5}$ and  set $x= \sqrt{5}\tan \theta$. This makes $dx = \sqrt{5}\sec^2\theta\ d\theta$. We will use the fact that $\sqrt{5+x^2} = \sqrt{5+5\tan^2\theta} = \sqrt{5\sec^2\theta} = \sqrt{5}\sec\theta.$ Substituting, we have:
\begin{align*}
\int \frac{1}{\sqrt{5+x^2}}\ dx &= \int \frac{1}{\sqrt{5+5\tan^2\theta}}\sqrt{5}\sec^2\theta\ d\theta \\
			&= \int \frac{\sqrt{5}\sec^2\theta}{\sqrt{5}\sec\theta} \ d\theta\\
			&= \int \sec\theta\ d\theta\\
			&= \ln\abs{\sec\theta+\tan\theta}+C.
\end{align*}
While the integration steps are over, we are not yet done. The original problem was stated in terms of $x$, whereas our answer is given in terms of $\theta$. We must convert back to $x$.

\mtable{A reference triangle for \autoref{ex_trigsub3}}{fig:tan_tri}{\begin{tikzpicture}
	\draw [very thick] (0,0) -- node[below,pos=.5] {$\sqrt5$} (3,0)
	 -- node [right,pos=.5] {$x$} (3,2)
	 -- node [pos=.5,above,sloped] {$\sqrt{x^2+5}$} cycle;
    \draw [thick] (2.7,0) -- (2.7,.3) -- (3,.3);
	\draw (.75,.25) node {$\theta$};
\end{tikzpicture}}
The reference triangle in \autoref{fig:tan_tri} will help. With $x=\sqrt{5}\tan\theta$, we have 
$$\tan \theta = \frac x{\sqrt{5}}\quad \text{and}\quad \sec\theta = \frac{\sqrt{x^2+5}}{\sqrt{5}}.$$
This gives
\begin{align*}
	\int\frac1{\sqrt{5+x^2}}\ dx
	&= \ln\abs{\sec\theta+\tan\theta}+C \\
	&= \ln\abs{\frac{\sqrt{x^2+5}}{\sqrt5}+ \frac x{\sqrt5}}+C.
\end{align*}
We can leave this answer as is, or we can use a logarithmic identity to simplify it. Note:
\begin{align*}
	\ln\abs{\frac{\sqrt{x^2+5}}{\sqrt5}+ \frac x{\sqrt5}}+C
	&= \ln\abs{\frac1{\sqrt5}\left(\sqrt{x^2+5}+ x\right)}+C \\
	&= \ln\abs{\frac1{\sqrt5}} + \ln\abs{\sqrt{x^2+5}+ x}+C\\
	&=	\ln\abs{\sqrt{x^2+5}+ x}+C,
\end{align*}
where the $\ln\big(1/\sqrt5\big)$ term is absorbed into the constant $C$. (In \autoref{sec:hyperbolic} we learned another way of approaching this problem.)}

\example{ex_trigsub2}{Using Trigonometric Substitution}{Evaluate $\ds \int \sqrt{4x^2-1}\ dx$.}
{We start by rewriting the integrand so that it looks like $\sqrt{x^2-a^2}$ for some value of $a$:
\begin{align*}
\sqrt{4x^2-1} &= \sqrt{4\left(x^2-\frac14\right)}\\
		&= 2\sqrt{x^2-\left(\frac12\right)^2}.
\end{align*}
So we have $a=1/2$, and following \autoref{idea:trigsub}(c), we set $x= \frac12\sec\theta$, and hence $dx = \frac12\sec\theta\tan\theta\ d\theta$. %The Key Idea also shows that $\sqrt{x^2-1/2^2} = \frac12\tan\theta$. 
We now rewrite the integral with these substitutions:
\begin{align*}
\int \sqrt{4x^2-1}\ dx &= \int 2\sqrt{x^2-\left(\frac12\right)^2}\ dx\\
			&= \int 2\sqrt{\frac14\sec^2\theta - \frac14}\left(\frac12\sec\theta\tan\theta\right)\ d\theta\\
			&=\int \sqrt{\frac14(\sec^2\theta-1)}\Big(\sec\theta\tan\theta\Big)\ d\theta\\
			&=\int\sqrt{\frac14\tan^2\theta}\Big(\sec\theta\tan\theta\Big)\ d\theta\\
			&=\int \frac12\tan^2\theta\sec\theta\ d\theta\\
			&=\frac12\int \Big(\sec^2\theta-1\Big)\sec\theta\ d\theta\\
			&=\frac12\int \big(\sec^3\theta - \sec\theta\big)\ d\theta.
\end{align*}
We integrated $\sec^3\theta$ in \autoref{ex_trigint6}, finding its antiderivatives to be
$$\int \sec^3\theta\ d\theta = \frac12\Big(\sec \theta\tan \theta + \ln|\sec \theta+\tan \theta|\Big)+C.$$
Thus
\flushinnerequ{%
\begin{align*}
\int \sqrt{4x^2-1}\ dx &=\frac12\int \big(\sec^3\theta - \sec\theta\big)\ d\theta\\
			&= \frac12\left(\frac12\Big(\sec \theta\tan \theta + \ln|\sec \theta+\tan \theta|\Big) -\ln|\sec \theta + \tan\theta|\right) + C\\
			%\end{align*}
			%\begin{align*}
			&= \frac14\left(\sec\theta\tan\theta -\ln|\sec\theta+\tan\theta|\right)+C.
\end{align*}}
We are not yet done. Our original integral is given in terms of $x$, whereas our final answer, as given, is in terms of $\theta$. We need to rewrite our answer in terms of $x$. With $a=1/2$, and $x=\frac12\sec\theta$, the reference triangle in \autoref{fig:sec_tri} shows that
\mtable{A reference triangle for \autoref{ex_trigsub2}}{fig:sec_tri}{\begin{tikzpicture}
	\draw [very thick] (0,0) -- node[below,pos=.5] {$1/2$} (3,0)
	 -- node [right,pos=.5] {$\sqrt{x^2-1/4}$} (3,2)
	 -- node [pos=.5,above] {$x$} cycle;
    \draw [thick] (2.7,0) -- (2.7,.3) -- (3,.3);
	\draw (.75,.25) node {$\theta$};
\end{tikzpicture}}
$$\tan \theta = \frac{\sqrt{x^2-\frac14}}{\frac12} = 2\sqrt{x^2-\frac14}\qquad \text{and}\qquad\sec\theta = 2x.$$
Therefore,
\begin{align*}
	\int \sqrt{4x^2-1}\ dx
	& =\frac14\Big(\sec\theta\tan\theta -\ln\abs{\sec\theta+\tan\theta}\Big)+C \\
	&=\frac14\Big(2x\cdot 2\sqrt{x^2-\frac14} - \ln\abs{2x + 2\sqrt{x^2-\frac14}}\Big)+C\\
	&= \frac14\Big(4x\sqrt{x^2-\frac14}-\ln\abs{2x + 2\sqrt{x^2-\frac14}}\Big)+C \\
	& = \frac14\Big(2x\sqrt{4x^2-1} - \ln\abs{2x + \sqrt{4x^2-1}}\Big)+C.\eoehere
\end{align*}}

\example{ex_trigsub4}{Using Trigonometric Substitution}{Evaluate $\ds \int \frac{\sqrt{4-x^2}}{x^2}\ dx$.}
{We use \autoref{idea:trigsub}(a) with $a=2$, $x=2\sin \theta$, $dx = 2\cos \theta$ and hence $\sqrt{4-x^2} = 2\cos\theta$. This gives
\begin{align*}
\int \frac{\sqrt{4-x^2}}{x^2}\ dx &= \int \frac{2\cos\theta}{4\sin^2\theta}(2\cos\theta)\ d\theta\\
		&= \int \cot^2\theta\ d\theta\\
		&=	\int (\csc^2\theta -1)\ d\theta\\
		&= -\cot\theta -\theta + C.
\end{align*}
\mtable{A reference triangle for \autoref{ex_trigsub4}}{fig:sin_tri}{\begin{tikzpicture}
	\draw [very thick] (0,0) -- node[below,pos=.5] {$\sqrt{4-x^2}$} (3,0)
	 -- node [right,pos=.5] {$x$} (3,2) -- node [pos=.5,above] {$2$} cycle;
    \draw [thick] (2.7,0) -- (2.7,.3) -- (3,.3);
	\draw (.75,.25) node {$\theta$};
\end{tikzpicture}}
We need to rewrite our answer in terms of $x$. Using the reference triangle found in \autoref{fig:sin_tri}, we have $\cot\theta = \sqrt{4-x^2}/x$ and $\theta = \sin^{-1}(x/2)$. Thus
$$\int \frac{\sqrt{4-x^2}}{x^2}\ dx = -\frac{\sqrt{4-x^2}}x-\sin^{-1}\left(\frac x2\right) + C.\eoehere$$}

Trigonometric Substitution can be applied in many situations, even those not of the form $\sqrt{a^2-x^2}$, $\sqrt{x^2-a^2}$ or $\sqrt{x^2+a^2}$. In the following example, we apply it to an integral we already know how to handle.

\example{ex_trigsub5}{Using Trigonometric Substitution}{Evaluate $\ds \int\frac1{x^2+1}\ dx$.}
{We know the answer already as $\tan^{-1}x+C$. We apply Trig\-o\-no\-metric Substitution here to show that we get the same answer without inherently relying on knowledge of the derivative of the arctangent function.

Using \autoref{idea:trigsub}(b), let $x=\tan\theta$, $dx=\sec^2\theta\ d\theta$ and note that $x^2+1 = \tan^2\theta+1 = \sec^2\theta$. Thus
\begin{align*}
\int \frac1{x^2+1}\ dx &= \int \frac{1}{\sec^2\theta}\sec^2\theta\ d\theta \\
			&= \int 1\ d\theta\\
			&= \theta + C.
\end{align*}
Since $x=\tan \theta$, $\theta = \tan^{-1}x$, and we conclude that $\ds \int\frac1{x^2+1}\ dx = \tan^{-1}x+C.$}

The next example is similar to the previous one in that it does not involve a square--root. It shows how several techniques and identities can be combined to obtain a solution.

\example{ex_trigsub7}{Using Trigonometric Substitution}{Evaluate $\ds\int\frac1{(x^2+6x+10)^2}\ dx.$}
{We start by completing the square, then make the substitution $u=x+3$, followed by the trigonometric substitution of $u=\tan\theta$:
\begin{align}
\int \frac1{(x^2+6x+10)^2}\ dx =\int \frac1{\big((x+3)^2+1\big)^2}\ dx&= \int \frac1{(u^2+1)^2}\ du. \notag
\intertext{Now make the substitution $u=\tan\theta$, $du=\sec^2\theta\ d\theta$:}
   &=	\int \frac1{(\tan^2\theta+1)^2}\sec^2\theta\ d\theta\notag\\
	&= \int\frac 1{(\sec^2\theta)^2}\sec^2\theta\ d\theta\notag\\
	&= \int \cos^2\theta\ d\theta.\notag
	\intertext{Applying a half--angle formula, we have}
	&= \int \left(\frac12 +\frac12\cos(2\theta)\right)\ d\theta \notag\\
	&= \frac12\theta + \frac14\sin(2\theta) + C.\label{eq:extrigsub7}
\end{align}
\mnote{\textbf{Note:} Remember the sine and cosine double angle identities:
\begin{align*}
 \sin2\theta &= 2\sin\theta\cos\theta\\
 \cos2\theta &= \cos^2\theta-\sin^2\theta \\
 &= 2\cos^2\theta-1 \\
 &= 1-2\sin^2\theta
\end{align*}
They are often needed for writing your final answer in terms of $x$.}
We need to return to the variable $x$. As $u=\tan\theta$, $\theta = \tan^{-1}u$. Using the identity $\sin(2\theta) = 2\sin\theta\cos\theta$ and using the reference triangle found in \autoref{idea:trigsub}(b), we have 
$$\frac14\sin(2\theta) = \frac12\frac u{\sqrt{u^2+1}}\cdot\frac 1{\sqrt{u^2+1}} = \frac12\frac u{u^2+1}.$$
Finally, we return to $x$ with the substitution $u=x+3$. We start with the expression in \autoeqref{eq:extrigsub7}:
\begin{align*}
	\frac12\theta + \frac14\sin(2\theta) + C
	&= \frac12\tan^{-1}u + \frac12\frac{u}{u^2+1}+C\\
	&= \frac12\tan^{-1}(x+3) + \frac{x+3}{2(x^2+6x+10)}+C.
\end{align*}
Stating our final result in one line,
\[
\int\frac1{(x^2+6x+10)^2}\ dx=\frac12\tan^{-1}(x+3)+\frac{x+3}{2(x^2+6x+10)}+C.\eoehere
\]}

Our last example returns us to definite integrals, as seen in our first example. Given a definite integral that can be evaluated using Trigonometric Substitution, we could first evaluate the corresponding indefinite integral (by changing from an integral in terms of $x$ to one in terms of $\theta$, then converting back to $x$) and then evaluate using the original bounds. It is much more straightforward, though, to change the bounds as we substitute.

\example{ex_trigsub6}{Definite integration and Trigonometric Substitution}{Evaluate $\ds\int_0^5\frac{x^2}{\sqrt{x^2+25}}\ dx$.}
{Using \autoref{idea:trigsub}(b), we set $x=5\tan\theta$, $dx = 5\sec^2\theta\ d\theta$, and note that $\sqrt{x^2+25} = 5\sec\theta$. As we substitute, we change the bounds of integration.

The lower bound of the original integral is $x=0$. As $x=5\tan\theta$, we solve for $\theta$ and find $\theta = \tan^{-1}(x/5)$. Thus the new lower bound is $\theta = \tan^{-1}(0) = 0$. The original upper bound is $x=5$, thus the new upper bound is $\theta = \tan^{-1}(5/5) = \pi/4$. 

Thus we have 
\begin{align*}
\int_0^5\frac{x^2}{\sqrt{x^2+25}}\ dx &= \int_0^{\pi/4} \frac{25\tan^2\theta}{5\sec\theta}5\sec^2\theta\ d\theta\\
		&= 25\int_0^{\pi/4} \tan^2\theta\sec\theta\ d\theta.
\end{align*}
We encountered this indefinite integral in \autoref{ex_trigsub2} where we found 
$$\int \tan^2\theta\sec\theta \ d\theta = \frac12\big(\sec\theta\tan\theta-\ln|\sec\theta+\tan\theta|\big).$$
So
\begin{align*}
	25\int_0^{\pi/4} \tan^2\theta\sec\theta\ d\theta
	&= \left.\frac{25}2\left(\sec\theta\tan\theta-\ln\abs{\sec\theta+\tan\theta}\right)\right|_0^{\pi/4}\\
	&= \frac{25}2\big(\sqrt2-\ln(\sqrt2+1)\big)
%	\\
%	&\approx 6.661
	.\eoehere
\end{align*}}

%The following equalities are very useful when evaluating integrals using Trigonometric Substitution.
%
%\keyidea{idea:useful_trigsub}{Useful Equalities with Trigonometric Substitution}
%{\begin{enumerate}
%	\item	$\sin(2\theta) = 2\sin\theta\cos\theta$
%	\item	$\cos(2\theta) = \cos^2\theta - \sin^2\theta = 2\cos^2\theta-1 = 1-2\sin^2\theta$
%	\item $\ds \int \sec^3\theta\ d\theta = \frac12\Big(\sec \theta\tan \theta + \ln\big|\sec \theta+\tan \theta\big|\Big)+C$
%	\item	$\ds \int \cos^2\theta\ d\theta = \int \frac12\big(1+\cos(2\theta)\big)\ d\theta = \frac12\big(\theta+\sin\theta\cos\theta\big)+C.$
%\end{enumerate}}

The next section introduces Partial Fraction Decomposition, which is an algebraic technique that turns ``complicated'' fractions into sums of ``simpler'' fractions, making integration easier.

\printexercises{exercises/06_08_exercises}

% this was orphaned from u-substitution

%\begin{example}[Integration by substitution: simplifying first]\label{ex_sub9}
%Evaluate $\ds\int \frac{x^3+4x^2+8x+5}{x^2+2x+1}\dd x$.
%\solution
%One may try to start by setting $u$ equal to either the numerator or denominator; in each instance, the result is not workable. 
%
%When dealing with rational functions (i.e., quotients made up of polynomial functions), it is an almost universal rule that everything works better when the degree of the numerator is less than the degree of the denominator. Hence we use polynomial division.
%
%We skip the specifics of the steps, but note that when $x^2+2x+1$ is divided into $x^3+4x^2+8x+5$, it goes in $x+2$ times with a remainder of $3x+3$. Thus 
%\[\frac{x^3+4x^2+8x+5}{x^2+2x+1} = x+2 + \frac{3x+3}{x^2+2x+1}.\]
%Integrating $x+2$ is simple. The fraction can be integrated by setting $u = x^2+2x+1$, giving $\dd u = (2x+2)\dd x$. This is very similar to the numerator. Note that $\dd u/2 = (x+1)\dd x$ and then consider the following:
%\begin{align*} % this had ``rule'' commands scattered throughout
%\int \frac{x^3+4x^2+8x+5}{x^2+2x+1}\dd x
% & = \int \left(x+2 + \frac{3x+3}{x^2+2x+1}\right)\dd x \\
% &= \int (x+2)\dd x + \int \frac{3(x+1)}{x^2+2x+1}\dd x \\
% & = \frac12x^2+2x+C_1 + \int \frac{3}{u}\frac{\dd u}{2} \\
% &= \frac12x^2+2x+C_1 + \frac32\ln\abs u + C_2 \\
% &= \frac12x^2+2x+\frac32\ln\abs{x^2+2x+1} + C.
%\end{align*}
%In some ways, we ``lucked out'' in that after dividing, substitution was able to be done. In later sections we'll develop techniques for handling rational functions where substitution is not directly feasible.
%\end{example}



\section{Partial Fraction Decomposition}\label{sec:partial_fraction}

In this section we investigate the antiderivatives of rational functions. Recall that rational functions are functions of the form $f(x)= \frac{p(x)}{q(x)}$, where $p(x)$ and $q(x)$ are polynomials and $q(x)\neq 0$. Such functions arise in many contexts, one of which is the solving of certain fundamental differential equations.

We begin with an example that demonstrates the motivation behind this section. Consider the integral $\ds\int \frac{1}{x^2-1}\dd x$. We do not have a simple formula for this (if the denominator were $x^2+1$, we would recognize the antiderivative as being the arctangent function). It can be solved using Trigonometric Substitution, but note how the integral is easy to evaluate once we realize:

%This integral is not difficult to evaluate once one realizes the following fact: 
\[\frac{1}{x^2-1} = \frac{1/2}{x-1} - \frac{1/2}{x+1}.\]
Thus 
\begin{align*}
\int\frac{1}{x^2-1}\dd x &= \int\frac{1/2}{x-1}\dd x - \int\frac{1/2}{x+1}\dd x \\
			&= \frac12\ln\abs{x-1}-\frac12\ln\abs{x+1}+C.
\end{align*}

This section teaches how to \emph{decompose}
\[\frac1{x^2-1}\quad  \text{into}\quad  \frac{1/2}{x-1}-\frac{1/2}{x+1}.\]

We start with a rational function $f(x)=\dfrac{p(x)}{q(x)}$, where $p$ and $q$ do not have any common factors. We first consider the degree of $p$ and $q$. 
\begin{itemize}
	\item If the $\deg(p)\ge\deg(q)$ then we use polynomial long division to divide $q$ into $p$ to determine a remainder $r(x)$ where $\deg(r)<\deg(q)$. We then write $f(x) =s(x)+\dfrac{r(x)}{q(x)}$ and apply partial fraction decomposition to $\dfrac{r(x)}{q(x)}$.
	\item If the $\deg(p)<\deg(q)$ we can apply partial fraction decomposition to $\dfrac{p(x)}{q(x)}$ without additional work.
\end{itemize}

Partial fraction decomposition is based on an algebraic theorem that guarantees that any polynomial, and hence $q$, can use real numbers to factor into the product of linear and irreducible quadratic factors.
\mnote[-.5\baselineskip]{An \emph{irreducible quadratic} is one that cannot factor into linear terms with real coefficients.}\index{irreducible quadratic}
The following Key Idea states how to decompose a rational function into a sum of rational functions whose denominators are all of lower degree than $q$.

{
\tcbset{grow to right by=6em}
\begin{keyidea}[Partial Fraction Decomposition]\label{idea:partial_fraction}
Let $\dfrac{p(x)}{q(x)}$ be a rational function, where $\deg(p)<\deg(q)$.
\begin{enumerate}
	\item \textbf {Factor} $\mathbf{q(x):}$ Write $q(x)$ as the product of its linear and irreducible quadratic factors of the form $(ax+b)^m$ and $(ax^2+bx+c)^n$ where $m$ and $n$ are the highest powers of each factor that divide $q$.
	\begin{itemize}
		\item \textbf{Linear Terms:} For each linear factor of $q(x)$ the decomposition of $\dfrac{p(x)}{q(x)}$ will contain the following terms:
		\[\frac{A_1}{(ax+b)}+\frac {A_2}{(ax+b)^2}+\dotsb\frac{A_m}{(ax+b)^m}\]
		\item \textbf{Irreducible Quadratic Terms:}  For each irreducible quadratic factor of $q(x)$ the decomposition of $\dfrac {p(x)}{q(x)}$ will contain the following terms:
		\[
		 \dfrac{B_1x+C_1}{(ax^2+bx+c)}+\frac{B_2x+C_2}{(ax^2+bx+c)^2}
		 +\dotsb\frac{B_nx+C_n}{(ax^2+bx+c)^n}
		\]
	\end{itemize}
	\item \textbf{Finding the Coefficients $\mathbf{A_i}$, $\mathbf{B_i}$, and $\mathbf{C_i}$:}
	\begin{itemize}
		\item Set $\dfrac{p(x)}{q(x)}$ equal to the sum of its linear and irreducible quadratic terms.
		\[
		 \frac{p(x)}{q(x)}
		 =\frac{A_1}{(ax+b)}+\dotsb\frac{A_m}{(ax+b)^m}
		 +\frac{B_1x+C_1}{(ax^2+bx+c)}+\dotsb\frac{B_nx+C_n}{(ax^2+bx+c)^n}
		\]
		\item Multiply this equation by the factored form of $q(x)$ and simplify to clear the denominators. 
		\item Solve for the coefficients $A_i, B_i,$ and $C_i$ by
		\begin{enumerate}
			\item multiplying out the remaining terms and collecting like powers of $x$, equating the resulting coefficients and solving the resulting system of linear equations, \textbf{or}
			\item substituting in values for $x$ that eliminate terms so the simplified equation can be solved for a coefficient.
		\end{enumerate}
	\end{itemize}
\end{enumerate}
\end{keyidea}
}

\youtubeVideo{6qVgHWxdlZ0}{Integration Using method of Partial Fractions}

The following examples will demonstrate how to put this Key Idea into practice. In \autoref{ex_pf1}, we focus on the setting up the decomposition of a rational function.

\begin{example}[Decomposing into partial fractions]\label{ex_pf1}
Decompose $\ds f(x)=\frac{1}{(x+5)(x-2)^3(x^2+x+2)(x^2+x+7)^2}$ without solving for the resulting coefficients.
\solution
The denominator is already factored, as both $x^2+x+2$ and $x^2+x+7$ are irreducible quadratics. We need to decompose $f(x)$ properly. Since $(x+5)$ is a linear factor that divides the denominator, there will be a
\[\frac{A}{x+5}\]
term in the decomposition.

As $(x-2)^3$ divides the denominator, we will have the following terms in the decomposition:
\[\frac{B}{x-2},\quad \frac{C}{(x-2)^2}\quad \text{and}\quad \frac{D}{(x-2)^3}.\]

The $x^2+x+2$ term in the denominator results in a $\ds\frac{Ex+F}{x^2+x+2}$ term.

Finally, the $(x^2+x+7)^2$ term results in the terms
\[\frac{Gx+H}{x^2+x+7}\quad \text{and}\quad \frac{Ix+J}{(x^2+x+7)^2}.\]
All together, we have
\begin{multline*}
	\frac{1}{(x+5)(x-2)^3(x^2+x+2)(x^2+x+7)^2}= \\
	\frac{A}{x+5} + \frac{B}{x-2}+ \frac{C}{(x-2)^2}+\frac{D}{(x-2)^3}+\\
	\frac{Ex+F}{x^2+x+2}+\frac{Gx+H}{x^2+x+7}+\frac{Ix+J}{(x^2+x+7)^2}
\end{multline*}
Solving for the coefficients $A$, $B$, \dots, $J$ would be a bit tedious but not ``hard.''  In the next example we demonstrate solving for the coefficients using both methods given in \autoref{idea:partial_fraction}.
\end{example}

%(-37 x-39)/(140800 (x^2+x+2))+(67804 x+21113)/(520524225 (x^2+x+7))+(89 x-32)/(296595 (x^2+x+7)^2)+665617/(5015768576 (x-2))-1/(5501034 (x+5))-1119/(6889792 (x-2)^2)+1/(9464 (x-2)^3)

\begin{example}[Decomposing into partial fractions]\label{ex_pf2}
Perform the partial fraction decomposition of $\ds \frac{1}{x^2-1}$.
\solution
The denominator can be written as the product of two linear factors: $x^2-1 = (x-1)(x+1)$. Thus 
\begin{equation}\label{eq:decomp2}
 \frac{1}{x^2-1} = \frac{A}{x-1} + \frac{B}{x+1}.
\end{equation}
Using the method described in \autoref{idea:partial_fraction} 2(a) to solve for $A$ and $B$, first multiply through by $x^2-1 = (x-1)(x+1)$:
\begin{align}\label{eq:pf2}
	1
	&= \frac{A(x-1)(x+1)}{x-1}+\frac{B(x-1)(x+1)}{x+1} \notag\\
	&= A(x+1) + B(x-1)\\
	&= Ax+A + Bx-B \notag\\
	&= (A+B)x + (A-B)\qquad\text{collect like terms.}\notag
\end{align}
The next step is key. %Note the equality we have:
%\[1 = (A+B)x+(A-B).\]
For clarity's sake, rewrite the equality we have as
\[0x+1 = (A+B)x+(A-B).\]
On the left, the coefficient of the $x$ term is 0; on the right, it is $(A+B)$. Since both sides are equal for all values of $x$, we must have that $0=A+B$. Likewise, on the left, we have a constant term of 1; on the right, the constant term is $(A-B)$. Therefore we have $1=A-B$.

We have two linear equations with two unknowns. This one is easy to solve by hand, leading to 
\[
 \begin{aligned}A+B&=0\\A-B&=1\end{aligned}
 \qquad\Rightarrow\qquad
 \begin{aligned}A&=1/2\\B&=-1/2.\end{aligned}
\]
Thus
\[\frac{1}{x^2-1}=\frac{1/2}{x-1}-\frac{1/2}{x+1}.\]

Before solving for $A$ and $B$ using the method described in \autoref{idea:partial_fraction} 2(b), we note that Equations \eqref{eq:decomp2} and \eqref{eq:pf2} are not equivalent. Only the second equation holds for all values of $x$, including $x=-1$ and $x=1$, by continuity of polynomials. Thus, we can choose values for $x$ that eliminate terms in the polynomial to solve for $A$ and $B$.
\[1=A(x+1) + B(x-1).\]
If we choose $x=-1$,
\begin{align*}
 1&=A(0) + B(-2) \\
 B&=-\frac{1}{2}.
\end{align*}
Next choose $x=1$:
\begin{align*}
 1&=A(2) + B(0) \\
 A&=\frac{1}{2}.
\end{align*}
Resulting in the same decomposition as above.
\end{example}

In \autoref{ex_pf3}, we solve for the decomposition coefficients using the system of linear equations (method 2a). The margin note explains how to solve using substitution (method 2b).

\begin{example}[Integrating using partial fractions]\label{ex_pf3}
Use partial fraction decomposition to integrate $\ds\int\frac{1}{(x-1)(x+2)^2}\dd x$.
\solution
We decompose the integrand as follows, as described by \autoref{idea:partial_fraction}:
\begin{equation}\label{eq:decomp3}
 \frac{1}{(x-1)(x+2)^2} = \frac{A}{x-1} + \frac{B}{x+2} + \frac{C}{(x+2)^2}.
\end{equation}
To solve for $A$, $B$ and $C$, we multiply both sides by $(x-1)(x+2)^2$ and collect like terms:
%
\mnote{\textbf{Note:} Equations \eqref{eq:decomp3} and \eqref{eq:pf3} are not equivalent for $x=1$ and $x=-2$. However, due to the continuity of polynomials we can let $x=1$ to simplify the right hand side to $A(1+2)^2=9A$. Since the left hand side is still $1$, we have $1=9A$, so that $A=1/9$.\bigskip\\
Likewise,when $x=-2$; this leads to the equation $1=-3C$. Thus $C = -1/3$.\bigskip\\
Knowing $A$ and $C$, we can find the value of $B$ by choosing yet another value of $x$, such as $x=0$, and solving for $B$.}
%
\begin{align}
	1
	&= A(x+2)^2 + B(x-1)(x+2) + C(x-1)\label{eq:pf3}\\
	&= Ax^2+4Ax+4A + Bx^2 + Bx-2B + Cx-C \notag \\
	&= (A+B)x^2 + (4A+B+C)x + (4A-2B-C)\notag
\end{align}
We have
\[0x^2+0x+ 1 = (A+B)x^2 + (4A+B+C)x + (4A-2B-C)\]
leading to the equations 
\[A+B = 0, \quad 4A+B+C = 0 \quad \text{and} \quad 4A-2B-C = 1.\]
These three equations of three unknowns lead to a unique solution:
\[A = 1/9,\quad B = -1/9 \quad \text{and} \quad C = -1/3.\]
Thus 
\[
\int\frac{1}{(x-1)(x+2)^2}\dd x = \int \frac{1/9}{x-1}\dd x + \int \frac{-1/9}{x+2}\dd x + \int \frac{-1/3}{(x+2)^2}\dd x.
\]

Each can be integrated with a simple substitution with $u=x-1$ or $u=x+2$.
% (or by directly applying \autoref{idea:linearsub} as the denominators are linear functions).
The end result is
\[\int\frac{1}{(x-1)(x+2)^2}\dd x = \frac19\ln\abs{x-1}-\frac19\ln\abs{x+2} +\frac1{3(x+2)}+C.\]
\end{example}

\begin{example}[Integrating using partial fractions]\label{ex_pf4}
Use partial fraction decomposition to integrate $\ds \int \frac{x^3}{(x-5)(x+3)}\dd x$.
\solution
\autoref{idea:partial_fraction} presumes that the degree of the numerator is less than the degree of the denominator. Since this is not the case here, we begin by using polynomial division to reduce the degree of the numerator. We omit the steps, but encourage the reader to verify that
\[\frac{x^3}{(x-5)(x+3)} = x+2+\frac{19x+30}{(x-5)(x+3)}.\]
Using \autoref{idea:partial_fraction}, we can rewrite the new rational function as:
\[\frac{19x+30}{(x-5)(x+3)} = \frac{A}{x-5} + \frac{B}{x+3}\]
for appropriate values of $A$ and $B$. Clearing denominators, we have 
\[19x+30=A(x+3)+B(x-5).\]
As in the previous examples we choose values of $x$ to eliminate terms in the polynomial.  If we choose $x=-3$,
\begin{align*}
 19(-3)+30&=A(0) + B(-8) \\
 B&= \frac{27}{8}.
\end{align*}
Next choose $x=5$:
\begin{align*}
 19(5)+30&=A(8) + B(0)\\
 A&= \frac{125}{8}.
\end{align*}
We can now integrate:
\begin{align*}
	\int \frac{x^3}{(x-5)(x+3)}\dd x
	&= \int\left(x+2+\frac{125/8}{x-5}+\frac{27/8}{x+3}\right)\dd x \\
	&= \frac{x^2}2+2x+\frac{125}{8}\ln\abs{x-5}+\frac{27}8\ln\abs{x+3}+C.
\end{align*}
\end{example}

Before the next example we remind the reader of a rational integrand evaluated by trigonometric substitution:
\[\int\frac1{x^2+a^2}\dd x=\frac1a\tan^{-1}\left(\frac xa\right) + C.\]

\begin{example}[Integrating using partial fractions]\label{ex_pf5}
Use partial fraction decomposition to evaluate $\ds \int\frac{7x^2+31x+54}{(x+1)(x^2+6x+11)}\dd x$.
\solution
The degree of the numerator is less than the degree of the denominator so we begin by applying \autoref{idea:partial_fraction}. We have:
\begin{align*}
\frac{7x^2+31x+54}{(x+1)(x^2+6x+11)} &= \frac{A}{x+1} + \frac{Bx+C}{x^2+6x+11}. \\
\intertext{Now clear the denominators.}
7x^2+31x+54 &= A(x^2+6x+11) + (Bx+C)(x+1).
\end{align*}
Again, we choose values of $x$ to eliminate terms in the polynomial.  If we choose $x=-1$,
\begin{align*}
 30&=6A + (-B+C)(0)\\
 A&= 5.
\end{align*}

Although none of the other terms can be zeroed out, we continue by letting $A=5$ and substituting helpful values of $x$. 
Choosing $x=0$, we notice
\begin{align*}
 54&= 55 +C \\
 C&= -1.
\end{align*}
Finally, choose $x=1$ (any value other than $-1$ and $0$ can be used, $1$ is easy to work with)
\begin{align*}
 92&=90 + (B-1)(2)\\
 B&= 2.
\end{align*}
Thus
\[
 \int\frac{7x^2+31x+54}{(x+1)(x^2+6x+11)}\dd x
 = \int\left(\frac{5}{x+1} + \frac{2x-1}{x^2+6x+11}\right)\dd x.
\]

The first term of this new integrand is easy to evaluate; it leads to a $5\ln\abs{x+1}$ term. The second term is not hard, but takes several steps and uses substitution techniques.

The integrand $\dfrac{2x-1}{x^2+6x+11}$ has a quadratic in the denominator and a linear term in the numerator. This leads us to try substitution. Let $u=x^2+6x+11$, so $du=(2x+6)\dd x$. The numerator is $2x-1$, not $2x+6$, but we can get a $2x+6$ term in the numerator by adding 0 in the form of ``$7-7$.''
\begin{align*}
	\frac{2x-1}{x^2+6x+11} &= \frac{2x-1+7-7}{x^2+6x+11} \\
	&= \frac{2x+6}{x^2+6x+11} - \frac{7}{x^2+6x+11}.
\end{align*}
We can now integrate the first term with substitution, yielding $\ln\abs{x^2+6x+11}$. The final term can be integrated using arctangent. First, complete the square in the denominator:
\[\frac{7}{x^2+6x+11} = \frac{7}{(x+3)^2+2}.\]
An antiderivative of the latter term can be found using \autoref{idea:trigsub} and substitution:
\[
 \int \frac7{x^2+6x+11}\dd x=\frac7{\sqrt2}\tan^{-1}\left(\frac{x+3}{\sqrt2}\right)+C.
\]

Let's start at the beginning and put all of the steps together.
\begin{align*}
	\int&\frac{7x^2+31x+54}{(x+1)(x^2+6x+11)}\dd x \\
	&= \int\left(\frac{5}{x+1} + \frac{2x-1}{x^2+6x+11}\right)\dd x \\
	&= \int\frac{5}{x+1}\dd x  + \int\frac{2x+6}{x^2+6x+11}\dd x -\int\frac{7}{x^2+6x+11}\dd x \\
	&= 5\ln\abs{x+1}+\ln\abs{x^2+6x+11}-\frac7{\sqrt2}\tan^{-1}\left(\frac{x+3}{\sqrt2}\right)+C.
\end{align*}
As with many other problems in calculus, it is important to remember that one is not expected to ``see'' the final answer immediately after seeing the problem. Rather, given the initial problem, we break it down into smaller problems that are easier to solve. The final answer is a combination of the answers of the smaller problems.
\end{example}

Partial Fraction Decomposition is an important tool when dealing with rational functions. Note that at its heart, it is a technique of algebra, not calculus, as we are rewriting a fraction in a new form. Regardless, it is very useful in the realm of calculus as it lets us evaluate a certain set of ``complicated'' integrals.  The next section will require the reader to determine an appropriate method for evaluating a variety of integrals.

\printexercises{exercises/06_04_exercises}

\section{Improper Integration}\label{sec:improper_integration}

We begin this section by considering the following definite integrals:
\begin{itemize}
\item	$\ds \int_0^{100}\frac1{1+x^2}\ dx \approx 1.5608,$
\item	$\ds \int_0^{1000}\frac1{1+x^2}\ dx \approx 1.5698,$
\item	$\ds \int_0^{10,000}\frac1{1+x^2}\ dx \approx 1.5707.$
\end{itemize}

Notice how the integrand is $1/(1+x^2)$ in each integral (which is sketched in \autoref{fig:improper1}). As the upper bound gets larger, one would expect the ``area under the curve'' would also grow. While the definite integrals do increase in value as the upper bound grows, they are not  increasing by much. In fact, consider:
\[\int_0^b \frac{1}{1+x^2}\ dx = \tan^{-1}x\Big|_0^b = \tan^{-1}b-\tan^{-1}0 = \tan^{-1}b.\]
As $b\rightarrow \infty$, $\tan^{-1}b \rightarrow \pi/2.$ Therefore it seems that as the upper bound $b$ grows, the value of the definite integral $\ds \int_0^b\frac{1}{1+x^2}\ dx$ approaches $\pi/2\approx 1.5708$. This should strike the reader as being a bit amazing: even though the curve extends ``to infinity,'' it has a finite amount of area underneath it.

\mtable{Graphing $\ds f(x)=\frac{1}{1+x^2}$.}{fig:improper1}{\begin{tikzpicture}
\begin{axis}[width=1.16\marginparwidth,tick label style={font=\scriptsize},
axis y line=middle,axis x line=middle,name=myplot,axis on top,
ymin=-.1,ymax=1.1,xmin=-1,xmax=11]
\addplot[draw={\colorone},fill={\coloronefill},domain=0:11,area style,thick]
 {1/(1+x*x)} |-(axis cs:0,0);
\end{axis}
\node [right] at (myplot.right of origin) {\scriptsize $x$};
\node [above] at (myplot.above origin) {\scriptsize $y$};
\end{tikzpicture}}

When we defined the definite integral $\ds\int_a^b f(x)\ dx$, we made two stipulations:
	\begin{enumerate}
	\item		The interval over which we integrated, $[a,b]$, was a finite interval, and
	\item		The function $f(x)$ was continuous on $[a,b]$ (ensuring that the range of $f$ was finite).
	\end{enumerate}
	
In this section we consider integrals where one or both of the above conditions do not hold. Such integrals are called \textbf{improper integrals.}

\clearpage

\subsection{Improper Integrals with Infinite Bounds}

%\setboxwidth{40pt}
\definition{def:imp_int1}{Improper Integrals with Infinite Bounds}
{\index{integration!improper}\index{improper integration}\index{convergence!of improper int.}\index{divergence!of improper int.}
\begin{enumerate}
\item		Let $f$ be a continuous function on $[a,\infty)$. For $t \geq a$ let \[\int_a^\infty f(x)\ dx = \lim_{t\to\infty}\int_a^t f(x)\ dx.\]

\item		Let $f$ be a continuous function on $(-\infty,b]$. For $t \leq b$ let
\[\int_{-\infty}^b f(x)\ dx = \lim_{t\to-\infty}\int_t^b f(x)\ dx.\]

\item		Let $f$ be a continuous function on $(-\infty,\infty)$. For any real number $c$ (which one doesn't matter), let
\[
\int_{-\infty}^\infty f(x)\ dx
= \lim_{a\to-\infty}\int_a^c f(x)\ dx\ +\ \lim_{b\to\infty}\int_c^b f(x)\ dx.
\]
\end{enumerate}}

An improper integral is said to \textbf{converge} if its corresponding limit exists; otherwise, it \textbf{diverges}. The improper integral in part 3 converges if and only if both of its limits exist.

\youtubeVideo{f6cGotvktxs}{Improper Integral --- Infinity in Upper and Lower Limits}

\example{ex_impint1}{Evaluating improper integrals}{Evaluate the following improper integrals.\\
\begin{minipage}[t]{.5\textwidth}
\begin{enumerate}
\item		$\ds\int_1^\infty \frac1{x^2}\ dx$
\item		$\ds\int_1^\infty \frac1x\ dx$
\end{enumerate}
\end{minipage}%
\begin{minipage}[t]{.5\textwidth}
\begin{enumerate}\addtocounter{enumi}{2}
\item		$\ds\int_{-\infty}^0 e^x\ dx$
\item		$\ds\int_{-\infty}^\infty \frac1{1+x^2}\ dx$
\end{enumerate}
\end{minipage}}
{\begin{enumerate}
	\item	\mbox{}\\[-3\baselineskip]
	%
\mtable{A graph of $f(x) = \frac{1}{x^2}$ in\\\autoref{ex_impint1} part 1.}{fig:impint1a}{\begin{tikzpicture}
\begin{axis}[width=1.16\marginparwidth,tick label style={font=\scriptsize},
axis y line=middle,axis x line=middle,name=myplot,axis on top,xtick={1,5,10},
ymin=-.1,ymax=1.1,xmin=-1,xmax=11]
\addplot [draw={\colorone},fill={\coloronefill},area style,domain=1:10.5,thick]
 {1/(x*x)} |- (axis cs:1,0);
\draw (axis cs:5,.75) node {\scriptsize $\ds f(x)=\frac{1}{x^2}$};
\end{axis}
\node [right] at (myplot.right of origin) {\scriptsize $x$};
\node [above] at (myplot.above origin) {\scriptsize $y$};
\end{tikzpicture}}
%
	\begin{align*}
		\int_1^\infty \frac{1}{x^2}\ dx
		& = \lim_{t\to\infty} \int_1^t\frac1{x^2}\ dx \\
		& = \lim_{t\to\infty} \left.\frac{-1}{x}\right|_1^t \\ 
		%&= \lim_{t\to\infty} \frac{-1}{x}\Big|_1^t \\
		&= \lim_{t\to\infty} \frac{-1}{t} + 1\\
		&= 1.
	\end{align*}
	A graph of the area defined by this integral is given in \autoref{fig:impint1a}.

	\item	\mbox{}\\[-3\baselineskip]
%
\mtable{A graph of $f(x) = \frac{1}{x}$ in\\\autoref{ex_impint1} part 2.}{fig:impint1b}{\begin{tikzpicture}
\begin{axis}[width=1.16\marginparwidth,tick label style={font=\scriptsize},
axis y line=middle,axis x line=middle,name=myplot,axis on top,xtick={1,5,10},
ymin=-.1,ymax=1.1,xmin=-1,xmax=11]
\addplot [draw={\colorone},fill={\coloronefill},area style,domain=1:10.5,thick]
 {1/x} |- (axis cs:1,0);
\draw (axis cs:5,.75) node {\scriptsize $\ds f(x)=\frac{1}{x^2}$};
\end{axis}
\node [right] at (myplot.right of origin) {\scriptsize $x$};
\node [above] at (myplot.above origin) {\scriptsize $y$};
\end{tikzpicture}}
%
	\begin{align*}
		\int_1^\infty \frac1x\ dx
		& = \lim_{t\to\infty}\int_1^t\frac1x\ dx \\
		&= \lim_{t\to\infty} \ln \abs{x}\Bigr|_1^t \\
		&= \lim_{t\to\infty} \ln (t)\\
		&= \infty.
	\end{align*}
	The limit does not exist, hence the improper integral $\ds\int_1^\infty\frac1x\ dx$ diverges. Compare the graphs in Figures \ref{fig:impint1a} and \ref{fig:impint1b}; notice how the values of $f(x) = 1/x$ are noticeably larger than those of $f(x)=1/x^2$. This difference is enough to cause the improper integral to diverge.

	\item	\mbox{}\\[-3\baselineskip]
%
\mtable{A graph of $f(x) = e^x$ in\\\autoref{ex_impint1} part 3.}{fig:impint1c}{\begin{tikzpicture}
\begin{axis}[width=1.16\marginparwidth,tick label style={font=\scriptsize},
axis y line=middle,axis x line=middle,name=myplot,axis on top,xtick={-1,-5,-10},
ytick={1},ymin=-.1,ymax=1.1,xmin=-11,xmax=1]
\addplot [draw={\colorone},fill={\coloronefill},area style,domain=-10.5:0,thick]
 {exp(x)} |- (axis cs:0,0);
\draw (axis cs:-5,.75) node {\scriptsize $\ds f(x)=e^x$};
\end{axis}
\node [right] at (myplot.right of origin) {\scriptsize $x$};
\node [above] at (myplot.above origin) {\scriptsize $y$};
\end{tikzpicture}}
%
	\begin{align*}
		\int_{-\infty}^0 e^x \ dx
		&= \lim_{t\to-\infty} \int_t^0 e^x\ dx \\
		&=  \lim_{t\to-\infty} e^x\Big|_t^0 \\
		&= \lim_{t\to-\infty} e^0-e^t \\
		&= 1.
	\end{align*}
	A graph of the area defined by this integral is given in \autoref{fig:impint1c}.

\clearpage

	\item	We will need to break this into two improper integrals and choose a value of $c$ as in part 3 of \autoref{def:imp_int1}. Any value of $c$ is fine; we choose $c=0$.
%
\mtable{A graph of $f(x) = \frac{1}{1+x^2}$ in \autoref{ex_impint1} part 4.}{fig:impint1d}{\begin{tikzpicture}
\begin{axis}[width=1.16\marginparwidth,tick label style={font=\scriptsize},
axis y line=middle,axis x line=middle,name=myplot,axis on top,ytick={1},
ymin=-.1,ymax=1.1,xmin=-11,xmax=11]
\addplot [draw={\colorone},fill={\coloronefill},area style,domain=-10.5:10.5,thick]
 {1/(1+x*x)} |- (axis cs:0,0);
\draw (axis cs:6,.75) node {\scriptsize $\ds f(x)=\frac{1}{1+x^2}$};
\end{axis}
\node [right] at (myplot.right of origin) {\scriptsize $x$};
\node [above] at (myplot.above origin) {\scriptsize $y$};
\end{tikzpicture}}
%
\begin{align*}
	\int_{-\infty}^\infty \frac1{1+x^2}\ dx
	&= \lim_{t\to-\infty} \int_t^0\frac{1}{1+x^2}\ dx + \lim_{t\to\infty} \int_0^t\frac{1}{1+x^2}\ dx \\
	&= \lim_{t\to-\infty} \tan^{-1}x\Big|_t^0 + \lim_{t\to\infty} \tan^{-1}x\Big|_0^t\\
	&= \lim_{t\to-\infty} \left(\tan^{-1}0-\tan^{-1}t\right) + \lim_{t\to\infty} \left(\tan^{-1}t-\tan^{-1}0\right)\\		
	&= \left(0-\frac{-\pi}2\right) + \left(\frac{\pi}2-0\right).\\
	&= \pi.
\end{align*}
A graph of the area defined by this integral is given in \autoref{fig:impint1d}.\eoehere
\end{enumerate}}

\autoref{sec:lhopitals_rule} introduced L'H\^opital's Rule, a method of evaluating limits that return indeterminate forms. It is not uncommon for the limits resulting from improper integrals to need this rule as demonstrated next.

\example{ex_impint2}{Improper integration and L'H\^opital's Rule}{Evaluate the improper integral $\ds \int_1^\infty \frac{\ln x}{x^2}\ dx.$}
{This integral will require the use of Integration by Parts. Let $u = \ln x$ and $dv = 1/x^2\ dx$. Then
%
\mtable{A graph of $f(x) = \frac{\ln x}{x^2}$ in \autoref{ex_impint2}.}{fig:impint2}{\begin{tikzpicture}
\begin{axis}[width=1.16\marginparwidth,tick label style={font=\scriptsize},
axis y line=middle,axis x line=middle,name=myplot,axis on top,xtick={1,5,10},
ymin=-.1,ymax=.5,xmin=-1,xmax=11]
\addplot [draw={\colorone},fill={\coloronefill},area style,domain=1:10.5,thick]
 {ln(x)/(x*x)} |- (axis cs:1,0);
\draw (axis cs:5,.3) node {\scriptsize $\ds f(x)=\frac{\ln x}{x^2}$};
\end{axis}
\node [right] at (myplot.right of origin) {\scriptsize $x$};
\node [above] at (myplot.above origin) {\scriptsize $y$};
\end{tikzpicture}}
%
\begin{align*}
	\int_1^\infty\frac{\ln x}{x^2}\ dx
	&= \lim_{t\to\infty}\int_1^t\frac{\ln x}{x^2}\ dx \\
	&=  \lim_{t\to\infty}\left(-\frac{\ln x}{x}\Big|_1^t +\int_1^t \frac{1}{x^2} \ dx \right)\\
	&=  \lim_{t\to\infty} \left.\left(-\frac{\ln x}{x} -\frac1x\right)\right|_1^t\\
	&=	\lim_{t\to\infty} \left(-\frac{\ln t}{t}-\frac1t - \left(-\ln 1-1\right)\right).
\end{align*}
The $1/t$ goes to 0, and $\ln 1=0$, leaving
% $\ds \lim_{t\to\infty} -\frac{\ln t}t + 1$. We need to evaluate
$\ds \lim_{t\to\infty} \frac{\ln t}{t}$ with L'H\^opital's Rule. We have:
\[\lim_{t\to\infty}\frac{\ln t}t \LHequals \lim_{t\to\infty} \frac{1/t}{1} = 0.\]
Thus the improper integral evaluates as:
\[\int_1^\infty\frac{\ln x}{x^2}\ dx = 1.\eoehere\]}

\subsection{Improper Integrals with Infinite Range}

We have just considered definite integrals where the interval of integration was infinite. We now consider another type of improper integration, where the range of the integrand is infinite.


\definition{def:imp_int2}{Improper Integration with Infinite Range}
{Let $f(x)$ be a continuous function on $[a,b]$ except at $c$, $a\leq c\leq b$, where $x=c$ is a vertical asymptote of $f$. Define\index{integration!improper}\index{improper integration}
\[\int_a^b f(x)\ dx = \lim_{t\to c^-}\int_a^t f(x)\ dx + \lim_{t\to c^+}\int_t^b f(x)\ dx.\]
%The integral converges if both limits exist and diverges otherwise.
}

Note that $c$ can be one of the endpoints ($a$ or $b$). In that case, there is only one limit to consider as part of the definition.

\example{ex_impint3}{Improper integration of functions with infinite range}{Evaluate the following improper integrals:
\[
 \text{1. }\int_0^1\frac1{\sqrt{x}}\ dx\qquad\qquad
 \text{2. }\int_{-1}^1\frac{1}{x^2}\ dx.
\]}
{\begin{enumerate}
\item		A graph of $f(x) = 1/\sqrt{x}$ is given in \autoref{fig:impint3}.
%
\mtable{A graph of $f(x)=\frac{1}{\sqrt{x}}$ in \autoref{ex_impint3}.}{fig:impint3}{\begin{tikzpicture}
\begin{axis}[width=1.16\marginparwidth,tick label style={font=\scriptsize},
axis y line=middle,axis x line=middle,name=myplot,axis on top,
ymin=-.1,ymax=11,xmin=-.1,xmax=1.1]
\addplot [draw={\colorone},fill={\coloronefill},area style,thick,domain=.01:1] {1/sqrt(x)} |- (axis cs:.01,0);
\draw (axis cs:.5,5) node {\scriptsize $\ds f(x)=\frac{1}{\sqrt{x}}$};
\end{axis}
\node [right] at (myplot.right of origin) {\scriptsize $x$};
\node [above] at (myplot.above origin) {\scriptsize $y$};
\end{tikzpicture}}
%
Notice that $f$ has a vertical asymptote at $x=0$. In some sense, we are trying to compute the area of a region that has no ``top.'' Could this have a finite value? 
\begin{align*}
	\int_0^1 \frac{1}{\sqrt{x}}\ dx
	&= \lim_{t\to0^+}\int_t^1 \frac1{\sqrt{x}}\ dx \\
	&= \lim_{t\to0^+} 2\sqrt{x}\Big|_t^1 \\
	&= \lim_{t\to0^+} 2\left(\sqrt{1}-\sqrt{t}\right)\\
	&=	2.
\end{align*}
It turns out that the region does have a finite area even though it has no upper bound (strange things can occur in mathematics when considering the infinite).

\item		The function $f(x) = 1/x^2$ has a vertical asymptote at $x=0$, as shown in \autoref{fig:impint3b}, so this integral is an improper integral. Let's eschew using limits for a moment and proceed without recognizing the improper nature of the integral. This leads to:
\begin{align*}
\int_{-1}^1\frac1{x^2}\ dx &= -\frac1x\Big|_{-1}^1\\
			&= -1 - (1)\\
			&=-2.
\end{align*}
%
\mtable[-4\baselineskip]{A graph of $f(x)=\frac{1}{x^2}$ in \autoref{ex_impint3}.}{fig:impint3b}{\begin{tikzpicture}
\begin{axis}[width=1.16\marginparwidth,tick label style={font=\scriptsize},
axis y line=middle,axis x line=middle,name=myplot,axis on top,
ymin=-.1,ymax=11,xmin=-1.1,xmax=1.1]
\addplot [draw={\colorone},fill={\coloronefill},smooth,thick,domain=-1:-.2]
 {1/(x*x)} |-(axis cs:-1,0);
\draw[draw={\coloronefill},fill={\coloronefill}](axis cs:-.25,0)rectangle(axis cs:.25,11);
\addplot [draw={\coloronefill},fill={\coloronefill},smooth,thick,domain=.2:1]
 {1/(x*x)} |-(axis cs:.2,0);
% for some reason, the previous puts a \colorone bar on the right
\addplot [draw={\colorone},smooth,thick,domain=.2:1] {1/(x*x)};
\draw (axis cs:.7,7) node {\scriptsize $\ds f(x)=\frac{1}{x^2}$};
\end{axis}
\node [right] at (myplot.right of origin) {\scriptsize $x$};
\node [above] at (myplot.above origin) {\scriptsize $y$};
\end{tikzpicture}}
%
Clearly the area in question is above the $x$-axis, yet the area is supposedly negative. In this example we noted the discontinuity of the integrand on $[-1,1]$ (its improper nature) but continued anyway to apply the Fundamental Theorem of Calculus. Violating the hypothesis of the FTC led us to an incorrect area of $-2$. If we now evaluate the integral using \autoref{def:imp_int2} we will see that the area is unbounded.
\begin{align*}
	\int_{-1}^1\frac1{x^2}\ dx
	&= \lim_{t\to0^-}\int_{-1}^t \frac1{x^2}\ dx + \lim_{t\to0^+}\int_t^1\frac1{x^2}\ dx \\
	&= \lim_{t\to0^-}-\frac1x\Big|_{-1}^t + \lim_{t\to0^+}-\frac1x\Big|_t^1\\
	&= \lim_{t\to0^-}\left(-\frac1t+1\right) + \lim_{t\to0^+}\left(-1+\frac1t\right).
\end{align*}
Neither limit converges hence the original improper integral diverges. The nonsensical answer we obtained by ignoring the improper nature of the integral is just that: nonsensical.\eoehere
\end{enumerate}}

\subsection{Understanding Convergence and Divergence}

Oftentimes we are interested in knowing simply whether or not an improper integral converges, and not necessarily the value of a convergent integral. We provide here several tools that help determine the convergence or divergence of improper integrals without integrating.

Our first tool is knowing the behavior of functions of the form $\dfrac1{x\primeskip^p}$.

\example{ex_impint4}{Improper integration of $1/x^p$}{Determine the values of $p$ for which $\ds \int_1^\infty \frac1{x^p}\ dx$ converges.}
{We begin by integrating and then evaluating the limit.
\begin{align*}
	\int_1^\infty \frac1{x\primeskip^p}\ dx
	&= \lim_{t\to\infty}\int_1^t\frac1{x\primeskip^p}\ dx\\
	&= \lim_{t\to\infty}\int_1^t x^{-p}\ dx \qquad \text{\small (assume $p\neq 1$)}\\
	&= \lim_{t\to\infty} \frac{1}{-p+1}x^{-p+1}\Big|_1^t\\
	&= \lim_{t\to\infty} \frac{1}{1-p}\big(t^{1-p}-1^{1-p}\big).\\
\end{align*}
%
\mtable{Plotting functions of the form $1/x\,^p$ in \autoref{ex_impint4}.}{fig:impint4}{\begin{tikzpicture}
\begin{axis}[width=1.16\marginparwidth,tick label style={font=\scriptsize},
axis y line=middle,axis x line=middle,name=myplot,axis on top,xtick={1},
ytick=\empty,ymin=-.1,ymax=6,xmin=-.1,xmax=2.1]
\addplot [dashed,thick,smooth,domain=.1:2] {1/x};
\addplot [draw={\colorone},thick,smooth,domain=.1:2] {1/x^1.5};
\addplot [draw={\colortwo},thick,smooth,domain=.05:2] {1/sqrt(x)};
\draw (axis cs:.7,5) node {\scriptsize $\ds f(x)=\frac{1}{x\,^q}$};
\draw (axis cs:.35,.8) node {\scriptsize $\ds f(x)=\frac{1}{x\,^p}$};
\draw (axis cs:1.5,3) node {\scriptsize $p<1<q$};
\end{axis}
\node [right] at (myplot.right of origin) {\scriptsize $x$};
\node [above] at (myplot.above origin) {\scriptsize $y$};
\end{tikzpicture}}%
%
When does this limit converge -- i.e., when is this limit \textit{not} $\infty$? This limit converges precisely when the power of $b$ is less than 0: when $1-p<0 \Rightarrow 1<p$. 

Our analysis shows that if $p>1$, then $\ds\int_1^\infty \frac1{x\primeskip^p}\ dx $ converges. When $p<1$ the improper integral diverges; we showed in \autoref{ex_impint1} that when $p=1$ the integral also diverges. 

\autoref{fig:impint4} graphs $y=1/x$ with a dashed line, along with graphs of $y=1/x^p$, $p<1$, and $y=1/x^q$, $q>1$. Somehow the dashed line forms a dividing line between convergence and divergence. %A function of the form $1/x^q$ will be under the dashed line on $[1,\infty)$ when $q>1$. Even if $q$ is ``very close'' to 1, the difference will be enough to force convergence.
}

The result of \autoref{ex_impint4} provides an important tool in determining the convergence of other integrals. A similar result is proved in the exercises about improper integrals of the form $\ds \int_0^1\frac1{x\primeskip^p}\ dx$. These results are summarized in the following Key Idea.

\setboxwidth{80pt}
\keyidea{idea:impint1}{Convergence of Improper Integrals $\ds \int_1^\infty\frac1{x\primeskip^p}\ dx$ and $\ds \int_0^1\frac1{x\primeskip^p}\ dx$.}
{\index{convergence!of improper int.}\index{divergence!of improper int.}\begin{enumerate}
\item		The improper integral $\ds \int_1^\infty\frac1{x\primeskip^p}\ dx$ converges when $p>1$ and diverges when $p\leq 1.$
\item		The improper integral $\ds \int_0^1\frac1{x\primeskip^p}\ dx$ converges when $p<1$ and diverges when $p\geq 1.$
\end{enumerate}}

A basic technique in determining convergence of improper integrals is to compare an integrand whose convergence is unknown to an integrand whose convergence is known. We often use integrands of the form $1/x\primeskip^p$ in comparisons as their convergence on certain intervals is known. This is described in the following theorem.

\mnote{\textbf{Note:} We used the upper and lower bound of ``1'' in \autoref{idea:impint1} for convenience. It can be replaced by any $a$ where $a>0$. }

\theorem{thm:impint_comparison}{Direct Comparison Test for Improper Integrals}
{Let $f$ and $g$ be continuous on $[a,\infty)$ where $0\leq f(x)\leq g(x)$ for all $x$ in $[a,\infty)$. 
\index{integration!improper}\index{convergence!Direct Comparison Test!for integration}\index{divergence!Direct Comparison Test!for integration}\index{Direct Comparison Test!for integration}\index{convergence!of improper int.}\index{divergence!of improper int.}
	\begin{enumerate}
	\item		If $\ds \int_a^\infty g(x)\ dx$ converges, then $\ds \int_a^\infty f(x)\ dx$ converges.
	\item		If $\ds \int_a^\infty f(x)\ dx$ diverges, then $\ds \int_a^\infty g(x)\ dx$ diverges.
	\end{enumerate}

%\item		Let $f$ and $g$ be continuous functions on $[a,b]$ except at $x=c$, where each has a vertical asymptote, and $0\leq f(x)\leq g(x)$ for all $x$ in $[a,b]$, $x\neq c$.  
%	\begin{itemize}
%	\item		If $\ds \int_a^b g(x)\ dx$ converges, then $\ds \int_a^b f(x)\ dx$ converges.
%	\item		If $\ds \int_a^b f(x)\ dx$ diverges, then $\ds \int_a^b g(x)\ dx$ diverges.
%	\end{itemize}
%\end{itemize}
}

\example{ex_impint5}{Determining convergence of improper integrals}{Determine the convergence of the following improper integrals.
\[
 \text{1. }\int_1^\infty e^{-x^2}\ dx\qquad\qquad
 \text{2. }\int_3^\infty \frac{1}{\sqrt{x^2-x}}\ dx
\]}
{\begin{enumerate}
\item		The function $f(x) = e^{-x^2}$ does not have an antiderivative expressible in terms of elementary functions, so we cannot integrate directly. It is comparable to $g(x)=1/x^2$, and as demonstrated in \autoref{fig:impint5}, $e^{-x^2} < 1/x^2$ on $[1,\infty)$. We know from \autoref{idea:impint1} that $\ds \int_1^\infty \frac{1}{x^2}\ dx$ converges, hence $\ds\int_1^\infty e^{-x^2}\ dx$ also converges.

\mtable{Graphs of $f(x) = e^{-x^2}$ and $f(x)= 1/x^2$ in \autoref{ex_impint5}.}{fig:impint5}{\begin{tikzpicture}
\begin{axis}[width=1.16\marginparwidth,tick label style={font=\scriptsize},
axis y line=middle,axis x line=middle,name=myplot,axis on top,
ymin=-.1,ymax=1.1,xmin=-.1,xmax=4.1]
\addplot [draw={\colorone},thick,smooth,domain=1:4] {exp(-x*x)};
\addplot [draw={\colortwo},thick,smooth,domain=1:4] {1/(x*x)};
\draw (axis cs:.7,.45) node {\scriptsize $\ds f(x)=e^{-x^2}$};
\draw (axis cs:1.9,.8) node {\scriptsize $\ds f(x)=\frac{1}{x^2}$};
\end{axis}
\node [right] at (myplot.right of origin) {\scriptsize $x$};
\node [above] at (myplot.above origin) {\scriptsize $y$};
\end{tikzpicture}}

\item		Note that for large values of $x$, $\ds \frac{1}{\sqrt{x^2-x}} \approx \frac{1}{\sqrt{x^2}} =\frac{1}{x}$. We know from \autoref{idea:impint1} and the subsequent note that  $\ds \int_3^\infty \frac1x\ dx$ diverges, so we seek to compare the original integrand to $1/x$.

It is easy to see that when $x>0$, we have $x = \sqrt{x^2} > \sqrt{x^2-x}$. Taking reciprocals reverses the inequality, giving
\[\frac1x < \frac1{\sqrt{x^2-x}}.\]

Using \autoref{thm:impint_comparison}, we conclude that since $\ds\int_3^\infty\frac1x\ dx$ diverges, $\ds\int_3^\infty\frac1{\sqrt{x^2-x}}\ dx$ diverges as well. \autoref{fig:impint5b} illustrates this.\eoehere

\mtable{Graphs of $f(x) = 1/\sqrt{x^2-x}$ and $f(x)= 1/x$ in \autoref{ex_impint5}.}{fig:impint5b}{\begin{tikzpicture}
\begin{axis}[width=1.16\marginparwidth,tick label style={font=\scriptsize},
axis y line=middle,axis x line=middle,name=myplot,axis on top,
ymin=-.1,ymax=.5,xmin=-.1,xmax=6.2]
\addplot [draw={\colorone},thick,smooth,domain=3:6] {1/sqrt(x*x-x)};
\addplot [draw={\colortwo},thick,smooth,domain=3:6] {1/x};
\draw (axis cs:4,.45) node {\scriptsize $\ds f(x)=\frac{1}{\sqrt{x^2-x}}$};
\draw (axis cs:2.5,.2) node {\scriptsize $\ds f(x)=\frac{1}{x}$};
\end{axis}
\node [right] at (myplot.right of origin) {\scriptsize $x$};
\node [above] at (myplot.above origin) {\scriptsize $y$};
\end{tikzpicture}}
\end{enumerate}}

Being able to compare ``unknown'' integrals to ``known'' integrals is very useful in determining convergence. However, some of our examples were a little ``too nice.'' For instance, it was convenient that $\ds \frac{1}x < \frac{1}{\sqrt{x^2-x}}$, but what if the ``$-x$'' were replaced with a ``$+2x+5$''? That is, what can we say about the convergence of $\ds \int_3^\infty\frac{1}{\sqrt{x^2+2x+5}}\ dx$? We have $\ds \frac{1}{x} > \frac1{\sqrt{x^2+2x+5}}$, so we cannot use \autoref{thm:impint_comparison}.

In cases like this (and many more) it is useful to employ the following theorem.

\theorem{thm:impint_limit}{Limit Comparison Test for Improper Integrals}
{Let $f$ and $g$ be continuous functions on $[a,\infty)$ where $f(x)>0$ and $g(x)>0$ for all $x$. If
\[\lim_{x\to\infty} \frac{f(x)}{g(x)} = L,\qquad 0<L<\infty,\]
	then
	\[\int_a^\infty f(x)\ dx \quad \text{and} \quad \int_a^\infty g(x)\ dx\]
	either both converge or both diverge.%
	\index{integration!improper}\index{convergence!Limit Comparison Test!for integration}\index{divergence!Limit Comparison Test!for integration}\index{Limit Comparison Test!for integration}\index{convergence!of improper int.}\index{divergence!of improper int.}}

\example{ex_impint6}{Determining convergence of improper integrals}{Determine the convergence of $\ds \int_3^{\infty} \frac{1}{\sqrt{x^2+2x+5}}\ dx$.}
{As $x$ gets large, the quadratic function will begin to behave much like $y=x$. So we compare \small$\ds\frac{1}{\sqrt{x^2+2x+5}}$\normalsize\ to \small$\ds\frac1x$\normalsize\ with the Limit Comparison Test:
\[
\lim_{x\to\infty} \frac{1/\sqrt{x^2+2x+5}}{1/x}
= \lim_{x\to\infty}\frac{x}{\sqrt{x^2+2x+5}}.
\]

The immediate evaluation of this limit returns $\infty/\infty$, an indeterminate form. Using L'H\^opital's Rule seems appropriate, but in this situation, it does not lead to useful results. (We encourage the reader to employ L'H\^opital's Rule at least once to verify this.)

The trouble is the square root function.
%To get rid of it, we employ the following fact: If $\ds \lim_{x\to c} f(x) = L$, then $\ds\lim_{x\to c} f(x)^2 = L^2.$ (This is true when either $c$ or $L$ is $\infty$.) So we consider now the limit
%\[\lim_{x\to\infty} \frac{x^2}{x^2+2x+5}.\]
%This converges to 1, meaning the original limit also converged to 1. As $x$ gets very large, the function $\frac1{\sqrt{x^2+2x+5}}$ looks very much like $\frac1x$.
We determine the limit by using a technique we learned in Calculus I:
%
\mtable{Graphing $f(x)=\frac{1}{\sqrt{x^2+2x+5}}$ and $f(x)=\frac1x$ in \autoref{ex_impint6}.}{fig:impint6}{\begin{tikzpicture}
\begin{axis}[width=1.16\marginparwidth,tick label style={font=\scriptsize},
axis y line=middle,axis x line=middle,name=myplot,axis on top,
ymin=-.1,ymax=.35,xmin=-.1,xmax=21]
\addplot [draw={\colortwo},thick,smooth,domain=3:20] {1/sqrt(x*x+2*x+5)};
\addplot [draw={\colorone},thick,smooth,domain=3:20] {1/x};
\draw (axis cs:10,.3) node {\scriptsize $\ds f(x)=\frac{1}{\sqrt{x^2+2x+5}}$};
\draw (axis cs:4,.09) node {\scriptsize $\ds f(x)=\frac{1}{x}$};
\end{axis}
\node [right] at (myplot.right of origin) {\scriptsize $x$};
\node [above] at (myplot.above origin) {\scriptsize $y$};
\end{tikzpicture}}
%
\[
 \lim_{x\to\infty}\frac x{\sqrt{x^2+2x+5}}
 =\lim_{x\to\infty}\frac{\frac xx}{\sqrt{\frac{x^2+2x+5}{x^2}}}
 =\lim_{x\to\infty}\frac1{\sqrt{1+\frac2x+\frac5{x^2}}}=1
\]
Since we know that $\ds\int_3^{\infty} \tfrac1x\ dx$ diverges, by the Limit Comparison Test we know that $\ds\int_3^\infty\tfrac1{\sqrt{x^2+2x+5}}\ dx$ also diverges. \autoref{fig:impint6} graphs $f(x)=1/\sqrt{x^2+2x+5}$ and $f(x)=1/x$, illustrating that as $x$ gets large, the functions become in\-dis\-tin\-guish\-a\-ble.}

Both the Direct and Limit Comparison Tests were given in terms of integrals over an infinite interval. There are versions that apply to improper integrals with an infinite range, but as they are a bit wordy and a little more difficult to employ, they are omitted from this text.\bigskip

This chapter has explored many integration techniques. We learned
% Substitution, which reverses the Chain Rule of differentiation, as well as
Integration by Parts, which reverses the Product Rule of differentiation. We also learned specialized techniques for handling trigonometric and rational functions. All techniques effectively have this goal in common: rewrite the integrand in a new way so that the integration step is easier to see and implement.

As stated before, integration is, in general, hard. It is easy to write a function whose antiderivative is impossible to write in terms of elementary functions, and even when a function does have an antiderivative expressible by elementary functions, it may be really hard to discover what it is. The powerful computer algebra system \textit{Mathematica}\textsuperscript{\textregistered} has approximately 1,000 pages of code dedicated to integration. 

Do not let this difficulty discourage you. There is great value in learning integration techniques, as they allow one to manipulate an integral in ways that can illuminate a concept for greater understanding. There is also great value in understanding the need for good numerical techniques: the Trapezoidal and Simpson's Rules are just the beginning of powerful techniques for approximating the value of integration.\bigskip

%The next chapter stresses the uses of integration. We generally do not find antiderivatives for antiderivative's sake, but rather because they provide the solution to some type of problem. The following chapter introduces us to several different problems whose solution is provided by integration.

\printexercises{exercises/06_07_exercises}

\section{Numerical Integration}\label{sec:numerical_integration}

The Fundamental Theorem of Calculus gives a concrete technique for finding the exact value of a definite integral. That technique is based on computing antiderivatives. Despite the power of this theorem, there are still situations where we must \textit{approximate} the value of the definite integral instead of finding its exact value. The first situation we explore is where we \textit{cannot} compute an antiderivative of the integrand. The second case is when we actually do not know the integrand, but only its value when evaluated at certain points.\index{integration!numerical}\index{numerical integration} %While we handle both situations in the same way, we address them separately here.
%
%\subsection{An Antiderivative That Cannot Be Computed}
\bigskip

%\autoref{sec:antider} introduced antiderivatives and the indefinite integral; given a function, its indefinite integral is a set of \textit{functions}. In \autoref{sec:def_int}, we learned about definite integrals: given a function and two bounds, the definite integral computes the ``area under the curve,'' i.e., a \textit{number}. The Fundamental Theorem of Calculus states that these two concepts are related: we use antiderivatives to evaluate definite integrals. Definite integrals have immense importance in mathematics, science and engineering. We will explore some of these applications in \autoref{SEVEN}. 
%
%This method of evaluating definite integrals has one significant shortcoming: not all functions have an antiderivative expressible in terms of elementary functions.  (Elementary functions are combinations of polynomials, $n^{\text{th}}$ root, rational, exponential, logarithmic and trigonometric functions.). To be clear, we are not claiming that the antiderivatives are just \textit{hard} to find; rather, we are saying that we \textit{cannot} write them out using functions that we normally use. 
An \sword{elementary function}\index{elementary function} is any function that is a combination of polynomials, $n^{\text{th}}$ roots, rational, exponential, logarithmic and trigonometric functions and their inverses. We can compute the derivative of any elementary function, but there are many elementary functions of which we cannot compute an antiderivative. For example, the following functions do not have antiderivatives that we can express with elementary functions:
\[e^{-x^2}, \quad \sin(x^3)\quad \text{and} \quad \frac{\sin x}{x}.\]

The simplest way to refer to the antiderivatives of $e^{-x^2}$ is to simply write $\ds\int e^{-x^2}\ dx$. 

\mtable[-2in]{Graphically representing three definite integrals that cannot be evaluated using antiderivatives.}{fig:numerical1}{\begin{tikzpicture}
\begin{axis}[width=1.16\marginparwidth,tick label style={font=\scriptsize},
axis y line=middle,axis x line=middle,name=myplot,axis on top,
ymin=-.2,ymax=1.2,xmin=-.2,xmax=1.1]
\addplot[draw={\colorone},fill={\coloronefill},area style,domain=0:1]
 {exp(-x*x)}\closedcycle;
\draw (axis cs:.75,1) node {\scriptsize $y=e^{-x^2}$};
\end{axis}
\node [right] at (myplot.right of origin) {\scriptsize $x$};
\node [above] at (myplot.above origin) {\scriptsize $y$};
\end{tikzpicture}
\bigskip\\
\begin{tikzpicture}
\begin{axis}[width=1.16\marginparwidth,tick label style={font=\scriptsize},
axis y line=middle,axis x line=middle,name=myplot,axis on top,
ymin=-.6,ymax=1.2,xmin=-1,xmax=1.7]
\addplot[draw={\colorone},fill={\coloronefill},area style,domain=-pi/4:pi/2]
 {sin(deg(x^3))}\closedcycle;
\draw (axis cs:.6,.9) node {\scriptsize $y=\sin(x^3)$};
\end{axis}
\node [right] at (myplot.right of origin) {\scriptsize $x$};
\node [above] at (myplot.above origin) {\scriptsize $y$};
\end{tikzpicture}
\bigskip\\
\begin{tikzpicture}
\begin{axis}[width=1.16\marginparwidth,tick label style={font=\scriptsize},
axis y line=middle,axis x line=middle,name=myplot,axis on top,xtick={5,10,15},
ymin=-.3,ymax=1.1,xmin=-.9,xmax=15.9]
\addplot[draw={\colorone},fill={\coloronefill},area style,domain=.5:4*pi]
 {sin(deg(x))/x}\closedcycle;
\draw (axis cs:10,.7) node {\scriptsize $\displaystyle y=\frac{\sin x}{x}$};
\end{axis}
\node [right] at (myplot.right of origin) {\scriptsize $x$};
\node [above] at (myplot.above origin) {\scriptsize $y$};
\end{tikzpicture}}

%How, then, can we solve problems involving definite integrals of such functions? We \textit{approximate.} 
This section outlines three common methods of approximating the value of definite integrals. We describe each as a systematic method of approximating area under a curve. By approximating this area accurately, we find an accurate approximation of the corresponding definite integral.

We will apply the methods we learn in this section to the following definite integrals:
\[
 \int_0^1 e^{-x^2} \ dx, \quad
 \int_{-\frac{\pi}{4}}^{\frac{\pi}{2}} \sin(x^3) \ dx,
 \quad \text{and} \quad
 \int_{0.5}^{4\pi} \frac{\sin(x)}{x} \ dx,
\]
% todo Tim maybe use \pi/4 to \pi/2 in (b) instead?
as pictured in \autoref{fig:numerical1}.

\subsection{The Left and Right Hand Rule Methods}

In \autoref{sec:riemann} we addressed the problem of evaluating definite integrals by approximating the area under the curve using rectangles. We revisit those ideas here before introducing other methods of approximating definite integrals. \index{numerical integration!Left/Right Hand Rule}\index{Right Hand Rule}\index{Left Hand Rule}\index{integration!numerical!Left/Right Hand Rule}

We start with a review of notation. Let $f$ be a continuous function on the interval $[a,b]$. We wish to approximate $\ds \int_a^b f(x)\ dx$. We partition $[a,b]$ into $n$ equally spaced subintervals, each of length $\ds\Delta x = \frac{b-a}{n}$. The endpoints of these subintervals are labeled as
\[x_0=a,\ x_1 = a+\Delta x,\ x_2 = a+ 2\Delta x,\ \dotsc,\ x_i = a+i\Delta x,\ \dotsc,\ x_n = b.\]

%\autoref{idea:riemann}
\autoref{sec:riemann} showed that to use the Left Hand Rule we use the summation $\ds \sum_{i=1}^n f(x_{i-1})\Delta x$ and to use the Right Hand Rule we use $\ds \sum_{i=1}^n f(x_i)\Delta x$. We review the use of these rules in the context of examples.

\example{ex_num1}{Approximating definite integrals with rectangles}{Approximate $\ds \int_0^1e^{-x^2}\ dx$ using the Left and Right Hand Rules with 5 equally spaced subintervals.}
{We begin by partitioning the interval $[0,1]$ into 5 equally spaced intervals. We have $\Delta x = \frac{1-0}5 = 1/5=0.2$, so
\[x_0=0,\ x_1=0.2,\ x_2=0.4,\ x_3=0.6,\ x_4=0.8,\text{ and }x_5=1.\]

\mtable{Approximating $\ds\int_0^1e^{-x^2}\ dx$ in \autoref{ex_num1} using (top) the left hand rule and (bottom) the right hand rule.}{fig:num1}{\begin{tikzpicture}
\begin{axis}[width=1.16\marginparwidth,tick label style={font=\scriptsize},
axis y line=middle,axis x line=middle,name=myplot,axis on top,xtick={.2,.4,.6,.8,1},
ymin=-.2,ymax=1.2,xmin=-.2,xmax=1.1]
\addplot[draw={\colorone},fill={\coloronefill},area style,domain=0:1]
 {exp(-x*x)}\closedcycle;
\foreach \x in {0,0.2,...,.8} {
 \edef\y{exp(-\x*\x)}
 \edef\temp{\noexpand\draw[thick,draw={\colortwo}]
  (axis cs:\x,0)rectangle(axis cs:{\x+.2},{\y});
 }\temp
}
\draw (axis cs:.75,1) node {\scriptsize $y=e^{-x^2}$};
\end{axis}
\node [right] at (myplot.right of origin) {\scriptsize $x$};
\node [above] at (myplot.above origin) {\scriptsize $y$};
\end{tikzpicture}
\bigskip\\
\begin{tikzpicture}
\begin{axis}[width=1.16\marginparwidth,tick label style={font=\scriptsize},
axis y line=middle,axis x line=middle,name=myplot,axis on top,xtick={.2,.4,.6,.8,1},
ymin=-.2,ymax=1.2,xmin=-.2,xmax=1.1]
\addplot[draw={\colorone},fill={\coloronefill},area style,domain=0:1]
 {exp(-x*x)}\closedcycle;
\foreach \x in {0.2,0.4,...,1} {
 \edef\y{exp(-\x*\x)}
 \edef\temp{\noexpand\draw[thick,draw={\colortwo}]
  (axis cs:\x,0)rectangle(axis cs:{\x-.2},{\y});
 }\temp
}
\draw (axis cs:.75,1) node {\scriptsize $y=e^{-x^2}$};
\end{axis}
\node [right] at (myplot.right of origin) {\scriptsize $x$};
\node [above] at (myplot.above origin) {\scriptsize $y$};
\end{tikzpicture}}

Using the Left Hand Rule, we have:
\begin{align*}
	\sum_{i=1}^n f(x_{i-1})\Delta x
	&= \big(f(x_0)+f(x_1)+f(x_2) + f(x_3) + f(x_4)\big)\Delta x \\
	&= \big(f(0) + f(0.2) + f(0.4) + f(0.6) + f(0.8)\big)\Delta x \\
	&\approx \big(1+0.961 + 0.852 + 0.698 + 0.527)(0.2)\\
	&\approx 0.808.
\end{align*}

Using the Right Hand Rule, we have:
\begin{align*}
	\sum_{i=1}^n f(x_i)\Delta x
	&= \big(f(x_1)+f(x_2) + f(x_3) + f(x_4) + f(x_5)\big)\Delta x \\
	&= \big(f(0.2) + f(0.4) + f(0.6) + f(0.8)+f(1)\big)\Delta x \\
	&\approx \big(0.961 +0.852 + 0.698 + 0.527 + 0.368)(0.2)\\
	&\approx 0.681.
\end{align*}

\autoref{fig:num1} shows the rectangles used in each method to approximate the definite integral. These graphs show that in this particular case, the Left Hand Rule is an over approximation and the Right Hand Rule is an under approximation. To get a better approximation, we could use more rectangles, as we did in \autoref{sec:riemann}. We could also average the Left and Right Hand Rule results together, giving
\[\frac{0.808 + 0.681}{2} = 0.7445.\]
The actual answer, accurate to 4 places after the decimal, is 0.7468, showing our average is a good approximation.}

\mtable[-.5in]{Table of values used to approximate $\ds\int_{-\frac{\pi}4}^{\frac{\pi}2}\sin(x^3)\ dx$ in \autoref{ex_num2}.}{fig:num2a}{%
	\begin{tabular}{lrll}\lxBeginTableHead
	$x_i$ & Exact & Approx. & $\sin(x_i^3)$ \\\lxEndTableHead\midrule
	$x_0$ & $-\pi/4\phantom{0}$ & $-0.785$ & $-0.466$ \\
	$x_1$ & $-7\pi/40$ & $-0.550$ & $-0.165$ \\
	$x_2$ & $-\pi/10$ & $-0.314$ & $-0.031$ \\
	$x_3$ & $-\pi/40$ & $-0.0785$ & $\phantom{-}0$ \\
	$x_4$ & $\pi/20$ & $\phantom{-}0.157$ & $\phantom{-}0.004$ \\
	$x_5$ & $\pi/8\phantom{0}$ & $\phantom{-}0.393$ & $\phantom{-}0.061$ \\
	$x_6$ & $\pi/5\phantom{0}$ & $\phantom{-}0.628$ & $\phantom{-}0.246$ \\
	$x_7$ & $11\pi/40$ & $\phantom{-}0.864$ & $\phantom{-}0.601$ \\
	$x_8$ & $7\pi/20$ & $\phantom{-}1.10$ & $\phantom{-}0.971$ \\
	$x_9$ & $17\pi/40$ & $\phantom{-}1.34$ & $\phantom{-}0.690$ \\
	$x_{10}$ & $\pi/2\phantom{0}$ & $\phantom{-}1.57$ & $-0.670$
\end{tabular}}

\example{ex_num2}{Approximating definite integrals with rectangles}{Approximate $\ds\int_{-\frac{\pi}4}^{\frac{\pi}2} \sin (x^3)\ dx$ using the Left and Right Hand Rules with 10 equally spaced subintervals.}
{We begin by finding $\Delta x$:
\[\frac{b-a}{n} = \frac{\pi/2 - (-\pi/4)}{10} = \frac{3\pi}{40}\approx 0.236.\]
It is useful to write out the endpoints of the subintervals in a table; in \autoref{fig:num2a}, we give the exact values of the endpoints, their decimal approximations, and decimal approximations of $\sin(x^3)$ evaluated at these points. 

\mtable{Approximating\\ $\ds\int_{-\frac{\pi}4}^{\frac{\pi}2}\sin(x^3)\ dx$ in \autoref{ex_num2} using (top) the left hand rule and (bottom) the right hand rule.}{fig:num2b}{%
\begin{tikzpicture}
\begin{axis}[width=1.16\marginparwidth,tick label style={font=\scriptsize},
axis y line=middle,axis x line=middle,name=myplot,axis on top,
ymin=-.7,ymax=1.2,xmin=-1,xmax=1.7]
\addplot[draw={\colorone},fill={\coloronefill},area style,domain=-pi/4:pi/2]
 {sin(deg(x^3))}\closedcycle;
\foreach \x / \y in %
                {-0.785/ -0.466, -0.550/-0.165, -0.314/ -0.0310,%
                 -0.0785/-0.000484, 0.157/ 0.00388, 0.393/ 0.0605,
                 0.628/ 0.246, 0.864/ 0.601, 1.10/ 0.971, 1.34/ 0.690} {
 \addplot [thick,draw={\colortwo}] coordinates
  {(\x+.2356,0) (\x+.2356,\y) (\x,\y) (\x,0) (\x+.2356,0)};
}
\draw (axis cs:.6,.9) node {\scriptsize $y=\sin(x^3)$};
\end{axis}
\node [right] at (myplot.right of origin) {\scriptsize $x$};
\node [above] at (myplot.above origin) {\scriptsize $y$};
\end{tikzpicture}
\\
\begin{tikzpicture}
\begin{axis}[width=1.16\marginparwidth,tick label style={font=\scriptsize},
axis y line=middle,axis x line=middle,name=myplot,axis on top,
ymin=-.7,ymax=1.2,xmin=-1,xmax=1.7]
\addplot[draw={\colorone},fill={\coloronefill},area style,domain=-pi/4:pi/2]
 {sin(deg(x^3))}\closedcycle;
\foreach \x / \y in %
               {-0.550/-0.165, -0.314/ -0.0310,%
                -0.0785/-0.000484, 0.157/ 0.00388, 0.393/ 0.0605,
                0.628/ 0.246, 0.864/ 0.601, 1.10/ 0.971, 1.34/ 0.690, 1.57/ -0.670} {
  \addplot [thick,draw={\colortwo}] coordinates
    {(\x-.2356,0) (\x-.2356,\y) (\x,\y) (\x,0) (\x-.2356,0)};
}
\draw (axis cs:.6,1.1) node {\scriptsize $y=\sin(x^3)$};
\end{axis}
\node [right] at (myplot.right of origin) {\scriptsize $x$};
\node [above] at (myplot.above origin) {\scriptsize $y$};
\end{tikzpicture}}

Once this table is created, it is straightforward to approximate the definite integral using the Left and Right Hand Rules. (Note: the table itself is easy to create, especially with a standard spreadsheet program on a computer. The last two columns are all that are needed.) The Left Hand Rule sums the first 10 values of $\sin(x_i^3)$ and multiplies the sum by $\Delta x$; the Right Hand Rule sums the last 10 values of $\sin(x_i^3)$ and multiplies by $\Delta x$. Therefore we have:
\begin{align*}
	\text{Left Hand Rule: }\int_{-\frac{\pi}4}^{\frac{\pi}2}\sin(x^3)\ dx
	& \approx (1.91)(0.236) = 0.451. \\
	\text{Right Hand Rule: }\int_{-\frac{\pi}4}^{\frac{\pi}2}\sin(x^3)\ dx
	&\approx (1.71)(0.236) = 0.404.
\end{align*}

The average of the Left and Right Hand Rules is 0.4275.  The actual answer, accurate to 3 places after the decimal, is 0.460. Our approximations were once again fairly good. The rectangles used in each approximation are shown in \autoref{fig:num2b}. It is clear from the graphs that using more rectangles (and hence, narrower rectangles) should result in a more accurate approximation.}

\subsection{The Trapezoidal Rule}

In \autoref{ex_num1} we approximated the value of $\ds \int_0^1 e^{-x^2}\ dx$ with 5 rectangles of equal width. \autoref{fig:num1} showed the rectangles used in the Left and Right Hand Rules. These graphs clearly show that rectangles do not match the shape of the graph all that well, and that accurate approximations will only come by using lots of rectangles. \index{Trapezoidal Rule}\index{numerical integration!Trapezoidal Rule}\index{integration!numerical!Trapezoidal Rule}

\mtable{Approximating $\int_0^1 e^{-x^2}\ dx$ using 5 trapezoids of equal widths.}{fig:num3a}{\begin{tikzpicture}
\begin{axis}[width=1.16\marginparwidth,tick label style={font=\scriptsize},
axis y line=middle,axis x line=middle,name=myplot,axis on top,xtick={.2,.4,...,1},
ymin=-.2,ymax=1.2,xmin=-.2,xmax=1.1]
\addplot[draw={\colorone},fill={\coloronefill},area style,domain=0:1]
 {exp(-x*x)}\closedcycle;
\foreach\x in{0,.2,...,.8}{
  \edef\y{{exp(-\x*\x)}}
  \edef\z{{exp(-(\x+.2)*(\x+.2))}}
  \edef\temp{\noexpand\addplot[thick,draw={\colortwo}] coordinates
   {(\x,0) (\x,\y) ({\x+.2},\z) ({\x+.2},0) (\x,0)};
 }\temp
}
\draw (axis cs:.75,1) node {\scriptsize $y=e^{-x^2}$};
\end{axis}
\node [right] at (myplot.right of origin) {\scriptsize $x$};
\node [above] at (myplot.above origin) {\scriptsize $y$};
\end{tikzpicture}}

Instead of using rectangles to approximate the area, we can instead use \textit{trapezoids.} In \autoref{fig:num3a}, we show the region under $f(x) = e^{-x^2}$ on $[0,1]$ approximated with 5 trapezoids of equal width; the top ``corners'' of each trapezoid lie on the graph of $f(x)$. It is clear from this figure that these trapezoids more accurately approximate the area under $f$ and hence should give a better approximation of $\int_0^1 e^{-x^2}\ dx$. (In fact, these trapezoids seem to give a \textit{great} approximation of the area.)

\youtubeVideo{8z6JRFvjkpc}{The Trapezoid Rule for Approximating Integrals}

\mtable{The area of a trapezoid is $\frac{a+b}2h$.}{fig:trapezoid}{\begin{tikzpicture}
\draw(0,0)--node[pos=.5,left]{\small $a$}(0,1)--(1,1.5)
 --node[pos=.5,right]{\small$b$}(1,0)--node[pos=.5,below]{\small$h$}(0,0);
\draw(0,.1)--(.1,.1)--(.1,0);
\draw(.9,0)--(.9,.1)--(1,.1);
%\draw(2.5,.75)node{Area = };
\end{tikzpicture}}

The formula for the area of a trapezoid is given in \autoref{fig:trapezoid}. We approximate $\int_0^1 e^{-x^2}\ dx$ with these trapezoids in the following example.

\example{ex_num3}{Approximating definite integrals using trapezoids}{Use 5 trapezoids of equal width to approximate $\ds \int_0^1e^{-x^2}\ dx$.}
{To compute the areas of the 5 trapezoids in \autoref{fig:num3a}, it will again be useful to create a table of values as shown in \autoref{fig:num3b}.

\mtable{A table of values of $e^{-x^2}$.}{fig:num3b}{%
\begin{tabular}{ll}\lxBeginTableHead
$x_i$ & $e^{-x_i^2}$ \\\lxEndTableHead\midrule
0 & 1\\
0.2 & 0.961 \\
0.4 & 0.852 \\
0.6 & 0.698 \\
0.8 & 0.527 \\
1 & 0.368
\end{tabular}}

The leftmost trapezoid has legs of length 1 and 0.961 and a height of 0.2. Thus, by our formula, the area of the leftmost trapezoid is:
\[\frac{1+0.961}{2}(0.2) = 0.1961.\]
Moving right, the next trapezoid has legs of length 0.961 and 0.852 and a height of 0.2. Thus its area is:
\[\frac{0.961+0.852}2(0.2) = 0.1813.\]

The sum of the areas of all 5 trapezoids is:
\begin{align*}
\frac{1+0.961}{2}(0.2) + \frac{0.961+0.852}2(0.2)+\frac{0.852+0.698}2(0.2)& \\
+\frac{0.698+0.527}2(0.2)+\frac{0.527+0.368}2(0.2)&= 0.7445.
\end{align*}
We approximate $\ds \int_0^1 e^{-x^2}\ dx \approx 0.7445.$}

There are many things to observe in this example. Note how each term in the final summation was multiplied by both 1/2 and by $\Delta x = 0.2$. We can factor these coefficients out, leaving a more concise summation as:\\
\flushinnerequ{\small
\[
\frac12(0.2)\Big[(1+0.961) + (0.961+0.852) + (0.852+0.698) + ( 0.698+ 0.527) +(0.527 + 0.368)\Big].
\]}\\
Now notice that all numbers except for the first and the last are added twice. Therefore we can write the summation even more concisely as
\[\frac{0.2}{2}\Big[1 + 2(0.961+0.852+0.698+0.527) + 0.368\Big].\]

This is the heart of the \textbf{Trapezoidal Rule}, wherein a definite integral $\ds \int_a^b f(x) \ dx$ is approximated by using trapezoids of equal widths to approximate the corresponding area under $f$. Using $n$ equally spaced subintervals with endpoints $x_0$, $x_1$, \dots, $x_n$, we again have $\ds \Delta x = \frac{b-a}n$. Thus:
\begin{align*}
	\int_a^b f(x)\ dx
	& \approx \sum_{i=1}^n \frac{f(x_{i-1})+f(x_i)}2\Delta x \\
	& = \frac{\Delta x}2 \sum_{i=1}^n \big(f(x_{i-1})+f(x_i)\big)\\
	& = \frac{\Delta x}2\left[f(x_0)+ 2\sum_{i=1}^{n-1} f(x_i) + f(x_n)\right].
\end{align*}

\example{ex_num4}{Using the Trapezoidal Rule}{Revisit \autoref{ex_num2} 
and approximate $\ds\int_{-\frac{\pi}{4}}^{\frac{\pi}{2}} \sin (x^3)\ dx$ using the Trapezoidal Rule and 10 equally spaced subintervals.}
{We refer back to \autoref{fig:num2a} for the table of values of $\sin(x^3)$. Recall that $\Delta x = 3\pi/40 \approx 0.236$. Thus we have:\small
\begin{align*}
	\int_{-\frac{\pi}4}^{\frac{\pi}2} & \sin (x^3)\ dx \\
	&\approx \frac{0.236}{2}\Big[-0.466 + 2\Big(-0.165+(-0.031)+\dotsb+%0.971+
0.69\Big)+(-0.67)\Big]\\
	&= 0.4275.\eoehere
\end{align*}
\normalsize}

Notice how ``quickly'' the Trapezoidal Rule can be implemented once the table of values is created. This is true for all the methods explored in this section; the real work is creating a table of $x_i$ and $f(x_i)$ values. Once this is completed, approximating the definite integral is not difficult. Again, using technology is wise. Spreadsheets can make quick work of these computations and make using lots of subintervals easy. 

Also notice the  approximations the Trapezoidal Rule gives. It is  the average of the approximations given by the Left and Right Hand Rules! This effectively renders the Left and Right Hand Rules obsolete. They are useful when first learning about definite integrals, but if a real approximation is needed, one is generally better off using the Trapezoidal Rule instead of either the Left or Right Hand Rule.

We will also show that the Trapezoidal Rule makes using the Midpoint Rule obsolete as well.  With much more work, it will turn out that the Midpoint Rule has only a marginal gain in accuracy.  But we will include it in our results for the sake of completeness.\bigskip

How can we improve on the Trapezoidal Rule, apart from using more and more trapezoids? The answer is clear once we look back and consider what we have \textit{really} done so far. The Left Hand Rule is not \textit{really} about using rectangles to approximate area. Instead, it approximates a function $f$ with constant functions on small subintervals and then computes the definite integral of these constant functions. The Trapezoidal Rule is really approximating a function $f$ with a linear function on a small subinterval, then computes the definite integral of this linear function. In both of these cases the definite integrals are easy to compute in geometric terms.

So we have a progression: we start by approximating $f$ with a constant function and then with a linear function. What is next? A quadratic function. By approximating the curve of a function with lots of parabolas, we generally get an even better approximation of the definite integral. We call this process \textbf{Simpson's Rule}, named after Thomas Simpson (1710-1761), even though others had used this rule as much as 100 years prior.

\subsection{Simpson's Rule}

Given one point, we can create a constant function that goes through that point. Given two points, we can create a linear function that goes through those points. Given three points, we can create a quadratic function that goes through those three points (given that no two have the same $x$-value).\index{Simpson's Rule}\index{numerical integration!Simpson's Rule}\index{integration!numerical!Simpson's Rule}

Consider three points $(x_1,y_1)$, $(x_2,y_2)$ and $(x_3,y_3)$ whose $x$-values are equ\-ally spaced and $x_1<x_2<x_3$. Let $f$ be the quadratic function that goes through these three points. An exercise will ask you to show that 
\begin{equation}
\int_{x_1}^{x_3} f(x)\ dx = \frac{x_3-x_1}{6}\big(y_1+4y_2+y_3\big).\label{eq:simpsons}
\end{equation}

\mtable{A graph of a function $f$ and a parabola that approximates it well on $[1,3]$.}{fig:numsimpsons}{\begin{tikzpicture}
\begin{axis}[width=1.16\marginparwidth,tick label style={font=\scriptsize},
axis y line=middle,axis x line=middle,name=myplot,axis on top,xtick={1,2,3},
ytick={1,2,3},ymin=-.2,ymax=4,xmin=.5,xmax=3.5]
\addplot [draw={\colorone},domain=.9:3.1,thick] {11-12*x+4.5*x^2-.5*x^3};
\addplot [draw={\colortwo},dashed,domain=.9:3.1,thick] {8-6.5*x+1.5*x^2};
\filldraw (axis cs:1,3) circle (1pt);
\filldraw (axis cs:2,1) circle (1pt);
\filldraw (axis cs:3,2) circle (1pt);
\end{axis}
\node [right] at (myplot.right of origin) {\scriptsize $x$};
\node [above] at (myplot.above origin) {\scriptsize $y$};
\end{tikzpicture}}

Consider \autoref{fig:numsimpsons}. A function $f$ goes through the 3 points shown and the parabola $g$ that also goes through those points is graphed with a dashed line. Using our equation from above, we know exactly that
\[\int_1^3 g(x) \ dx = \frac{3-1}{6}\big(3+4(1)+2\big)= 3.\]
Since $g$ is a good approximation for $f$ on $[1,3]$, we can state that
\[\int_1^3 f(x)\ dx \approx 3.\]

Notice how the interval $[1,3]$ was split into two subintervals as we needed 3 points. Because of this, whenever we use Simpson's Rule, we need to break the interval into an even number of subintervals. 

In general, to approximate $\ds \int_a^b f(x)\ dx$ using Simpson's Rule, subdivide $[a,b]$ into $n$ subintervals, where $n$ is even and each subinterval has width $\Delta x = (b-a)/n$. We approximate $f$ with $n/2$ parabolic curves, using \autoeqref{eq:simpsons} to compute the area under these parabolas. Adding up these areas gives the formula:
\begin{multline*}
\int_a^b f(x) \ dx\approx\\
\frac{\Delta x}3
\left[f(x_0)+4f(x_1)+2f(x_2)+4f(x_3)+\dotsb+2f(x_{n-2})+4f(x_{n-1})+f(x_n)\right].
\end{multline*}
Note how the coefficients of the terms in the summation have the pattern 1, 4, 2, 4, 2, 4, \dots, 2, 4, 1.

Let's demonstrate Simpson's Rule with a concrete example.

\example{ex_num5}{Using Simpson's Rule}{Approximate $\ds\int_0^1 e^{-x^2}\ dx$ using Simpson's Rule and 4 equally spaced subintervals.}
{We begin by making a table of values as we have in the past, as shown in \autoref{fig:num5a}(a). Simpson's Rule states that
\mtable{A table of values to approximate $\int_0^1e^{-x^2}\ dx$ in \autoref{ex_num5}, along with a graph of the function.}{fig:num5a}{%
\begin{tabular}{ll}\lxBeginTableHead
$x_i$ & $e^{-x_i^2}$ \\\lxEndTableHead\midrule
0 & 1 \\
0.25 & 0.939 \\
0.5 & 0.779 \\
0.75 & 0.570 \\
1 & 0.368
\end{tabular}
\\ (a) \\
\begin{tikzpicture}
\begin{axis}[width=1.16\marginparwidth,tick label style={font=\scriptsize},
axis y line=middle,axis x line=middle,name=myplot,axis on top,xtick={.25,.5,.75,1},
ymin=-.2,ymax=1.2,xmin=-.2,xmax=1.1]
\addplot[draw={\colorone},fill={\coloronefill},area style,domain=0:1]
 {exp(-x*x)}\closedcycle;
\addplot [draw={\colortwo},domain=0:.5,thick] {1-0.042*x-0.8*x^2};
\addplot [draw={\colortwo},domain=.5:1,thick] {1.218-0.907*x+0.0569*x^2};
\addplot [thick,draw={\colortwo}] coordinates {(0,1) (0,0) (.5,0) (.5,.779)};
\addplot [thick,draw={\colortwo}] coordinates {(.5,0) (1,0) (1,.368)};
\draw (axis cs:.75,1) node {\scriptsize $y=e^{-x^2}$};
\filldraw (axis cs:0,1) circle (1pt)
          (axis cs:.25,.939) circle (1pt)
          (axis cs:.5,.779) circle (1pt)
          (axis cs:.75,.57) circle (1pt)
          (axis cs:1,.368) circle (1pt);
\end{axis}
\node [right] at (myplot.right of origin) {\scriptsize $x$};
\node [above] at (myplot.above origin) {\scriptsize $y$};
\end{tikzpicture}
\\ (b)}
\[
 \int_0^1e^{-x^2}\ dx
 \approx \frac{0.25}{3}\Big[1+4(0.939)+2(0.779)+4(0.570) + 0.368\Big]
 = 0.7468\overline{3}.
\]

Recall in \autoref{ex_num1} we stated that the correct answer, accurate to 4 places after the decimal, was 0.7468. Our approximation with Simpson's Rule, with 4 subintervals, is better than our approximation with the Trapezoidal Rule using 5.

\autoref{fig:num5a}(b) shows $f(x) = e^{-x^2}$ along with its approximating parabolas, demonstrating how good our approximation is. The approximating curves are nearly indistinguishable from the actual function.}

\example{ex_num6}{Using Simpson's Rule}{Approximate $\ds\int_{-\frac{\pi}4}^{\frac{\pi}2} \sin (x^3)\ dx$ using Simpson's Rule and 10 equally spaced intervals.}
{\mtable{A table of values to approximate $\int_{-\frac{\pi}4}^{\frac{\pi}2}\sin(x^3)\ dx$ in \autoref{ex_num6}, along with a graph of the function.}{fig:num6a}{%
\begin{tabular}{ll}\lxBeginTableHead
  $x_i$ & $\sin(x_i^3)$ \\\lxEndTableHead\midrule
  $-0.785$ & $-0.466$ \\
  $-0.550$ & $-0.165$   \\
  $-0.314$ & $-0.031$ \\
  $-0.0785$ & $\phantom{-}0$   \\
  $\phantom{-}0.157$ & $\phantom{-}0.004$ \\
  $\phantom{-}0.393$ & $\phantom{-}0.061$ \\
  $\phantom{-}0.628$ & $\phantom{-}0.246$ \\
  $\phantom{-}0.864$ & $\phantom{-}0.601$ \\
  $\phantom{-}1.10$ & $\phantom{-}0.971$ \\
  $\phantom{-}1.34$ & $\phantom{-}0.690$ \\
  $\phantom{-}1.57$ & $-0.670$ \\
\end{tabular}
\\ (a) \\
\begin{tikzpicture}
\begin{axis}[width=1.16\marginparwidth,tick label style={font=\scriptsize},
axis y line=middle,axis x line=middle,name=myplot,axis on top,
ymin=-.7,ymax=1.2,xmin=-1,xmax=1.7]
\addplot[draw={\colorone},fill={\coloronefill},area style,domain=-pi/4:pi/2]
 {sin(deg(x^3))}\closedcycle;
\addplot [draw={\colortwo},domain=-.785:-.314,thick] {-.11-.721*x-1.49*x^2};
\addplot [thick,draw={\colortwo}] coordinates {(-.785,-.466) (-.785,0) (-.314,0) (-.314,-.03)};
\addplot [draw={\colortwo},domain=-.314:.157,thick] {.004+.037*x-.236*x^2};
\addplot [thick,draw={\colortwo}] coordinates {(-.314,0) (.157,0) (.157,.004)};
\addplot [draw={\colortwo},domain=.157:.628,thick] {.037-.395*x+1.156*x^2};
\addplot [thick,draw={\colortwo}] coordinates {(.157,0) (.628,0) (.628,.246)};
\addplot [draw={\colortwo},domain=.628:1.1,thick] {-.632+1.32*x+.13*x^2};
\addplot [thick,draw={\colortwo}] coordinates {(.628,0) (1.1,0) (1.1,.971)};
\addplot [draw={\colortwo},domain=1.1:1.57,thick] {-11.98+22.46*x-9.72*x^2};
\addplot [thick,draw={\colortwo}] coordinates {(1.1,0) (1.57,0) (1.57,-.67)};
\filldraw (axis cs:-.785,-.466 ) circle (1pt)
          (axis cs:-.55,-.165 ) circle (1pt)
          (axis cs:-.314,-.031 ) circle (1pt)
          (axis cs:-.0785,0 ) circle (1pt)
          (axis cs:.157,.004 ) circle (1pt)
          (axis cs:.393,.061 ) circle (1pt)
          (axis cs:.628,.246 ) circle (1pt)
          (axis cs:.864,.601 ) circle (1pt)
          (axis cs:1.1,.971 ) circle (1pt)
          (axis cs:1.34,.69 ) circle (1pt)
          (axis cs:1.57,-.67 ) circle (1pt);
\draw (axis cs:.6,.9) node {\scriptsize $y=\sin(x^3)$};
\end{axis}
\node [right] at (myplot.right of origin) {\scriptsize $x$};
\node [above] at (myplot.above origin) {\scriptsize $y$};
\end{tikzpicture}
\\ (b)}%
\autoref{fig:num6a}(a) shows the table of values that we used in the past for this problem, shown here again for convenience. Again, $\Delta x = (\pi/2+\pi/4)/10 \approx 0.236$.

Simpson's Rule states that
{\small\begin{align*}
	\int_{-\frac{\pi}4}^{\frac{\pi}2} \sin (x^3)\ dx
	&\approx \frac{0.236}3\Big[(-0.466)+4(-0.165)+2(-0.031) + \dotsb \\
	&\qquad\qquad \dotsb + 2(0.971) + 4(0.69) + (-0.67)\Big]\\
	&= 0.4701
\end{align*}}

Recall that the actual value, accurate to 3 decimal places, is 0.460. Our approximation is within one 1/100$^\text{th}$ of the correct value. The graph in \autoref{fig:num6a}(b) shows how closely the parabolas match the shape of the graph.}

%\noindent\begin{minipage}[t]{\linewidth}\noindent%
%\captionsetup{type=figure}%
%
%\caption{Approximating $\int_{-\frac{\pi}4}^{\frac{\pi}2}\sin(x^3)\ dx$ in \autoref{ex_num6} with Simpson's Rule and 10 equally spaced intervals.}
%\label{fig:num6b}
%\end{minipage}

\subsection{Summary and Error Analysis}

We summarize the key concepts of this section thus far in the following Key Idea.

{
\tcbset{grow to right by=140pt}
\begin{keyidea}[Numerical Integration]\label{idea:numerical}
Let $f$ be a continuous function on $[a,b]$, let $n$ be a positive integer, and let $\ds\Delta x = \frac{b-a}{n}$.
Set $x_0=a$, $x_1=a+\Delta x$, \dots, $x_i = a+i\Delta x$, $x_n=b$.
Consider $\ds\int_a^b f(x)\ dx$.\\
\index{integration!numerical!Left Hand Rule}
\index{integration!numerical!Right Hand Rule}
\index{integration!numerical!Midpoint Rule}
\index{integration!numerical!Trapezoidal Rule}
\index{integration!numerical!Simpson's Rule}
\index{Left/Right Hand Rule}\index{Midpoint Rule}
\index{Trapezoidal Rule}\index{Simpson's Rule}
\index{numerical integration!Left Hand Rule}
\index{numerical integration!Right Hand Rule}
\index{numerical integration!Midpoint Rule}
\index{numerical integration!Trapezoidal Rule}
\index{numerical integration!Simpson's Rule}
\begin{tabular}{ll}
Left Hand Rule: &
$\ds\int_a^b f(x)\ dx \approx \Delta x\Big[f(x_0) + f(x_1) + \dotsb + f(x_{n-1})\big]$.\\
Right Hand Rule: &
$\ds\int_a^b f(x)\ dx \approx \Delta x\Big[f(x_1) + f(x_2) + \dotsb + f(x_n)\big]$.\\
Midpoint Rule: &
$\ds\int_a^b f(x)\ dx \approx \Delta x\Big[f\big(\frac{x_0+x_1}2\big) + f\big(\frac{x_1+x_2}2\big) + \dotsb + f\big(\frac{x_{n-1}+x_n}2\big)\big]$.\\
Trapezoidal Rule: &
$\ds\int_a^b f(x)\ dx \approx \frac{\Delta x}2\Big[f(x_0) + 2f(x_1) + 2f(x_2) +\dotsb + 2f(x_{n-1})+ f(x_n)\big]$.\\
Simpson's Rule: &
$\ds\int_a^b f(x)\ dx \approx \frac{\Delta x}3\Big[f(x_0) + 4f(x_1) + 2f(x_2) +\dotsb + 4f(x_{n-1})+ f(x_n)\big]$ {\small ($n$ even)}.
\end{tabular}
\end{keyidea}
}

In our examples, we approximated the value of a definite integral using a given method then compared it to the ``right'' answer. This should have raised several questions in the reader's mind, such as:
\begin{enumerate}
	\item	How was the ``right'' answer computed?
	\item	If the right answer can be found, what is the point of approximating?
	\item	If there is value to approximating, how are we supposed to know if the approximation is any good?
\end{enumerate}

These are good questions, and their answers are educational. In the examples, \textit{the} right answer was never computed. Rather, an approximation accurate to a certain number of places after the decimal was given. In \autoref{ex_num1}, we do not know the \textit{exact} answer, but we know it starts with 0.7468. These more accurate approximations were computed using numerical integration but with more precision (i.e., more subintervals and the help of a computer). 

Since the exact answer cannot be found, approximation still has its place. How are we to tell if the approximation is any good?

``Trial and error'' provides one way. %(No, this does not refer to seeing whether or not the bridge collapses.) 
Using technology, make an approximation with, say, 10, 100, and 200 subintervals. This likely will not take much time at all, and a trend should emerge. If a trend does not emerge, try using yet more subintervals. Keep in mind that trial and error is never foolproof; you might stumble upon a problem in which a trend will not emerge.

A second method is to use Error Analysis. While the details are beyond the scope of this text, there are some formulas that give \textit{bounds} for how good your approximation will be. For instance, the formula might state that the approximation is within 0.1 of the correct answer. If the approximation is 1.58, then one knows that the correct answer is between 1.48 and 1.68. By using lots of subintervals, one can get an approximation as accurate as one likes. \autoref{thm:numerical_error} states what these bounds are.

\begin{theorem}[Error Bounds in Numerical Integration]\label{thm:numerical_error}
Suppose that $M_n$ is an upper bound on $\abs{f^{(n)}(x)}$ on $[a,b]$.  Then a bound for the error of the numerical method of integration is given by:
\index{integration!numerical!Left Hand Rule}
\index{integration!numerical!Right Hand Rule}
\index{integration!numerical!Midpoint Rule}
\index{integration!numerical!Trapezoidal Rule}
\index{integration!numerical!Simpson's Rule}
\index{Left/Right Hand Rule!error bounds}
\index{Midpoint Rule!error bounds}
\index{Trapezoidal Rule!error bounds}
\index{Simpson's Rule!error bounds}
\index{numerical integration!Left Hand Rule!error bounds}
\index{numerical integration!Right Hand Rule!error bounds}
\index{numerical integration!Midpoint Rule!error bounds}
\index{numerical integration!Trapezoidal Rule!error bounds}
\index{numerical integration!Simpson's Rule!error bounds}
\begin{center}
\begin{tabular}{lc}
Method & Error Bound \\\midrule
Left/Right Hand Rule & $\dfrac{M_1(b-a)^2}{2n}$ \\
Midpoint Rule & $\dfrac{M_2(b-a)^3}{24n^2}$ \\
Trapezoidal Rule & $\dfrac{M_2(b-a)^3}{12n^2}$ \\
Simpson's Rule & $\dfrac{M_4(b-a)^5}{180n^4}$
\end{tabular}
\end{center}
\end{theorem}

There are some key things to note about this theorem.
\begin{enumerate}
	\item	The larger the interval, the larger the error. This should make sense intuitively.
	\item	The error shrinks as more subintervals are used (i.e., as $n$ gets larger).
	\item	When $n$ doubles, the Left and Right Hand Rules double in accuracy, the Midpoint and Trapezoidal Rules quadruple in accuracy, and Simpson's Rule is 16 times more accurate.
	\item	The error in Simpson's Rule has a term relating to the 4$^{\text{th}}$ derivative of $f$. Consider a cubic polynomial: its $4^{\text{th}}$ derivative is 0. Therefore, the error in approximating the definite integral of a cubic polynomial with Simpson's Rule is 0 --- Simpson's Rule computes the exact answer!
\end{enumerate}

We revisit Examples \ref{ex_num3} and \ref{ex_num5} and compute the error bounds using \autoref{thm:numerical_error} in the following example.

\example{ex_num7}{Computing error bounds}{Find the error bounds when approximating $\ds \int_0^1 e^{-x^2}\ dx$ using the Trapezoidal Rule and 5 subintervals, and using Simpson's Rule with 4 subintervals.}
{\textbf{Trapezoidal Rule with $n=5$:}

We start by computing the $2^\text{nd}$ derivative of $f(x) = e^{-x^2}$:
%
\mtable{Graphing $\fpp(x)$ in \autoref{ex_num7} to help establish error bounds.}{fig:num7a}{\begin{tikzpicture}
\begin{axis}[width=1.16\marginparwidth,tick label style={font=\scriptsize},
axis y line=middle,axis x line=middle,name=myplot,axis on top,
ymin=-2.1,ymax=.7,xmin=-.1,xmax=1.1]
\addplot[draw={\colorone},smooth,thick,domain=0:1] {exp(-x*x)*(4*x*x-2)};
\draw (axis cs:.8,-1.5) node {\scriptsize $y=e^{-x^2}(4x^2-2)$};
\end{axis}
\node [right] at (myplot.right of origin) {\scriptsize $x$};
\node [above] at (myplot.above origin) {\scriptsize $y$};
\end{tikzpicture}}
%
\[\fpp(x) = e^{-x^2}(4x^2-2).\]
\autoref{fig:num7a} shows a graph of $\fpp(x)$ on $[0,1]$. It is clear that the largest value of $\fpp$, in absolute value, is 2. Thus we let $M=2$ and apply the error formula from \autoref{thm:numerical_error}.
\[E_T = \frac{(1-0)^3}{12\cdot 5^2}\cdot 2 = 0.00\overline{6}.\]

Our error estimation formula states that our approximation of 0.7445 found in \autoref{ex_num3} is within 0.0067 of the correct answer, hence we know that
\[0.7445-0.0067 = .7378 \leq \int_0^1e^{-x^2}\ dx \leq 0.7512 = 0.7445 + 0.0067.\]
We had earlier computed the exact answer, correct to 4 decimal places, to be 0.7468, affirming the validity of \autoref{thm:numerical_error}.\bigskip\\
\textbf{Simpson's Rule with $n=4$:}

We start by computing the $4^\text{th}$ derivative of $f(x) = e^{-x^2}$:
%
\mtable{Graphing $f\,^{(4)}(x)$ in \autoref{ex_num7} to help establish error bounds.}{fig:num7b}{\begin{tikzpicture}
\begin{axis}[width=1.16\marginparwidth,tick label style={font=\scriptsize},
axis y line=middle,axis x line=middle,name=myplot,axis on top,minor y tick num=4,
ymin=-8,ymax=13.5,xmin=-.1,xmax=1.1]
\addplot[draw={\colorone},smooth,thick,domain=0:1] {exp(-x*x)*(16*x^4-48*x*x+12)};
\draw (axis cs:.7,11) node {\scriptsize $y=e^{-x^2}(16x^4-48x^2+12)$};
\end{axis}
\node [right] at (myplot.right of origin) {\scriptsize $x$};
\node [above] at (myplot.above origin) {\scriptsize $y$};
\end{tikzpicture}}
%
\[f\,^{(4)}(x) = e^{-x^2}(16x^4-48x^2+12).\]
\autoref{fig:num7b} shows a graph of $f\,^{(4)}(x)$ on $[0,1]$. It is clear that the largest value of $f\,^{(4)}$, in absolute value, is 12. Thus we let $M=12$ and apply the error formula from \autoref{thm:numerical_error}.
\[E_s = \frac{(1-0)^5}{180\cdot 4^4}\cdot 12 = 0.00026.\]

Our error estimation formula states that our approximation of $0.7468\overline{3}$ found in \autoref{ex_num5} is within 0.00026 of the correct answer, hence we know that
\[0.74683-0.00026 = .74657 \leq \int_0^1e^{-x^2}\ dx \leq 0.74709 = 0.74683 + 0.00026.\]
Once again we affirm the validity of \autoref{thm:numerical_error}.}

% todo write an example of ``find the n so that the estimate is within ? of the actual''

At the beginning of this section we mentioned two main situations where numerical integration was desirable. We have considered the case where an antiderivative of the integrand cannot be computed. We now investigate the situation where the integrand is not known. This is, in fact, the most widely used application of Numerical Integration methods. ``Most of the time'' we observe behavior but do not know ``the'' function that describes it. We instead collect data about the behavior and make approximations based off of this data. We demonstrate this in an example.

\example{ex_num8}{Approximating distance traveled}{One of the authors drove his daughter home from school while she recorded their speed every 30 seconds. The data is given in \autoref{fig:num8}.
%
\mtable{Speed data collected at 30 second intervals for \autoref{ex_num8}.}{fig:num8}{\begin{tabular}{rr}\lxBeginTableHead
Time & Speed\\(min)&(mph)
 \\\lxEndTableHead\midrule
0\phantom{.5} & 0 \\
 0.5 & 25 \\ 1\phantom{.5} & 22 \\
 1.5 & 19 \\ 2\phantom{.5} & 39 \\
 2.5 & 0 \\ 3\phantom{.5} & 43\\
 3.5 & 59 \\ 4\phantom{.5} & 54 \\
 4.5 & 51 \\ 5\phantom{.5} & 43 \\
 5.5 & 35 \\ 6\phantom{.5} & 40 \\
 6.5 & 43 \\ 7\phantom{.5} & 30 \\
 7.5 & 0 \\ 8\phantom{.5} & 0 \\
 8.5 & 28 \\ 9\phantom{.5} & 40 \\
 9.5 & 42 \\ 10\phantom{.5} & 40 \\
 10.5 & 39 \\ 11\phantom{.5} & 40 \\
 11.5 & 23 \\ 12\phantom{.5} & 0 \\ 
\end{tabular}}
%
Approximate the distance they traveled.}
{Recall that by integrating a speed function we get distance traveled. We have information about $v(t)$; we will use Simpson's Rule to approximate $\ds \int_a^b v(t)\ dt.$ 

The most difficult aspect of this problem is converting the given data into the form we need it to be in. The speed is measured in miles per hour, whereas the time is measured in 30 second increments.

We need to compute $\Delta x = (b-a)/n$. Clearly, $n=24$. What are $a$ and $b$? Since we start at time $t=0$, we have that $a=0$. The final recorded time came after 24 periods of 30 seconds, which is 12 minutes or 1/5 of an hour. Thus we have
\[\Delta x = \frac{b-a}{n} = \frac{1/5-0}{24} = \frac1{120}; \quad \frac{\Delta x}{3} = \frac{1}{360}.\]

Thus the distance traveled is approximately:
\begin{align*}
\int_0^{0.2}v(t)\ dt &\approx \frac{1}{360}\Big[f(x_1)+4f(x_2) + 2f(x_3) + \dots + 4f(x_n)+f(x_{n+1})\Big]\\
					&= \frac{1}{360}\Big[0+4\cdot25+2\cdot 22 + \dots + 2\cdot40+4\cdot 23 + 0\Big]\\
					&\approx 6.2167 \ \text{miles.}
\end{align*}
We approximate the author drove 6.2 miles. (Because we are sure the reader wants to know, the author's odometer recorded the distance as about 6.05 miles.)}

%We started this chapter learning about antiderivatives and indefinite integrals. We then seemed to change focus by looking at areas between the graph of a function and the $x$-axis. We defined these areas as the definite integral of the function, using a notation very similar to the notation of the indefinite integral. The Fundamental Theorem of Calculus tied these two seemingly separate concepts together: we can find areas under a curve, i.e., we can evaluate a definite integral, using antiderivatives. 
%
%We ended the chapter by noting that antiderivatives are sometimes more than difficult to find: they are impossible. Therefore we developed numerical techniques that gave us good approximations of definite integrals.
%
%We used the definite integral to compute areas, and also to compute displacements and distances traveled. There is far more we can do than that. In \autoref{chapter:app_of_int} we'll see more applications of the definite integral. Before that, in \autoref{chapter:anti_tech} we'll learn advanced techniques of integration, analogous to learning rules like the Product, Quotient and Chain Rules of differentiation.

\printexercises{exercises/05_05_exercises}

%\printexercisesreview{exercises/06_cr_exercises}

\apexchapter{Sequences and Series}{chapter:sequences_series}
\iflatexml\section*{Chapter Introduction}\fi

This chapter introduces \sword{sequences} and \sword{series}, important mathematical constructions that are useful when solving a large variety of mathematical problems. The content of this chapter is considerably different from the content of the chapters before it. While the material we learn here definitely falls under the scope of ``calculus,'' we will make very little use of derivatives or integrals. Limits are extremely important, though, especially limits that involve infinity. 

One of the problems addressed by this chapter is this: suppose we know information about a function and its derivatives at a point, such as  $f(1) = 3$, $\fp(1) = 1$, $\fp'(1) = -2$, $\fp''(1) = 7$, and so on. What can I say about $f(x)$ itself? Is there any reasonable approximation of the value of $f(2)$? The topic of Taylor Series addresses this problem, and allows us to make excellent approximations of functions when limited knowledge of the function is available.

\section{Sequences}\label{sec:sequences}

We commonly refer to a set of events that occur one after the other as a \textit{sequence} of events. In mathematics, we use the word \textit{sequence} to refer to an ordered set of numbers, i.e., a set of numbers that ``occur one after the other.''

For instance, the numbers 2, 4, 6, 8, \ldots, form a sequence. The order is important; the first number is 2, the second is 4, etc. It seems natural to seek a formula that describes a given sequence, and often this can be done. For instance, the sequence above could be described by the function $a(n) = 2n$, for the values of $n = 1, 2, \dotsc$ (it could also be described by $n^4-10 n^3+35 n^2-48n+24$, to give one of infinitely many other options). To find the 10$^\text{th}$ term in the sequence, we would compute $a(10)$. This leads us to the following, formal definition of a sequence.

\mnote{\textbf{Notation:} We use $\mathbb{N}$ to describe the set of natural numbers, that is, the integers 1, 2, 3, \ldots}

\definition{def:sequence}{Sequence}
{A \textbf{sequence} is a function $a(n)$ whose domain is $\mathbb{N}$. The \textbf{range} of a sequence is the set of all distinct values of $a(n)$.
\index{sequences!definition}\\

The \textbf{terms} of a sequence are the values $a(1)$, $a(2)$, \ldots, which are usually denoted with subscripts as $a_1$, $a_2$, \ldots.\\

A sequence $a(n)$ is often denoted as $\{a_n\}$.}

\youtubeVideo{9K1xx6wfN-U}{Sequences --- Examples showing convergence or divergence}

\mnote[-.3in]{\textbf{Factorial:} The expression $3!$ refers to the number $3\cdot2\cdot1 = 6$.
\index{factorial}\index{aa@$"!$}\\
In general, $n! = n\cdot (n-1)\cdot(n-2)\dotsm 2\cdot1$, where $n$ is a natural number.\\
We define $0! = 1$. While this does not immediately make sense, it makes many mathematical formulas work properly.}

\example{ex_seq1}{Listing terms of a sequence}{List the first four terms of the following sequences.
\mtable{Plotting sequences in \autoref{ex_seq1}.}{fig:seq1b}{%
\myincludegraphics{figures/figseq1a} \\  (a) \\
\myincludegraphics{figures/figseq1b} \\ (b) \\
\myincludegraphics{figures/figseq1c} \\ (c)}
\[
 \text{1. }\{a_n\} = \left\{\frac{3^n}{n!}\right\}\qquad
 \text{2. }\{a_n\} = \{4+(-1)^n\}\qquad
 \text{3. }\{a_n\} = \left\{\frac{(-1)^{n(n+1)/2}}{n^2}\right\}
\]}
{\begin{enumerate}
\item		$\ds a_1=\frac{3^1}{1!} = 3$;\qquad	$\ds a_2= \frac{3^2}{2!} = \frac92$;\qquad $\ds a_3 = \frac{3^3}{3!} = \frac92$; \qquad $\ds a_4 = \frac{3^4}{4!} = \frac{27}8$

We can plot the terms of a sequence with a scatter plot. The ``$x$''-axis is used for the values of $n$, and the values of the terms are plotted on the $y$-axis. To visualize this sequence, see \autoref{fig:seq1b}(a).

\item		$a_1= 4+(-1)^1 = 3$;\qquad $a_2 = 4+(-1)^2 = 5$; 

\noindent $a_3=4+(-1)^3 = 3$; \qquad $a_4 = 4+(-1)^4 = 5$.\\
Note that the range of this sequence is finite, consisting of only the values 3 and 5. This sequence is plotted in \autoref{fig:seq1b}(b).

\item		$\ds a_1= \frac{(-1)^{1(2)/2}}{1^2} = -1$; \qquad $\ds a_2 = \frac{(-1)^{2(3)/2}}{2^2} =-\frac14$

\noindent $\ds a_3 = \frac{(-1)^{3(4)/2}}{3^2} = \frac19$ \qquad $\ds a_4 = \frac{(-1)^{4(5)/2}}{4^2} = \frac1{16}$; 

\noindent $\ds a_5 = \frac{(-1)^{5(6)/2}}{5^2}=-\frac1{25}$.

\noindent We gave one extra term to begin to show the pattern of signs is ``$-$, $-$, $+$, $+$, $-$, $-$, $\ldots$'', due to the fact that the exponent of $-1$ is a special quadratic. This sequence is plotted in \autoref{fig:seq1b}(c).\eoehere
\end{enumerate}}

\example{ex_seq2}{Determining a formula for a sequence}{Find the $n^\text{th}$ term of the following sequences, i.e., find a function that describes each of the given sequences.

\begin{enumerate}
\item		2, 5, 8, 11, 14, $\ldots$
\item		2, $-5$, 10, $-17$, 26, $-37$, $\ldots$
\item		1, 1, 2, 6, 24, 120, 720, $\ldots$
\item		$\ds \frac52$, $\ds \frac52$, $\ds \frac{15}8$, $\ds \frac54$, $\ds \frac{25}{32}$, $\ldots$
\end{enumerate}
}
{We should first note that there is never exactly one function that describes a finite set of numbers as a sequence. There are many sequences that start with 2, then 5, as our first example does. We are looking for a simple formula that describes the terms given, knowing there is possibly more than one answer.
\begin{enumerate}
\item		Note how each term is 3 more than the previous one. This implies a linear function would be appropriate: $a(n) = a_n = 3n + b$ for some appropriate value of $b$. As we want $a_1=2$, we set $b=-1$. Thus $a_n = 3n-1$.

\item		First notice how the sign changes from term to term. This is most commonly accomplished by multiplying the terms by either $(-1)^n$ or $(-1)^{n+1}$. Using $(-1)^n$ multiplies the odd terms by $(-1)$; using $(-1)^{n+1}$ multiplies the even terms by $(-1)$. As this sequence has negative even terms, we will multiply by $(-1)^{n+1}$. 

After this, we might feel a bit stuck as to how to proceed. At this point, we are just looking for a pattern of some sort: what do the numbers 2, 5, 10, 17, etc., have in common? There are many correct answers, but the one that we'll use here is that each is one more than a perfect square. That is, $2=1^2+1$, $5=2^2+1$, $10=3^2+1$, etc. Thus our formula is $a_n= (-1)^{n+1}(n^2+1)$.

\item		One who is familiar with the factorial function will readily recognize these numbers. They are $0!$, $1!$, $2!$, $3!$, etc. Since our sequences start with $n=1$, we cannot write $a_n = n!$, for this misses the $0!$ term. Instead, we shift by 1, and write $a_n = (n-1)!$.

\item		This one may appear difficult, especially as the first two terms are the same, but a little ``sleuthing'' will help. Notice how the terms in the numerator are always multiples of 5, and the terms in the denominator are always powers of 2. Does something as simple as $a_n = \frac{5n}{2^n}$ work?

When $n=1$, we see that we indeed get $5/2$ as desired. When $n=2$, we get $10/4 = 5/2$. Further checking shows that this formula indeed matches the other terms of the sequence.\eoehere
\end{enumerate}}

A common mathematical endeavor is to create a new mathematical object (for instance, a sequence) and then apply previously known mathematics to the new object. We do so here. The fundamental concept of calculus is the limit, so we will investigate what it means to find the limit of a sequence.

%\setboxwidth{80pt}
\definition{def:seq_limit}{Limit of a Sequence, Convergent, Divergent}
{Let $\{a_n\}$ be a sequence and let $L$ be a real number. Given any $\epsilon>0$, if an $m$ can be found such that $|a_n-L|<\epsilon$ for all $n>m$, then we say the \textbf{limit of $\{a_n\}$, as $n$ approaches infinity, is $L$}, denoted $$\lim_{n\to\infty}a_n = L.$$

If $\ds\lim_{n\to\infty} a_n$ exists, we say the sequence \sword{converges}; otherwise, the sequence \sword{diverges}.\index{limit!of sequence}\index{sequences!limit}\index{convergence!of sequence}\index{divergence!of sequence}\index{sequences!convergent}\index{sequences!divergent}
}

This definition states, informally, that if the limit of a sequence is $L$, then if you go far enough out along the sequence, all subsequent terms will be \emph{really close} to $L$. Of course, the terms ``far enough'' and ``really close'' are subjective terms, but hopefully the intent is clear.

This definition is reminiscent of the $\epsilon$--$\delta$ proofs of \autoref{chapter:limits}. In that chapter we developed other tools to evaluate limits apart from the formal definition; we do so here as well.

%\setboxwidth{30pt}
\theorem{thm:seq_limit}{Limit of a Sequence}
{Let $\{a_n\}$ be a sequence and let $f(x)$ be a function whose domain contains the positive real numbers where $f(n) = a_n$ for all $n$ in $\mathbb{N}$. \\

If $\ds \lim_{x\to\infty} f(x) = L$, then $\ds\lim_{n\to\infty} a_n = L$.
%\begin{enumerate}
%\item		If $\ds \lim_{x\to\infty} f(x) = L$, then $\ds\lim_{n\to\infty} a_n = L$.
%\item		If $\ds \lim_{x\to\infty} f(x)$ does not exist, then $\{a_n\}$ diverges.
%\end{enumerate}
}

\autoref{thm:seq_limit} allows us, in certain cases, to apply the tools developed in \autoref{chapter:limits} to limits of sequences. Note two things \textit{not} stated by the theorem:
	\begin{enumerate}
		\item If $\ds \lim_{x\to\infty}f(x)$ does not exist, we cannot conclude that $\ds\lim_{n\to\infty} a_n$ does not exist. It may, or may not, exist. For instance, we can define a sequence $\{a_n\} = \{\cos(2\pi n)\}$. Let $f(x) = \cos (2\pi x)$. Since the cosine function oscillates over the real numbers, the limit $\ds \lim_{x\to\infty}f(x)$ does not exist. 
		
		However, for every positive integer $n$, $\cos(2\pi n) = 1$, so $\ds \lim_{n\to\infty} a_n = 1$.
		
		%For every positive integer $n$, $\cos(2\pi n) = 1$, so $\ds \lim_{n\to\infty} a_n = 1$. 
		%
		%It is natural to set $f(x) = \cos (2\pi x)$. Since the cosine function oscillates over the real numbers, the limit $\ds \lim_{x\to\infty}f(x)$ does not exist.
		\item	If we cannot find a function $f(x)$ whose domain contains the positive real numbers where $f(n) = a_n$ for all $n$ in $\mathbb{N}$, we cannot conclude $\ds\lim_{n\to\infty} a_n$ does not exist. It may, or may not, exist.
	\end{enumerate}

%When we considered limits before, the domain of the function was an interval of real numbers. Now, as we consider limits, the domain is restricted to $\mathbb{N}$, the natural numbers. \autoref{thm:seq_limit} states that if we can extend a function whose domain is $\mathbb{N}$ to a

%\autoref{thm:seq_limit} states that this restriction of the domain does not affect the outcome of the limit and whatever tools we developed in \autoref{chapter:limits} to evaluate limits can be applied here as well.\\

%Considering again \autoref{ex_seq3}, we can now state when $\{a_n\} = \{\frac1n\}$, \mbox{$\ds \lim_{n\to\infty} a_n = 0$}.\\

\example{ex_seq4}{Determining convergence/divergence of a sequence}{Determine the convergence or divergence of the following sequences.
%
\mtable{Scatter plots of the sequences in \autoref{ex_seq4}.}{fig:seq4}{%
 \begin{tikzpicture}
  \begin{axis}[width=1.16\marginparwidth,tick label style={font=\scriptsize},
    axis y line=middle,axis x line=middle,name=myplot,axis on top,%
    xtick={20,40,60,80,100},ymin=-11,ymax=11,xmin=-.1,xmax=110]
   \addplot [only marks,{\colorone},mark size={.75pt},domain=1:100,samples=21]
    {(3*x^2 - 2*x + 1)/(x^2 - 1000)};
   \draw (axis cs:60,-8) node {\scriptsize $a_n=\dfrac{3n^2-2n+1}{n^2-1000}$};
  \end{axis}
  \node [right] at (myplot.right of origin) {\scriptsize $n$};
  \node [above] at (myplot.above origin) {\scriptsize $y$};
 \end{tikzpicture}
\\(a)\\
\myincludegraphics{figures/figseq4b}
\\(b)\\
\myincludegraphics{figures/figseq4c}
\\(c)}
%
\[
 \text{1. }\{a_n\} = \left\{\frac{3n^2-2n+1}{n^2-1000}\right\}\qquad
 \text{2. }\{a_n\} = \{\cos n \}\qquad
 \text{3. }\{a_n\} = \left\{\frac{(-1)^n}{n}\right\}
\]}
{\begin{enumerate}
\item		Using \autoref{thm:lim_rational_fn_at_infty}, we can state that $\ds\lim_{x\to\infty} \frac{3x^2-2x+1}{x^2-1000} = 3$. (We could have also directly applied L'H\^opital's Rule.) Thus the sequence $\{a_n\}$ converges, and its limit is 3. A scatter plot of every 5 values of $a_n$ is given in \autoref{fig:seq4} (a). The values of $a_n$ vary widely near $n=30$, ranging from about $-73$ to $125$, but as $n$ grows, the values approach 3.

\item		The limit $\ds\lim_{x\to\infty}\cos x$ does not exist, as $\cos x$ oscillates (and takes on every value in $[-1,1]$ infinitely many times). Thus we cannot apply \autoref{thm:seq_limit}. 

The fact that the cosine function oscillates strongly hints that $\cos n$, when $n$ is restricted to $\mathbb{N}$, will also oscillate. \autoref{fig:seq4} (b), where the sequence is plotted, shows that this is true. Because only discrete values of cosine are plotted, it does not bear strong resemblance to the familiar cosine wave.

Based on the graph, we suspect that $\ds \lim_{n\to\infty} a_n$ does not exist, but we have not decisively proven it yet.

%conclude that the sequence $\{\cos n\}$ diverges. (And in this particular case, since the domain is restricted to $\mathbb{N}$, no value of $\cos n$ is repeated!) This sequence is plotted in \autoref{fig:seq4} (b); because only discrete values of cosine are plotted, it does not bear strong resemblance to the familiar cosine wave.

\item		We cannot actually apply \autoref{thm:seq_limit} here, as the function $f(x) = (-1)^x/x$ is not well defined. (What does $(-1)^{\sqrt{2}}$ mean? In actuality, there is an answer, but it involves \emph{complex analysis}, beyond the scope of this text.) So for now we say that we cannot determine the limit. (But we will be able to very soon.) By looking at the plot in \autoref{fig:seq4} (c), we would like to conclude that the sequence converges to 0. That is true, but at this point we are unable to decisively say so.\eoehere
\end{enumerate}}

It seems that  %$\ds \left\{\frac{(-1)^n}{n}\right\}$ 
$\{(-1)^n/n\}$ converges to 0 but we lack the formal tool to prove it. The following theorem gives us that tool.

\theorem{thm:abs_val_seq}{Absolute Value Theorem}
{Let $\{a_n\}$ be a sequence. If $\ds \lim_{n\to\infty} |a_n| = 0$, then $\ds \lim_{n\to\infty} a_n = 0$\index{Absolute Value Theorem}\index{limit!Absolute Value Theorem}\index{sequence!Absolute Value Theorem}}

% this proof is exercise 45, but copying the proof is not the worst learning outcome
\begin{proof}
We know $-\abs{a_n}\leq a_n\leq\abs{a_n}$ and $\ds \lim_{n\to \infty} (-\abs{a_n})=-\lim_{n\to\infty}\abs{a_n}=0$. Thus by the Squeeze Theorem $\ds \lim_{n\to\infty} a_n =0$.
\end{proof}

\example{ex_seq5}{Determining the convergence/divergence of a sequence}{Determine the convergence or divergence of the following sequences.
\[
 \text{1. }\{a_n\} = \left\{\frac{(-1)^n}{n}\right\}\qquad
 \text{2. }\{a_n\} = \left\{\frac{(-1)^n(n+1)}{n}\right\}
\]}
{\begin{enumerate}
\item		This appeared in \autoref{ex_seq4}. We want to apply \autoref{thm:abs_val_seq}, so consider the limit of $\{|a_n|\}$:
\begin{align*}
\lim_{n\to\infty} |a_n| &= \lim_{n\to\infty} \left|\frac{(-1)^n}{n}\right| \\
					&= \lim_{n\to\infty} \frac{1}{n} \\
					&= 0.
\end{align*}
Since this limit is 0, we can apply \autoref{thm:abs_val_seq} and state that $\ds\lim_{n\to\infty} a_n=0$.

\item Because of the alternating nature of this sequence (i.e., every other term is multiplied by $-1$), we cannot simply look at the limit $\ds \lim_{x\to\infty} \frac{(-1)^x(x+1)}{x}$. We can try to apply the techniques of \autoref{thm:abs_val_seq}:
\begin{align*}
	\lim_{n\to\infty}\abs{a_n}
	&= \lim_{n\to\infty} \abs{\frac{(-1)^n(n+1)}{n}} \\
	&= \lim_{n\to\infty} \frac{n+1}{n}\\
	&= 1.
\end{align*}
\mfigure{0in}{A plot of a sequence in \autoref{ex_seq5}, part 2.}{fig:seq5}{figures/figseq5}
We have concluded that when we ignore the alternating sign, the se\-quence approaches 1. This means we cannot apply \autoref{thm:abs_val_seq}; it states the the limit must be 0 in order to conclude anything.

Since we know that the signs of the terms alternate \emph{and} we know that the limit of $\abs{a_n}$ is 1, we know that as $n$ approaches infinity, the terms will alternate between values close to 1 and $-1$, meaning the sequence diverges. A plot of this sequence is given in \autoref{fig:seq5}.\eoehere
\end{enumerate}}

We continue our study of the limits of sequences by considering some of the properties of these limits.

\theorem{thm:seq_properties}{Properties of the Limits of Sequences}
{Let $\{a_n\}$ and $\{b_n\}$ be sequences such that $\ds \lim_{n\to\infty} a_n = L$, $\ds \lim_{n\to\infty} b_n = K$, and let $c$ be a real number.\\
\begin{minipage}[t]{.5\linewidth}
\begin{enumerate}
\item		$\ds \lim_{n\to\infty} (a_n\pm b_n) = L\pm K$
\index{sequences!limit properties}
\item		$\ds \lim_{n\to\infty} (a_n\cdot b_n) = L\cdot K$
\end{enumerate}
\end{minipage}%
\begin{minipage}[t]{.5\linewidth}
\begin{enumerate}\addtocounter{enumi}{2}
\item		$\ds \lim_{n\to\infty} (a_n/b_n) = L/K$, $K\neq 0$
\item		$\ds \lim_{n\to\infty} c\cdot a_n = c\cdot L$
\end{enumerate}
\end{minipage}}

\example{ex_seq6}{Applying properties of limits of sequences}{Let the following limits be given:
\begin{itemize}
\item	 	$\ds \lim_{n\to\infty} a_n = 0$;
\item		$\ds \lim_{n\to\infty} b_n = e$; and
\item	  $\ds \lim_{n\to\infty} c_n = 5$.
\end{itemize}
Evaluate the following limits.
\[
 \text{1.}\quad\lim_{n\to\infty} (a_n+b_n)\qquad
 \text{2.}\quad\lim_{n\to\infty} (b_n\cdot c_n)\qquad
 \text{3.}\quad\lim_{n\to\infty} (1000\cdot a_n)
\]}
{We will use \autoref{thm:seq_properties} to answer each of these.
\begin{enumerate} 
\item		Since $\ds \lim_{n\to\infty} a_n = 0$ and $\ds \lim_{n\to\infty} b_n = e$, we conclude that $\ds \lim_{n\to\infty} (a_n+b_n) = 0+e = e.$ So even though we are adding something to each term of the sequence $b_n$, we are adding something so small that the final limit is the same as before.

\item		Since $\ds \lim_{n\to\infty} b_n = e$ and $\ds \lim_{n\to\infty} c_n = 5$, we conclude that $\ds \lim_{n\to\infty} (b_n\cdot c_n) = e\cdot 5 = 5e.$

\item		Since $\ds \lim_{n\to\infty} a_n = 0$, we have $\ds \lim_{n\to\infty} 1000a_n =1000\cdot 0 = 0$. It does not matter that we multiply each term by 1000; the sequence still approaches 0. (It just takes longer to get close to 0.)\eoehere
\end{enumerate}}

% todo write an introductory for the theorem ``Convergence of Geometric Sequences''

\theorem{thm:geom_seq}{Convergence of Geometric Sequences}{The sequence $\{r^n\}$ is convergent if $-1<r\leq 1$ and divergent for all other values of $r$.  Furthermore,
\[
\lim_{n\to \infty} r^n=\begin{cases} 
0&  -1<r<1\\
1& r=1
\end{cases}\]}

\begin{proof}
We can see from \autoref{ki_exp_func_props} and by letting $a=r$ that
\[
\lim_{n\to \infty} r^n =
\begin{cases}
\infty &  r>1\\
0 & 0<r<1.
\end{cases}
\]
We also know that $\ds  \lim_{x\to \infty} 1^n=1$ and $\ds  \lim_{x\to \infty} 0^n=0$. If $-1<r<0$, we know $0<\abs r<1$ so $\ds \lim_{x\to \infty}\abs{r^n}=\lim_{x\to \infty}\abs r^n=0$ and thus by \autoref{thm:abs_val_seq},$\ds \lim_{x\to \infty} r^n=0$. If $r\le-1$, $\ds \lim_{x\to \infty} r^n$ does not exist. Therefore, the sequence $\{ r^n\}$ is convergent if $-1<r\leq 1$ and divergent for all other values of $r$.
\end{proof}

There is more to learn about sequences than just their limits. We will also study their range and the relationships terms have with the terms that follow. We start with some definitions describing properties of the range.

\definition{def:bounded}{Bounded and Unbounded Sequences}
{A sequence $\{a_n\}$ is said to be \textbf{bounded} if there exists real numbers $m$ and $M$ such that $m < a_n < M$ for all $n$ in $\mathbb{N}$.\\

A sequence $\{a_n\}$ is said to be \textbf{unbounded} if it is not bounded.\\

A sequence $\{a_n\}$ is said to be \textbf{bounded above} if there exists an $M$ such that $a_n < M$ for all $n$ in $\mathbb{N}$; it is \textbf{bounded below} if there exists an $m$ such that $m<a_n$ for all $n$ in $\mathbb{N}$.
\index{sequences!boundedness}\index{bounded sequence}\index{unbounded sequence}}

It follows from this definition that an unbounded sequence may be bounded above or bounded below; a sequence that is both bounded above and below is simply a bounded sequence.

\example{ex_seq3}{Determining boundedness of sequences}{Determine the boundedness of the following sequences.
\[
 \text{1.}\quad\{a_n\} = \left\{\frac1n\right\}\qquad\qquad
 \text{2.}\quad\{a_n\} = \{2^n\}
\]
%
\mtable{A plot of $\{a_n\} = \{1/n\}$ and $\{a_n\} = \{2^n\}$ from \autoref{ex_seq3}.}{fig:seq3}{\myincludegraphics{figures/figseq3a}\\
(a)\\[10pt]
\myincludegraphics{figures/figseq3b}\\
(b)}}
{\begin{enumerate}
\item	The terms of this sequence are always positive but are decreasing, so we have $0<a_n<2$ for all $n$. Thus this sequence is bounded. \autoref{fig:seq3}(a) illustrates this.

\item	The terms of this sequence obviously grow without bound. However, it is also true that these terms are all positive, meaning $0<a_n$. Thus we can say the sequence is unbounded, but also bounded below. \autoref{fig:seq3}(b) illustrates this.\eoehere
\end{enumerate}}

The previous example produces some interesting concepts. First, we can recognize that the sequence $\ds\left\{1/n\right\}$ converges to 0. This says, informally, that ``most'' of the terms of the sequence are ``really close'' to 0. This implies that the sequence is bounded, using the following logic. First, ``most'' terms are near 0, so we could find some sort of bound on these terms (using \autoref{def:seq_limit}, the bound is $\epsilon$). That leaves a ``few'' terms that are not near 0 (i.e., a \emph{finite} number of terms). A finite list of numbers is always bounded. 

This logic suggests that if a sequence converges, it must be bounded. This is indeed true, as stated by the following theorem.

\theorem{thm:converge_bounded}{Convergent Sequences are Bounded}
{Let $\ds \left\{a_n\right\}$ be a convergent sequence. Then $\{a_n\}$ is bounded.
\index{bounded sequence!convergence}\index{convergence!of sequence}\index{sequences!convergent}
}

\mnote{\textbf{Note:} Keep in mind what \autoref{thm:converge_bounded} does \emph{not} say. It does not say that bounded sequences must converge, nor does it say that if a sequence does not converge, it is not bounded.}

In \autoref{sec:lhopitals_rule} \autoref{ex_LHR4} part 1, we found that $\ds \lim_{x\to \infty} (1+1/x)^x=e$. If we consider the sequence $\ds \{b_n\}=\{(1+1/n)^n\}$, we see that $\ds \lim_{n\to \infty}=e$. Even though it may be difficult to intuitively grasp the behavior of this sequence, we know immediately that it is bounded.

Another interesting concept to come out of \autoref{ex_seq3} again involves the sequence $\{1/n\}$. We stated, without proof, that the terms of the sequence were decreasing. That is, that $a_{n+1} < a_n$ for all $n$. (This is easy to show. Clearly $n < n+1$. Taking reciprocals flips the inequality: $1/n > 1/(n+1)$. This is the same as $a_n > a_{n+1}$.) Sequences that either steadily increase or decrease are important, so we give this property a name.

\definition{def:monotonic}{Monotonic Sequences}
{\mbox{}\\[-2\baselineskip]\begin{enumerate}
\item		A sequence $\{a_n\}$ is \textbf{monotonically increasing} if $a_n \leq a_{n+1}$ for all $n$, i.e.,
 $$a_1 \leq a_2 \leq a_3 \leq \cdots a_n \leq a_{n+1} \cdots$$
 \item	A sequence $\{a_n\}$ is \textbf{monotonically decreasing} if $a_n \geq a_{n+1}$ for all $n$, i.e.,
 $$a_1 \geq a_2 \geq a_3 \geq \cdots a_n \geq a_{n+1} \cdots$$
 \item	A sequence is \textbf{monotonic} if it is monotonically increasing or monotonically decreasing.
\index{sequences!monotonic}\index{monotonic sequence}
 \end{enumerate}}

\mnote{\textbf{Note:} It is sometimes useful to call a monotonically increasing sequence \emph{strictly increasing} if $a_n < a_{n+1}$ for all $n$; i.e, we remove the possibility that subsequent terms are equal.

A similar statement holds for \emph{strictly decreasing.}}

\example{ex_seq7}{Determining monotonicity}{Determine the monotonicity of the following sequences.\\
\begin{minipage}[t]{.5\linewidth}
\begin{enumerate}
\item $\ds \{a_n\} = \left\{\frac{n+1}n\right\}$
\item	$\ds \{a_n\} = \left\{\frac{n^2+1}{n+1}\right\}$	
\end{enumerate}
\end{minipage}%
\begin{minipage}[t]{.5\linewidth}
\begin{enumerate}\addtocounter{enumi}{2}
\item $\ds \{a_n\} = \left\{\frac{n^2-9}{n^2-10n+26}\right\}$
\item	$\ds \{a_n\} = \left\{\frac{n^2}{n!}\right\}$	
\end{enumerate}
\end{minipage}}
{In each of the following, we will examine $a_{n+1}-a_n$. If $a_{n+1}-a_n \ge0$, we conclude that $a_n\le a_{n+1}$ and hence the sequence is increasing. If $a_{n+1}-a_n\le0$, we conclude that $a_n\ge a_{n+1}$ and the sequence is decreasing. Of course, a sequence need not be monotonic and perhaps neither of the above will apply.

%\mtable{Plots of sequences in \autoref{ex_seq7}.}{fig:seq7}{%
%\begin{tabular}{c}
%\myincludegraphics{figures/figseq7a}\\
%(a)\rule[-25pt]{0pt}{10pt}\\ 
%\myincludegraphics{figures/figseq7b}\\
%(b)\rule[-25pt]{0pt}{10pt}\\ 
%\myincludegraphics{figures/figseq7c}\\
%(c)\\
%\end{tabular}}

We also give a scatter plot of each sequence. These are useful as they suggest a pattern of monotonicity, but analytic work should be done to confirm a graphical trend.

\begin{enumerate}
\item
\mfigure{0in}{A plot of $\{a_n\}=\{\frac{n+1}n\}$ in \autoref{ex_seq7}(a).}{fig:seq7a}{figures/figseq7a}
	\hfill	$\ds\begin{aligned}[t] a_{n+1}-a_n &= \frac{n+2}{n+1} - \frac{n+1}{n} \\		
	&= \frac{(n+2)(n)-(n+1)^2}{(n+1)n} \\
	&=	\frac{-1}{n(n+1)} \\
	&<0 \quad\text{ for all $n$.}
\end{aligned}$ \hfill\null
				
Since $a_{n+1}-a_n<0$ for all $n$, we conclude that the sequence is decreasing.

\item
	\hfill $\ds \begin{aligned}[t]	
	a_{n+1}-a_n &= \frac{(n+1)^2+1}{n+2} - \frac{n^2+1}{n+1} \\		
	&= \frac{\big((n+1)^2+1\big)(n+1)- (n^2+1)(n+2)}{(n+1)(n+2)}\\
	&=	\frac{n^2+4n+1}{(n+1)(n+2)} \\
	&> 0 \quad \text{ for all $n$.}
\end{aligned}$\hfill \null
\mfigure{-.5in}{A plot of $\{a_n\}=\{\frac{n^2+1}{n+1}\}$ in \autoref{ex_seq7}(b).}{fig:seq7b}{figures/figseq7b}
					
Since $a_{n+1}-a_n>0$ for all $n$, we conclude the sequence is increasing.

\item
\mfigure{0in}{A plot of $\{a_n\}=\{\frac{n^2-9}{n^2-10n+26}\}$ in \autoref{ex_seq7}(c).}{fig:seq7c}{figures/figseq7c}
	We can clearly see in \autoref{fig:seq7c}, where the sequence is plotted, that it is not monotonic. However, it does seem that after the first 4 terms it is decreasing. To understand why, perform the same analysis as done before:

\hfill $\ds \begin{aligned}[t]	
	a_{n+1}-a_n &= \frac{(n+1)^2-9}{(n+1)^2-10(n+1)+26} - \frac{n^2-9}{n^2-10n+26} \\	
	&= \frac{n^2+2n-8}{n^2-8n+17}-\frac{n^2-9}{n^2-10n+26}\\
	&= \frac{(n^2+2n-8)(n^2-10n+26)-(n^2-9)(n^2-8n+17)}{(n^2-8n+17)(n^2-10n+26)}\\
	&= \frac{-10n^2+60n-55}{(n^2-8n+17)(n^2-10n+26)}.
\end{aligned}$\hfill \null		

We want to know when this is greater than, or less than, 0. The denominator is always positive, therefore we are only concerned with the numerator. Using the quadratic formula, we can determine that $-10n^2+60n-55=0$ when $n\approx 1.13, 4.87$. So for $n<1.13$, the sequence is decreasing. Since we are only dealing with the natural numbers, this means that $a_1 > a_2$.

Between $1.13$ and $4.87$, i.e., for $n=2$, 3 and 4, we have that $a_{n+1}>a_n$ and the sequence is increasing. (That is, when $n=2$, 3 and 4, the numerator $-10n^2+60n+55$ from the fraction above is $>0$.)

When $n> 4.87$, i.e, for $n\geq 5$, we have that $-10n^2+60n+55<0$, hence $a_{n+1}-a_n<0$, so the sequence is decreasing.

In short, the sequence is simply not monotonic. However, it is useful to note that for $n\geq 5$, the sequence is monotonically decreasing. 

\item
\mfigure{-1in}{A plot of $\{a_n\}=\{n^2/n!\}$ in \autoref{ex_seq7}(d).}{fig:seq7d}{figures/figseq7d}
	Again, the plot in \autoref{fig:seq7d} shows that the sequence is not monotonic, but it suggests that it is monotonically decreasing after the first term. We perform the usual analysis to confirm this.

\hfill $\ds \begin{aligned}[t]	
	a_{n+1}-a_n &= \frac{(n+1)^2}{(n+1)!} - \frac{n^2}{n!} \\
	&= \frac{(n+1)^2-n^2(n+1)}{(n+1)!} \\
	&=	\frac{-n^3+2n+1}{(n+1)!}
\end{aligned}$\hfill \null

When $n=1$, the above expression is $>0$; for $n\geq 2$, the above expression is $<0$. Thus this sequence is not monotonic, but it is monotonically decreasing after the first term.\eoehere
\end{enumerate}}

Knowing that a sequence is monotonic can be useful. In particular, if we know that a sequence is bounded and monotonic, we can conclude it converges. Consider, for example, a sequence that is monotonically decreasing and is bound\-ed below. We know the sequence is always getting smaller, but that there is a bound to how small it can become. This is enough to prove that the sequence will converge, as stated in the following theorem.

\theorem{thm:monotonic_converge}{Bounded Monotonic Sequences are Convergent}
{Let $\{a_n\}$ be a bounded, monotonic sequence. Then $\{a_n\}$ converges; i.e., $\ds \lim_{n \to\infty}a_n$ exists.
\index{sequences!convergent}\index{convergence!of monotonic sequences}}

Consider once again the sequence $\{a_n\} = \{1/n\}$. It is easy to show it is monotonically decreasing and that it is always positive (i.e., bounded below by 0). Therefore we can conclude by \autoref{thm:monotonic_converge} that the sequence converges. We already knew this by other means, but in the following section this theorem will become very useful.

Convergence of a sequence does not depend on the first $N$ terms of a sequence.  For example, we could adapt the sequence of the previous paragraph to be
\[
 1,\ 10,\ 100,\ 1000,\ \frac15,\ \frac16,\ \frac17,\ \frac18,\ \frac19,\ \frac1{10},
 \dotsc
\]
Because we only changed three of the first 4 terms, we have not affected whe\-ther the sequence converges or diverges.

Sequences are a great source of mathematical inquiry. The On-Line Encyclopedia of Integer Sequences (\url{http://oeis.org}) contains thousands of sequences and their formulae. (As of this writing, there are 257,537 sequences in the database.) Perusing this database quickly demonstrates that a single sequence can represent several different ``real life'' phenomena. 

Interesting as this is, our interest actually lies elsewhere. We are more interested in the \emph{sum} of a sequence. That is, given a sequence $\{a_n\}$, we are very interested in $a_1+a_2+a_3+\dotsb$. Of course, one might immediately counter with ``Doesn't this just add up to `infinity'?'' Many times, yes, but there are many important cases where the answer is no. This is the topic of \emph{series}, which we begin to investigate in the next section.

\printexercises{exercises/08_01_exercises}

\section{Infinite Series}\label{sec:series}

Given the sequence $\{a_n\} = \{1/2^n\} = 1/2,\ 1/4,\ 1/8,\ \ldots$, consider the following sums:

$$\begin{array}{ccccc}
a_1				&=& 1/2					&=& 1/2\\
a_1+a_2			&=& 1/2+1/4				&=& 3/4\\
a_1+a_2+a_3		&=& 1/2+1/4+1/8			&=& 7/8\\
a_1+a_2+a_3+a_4	&=& 1/2+1/4+1/8+1/16		&=& 15/16
\end{array}$$
Later, we will be able to show that
$$a_1+a_2+a_3+\dotsb+a_n = \frac{2^n-1}{2^n} = 1-\frac{1}{2^n}.$$
Let $S_n$ be the sum of the first $n$ terms of the sequence $\{1/2^n\}$. From the above, we see that $S_1=1/2$, $S_2 = 3/4$, and that $S_n = 1-1/2^n$. 

Now consider the following limit: $\ds \lim_{n\to\infty}S_n = \lim_{n\to\infty}\big(1-1/2^n\big) = 1$. This limit can be interpreted as saying something amazing: \emph{the sum of \emph{all} the terms of the sequence $\{1/2^n\}$ is 1.} 


This example illustrates some interesting concepts that we explore in this section. We begin this exploration with some definitions.

% make sure the next stays on the opening page

%\setboxwidth{50pt}
\definition{def:series}{Infinite Series, $n^\text{th}$ Partial Sums, Convergence, Divergence}
{Let $\{a_n\}$ be a sequence.
\index{series!definition}\index{series!partial sums}\index{series!convergent}\index{series!divergent}\index{convergence!of series}\index{divergence!of series}
\begin{enumerate}
\item		The sum $\ds \sum_{n=1}^\infty a_n$ is an \textbf{infinite series} (or, simply \textbf{series}).
\item		Let $\ds S_n = \sum_{i=1}^n a_i$\,; the sequence $\{S_n\}$ is the sequence of \textbf{$n^\text{th}$ partial sums} of $\{a_n\}$.
\item		If the sequence $\{S_n\}$ converges to $L$, we say the series $\ds \sum_{n=1}^\infty a_n$ \textbf{converges} to $L$, and we write $\ds \sum_{n=1}^\infty a_n = L$.
\item		If the sequence $\{S_n\}$ diverges, the series $\ds \sum_{n=1}^\infty a_n$ \textbf{diverges}.
\end{enumerate}}

Using our new terminology, we can state that the series $\ds \sum_{n=1}^\infty 1/2^n$ converges, and $\ds \sum_{n=1}^\infty 1/2^n = 1.$

\youtubeVideo{cyoiIBs7kIg}{Finding a Formula for a Partial Sum of a Telescoping Series}

We will explore a variety of series in this section. We start with two series that diverge, showing how we might discern divergence.

\example{ex_series1}{Showing series diverge}{\mbox{}\\[-2\baselineskip]
\begin{enumerate}
\item		Let $\{a_n\} = \{n^2\}$. Show $\ds \sum_{n=1}^\infty a_n$ diverges.
\item		Let $\{b_n\} = \{(-1)^{n+1}\}$. Show $\ds \sum_{n=1}^\infty b_n$ diverges.
\end{enumerate}}
{\begin{enumerate}
\item	Consider $S_n$, the $n^\text{th}$ partial sum.
\mfigure{0in}{Scatter plots relating to the series of \autoref{ex_series1} part 1.}{fig:series1a}{figures/figseries1a}
\begin{align*}
	S_n &= a_1+a_2+a_3+\dotsb+a_n \\		
	&= 1^2+2^2+3^2\dotsb+ n^2 \\
	&= \frac{n(n+1)(2n+1)}{6}. \qquad\text{by \autoref{thm:summation}}
\end{align*}
Since $\ds \lim_{n\to\infty}S_n = \infty$, we conclude that the series $\ds \sum_{n=1}^\infty n^2$ diverges. It is instructive to write $\ds \sum_{n=1}^\infty n^2=\infty$ for this tells us \emph{how} the series diverges: it grows without bound.

A scatter plot of the sequences $\{a_n\}$ and $\{S_n\}$ is given in \autoref{fig:series1a}. The terms of $\{a_n\}$ are growing, so the terms of the partial sums $\{S_n\}$ are growing even faster, illustrating that the series diverges.

\item	The sequence $\{b_n\}$ starts with 1, $-1$, 1, $-1$, $\ldots$. Consider some of the partial sums $S_n$ of $\{b_n\}$:
\begin{align*}
S_1 &= 1\\
S_2 &= 0\\
S_3 &= 1\\
S_4 &= 0
\end{align*}
\mfigure{0in}{Scatter plots relating to the series of \autoref{ex_series1} part 2.}{fig:series1b}{figures/figseries1b}
This pattern repeats; we find that
$S_n = \begin{cases}
1  & n\ \text{ is odd}\\
0  & n\ \text{ is even}
\end{cases}$.
As $\{S_n\}$ oscillates, repeating 1, 0, 1, 0, $\ldots$, we conclude that $\ds\lim_{n\to\infty}S_n$ does not exist, hence $\ds\sum_{n=1}^\infty (-1)^{n+1}$ diverges.		

A scatter plot of the sequence $\{b_n\}$ and the partial sums $\{S_n\}$ is given in \autoref{fig:series1b}. When $n$ is odd, $b_n = S_n$ so the marks for $b_n$ are drawn oversized to show they coincide.\eoehere
\end{enumerate}}

While it is important to recognize when a series diverges, we are generally more interested in the series that  converge. In this section we will demonstrate a few general techniques for determining convergence; later sections will delve deeper into this topic.

\subsection*{Geometric Series}

One important type of series is a \emph{geometric series}.

\definition{def:geom_series}{Geometric Series}
{A \textbf{geometric series} is a series of the form 
$$\sum_{n=0}^\infty ar^n = a+ar+ar^2+ar^3+\dotsb+ar^n+\dotsb$$
Note that the index starts at $n=0$, if the index starts at $n=1$ we have $\ds \sum_{n=1}^\infty ar^{n-1}$.%
\index{series!geometric}\index{geometric series}
}

We started this section with a geometric series, although we dropped the first term of $1$. One reason geometric series are important is that they have nice convergence properties.

% todo Tim the statement has $S_n=\sum_0^n$.  The proof has $S_n=\sum_0^{n-1}$
\theorem{thm:geom_series}{Convergence of Geometric Series}
{Consider the geometric series $\ds \sum_{n=0}^\infty ar^n$.
\index{series!geometric}\index{geometric series}\index{convergence!of geometric series}\index{divergence!of geometric series}
\begin{enumerate}
\item	If $r\neq1$, the $n^\text{th}$ partial sum is: $\ds S_n = \frac{a(1-r\,^{n+1})}{1-r}$.
\item	The series converges if, and only if, $\abs r<1$. When $\abs r<1$, 
\[\sum_{n=0}^\infty ar^n = \frac{a}{1-r}.\]
\end{enumerate}}

\begin{proof}
If $r=1$, then $S_n=a+a+a+\dotsb+a=na$. Since $\lim_{n\to \infty} S_n=\pm \infty$, the geometric series diverges.\\
If $r\neq 1$, we have
$$S_n=a+ar+ar^2+\dotsb+ar^{n-1}.$$
Multiply each term by $r$ and we have 
$$rS_n=ar+ar^2+ar^3\dotsb+ar^n.$$
Subtract these two equations and solve for $S_n$.
\begin{align*}
S_n-rS_n &=a-ar^n \\
S_n &=\frac{a(1-r^n)}{1-r}\\
\end{align*}
From \autoref{thm:geom_seq}, we know that if $-1<r<1$, then $\ds \lim_{n\to \infty} r^n=0$ so
$$\lim_{n\to \infty} S_n=\lim_{n\to \infty}=\frac{a(1-r^n)}{1-r}=\frac{a}{1-r}- \frac{a}{1-r}\lim_{n\to \infty}r^n=\frac{a}{1-r}.$$
So when $|r|<1$ the geometric series converges and its sum is $\ds \frac{a}{1-r}$.

If either $r\leq -1$ or $r>1$, the sequence $\{r^n\}$ is divergent by \autoref{thm:geom_seq}. Thus $\ds \lim_{n\to \infty} S_n$ does not exist, so the geometric series diverges if $r\leq -1$ or $r>1$.
\end{proof}

According to \autoref{thm:geom_series}, the series 
$$\ds\sum_{n=0}^\infty \frac{1}{2^n} =\sum_{n=0}^\infty \left(\frac 12\right)^2= 1+\frac12+\frac14+\dotsb$$ converges as $r=1/2$, and $\ds \sum_{n=0}^\infty \frac{1}{2^n} = \frac{1}{1-1/2} = 2.$ This concurs with our introductory example; while there we got a sum of 1, we skipped the first term of 1.\\

\mtable{Scatter plots relating to the series in \autoref{ex_series2}.}{fig:series2}{%
\myincludegraphics{figures/figseries2a}
\smallskip\\(a)\bigskip\\
\myincludegraphics{figures/figseries2b}
\smallskip\\(b)\bigskip\\
\myincludegraphics{figures/figseries2c}
\smallskip\\(c)}

\example{ex_series2}{Exploring geometric series}{Check the convergence of the following series. If the series converges, find its sum.
\[
 \text{1. }\sum_{n=2}^\infty \left(\frac34\right)^n\qquad
 \text{2. }\sum_{n=0}^\infty \left(\frac{-1}{2}\right)^n\qquad
 \text{3. }\sum_{n=0}^\infty 3^n
\]}
{\begin{enumerate}
\item		Since $r=3/4<1$, this series converges. By \autoref{thm:geom_series}, we have that
$$\sum_{n=0}^\infty \left(\frac34\right)^n = \frac{1}{1-3/4} = 4.$$ However, note the subscript of the summation in the given series: we are to start with $n=2$. Therefore we subtract off the first two terms, giving:
$$\sum_{n=2}^\infty \left(\frac34\right)^n = 4 - 1 - \frac34 = \frac94.$$
This is illustrated in \autoref{fig:series2}(a).

\item	Since $\abs r = 1/2 < 1$, this series converges, and by \autoref{thm:geom_series},
$$\sum_{n=0}^\infty \left(\frac{-1}{2}\right)^n = \frac{1}{1-(-1/2)} = \frac23.$$
The partial sums of this series are plotted in \autoref{fig:series2}(b). Note how the partial sums are not purely increasing as some of the terms of the sequence $\{(-1/2)^n\}$ are negative.

\item		Since $r>1$, the series diverges. (This makes ``common sense''; we expect the sum $$1+3+9+27 + 81+243+\dotsb$$ to diverge.) This is illustrated in \autoref{fig:series2}(c).\eoehere
\end{enumerate}}

Later sections will provide tests by which we can determine whether or not a given series converges. This, in general, is much easier than determining \emph{what} a given series converges to. There are many cases, though, where the sum can be determined.

\example{ex_series3}{Telescoping series}{Evaluate the sum $\ds \sum_{n=1}^\infty \left(\frac1n-\frac1{n+1}\right)$.
\index{series!telescoping}\index{telescoping series}}
{It will help to write down some of the first few partial sums of this series.
\begin{align*}
S_1 &=	\frac11-\frac12 & & = 1-\frac12\\
S_2 &=	\left(\frac11-\frac12\right) + \left(\frac12-\frac13\right) & & = 1-\frac13\\
S_3 &=	\left(\frac11-\frac12\right) + \left(\frac12-\frac13\right)+\left(\frac13-\frac14\right) & &= 1-\frac14\\
S_4 &=	\left(\frac11-\frac12\right) + \left(\frac12-\frac13\right)+\left(\frac13-\frac14\right) +\left(\frac14-\frac15\right)& &= 1-\frac15
\end{align*}
\mfigure{0in}{Scatter plots relating to the series of \autoref{ex_series3}.}{fig:series3}{figures/figseries3}
Note how most of the terms in each partial sum subtract out. In general, we see that $\ds S_n = 1-\frac{1}{n+1}$. The sequence $\{S_n\}$ converges,  as $\ds \lim_{n\to\infty}S_n = \lim_{n\to\infty}\left(1-\frac1{n+1}\right) = 1$, and so we conclude that $\ds \sum_{n=1}^\infty \left(\frac1n-\frac1{n+1}\right) = 1$. Partial sums of the series are plotted in \autoref{fig:series3}.}

The series in \autoref{ex_series3} is an example of a \sword{telescoping series}. Informally, a telescoping series is one in which the partial sums reduce to just a finite number of terms. The partial sum $S_n$ did not contain $n$ terms, but rather just two: 1 and $1/(n+1)$.\index{series!telescoping}\index{telescoping series}

When possible, seek a way to write an explicit formula for the $n^\text{th}$ partial sum $S_n$. This makes evaluating the limit $\ds\lim_{n\to\infty} S_n$ much more approachable. We do so in the next example.

%\noindent\textbf{Note on notation:} Most of the series we encounter will start with $n=1$. For ease of notation, we will often write $\sum a_n$ instead of writing $\ds\sum_{n=1}^\infty a_n$.\\


\example{ex_series4}{Evaluating series}{Evaluate each of the following infinite series.
\[
 \text{1.}\quad\sum_{n=1}^\infty \frac{2}{n^2+2n}\qquad\qquad
 \text{2.}\quad\sum_{n=1}^\infty \ln\left(\frac{n+1}{n}\right)
\]}
{\begin{enumerate}
\item		We can decompose the fraction $2/(n^2+2n)$ as
$$\frac2{n^2+2n} = \frac1n-\frac1{n+2}.$$
(See \autoref{sec:partial_fraction}, Partial Fraction Decomposition, to recall how  this is done, if necessary.)

Expressing the terms of $\{S_n\}$ is now more instructive:\\
\flushinnerequ{\footnotesize
\begin{align*}
S_1 &= 1-\frac13 &&= 1-\frac13\\
S_2 &= \left(1-\frac13\right) + \left(\frac12-\frac14\right) &&= 1+\frac12-\frac13-\frac14\\
S_3 &= \left(1-\frac13\right) + \left(\frac12-\frac14\right) + \left(\frac13-\frac15\right)
&&= 1+\frac12-\frac14-\frac15\\
S_4 &= \left(1-\frac13\right) + \left(\frac12-\frac14\right) + \left(\frac13-\frac15\right) + \left(\frac14-\frac16\right)
&&= 1+\frac12-\frac15-\frac16\\
S_5 &= \left(1-\frac13\right) + \left(\frac12-\frac14\right) + \left(\frac13-\frac15\right) + \left(\frac14-\frac16\right) + \left(\frac15-\frac17\right)
&&= 1+\frac12-\frac16-\frac17
\end{align*}}\\
\mfigure{0in}{Scatter plots relating to the series of \autoref{ex_series4} part 1.}{fig:series4a}{figures/figseries4a}
We again have a telescoping series. In each partial sum, most of the terms pair up to add to zero and we obtain the formula $\ds S_n = 1+\frac12-\frac1{n+1}-\frac1{n+2}.$ Taking limits allows us to determine the convergence of the series:\\
\flushinnerequ{%
$$\lim_{n\to\infty}S_n = \lim_{n\to\infty} \left(1+\frac12-\frac1{n+1}-\frac1{n+2}\right) = \frac32,\quad \text{so } \sum_{n=1}^\infty \frac1{n^2+2n} = \frac32.$$}
This is illustrated in \autoref{fig:series4a}.

\item	We begin by writing the first few partial sums of the series:
\begin{align*}
S_1 &= \ln\left(2\right) \\
S_2 &= \ln\left(2\right)+\ln\left(\frac32\right) \\
S_3 &= \ln\left(2\right)+\ln\left(\frac32\right)+\ln\left(\frac43\right) \\
S_4 &= \ln\left(2\right)+\ln\left(\frac32\right)+\ln\left(\frac43\right)
+\ln\left(\frac54\right) 
\end{align*}
At first, this does not seem helpful, but recall the logarithmic identity: $\ln x+\ln y = \ln (xy).$ Applying this to $S_4$ gives:
$$S_4 = \ln\left(2\right)+\ln\left(\frac32\right)+\ln\left(\frac43\right)
+\ln\left(\frac54\right)
=\ln\left(\frac21\cdot\frac32\cdot\frac43\cdot\frac54\right)
=\ln\left(5\right).$$
We must generalize this for $S_n$.\\
\flushinnerequ{%
$$S_n=\ln\left(2\right)+\ln\left(\frac32\right)+\cdots +\ln \left(\frac{n+1}{n}\right)=\ln\left(\frac21\cdot\frac32 \cdots  \frac{n}{n-1}\cdot \frac{n+1}{n}\right)=\ln ( n+1)$$}

%\clearpage

\mfigure{0in}{Scatter plots relating to the series of \autoref{ex_series4} part 2.}{fig:series4b}{figures/figseries4b}
We can conclude that $\{S_n\} = \big\{\ln (n+1)\big\}$. This sequence  does not converge, as $\ds \lim_{n\to\infty}S_n=\infty$. Therefore  $\ds\sum_{n=1}^\infty  \ln\left(\frac{n+1}{n}\right)=\infty$; the series diverges. Note in \autoref{fig:series4b} how the sequence of partial sums grows slowly; after 100 terms, it is not yet over 5. Graphically we may be fooled into thinking the series converges, but our analysis above shows that it does not.\eoehere
\end{enumerate}}

We are learning about a new mathematical object, the series. As done before, we apply ``old'' mathematics to this new topic.

\theorem{thm:series_prop}{Properties of Infinite Series}
{Suppose that \quad$\ds \sum_{n=1}^\infty a_n$\quad and \quad $\ds\sum_{n=1}^\infty b_n$ are convergent series, and that \quad$\ds \sum_{n=1}^\infty a_n = L$,\quad  $\ds\sum_{n=1}^\infty b_n = K$, and $c$ is a constant.
\index{series!properties}\index{Sum/Difference Rule!of series}
\index{Constant Multiple Rule!of series}
\begin{enumerate}
\item  Constant Multiple Rule: $\ds\sum_{n=1}^\infty c\cdot a_n = c\cdot\sum_{n=1}^\infty a_n = c\cdot L.$
\item		Sum/Difference Rule: $\ds\sum_{n=1}^\infty \big(a_n\pm b_n\big) = \sum_{n=1}^\infty a_n \pm \sum_{n=1}^\infty b_n = L \pm K.$
\end{enumerate}}

Before using this theorem, we will consider the harmonic series $\ds \sum_{n\to \infty} \frac{1}{n}$.

\example{ex_harm_series}{Divergence of the Harmonic Series}{Show that the harmonic series $\ds \sum_{n\to \infty} \frac{1}{n}$ diverges.}{We will use a proof by contradiction here. Suppose the harmonic series converges to $S$. That is $$S=1+\frac12+ \frac13 +\frac14+\frac15+\frac16+\frac17+\frac18+\cdots$$
We then have
\begin{align*}
S &\geq 1+\frac12+\frac14+\frac14+\frac16+\frac16+\frac18+\frac18+\cdots\\
&=1+\frac12+\frac12\phantom{+\frac14}+\frac13\phantom{+\frac16}
+\frac14\phantom{+\frac18}+\cdots\\
&=\frac12+S
\end{align*}
This gives us $S\geq \frac12+S$ which can never be true, thus our assumption that the harmonic series converges must be false. Therefore, the harmonic series diverges.}

%Before using this theorem, we provide a few ``famous'' series.
%
%\setboxwidth{20pt}
%\keyidea{idea:famous_series}{Important Series}
%{\begin{enumerate}
%\item	\parbox{90pt}{$\ds\sum_{n=0}^\infty \frac1{n!} = e$. } (Note that the index starts with $n=0$.)
%\item	$\ds\sum_{n=1}^\infty \frac1{n^2} = \frac{\pi^2}{6}$.
%\item	$\ds\sum_{n=1}^\infty \frac{(-1)^{n+1}}{n^2} = \frac{\pi^2}{12}$.
%\item	$\ds\sum_{n=0}^\infty \frac{(-1)^{n}}{2n+1} = \frac{\pi}{4}$.
%\item	\parbox{90pt}{$\ds\sum_{n=1}^\infty \frac{1}{n} $ \quad diverges.} (This is called the \emph{Harmonic Series}.)\index{Harmonic Series}
%\item	\parbox{90pt}{$\ds\sum_{n=1}^\infty \frac{(-1)^{n+1}}{n} = \ln 2$.} (This is called the \emph{Alternating Harmonic Series}.)\index{Alternating Harmonic Series}
%\end{enumerate}}
%
%\example{ex_series5}{Evaluating series}{Evaluate the given series.
%
%\[
%\text{1.}\quad\sum_{n=1}^\infty \frac{(-1)^{n+1}\big(n^2-n\big)}{n^3}\qquad
%\text{2.}\quad\sum_{n=1}^\infty \frac{1000}{n!}\qquad
%\text{3.}\quad\frac1{16}+\frac1{25}+\frac1{36}+\frac1{49}+\dotsb
%\]}
%{\begin{enumerate}
%\item	We start by using algebra to break the series apart:
%\begin{align*}
%\sum_{n=1}^\infty \frac{(-1)^{n+1}\big(n^2-n\big)}{n^3} &= \sum_{n=1}^\infty\left(\frac{(-1)^{n+1}n^2}{n^3}-\frac{(-1)^{n+1}n}{n^3}\right) \\
%						&= \sum_{n=1}^\infty\frac{(-1)^{n+1}}{n}-\sum_{n=1}^\infty\frac{(-1)^{n+1}}{n^2} \\
%						&= \ln(2) - \frac{\pi^2}{12}	\approx	-0.1293.
%\end{align*}
%
%This is illustrated in \autoref{fig:series5}(a).
%%\mfigure{0in}{Scatter plots relating to the series of \autoref{ex_series5} part 1.}{fig:series5a}{figures/figseries5a}
%
%\item		This looks very similar to the series that involves $e$ in \autoref{idea:famous_series}. Note, however, that the series given in this example starts with $n=1$ and not $n=0$. The first term of the series in the Key Idea is $1/0! = 1$, so we will subtract this from our result below:
%\begin{align*}
%		\sum_{n=1}^\infty \frac{1000}{n!} &= 1000\cdot\sum_{n=1}^\infty \frac{1}{n!} \\
%							&= 1000\cdot (e-1) \approx  1718.28.
%\end{align*}
%This is illustrated in \autoref{fig:series5}(b). The graph shows how this particular series converges very rapidly.
%%\mfigure{0in}{Scatter plots relating to the series of \autoref{ex_series5} part 2.}{fig:series5b}{figures/figseries5b}
%\mtable{Scatter plots relating to the series in \autoref{ex_series5}.}{fig:series5}{%
%\begin{tabular}{c}
%\myincludegraphics{figures/figseries5a}\\[10pt]
%(a)\\[15pt]
%\myincludegraphics{figures/figseries5b}\\[10pt]
%(b)
%\end{tabular}
%}
%
%\item		The denominators in each term are perfect squares; we are adding $\ds \sum_{n=4}^\infty \frac{1}{n^2}$ (note  we start with $n=4$, not $n=1$). This series will converge. Using the formula from \autoref{idea:famous_series}, we have the following:
%\begin{align*}
%\sum_{n=1}^\infty \frac1{n^2} &= \sum_{n=1}^3 \frac1{n^2} +\sum_{n=4}^\infty \frac1{n^2} \\
%\sum_{n=1}^\infty \frac1{n^2} - \sum_{n=1}^3 \frac1{n^2} &=\sum_{n=4}^\infty \frac1{n^2} \\
%\frac{\pi^2}{6} - \left(\frac11+\frac14+\frac19\right) &= \sum_{n=4}^\infty \frac1{n^2} \\
%\frac{\pi^2}{6} - \frac{49}{36} &= \sum_{n=4}^\infty \frac1{n^2} \\
%0.2838&\approx \sum_{n=4}^\infty \frac1{n^2} 
%\end{align*}
%\end{enumerate}}

It may take a while before one is comfortable with this statement, whose truth lies at the heart of the study of infinite series: \emph{it is possible that the sum of an infinite list of nonzero numbers is finite.} We have seen this repeatedly in this section, yet it still may ``take some getting used to.''

As one contemplates the behavior of series, a few facts become clear. 
\begin{enumerate}
\item		In order to add an infinite list of nonzero numbers and get a finite result, ``most'' of those numbers must be ``very near'' 0. 
\item		If a series diverges, it means that the sum of an infinite list of numbers is not finite (it may approach $\pm \infty$ or it may oscillate), and:
\begin{enumerate}
	\item	The series will still diverge if the first term is removed.
	\item	The series will still diverge if the first 10 terms are removed.
	\item	The series will still diverge if the first $1{,}000{,}000$ terms are removed.
	\item	The series will still diverge if any finite number of terms from anywhere in the series are removed.
\end{enumerate}
\end{enumerate}

These concepts are very important and lie at the heart of the next two theorems.

\theorem{thm:series_conv}{Convergence of Sequence}{If the series $\ds \sum_{n\to\infty}a_n$ converges, then $\lim_{n\to\infty}a_n=0$.}

\begin{proof}
Let $S_n=a_1+a_2+\cdots+a_n$. We have 
\begin{align*}
S_n&=a_1+a_2+\cdots+a_{n-1}+a_n\\
S_n&=S_{n-1}+a_n\\
a_n&=S_n-S_{n-1}
\end{align*}
Since  $\ds \sum_{n\to\infty}a_n$ converges, the sequence $\{ S_n\}$ converges.  Let $\ds \lim_{n\to\infty} S_n=S$. As $n\to \infty$, $n-1$ also goes to $\infty$, so $\ds \lim_{n\to\infty} S_{n-1}=S$. We now have
\begin{align*}
\lim_{n\to\infty} a_n&= \lim_{n\to\infty}(S_n-S_{n-1})\\
&= \lim_{n\to\infty}S_n - \lim_{n\to\infty} S_{n-1}\\
&=S-S=0\qedhere
\end{align*}
\end{proof}

\theorem{thm:series_nth_term}{Test for Divergence}{If $ \ds \lim_{n\to\infty} a_n$ does not exist or $\ds  \lim_{n\to\infty}a_n\neq0$, then the series $\ds \sum_{n=1}^\infty a_n$ diverges.}

The Test for Divergence follows from \autoref{thm:series_conv}. If the series does not diverge, it must converge and therefore $\ds  \lim_{n\to\infty}a_n=0$.

Note that the two statements in Theorems \ref{thm:series_conv} and \ref{thm:series_nth_term} are really the same. In order to converge, the limit of the terms of the sequence must approach 0; if they do not, the series will not converge. 

Looking back, we can apply this theorem to the series in \autoref{ex_series1}. In that example, we had $\{a_n\} = \{n^2\}$ and $\{b_n\} = \{(-1)^{n+1}\}$.
$$\lim_{n\to\infty} a_n=\lim_{n\to\infty} n^2=\infty$$
and
$$\lim_{n\to\infty} b_n=\lim_{n\to\infty}(-1)^{n+1}\text{ which does not exist.}$$
Thus by the Test for Divergence, both series will diverge.

\textbf{Important!} This theorem \emph{does not state} that if $\ds \lim_{n\to\infty} a_n = 0$ then $\ds \sum_{n=1}^\infty  a_n $ converges. The standard example of this is the Harmonic Series, as given in \autoref{ex_harm_series}. The Harmonic Sequence, $\{1/n\}$, converges to 0; the Harmonic Series, $\ds \sum_{n=1}^\infty 1/n$, diverges.

\theorem{thm:series_behavior}{Infinite Nature of Series}
{The convergence or divergence remains unchanged by the insertion or deletion of any finite number of terms. That is:
\begin{enumerate}
	\item	A divergent series will remain divergent with the insertion or deletion of any finite number of terms.
	\item	A convergent series will remain convergent with the insertion or deletion of any finite number of terms. (Of course, the \emph{sum} will likely change.)
\end{enumerate}}

In other words, when we are only interested in the convergence or divergence of a series, it is safe to ignore the first few billion terms.

\example*{ex_trunc_harm}{Removing Terms from the Harmonic Series}{Consider once more the Harmonic Series $\ds\sum_{n=1}^\infty\frac1n$\vspace{-.8\baselineskip} which diverges; that is, the partial sums $S_N=\ds\sum_{n=1}^N\frac1n$ grow (very, very slowly) without bound. One might think that by removing the ``large'' terms of the sequence that perhaps the series will converge. This is simply not the case. For instance, the sum of the first 10 million terms of the Harmonic Series is about 16.7. Removing the first 10 million terms from the Harmonic Series changes the partial sums,  effectively subtracting 16.7 from the sum. However, a sequence that is growing without bound will still grow without bound when 16.7 is subtracted from it. 

The equation below illustrates this. Even though we have subtracted off the first 10 million terms, this only subtracts a constant off of an expression that is still growing to infinity. Therefore, the modified series is still growing to infinity.
\begin{multline*}
 \sum_{n=10,000,001}^\infty\frac1n
 =\lim_{N\to\infty}\sum_{n=10,000,001}^N\frac1n
 =\lim_{N\to\infty}\sum_{n=1}^N\frac1n-\sum_{n=1}^{10,000,001}\frac1n \\
 =\lim_{N\to\infty}\sum_{n=1}^N\frac1n-16.7
 =\infty.\eoehere
\end{multline*}}

This section introduced us to series and defined a few special types of series whose convergence properties are well known. We know when a geometric series converges or diverges. Most series that we encounter are not one of these types, but we are still interested in knowing whether or not they converge. The next three sections introduce tests that help us determine whether or not a given series converges. 

\printexercises{exercises/08_02_exercises}

\input{text/08_Integral_Comparison_Tests}
\input{text/08_Ratio_Root_Tests}
\input{text/08_Alternating_Series}
\section{Power Series}\label{sec:power_series}

So far, our study of series has examined the question of ``Is the sum of these infinite terms finite?,'' i.e., ``Does the series converge?'' We now approach series from a different perspective: as a function. Given a value of $x$, we evaluate $f(x)$ by finding the sum of a particular series that depends on $x$ (assuming the series converges). We start this new approach to series with a definition.

\definition{def:power_series}{Power Series}
{Let $\{a_n\}$ be a sequence, let $x$ be a variable, and let $c$ be a real number.
\index{power series}\index{series!power}
	\begin{enumerate}
		\item The \sword{power series in $x$} is the series
		$$\sum_{n=0}^\infty a_nx^n = a_0+a_1x+a_2x^2+a_3x^3+\dotsb$$
		
		\item The \sword{power series in $x$ centered at $c$} is the series
		$$\sum_{n=0}^\infty a_n(x-c)^n = a_0+a_1(x-c)+a_2(x-c)^2+a_3(x-c)^3+\dotsb$$
	\end{enumerate}
}

\example{ex_ps1}{Examples of power series}{Write out the first five terms of the following power series:
\[
 \text{1. }\sum_{n=0}^\infty x^n \qquad\qquad
 \text{2. }\sum_{n=1}^\infty (-1)^{n+1}\frac{(x+1)^n}n\qquad\qquad
 \text{3. }\sum_{n=0}^\infty (-1)^{n+1} \frac{(x-\pi)^{2n}}{(2n)!}.
\]}
{\begin{enumerate}
	\item One of the conventions we adopt is that $x^0=1$ regardless of the value of $x$. Therefore
	$$\sum_{n=0}^\infty x^n = 1+x+x^2+x^3+x^4+\dotsb$$
	This is a geometric series in $x$.
	
	\item	This series is centered at $c=-1$. Note how this series starts with $n=1$. We could rewrite this series starting at $n=0$ with the understanding that $a_0=0$, and hence the first term is $0$.\\
	\flushinnerequ{%
	$$\sum_{n=1}^\infty (-1)^{n+1}\frac{(x+1)^n}n = (x+1) - \frac{(x+1)^2}{2} + \frac{(x+1)^3}{3} - \frac{(x+1)^4}{4}+\frac{(x+1)^5}{5}\dotsb$$}
	
	\item		This series is centered at $c=\pi$. Recall that $0!=1$.\\
	\flushinnerequ{%
	$$\sum_{n=0}^\infty (-1)^{n+1} \frac{(x-\pi)^{2n}}{(2n)!} = -1+\frac{(x-\pi)^2}{2} - \frac{(x-\pi)^4}{24}+ \frac{(x-\pi)^6}{6!}-\frac{(x-\pi)^8}{8!}\dotsb\eoehere $$}
\end{enumerate}}

We introduced power series as a type of function, where a value of $x$ is given and the sum of a series is returned. Of course, not every series converges. For instance, in part 1 of \autoref{ex_ps1}, we recognized the series $\ds \sum_{n=0}^\infty x^n$ as a geometric series in $x$. \autoref{thm:geom_series} states that this series converges only when $|x|<1$. 

This raises the question: ``For what values of $x$ will a given power series converge?,'' which  leads us to a theorem and definition.

\theorem{thm:radius_converge}{Convergence of Power Series}
{Let a power series $\ds \sum_{n=0}^\infty a_n(x-c)^n$ be given. Then one of the following is true:
\index{convergence!of power series}\index{power series!convergence}\index{series!power}
\begin{enumerate}
	\item The series converges only at $x=c$.
	\item	There is an $R>0$ such that the series converges for all $x$ in \\	
	$(c-R,c+R)$ and diverges for all $x<c-R$ and $x>c+R$.
	\item	The series converges for all $x$.
\end{enumerate}
}

%A corollary to \autoref{thm:radius_converge} is this: given a series $\ds \sum_{n=0}^\infty a_n(x-c)^n$ for some real number $c$, one of the following is true:
	%\begin{enumerate}
		%\item The series converges only at $x=c$.
		%\item	There is an $R>0$ such that the series converges for all $x$ in $(c-R, c+R)$ and diverges for all $x<c-R$ and $x>c+R$.
		%\item	The series converges for all $x$.
	%\end{enumerate}
	
The value of $R$ is important when understanding a power series, hence it is given a name in the following definition. Also, note that part 2 of \autoref{thm:radius_converge} makes a statement about the interval $(c-R,c+R)$, but the not the endpoints of that interval. A series may or may not converge at these endpoints.
	
\definition{def:radius_converge}{Radius and Interval of Convergence}
{\mbox{}\\[-2\baselineskip]\index{convergence!radius of}\index{convergence!interval of}\index{radius of convergence}\index{interval of convergence}\index{series!radius of convergence}\index{series!interval of convergence}
\begin{enumerate}
	\item The number $R$ given in \autoref{thm:radius_converge} is the \sword{radius of convergence} of a given series. When a series converges for only $x=c$, we say the radius of convergence is 0, i.e.,  $R=0$. When a series converges for all $x$, we say the series has an infinite radius of convergence, i.e., $R=\infty$.
	\item	The \sword{interval of convergence} is the set of all values of $x$ for which the series converges.
\end{enumerate}}

To find the values of $x$ for which a given series converges, we will use the convergence tests we studied previously (especially the Ratio Test). However, the tests all required that the terms of a series be positive. The following theorem gives us a work--around to this problem.

\theorem{thm:abs_power}{The Radius of Convergence of a Series and Absolute Convergence}
{%{\begin{itemize}
	%\item The series $\ds \sum_{n=0}^\infty a_nx^n$ and $\ds \sum_{n=0}^\infty \big|a_nx^n\big|$ have the same radius of convergence $R$.
	%\item 
	The series $\ds \sum_{n=0}^\infty a_n(x-c)^n$ and $\ds \sum_{n=0}^\infty \abs{a_n(x-c)^n}$ have the same radius of convergence $R$.
%\end{itemize}
}

\autoref{thm:abs_power} allows us to find the radius of convergence $R$ of a series by applying the Ratio Test (or any applicable test) to the absolute value of the terms of the series. We practice this in the following example.

\youtubeVideo{01LzAU__J-0}{Power Series --- Finding the Interval of Convergence}

\clearpage

\example{ex_ps2}{Determining the radius and interval of convergence.}{Find the radius and interval of convergence for each of the following series:
\[
 \text{1. }\sum_{n=0}^\infty \frac{x^n}{n!} \qquad
 \text{2. }\sum_{n=1}^\infty (-1)^{n+1}\frac{x^n}{n}\qquad
 \text{3. }\sum_{n=0}^\infty 2^n(x-3)^n\qquad
 \text{4. }\sum_{n=0}^\infty n!x^n
\]}
{\begin{enumerate}
	\item We apply the Ratio Test to the series $\ds \sum_{n=0}^\infty \left|\frac{x^n}{n!}\right|$:
		\begin{align*}
		\lim_{n\to\infty} \frac{\big|x^{n+1}/(n+1)!\big|}{\big|x^n/n!\big|} &= \lim_{n\to\infty} \left|\frac{x^{n+1}}{x^n}\cdot\frac{n!}{(n+1)!}\right|\\
			&= \lim_{n\to\infty} \left|\frac x{n+1}\right|\\
			&= 0 \text{ for all } x.
		\end{align*}
		The Ratio Test shows us that regardless of the choice of $x$, the series converges. Therefore the radius of convergence is $R=\infty$, and the interval of convergence is $(-\infty,\infty)$.
		
	\item		We apply the Ratio Test to the series $\ds \sum_{n=1}^\infty \left|(-1)^{n+1}\frac{x^n}{n}\right| = \sum_{n=1}^\infty \left|\frac{x^n}{n}\right|$:
	\begin{align*}
	\lim_{n\to\infty} \frac{\big|x^{n+1}/(n+1)\big|}{\big|x^n/n\big|} &= \lim_{n\to\infty} \left|\frac{x^{n+1}}{x^n}\cdot \frac{n}{n+1}\right| \\
			&= \lim_{n\to\infty} |x|\frac{n}{n+1}\\
			&= |x|.
	\end{align*}
	The Ratio Test states a series converges if the limit of $|a_{n+1}/a_n| = L<1$. We found the limit above to be $|x|$; therefore, the power series converges when $|x| <1$, or when $x$ is in $(-1,1)$. Thus the radius of convergence is $R=1$.
	
	To determine the interval of convergence, we need to check the endpoints of $(-1,1)$. When $x=-1$, we have the opposite of the Harmonic Series:
	\begin{align*}
	\sum_{n=1}^\infty (-1)^{n+1}\frac{(-1)^n}{n}
	&= \sum_{n=1}^\infty \frac{(-1)^{2n+1}}{n}\\
	&= \sum_{n=1}^\infty \frac{-1}{n}\\
	&= -\infty.
	\end{align*}
	The series diverges when $x=-1$.
	
	When $x=1$, we have the series $\ds \sum_{n=1}^\infty (-1)^{n+1}\frac{(1)^n}{n}$, which is the Alternating Harmonic Series, which converges. Therefore the interval of convergence is $(-1,1]$.
	
	\item		We apply the Ratio Test to the series $\ds\sum_{n=0}^\infty \big|2^n(x-3)^n\big|$:
	\begin{align*}
	\lim_{n\to\infty} \frac{\big| 2^{n+1}(x-3)^{n+1}\big|}{\big|2^n(x-3)^n\big|} &= \lim_{n\to\infty} \left|\frac{2^{n+1}}{2^n}\cdot\frac{(x-3)^{n+1}}{(x-3)^n}\right|\\
			&=\lim_{n\to\infty} \big|2(x-3)\big|.
	\end{align*}
	
According to the Ratio Test, the series converges when $\big|2(x-3)\big|<1 \implies \big|x-3\big| < 1/2$. The series is centered at 3, and $x$ must be within $1/2$ of 3 in order for the series to converge. Therefore the radius of convergence is $R=1/2$, and we know that the series converges absolutely for all $x$ in $(3-1/2,3+1/2) = (2.5, 3.5)$.

We check for convergence at the endpoints to find the interval of convergence. When $x=2.5$, we have:
\begin{align*}
\sum_{n=0}^\infty 2^n(2.5-3)^n &= \sum_{n=0}^\infty 2^n(-1/2)^n \\
			&=\sum_{n=0}^\infty (-1)^n,
\end{align*}
which diverges. A similar process shows that the series also diverges at $x=3.5$. Therefore the interval of convergence is $(2.5, 3.5)$.

\item		We apply the Ratio Test to $\ds \sum_{n=0}^\infty \big|n!x^n\big|$:
\begin{align*}
\lim_{n\to\infty} \frac{\big| (n+1)!x^{n+1}\big|}{\big|n!x^n\big|} &= \lim_{n\to\infty} \big|(n+1)x\big|\\
		&= \infty\ \text{ for all $x$, except $x=0$.}
\end{align*}

The Ratio Test shows that the series diverges for all $x$ except $x=0$. Therefore the radius of convergence is $R=0$.\eoehere
\end{enumerate}}

\subsection*{Power Series as Functions}

We can use a power series to define a function:
$$f(x) = \sum_{n=0}^\infty a_nx^n$$
where the domain of $f$ is a subset of the interval of convergence of the power series. One can apply calculus techniques to such functions; in particular, we can find derivatives and antiderivatives. 

\theorem{thm:calc_power_series}{\parbox[t]{210pt}{Derivatives and Indefinite Integrals of Power Series Functions}}
{Let $\ds f(x) = \sum_{n=0}^\infty a_n(x-c)^n$ be a function defined by a power series, with radius of convergence $R$.
\index{series!power!derivatives and integrals}\index{integration!of power series}\index{derivative!power series}\index{power series!derivatives and integrals}
\begin{enumerate}
	\item $f(x)$ is continuous and differentiable on $(c-R,c+R)$.
	\item	$\ds \fp(x) = \sum_{n=1}^\infty a_n\cdot n\cdot (x-c)^{n-1}$, with radius of convergence $R$.
	\item	$\ds \int f(x)\ dx = C+\sum_{n=0}^\infty a_n\frac{(x-c)^{n+1}}{n+1}$, with radius of convergence $R$.
\end{enumerate}}

A few notes about \autoref{thm:calc_power_series}:
\begin{enumerate}
	\item The theorem states that differentiation and integration do not change the radius of convergence. It does not state anything about the \emph{interval} of convergence. They are not always the same.
	\item	Notice how the summation for $\fp(x)$ starts with $n=1$. This is because the constant term $a_0$ of $f(x)$ goes to 0.
	\item	Differentiation and integration are simply calculated term--by--term using previous rules of integration and differentiation.
\end{enumerate}

\example{ex_ps3}{Derivatives and indefinite integrals of power series}{Let $\ds f(x) = \sum_{n=0}^\infty x^n$. Find the following along with their respective intervals of convergence.
\[
 \text{1.}\quad\fp(x)
 \qquad\text{and}\qquad
 \text{2.}\quad\ds F(x) =\int f(x)\ dx
\]}
{We find the derivative and indefinite integral of $f(x)$, following \autoref{thm:calc_power_series}.

\begin{enumerate}
\item\mbox{}\\[-3\baselineskip]
\begin{align*}
f(x)&=1+x+\,x^2+\,x^3+\,x^4+\dotsb = \sum_{n=0}^\infty x^n\\
\fp(x) &= 0+1+2x+3x^2+4x^3+\dotsb=\sum_{n=1}^\infty nx^{n-1} 
\end{align*}

In \autoref{ex_ps1}, we recognized that $\ds \sum_{n=0}^\infty x^n$ is a geometric series in $x$. We know that such a geometric series converges when $|x|<1$; that is, the interval of convergence is $(-1,1)$.

To determine the interval of convergence of $\fp(x)$, we consider the endpoints of $(-1,1)$.
When $x=-1$ we have $$\fp(-1) = \sum_{n=1}^\infty n(-1)^{n-1}$$
which diverges by the Test for Divergence
and when $x=1$ we have
$$\fp(1) = \sum_{n=1}^\infty n$$
which also diverges by the Test for Divergence. Therefore, the interval of convergence of $\fp(x)$ is $(-1,1)$. 

\item\mbox{}\\[-3\baselineskip]
\begin{align*}
 f(x)&=\phantom{C+{}}1+\,x+\,x^2+\,x^3+\dotsb = \sum_{n=0}^\infty x^n\\
 F(x) = \int f(x)\ dx &= C+ x+\frac{x^2}{2}+\frac{x^3}3+\frac{x^4}4+\dotsb \\
 &= C+\sum_{n=0}^\infty \frac{x^{n+1}}{n+1}=C+\sum_{n=1}^\infty \frac{x^{n}}{n}  
\end{align*}

To find the interval of convergence of $F(x)$, we again consider the endpoints of $(-1,1)$.
When $x=-1$ we have
$$F(-1) = C+\sum_{n=1}^\infty \frac{(-1)^{n}}{n}$$
The value of $C$ is irrelevant; notice that the rest of the series is an Alternating Series that whose terms converge to 0. By the Alternating Series Test, this series converges. (In fact, we can recognize that the terms of the series after $C$ are the opposite of the Alternating Harmonic Series. We can thus say that $F(-1) = C-\ln 2$.)
$$F(1) = C+\sum_{n=1}^\infty \frac{1}{n} $$
Notice that this summation is $C\ +$ the Harmonic Series, which diverges. Since $F$ converges for $x=-1$ and diverges for $x=1$, the interval of convergence of $F(x)$ is $[-1,1)$.\eoehere
\end{enumerate}}

The previous example showed how to take the derivative and indefinite integral of a power series without motivation for why we care about such operations. We may care for the sheer mathematical enjoyment ``that we can'', which is motivation enough for many. However, we would be remiss to not recognize that we can learn a great deal from taking derivatives and indefinite integrals.\bigskip

Recall that $\ds f(x) = \sum_{n=0}^\infty x^n$ in \autoref{ex_ps3} is a geometric series. According to \autoref{thm:geom_series}, this series converges to $1/(1-x)$ when $|x|<1$. Thus we can say
$$	f(x) = \sum_{n=0}^\infty x^n = \frac 1{1-x},\quad \text{ on }\quad (-1,1).$$

Integrating the power series, (as done in \autoref{ex_ps3},) we find
\begin{equation} F(x)  = C_1+\sum_{n=0}^\infty \frac{x^{n+1}}{n+1},\label{eq:ps3a}\end{equation}
while integrating the function $f(x) = 1/(1-x)$ gives
\begin{equation} F(x)  = -\ln\abs{1-x} + C_2.\label{eq:ps3b}\end{equation}

Equating Equations \eqref{eq:ps3a} and \eqref{eq:ps3b}, we have 
$$F(x) = C_1+\sum_{n=0}^\infty \frac{x^{n+1}}{n+1} = -\ln\abs{1-x} + C_2.$$
Letting $x=0$, we have $F(0) = C_1 = C_2$. This implies that we can drop the constants and conclude
$$\sum_{n=0}^\infty \frac{x^{n+1}}{n+1} = -\ln\abs{1-x}.$$
We established in \autoref{ex_ps3} that the series $\ds\sum_{n=0}^\infty \frac{x^{n+1}}{n+1}$ converges at $x=-1$; substituting $x=-1$ on both sides of the above equality gives
$$-1+\frac12-\frac13+\frac14-\frac15+\dotsb = -\ln 2.$$
On the left we have the opposite of the Alternating Harmonic Series; on the right, we have $-\ln 2$. We conclude that 
$$1-\frac12+\frac13-\frac14+\dotsb = \ln 2.$$
In \autoref{ex_alt1} of \autoref{sec:alt_series} we said the Alternating Harmonic Series converges to $\ln 2$, but did not show why this was the case. The work above shows how we conclude that the Alternating Harmonic Series Converges to $\ln 2$. \index{Alternating Harmonic Series}

We use this type of analysis in the next example.

\example{ex_ps4}{Analyzing power series functions}{Let $\ds f(x) = \sum_{n=0}^\infty \frac{x^n}{n!}$. Find $\ds \fp(x)$ and $\ds \int f(x)\ dx$, and use these to analyze the behavior of $f(x)$.}
{We start by making two notes: first, in \autoref{ex_ps2}, we found the interval of convergence of this power series is $(-\infty,\infty)$. Second, we will find it useful later to have a  few terms of the series written out:
\begin{equation}
\sum_{n=0}^\infty \frac{x^n}{n!}
= 1 + x + \frac{x^2}2+\frac{x^3}{6} + \frac{x^4}{24} +\dotsb\label{eq:ps4}
\end{equation}

We now find the derivative:
\begin{align*}
\fp(x) &= \sum_{n=1}^\infty n\frac{x^{n-1}}{n!} \\
&=\sum_{n=1}^\infty \frac{x^{n-1}}{(n-1)!} = 1+x+\frac{x^2}{2!}+\dotsb. 
\intertext{Since the series starts at $n=1$ and each term refers to $(n-1)$, we can re-index the series starting with $n=0$:}
		&= \sum_{n=0}^\infty \frac{x^{n}}{n!}\\
		&= f(x).
\end{align*}
We found the derivative of $f(x)$ is $f(x)$. The only functions for which this is true are of the form $y=ce^x$ for some constant $c$. As $f(0) = 1$ (see \autoeqref{eq:ps4}), $c$ must be 1. Therefore we conclude that 
$$f(x) = \sum_{n=0}^\infty \frac{x^n}{n!} = e^x$$% \quad\text{for all $x$}.$$
for all $x$.

We can also find $\ds \int f(x)\ dx$:
\begin{align*}
\int f(x)\ dx &= C+\sum_{n=0}^\infty \frac{x^{n+1}}{n!(n+1)} \\
				&= C+ \sum_{n=0}^\infty \frac{x^{n+1}}{(n+1)!}
\end{align*}
We write out a few terms of this last series:
$$C+ \sum_{n=0}^\infty \frac{x^{n+1}}{(n+1)!} = C+ x+ \frac{x^2}2+\frac{x^3}{6}+\frac{x^4}{24}+\dotsb$$
The integral of $f(x)$ differs from $f(x)$ only by a constant, again indicating that $f(x) = e^x$.}


\autoref{ex_ps4} and the work following \autoref{ex_ps3} established relationships between a power series function and ``regular'' functions that we have dealt with in the past. In general, given a power series function, it is difficult (if not impossible) to express the function in terms of elementary functions. We chose examples where things worked out nicely.

%In this section's last example, we show how to solve a simple differential equation with a power series.
%
%\example{ex_ps5}{Solving a differential equation with a power series.}
%{Give the first 4 terms of the power series solution to $y\primeskip' = 2y$, where $y(0) = 1$.
%}
%{The differential equation $y\primeskip' = 2y$ describes a function $y=f(x)$ where the derivative of $y$ is twice $y$ and $y(0)=1$. This is a rather simple differential equation; with a bit of thought one should realize that if $y=Ce^{2x}$, then $y\primeskip' = 2Ce^{2x}$, and hence $y\primeskip' = 2y$. By letting $C=1$ we satisfy the initial condition of $y(0)=1$.
%
%Let's ignore the fact that we already know the solution and find a power series function that satisfies the equation. The solution we seek will have the form
%$$f(x)  = \sum_{n=0}^\infty a_nx^n = a_0+a_1x+a_2x^2+a_3x^3+\dotsb$$
%for unknown coefficients $a_n$. We can find $\fp(x)$ using \autoref{thm:calc_power_series}:
%$$\fp(x) = \sum_{n=1}^\infty a_n\cdot n\cdot x^{n-1} = a_1+2a_2x+3a_3x^2+4a_4x^3\dotsb.$$
%Since $\fp(x) = 2f(x)$, we have
%\begin{align*}
%a_1+2a_2x+3a_3x^2+4a_4x^3\dotsb &= 2\big(a_0+a_1x+a_2x^2+a_3x^3+\dotsb\big)\\
%			&=2a_0+2a_1x+2a_2x^2+2a_3x^3+\dotsb
%\end{align*}
%The coefficients of like powers of $x$ must be equal, so we find that
%$$a_1 = 2a_0,\quad 2a_2 = 2a_1,\quad 3a_3 = 2a_2,\quad 4a_4 = 2a_3,\quad \text{etc.}$$
%The initial condition $y(0) = f(0) = 1$ indicates that $a_0 = 1$; with this, we can find the values of the other coefficients:
%\begin{align*}
%a_0 = 1 \text{ and } a_1=2a_0 &\Rightarrow a_1 = 2;\\
%a_1 = 2 \text{ and } 2a_2 = 2a_1 &\Rightarrow a_2=4/2 =2;\\
%a_2=2 \text{ and } 3a_3 = 2a_2 &\Rightarrow a_3=8/(2\cdot3)=4/3;\\
%a_3=4/3 \text{ and } 4a_4 = 2a_3 &\Rightarrow a_4 =16/(2\cdot3\cdot4)= 2/3. 
%\end{align*}
%Thus the first 5 terms of the power series solution to the differential equation $y\primeskip'=2y$ is 
%$$f(x) = 1+ 2x+2x^2 + \frac43x^3+\frac23x^4+\dotsb$$
%In \autoref{sec:taylor_series}, as we study Taylor Series, we will learn how to recognize this series as describing $y=e^{2x}$.}

\subsection*{Representations of Functions with Power Series}

%Our last example illustrates that it
It can be difficult to recognize an elementary function by its power series expansion. It is far easier to start with a known function, expressed in terms of elementary functions, and represent it as a power series function. One may wonder why we would bother doing so, as the latter function probably seems more complicated.
%In the next two sections, we show both \emph{how} to do this and \emph{why} such a process can be beneficial.

%We opened the last section by saying that we were going to start thinking about applications of series and then promptly spent the section talking about convergence again.  It's now time to actually start with the applications of series. With this section  we will start talking about how to represent functions with power series.  The natural question of why we might want to do this will be answered later once we actually learn how to do this.

Let's start off with a series we already know how to do, although when we first ran across this series we didn't think of it as a power series nor did we acknowledge that it represented a function. Recall that the geometric series is
$$\sum_{n=0}^\infty ar^n = \frac a{1-r} \qquad \text{provided }\abs r<1.$$
We also know that if $\abs r\geq 1$ the series diverges. Now, if we take $a=1$ and $r=x$ this becomes,
\begin{equation}\label{eq_power_eq}
\sum_{n=0}^\infty x^n=\frac1{1-x}\qquad\text{provided }\abs x<1
\end{equation}

Turning this around we can see that we can represent the function
\begin{equation}\label{eq_power_func}
f(x) = \frac1{1-x}
\end{equation}
with the power series
\begin{equation}\label{eq_power_series}
\sum_{n=0}^{\infty}x^n \qquad \text{provided }\abs x<1.
\end{equation}

This provision is important.  We can clearly plug any number other than $x=1$ into the function, however, we will only get a convergent power series if $\abs x<1$.  This means the equality in \autoeqref{eq_power_eq} will only hold if $\abs x<1$.  For any other value of $x$ the equality won't hold.  Note as well that we can also use this to acknowledge that the radius of convergence of this power series is $R=1$ and the interval of convergence is $\abs x<1$.

This idea of convergence is important here.  We will be representing many functions as power series and it will be important to recognize that the representations will often only be valid for a range of $x$'s and that there may be values of $x$ that we can plug into the function that we can't plug into the power series representation.

In this section we are going to concentrate on representing functions with power series where the function can be related back to a geometric series. In this way we will hopefully become familiar with some of the kinds of manipulations that we will sometimes need when working with power series. We will see in \autoref{sec:taylor_series} that this strategy is useful for integrating functions that don't have elementary derivatives.

%In this section we are going to concentrate on representing functions with power series where the function can be related back to \autoeqref{eq_power_func}.  

%In this way we will hopefully become familiar with some of the kinds of manipulations that we will sometimes need when working with power series. We now consider several examples.

\example{ex_power_cube}{Finding a Power Series}{Find a power series representation for $g(x) = \dfrac1{1+x^3}$ and determine its interval of convergence.}{We want to relate this function back to \autoeqref{eq_power_func}.  This is actually easier than it might look.  Recall that the $x$ in \autoeqref{eq_power_func} is simply a variable and can represent anything.  So, a quick rewrite of $g(x)$ gives,
$$g(x)=\frac1{1-(-x^3)}$$
and so the $-x^3$ holds the same place as the $x$ in \autoeqref{eq_power_func}.  Therefore, all we need to do is replace the $x$ in \autoeqref{eq_power_series} and we've got a power series representation for $g(x)$.  
$$g(x)=\sum_{n=0}^{\infty}\left( -x^3 \right)^n \qquad \text{provided }\abs{-x^3}<1$$

Notice that we replaced both the $x$ in the power series and in the interval of convergence. All we need to do now is a little simplification.
$$g(x)=\sum_{n=0}^{\infty}\left( -1 \right)^n x^{3n} \qquad\text{provided }\abs x<1$$
So, in this case the interval of convergence is the same as the original power series.  This usually won't happen.  More often than not the new interval of convergence will be different from the original interval of convergence.}

\example{ex_power_numerator}{Finding a Power Series}{Find a power series representation for $h(x)=\dfrac{2x^2}{1+x^3}$ and determine its interval of convergence.}{This function is similar to the previous function, however the numerator is different.  Since \autoeqref{eq_power_func} doesn't have an $x$ in the numerator it appears that we can't relate this function back to that.  However, now that we've worked the first example this one is actually very simple since we can use the result of the answer from that example.  To see how to do this let's first rewrite the function a little. 
$$h(x)=2x^2\frac1{1+x^3}.$$
Now, from the first example we've already got a power series for the second term so let's use that to write the function as, 
$$h(x)=2x^2\sum_{n=0}^\infty\left( -1 \right)^n x^{3n} \qquad\text{provided }\abs x<1$$

Notice that the presence of $x$'s outside of the series will NOT affect its convergence and so the interval of convergence remains the same.  
The last step is to bring the coefficient into the series and we'll be done.  When we do this make sure and combine the $x$'s as well.  We typically only want a single $x$ in a power series.
$$h(x)=\sum_{n=0}^\infty 2\left( -1 \right)^n x^{3n+2} \qquad\text{provided }\abs x<1.$$
As we saw in the previous example we can often use previous results to help us out.  This is an important idea to remember as it can often greatly simplify our work.}

\example{ex_power_five}{Finding a Power Series}{Find a power series representation for $f(x)=\dfrac x{5-x}$ and determine its interval of convergence.}{So again, we have an $x$ in the numerator. As with the last example factor $x$ out and we have $f(x)=x\dfrac1{5-x}$. If we had a power series representation for $g(x)=\dfrac1{5-x}$ we could get a power series representation for $f(x)$. We need the number in the denominator to be a one so we rewrite the denominator.  
$$g(x)=\frac15\frac1{1-\frac x5}$$

Now all we need to do to get a power series representation is to replace the $x$ in \autoeqref{eq_power_series} with $\dfrac x5$.  Doing this gives
$$g(x)=\frac15\sum_{n=0}^\infty\left(\frac x5\right)^n\qquad\text{provided } \abs{\frac x5}<1.$$

Now simplify the series.
\begin{align*}
	g(x)
	& = \frac15\sum_{n=0}^\infty\frac{x^n}{5^n} \\
	& = \sum_{n=0}^\infty\frac{x^n}{5^{n+1}} 
\end{align*}

The interval of convergence for this series is
$$\abs{\frac x5}<1 \qquad \Rightarrow \qquad \frac15\abs x<1 \qquad \Rightarrow \qquad\abs x<5$$

We now have a power series representation for $g(x)$ but we need to find a power series representation for the original function.  All we need to do for this is to multiply the power series representative for $g(x)$ by $x$ and we'll have it.  
\begin{align*}
	f(x)
	&= x \frac1{5-x} \\
	&= x \sum_{n=0}^\infty\frac{x^n}{5^{n+1}}\\ 
	& = \sum_{n=0}^\infty\frac{x^{n+1}}{5^{n+1}}
\end{align*}
The interval of convergence doesn't change and so it will be $\abs x<5$.}

We now consider several examples where differentiation and integration of power series from \autoref{thm:calc_power_series} are used to write the power series for a function. % in \autoref{sec:power_series}.

\example{ex_power_deriv}{Differentiating a Power Series}{Find a power series representation for $g(x)=\dfrac1{(1-x)^2}$ and determine its radius of convergence.}{We know that
$$\frac1{(1-x)^2} = \frac{d}{dx} \left( \frac{1}{1-x} \right).$$
Since we have a power series  representation for $\dfrac1{1-x}$, we can differentiate that power series to get a power series representation for $g(x)$.  
\begin{align*}
	g(x)
	& = \frac{1}{1-x} \\
    & = \frac{d}{dx} \left( \frac{1}{1-x} \right) \\
	& =  \frac{d}{dx} \left( \sum_{n=0}^{\infty} x^n \right)\\
    & =  \sum_{n=1}^{\infty} nx^{n-1}
\end{align*}

Since the original power series had a radius of convergence of $R=1$ the derivative, and hence $g(x)$, will also have a radius of convergence of $R=1$.}

\example{ex_power_int}{Integrating a Power Series}{Find a power series representation for $h(x)=\ln(5-x)$ and determine its radius of convergence.}{In this case we need the fact that
$$\int\frac1{5-x}\ dx=-\ln(5-x).$$
Recall that we found a power series representation for $\dfrac1{5-x}$ in \autoref{ex_power_five}. We now have
\begin{align*}
	\ln(5-x)
	& =  -\ds\int \frac{1}{5-x}\ dx \\
    & =  -\ds\int \sum_{n=0}^{\infty} \frac{x^n}{5^{n+1}}\ dx \qquad\text{where $\abs x<5$}\\
	& =  C - \sum_{n=0}^{\infty} \frac{x^{n+1}}{(n+1)5^{n+1}} \qquad\text{where $\abs x<5$}
\end{align*}

We can find the constant of integration, $C$, by substituting in a value of $x$.  A good choice is $x=0$ as the series is usually easy to evaluate there.  
\begin{align*}
	\ln(5-0) & = C - \sum_{n=0}^{\infty} \frac{0^{n+1}}{(n+1)5^{n+1}} \\
	\ln (5-0) & =  C 
\end{align*}

So, the final answer is, 
$$\ln(5-x) =\ln(5) - \sum_{n=0}^{\infty} \frac{x^{n+1}}{(n+1)5^{n+1}},$$
and the radius of convergence is 5. Notice that $x=-5$ allows for convergence so the interval of convergence is $[-5,5)$.}

% todo write a transition paragraph

\printexercises{exercises/08_06_exercises}

\section{Taylor Polynomials}\label{sec:taylor_poly}

Consider a function $y=f(x)$ and a point $\bigl(c,f(c)\bigr)$. The derivative, $\fp(c)$, gives the instantaneous rate of change of $f$ at $x=c$. Of all lines that pass through the point $\bigl(c,f(c)\bigr)$, the line that best approximates $f$ at this point is the tangent line; that is, the line whose slope (rate of change) is $\fp(c)$.

\mtable[-\baselineskip]{Plotting $y=f(x)$ and a table of derivatives of $f$ evaluated at 0.}{fig:taypolyintroa}{\pdftooltip{\begin{tikzpicture}
\begin{axis}[width=1.16\marginparwidth,tick label style={font=\scriptsize},
axis y line=middle,axis x line=middle,name=myplot,axis on top,
ymin=-5.5,ymax=8.5,xmin=-4.2,xmax=4.4]
\addplot [draw={\colorone},domain=-4:4,smooth,thick,samples=50] {exp(x)*sin(deg(x))*cos(deg(x))+2};
\addplot [draw={\colortwo},domain=-4:4,thick] {x+2};
\draw (axis cs:-3.,3) node {\scriptsize $y=f(x)$};
\draw (axis cs:-2.75,-2.2) node {\scriptsize $y=p_1(x)$};
\end{axis}
\node [right] at (myplot.right of origin) {\scriptsize $x$};
\node [above] at (myplot.above origin) {\scriptsize $y$};
\end{tikzpicture}}{ALT-TEXT-TO-BE-DETERMINED}
\begin{align*}
f(0)&=2 & \fp''(0)&=-1\\
\fp(0)&=1 & f\,^{(4)}(0)&=-12 \\
\fpp(0)&=2 & f\,^{(5)}(0)&=-19\vspace{-.5\baselineskip}
\end{align*}}

In \autoref{fig:taypolyintroa}, we see a function $y=f(x)$ graphed. The table below the graph shows that $f(0)=2$ and $\fp(0) = 1$; therefore, the tangent line to $f$ at $x=0$ is $p_1(x) = 1(x-0)+2 = x+2$. The tangent line is also given in the figure. Note that ``near'' $x=0$, $p_1(x) \approx f(x)$; that is, the tangent line approximates $f$ well.

One shortcoming of this approximation is that the tangent line only matches the slope of $f$; it does not, for instance, match the concavity of $f$. We can find a polynomial, $p_2(x)$, that does match the concavity without much difficulty, though. The table in \autoref{fig:taypolyintroa} gives the following information:
\[f(0) = 2 \qquad \fp(0) = 1\qquad \fp'(0) = 2.\]
Therefore, we want our polynomial $p_2(x)$ to have these same properties. That is, we need
\[p_2(0) = 2 \qquad p_2'(0) = 1 \qquad p_2''(0) = 2.\]

This is simply an initial-value problem. We can solve this using the techniques first described in \autoref{sec:antider}. To keep $p_2(x)$ as simple as possible, we'll assume that not only  $p_2''(0)=2$, but that $p_2''(x)=2$. That is, the second derivative of $p_2$ is  constant.

\mtable{Plotting $f$, $p_2$, and $p_4$.}{fig:taypolyintrob}{\pdftooltip{\begin{tikzpicture}
\begin{axis}[width=1.16\marginparwidth,tick label style={font=\scriptsize},
axis y line=middle,axis x line=middle,name=myplot,axis on top,
ymin=-5.5,ymax=8.5,xmin=-4.2,xmax=4.4]
\addplot [draw={\colorone},domain=-4:4,smooth,thick,samples=50]
 {exp(x)*sin(deg(x))*cos(deg(x))+2};
\addplot [draw={\colortwo},domain=-4:4,thick,smooth] {x^2+x+2};
\addplot [draw={\colortwo!40},domain=-4:4,thick,smooth] {-x^4/2-x^3/6+x^2+x+2};
\draw (axis cs:-3.,3.5) node {\scriptsize $y=p_2(x)$};
\draw (axis cs:-3.1,-2.5) node {\scriptsize $y=p_4(x)$};
\end{axis}
\node [right] at (myplot.right of origin) {\scriptsize $x$};
\node [above] at (myplot.above origin) {\scriptsize $y$};
\end{tikzpicture}}{ALT-TEXT-TO-BE-DETERMINED}}

If $p_2''(x) = 2$, then $p_2'(x) = 2x+C$ for some constant $C$. Since we have determined that $p_2'(0) = 1$, we find that $C=1$ and so $p_2'(x) = 2x+1$. Finally, we can compute $p_2(x) = x^2+x+C$. Using our initial values, we know $p_2(0) = 2$ so $C=2.$ We conclude that $p_2(x) = x^2+x+2.$ This function is plotted with $f$ in \autoref{fig:taypolyintrob}.

We can repeat this approximation process by creating polynomials of higher degree that match more of the derivatives of $f$ at $x=0$. In general, a polynomial of degree $n$ can be created to match the first $n$ derivatives of $f$. \autoref{fig:taypolyintrob} also shows $p_4(x)= -x^4/2-x^3/6+x^2+x+2$, whose first four derivatives at 0 match those of $f$. (Using the table in \autoref{fig:taypolyintroa}, start with $p_4^{(4)}(x)=-12$ and solve the related initial-value problem.)

\mtable[2\baselineskip]{Plotting $f$ and $p_{13}$.}{fig:taypolyintroc}{\pdftooltip{\begin{tikzpicture}
\begin{axis}[width=1.16\marginparwidth,tick label style={font=\scriptsize},
axis y line=middle,axis x line=middle,name=myplot,axis on top,
ymin=-5.5,ymax=8.5,xmin=-4.2,xmax=4.4]
\addplot [draw={\colorone},domain=-4:4,smooth,thick,samples=50] {exp(x)*sin(deg(x))*cos(deg(x))+2};
\addplot [draw={\colortwo},domain=-4:4,thick,smooth] coordinates {(-3.52,-6.364)(-3.36,-2.334)(-3.2,-0.1706)(-3.04,0.9569)(-2.88,1.528)(-2.72,1.809)(-2.56,1.944)(-2.4,2.009)(-2.24,2.038)(-2.08,2.048)(-1.92,2.046)(-1.76,2.031)(-1.6,2.006)(-1.44,1.969)(-1.28,1.924)(-1.12,1.872)(-0.96,1.82)(-0.8,1.775)(-0.64,1.747)(-0.48,1.747)(-0.32,1.783)(-0.16,1.866)(0,2.)(0.16,2.185)(0.32,2.411)(0.48,2.662)(0.64,2.908)(0.8,3.112)(0.96,3.227)(1.12,3.202)(1.28,2.988)(1.44,2.546)(1.6,1.855)(1.76,0.9264)(1.92,-0.1926)(2.08,-1.406)(2.24,-2.565)(2.4,-3.474)(2.56,-3.891)(2.72,-3.542)(2.88,-2.133)(3.04,0.6461)(3.2,5.139)(3.36,11.79)};
% the degree 13 polynomial
\draw (axis cs:-2,-2.75) node {\scriptsize $y=p_{13}(x)$};
\end{axis}
\node [right] at (myplot.right of origin) {\scriptsize $x$};
\node [above] at (myplot.above origin) {\scriptsize $y$};
\end{tikzpicture}}{ALT-TEXT-TO-BE-DETERMINED}}

As we use more and more derivatives, our polynomial approximation to $f$ gets better and better. In this example, the interval on which the approximation is ``good'' gets bigger and bigger. \autoref{fig:taypolyintroc} shows $p_{13}(x)$; we can visually affirm that this polynomial approximates $f$ very well on $[-2,3]$. The polynomial $p_{13}(x)$ is fairly complicated:
\[\scriptstyle
\frac{16901x^{13}}{6227020800}+\frac{13x^{12}}{1209600}-\frac{1321x^{11}}{39916800}-\frac{779x^{10}}{1814400}-\frac{359x^9}{362880}+\frac{x^8}{240}+\frac{139x^7}{5040}+\frac{11 x^6}{360}-\frac{19x^5}{120}-\frac{x^4}{2}-\frac{x^3}{6}+x^2+x+2.
\]

The polynomials we have created are examples of \emph{Taylor polynomials}, named after the British mathematician Brook Taylor who made important discoveries about such functions. While we created the above Taylor polynomials by solving initial-value problems, it can be shown that Taylor polynomials follow a general pattern that makes their formation much more direct. This is described in the following definition.

{
\tcbset{grow to right by=10em}
\begin{definition}[Taylor Polynomial, Maclaurin Polynomial]\label{def:taypoly}
Let $f$ be a function whose first $n$ derivatives exist at $x=c$.
\index{Taylor Polynomial!definition}\index{Maclaurin Polynomial!definition} \index{Maclaurin Polynomial|seealso{Taylor Polynomial}}
\begin{enumerate}
	\item	The \textbf{Taylor polynomial of degree $n$ of $f$ at $x=c$} is 
	\begin{align*}
	p_n(x)
	&= f(c) + \fp(c)(x-c) + \frac{\fpp(c)}{2!}(x-c)^2+\frac{\fp''(c)}{3!}(x-c)^3+\dotsb+\frac{f\,^{(n)}(c)}{n!}(x-c)^n \\
	&=\sum_{k=0}^n\frac{f\,^{(k)}(c)}{k!}(x-c)^k.
	\end{align*}
	\item	A special case of the Taylor polynomial is the Maclaurin polynomial, where $c=0$. That is, the \textbf{Maclaurin polynomial of degree $n$ of $f$} is 
	\begin{align*}
	p_n(x)
	&= f(0) + \fp(0)x + \frac{\fpp(0)}{2!}x^2+\frac{\fp''(0)}{3!}x^3+\dotsb+\frac{f\,^{(n)}(0)}{n!}x^n \\
	&=\sum_{k=0}^n\frac{f\,^{(k)}(0)}{k!}x^k.
	\end{align*}
\end{enumerate}
\end{definition}
}

\mnote{\textbf{Note:} The summations in this definition use the convention that $x^0=1$ even when $x=0$ and that $f^{(0)}=f$.  They also use the definition that $0!=1$.}
Generally, we order the terms of a polynomial to have decreasing degrees, and that is how we began this section.  This definition, and the rest of this chapter, reverses this order to reflect the greater importance of the lower degree terms in the polynomials that we will be finding.

\youtubeVideo{UINFWG0ErSA}{Taylor Polynomial to Approximate a Function, Ex 3}

We will practice creating Taylor and Maclaurin polynomials in the following examples.

\begin{example}[Finding and using Maclaurin polynomials]\label{ex_taypoly1}
\mbox{}\\[-2\baselineskip]\parbox[t]{\linewidth}{%
\begin{enumerate}
	\item	Find the $n^\text{th}$ Maclaurin polynomial for $f(x) = e^x$.
	\item	Use $p_5(x)$ to approximate the value of $e$.
\end{enumerate}}\vspace{0pt}
\solution
\mtable{The derivatives of $f(x)=e^x$ evaluated at $x=0$.}{fig:taypoly1a}{%
\begin{align*}
% todo Tim latexml https://github.com/brucemiller/LaTeXML/issues/1763
f(x)&=e^x \quad\Rightarrow & f(0)&=1 \\
\fp(x)&=e^x \quad\Rightarrow & \fp(0)&=1 \\
\fpp(x)&=e^x \quad\Rightarrow & \fpp(0)&=1 \\
\vdots && \vdots & \\
f\,^{(n)}(x)&=e^x \quad\Rightarrow & f\,^{(n)}(0)&=1
%f(x)&=e^x & \Rightarrow && f(0)&=1 \\
%\fp(x)&=e^x & \Rightarrow && \fp(0)&=1 \\
%\fpp(x)&=e^x & \Rightarrow && \fpp(0)&=1 \\
%\vdots &&&& \vdots & \\
%f\,^{(n)}(x)&=e^x & \Rightarrow && f\,^{(n)}(0)&=1
\end{align*}}
\begin{enumerate}
\item We start with creating a table of the derivatives of $e^x$ evaluated at $x=0$. In this particular case, this is relatively simple, as shown in \autoref{fig:taypoly1a}. By the definition of the Maclaurin polynomial, we have 
\[
	p_n(x)
	=\sum_{k=0}^n\frac{f\,^{(k)}(0)}{k!}x^k
	=\sum_{k=0}^n\frac1{k!}x^k.
\]

\item	Using our answer from part 1, we have
\[p_5(x) = 1+x+\frac{1}{2}x^2+\frac{1}{6}x^3 + \frac{1}{24}x^4 + \frac{1}{120}x^5.\]
To approximate the value of $e$, note that $e = e^1 = f(1) \approx p_5(1).$ It is very straightforward to evaluate $p_5(1)$:
%
\mtable{A plot of $f(x)=e^x$ and its 5$^\text{th}$ degree Maclaurin polynomial $p_5(x)$.}{fig:taypoly1b}{\pdftooltip{\begin{tikzpicture}
\begin{axis}[width=1.16\marginparwidth,tick label style={font=\scriptsize},
axis y line=middle,axis x line=middle,name=myplot,axis on top,
ymin=-3,ymax=11,xmin=-3.75,xmax=2.9]
\addplot [draw={\colorone},domain=-3.5:2.5,smooth,thick,samples=50] {exp(x)};
\addplot [draw={\colortwo},domain=-4:4,smooth,thick] {1+x+x^2/2+x^3/6+x^4/24+x^5/120};
\draw (axis cs:-3.,-2) node {\scriptsize $y=p_5(x)$};
\end{axis}
\node [right] at (myplot.right of origin) {\scriptsize $x$};
\node [above] at (myplot.above origin) {\scriptsize $y$};
\end{tikzpicture}}{ALT-TEXT-TO-BE-DETERMINED}}
%
\[p_5(1) = 1+1+\frac12+\frac16+\frac1{24}+\frac1{120} = \frac{163}{60} \approx 2.71667.\]
This is an error of about $0.0016$, or $0.06\%$ of the true value.

A plot of $f(x)=e^x$ and $p_5(x)$ is given in \autoref{fig:taypoly1b}.
\end{enumerate}
\end{example}

\begin{example}[Finding and using Taylor polynomials]\label{ex_taypoly2}
\mbox{}\\[-2\baselineskip]\parbox[t]{\linewidth}{\begin{enumerate}
	\item	Find the $n^\text{th}$ Taylor polynomial of $y=\ln x$ at $x=1$.
	\item	Use $p_6(x)$ to approximate the value of $\ln 1.5$.
	\item	Use $p_6(x)$ to approximate the value of $\ln 2$. 
\end{enumerate}}\vspace{0pt}
\solution
\begin{enumerate}
\item	We begin by creating a table of derivatives of $\ln x$ evaluated at $x=1$. While this is not as straightforward as it was in the previous example, a pattern does emerge, as shown in \autoref{fig:taypoly2a}.
\mtable[-5\baselineskip]{Derivatives of $\ln x$ evaluated at $x=1$.}{fig:taypoly2a}{%
\begin{align*}
% todo Tim latexml https://github.com/brucemiller/LaTeXML/issues/1763
f(x) &= \makebox[3em]{$\ln x$} \quad \Rightarrow & f(1)&=\phantom-0\\
\fp(x) &= \makebox[3em]{$\phantom-1/x$} \quad \Rightarrow & \fp(1) &= \phantom-1\\
\fp'(x) &= \makebox[3em]{$-1/x^2$} \quad \Rightarrow & \fp'(1) &= -1\\
\fp''(x) &= \makebox[3em]{$\phantom-2/x^3$} \quad \Rightarrow & \fp''(1) &= \phantom-2\\
f\,^{(4)}(x) &= \makebox[3em]{$-6/x^4$} \quad \Rightarrow & f\,^{(4)}(1) &= -6\\
%f\,^{(5)}(x) &= \makebox[3em]{$24/x^5$} \quad\Rightarrow  &f\,^{(5)}(1) &= 24\\
\vdots && \vdots & \\
f\,^{(n)}(x) &= \makebox[3em]{}\quad \Rightarrow & f\,^{(n)}(1) &= \smallskip\\
\makebox[1pt]{$\dfrac{(-1)^{n+1}(n-1)!}{x^n}$} && \makebox[1pt]{$(-1)^{n+1}(n-1)!$} &
%f(x) &= \ln x & \Rightarrow && f(1)&=\phantom-0\\
%\fp(x) &= \phantom-1/x & \Rightarrow && \fp(1) &= \phantom-1\\
%\fp'(x) &= -1/x^2 & \Rightarrow && \fp'(1) &= -1\\
%\fp''(x) &= \phantom-2/x^3 & \Rightarrow && \fp''(1) &= \phantom-2\\
%f\,^{(4)}(x) &= -6/x^4 & \Rightarrow && f\,^{(4)}(1) &= -6\\
%\vdots &&&& \vdots & \\
%f\,^{(n)}(x) &= & \Rightarrow && f\,^{(n)}(1) &= \smallskip\\
%\makebox[1pt]{$\dfrac{(-1)^{n+1}(n-1)!}{x^n}$} &&&& \makebox[1pt]{$(-1)^{n+1}(n-1)!$} &
\end{align*}}

Using \autoref{def:taypoly}, we have
\[
	p_n(x)
	= \sum_{k=0}^n\frac{f\,^{(k)}(c)}{k!}(x-c)^k
	= \sum_{k=1}^n\frac{(-1)^{k+1}}k(x-1)^k.
\]

\item	We can compute $p_6(x)$ using our work above:
\[
p_6(x)
 = (x-1)-\frac12(x-1)^2+\frac13(x-1)^3-\frac14(x-1)^4+\frac15(x-1)^5-\frac16(x-1)^6.
\]
Since $p_6(x)$ approximates $\ln x$ well near $x=1$, we approximate $\ln 1.5 \approx p_6(1.5)$:
\begin{align*}
	p_6(1.5)
	&= (1.5-1)-\frac12(1.5-1)^2+\frac13(1.5-1)^3 \\
	&\qquad\qquad{}-\frac14(1.5-1)^4+\frac15(1.5-1)^5-\frac16(1.5-1)^6\\
	&=\frac{259}{640}\\
	&\approx 0.404688.
\end{align*}
%
\mtable{A plot of $y=\ln x$ and its 6$^\text{th}$ degree Taylor polynomial at $x=1$.}{fig:taypoly2b}{\pdftooltip{\begin{tikzpicture}
\begin{axis}[width=1.16\marginparwidth,tick label style={font=\scriptsize},
axis y line=middle,axis x line=middle,name=myplot,axis on top,
ymin=-4.5,ymax=2.4,xmin=-.5,xmax=3.2]
\addplot [draw={\colorone},domain=0.01:3,smooth,thick,samples=50] {ln(x)};
\addplot [draw={\colortwo},domain=-.5:3.2,smooth,thick]
 {(x-1)-(x-1)^2/2+(x-1)^3/3-(x-1)^4/4+(x-1)^5/5-(x-1)^6/6};
\draw (axis cs:2.5,1.65) node {\scriptsize $y=\ln x$};
\draw (axis cs:2.2,-2.75) node {\scriptsize $y=p_{6}(x)$};
\end{axis}
\node [right] at (myplot.right of origin) {\scriptsize $x$};
\node [above] at (myplot.above origin) {\scriptsize $y$};
\end{tikzpicture}}{ALT-TEXT-TO-BE-DETERMINED}}
%
This is a good approximation as a calculator shows that $\ln 1.5 \approx 0.4055.$  %This is an error of $0.0008$, or $0.2\%$. % this is in next example
\autoref{fig:taypoly2b} plots $y=\ln x$ with $y=p_6(x)$. We can see that $\ln 1.5\approx p_6(1.5)$.

\item	
We approximate $\ln 2$ with $ p_6(2)$:
\begin{align*}
p_6(2) &= (2-1)-\frac12(2-1)^2+\frac13(2-1)^3 \\
			&\qquad\qquad{}-\frac14(2-1)^4+\frac15(2-1)^5-\frac16(2-1)^6\\
			&=	1-\frac12+\frac13-\frac14+\frac15-\frac16 \\
			&= \frac{37}{60}\\ 
			&\approx 0.616667.
\end{align*}
This approximation is not terribly impressive: a hand held calculator shows that $\ln 2 \approx 0.693147.$
% This is an error of $0.08$, or $11\%$. % this is in next example
The graph in \autoref{fig:taypoly2b} shows that $p_6(x)$ provides less accurate approximations of $\ln x$ as $x$ gets close to 0 or 2. 

% todo use 20th degree polynomial instead of coordinates in Figure 9.9.8 fig:taypoly2c
\mtable[-.5in]{A plot of $y=\ln x$ and its 20$^\text{th}$ degree Taylor polynomial at $x=1$.}{fig:taypoly2c}{\pdftooltip{\begin{tikzpicture}
\begin{axis}[width=1.16\marginparwidth,tick label style={font=\scriptsize},
axis y line=middle,axis x line=middle,name=myplot,axis on top,
ymin=-4.5,ymax=2.4,xmin=-.5,xmax=3.2]
\addplot [draw={\colorone},domain=0.01:3,smooth,thick,samples=50] {ln(x)};
\addplot [draw={\colortwo},smooth,thick] coordinates {
 (-0.108,-7.567)(-0.052,-4.958)(0.004,-3.519)(0.06,-2.671)(0.116,-2.13)
 (0.172,-1.756)(0.228,-1.478)(0.284,-1.259)(0.34,-1.079)(0.396,-0.9263)
 (0.452,-0.7941)(0.508,-0.6773)(0.564,-0.5727)(0.62,-0.478)(0.676,-0.3916)
 (0.732,-0.312)(0.788,-0.2383)(0.844,-0.1696)(0.9,-0.1054)(0.956,-0.045)
 (1.012,0.01193)(1.068,0.06579)(1.124,0.1169)(1.18,0.1655)(1.236,0.2119)
 (1.292,0.2562)(1.348,0.2986)(1.404,0.3393)(1.46,0.3784)(1.516,0.4161)
 (1.572,0.4523)(1.628,0.4874)(1.684,0.5212)(1.74,0.5538)(1.796,0.5853)
 (1.852,0.6154)(1.908,0.6427)(1.964,0.6635)(2.02,0.6665)(2.076,0.621)
 (2.132,0.4476)(2.188,-0.04877)(2.244,-1.326)(2.3,-4.421)};
% the 20th degree Taylor approximation
\draw (axis cs:2.5,1.65) node {\scriptsize $y=\ln x$};
\draw (axis cs:1.7,-2.75) node {\scriptsize $y=p_{20}(x)$};
\end{axis}
\node [right] at (myplot.right of origin) {\scriptsize $x$};
\node [above] at (myplot.above origin) {\scriptsize $y$};
\end{tikzpicture}}{ALT-TEXT-TO-BE-DETERMINED}}

Surprisingly enough, even the 20$^\text{th}$ degree Taylor polynomial fails to approximate $\ln x$ for $x>2$, as shown in \autoref{fig:taypoly2c}. We'll soon discuss why this is.
\end{enumerate}
\end{example}

Taylor polynomials are used to approximate functions $f(x)$ in mainly two situations:
\begin{enumerate}
	\item	When $f(x)$ is known, but perhaps ``hard'' to compute directly. For instance, we can define $y=\cos x$ as either the ratio of sides of a right triangle (``adjacent over hypotenuse'') or with the unit circle. However, neither of these provides a convenient way of computing $\cos 2$. A Taylor polynomial of sufficiently high degree can provide a reasonable method of computing such values using only operations usually hard-wired into a computer ($+$, $-$, $\times$ and $\div$).
	
	\item	When $f(x)$ is not known, but information about its derivatives is known. This occurs more often than one might think, especially in the study of differential equations.
\end{enumerate}

\mnote{\textbf{Note:} Even though Taylor polynomials \emph{could} be used in calculators and computers to calculate values of trigonometric functions, in practice they generally aren't. Other more efficient and accurate methods have been developed, such as the CORDIC algorithm.}
	
In both situations, a critical piece of information to have is ``How good is my approximation?'' If we use a Taylor polynomial to compute $\cos 2$, how do we know how accurate the approximation is? 

We had the same problem with Numerical Integration. \autoref{thm:numerical_error} provided bounds on the error when using, say, Simpson's Rule to approximate a definite integral. These bounds allowed us to determine that, for example, using $10$ subintervals provided an approximation within $\pm .01$ of the exact value. The following theorem gives similar bounds for Taylor (and hence Maclaurin) polynomials.

{\tcbset{grow to right by=.5em}
\begin{theorem}[Taylor's Theorem]\label{thm:taylorthm}
\mbox{}\\[-2\baselineskip]\index{Taylor Polynomial!Taylor's Theorem}\index{Taylor's Theorem}
\begin{enumerate}
	\item	Let $f$ be a function whose $(n+1)^{\text{th}}$ derivative exists on an open interval $I$ and let $c$ be in $I$. Then, for each $x$ in $I$, there exists $z_x$ between $x$ and $c$ such that\vspace{-.3\baselineskip}
\[R_n(x) = f(x) - \sum_{k=0}^n\frac{f\,^{(k)}(c)}{k!}(x-c)^k = \frac{f\,^{(n+1)}(z_x)}{(n+1)!}(x-c)^{n+1}.\]
%\[f(x) = \sum_{k=0}^n\frac{f\,^{(k)}(c)}{k!}(x-c)^k+R_n(x),\]
%where $\ds R_n(x) = \frac{f\,^{(n+1)}(z_x)}{(n+1)!}(x-c)^{n+1}.$
	\item	$\abs{R_n(x)}\leq\dfrac{\max_z\abs{\,f\,^{(n+1)}(z)}}{(n+1)!}\abs{x-c}^{n+1}$, where $z$ is between $x$ and $c$.
\end{enumerate}
\end{theorem}}

The first part of Taylor's Theorem states that $f(x) = p_n(x) + R_n(x)$, where $p_n(x)$ is the $n^\text{th}$ order Taylor polynomial and $R_n(x)$ is the remainder, or error, in the Taylor approximation. The second part gives bounds on how big that error can be. If the $(n+1)^\text{th}$ derivative is large, the error may be large; if $x$ is far from $c$, the error may also be large. However, the $(n+1)!$ term in the denominator tends to ensure that the error gets smaller as $n$ increases.

The following example computes error estimates for the approximations of $\ln 1.5$ and $\ln 2$ made in \autoref{ex_taypoly2}.

\begin{example}[Finding error bounds of a Taylor polynomial]\label{ex_taypoly3}
Use \autoref{thm:taylorthm} to find error bounds when approximating $\ln 1.5$ and $\ln 2$ with $p_6(x)$, the Taylor polynomial of degree 6 of $f(x)=\ln x$ at $x=1$, as calculated in \autoref{ex_taypoly2}.
\solution
\begin{enumerate}
\item	We start with the approximation of $\ln 1.5$ with $p_6(1.5)$.
% Taylor's Theorem references an open interval $I$ that contains both $x$ and $c$. The smaller the interval we use the better; it will give us a more accurate (and smaller) approximation of the error. We let $I = (0.9,1.6)$, as this interval contains both $c=1$ and $x=1.5$. 
%
Taylor's Theorem references $\max\abs{f\,^{(n+1)}(z)}$. In our situation, this is asking ``How big can the $7^\text{th}$ derivative of $y=\ln x$ be on the interval $[1,1.5]$?'' The seventh derivative is $y = 6!/x^7$. The largest absolute value it attains on $I$ is 720. Thus we can bound the error as:
\begin{align*}
	\abs{R_6(1.5)}
	&\leq \frac{\max\abs{f\,^{(7)}(z)}}{7!}\abs{1.5-1}^7\\
	&\leq \frac{720}{5040}\cdot\frac1{2^7}\\
	&\approx 0.001.
\end{align*}
We computed $p_6(1.5) = 0.404688$; using a calculator, we find $\ln 1.5 \approx 0.405465$, so the actual error is about $0.000778$ (or $0.2\%$), which is less than our bound of $0.001$. This affirms Taylor's Theorem; the theorem states that our approximation would be within about one thousandth of the actual value, whereas the approximation was actually closer.

	\item	%We again find an interval $I$ that contains both $c=1$ and $x=2$; we choose $I = (0.9,2.1)$.
	The maximum value of the seventh derivative of $f$ on $[1,2]$ %this interval
	is again 720 (as the largest values come at $x=1$). Thus 
\begin{align*}
	\abs{R_6(2)}
	&\leq \frac{\max\abs{f\,^{(7)}(z)}}{7!}\abs{2-1}^7\\
	&\leq \frac{720}{5040}\cdot1^7\\
	&\approx0.15.
\end{align*}
This bound is not as nearly as good as before. Using the degree 6 Taylor polynomial at $x =1$ will bring us within 0.15 of the correct answer. As $p_6(2)\approx 0.61667$, our error estimate guarantees that the actual value of $\ln 2$ is somewhere between $0.46$ and $0.76$. These bounds are not particularly useful.

In reality, our approximation was only off by about $0.07$ (or $11\%$). However, we are approximating ostensibly because we do not know the real answer. In order to be assured that we have a good approximation, we would have to resort to using a polynomial of higher degree.
\end{enumerate}
\end{example}

We practice again. This time, we use Taylor's theorem to find $n$ that guarantees our approximation is within a certain amount.

\begin{example}[Finding sufficiently accurate Taylor polynomials]\label{ex_taypoly4}
Find $n$ such that the $n^\text{th}$ Taylor polynomial of $f(x)=\cos x$ at $x=0$ approximates $\cos 2$ to within $0.001$ of the actual answer. What is $p_n(2)$?
\solution
Following Taylor's theorem, we need bounds on the size of the derivatives of $f(x)=\cos x$. In the case of this trigonometric function, this is easy. All derivatives of cosine are $\pm \sin x$ or $\pm \cos x$. In all cases, these functions are never greater than 1 in absolute value. We want the error to be less than $0.001$. To find the appropriate $n$, consider the following inequalities:
\begin{align*}
\frac{\max\abs{f\,^{(n+1)}(z)}}{(n+1)!}\abs{2-0}^{n+1} &\leq 0.001 \\
\frac1{(n+1)!}\cdot2^{n+1} &\leq 0.001
\end{align*}
We find an $n$ that satisfies this last inequality with trial-and-error. When $n=8$, we have $\ds \frac{2^{8+1}}{(8+1)!} \approx 0.0014$; when $n=9$, we have $\ds \frac{2^{9+1}}{(9+1)!} \approx 0.000282 <0.001$. Thus we want to approximate $\cos 2$ with $p_9(2)$.\bigskip

\mtable[-2\baselineskip]{A table of the derivatives of $f(x)=\cos x$ evaluated at $x=0$.}{fig:taypoly4a}{\begin{align*}
f(x) &= \phantom-\cos x & \Rightarrow && f(0) &= \phantom-1\\
\fp(x) &= -\sin x & \Rightarrow && \fp(0) &= \phantom-0\\
\fp'(x) &= -\cos x & \Rightarrow && \fp'(0) &= -1\\
\fp''(x) &= \phantom-\sin x & \Rightarrow && \fp''(0) &= \phantom-0\\
f\,^{(4)}(x) &= \phantom-\cos x & \Rightarrow && f\,^{(4)}(0) &= \phantom-1\\
f\,^{(5)}(x) &= -\sin x & \Rightarrow && f\,^{(5)}(0) &= \phantom-0\\
f\,^{(6)}(x) &= -\cos x & \Rightarrow && f\,^{(6)}(0) &= -1\\
f\,^{(7)}(x) &= \phantom-\sin x & \Rightarrow && f\,^{(7)}(0) &= \phantom-0\\
f\,^{(8)}(x) &= \phantom-\cos x & \Rightarrow && f\,^{(8)}(0) &= \phantom-1\\
f\,^{(9)}(x) &= -\sin x & \Rightarrow && f\,^{(9)}(0) &= \phantom-0
\end{align*}}

We now set out to compute $p_9(x)$. We again need a table of the derivatives of $f(x)=\cos x$ evaluated at $x=0$. A table of these values is given in \autoref{fig:taypoly4a}. Notice how the derivatives, evaluated at $x=0$, follow a certain pattern. All the odd powers of $x$ in the Taylor polynomial will disappear as their coefficient is 0. While our error bounds state that we need $p_9(x)$, our work shows that this will be the same as $p_8(x)$.

Since we are forming our polynomial at $x=0$, we are creating a Maclaurin polynomial, and:\vspace{-.3\baselineskip}
\[
	p_8(x) = \sum_{k=0}^8\frac{f\,^{(k)}(0)}{k!}x^k
	=  1-\frac{1}{2!}x^2+\frac{1}{4!}x^4-\frac{1}{6!}x^6+\frac{1}{8!}x^8
\]

We finally approximate $\cos 2$:
\[\cos 2 \approx p_8(2) = -\frac{131}{315} \approx -0.41587.\]
%
% todo Tim unnest tikzpicture
\mtable{A graph of $f(x)= \cos x$ and its degree 8 Maclaurin polynomial.}{fig:taypoly4b}{\pdftooltip{\begin{tikzpicture}
\begin{axis}[width=1.16\marginparwidth,tick label style={font=\scriptsize},
axis y line=middle,axis x line=middle,name=myplot,axis on top,
xtick={-5,-4,-3,-2,-1,1,2,3,4,5},ymin=-1.1,ymax=1.5,xmin=-5.5,xmax=5.5]
\addplot [draw={\colorone},domain=-5:5,smooth,thick,samples=50] {cos(deg(x))};
\addplot [draw={\colortwo},domain=-4:4,smooth,thick] {1-x^2/2+x^4/24-x^6/720+x^8/40320};
\draw (axis cs:-3.,1) node {\scriptsize $y=p_8(x)$};
\end{axis}
\node [right] at (myplot.right of origin) {\scriptsize $x$};
\node [above] at (myplot.above origin) {\scriptsize $y$};
\node [below] at (myplot.below origin) {
 \begin{tikzpicture}
  \draw[thick,draw={\colorone}] (0,0)--(10pt,0)
   node [right,black] {\scriptsize $f(x)= \cos x$};
 \end{tikzpicture}};
\end{tikzpicture}}{ALT-TEXT-TO-BE-DETERMINED}}
%
Our error bound guarantees that this approximation is within $0.001$ of the correct answer. Technology shows us that our approximation is actually within about $0.0003$ (or $0.07\%$) of the correct answer.

\autoref{fig:taypoly4b} shows a graph of $y=p_8(x)$ and $y=\cos x$. Note how well the two functions agree on about $(-\pi,\pi)$.
\end{example}

\begin{example}[Finding and using Taylor polynomials]\label{ex_taypoly5}
\mbox{}\\[-2\baselineskip]\parbox[t]{\linewidth}{\begin{enumerate}
	\item	Find the degree 4 Taylor polynomial, $p_4(x)$, for $f(x)=\sqrt{x}$ at $x=4.$
	\item	Use $p_4(x)$ to approximate $\sqrt{3}$.
	\item	Find bounds on the error when approximating $\sqrt{3}$ with $p_4(3)$.
\end{enumerate}}\vspace{0pt}
\solution
\begin{enumerate}
	\item	We begin by evaluating the derivatives of $f$ at $x=4$. This is done in \autoref{fig:taypoly5a}. These values allow us to form the Taylor polynomial $p_4(x)$:
	%
\mtable[-2\baselineskip]{A table of the derivatives of $f(x)=\sqrt{x}$ evaluated at $x=4$.}{fig:taypoly5a}{%
\begin{align*}
f(x) &= \sqrt{x} & \Rightarrow && f(4) &= 2\\
\ds\fp(x) &= \frac{1}{2\sqrt{x}} & \Rightarrow && \ds\fp(4) &= \frac{1}{4}\\
\ds\fp'(x) &= \frac{-1}{4x^{3/2}} & \Rightarrow && \ds\fp'(4) &= \frac{-1}{32}\\
\ds\fp''(x) &= \frac3{8x^{5/2}} & \Rightarrow && \ds\fp''(4) &= \frac{3}{256}\\
\ds f\,^{(4)}(x) &= \frac{-15}{16x^{7/2}} & \Rightarrow
&& \ds f\,^{(4)}(4) &= \frac{-15}{2048}\vspace{-.5\baselineskip}
\end{align*}}
%
\begin{multline*}
p_4(x) = \\
2 + \frac14(x-4) +\frac{-1/32}{2!}(x-4)^2+\frac{3/256}{3!}(x-4)^3+\frac{-15/2048}{4!}(x-4)^4.
\end{multline*}

	\item	As $p_4(x) \approx \sqrt{x}$ near $x=4$, we approximate $\sqrt{3}$ with $p_4(3) = 1.73212$.

	\item	%To find a bound on the error, we need an open interval that contains $x=3$ and $x=4$. We set $I = (2.9,4.1)$.
	The largest value the fifth derivative of $f(x)=\sqrt{x}$ takes on $[3,4]$ is when $x=3$, at about $0.0234$. Thus
\[\abs{R_4(3)} \leq \frac{0.0234}{5!}\abs{3-4}^5 \approx 0.00019.\]
%
\mtable[\baselineskip]{A graph of $f(x)=\sqrt{x}$ and its degree 4 Taylor polynomial at $x=4$.}{fig:taypoly5b}{\pdftooltip{\begin{tikzpicture}
\begin{axis}[width=1.16\marginparwidth,tick label style={font=\scriptsize},
axis y line=middle,axis x line=middle,name=myplot,axis on top,
ymin=-.1,ymax=3.5,xmin=-1,xmax=11]
\addplot [draw={\colorone},domain=0:10,smooth,thick,samples=50] {sqrt(x)};
\addplot [draw={\colortwo},smooth,thick,domain=0:10]
 {2+(x-4)/4-(x-4)^2/64+3*(x-4)^3/1536-15*(x-4)^4/49152};
\draw (axis cs:8.,1) node {\begin{tikzpicture}
  \draw[thick,draw={\colorone}](0,0)--(10pt,0)node[right,black]{\scriptsize$y=\sqrt x$};
  \draw[draw={\colortwo},thick](0,-10pt)--(10pt,-10pt)
   node[right,black]{\scriptsize$y=p_4(x)$};
 \end{tikzpicture}};
\end{axis}
\node [right] at (myplot.right of origin) {\scriptsize $x$};
\node [above] at (myplot.above origin) {\scriptsize $y$};
\end{tikzpicture}}{ALT-TEXT-TO-BE-DETERMINED}}%
%
This shows our approximation is accurate to at least the first 2 places after the decimal. It turns out that our approximation has an error of $0.00007$, or $0.004\%$. A graph of $f(x)=\sqrt x$ and $p_4(x)$ is given in \autoref{fig:taypoly5b}. Note how the two functions are nearly indistinguishable on $(2,7)$.
\end{enumerate}
\end{example}

%Our final example gives a brief introduction to using Taylor polynomials to solve differential equations.
%
%\begin{example}[Approximating an unknown function]\label{ex_taypoly6}
%A function $y=f(x)$ is unknown save for the following two facts.
%\begin{enumerate}
%\item		$y(0) = f(0) = 1$, and
%\item		$y\primeskip'= y^2$
%\end{enumerate}
%(This second fact says that amazingly, the derivative of the function is actually the function squared!)
%
%Find the degree 3 Maclaurin polynomial $p_3(x)$ of $y=f(x)$.
%\solution
%One might initially think that not enough information is given to find $p_3(x)$. However, note how the second fact above actually lets us know what $y\primeskip'(0)$ is:
%\[y\primeskip' = y^2 \Rightarrow y\primeskip'(0) = y^2(0).\]
%Since $y(0) = 1$, we conclude that $y\primeskip'(0) = 1$.
%
%Now we find information about $y\primeskip''$. Starting with $y\primeskip'=y^2$, take derivatives of both sides, \emph{with respect to $x$}. That means we must use implicit differentiation.
%\begin{align*}
%y\primeskip' &= y^2\\
%\frac{\dd}{\dd x}\bigl(y\primeskip'\bigr) &= \frac{\dd}{\dd x}\bigl(y^2\bigr)\\
%y\primeskip'' &= 2y\cdot y\primeskip'.
%\intertext{Now evaluate both sides at $x=0$:}
%y\primeskip''(0) &= 2y(0)\cdot y\primeskip'(0)\\
%y\primeskip''(0) &= 2
%\end{align*}
%We repeat this once more to find $y\primeskip'''(0)$. We again use implicit differentiation; this time the Product Rule is also required.
%\begin{align*}
%\frac{\dd}{\dd x}\bigl(y\primeskip''\bigr) &= \frac{\dd}{\dd x} \bigl(2yy\primeskip'\bigr)\\
%y\primeskip''' &= 2y\primeskip'\cdot y\primeskip' + 2y\cdot y\primeskip''.
%\intertext{Now evaluate both sides at $x=0$:}
%y\primeskip'''(0) &= 2y\primeskip'(0)^2 + 2y(0)y\primeskip''(0)\\
%y\primeskip'''(0) &=	2+4=6
%\end{align*}
%In summary, we have:
%\[y(0) = 1 \qquad y\primeskip'(0) = 1  \qquad y\primeskip''(0) = 2 \qquad y\primeskip'''(0) = 6.\]
%We can now form $p_3(x)$:
%\begin{align*}
%p_3(x) &= 1 + x + \frac{2}{2!}x^2 + \frac{6}{3!}x^3\\
%				&= 1+x+x^2+x^3.
%\end{align*}
%\mtable{A graph of $y=-1/(x-1)$ and $y=p_3(x)$ from \autoref{ex_taypoly6}.}{fig:taypoly6}{\pdftooltip{\begin{tikzpicture}
%\begin{axis}[width=1.16\marginparwidth,tick label style={font=\scriptsize},
%axis y line=middle,axis x line=middle,name=myplot,axis on top,
%ymin=-.1,ymax=3.5,xmin=-1.1,xmax=1.1]
%\addplot [draw={\colorone},domain=-1:.7,smooth,thick,samples=50] {-1/(x-1)};
%\addplot [draw={\colortwo},smooth,thick] coordinates {(-1.,0)(-0.96,0.07686)(-0.92,0.1477)(-0.88,0.2129)(-0.84,0.2729)(-0.8,
%0.328)(-0.76,0.3786)(-0.72,0.4252)(-0.68,0.468)(-0.64,0.5075)(-0.6,0.
%544)(-0.56,0.578)(-0.52,0.6098)(-0.48,0.6398)(-0.44,0.6684)(-0.4,0.
%696)(-0.36,0.7229)(-0.32,0.7496)(-0.28,0.7764)(-0.24,0.8038)(-0.2,0.
%832)(-0.16,0.8615)(-0.12,0.8927)(-0.08,0.9259)(-0.04,0.9615)(0,1.)(0.
%04,1.042)(0.08,1.087)(0.12,1.136)(0.16,1.19)(0.2,1.248)(0.24,1.311)(0.
%28,1.38)(0.32,1.455)(0.36,1.536)(0.4,1.624)(0.44,1.719)(0.48,1.821)(0.
%52,1.931)(0.56,2.049)(0.6,2.176)(0.64,2.312)(0.68,2.457)(0.72,2.612)(
%0.76,2.777)(0.8,2.952)(0.84,3.138)(0.88,3.336)(0.92,3.545)(0.96,3.766)(1.,4.)};
%\draw (axis cs:.75,.9) node {\begin{tikzpicture}
%	\draw[thick,draw={\colorone}] (0,0)--(10pt,0) node [right,black] {\scriptsize $\displaystyle y= \frac{1}{1-x}$};
%	\draw[draw={\colortwo},thick] (0,-15pt)--(10pt,-15pt) node [right,black] {\scriptsize $y= p_3(x)$};
%\end{tikzpicture}};
%\end{axis}
%\node [right] at (myplot.right of origin) {\scriptsize $x$};
%\node [above] at (myplot.above origin) {\scriptsize $y$};
%\end{tikzpicture}}{ALT-TEXT-TO-BE-DETERMINED}}
%It turns out that the differential equation we started with, $y\primeskip'=y^2$, where $y(0)=1$, can be solved without too much difficulty: $\ds y = \frac{1}{1-x}$. \autoref{fig:taypoly6} shows this function plotted with $p_3(x)$. Note how similar they are near $x=0$.
%\end{example}
%
%It is beyond the scope of this text to pursue error analysis when using Taylor polynomials to approximate solutions to differential equations. This topic is often broached in introductory Differential Equations courses and usually covered in depth in Numerical Analysis courses. Such an analysis is very important; one needs to know how good their approximation is. We explored this example simply to demonstrate the usefulness of Taylor polynomials.

\bigskip

Most of this chapter has been devoted to the study of infinite series. This section has stepped aside from this study, focusing instead on finite summation of terms. In the next section, we will combine power series and Taylor polynomials into \textbf{Taylor Series}, where we represent a function with an infinite series.

\printexercises{exercises/08-07-exercises}

% todo for exercises 21-24, reach back to lin approx / Newton’s method and redo the appropriate problems with Taylor series

\section{Taylor Series}\label{sec:taylor_series}

In \autoref{sec:power_series}, we showed how certain functions can be represented by a power series function. In \autoref{sec:taylor_poly}, we showed how we can approximate functions with polynomials, given that enough derivative information is available. In this section we combine these concepts: if a function $f(x)$ is infinitely differentiable, we show how to represent it with a power series function.

\begin{definition}[Taylor and Maclaurin Series]\label{def:taylor_series}
Let $f(x)$ have derivatives of all orders at $x=c$.
\index{Taylor Series!definition}\index{Maclaurin Series!definition}\index{Maclaurin Series|seealso{Taylor Series}}\index{series!Taylor}\index{series!Maclaurin}
\begin{enumerate}
	\item The \textbf{Taylor Series of $f(x)$, centered at $c$} is
	\[\sum_{n=0}^\infty \frac{f\,^{(n)}(c)}{n!}(x-c)^n.\]
	\item	Setting $c=0$ gives the \textbf{Maclaurin Series of $f(x)$}:
	\[\sum_{n=0}^\infty \frac{f\,^{(n)}(0)}{n!}x^n.\]
\end{enumerate}
\end{definition}

\youtubeVideo{Os8OtXFBLkY}{Taylor and Maclaurin Series --- Example 2}

The difference between a Taylor polynomial and a Taylor series is the former is a polynomial, containing only a finite number of terms, whereas the latter is a series, a summation of an infinite set of terms. When creating the Taylor polynomial of degree $n$ for a function $f(x)$ at $x=c$, we needed to evaluate $f$, and the first $n$ derivatives of $f$, at $x=c$. When creating the Taylor series of $f$, we need to find a pattern that describes the $n^\text{th}$ derivative of $f$ at $x=c$. We demonstrate this in the next two examples.

\begin{example}[The Maclaurin series of $f(x) = \cos x$]\label{ex_ts1}
Find the Maclaurin series of $f(x)=\cos x$.
\solution
In \autoref{ex_taypoly4} we found the $8^\text{th}$ degree Maclaurin polynomial of $\cos x$. In doing so, we created the table shown in \autoref{fig:ts1}.
%
\mtable{A table of the derivatives of $f(x)=\cos x$ evaluated at $x=0$.}{fig:ts1}{%
\begin{align*}
f(x) &= \phantom-\cos x &\Rightarrow & &f(0) &= \phantom-1\\
\fp(x) &= -\sin x &\Rightarrow & &\fp(0) &= \phantom-0\\
\fp'(x) &= -\cos x &\Rightarrow & &\fp'(0) &= -1\\
\fp''(x) &= \phantom-\sin x &\Rightarrow & &\fp''(0) &= \phantom-0\\
f\,^{(4)}(x) &= \phantom-\cos x &\Rightarrow & &f\,^{(4)}(0) &= \phantom-1\\
f\,^{(5)}(x) &= -\sin x &\Rightarrow & &f\,^{(5)}(0) &= \phantom-0\\
f\,^{(6)}(x) &= -\cos x &\Rightarrow & &f\,^{(6)}(0) &= -1\\
f\,^{(7)}(x) &= \phantom-\sin x &\Rightarrow & &f\,^{(7)}(0) &= \phantom-0\\
f\,^{(8)}(x) &= \phantom-\cos x &\Rightarrow & &f\,^{(8)}(0) &= \phantom-1\\
f\,^{(9)}(x) &= -\sin x &\Rightarrow & &f\,^{(9)}(0) &= \phantom-0
\end{align*}}
%
Notice how $f\,^{(n)}(0)=0$ when $n$ is odd,  $f\,^{(n)}(0)=1$ when $n$ is divisible by $4$, and $f\,^{(n)}(0)=-1$ when $n$ is even but not divisible by 4. Thus the Maclaurin series of $\cos x$ is
\[1-\frac{x^2}2+\frac{x^4}{4!}-\frac{x^6}{6!}+\frac{x^8}{8!} - \dotsb\]
We can go further and write this as a summation. Since we only need the terms where the power of $x$ is even, we write the power series in terms of $x^{2n}$:
\[\sum_{n=0}^\infty (-1)^{n}\frac{x^{2n}}{(2n)!}.\]
\end{example}

\begin{example}[The Taylor series of $f(x)=\ln x$ at $x=1$]\label{ex_ts2}
Find the Taylor series of $f(x) = \ln x$ centered at $x=1$.
\solution
\autoref{fig:ts2} shows the $n^\text{th}$ derivative of $\ln x$ evaluated at $x=1$ for $n=0,\dotsc,5$, along with an expression for the $n^\text{th}$ term:
\[f\,^{(n)}(1) = (-1)^{n+1}(n-1)!\quad \text{for $n\geq 1$.}\]
Remember that this is what distinguishes Taylor series from Taylor polynomials; we are very interested in finding a pattern for the $n^\text{th}$ term, not just finding a finite set of coefficients for a polynomial.
\mtable{Derivatives of $\ln x$ evaluated at $x=1$.}{fig:ts2}{%
\begin{align*}
% todo Tim latexml https://github.com/brucemiller/LaTeXML/issues/1763
f(x) &= \makebox[3em]{$\ln x$} \quad \Rightarrow & f(1)&=\phantom-0\\
\fp(x) &= \makebox[3em]{$\phantom-1/x$} \quad \Rightarrow & \fp(1) &= \phantom-1\\
\fp'(x) &= \makebox[3em]{$-1/x^2$} \quad \Rightarrow & \fp'(1) &= -1\\
\fp''(x) &= \makebox[3em]{$\phantom-2/x^3$} \quad \Rightarrow & \fp''(1) &= \phantom-2\\
f\,^{(4)}(x) &= \makebox[3em]{$-6/x^4$} \quad \Rightarrow & f\,^{(4)}(1) &= -6\\
f\,^{(5)}(x) &= \makebox[3em]{$24/x^5$} \quad\Rightarrow  &f\,^{(5)}(1) &= 24\\
\vdots && \vdots & \\
f\,^{(n)}(x) &= \makebox[3em]{}\quad \Rightarrow & f\,^{(n)}(1) &= \smallskip\\
\makebox[1pt]{$\dfrac{(-1)^{n+1}(n-1)!}{x^n}$} && \makebox[1pt]{$(-1)^{n+1}(n-1)!$} &
%f(x) &= \ln x &\Rightarrow & &f(1) &= \phantom-0\\
%\fp(x) &= \phantom-1/x &\Rightarrow & &\fp(1) &= \phantom-1\\
%\fp'(x) &= -1/x^2 &\Rightarrow & &\fp'(1) &= -1\\
%\fp''(x) &= \phantom-2/x^3 &\Rightarrow & &\fp''(1) &= \phantom-2\\
%f\,^{(4)}(x) &= -6/x^4 &\Rightarrow & &f\,^{(4)}(1) &= -6\\
%f\,^{(5)}(x) &= 24/x^5 &\Rightarrow & &f\,^{(5)}(1) &= 24\\
%\ \vdots & & & & \ \vdots & \\
%f\,^{(n)}(x) &=  &\Rightarrow & & f\,^{(n)}(1) &= \smallskip\\
%\makebox[1pt]{$\dfrac{(-1)^{n+1}(n-1)!}{x^n}$} &&&& \makebox[1pt]{$(-1)^{n+1}(n-1)!$} &
\end{align*}}
Since $f(1) = \ln 1 = 0$, we skip the first term and start the summation with $n=1$, giving the Taylor series for $\ln x$, centered at $x=1$, as 
\[\sum_{n=1}^\infty (-1)^{n+1}(n-1)!\frac{1}{n!}(x-1)^n = \sum_{n=1}^\infty (-1)^{n+1}\frac{(x-1)^n}{n}.\]
\end{example}

It is important to note that \autoref{def:taylor_series} defines a Taylor series given a function $f(x)$; however, we \emph{cannot} yet state that $f(x)$ \emph{is equal} to its Taylor series. We will find that ``most of the time'' they are equal, but we need to consider the conditions that allow us to conclude this.

\autoref{thm:taylorthm} states that the error between a function and its $n^\text{th}$-degree Taylor polynomial is $R_n(x)$, where
\[\abs{R_n(x)}\leq \frac{\max\abs{\,f\,^{(n+1)}(z)}}{(n+1)!}\abs{x-c}^{n+1}.\]

If $R_n(x)$ goes to 0 for each $x$ in an interval $I$ as $n$ approaches infinity, we conclude that the function is equal to its Taylor series expansion.

\begin{theorem}[Function and Taylor Series Equality]\label{thm:function_series_equality}
Let $f(x)$ have derivatives of all orders at $x=c$, let $R_n(x)$ be as stated in \autoref{thm:taylorthm}, and let $I$ be an interval on which the Taylor series of $f(x)$ converges. 
If $\ds\lim_{n\to\infty} R_n(x) = 0$ for all $x$ in $I$, then 
\index{Taylor Series!equality with generating function}
\[f(x) = \sum_{n=0}^\infty \frac{f\,^{(n)}(c)}{n!}(x-c)^n\ \text{ on $I$.}\]
\end{theorem}

We demonstrate the use of this theorem in an example.

\begin{example}[Establishing equality of a function and its Taylor series]\label{ex_ts3}
Show that, for all $x$, $f(x) = \cos x$ is equal to its Maclaurin series as found in \autoref{ex_ts1}.
\solution
Given a value $x$, the magnitude of the error term $R_n(x)$ is bounded by
\[\abs{R_n(x)}\leq \frac{\max\abs{\,f\,^{(n+1)}(z)}}{(n+1)!}\abs{x}^{n+1}.\]
Since all derivatives of $\cos x$ are $\pm \sin x$ or $\pm\cos x$, whose magnitudes are bounded by $1$, we can state
\[\abs{R_n(x)}\leq \frac{1}{(n+1)!}\abs{x}^{n+1}\]
which implies
\begin{equation*}
 -\frac{\abs{x}^{n+1}}{(n+1)!} \leq R_n(x) \leq\frac{\abs{x}^{n+1}}{(n+1)!}.%\label{eq:coseqtaylor}
\end{equation*}
For any $x$, $\ds\lim_{n\to\infty} \frac{x^{n+1}}{(n+1)!} = 0$. Applying the Squeeze Theorem to our last inequality%\autoeqref{eq:coseqtaylor}
, we conclude that $\ds \lim_{n\to\infty} R_n(x) = 0$ for all $x$, and hence
\[\cos x = \sum_{n=0}^\infty (-1)^{n}\frac{x^{2n}}{(2n)!}\quad \text{for all $x$}.\]
\end{example}

It is natural to assume that a function is  equal to its Taylor series on the series' interval of convergence, but this is not necessarily the case. In order to properly establish equality, one must use \autoref{thm:function_series_equality}. This is a bit disappointing, as we developed beautiful techniques for determining the interval of convergence of a power series, and proving that $R_n(x)\to 0$ can be cumbersome as it deals with high order derivatives of the function.

There is good news. A function $f(x)$ that is equal to its Taylor series, centered at any point the domain of $f(x)$, is said to be an \textbf{analytic function},\index{analytic function} and most, if not all, functions that we encounter within this course are analytic functions. Generally speaking, any function that one creates with elementary functions (polynomials, exponentials, trigonometric functions, etc.) that is not piecewise defined is probably analytic. For most functions, we assume the function is equal to its Taylor series on the series' interval of convergence and only use \autoref{thm:function_series_equality} when we suspect something may not work as expected.  The converse is also true: if a function is equal to \emph{some} power series on an interval, then that power series is the Taylor series of the function.

We develop the Taylor series for one more important function, then give a table of the Taylor series for a number of common functions.\index{Binomial Series}\index{series!Binomial}

\begin{example}[The Binomial Series]\label{ex_ts4}
Find the Maclaurin series of $f(x) = (1+x)^k$, $k\neq 0$.
\solution
When $k$ is a positive integer, the Maclaurin series is finite. For instance, when $k=4$, we have 
\[f(x) = (1+x)^4 = 1+4x+6x^2+4x^3+x^4.\]
The coefficients of $x$ when $k$ is a positive integer are known as the \emph{binomial coefficients}, giving the series we are developing its name.

When $k=1/2$, we have $f(x) = \sqrt{1+x}$. Knowing a series representation of this function would give a useful way of approximating $\sqrt{1.3}$, for instance.

To develop the Maclaurin series for $f(x) = (1+x)^k$ for any value of $k\neq0$, we consider the derivatives of $f$ evaluated at $x=0$:
{%
\small%
\begin{align*}
f(x) &= (1+x)^k & f(0) &= 1\\
\fp(x) &= k(1+x)^{k-1} & \fp(0) &=k\\
\fp'(x) &= k(k-1)(1+x)^{k-2} & \fp'(0) &=k(k-1)\\
\fp''(x) &= k(k-1)(k-2)(1+x)^{k-3} & \fp''(0) &=k(k-1)(k-2)\\
&\vdots & &\vdots\\
f\,^{(n)}(x) &= k(k-1)\dotsm\bigl(k-(n-1)\bigr)(1+x)^{k-n}
\mkern-3\thickmuskip&&\vdots \\
&& f\,^{(n)}(0) &=k(k-1)\dotsm\bigl(k-(n-1)\bigr)
\end{align*}}%
Thus the Maclaurin series for $f(x) = (1+x)^k$ is
\[
1+ kx + \frac{k(k-1)}{2!}x^2 + \frac{k(k-1)(k-2)}{3!}x^3 + \dotsb + \frac{k(k-1)\dotsm\bigl(k-(n-1)\bigr)}{n!}x^n+\dotsb
\]

It is important to determine the interval of convergence of this series. With 
\[a_n = \frac{k(k-1)\dotsm\bigl(k-(n-1)\bigr)}{n!}x^n,\]
we apply the Ratio Test:
\begin{align*}
	\lim_{n\to\infty}\frac{\abs{a_{n+1}}}{\abs{a_n}}
	&=\lim_{n\to\infty}\frac{\abs{\frac{k(k-1)\dotsm(k-n)}{(n+1)!}x^{n+1}}}{\abs{\frac{k(k-1)\dotsm\bigl(k-(n-1)\bigr)}{n!}x^n}}\\
		&=\lim_{n\to\infty} \abs{\frac{k-n}{n+1}x}\\
		&= \abs x.
\end{align*}

The series converges absolutely when the limit of the Ratio Test is less than 1; therefore, we have absolute convergence when $\abs x<1$. 

While outside the scope of this text, the interval of convergence depends on the value of $k$. When $k>0$, the interval of convergence is $[-1,1]$. When $-1<k<0$, the interval of convergence is $(-1,1]$. If $k\leq -1$, the interval of convergence is $(-1,1)$.
%When $x=1$, we can apply the Alternating Series Test and find the series converges. When $x=-1$, it can be shown (with some difficulty) that the series also converges. Therefore the interval of convergence is $[-1,1]$. We can apply \autoref{thm:function_series_equality} to prove equality between $f(x)$ and the series (or apply the discussion following the theorem). 
\end{example}

We learned that Taylor polynomials offer a way of approximating a ``difficult to compute'' function with a polynomial. Taylor series offer a way of exactly representing a function with a series. One probably can see the use of a good approximation; is there any use of representing a function exactly as a series?

While we appreciate the mathematical beauty of Taylor series (which is reason enough to study them), there are practical uses as well. They provide a valuable tool for solving a variety of problems, including problems relating to integration and differential equations.

In \autoref{idea:common_taylor} (on the following page) we give  a table of the Maclaurin series of a number of common functions. We then give a theorem about the ``algebra of power series,'' that is, how we can combine power series to create power series of new functions. This allows us to find the Taylor series of functions like $f(x) = e^x\cos x$ by knowing the Taylor series of $e^x$ and $\cos x$.

Before we investigate combining functions, consider the Taylor series for the arctangent function (see \autoref{idea:common_taylor}). Knowing that $\tan^{-1}(1) = \pi/4$, we can use this series to approximate the value of $\pi$:
\begin{align*}
\frac{\pi}4 &= \tan^{-1}(1) = 1-\frac13+\frac15-\frac17+\frac19-\dotsb\\
\pi &= 4\left(1-\frac13+\frac15-\frac17+\frac19-\dotsb\right)
\end{align*} 
Unfortunately, this particular expansion of $\pi$ converges very slowly. The first 
100 terms approximate $\pi$ as $3.13159$, which is not particularly good.

{
\tcbset{grow to right by=12em}
\begin{keyidea}[Important Maclaurin Series Expansions]\label{idea:common_taylor}
%
\noindent\begin{tabular}{llc}
\textbf{Function and Series} & \textbf{First Few Terms} & \parbox{55pt}{\centering\textbf{Interval of}\\\textbf{Convergence}} \\
$\ds e^x = \sum_{n=0}^\infty \frac{x^n}{n!}$ & $\ds 1+ x+\frac{x^2}{2!} + \frac{x^3}{3!}+\dotsb$ & $(-\infty,\infty)$\medskip\\ % Ex {ex_taypoly1} (ish) 9.9.1
$\ds \sin x = \sum_{n=0}^\infty (-1)^n\frac{x^{2n+1}}{(2n+1)!}$ & $\ds x-\frac{x^3}{3!}+\frac{x^5}{5!} - \frac{x^7}{7!}+\dotsb$ & $(-\infty,\infty)$\medskip\\
$\ds \cos x = \sum_{n=0}^\infty (-1)^n\frac{x^{2n}}{(2n)!}$ & $\ds 1-\frac{x^2}{2!}+\frac{x^4}{4!} - \frac{x^6}{6!} +\dotsb$ & $(-\infty,\infty)$\smallskip\\ % ex {ex_ts1} 9.10.1
$\ds \ln(x+1) = \sum_{n=1}^\infty(-1)^{n+1}\frac{x^n}{n}$ & $\ds x-\frac{x^2}{2}+\frac{x^3}{3}-\dotsb$& $(-1,1]$\smallskip\\ % Ex 9.10.2 (ish)
$\ds \frac{1}{1-x} = \sum_{n=0}^\infty x^n$ &$\ds 1+x+x^2+x^3+\dotsb$& $(-1,1)$\\ % equ {eq:power_series_geometric} 9.8.1
\small$\ds (1+x)^k=\sum_{n=0}^\infty \frac{k(k-1)\dotsm\bigl(k-(n-1)\bigr)}{n!}x^n$ \normalsize& $\ds 1+kx+\frac{k(k-1)}{2!}x^2 + \dotsb$ & {$\begin{cases}(-1,1)&\phantom{-}k\le-1\\{}(-1,1]&-1<k<0\\{}[-1,1]&\phantom{-}0<k\end{cases}$}\\ % ex {ex_ts4} 9.10.4
$\ds \tan^{-1}x = \sum_{n=0}^\infty (-1)^n\frac{x^{2n+1}}{2n+1}$ & $\ds x-\frac{x^3}{3}+\frac{x^5}{5}-\frac{x^7}{7}+\dotsb$ & $[-1,1]$
\end{tabular}\index{Taylor Series!common series}
\end{keyidea}
}

% todo Tim should we mention where in the text we derive the ``Important Maclaurin Series Expansions''
% todo Tim Do we want to derive the Maclaurin series for sin and arctan?

\begin{theorem}[Algebra of Power Series]\label{thm:series_alg}
Let $\ds f(x) = \sum_{n=0}^\infty a_nx^n$ and $\ds g(x) = \sum_{n=0}^\infty b_nx^n$ converge absolutely for $\abs x<R$, and let $h(x)$ be continuous.
\index{power series!algebra of} 
\begin{enumerate}
	\item	$\ds f(x)\pm g(x) = \sum_{n=0}^\infty (a_n\pm b_n)x^n$ \quad for $\abs x<R$.
	\item	$\ds f(x)g(x) = \left(\sum_{n=0}^\infty a_nx^n\right)\left(\sum_{n=0}^\infty b_nx^n\right) =\\
	\mbox{}\qquad
	\sum_{n=0}^\infty\bigl(a_0b_n+a_1b_{n-1}+\dotsb+ a_nb_0\bigr)x^n$ for $\abs x<R$.
	\item	$\ds f\bigl(h(x)\bigr) = \sum_{n=0}^\infty a_n\bigl(h(x)\bigr)^n$ \quad for $\abs{h(x)}<R$.
\end{enumerate}
\end{theorem}

\begin{example}[Combining Taylor series]\label{ex_ts5}
Write out the first 3 terms of the Maclaurin Series for $f(x) = e^x\cos x$ using \autoref{idea:common_taylor} and \autoref{thm:series_alg}.
\solution
\autoref{idea:common_taylor} informs us that 
\[e^x = 1+x+\frac{x^2}{2!}+\frac{x^3}{3!}+\dotsb\qquad \text{and}\qquad \cos x = 1-\frac{x^2}{2!}+\frac{x^4}{4!}+\dotsb.\]
Applying \autoref{thm:series_alg}, we find that
\begin{align*}
e^x\cos x &= \left(1+x+\frac{x^2}{2!}+\frac{x^3}{3!}+\dotsb\right)\left(1-\frac{x^2}{2!}+\frac{x^4}{4!}+\dotsb\right).
\intertext{Distribute the right hand expression across the left:}
	&= 1\left(1-\frac{x^2}{2!}+\frac{x^4}{4!}+\dotsb\right)
	+x\left(1-\frac{x^2}{2!}+\frac{x^4}{4!}+\dotsb\right)\\
	&\phantom{=}+\frac{x^2}{2!}\left(1-\frac{x^2}{2!}+\frac{x^4}{4!}+\dotsb\right)
	+\frac{x^3}{3!}\left(1-\frac{x^2}{2!}+\frac{x^4}{4!}+\dotsb\right)\\
	&\phantom{=}+ \frac{x^4}{4!}\left(1-\frac{x^2}{2!}+\frac{x^4}{4!}+\dotsb\right)+\dotsb
	\intertext{Distribute again and collect like terms.}
	&= 1 + x -\frac{x^3}{3}-\frac{x^4}{6} - \frac{x^5}{30}+\frac{x^7}{630}+\dotsb
\end{align*}
While this process is a bit tedious, it is much faster than evaluating all the necessary derivatives of $e^x\cos x$ and computing the Taylor series directly.

Because the series for $e^x$ and $\cos x$ both converge on $(-\infty,\infty)$, so does the series expansion for $e^x\cos x$.
\end{example}

\begin{example}[Creating new Taylor series]\label{ex_ts6}
Use \autoref{thm:series_alg} to create the Taylor series for $y=\sin(x^2)$ centered at $x=0$ and a series for $y=\ln (\sqrt{x})$ centered at $c=1$.
\solution
Given that 
\[\sin x = \sum_{n=0}^\infty (-1)^n\frac{x^{2n+1}}{(2n+1)!} = x-\frac{x^3}{3!}+\frac{x^5}{5!} -\frac{x^7}{7!}+\dotsb,\]
we simply substitute $x^2$ for $x$ in the series, giving
\[
\sin (x^2) = \sum_{n=0}^\infty (-1)^n\frac{(x^2)^{2n+1}}{(2n+1)!} = \sum_{n=0}^\infty (-1)^n\frac{x^{4n+2}}{(2n+1)!} = x^2-\frac{x^6}{3!}+\frac{x^{10}}{5!} -\frac{x^{14}}{7!}\dotsb.
\]
Since the Taylor series for $\sin x$ has an infinite radius of convergence, so does the Taylor series for $\sin(x^2)$.\bigskip

The Taylor expansion for $\ln(x+1)$ given in \autoref{idea:common_taylor} is centered at $x=0$, so we can center the series for $\ln (\sqrt{x})$ at $x=1$.
With 
\[\ln x = \sum_{n=1}^\infty(-1)^{n+1}\frac{(x-1)^n}{n} = (x-1)- \frac{(x-1)^2}{2} +\frac{(x-1)^3}{3}-\dotsb,\]
\mnote{\textbf{Note:} In \autoref{ex_ts6}, one could create a series for $\ln(\sqrt{x})$ by simply recognizing that $\ln(\sqrt{x}) = \ln (x^{1/2}) = 1/2\ln x$, and hence multiplying the Taylor series for $\ln x$ by $1/2$. This example was chosen to demonstrate other aspects of series, such as the fact that the interval of convergence changes.}
we substitute $\sqrt{x}$ for $x$ to obtain
\[\ln (\sqrt{x}) = \sum_{n=1}^\infty(-1)^{n+1}\frac{(\sqrt{x}-1)^n}{n} = (\sqrt{x}-1)- \frac{(\sqrt{x}-1)^2}{2} +\frac{(\sqrt{x}-1)^3}{3}-\dotsb.\]
While this is not strictly a power series because of the $\sqrt x$, it is a series that allows us to study the function $\ln(\sqrt{x})$. Since the interval of convergence of $\ln x$ is $(0,2]$, and the range of $\sqrt{x}$ on $(0,4]$ is $(0,2]$, the interval of convergence of this series expansion of $\ln(\sqrt{x})$ is $(0,4]$.
\end{example}

\begin{example}[Using Taylor series to evaluate definite integrals]\label{ex_ts7}
Use the Taylor series of $e^{-x^2}$ to evaluate $\ds \int_0^1e^{-x^2}\dd x$.
\solution
We learned, when studying Numerical Integration, that $e^{-x^2}$ does not have an antiderivative expressible in terms of elementary functions. This means any definite integral of this function must have its value approximated, and not computed exactly.

We can quickly write out the Taylor series for $e^{-x^2}$ using the Taylor series of $e^x$:\vspace{-.5\baselineskip}
\begin{align*}
e^x &= \sum_{n=0}^\infty \frac{x^n}{n!} = 1+x+\frac{x^2}{2!}+\frac{x^3}{3!}+\dotsb
\intertext{and so}
e^{-x^2} &= \sum_{n=0}^\infty \frac{(-x^2)^n}{n!} \\
				&= \sum_{n=0}^\infty (-1)^n\frac{x^{2n}}{n!}\\
				&= 1-x^2+\frac{x^4}{2!}-\frac{x^6}{3!}+\dotsb.
\end{align*}
We use \autoref{thm:calc_power_series} to integrate:
\[\int e^{-x^2}\dd x = C + x - \frac{x^3}{3}+\frac{x^5}{5\cdot2!}-\frac{x^7}{7\cdot3!}+\dotsb +(-1)^n\frac{x^{2n+1}}{(2n+1)n!}+\dotsb\]
This \emph{is} the antiderivative of $e^{-x^2}$; while we can write it out as a series, we cannot write it out in terms of elementary functions. We can evaluate the definite integral $\ds \int_0^1e^{-x^2}\dd x$ using this antiderivative; substituting 1 and 0 for $x$ and subtracting gives
\[\int_0^1e^{-x^2}\dd x = 1-\frac{1}{3}+\frac{1}{5\cdot 2!}-\frac{1}{7\cdot3!} + \frac{1}{9\cdot4!}-\dotsb.\]
Summing the 5 terms shown above gives the approximation of $0.74749.$ Since this is an alternating series, we can use the Alternating Series Approximation Theorem, (\autoref{thm:alt_series_approx}), to determine how accurate this approximation is. The next term of the series is $ 1/(11\cdot5!) \approx 0.00075758$. Thus we know our approximation is within $0.00075758$ of the actual value of the integral. This is arguably much less work than using Simpson's Rule to approximate the value of the integral.
\end{example}

Another advantage to using Taylor series instead of Simpson's Rule is for making subsequent approximations.  We found in \autoref{ex_num7} that the error in using Simpson's Rule for $\ds\int_0^1 e^{-x^2}\dd x$ with four intervals was $0.00026$.  If we wanted to decrease that error, we would need to use more intervals, essentially starting the problem over.  Using a Taylor series, if we wanted a more accurate approximation, we can just subtract the next term $1/(11\cdot5!)$ to get an approximation of $0.7467$, with an error of at most $1/(13\cdot6!)\approx0.0001$.\bigskip

%\begin{example}[Using Taylor series to solve differential equations]\label{ex_ts8}
%Solve the differential equation $y\primeskip'=2y$ in terms of a power series, and use the theory of Taylor series to recognize the solution in terms of an elementary function.
%\solution
%We found the first 5 terms of the power series solution to this differential equation in \autoref{ex_ps5} in \autoref{sec:power_series}. These are:
%\[a_0=1,\quad a_1 = 2,\quad a_2 = \frac42=2,\quad a_3=\frac{8}{2\cdot3}=\frac43,\quad a_4=\frac{16}{2\cdot3\cdot4} = \frac23.\]
%We include the ``unsimplified'' expressions for the coefficients found in \autoref{ex_ps5} as we are looking for a pattern. It can be shown that $a_n = 2^n/n!$. Thus the solution, written as a power series, is
%\[y = \sum_{n=0}^\infty \frac{2^n}{n!}x^n = \sum_{n=0}^\infty \frac{(2x)^n}{n!}.\]
%Using \autoref{idea:common_taylor} and \autoref{thm:series_alg}, we recognize $f(x) = e^{2x}$:
%\[e^x = \sum_{n=0}^\infty \frac{x^n}{n!} \qquad \Rightarrow \qquad e^{2x} = \sum_{n=0}^\infty \frac{(2x)^n}{n!}.\]
%\end{example}

Finding a pattern in the coefficients that match the series expansion of a known function, such as those shown in \autoref{idea:common_taylor}, can be difficult. What if the coefficients %in the previous example were
are given in their reduced form; how could we still recover the function?% $y=e^{2x}$?

Suppose that all we know is that 
\[a_0=1,\quad a_1=2,\quad a_2=2,\quad a_3=\frac43,\quad a_4=\frac23.\]
\autoref{def:taylor_series} states that each term of the Taylor expansion of a function includes an $n!$. This allows us to say that
\[a_2=2=\frac{b_2}{2!},\quad a_3 = \frac43=\frac{b_3}{3!},\quad \text{and}\quad a_4 = \frac23=\frac{b_4}{4!}\]
for some values $b_2$, $b_3$ and $b_4$.
Solving for these values, we see that $b_2=4$, $b_3 = 8$ and $b_4=16$. That is, we are recovering the pattern $b_n=2^n$, %we had previously seen,
allowing us to write 
\begin{align*}
	f(x) = \sum_{n=0}^\infty a_nx^n
	&= \sum_{n=0}^\infty \frac{b_n}{n!}x^n \\
	&= 1+2x+ \frac{4}{2!}x^2 + \frac{8}{3!}x^3+\frac{16}{4!}x^4 + \dotsb
\end{align*}
From here it is easier to recognize that the series is describing an exponential function.

%There are simpler, more direct ways of solving the differential equation $y\primeskip' = 2y$. We applied power series techniques to this equation to demonstrate its utility, and went on to show how \emph{sometimes} we are able to recover the solution in terms of elementary functions using the theory of Taylor series. Most differential equations faced in real scientific and engineering situations are much more complicated than this one, but power series can offer a valuable tool in finding, or at least approximating, the solution.\bigskip

This chapter introduced sequences, which are ordered lists of numbers, followed by series, wherein we add up the terms of a sequence. We quickly saw that such sums do not always add up to ``infinity,'' but rather converge. We studied tests for convergence, then ended the chapter with a formal way of defining functions based on series. Such ``series-defined functions'' are a valuable tool in solving a number of different problems throughout science and engineering.

Coming in the next chapters are new ways of defining curves in the plane apart from using functions of the form $y=f(x)$. Curves created by these new methods can be beautiful, useful, and important. 

% todo have the exercises find the interval of convergence where appropriate

\printexercises{exercises/08-08-exercises}

%\printexercisesreview{exercises/08_cr_exercises}

\apexchapter{Curves in the Plane}{chapter:planar_curves}
\section{Chapter Prerequisites --- Conic Sections}
\label{sec:conic_sections}

\prereqIntro

The ancient Greeks recognized that interesting shapes can be formed by intersecting a plane with a 
\textit{double napped} cone (i.e., two identical cones placed tip--to--tip as shown in the following figures). As these shapes are formed as sections of conics, they have earned the official name ``conic sections.''

The three ``most interesting'' conic sections are given in the top row of \autoref{fig:nondeg_conic}. They are the parabola, the ellipse (which includes circles) and the hyperbola. In each of these cases, the plane does not intersect the tips of the cones (usually taken to be the origin).

\begin{lxfigure}
\flushinner{%
\small
\begin{tabular}{cccc}
\myincludegraphics[scale=.25]{figures/conic_parabola}&
\myincludegraphics[scale=.25]{figures/conic_ellipse}&
\myincludegraphics[scale=.25]{figures/conic_circle}&
\myincludegraphics[scale=.25]{figures/conic_hyperbola} \\
Parabola & Ellipse & Circle &  Hyperbola \\
\myincludegraphics[scale=.25]{figures/conic_singlept}&
\myincludegraphics[scale=.25]{figures/conic_oneline}&
\myincludegraphics[scale=.25]{figures/conic_crossedlines} \\
Point & Line & Crossed Lines 
\end{tabular}}
\caption{Conic Sections}\label{fig:nondeg_conic}
\end{lxfigure}
% todo convert conic figures to asymptote (or at least tikz)

When the plane does contain the origin, three \textbf{degenerate} cones can be formed as shown the bottom row of \autoref{fig:nondeg_conic}: a point, a line, and crossed lines. We focus here on the nondegenerate cases.\index{conic sections}\index{conic sections!degenerate}

While the above geometric constructs define the conics in an intuitive, visual way, these constructs are not very helpful when trying to analyze the shapes algebraically or consider them as the graph of a function. It can be shown that all conics can be defined by the general second--degree equation
\[Ax^2+Bxy+Cy^2+Dx+Ey+F=0.\]
While this algebraic definition has its uses, most find another geometric perspective of the conics more beneficial.

Each nondegenerate conic can be defined as the \textbf{locus}, or set, of points that satisfy a certain distance property. These distance properties can be used to generate an algebraic formula, allowing us to study each conic as the graph of a function.

\subsection{Parabolas}

\definition{def:parabola}{Parabola}
{A \textbf{parabola} is the locus of all points equidistant from a point (called a \textbf{focus}) and a line (called the \textbf{directrix}) that does not contain the focus.
\index{conic sections!parabola}\index{parabola!definition}\index{directrix}\index{focus}
}

\mtable{Illustrating the definition of the parabola and establishing an algebraic formula.}{fig:parabola_def}{\begin{tikzpicture}[scale=.8]
	\draw [thick,draw={\colorone}] (-3,-1)
	 node [above,black,shift={(12pt,0pt)}] {\scriptsize Directrix}
	 -- (3,-1);
	\filldraw [black] (0,1) circle (1.5pt) node [above left] {\scriptsize Focus};
	\draw [thick,draw={\colortwo}](-3,2.25) parabola bend (0,0) (3,2.25);
	\filldraw (0,0) circle (1.5pt) node [below left] {\scriptsize Vertex};
	\draw (.3,.5) node[] {\scriptsize $\left.\rule{0pt}{12pt}\right\}p$};
	\draw (.3,-.5) node[] {\scriptsize $\left.\rule{0pt}{12pt}\right\}p$};
	\coordinate (A) at (2.5,1.5625);
	\filldraw [black] (A) circle (1.5pt) node [right] {\scriptsize $(x,y)$};
	\draw [thick,dashed](0,1) -- (A) node [pos=.5,above] {\scriptsize $d$}
	 -- (2.5,-1) node [pos=.5,right] {\scriptsize $d$};
	\draw [thick,dashed] (0,-1.5) -- node [pos=.85,above,rotate=90]
	 {\scriptsize Axis of}
	 node [pos=.85,below,rotate=90] {\scriptsize Symmetry} (0,3.2);
\end{tikzpicture}}

\autoref{fig:parabola_def} illustrates this definition. The point halfway between the focus and the directrix is the \textbf{vertex}. The line through the focus, perpendicular to the directrix, is the \textbf{axis of symmetry}, as the portion of the parabola on one side of this line is the mirror--image of the portion on the opposite side.

\ifthenelse{\boolean{abridgeConics}}{}{%
The definition leads us to an algebraic formula for the parabola. Let $P=(x,y)$ be a point on a parabola whose focus is at $F=(0,p)$ and whose directrix is at $y=-p$. (We'll assume for now that the focus lies on the $y$-axis; by placing the focus $p$ units above the $x$-axis and the directrix $p$ units below this axis, the vertex will be at $(0,0)$.)

We use the Distance Formula to find the distance $d_1$ between $F$ and $P$:
\[d_1=\sqrt{(x-0)^2+(y-p)^2}.\]
The distance $d_2$ from $P$ to the directrix is more straightforward:
\[d_2=y-(-p) = y+p.\]
These two distances are equal. Setting $d_1=d_2$, we can solve for $y$ in terms of $x$:
\begin{align*}
	d_1&= d_2 \\
	\sqrt{x^2+(y-p)^2} &= y+p 
	\intertext{Now square both sides.}
	x^2+(y-p)^2 &= (y+p)^2 \\
	x^2+y^2-2yp+p^2 &= y^2+2yp+p^2\\
	x^2 &=4yp\\
	y&= \frac{1}{4p}x^2.
\end{align*}%
}
The geometric definition of the parabola and distance formula can be used to derive the quadratic function whose graph is a parabola with vertex at the origin.
\[y=\frac{1}{4p}x^2.\]
Applying transformations of functions we get the following standard form of the parabola.

\keyidea{idea:parabola}{General Equation of a Parabola}
{\begin{enumerate}
	\item	\textbf{Vertical Axis of Symmetry:} The equation of the parabola with vertex at $(h,k)$, directrix $y=k-p$, and focus at $(h,k+p)$ in standard form is \[y=\frac{1}{4p}(x-h)^2+k.\]
	\item	\textbf{Horizontal Axis of Symmetry:} The  equation of the parabola with vertex at $(h,k)$, directrix $x=h-p$, and focus at $(h+p,k)$ in standard form is \[x=\frac{1}{4p}(y-k)^2+h.\]
\end{enumerate}
Note: $p$ is not necessarily a positive number.\index{parabola!general equation}}

\example{ex_conic1}{Finding the equation of a parabola}{Give the equation of the parabola with focus at $(1,2)$ and directrix at $y=3$.}
{The vertex is located halfway between the focus and directrix, so $(h,k) = (1,2.5)$. This gives $p=-0.5$. Using \autoref{idea:parabola} we have the equation of the parabola as
%
\mtable{The parabola described in \autoref{ex_conic1}.}{fig:conic1}{\begin{tikzpicture}
\begin{axis}[width=1.16\marginparwidth,tick label style={font=\scriptsize},
axis y line=middle,axis x line=middle,name=myplot,axis equal,
ymin=-6.5,ymax=2.9,xmin=-3.5,xmax=5.5]
\addplot [thick, draw={\colorone},smooth,domain=-3:5] {-.5*(x-1)^2+2.5};
\end{axis}
\node [right] at (myplot.right of origin) {\scriptsize $x$};
\node [above] at (myplot.above origin) {\scriptsize $y$};
\end{tikzpicture}}
%
\[y=\frac{1}{4(-0.5)}(x-1)^2+2.5 = -\frac12(x-1)^2+2.5.\]
The parabola is sketched in \autoref{fig:conic1}.}

\ifthenelse{\boolean{abridgeConics}}{}{%
\example{ex_conic2}{Finding the focus and directrix of a parabola}{Find the focus and directrix of the parabola $x=\frac18y^2-y+1$. The point $(7,12)$ lies on the graph of this parabola; verify that it is equidistant from the focus and directrix.}
{We need to put the equation of the parabola in its general form. This requires us to complete the square:
\begin{align*}
	x &= \frac18y^2-y+1 \\
	&= \frac18\big(y^2-8y+8\big)\\
	&=	\frac18\big(y^2-8y+16 -16+8\big)\\
	&=	\frac18\big((y-4)^2 - 8\big)\\
	&=	\frac18(y-4)^2 -1.
\end{align*}
Hence the vertex is located at $(-1,4)$.  We have $\frac18=\frac1{4p}$, so $p=2$. We conclude that the focus is located at $(1,4)$ and the directrix is  $x=-3$. The parabola is graphed in \autoref{fig:conic2}, along with its focus and directrix.\bigskip

\mtable{The parabola described in \autoref{ex_conic2}. The distances from a point on the parabola to the focus and directrix is given.}{fig:conic2}{\begin{tikzpicture}
\begin{axis}[width=1.16\marginparwidth,tick label style={font=\scriptsize},
axis y line=middle,axis x line=middle,name=myplot,xtick={-10,-5,5,10},
ymin=-5.5,ymax=14.5,xmin=-11,xmax=14,axis equal]
\addplot [thick, draw={\colorone},smooth,domain=-5:14] ({x^2/8-x+1},x);
\addplot [thick, draw={\colortwo},smooth,domain=-5:14] (-3,x);
\draw [dashed,draw={\colorone!70},thick] (axis cs:-3,12)
 -- node [above,pos=.5,black] {\scriptsize 10} (axis cs:7,12)
 -- node [pos=.6,right,black] {\scriptsize 10} (axis cs: 1,4);
\filldraw [draw={\colortwo}] (axis cs: 1,4) circle (1.5pt)
 (axis cs: 7,12) circle (1.5pt);
\end{axis}
\node [right] at (myplot.right of origin) {\scriptsize $x$};
\node [above] at (myplot.above origin) {\scriptsize $y$};
\end{tikzpicture}}

The point $(7,12)$ lies on the graph and is $7-(-3)=10$ units from the directrix. The distance from $(7,12)$ to the focus is:
\[\sqrt{(7-1)^2 + (12-4)^2} = \sqrt{100}=10.\]
Indeed, the point on the parabola is equidistant from the focus and directrix.}
	
\subsection{Reflective Property}

One of the fascinating things about the nondegenerate conic sections is their reflective properties. Parabolas have the following reflective property:

\begin{quote}
	Any ray emanating from the focus that intersects the parabola reflects off along a line perpendicular to the directrix.
\end{quote}

This is illustrated in \autoref{fig:conic_reflect}. The following theorem states this more rigorously.

\mtable{Illustrating the parabola's reflective property.}{fig:conic_reflect}{\begin{tikzpicture}
\begin{axis}[width=1.16\marginparwidth,tick label style={font=\scriptsize},
axis y line=none,axis x line=none,name=myplot,axis equal,
ymin=-6.5,ymax=14.5,xmin=-4,xmax=12]
\coordinate (focus) at (axis cs: 1,4);
\addplot [very thick, draw={\colorone},smooth,domain=-6:14] ({x^2/8-x+1},x);
\addplot [very thick, draw={\colortwo},smooth,domain=-6:14] (-3,x);
\draw [thick](axis cs: 12,12) -- (axis cs: 7,12) -- (focus)
			(axis cs: 12,8) -- (axis cs: 1,8) -- (focus)
			(axis cs: 12,3) -- (axis cs: -.875,3)
			(axis cs: -.875,3)-- (focus)
			(axis cs: 12,-3) -- (axis cs: 5.125,-3) -- (focus);
\filldraw [draw={\colortwo}] (focus) circle (1.5pt);
\end{axis}
\end{tikzpicture}}
	
\theorem{thm:parabola_reflect}{Reflective Property of the Parabola}
{Let $P$ be a point on a parabola. The tangent line to the parabola at $P$ makes equal angles with the following two lines:
\begin{enumerate}
	\item	The line containing $P$ and the focus $F$, and
	\item	The line perpendicular to the directrix through $P$.
	\index{parabola!reflective property}
\end{enumerate}}

% todo prove the reflective property of parabolas

Because of this reflective property, paraboloids (the 3D analogue of parabolas) make for useful flashlight reflectors as the light from the bulb, ideally located at the focus, is reflected along parallel rays. Satellite dishes also have paraboloid shapes. Signals coming from satellites effectively approach the dish along parallel rays. The dish then \textit{focuses} these rays at the focus, where the sensor is located.
}

\subsection{Ellipses}

\definition{def:ellipse}{Ellipse}
{An \textbf{ellipse} is the locus of all points whose sum of distances from two fixed points, each a \textbf{focus} of the ellipse, is constant.\index{conic sections!ellipse}\index{ellipse!definition}\index{focus}}

An easy way to visualize this construction of an ellipse is to pin both ends of a string to a board. The pins become the foci. Holding a pencil tight against the string places the pencil on the ellipse; the sum of distances from the pencil to the pins is constant: the length of the string. See \autoref{fig:ellipse_def}.

\ifthenelse{\boolean{abridgeConics}}{}{%
We can again find an algebraic equation for an ellipse using this geometric definition. Let the foci be located along the $x$-axis, $c$ units from the origin. Let these foci be labeled as $F_1 = (-c,0)$ and $F_2=(c,0)$. Let $P=(x,y)$ be a point on the ellipse. The sum of distances from $F_1$ to $P$ ($d_1$) and from $F_2$ to $P$ ($d_2$) is a constant $d$. That is, $d_1+d_2=d$. Using the Distance Formula, we have 
\[\sqrt{(x+c)^2+y^2} + \sqrt{(x-c)^2+y^2} = d.\]
Using a fair amount of algebra can produce the following equation of an ellipse (note that the equation is an implicitly defined function; it has to be, as an ellipse fails the Vertical Line Test):
\[
\frac{x^2}{\left(\frac d2\right)^2} + \frac{y^2}{\left(\frac d2\right)^2-c^2} = 1.
\]
This is not particularly illuminating, but by making the substitution $a=d/2$ and $b=\sqrt{a^2-c^2}$, we can rewrite the above equation as 
\[\frac{x^2}{a^2} + \frac{y^2}{b^2} = 1.\]
}

As shown in \autoref{fig:ellipse_label}, the values of $a$ and $b$ have
% geometric
meaning. In general, the two foci of an ellipse lie on the \textbf{major axis} of the ellipse, and the midpoint of the segment joining the two foci is the \textbf{center}. The major axis intersects the ellipse at two points, each of which is a \textbf{vertex}. The line segment through the center and perpendicular to the major axis is the \textbf{minor axis}. The ``constant sum of distances'' that defines the ellipse is the length of the major axis, i.e., $2a$.
%
% this is not the figure that was just referenced, but the figure from before the previous paragraph.  But I can't get TeX to realize what page it goes on otherwise
\mtable[-6\baselineskip]{Illustrating the construction of an ellipse with pins, pencil and string.}{fig:ellipse_def}{\begin{tikzpicture}
	\draw [thick,dashed] (0,0) circle [x radius=2,y radius=1.5];
	\filldraw (1.3,0)circle(1.5pt) (-1.3,0)circle(1.5pt);
	\draw  (1.3,0) -- node [right,pos=.5] {\scriptsize $d_1$}
	 (1.39,1.07) -- node [above,pos=.5] {\scriptsize $d_2$} (-1.3,0);
	\draw (0,-.7) node {\scriptsize $d_1+d_2=$ constant};
	\begin{scope}[shift={(1.9cm,1.225cm)},rotate=120,xscale=.25,yscale=.5]
		\draw (0,0) -- (0,-2) -- (1,-2)--(1,0) -- (.5,1) -- cycle;
		\draw [fill=black] (.3,.6) -- ( .5,1)--(.7,.6)--cycle;
		\draw (0,0) cos (.5,-.1) sin (1,0);
	\end{scope}
\end{tikzpicture}}

\mtable{Labeling the significant features of an ellipse.}{fig:ellipse_label}{\begin{tikzpicture}
 \draw [thick,draw={\colorone}] (0,0) circle [x radius=2,y radius=1.5];
 \draw [thick,dashed] (-2.4,0) -- (2.4,0) (0,1.8) -- (0,-1.9);
 \filldraw [draw={\colortwo}] (1.3,0) circle (1.5pt) (-1.3,0) circle (1.5pt);
 \draw [->,>=latex] (-2,-2) node [fill=white] {\scriptsize Major axis}
  -- (-2.2,-.1);
 \draw [->,>=latex] (2,-2) node [fill=white] {\scriptsize Minor axis}
  -- (.1,-1.8);
 \draw [->,>=latex] (-2,2)  -- (-2,.3);
 \draw [->,>=latex] (-2,2) node [fill=white] {\scriptsize Vertices} -- (1.9,.1);
 \draw [->,>=latex] (2,2)  -- (-1.2,.1);
 \draw [->,>=latex] (2,2) node [fill=white] {\scriptsize Foci} -- (1.3,.1);
 \filldraw [draw={\colorone}] (2,0) circle (1.5pt) (-2,0) circle (1.5pt);
 \draw (-1,-.25) node {\scriptsize $\underbrace{\rule{1.8cm}{0pt}}_a$};
 \draw (.65,-.25) node {\scriptsize $\underbrace{\rule{1.1cm}{0pt}}_c$};
 \draw (-.25,.75) node [] {\scriptsize $b\left\{\rule[-.65cm]{0pt}{1.3cm}\right.$};
\end{tikzpicture}}

Allowing for the shifting of the ellipse gives the following standard equations.
	
\keyidea{idea:ellipse}{Standard Equation of the Ellipse}
{The equation of an ellipse centered at $(h,k)$ with major axis of length $2a$ and minor axis of length $2b$ in standard form is:
\begin{enumerate}
	\item	\textbf{Horizontal major axis:} $\ds \frac{(x-h)^2}{a^2}+\frac{(y-k)^2}{b^2}=1.$
	\item	\textbf{Vertical major axis:} $\ds \frac{(x-h)^2}{b^2}+\frac{(y-k)^2}{a^2}=1.$
\end{enumerate}
The foci lie along the major axis, $c$ units from the center, where $c^2=a^2-b^2$.
\index{ellipse!standard equation}}

\example{ex_conic3}{Finding the equation of an ellipse}{Find the general equation of the ellipse graphed in \autoref{fig:conic3}.
%
\mtable{The ellipse used in \autoref{ex_conic3}.}{fig:conic3}{\begin{tikzpicture}
\begin{axis}[width=1.16\marginparwidth,tick label style={font=\scriptsize},
axis y line=middle,axis x line=middle,name=myplot,xtick={-6,-4,-2,2,4,6},
ytick={-4,-2,2,4,6},ymin=-4.9,ymax=6.9,xmin=-6.9,xmax=6.9]
\addplot [thick, draw={\colorone},smooth,domain=0:360,samples=60]
 ({2*cos(x)-3},{5*sin(x)+1});
\filldraw (axis cs:-3,1)  circle (1.5pt);
\end{axis}
\node [right] at (myplot.right of origin) {\scriptsize $x$};
\node [above] at (myplot.above origin) {\scriptsize $y$};
\end{tikzpicture}}}
{The center is located at $(-3,1)$. The distance from the center to a vertex is 5 units, hence $a=5$. The minor axis seems to have length 4, so $b=2$. Thus the equation of the ellipse is
\[\frac{(x+3)^2}{4}+\frac{(y-1)^2}{25} = 1.\eoehere\]}

\example{ex_conic4}{Graphing an ellipse}{Graph the ellipse defined by $4x^2+9y^2-8x-36y=-4$.}
{It is simple to graph an ellipse once it is in standard form. In order to put the given equation in standard form, we must complete the square with both the $x$ and $y$ terms. We first rewrite the equation by regrouping:
\[4x^2+9y^2-8x-36y=-4 \quad \Rightarrow \quad (4x^2-8x) + (9y^2-36y) = -4.\]
Now we complete the squares.
\begin{align*}
	(4x^2-8x) + (9y^2-36y) &= -4\\
	4(x^2-2x) + 9(y^2-4y) &= -4 \\
	4(x^2-2x +1 - 1) + 9(y^2-4y+4-4) &= - 4\\
	4\big((x-1)^2-1\big) + 9\big((y-2)^2-4\big) &= -4\\
	4(x-1)^2 -4 + 9(y-2)^2-36 &= -4 \\
	4(x-1)^2 + 9(y-2)^2 &= 36 \\
	\frac{(x-1)^2}{9} + \frac{(y-2)^2}{4} &= 1.
\end{align*}
%
\mtable{Graphing the ellipse in \autoref{ex_conic4}.}{fig:conic4}{\begin{tikzpicture}
\begin{axis}[width=1.16\marginparwidth,tick label style={font=\scriptsize},
axis y line=middle,axis x line=middle,name=myplot,xtick={-2,-1,1,2,3,4},
ytick={-1,1,2,3,4},ymin=-2,ymax=5,xmin=-3,xmax=5]
\addplot [thick, draw={\colorone},smooth,domain=0:360,samples=60]
 ({3*cos(x)+1},{2*sin(x)+2});
\filldraw (axis cs:1,2)  circle (1.5pt)
		(axis cs:3.24,2) circle (1pt)
		(axis cs:-1.24,2) circle (1pt);
\filldraw [draw={\colorone}] (axis cs: 4,2) circle (1.5pt) (axis cs: -2,2) circle (1.5pt);
\end{axis}
\node [right] at (myplot.right of origin) {\scriptsize $x$};
\node [above] at (myplot.above origin) {\scriptsize $y$};
\end{tikzpicture}}%
%
We see the center of the ellipse is at $(1,2)$. We have $a=3$ and $b=2$; the major axis is horizontal, so the vertices are located at $(-2,2)$ and $(4,2)$. We find $c=\sqrt{9-4} = \sqrt{5}\approx 2.24.$ The foci are located along the major axis, approximately $2.24$ units from the center, at $(1\pm 2.24,2)$. This is all graphed in \autoref{fig:conic4}.}

\ifthenelse{\boolean{abridgeConics}}{}{%
\subsection{Eccentricity}

When $a=b$, we have a circle. The general equation becomes
\[\frac{(x-h)^2}{a^2} + \frac{(y-k)^2}{a^2} = 1 \quad \Rightarrow \quad (x-h)^2 + (y-k)^2 = a^2,\]
the familiar equation of the circle centered at $(h,k)$ with radius $a$.  Since $a=b$, $c = \sqrt{a^2-b^2}=0$. The circle has ``two'' foci, but they lie on the same point, the center of the circle. 

Consider \autoref{fig:ellipse_ecc}, where several ellipses are graphed with $a=1$. In (a), we have $c=0$ and the ellipse is a circle. As $c$ grows, the resulting ellipses look less and less circular. A measure of this ``noncircularness'' is \textit{eccentricity}.

\begin{lxfigure}
 \centering
 \begin{tabular}{cc}
\begin{tikzpicture}
\begin{axis}[width=\marginparwidth,tick label style={font=\scriptsize},
axis y line=middle,axis x line=middle,name=myplot,xtick={-1,1},ytick={-1,1},
ymin=-1.1,ymax=1.1,xmin=-1.35,xmax=1.35,axis equal]
\draw[thick,draw={\colorone}](axis cs:0,0)circle(1);
\filldraw (axis cs:0,0)circle(1.5pt);
\draw (axis cs:.8,-.9) node {\scriptsize $e=0$};
\end{axis}
\node [right] at (myplot.right of origin) {\scriptsize $x$};
\node [above] at (myplot.above origin) {\scriptsize $y$};
\end{tikzpicture}
  &
\begin{tikzpicture}
\begin{axis}[width=\marginparwidth,tick label style={font=\scriptsize},
axis y line=middle,axis x line=middle,name=myplot,xtick={-1,1},ytick={-1,1},
ymin=-1.1,ymax=1.1,xmin=-1.35,xmax=1.35,axis equal]
\addplot [thick, draw={\colorone},smooth,domain=0:360,samples=60]
 ({cos(x)},{.95*sin(x)});
\filldraw (axis cs:.3,0)circle(1.5pt) (axis cs:-.3,0)circle(1.5pt);
\draw (axis cs:.8,-.9) node {\scriptsize $e=0.3$};					
\end{axis}
\node [right] at (myplot.right of origin) {\scriptsize $x$};
\node [above] at (myplot.above origin) {\scriptsize $y$};
\end{tikzpicture}
  \\(a) & (b)\\
\begin{tikzpicture}
\begin{axis}[width=\marginparwidth,tick label style={font=\scriptsize},
axis y line=middle,axis x line=middle,name=myplot,xtick={-1,1},ytick={-1,1},
ymin=-1.1,ymax=1.1,xmin=-1.35,xmax=1.35,axis equal]
\addplot [thick, draw={\colorone},smooth,domain=0:360,samples=60]
 ({cos(x)},{.6*sin(x)});
\filldraw (axis cs:.8,0)circle(1.5pt) (axis cs:-.8,0)circle(1.5pt);
\draw (axis cs:.8,-.9) node {\scriptsize $e=0.8$};					
\end{axis}
\node [right] at (myplot.right of origin) {\scriptsize $x$};
\node [above] at (myplot.above origin) {\scriptsize $y$};
\end{tikzpicture}
  &
\begin{tikzpicture}
\begin{axis}[width=\marginparwidth,tick label style={font=\scriptsize},
axis y line=middle,axis x line=middle,name=myplot,xtick={-1,1},ytick={-1,1},
ymin=-1.1,ymax=1.1,xmin=-1.35,xmax=1.35,axis equal]
\addplot [thick, draw={\colorone},smooth,domain=0:360,samples=60]
 ({cos(x)},{.141*sin(x)});
\filldraw (axis cs:.99,0)circle(1.5pt) (axis cs:-.99,0)circle(1.5pt);
\draw (axis cs:.8,-.9) node {\scriptsize $e=0.99$};
\end{axis}
\node [right] at (myplot.right of origin) {\scriptsize $x$};
\node [above] at (myplot.above origin) {\scriptsize $y$};
\end{tikzpicture}
  \\(c) & (d)
 \end{tabular}
 \caption{Understanding the eccentricity of an ellipse.}
 \label{fig:ellipse_ecc}
\end{lxfigure}

\definition{def:eccentricity_ellipse}{Eccentricity of an Ellipse}
{The eccentricity $e$ of an ellipse  is $\ds e=\frac{c}{a}$.
\index{ellipse!eccentricity}\index{eccentricity}}

The eccentricity of a circle is 0; that is, a circle has no ``noncircularness.'' As $c$ approaches $a$, $e$ approaches 1, giving rise to a very noncircular ellipse, as seen in \autoref{fig:ellipse_ecc} (d). 

It was long assumed that planets had circular orbits. This is known to be incorrect; the orbits are elliptical. Earth has an eccentricity of $0.0167$ -- it has a nearly circular orbit.   Mercury's orbit is the most eccentric, with $e=0.2056$. (Pluto's eccentricity is greater, at $e=0.248$, the greatest of all the currently known dwarf planets.) The planet with the most circular orbit is Venus, with $e=0.0068$. The Earth's moon has an eccentricity of $e=0.0549$, also very circular.

\subsection{Reflective Property}

The ellipse also possesses an interesting reflective property. Any ray emanating from one focus of an ellipse reflects off the ellipse along a line through the other focus, as illustrated in \autoref{fig:ellipse_reflect}. This property is given formally in the following theorem.

\mtable{Illustrating the reflective property of an ellipse.}{fig:ellipse_reflect}{\begin{tikzpicture}[scale=1.2]%[width=1.16\marginparwidth]
\draw [thick] (0,0) circle[x radius=2cm,y radius = 1.2cm];
\filldraw (1.6,0)  circle (1.5pt) node [above left] {\scriptsize $F_2$}
			(-1.6,0)  circle (1.5pt) node [above] {\scriptsize $F_1$};
\draw[->,>=latex,draw={\colorone},thick] (1.6,0) -- ({2*cos(30)},{1.2*sin(30)});
\draw[->,>=latex,draw={\colorone},thick]({2*cos(30)},{1.2*sin(30)}) -- (-1.55,0);
\draw[->,>=latex,draw={\colortwo},thick] (-1.6,0) -- ({2*cos(250)},{1.2*sin(250)});
\draw[->,>=latex,draw={\colortwo},thick]  ({2*cos(250)},{1.2*sin(250)}) -- (1.55,0);
\end{tikzpicture}}

\theorem{thm:ellipse_reflect}{Reflective Property of an Ellipse}
{Let $P$ be a point on a ellipse with foci $F_1$ and $F_2$. The tangent line to the ellipse at $P$ makes equal angles with the following two lines:\index{ellipse!reflective property}
\begin{enumerate}
	\item The line through $F_1$ and $P$, and
	\item The line through $F_2$ and $P$. 
\end{enumerate}}

% todo prove the reflective property of ellipses

This reflective property is useful in optics and is the basis of the phenomena experienced in whispering halls.
}

% todo prove that a parabola is the limiting case of an ellipse

\subsection{Hyperbolas}

The definition of a hyperbola is very similar to the definition of an ellipse; we essentially just change the word ``sum'' to ``difference.''

\definition{def:hyperbola}{Hyperbola}
{A \textbf{hyperbola} is the locus of all points where the absolute value of the difference of distances from two fixed points, each a focus of the hyperbola, is constant.
\index{conic sections!hyperbola}\index{hyperbola!definition}\index{focus}}

We do not have a convenient way of visualizing the construction of a hyperbola as we did for the ellipse. The geometric definition does allow us to find an algebraic expression that describes it. It will be useful to define some terms first.

\mtable{Labeling the significant features of a hyperbola.}{fig:hyperbola_def}{\begin{tikzpicture}
\begin{axis}[width=1.16\marginparwidth,tick label style={font=\scriptsize},
axis y line=none,axis x line=none,name=myplot,
ymin=-3.2,ymax=3.2,xmin=-3.2,xmax=3.2]
\addplot [thick, draw={\colorone},smooth,domain=-70:70,samples=60] ({sec(x)},{tan(x)});
\addplot [thick, draw={\colorone},smooth,domain=-70:70,samples=60] ({-sec(x)},{tan(x)});
\filldraw (axis cs:0,0)  circle (1.5pt);
\draw [thick,dashed] 	(axis cs:-3.2,0)
 --  node [above,pos=.88] {\scriptsize Transverse}
 node [below,pos=.88] {\scriptsize axis}(axis cs:3.2,0)
 (axis cs:0,-3.2)
 -- node [below,rotate=90,pos=.8] {\scriptsize  axis }
 node [above,rotate=90,pos=.8] {\scriptsize Conjugate}(axis cs:0,3.2);
\filldraw [draw={\colorone}] (axis cs:1,0)circle(1.5pt) (axis cs:-1,0)circle(1.5pt);
\filldraw [draw={\colortwo}] (axis cs:1.4,0)circle(1.5pt) node (f1) {}
						(axis cs:-1.4,0)  circle (1.5pt) node (f2) {};
\draw [->,>=latex] (axis cs: 1.25,-2.8) -- (axis cs: -1.35,-.1);
\draw [->,>=latex] (axis cs: 1.25,-2.8) node [fill=white] {\scriptsize Foci }
 -- (axis cs: 1.4,-.1);	
\draw [->,>=latex] (axis cs: -1.25,-2.8) -- (axis cs: -1,-.1);
\draw [->,>=latex] (axis cs: -1.25,-2.8) node [fill=white] {\scriptsize Vertices}
 -- (axis cs: .95,-.1);	
\draw (axis cs:-.5,.4) node {\scriptsize $\overbrace{\rule{18pt}{0pt}}^a$};
\draw (axis cs:.7,.4) node {\scriptsize $\overbrace{\rule{25pt}{0pt}}^c$};
\end{axis}
\end{tikzpicture}}

The two foci lie on the \textbf{transverse axis} of the hyperbola; the midpoint of the line segment joining the foci is the \textbf{center} of the hyperbola. The transverse axis intersects the hyperbola at two points, each a \textbf{vertex} of the hyperbola. The line through the center and perpendicular to the transverse axis is the \textbf{conjugate axis.} This is illustrated in \autoref{fig:hyperbola_def}. It is easy to show that the constant difference of distances used in the definition of the hyperbola is the distance between the vertices, i.e., $2a$.

\keyidea{idea:hyperbola}{Standard Equation of a Hyperbola}
{The equation of a hyperbola centered at $(h,k)$ in standard form is:
\index{hyperbola!standard equation}
\begin{enumerate}
	\item \parbox{120pt}{\textbf{Horizontal Transverse Axis:}} $\ds \frac{(x-h)^2}{a^2} - \frac{(y-k)^2}{b^2} = 1.$
	\item	\parbox{120pt}{\textbf{Vertical Transverse Axis:}} $\ds \frac{(y-k)^2}{a^2}-\frac{(x-h)^2}{b^2} = 1.$
\end{enumerate}
The vertices are located $a$ units from the center and the foci are located $c$ units from the center, where $c^2 = a^2+b^2$.}

\subsection{Graphing Hyperbolas}

\mtable[-1in]{Graphing the hyperbola $\frac{x^2}9-\frac{y^2}1 = 1$ along with its asymptotes, $y=\pm x/3$.}{fig:hyperbola_asy1}{\begin{tikzpicture}
\begin{axis}[width=1.16\marginparwidth,tick label style={font=\scriptsize},
axis y line=middle,axis x line=middle,name=myplot,
ymin=-3.2,ymax=3.2,xmin=-8.2,xmax=8.2]
\addplot [thick, draw={\colorone},smooth,domain=-70:70,samples=60] ({3*sec(x)},{tan(x)});
\addplot [thick, draw={\colorone},smooth,domain=-70:70,samples=60] ({-3*sec(x)},{tan(x)});
\addplot [thick,dashed,draw={\colortwo},domain=-8:8] {x/3};
\addplot [thick,dashed,draw={\colortwo},domain=-8:8] {-x/3};
\end{axis}
\node [right] at (myplot.right of origin) {\scriptsize $x$};
\node [above] at (myplot.above origin) {\scriptsize $y$};
\end{tikzpicture}}

Consider the hyperbola $\frac{x^2}9-\frac{y^2}1 = 1$. Solving for $y$, we find $y=\pm\sqrt{x^2/9-1}$. As $x$ grows large, the ``$-1$'' part of the equation for $y$ becomes less significant and $y\approx \pm\sqrt{x^2/9} = \pm x/3$. That is, as $x$ gets large, the graph of the hyperbola looks very much like the lines $y=\pm x/3$. These lines are asymptotes of the hyperbola, as shown in \autoref{fig:hyperbola_asy1}.

\mtable{Using the asymptotes of a hyperbola as a graphing aid.}{fig:hyperbola_asy2}{\begin{tikzpicture}
\begin{axis}[width=1.16\marginparwidth,tick label style={font=\scriptsize},
axis y line=middle,axis x line=middle,name=myplot,xtick=\empty,extra x ticks={2,8,5},
extra x tick labels={$h-a$, $h+a$,$h$},ytick=\empty,extra y ticks={2,4,6},
extra y tick labels={$k-b$, $k$, $k+b$},ymin=-2.2,ymax=10.2,xmin=-3.2,xmax=13.2]
\addplot [thick, draw={\colorone},smooth,domain=-70:70,samples=60]
 ({3*sec(x)+5},{2*tan(x)+4});
\addplot [thick, draw={\colorone},smooth,domain=-70:70,samples=60]
 ({-3*sec(x)+5},{2*tan(x)+4});
\draw [thick,dashed] (axis cs: 2,6) -- (axis cs:8,6) -- (axis cs:8,2)
 -- (axis cs: 2,2) -- cycle;
\addplot [thick,dashed,draw={\colortwo},domain=-3:13] {2*(x-5)/3+4};
\addplot [thick,dashed,draw={\colortwo},domain=-3:13] {-2*(x-5)/3+4};
\filldraw (axis cs:5,4) circle (1.5pt);
\filldraw [draw={\colorone}] (axis cs:2,4) circle (1.5pt);
\filldraw [draw={\colorone}] (axis cs:8,4) circle (1.5pt);
\end{axis}
\node [right] at (myplot.right of origin) {\scriptsize $x$};
\node [above] at (myplot.above origin) {\scriptsize $y$};
\end{tikzpicture}}

This is a valuable tool in sketching. Given the equation of a hyperbola in general form, draw a rectangle centered at $(h,k)$ with sides of length $2a$ parallel to the transverse axis and sides of length $2b$ parallel to the conjugate axis. (See \autoref{fig:hyperbola_asy2} for an example with a horizontal transverse axis.) The diagonals of the rectangle lie on the asymptotes. 

These lines pass through $(h,k)$.  When the transverse axis is horizontal, the slopes are $\pm b/a$; when the transverse axis is vertical, their slopes are $\pm a/b$. This gives equations:
\begin{center}
\begin{tabular}{cc}
\parbox{100pt}{\centering Horizontal \\ Transverse Axis} & \parbox{100pt}{\centering Vertical \\ Transverse Axis} \\ \ \\
$\ds y=\pm\frac ba(x-h)+k$  &$\ds  y=\pm\frac ab(x-h)+k.$
\end{tabular}
\end{center}

\mtable{Graphing the hyperbola in \autoref{ex_conic5}.}{fig:conic5}{\begin{tikzpicture}
\begin{axis}[width=1.16\marginparwidth,tick label style={font=\scriptsize},
axis y line=middle,axis x line=middle,name=myplot,minor x tick num=4,
minor y tick num=4,ymin=-7.2,ymax=11.2,xmin=-5.2,xmax=7.2]
\addplot [thick, draw={\colorone},smooth,domain=-70:70,samples=60]
 ({2*tan(x)+1},{5*sec(x)+2});
\addplot [thick, draw={\colorone},smooth,domain=-70:70,samples=60]
 ({2*tan(x)+1},{-5*sec(x)+2});
\draw [thick,dashed] (axis cs: -1,7) -- (axis cs:3,7) -- (axis cs:3,-3)
 -- (axis cs: -1,-3) -- cycle;
\addplot [thick,dashed,draw={\colortwo},domain=-3:13] {5*(x-1)/2+2};
\addplot [thick,dashed,draw={\colortwo},domain=-3:13] {-5*(x-1)/2+2};
\filldraw (axis cs:1,2) circle (1.5pt);
\filldraw [draw={\colorone}] (axis cs:1,7) circle (1.5pt);
\filldraw [draw={\colorone}] (axis cs:1,-3) circle (1.5pt);
\filldraw [draw={\colortwo}] (axis cs:1,7.4) circle (1.5pt);
\filldraw [draw={\colortwo}] (axis cs:1,-3.4) circle (1.5pt);
\end{axis}
\node [right] at (myplot.right of origin) {\scriptsize $x$};
\node [above] at (myplot.above origin) {\scriptsize $y$};
\end{tikzpicture}}

\example{ex_conic5}{Graphing a hyperbola}{Sketch the hyperbola given by $\ds \frac{(y-2)^2}{25}-\frac{(x-1)^2}{4}=1.$}
{The hyperbola is centered at $(1,2)$; $a=5$ and $b=2$.
In \autoref{fig:conic5} we draw the prescribed rectangle centered at $(1,2)$ along with the asymptotes defined by its diagonals. The hyperbola has a vertical transverse axis, so the vertices are located at $(1,7)$ and $(1,-3)$. This is enough to make a good sketch.

We also find the location of the foci: as $c^2= a^2+b^2$, we have $c=\sqrt{29}\approx 5.4$. Thus the foci are located at $(1,2\pm 5.4)$ as shown in the figure.}

\mtable{Graphing the hyperbola in \autoref{ex_conic6}.}{fig:conic6}{\begin{tikzpicture}
\begin{axis}[width=1.16\marginparwidth,tick label style={font=\scriptsize},
axis y line=middle,axis x line=middle,name=myplot,minor x tick num=1,
minor y tick num=4,ymin=-10.9,ymax=11.9,xmin=-4.2,xmax=4.2]
\addplot [thick, draw={\colorone},smooth,domain=-75:75,samples=60]
 ({sec(x)},{3*tan(x)+1});
\addplot [thick, draw={\colorone},smooth,domain=-75:75,samples=60]
 ({-sec(x)},{3*tan(x)+1});
\draw [thick,dashed] (axis cs: -1,4) -- (axis cs:1,4) -- (axis cs:1,-2)
 -- (axis cs: -1,-2) -- cycle;
\addplot [thick,dashed,draw={\colortwo},domain=-4:4] {3*(x-0)/1+1};
\addplot [thick,dashed,draw={\colortwo},domain=-4:4] {-3*(x-0)/1+1};
\filldraw (axis cs:0,1) circle (1.5pt);
\filldraw [draw={\colorone}] (axis cs:-1,1) circle (1.5pt);
\filldraw [draw={\colorone}] (axis cs:1,1) circle (1.5pt);
\filldraw [draw={\colortwo}] (axis cs:3.2,1) circle (1.5pt);
\filldraw [draw={\colortwo}] (axis cs:-3.2,1) circle (1.5pt);
\end{axis}
\node [right] at (myplot.right of origin) {\scriptsize $x$};
\node [above] at (myplot.above origin) {\scriptsize $y$};
\end{tikzpicture}}

\example{ex_conic6}{Graphing a hyperbola}{Sketch the hyperbola given by $9x^2-y^2+2y=10.$}
{We must complete the square to put the equation in general form. (We recognize this as a hyperbola since it is a general quadratic equation and the $x^2$ and $y^2$ terms have opposite signs.)
\begin{align*}
	9x^2-y^2+2y &=10\\
	9x^2- (y^2-2y) &= 10\\
	9x^2 - (y^2-2y+1-1) &= 10\\
	9x^2 -\big((y-1)^2-1\big) &= 10\\
	9x^2 - (y-1)^2 &= 9\\
	x^2 - \frac{(y-1)^2}{9} &=1
\end{align*}

We see the hyperbola is centered at $(0,1)$, with a horizontal transverse axis, where $a=1$ and $b=3$. The appropriate rectangle is sketched in \autoref{fig:conic6} along with the asymptotes of the hyperbola. The vertices are located at $(\pm 1,1)$. We have $c=\sqrt{10}\approx 3.2$, so the foci are located at $(\pm 3.2,1)$ as shown in \autoref{fig:conic6}.}

\ifthenelse{\boolean{abridgeConics}}{}{%
\subsection{Eccentricity}

\definition{def:hyperbola_eccentricity}{Eccentricity of a Hyperbola}
{The eccentricity of a hyperbola is $\ds e=\frac ca$.\index{hyperbola!eccentricity}\index{eccentricity}}

Note that this is the definition of eccentricity as used for the ellipse.  When $c$ is close in value to $a$ (i.e., $e\approx 1$), the hyperbola is very narrow (looking almost like crossed lines). \autoref{fig:hyperbola_ecc} shows hyperbolas centered at the origin with $a=1$. The graph in (a) has $c=1.05$, giving an eccentricity of $e=1.05$, which is close to 1. As $c$ grows larger, the hyperbola widens and begins to look like parallel lines, as shown in part (d) of the figure.

\begin{lxfigure}
 \begin{tabular}{cc}
\begin{tikzpicture}
\begin{axis}[width=1.16\marginparwidth,tick label style={font=\scriptsize},
axis y line=middle,axis x line=middle,name=myplot,xtick={-10,-5,5,10},
minor x tick num=4,ytick={-10,-5,5,10},minor y tick num=4,
ymin=-11.9,ymax=11.9,xmin=-11.9,xmax=11.9,axis equal]
\addplot [thick, draw={\colorone},smooth,domain=-85:85,samples=60]
 ({sec(x)},{.32*tan(x)});
\addplot [thick, draw={\colorone},smooth,domain=-85:85,samples=60]
 ({-sec(x)},{.32*tan(x)});
\filldraw (axis cs:0,0) circle (1.5pt);
\filldraw [draw={\colorone}] (axis cs:-1,0) circle (1.5pt);
\filldraw [draw={\colorone}] (axis cs:1,0) circle (1.5pt);
\filldraw [draw={\colortwo}] (axis cs:1.05,0) circle (1.5pt);
\filldraw [draw={\colortwo}] (axis cs:-1.05,0) circle (1.5pt);
\draw (axis cs: 9,-6) node {\scriptsize $e = 1.05$};					
\end{axis}
\node [right] at (myplot.right of origin) {\scriptsize $x$};
\node [above] at (myplot.above origin) {\scriptsize $y$};
\end{tikzpicture}
  &
\begin{tikzpicture}
\begin{axis}[width=1.16\marginparwidth,tick label style={font=\scriptsize},
axis y line=middle,axis x line=middle,name=myplot,xtick={-10,-5,5,10},
minor x tick num=4,ytick={-10,-5,5,10},minor y tick num=4,
ymin=-11.9,ymax=11.9,xmin=-11.9,xmax=11.9,axis equal]
\addplot [thick, draw={\colorone},smooth,domain=-85:85,samples=60]
 ({sec(x)},{1.12*tan(x)});
\addplot [thick, draw={\colorone},smooth,domain=-85:85,samples=60]
 ({-sec(x)},{1.12*tan(x)});
\filldraw (axis cs:0,0) circle (1.5pt);
\filldraw [draw={\colorone}] (axis cs:-1,0) circle (1.5pt);
\filldraw [draw={\colorone}] (axis cs:1,0) circle (1.5pt);
\filldraw [draw={\colortwo}] (axis cs:1.5,0) circle (1.5pt);
\filldraw [draw={\colortwo}] (axis cs:-1.5,0) circle (1.5pt);
\draw (axis cs: 9,-6) node {\scriptsize $e = 1.5$};					
\end{axis}
\node [right] at (myplot.right of origin) {\scriptsize $x$};
\node [above] at (myplot.above origin) {\scriptsize $y$};
\end{tikzpicture}
  \\(a)&(b) \\
\begin{tikzpicture}
\begin{axis}[width=1.16\marginparwidth,tick label style={font=\scriptsize},
axis y line=middle,axis x line=middle,name=myplot,xtick={-10,-5,5,10},
minor x tick num=4,ytick={-10,-5,5,10},minor y tick num=4,
ymin=-11.9,ymax=11.9,xmin=-11.9,xmax=11.9,axis equal]
\addplot [thick, draw={\colorone},smooth,domain=-85:85,samples=60]
 ({sec(x)},{2.83*tan(x)});
\addplot [thick, draw={\colorone},smooth,domain=-85:85,samples=60]
 ({-sec(x)},{2.83*tan(x)});
\filldraw (axis cs:0,0) circle (1.5pt);
\filldraw [draw={\colorone}] (axis cs:-1,0) circle (1.5pt);
\filldraw [draw={\colorone}] (axis cs:1,0) circle (1.5pt);
\filldraw [draw={\colortwo}] (axis cs:3,0) circle (1.5pt);
\filldraw [draw={\colortwo}] (axis cs:-3,0) circle (1.5pt);
\draw (axis cs: 9,-6) node {\scriptsize $e = 3$};					
\end{axis}
\node [right] at (myplot.right of origin) {\scriptsize $x$};
\node [above] at (myplot.above origin) {\scriptsize $y$};
\end{tikzpicture}
  &
\begin{tikzpicture}
\begin{axis}[width=1.16\marginparwidth,tick label style={font=\scriptsize},
axis y line=middle,axis x line=middle,name=myplot,xtick={-10,-5,5,10},
minor x tick num=4,ytick={-10,-5,5,10},minor y tick num=4,
ymin=-11.9,ymax=11.9,xmin=-11.9,xmax=11.9,axis equal]
\addplot [thick, draw={\colorone},smooth,domain=-85:85,samples=60] ({sec(x)},{9.95*tan(x)});
\addplot [thick, draw={\colorone},smooth,domain=-85:85,samples=60] ({-sec(x)},{9.95*tan(x)});
\filldraw (axis cs:0,0) circle (1.5pt);
\filldraw [draw={\colorone}] (axis cs:-1,0) circle (1.5pt);
\filldraw [draw={\colorone}] (axis cs:1,0) circle (1.5pt);
\filldraw [draw={\colortwo}] (axis cs:10,0) circle (1.5pt);
\filldraw [draw={\colortwo}] (axis cs:-10,0) circle (1.5pt);
\draw (axis cs: 9,-6) node {\scriptsize $e = 10$};					
\end{axis}
\node [right] at (myplot.right of origin) {\scriptsize $x$};
\node [above] at (myplot.above origin) {\scriptsize $y$};
\end{tikzpicture}
  \\(c)&(d)
 \end{tabular}
 \caption{Understanding the eccentricity of a hyperbola.}
 \label{fig:hyperbola_ecc}
\end{lxfigure}

\subsection{Reflective Property}

Hyperbolas share a similar reflective property with ellipses. However, in the case of a hyperbola, a ray emanating from a focus that intersects the hyperbola reflects along a line containing the other focus, but moving \textit{away} from that focus. This is illustrated in \autoref{fig:hyperbola_reflect} (on the next page). Hyperbolic mirrors are commonly used in telescopes because of this reflective property. It is stated formally in the following theorem.

\theorem{thm:reflective_hyperbola}{Reflective Property of Hyperbolas}
{Let $P$ be a point on a hyperbola with foci $F_1$ and $F_2$. The tangent line to the hyperbola at $P$ makes equal angles with the following two lines:\index{hyperbola!reflective property}
\begin{enumerate}
	\item	The line through $F_1$ and $P$, and
	\item	The line through $F_2$ and $P$.
\end{enumerate}}

% todo prove the reflective property of hyperbolas

% todo prove that a parabola is the limiting case of a hyperbola

\subsection{Location Determination}

Determining the location of a known event has many practical uses (locating the epicenter of an earthquake, an airplane crash site, the position of the person speaking in a large room, etc.).

To determine the location of an earthquake's epicenter, seismologists use \textit{trilateration} (not to be confused with \textit{triangulation}). A seismograph allows one to determine how far away the epicenter was; using three separate readings, the location of the epicenter can be approximated.

A key to this method is knowing distances. What if this information is not available? Consider three microphones at positions $A$, $B$ and $C$ which all record a noise (a person's voice, an explosion, etc.) created at unknown location $D$. The microphone does not ``know'' when the sound was \textit{created}, only when the sound was \textit{detected}. How can the location be determined in such a situation?

\mtable{Illustrating the reflective property of a hyperbola.}{fig:hyperbola_reflect}{\begin{tikzpicture}
\begin{axis}[width=1.16\marginparwidth,tick label style={font=\scriptsize},
axis y line=none,axis x line=none,name=myplot,
ymin=-11.9,ymax=11.9,xmin=-5,xmax=5]
\addplot [thick, smooth,domain=-85:85,samples=60] ({sec(x)},{2.83*tan(x)});
\addplot [thick,smooth,domain=-85:85,samples=60] ({-sec(x)},{2.83*tan(x)});
\filldraw (axis cs:0,0) circle (1.5pt);
\filldraw [draw={\colorone}] (axis cs:-1,0) circle (1.5pt);
\filldraw [draw={\colorone}] (axis cs:1,0) circle (1.5pt);
\draw [draw={\colortwo},thick,->,>=latex] (axis cs:-3,0) -- (axis cs: 1.4,2.83);
\draw [draw={\colortwo},thick,->,>=latex] (axis cs:1.4,2.83)--(axis cs: -1.8,8.49);
\draw [dashed,draw={\colortwo},thick] (axis cs:1.4,2.83)--(axis cs: 3,0);
\filldraw [] (axis cs:1.4,2.83) circle (1.5pt);
\filldraw [] (axis cs:3,0) circle (1.5pt) node [below] {\scriptsize $F_2$};
\filldraw [] (axis cs:-3,0) circle (1.5pt) node [below] {\scriptsize $F_1$};
\end{axis}
\end{tikzpicture}}

If each location has a clock set to the same time, hyperbolas can be used to determine the location. Suppose the microphone at position $A$ records the sound at exactly 12:00, location $B$ records the time exactly 1 second later, and location $C$ records the noise exactly 2 seconds after that. We are interested in the \textit{difference} of times. Since the speed of sound is approximately 340 m/s, we can conclude quickly that the sound was created 340 meters closer to position $A$ than position $B$. If $A$ and $B$ are a known distance apart (as shown in \autoref{fig:hyperbola_locate} (a)), then we can determine a hyperbola on which $D$ must lie. 

The ``difference of distances'' is 340; this is also the distance between vertices of the hyperbola. So we know $2a=340$. Positions $A$ and $B$ lie on the foci, so $2c=1000$. From this we can find $b\approx 470$ and can sketch the hyperbola, given in part (b) of the figure. We only care about the side closest to $A$. (Why?)

We can also find the hyperbola defined by positions $B$ and $C$. In this case, $2a = 680$ as the sound traveled an extra 2 seconds to get to $C$. We still have $2c=1000$, centering this hyperbola at $(-500,500)$. We find $b\approx 367$. This hyperbola is sketched in part (c) of the figure. The intersection point of the two graphs is the location of the sound, at approximately $(188,-222.5)$.

\begin{lxfigure}
\flushinner{%
\begin{tabular}{c @{\hspace{1em}} c @{\hspace{1em}} c}
\begin{tikzpicture}
\begin{axis}[width=1.16\marginparwidth,tick label style={font=\scriptsize},
axis y line=middle,axis x line=middle,name=myplot,
ymin=-1100,ymax=1100,xmin=-1100,xmax=1100]
\filldraw (axis cs:500,0) circle (1.5pt) node [above] {\scriptsize $A$};
\filldraw (axis cs:-500,0) circle (1.5pt) node [above] {\scriptsize $B$};
\filldraw (axis cs:-500,1000) circle (1.5pt) node [left] {\scriptsize $C$};
\end{axis}
\end{tikzpicture}
 &
\begin{tikzpicture}
\begin{axis}[width=1.16\marginparwidth,tick label style={font=\scriptsize},
axis y line=middle,axis x line=middle,name=myplot,
ymin=-1100,ymax=1100,xmin=-1100,xmax=1100]
\addplot [thick, smooth,domain=-85:85,samples=60] ({170*sec(x)},{470*tan(x)});
\addplot [dashed,thick,smooth,domain=-85:85,samples=60] ({-170*sec(x)},{470*tan(x)});
\filldraw (axis cs:500,0) circle (1.5pt) node [above] {\scriptsize $A$};
\filldraw (axis cs:-500,0) circle (1.5pt) node [above] {\scriptsize $B$};
\filldraw (axis cs:-500,1000) circle (1.5pt) node [left] {\scriptsize $C$};
\end{axis}
\end{tikzpicture}
 &
\begin{tikzpicture}
\begin{axis}[width=1.16\marginparwidth,tick label style={font=\scriptsize},
axis y line=middle,axis x line=middle,name=myplot,
ymin=-1100,ymax=1100,xmin=-1100,xmax=1100]
\addplot [thick, smooth,domain=-85:85,samples=60] ({170*sec(x)},{470*tan(x)});
\addplot [dashed,thick, smooth,domain=-85:85,samples=60]
 ({367*tan(x)-500},{340*sec(x)+500});
\addplot [thick, smooth,domain=-85:85,samples=60]
 ({367*tan(x)-500},{-340*sec(x)+500});
\filldraw (axis cs:500,0) circle (1.5pt) node [above] {\scriptsize $A$};
\filldraw (axis cs:-500,0) circle (1.5pt) node [above] {\scriptsize $B$};
\filldraw (axis cs:-500,1000) circle (1.5pt) node [left] {\scriptsize $C$};
\filldraw (axis cs:188,-222) circle (1.5pt) node [below left] {\scriptsize $D$};
\end{axis}
\end{tikzpicture}
\\(a) & (b) & (c)
\end{tabular}}
\caption{Using hyperbolas in location detection.}%
\label{fig:hyperbola_locate}
\end{lxfigure}
}

This chapter explores curves in the plane, in particular curves that cannot be described by functions of the form $y=f(x)$. In this section, we learned of ellipses and hyperbolas that are defined implicitly, not explicitly. In the following sections, we will learn completely new ways of describing curves in the plane, using \emph{parametric equations} and \emph{polar coordinates}, then study these curves using calculus techniques.

\printexercises{exercises/09_01_exercises}

\section{Parametric Equations}\label{sec:param_eqs}

We are familiar with sketching  shapes, such as parabolas, by following this basic procedure:

\[
\begin{gathered}\text{Choose}\\x\end{gathered}
\longrightarrow
\begin{gathered}
\text{Use a function}\\f\text{ to find }y\\\bigl(y=f(x)\bigr)
\end{gathered}
\longrightarrow
\begin{gathered}\text{Plot point}\\(x,y)\end{gathered}
\]

In the rectangular coordinate system, the \textbf{rectangular equation} $y=f(x)$ works well for some shapes like a parabola with a vertical axis of symmetry, but in Precalculus and the review of conic sections in \autoref{sec:conic_sections}, we encountered several shapes that could not be sketched in this manner. (To plot an ellipse using the above procedure, we need to plot the ``top'' and ``bottom'' separately.)\index{curve!rectangular equation}

In this section we introduce a new sketching procedure:

\ifbool{latexml}{
\[
\begin{gathered}\text{Choose}\\t\end{gathered}\quad
\begin{gathered}
\rotatebox{-20}{\scalebox{2}{$\nearrow$}}\\
\rotatebox{20}{\scalebox{2}{$\searrow$}}
\end{gathered}\quad
{\addtolength{\jot}{-.5ex}
\begin{gathered}
\text{Use a function}\\f\text{ to find }x\\\bigl(x=f(t)\bigr)\\~\\
\text{Use a function}\\g\text{ to find }y\\\bigl(y=g(t)\bigr)
\end{gathered}
}\quad
\begin{gathered}
\rotatebox{20}{\scalebox{2}{$\searrow$}}\\
\rotatebox{-20}{\scalebox{2}{$\nearrow$}}
\end{gathered}\quad
\begin{gathered}\text{Plot point}\\(x,y)\end{gathered}
\]
}{
\begin{center}
\pdftooltip{\begin{tikzpicture}[>=latex]
\draw (0,0) node (A) [align=center] {Choose \\$t$} 
      (3,1) node[align=center] (B1) {Use a function\\ $f$ to find $x$\\$\bigl(x=f(t)\bigr)$}
			(3,-1) node[align=center] (B2) {Use a function\\ $g$ to find $y$\\$\bigl(y=g(t)\bigr)$}
			(6.25,0) node [align=center] (C) {Plot point \\ $(x,y)$};
\draw [->](A) --(B1);
\draw [->](A) --(B2);
\draw [->](B1) -- (C);
\draw [->](B2) -- (C);
\end{tikzpicture}}{ALT-TEXT-TO-BE-DETERMINED}
\end{center}
}

Here, $x$ and $y$ are found separately but then plotted together. This leads us to a definition.

\begin{definition}[Parametric Equations and Curves]\label{def:param_eq}%%
Let $f$ and $g$ be continuous functions on an interval $I$. The \textbf{graph} of the \textbf{parametric equations} $x=f(t)$ and $y=g(t)$ is the set of all points $\bigl(x,y\bigr) = \bigl(f(t),g(t)\bigr)$ in the Cartesian plane, as the \textbf{parameter} $t$ varies over $I$. A \textbf{curve} is a graph along with the parametric equations that define it.
\index{curve!parametrically defined}\index{parametric equations!definition}
\end{definition}

\youtubeVideo{9kKZHQtYP7g}{Parametric Equations --- Some basic questions}

This is a formal definition of the word \emph{curve}. When a curve lies in a plane (such as the Cartesian plane), it is often referred to as a \textbf{plane curve}. Examples will help us understand the concepts introduced in the definition.

\mtable[-.5in]{A table of values of the parametric equations in \autoref{ex_pareq1} along with a sketch of their graph.}{fig:pareq1}{%
\tagpdfsetup{table/header-rows={1}}
\begin{tabular}{r @{\hspace{2em}} rr}
$t$ & $x$ & $y$ \\ \midrule
$-2$ & \phantom{$-$}4 & $-1$ \\
$-1$ & 1 & 0 \\
0 & 0 & 1 \\
1 & 1 & 2 \\
2 & 4 & 3
\end{tabular}\\(a)\\
\pdftooltip{\begin{tikzpicture}
\begin{axis}[width=1.16\marginparwidth,tick label style={font=\scriptsize},
axis y line=middle,axis x line=middle,name=myplot,
ymin=-2.2,ymax=4.2,xmin=-.9,xmax=5.2]
\addplot [thick,draw={\colorone}, smooth,domain=-2:2,] ({x^2},{x+1});
\draw [->,>=latex] (axis cs:3.9204,2.98) -- (axis cs:3.96,2.99);
\draw [->,>=latex] (axis cs:2.25,-.5) -- (axis cs:2.22,-.49);
\filldraw (axis cs:4,-1) circle (1.5pt) node [below] {\scriptsize $t=-2$};
\filldraw (axis cs:1,0) circle (1.5pt) node [shift={(0pt,-10pt)}]
 {\scriptsize $t=-1$};
\filldraw (axis cs:0,1) circle (1.5pt) node [left] {\scriptsize $t=0$};
\filldraw (axis cs:1,2) circle (1.5pt) node [above] {\scriptsize $t=1$};
\filldraw (axis cs:4,3) circle (1.5pt) node [above ] {\scriptsize $t=2$};
\end{axis}
\node [right] at (myplot.right of origin) {\scriptsize $x$};
\node [above] at (myplot.above origin) {\scriptsize $y$};
\end{tikzpicture}}{ALT-TEXT-TO-BE-DETERMINED}
\\(b)}

\begin{example}[Plotting parametric functions]\label{ex_pareq1}%%
Plot the graph of the parametric equations $x=t^2$, $y=t+1$ for $t$ in $[-2,2]$.
\solution
We plot the graphs of parametric equations in much the same manner as we plotted graphs of functions like $y=f(x)$: we make a table of values, plot points, then connect these points with a ``reasonable'' looking curve. \autoref{fig:pareq1}(a) shows such a table of values; note how we have 3 columns.

The points $(x,y)$ from the table are plotted in \autoref{fig:pareq1}(b). The points have been connected with a smooth curve. Each point has been labeled with its corresponding $t$-value. These values, along with the two arrows along the curve, are used to indicate the \textbf{orientation} of the graph. This information describes the \textbf{path} of a particle traveling along the curve.
\end{example}

We often use the letter $t$ as the parameter as we often regard $t$ as representing \emph{time}. Certainly there are many contexts in which the parameter is not time, but it can be helpful to think in terms of time as one makes sense of parametric plots and their orientation (for instance, ``At time $t=0$ the position is $(1,2)$ and at time $t=3$ the position is $(5,1)$.'').

\begin{example}[Plotting parametric functions]\label{ex_pareq2}%
%
\mtable{A table of values of the parametric equations in \autoref{ex_pareq2} along with a sketch of their graph.}{fig:pareq2}{%
\tagpdfsetup{table/header-rows={1}}
\begin{tabular}{c @{\hspace{2em}} cc}
$t$ & $x$ & $y$ \\ \hline
0 & 1 & 2 \\
$\pi/4$ & $1/2$ & $1+\sqrt{2}/2$ \\
$\pi/2$ & 0 & 1\\
$3\pi/4$ & $1/2$ & $1-\sqrt{2}/2$ \\
$\pi$ & 1 & 0
\end{tabular}
\\(a)\\
\pdftooltip{\begin{tikzpicture}
\begin{axis}[width=1.16\marginparwidth,tick label style={font=\scriptsize},
axis y line=middle,axis x line=middle,name=myplot,
ymin=-.1,ymax=2.1,xmin=-.1,xmax=1.6]
\addplot [thick,draw={\colorone}, smooth,domain=0:180,] ({(cos(x))^2},{cos(x)+1});
\draw [->,>=latex] (axis cs:0.998271, 0.00086485) -- (axis cs:0.999534, 0.000233112);
\draw [->,>=latex] (axis cs:0.770151, 1.87758) -- (axis cs:0.75311, 1.86782);
\filldraw (axis cs:1,2) circle (1.5pt) node [right] {\scriptsize $t=0$};
\filldraw (axis cs:.5,1.71) circle (1.5pt) node [below right]
 {\scriptsize $t=\pi/4$};
\filldraw (axis cs:0,1) circle (1.5pt) node [right] {\scriptsize $t=\pi/2$};
\filldraw (axis cs:.5,.292) circle (1.5pt) node [above right]
 {\scriptsize $t=3\pi/4$};
\filldraw (axis cs:1,0) circle (1.5pt) node [above right] {\scriptsize $t=\pi$};
\end{axis}
\node [right] at (myplot.right of origin) {\scriptsize $x$};
\node [above] at (myplot.above origin) {\scriptsize $y$};
\end{tikzpicture}}{ALT-TEXT-TO-BE-DETERMINED}
\\(b)}
%
Sketch the graph of the parametric equations $x=\cos^2t$, $y=\cos t+1$ for $t$ in $[0,\pi]$.
\solution
We again start by making a table of values in \autoref{fig:pareq2}(a), then plot the points $(x,y)$ on the Cartesian plane in \autoref{fig:pareq2}(b).

The curves in Examples \ref{ex_pareq1} and \ref{ex_pareq2} are portions of the same parabola $(y-1)^2+x=1$. While the \emph{parabola} is the same, the \emph{curves} are different. In \autoref{ex_pareq1}, if we let $t$ vary over all real numbers, we'd obtain the entire parabola. In this example, letting $t$ vary over all real numbers would still produce the same graph; this portion of the parabola would be traced, and re-traced, infinitely often. The orientation shown in \autoref{fig:pareq2} shows the orientation on $[0,\pi]$, but this orientation is reversed on $[\pi,2\pi]$.
\end{example}

\subsection{Converting between rectangular and parametric equations}

It is sometimes useful to transform rectangular form equations (i.e., $y=f(x)$) into parametric form equations, and vice-versa. Converting from rectangular to parametric can be very simple: given $y=f(x)$, the parametric equations $x=t$, $y=f(t)$ produce the same graph. As an example, given $y=x^2-x-6$, the parametric equations $x=t, y=t^2-t-6$ produce the same parabola. However, other parameterizations can be used. The following example demonstrates one possible alternative.

\mtable{The equation $f(x)=x^2-x-6$ with different parameterizations.}{fig_parab_param}{\pdftooltip{\begin{tikzpicture}
 \begin{axis}[width=1.16\marginparwidth,tick label style={font=\scriptsize},
   axis y line=middle,axis x line=middle,name=myplot,
   ymin=-7.2,ymax=5.5,xmin=-3.75,xmax=4.5]
  \addplot [thick,draw={\colorone}, smooth,domain=-1.5:4.5,] ({x-1},{x^2-3*x-4});
  \draw [->,>=latex] (axis cs:-0.82, -4.51) -- (axis cs:-0.813, -4.53);
  \draw [->,>=latex] (axis cs:2.42, -2.57) -- (axis cs:2.55, -2.13);
  \filldraw (axis cs:-2,0) circle (1.5pt) node [above left] {\scriptsize $t=-1$};
  \filldraw (axis cs:0,-6) circle (1.5pt) node [below left] {\scriptsize $t=1$};
  \filldraw (axis cs:2,-4) circle (1.5pt) node [right] {\scriptsize $t=3$};
  \filldraw (axis cs:3,0) circle (1.5pt) node [above right] {\scriptsize $t=4$};
 \end{axis}
 \node [right] at (myplot.right of origin) {\scriptsize $x$};
 \node [above] at (myplot.above origin) {\scriptsize $y$};
\end{tikzpicture}}{ALT-TEXT-TO-BE-DETERMINED}
\\(a)\\
\pdftooltip{\begin{tikzpicture}
 \begin{axis}[width=1.16\marginparwidth,tick label style={font=\scriptsize},
   axis y line=middle,axis x line=middle,name=myplot,
   ymin=-7.2,ymax=5.5,xmin=-3.75,xmax=4.5]
  \addplot [thick,draw={\colorone}, smooth,domain=-.5:5.5,] ({3-x},{x^2-5*x});
  \draw [->,>=latex] (axis cs:-1.20, -3.43) -- (axis cs:-1.21, -3.4);
  \draw [->,>=latex] (axis cs:2.55, -2.13) -- (axis cs:2.42, -2.57);
  \filldraw (axis cs:-2,0) circle (1.5pt) node [above left] {\scriptsize $t=5$};
  \filldraw (axis cs:0,-6) circle (1.5pt) node [below left] {\scriptsize $t=3$};
  \filldraw (axis cs:2,-4) circle (1.5pt) node [right] {\scriptsize $t=1$};
  \filldraw (axis cs:3,0) circle (1.5pt) node [above right] {\scriptsize $t=0$};
 \end{axis}
 \node [right] at (myplot.right of origin) {\scriptsize $x$};
 \node [above] at (myplot.above origin) {\scriptsize $y$};
\end{tikzpicture}}{ALT-TEXT-TO-BE-DETERMINED}
\\(b)\\
\pdftooltip{\begin{tikzpicture}
 \begin{axis}[width=1.16\marginparwidth,tick label style={font=\scriptsize},
   axis y line=middle,axis x line=middle,name=myplot,
   ymin=-7.2,ymax=5.5,xmin=-3.75,xmax=4.5]
  \addplot [thick,draw={\colorone}, smooth,domain=-5.75:5.75,] ({(x+1)/2},{.25*x^2-6.25});
  \draw [->,>=latex] (axis cs:-0.82, -4.51) -- (axis cs:-0.813, -4.53);
  \draw [->,>=latex] (axis cs:2.42, -2.57) -- (axis cs:2.55, -2.13);
  \filldraw (axis cs:-2,0) circle (1.5pt) node [above left] {\scriptsize $t=-5$};
  \filldraw (axis cs:0,-6) circle (1.5pt) node [below left] {\scriptsize $t=-1$};
  \filldraw (axis cs:2,-4) circle (1.5pt) node [right] {\scriptsize $t=3$};
  \filldraw (axis cs:3,0) circle (1.5pt) node [above right] {\scriptsize $t=5$};
 \end{axis}
 \node [right] at (myplot.right of origin) {\scriptsize $x$};
 \node [above] at (myplot.above origin) {\scriptsize $y$};
\end{tikzpicture}}{ALT-TEXT-TO-BE-DETERMINED}
\\(c)}

\begin{example}[Converting from rectangular to parametric]\label{ex_pareq5}%
Find parametric equations for $f(x)=x^2-x-6$.
\solution
Solution 1: For any choice for $x$ we can determine the corresponding $y$ by substitution. If we choose $x=t-1$ then $y=(t-1)^2-(t-1)-6=t^2-3t-4$. Thus $f(x)$ can be represented by the parametric equations
\[x=t-1 \qquad y=t^2-3t-4.\]
On the graph of this parameterization (\autoref{fig_parab_param}(a)) the points have been labeled with the corresponding $t$-values and arrows indicate the path of a particle traveling on this curve. The particle would move from the upper left, down to the vertex at $(.5,-6.25)$ and then up to the right.

Solution 2: If we choose $x=3-t$ then $y=(3-t)^2-(3-t)-6=t^2-5t$. Thus $f(x)$ can also be represented by the parametric equations
\[x=3-t \qquad y=t^2-5t.\]
On the graph of this parameterization (\autoref{fig_parab_param}(b)) the points have been labeled with the corresponding $t-$values and arrows indicate the path of a particle traveling on this curve. The particle would move down from the upper right, to the vertex at $(.5,-6.25)$ and then up to the left.

Solution 3: We can also parameterize any $y=f(x)$ by setting $t=\frac{\dd y}{\dd x}$. That is, $t=a$ corresponds to the point on the graph whose tangent line has a slope $a$. Computing $\frac{\dd y}{\dd x}$, $\fp(x) = 2x-1$ we set $t=2x-1$. Solving for $x$ we find $x=\frac{t+1}{2}$ and by substitution $y=\frac{1}{4}t^2 - \frac{25}{4}$. Thus $f(x)$ can be represented by the parametric equations
\[x=\frac{t+1}{2} \qquad y=\frac{1}{4}t^2 - \frac{25}{4}.\]

The graph of this parameterization is shown in \autoref{fig_parab_param}(c). To find the point where the tangent line has a slope of $0$, we set $t=0$. This gives us the point $(.5, -6.25)$ which is the vertex of $f(x)$.
\end{example}

\begin{example}[Converting from rectangular to parametric]\label{ex_par_circle}%
Find parametric equations for the circle $x^2+y^2=4$.
\solution
We will present three different approaches:

\textbf{Solution 1:} Consider the equivalent equation $\biggl(\dfrac{x}{2}\biggr)^2+ \biggl(\dfrac{y}{2}\biggr)^2=1$ and the Pythagorean Identity, $\sin^2t+\cos^2 t=1$. We set $\cos t=\frac{x}{2}$ and $\sin t=\frac{y}{2}$, which gives $x=2\cos t$ and $y=2\sin t$. To trace the circle once, we must have $0\leq t \leq 2\pi$. Note that when $t=0$ a particle tracing the curve would be at the point $(2,0)$ and would move in a counterclockwise direction.

\textbf{Solution 2:} Another parameterization of the same circle would be $x=2\sin t$ and $y=2\cos t$ for $0\leq t \leq 2\pi$. When $t=0$ a particle would be at the point $(0,2)$ and would move in a clockwise direction.

\textbf{Solution 3:} We could let $x=-2\sin t$ and $y=2\cos t$ for $0\leq t \leq 2\pi$. Also note that we could use $x=2\cos 2t$ and $y=2\sin 2t$ for $0\leq t \leq \pi$.
\end{example}

As we have shown in the previous examples, there are many different ways to parameterize any given curve. We sometimes choose the parameter to accurately model physical behavior.

\begin{example}[Converting from rectangular to parametric]\label{ex_para_ellipse}%
Find a parameterization that traces the ellipse $\dfrac{(x-2)^2}{9}+\dfrac{(y+3)^2}{4}=1$ starting at the point $(-1,-3)$ in a clockwise direction.
\solution
Applying the Pythagorean Identity, $\cos^2t+\sin^2t=1$, we set $\cos^2 t=\dfrac{(x-2)^2}{9}$ and $\sin^2 t=\dfrac{(y+3)^2}{4}$. Solving these equations for $x$ and $y$ we set $x=-3\cos t+2$ and $y=2\sin t-3$  for $0\leq t\leq 2\pi$.
\end{example}

\begin{example}[Converting from rectangular to parametric]\label{ex_para_hyper}%
Find a parameterization for the hyperbola $\dfrac{(x-2)^2}{9}-\dfrac{(y-3)^2}{4}=1$.
\solution
We use the form of the Pythagorean Identity $\sec^2t-\tan^2t=1$. We let  $\ds \sec^2 t=\frac{(x-2)^2}{9}$ and $\tan^2 t=\dfrac{(y-3)^2}{4}$. Solving these equations for $x$ and $y$ we have $x=3\sec t +2$ and $y=2\tan t +3$ for $0\leq t\leq 2\pi$ and $t\neq\frac\pi2,\frac{3\pi}2$.
\end{example}

\begin{example}[Converting from rectangular to parametric]\label{ex_pareq6}%
An object is fired from a height of 0ft and lands 6 seconds later, 192ft away. Assuming ideal projectile motion, the height, in feet, of the object can be described by $h(x) = -x^2/64+3x$, where $x$ is the distance in feet from the initial location. (Thus $h(0) = h(192) = 0$ft.) Find parametric equations $x=f(t)$, $y=g(t)$ for the path of the projectile where $x$ is the horizontal distance the object has traveled at time $t$ (in seconds) and $y$ is the height at time $t$.
\solution
Physics tells us that the horizontal motion of the projectile is linear; that is, the horizontal speed of the projectile is constant. Since the object travels 192ft in 6s, we deduce that the object is moving horizontally at a rate of 32ft/s, giving the equation $x=32t$. As $y=-x^2/64+3x$, we find $y= -16t^2+96t$. We can quickly verify that $y\primeskip''=-32$ft/s$^2$, the acceleration due to gravity, and that the projectile reaches its maximum at $t=3$, halfway along its path.

\mtable{Graphing projectile motion in \autoref{ex_pareq6}.}{fig:pareq6}{\pdftooltip{\begin{tikzpicture}
\begin{axis}[width=\marginparwidth,tick label style={font=\scriptsize},
axis y line=middle,axis x line=middle,name=myplot,
ymin=0,ymax=160,xmin=0,xmax=210]
\addplot [thick, draw={\colorone},smooth,domain=0:6] ({32*x},{-16*x^2+96*x});
\draw [->,>=latex] (axis cs:32,80) -- (axis cs:35.2, 86.24);
\filldraw (axis cs:64, 128) circle (1.5pt) node [above left] {\scriptsize $t=2$};
%\draw (axis cs: 100,50) node [align=left]
% {\scriptsize $x=32t$\\ \scriptsize $y=-16t^2+96t$};
\draw (axis cs: 100,50) node [align=left]
 {\scriptsize $\begin{aligned}x&=32t\\y&=-16t^2+96t\end{aligned}$};
\end{axis}
\node [right] at (myplot.right of origin) {\scriptsize $x$};
\node [above] at (myplot.above origin) {\scriptsize $y$};
\end{tikzpicture}}{ALT-TEXT-TO-BE-DETERMINED}}

These parametric equations make certain determinations about the object's location easy: 2 seconds into the flight the object is at the point $\bigl(x(2),y(2)\bigr) = \bigl(64,128\bigr)$. That is, it has traveled horizontally 64ft and is at a height of 128ft, as shown in \autoref{fig:pareq6}.
\end{example}

It is  sometimes necessary to convert given parametric equations into rectangular form. This can be decidedly more difficult, as some ``simple'' looking parametric equations can have very ``complicated'' rectangular equations. This conversion is often referred to as ``eliminating the parameter,'' as we are looking for a relationship between $x$ and $y$ that does not involve the parameter $t$.

\begin{example}[Eliminating the parameter]\label{ex_pareq7}%
Find a rectangular equation for the curve described by
\[x= \frac{1}{t^2+1}\quad \text{and}\quad y=\frac{t^2}{t^2+1}.\]
\solution
There is not a set way to eliminate a parameter. One method is to solve for $t$ in one equation and then substitute that value in the second. We use that technique here, then show a second, simpler method.

Starting with $x= 1/(t^2+1)$, solve for $t$: $ t = \pm\sqrt{1/x-1}$. Substitute this value for $t$ in the equation for $y$:
{\allowdisplaybreaks
\begin{align*}
 y &= \frac{t^2}{t^2 +1} \\
		&= \frac{1/x-1}{1/x-1+1} \\
		&= \frac{1/x - 1}{1/x} \\
		&= \left(\frac1x-1\right)\cdot x \\
		&= 1-x.
\end{align*}}

\mtable{Graphing parametric and rectangular equations for a graph in \autoref{ex_pareq7}.}{fig:pareq7}{\pdftooltip{\begin{tikzpicture}
\begin{axis}[width=1.16\marginparwidth,tick label style={font=\scriptsize},
axis y line=middle,axis x line=middle,name=myplot,
ymin=-1.1,ymax=2.1,xmin=-2.1,xmax=2.1,axis equal]
\addplot [ draw={\colorone},smooth,domain=-2:2] {1-x};
\addplot [draw={\colorone},ultra thick, smooth,domain=0:1] {1-x};
\draw [fill=white] (axis cs:0,1) circle (1.5pt);
\filldraw [fill=black] (axis cs:1,0) circle (1.5pt);
\draw (axis cs:1,0) node [align=left,shift={(5pt,40pt)}]
 {\scriptsize $x=\dfrac{1}{t^2+1}$};
\draw (axis cs:1,0) node [align=left,shift={(5pt,20pt)}]
 {\scriptsize $y=\dfrac{t^2}{t^2+1}$};
%\filldraw [fill=black] (axis cs:1,0) circle (1.5pt)
% node [align=left,shift={(5pt,30pt)}]
% {\scriptsize $x=\dfrac{1}{t^2+1}$\\[2pt] \scriptsize $y=\dfrac{t^2}{t^2+1}$};
\draw (axis cs: -1,1.4) node {\scriptsize $y=1-x$};
\end{axis}
\node [right] at (myplot.right of origin) {\scriptsize $x$};
\node [above] at (myplot.above origin) {\scriptsize $y$};
\end{tikzpicture}}{ALT-TEXT-TO-BE-DETERMINED}}

Thus $y=1-x$. One may have recognized this earlier by manipulating the equation for $y$:
\[y = \frac{t^2}{t^2+1} = 1-\frac{1}{t^2+1} = 1-x.\]
This is a shortcut that is very specific to this problem; sometimes shortcuts exist and are worth looking for.

We should be careful to limit the domain of the function $y=1-x$. The parametric equations limit $x$ to values in $(0,1]$, thus to produce the same graph we should limit the domain of $y=1-x$ to the same. 

The graphs of these functions are given in \autoref{fig:pareq7}. The portion of the graph defined by the parametric equations is given in a thick line; the graph defined by $y=1-x$ with unrestricted domain is given in a thin line.
\end{example}

\begin{example}[Eliminating the parameter]\label{ex_pareq8}%
Eliminate the parameter in $x=4\cos t+3$, $y= 2\sin t+1$
\solution
We should not try to solve for $t$ in this situation as the resulting algebra/trig would be messy. Rather, we solve for $\cos t$ and $\sin t$ in each equation, respectively. This gives
\[\cos t = \frac{x-3}{4} \quad \text{and}\quad \sin t=\frac{y-1}{2}.\]
The Pythagorean Theorem gives $\cos^2t+\sin^2t=1$, so:
%
\mtable{Graphing the parametric equations $x=4\cos t+3$, $y=2\sin t+1$ in \autoref{ex_pareq8}.}{fig:pareq8}{\pdftooltip{\begin{tikzpicture}
\begin{axis}[width=1.16\marginparwidth,tick label style={font=\scriptsize},
axis y line=middle,axis x line=middle,name=myplot,minor x tick num=1,
minor y tick num=1,ymin=-3.1,ymax=5.1,xmin=-1.1,xmax=8.5]
\addplot [thick,draw={\colorone},smooth,domain=0:360,samples=60]
 ({4*cos(x)+3},{2*sin(x)+1});
\end{axis}
\node [right] at (myplot.right of origin) {\scriptsize $x$};
\node [above] at (myplot.above origin) {\scriptsize $y$};
\end{tikzpicture}}{ALT-TEXT-TO-BE-DETERMINED}}
%
\begin{align*}
\cos^2t+\sin^2t &=1 \\
\left(\frac{x-3}{4}\right)^2 +\left(\frac{y-1}{2}\right)^2 &=1\\
\frac{(x-3)^2}{16}+\frac{(y-1)^2}{4} &=1
\end{align*}
This final equation should look familiar --- it is the equation of an ellipse. \autoref{fig:pareq8} plots the parametric equations, demonstrating that the graph is indeed of an ellipse with a horizontal major axis and center at $(3,1)$.
\end{example}

%The Pythagorean Theorem can also be used to identify parametric equations for hyperbolas. We give the parametric equations for ellipses and hyperbolas in the following Key Idea.

%\begin{keyidea}[Parametric Equations for Ellipses]\label{idea:par_ellipse}%
%The parametric equations 
%\[x=a\cos t+h, \quad y=b\sin t+k\]
%define an ellipse with horizontal axis of length $2a$ and vertical axis of length $2b$, centered at $(h,k)$.
%\index{ellipse!parametric equations}%\\
%\end{keyidea}
%
%\begin{keyidea}[Parametric Equations for Hyperbolas]\label{idea:par_hyperbola}%
%The parametric equations
%\[x= a\tan t+h,\quad y=\pm b\sec t+k\]
%define a hyperbola with vertical transverse axis centered at $(h,k)$, and 
%\[x=\pm a\sec t+h, \quad y=b\tan t + k\]
%defines a hyperbola with horizontal transverse axis. Each has asymptotes at $y=\pm b/a(x-h)+k$.
%\index{hyperbola!parametric equations}
%\end{keyidea}

%\begin{keyidea}[Parametric Equations of Ellipses and Hyperbolas]\label{idea:par_ellipse_hyperbola}%
%\begin{itemize}
%	\item The parametric equations 
%\[x=a\cos t+h, \quad y=b\sin t+k\]
%define an ellipse with horizontal axis of length $2a$ and vertical axis of length $2b$, centered at $(h,k)$.
%\index{ellipse!parametric equations}
%	\item The parametric equations
%\[x= a\tan t+h,\quad y=\pm b\sec t+k\]
%define a hyperbola with vertical transverse axis centered at $(h,k)$, and 
%\[x=\pm a\sec t+h, \quad y=b\tan t + k\]
%defines a hyperbola with horizontal transverse axis. Each has asymptotes at $y=\pm b/a(x-h)+k$.
%\index{hyperbola!parametric equations}
%\end{itemize}
%\end{keyidea}
%%%%% End with this??
%%%%%

\subsection{Graphs of Parametric Equations}

These examples begin to illustrate the powerful nature of parametric equations. Their graphs are far more diverse than the graphs of functions produced by ``$y=f(x)$'' functions.

%\paragraph{Technology Note:} Most graphing utilities can graph functions given in parametric form. Often the word ``parametric'' is abbreviated as ``PAR'' or ``PARAM'' in the  options. The user usually needs to determine the graphing window (i.e, the minimum and maximum $x$- and $y$-values), along with the values of $t$ that are to be plotted. The user is often prompted to give a $t$ minimum, a $t$ maximum, and a ``$t$-step'' or ``$\Delta t$.'' Graphing utilities effectively plot parametric functions just as we've shown here: they plots lots of points. A smaller $t$-step plots more points, making for a smoother graph (but may take longer). In \autoref{fig:pareq1}, the $t$-step is 1; in \autoref{fig:pareq2}, the $t$-step is $\pi/4$.\\

One nice feature of parametric equations is that their graphs are easy to shift. While this is not too difficult in the ``$y=f(x)$'' context, the resulting function can look rather messy. (Plus, to shift to the right by two, we replace $x$ with $x-2$, which is counterintuitive.) The following example demonstrates this.

\begin{example}[Shifting the graph of parametric functions]\label{ex_pareq3}%
Sketch the graph of the parametric equations $x=t^2+t$, $y=t^2-t$.  Find new parametric equations that shift this graph to the right 3 units and down 2.
%
\mtable{Illustrating how to shift graphs in \autoref{ex_pareq3}.}{fig:pareq3}{%
\pdftooltip{\begin{tikzpicture}
\begin{axis}[width=1.16\marginparwidth,tick label style={font=\scriptsize},
axis y line=middle,axis x line=middle,name=myplot,xtick={2,4,6,8,10},
ymin=-2.6,ymax=6.5,xmin=-.6,xmax=10.5]
\addplot [thick,draw={\colorone}, smooth,domain=-3:3,] ({x^2+x},{x^2-x});
\draw [->,>=latex] (axis cs:0.96, 4.16) -- (axis cs:0.9381, 4.1181);
\draw [->,>=latex] (axis cs:6,2) -- (axis cs:6.05,2.03);
\draw (axis cs: 4,5) node [align=left]
 {\scriptsize $\begin{aligned}x&=t^2+t\\y&=t^2-t\end{aligned}$};
% {\scriptsize $x=t^2+t$\\ \scriptsize $y=t^2-t$};
\end{axis}
\node [right] at (myplot.right of origin) {\scriptsize $x$};
\node [above] at (myplot.above origin) {\scriptsize $y$};
\end{tikzpicture}}{ALT-TEXT-TO-BE-DETERMINED}
\\(a) \\
\pdftooltip{\begin{tikzpicture}
\begin{axis}[width=1.16\marginparwidth,tick label style={font=\scriptsize},
axis y line=middle,axis x line=middle,name=myplot,xtick={2,4,6,8,10},
ymin=-2.6,ymax=6.5,xmin=-.6,xmax=10.5]
\addplot [thick,draw={\colorone}, smooth,domain=-3:3,] ({x^2+x+3},{x^2-x-2});
\draw [->,>=latex] (axis cs:3.96, 2.16) -- (axis cs:3.9381, 2.1181);
\draw [->,>=latex] (axis cs:9,0) -- (axis cs:9.05,0.03);
\draw (axis cs: 2.5,5) node [align=left]
 {\scriptsize $\begin{aligned}x&=t^2+t+3\\y&=t^2-t-2\end{aligned}$};
% {\scriptsize $x=t^2+t+3$\\ \scriptsize $y=t^2-t-2$};
\end{axis}
\node [right] at (myplot.right of origin) {\scriptsize $x$};
\node [above] at (myplot.above origin) {\scriptsize $y$};
\end{tikzpicture}}{ALT-TEXT-TO-BE-DETERMINED}
\\(b)}
%
\solution
We see the graph in \autoref{fig:pareq3}(a). It is a parabola with an axis of symmetry along the line $y=x$; the vertex is at $(0,0)$. It should be noted that finding the vertex is not a trivial matter and not something you will be asked to do in this text.

In order to shift the graph to the right 3 units, we need to increase the $x$-value by 3 for every point. The straightforward way to accomplish this is simply to add 3 to the function defining $x$: $x = t^2+t+3$. To shift the graph down by 2 units, we wish to decrease each $y$-value by 2, so we subtract 2 from the function defining $y$: $y = t^2-t-2$. Thus our parametric equations for the shifted graph are $x=t^2+t+3$, $y=t^2-t-2$. This is graphed in \autoref{fig:pareq3} (b). Notice how the vertex is now at $(3,-2)$.
\end{example}

Because the $x$- and $y$-values of a graph are determined independently, the graphs of parametric functions often possess features not seen on ``$y=f(x)$'' type graphs. The next example demonstrates how such graphs can arrive at the same point more than once.

\begin{example}[Graphs that cross themselves]\label{ex_pareq4}%
Plot the parametric functions $x=t^3-5t^2+3t+11$ and $y=t^2-2t+3$ and determine the $t$-values where the graph crosses itself.
\solution
Using the methods developed in this section, we again plot points and graph the parametric equations as shown in \autoref{fig:pareq4}. It appears that the graph crosses itself at the point $(2,6)$, but we'll need to analytically determine this.

We are looking for two different values, say, $s$ and $t$, where $x(s) = x(t)$ and $y(s) = y(t)$. That is, the $x$-values are the same precisely when the $y$-values are the same. This gives us a system of 2 equations with 2 unknowns:
%
\mtable{A graph of the parametric equations from \autoref{ex_pareq4}.}{fig:pareq4}{\pdftooltip{\begin{tikzpicture}
\begin{axis}[width=1.16\marginparwidth,tick label style={font=\scriptsize},
axis y line=middle,axis x line=middle,name=myplot,minor x tick num=4,
minor y tick num=4,ymin=-.9,ymax=16,xmin=-6,xmax=16]
\addplot [thick, draw={\colorone},smooth,domain=-2:5,samples=50]
 ({x^3-5*x^2+3*x+11},{x^2-2*x+3});
\draw [->,>=latex] (axis cs:-4.62287, 7.5225) -- (axis cs:-4.4041, 7.4756);
\draw [->,>=latex] (axis cs:7,11) -- (axis cs:7.1107, 11.0601);
\draw (axis cs: 7,14) node [align=left]
 {\scriptsize $\begin{aligned}x&=t^3-5t^2+3t+11\\y&=t^2-2t+3\end{aligned}$};
\end{axis}
\node [right] at (myplot.right of origin) {\scriptsize $x$};
\node [above] at (myplot.above origin) {\scriptsize $y$};
\end{tikzpicture}}{ALT-TEXT-TO-BE-DETERMINED}}
%
\begin{align*}
s^3-5s^2+3s+11 &= t^3-5t^2+3t+11 \\
s^2-2s+3 &= t^2-2t+3
\end{align*}

Solving this system is not trivial but involves only algebra. Using the quadratic formula, one can solve for $t$ in the second equation and find that $\ds t = 1\pm \sqrt{s^2-2s+1}$. This can be substituted into the first equation, revealing that the graph crosses itself at $t=-1$ and $t=3$. We confirm our result by computing $x(-1) = x(3)=2$ and $y(-1) = y(3) = 6$.
\end{example}

We now present a small gallery of ``interesting'' and ``famous'' curves along with parametric equations that produce them.\bigskip\vfill

\noindent\flushinner{%
\newcommand{\scalefactor}{1.16}
\tagpdfsetup{table/tagging=presentation}
 \begin{tabular}{ c c c }
\pdftooltip{\begin{tikzpicture}
\begin{axis}[width=\scalefactor\marginparwidth,tick label style={font=\scriptsize},
axis y line=middle,axis x line=middle,name=myplot,xtick={-1,1},ytick={-1,1},
ymin=-1.1,ymax=1.1,xmin=-1.1,xmax=1.1,axis equal]
\addplot [thick,draw={\colorone}, smooth,domain=0:360,samples=360]
 ({(cos(x))^3},{(sin(x))^3});
\end{axis}
\node [right] at (myplot.right of origin) {\scriptsize $x$};
\node [above] at (myplot.above origin) {\scriptsize $y$};
\end{tikzpicture}}{ALT-TEXT-TO-BE-DETERMINED}
  &
  \pdftooltip{\begin{tikzpicture}
   \begin{axis}[width=\scalefactor\marginparwidth,tick label style={font=\scriptsize},
                axis y line=middle,axis x line=middle,name=myplot,
                xtick={3.14,6.28},xticklabels={$\pi$,$2\pi$},
                ymin=-.2,ymax=4.2,xmin=-.1,xmax=6.3,axis equal image]
    \addplot [thick,draw={\colorone}, smooth,domain=0:6.28,samples=20]
     ({x-sin(deg(x))},{1-cos(deg(x))});
   \end{axis}
   \node [right] at (myplot.right of origin) {\scriptsize $x$};
   \node [above] at (myplot.above origin) {\scriptsize $y$};
  \end{tikzpicture}}{ALT-TEXT-TO-BE-DETERMINED} &
  \pdftooltip{\begin{tikzpicture}
   \begin{axis}[width=\scalefactor\marginparwidth,tick label style={font=\scriptsize},
                axis y line=middle,axis x line=middle,name=myplot,
                %xtick={-1,1},ytick={-1,1},
                ymin=-.1,ymax=4.1,xmin=-3.1,xmax=3.1,axis equal image]
    \addplot [thick,draw={\colorone}, smooth,domain=-1.55:1.55,samples=20]
     ({2*x},{2/(1+x^2)});
   \end{axis}
   \node [right] at (myplot.right of origin) {\scriptsize $x$};
   \node [above] at (myplot.above origin) {\scriptsize $y$};
  \end{tikzpicture}}{ALT-TEXT-TO-BE-DETERMINED} \\
  \parbox{150pt}{\centering Astroid\\$x=\cos^3 t$\\$y=\sin^3t$} &
  \parbox{150pt}{\centering Cycloid\\$x=r(t-\sin t)$\\$y=r(1-\cos t)$} &
  \parbox{150pt}{\centering Witch of Agnesi\\$x=2at$\\$y=2a/(1+t^2)$} \\
  \\
\pdftooltip{\begin{tikzpicture}
\begin{axis}[width=\scalefactor\marginparwidth,tick label style={font=\scriptsize},
axis y line=middle,axis x line=middle,name=myplot,
ymin=-7.1,ymax=7.1,xmin=-7.1,xmax=7.5,axis equal]
\addplot [thick, draw={\colorone},smooth,domain=0:1440,samples=120]
 ({2*cos(x)+5*cos(2*x/3)},{2*sin(x)-5*sin(2*x/3)});
\end{axis}
\node [right] at (myplot.right of origin) {\scriptsize $x$};
\node [above] at (myplot.above origin) {\scriptsize $y$};
\end{tikzpicture}}{ALT-TEXT-TO-BE-DETERMINED}
  &
\pdftooltip{\begin{tikzpicture}
\begin{axis}[width=\scalefactor\marginparwidth,tick label style={font=\scriptsize},
axis y line=middle,axis x line=middle,name=myplot,
ymin=-7.1,ymax=7.1,xmin=-7.1,xmax=7.5,axis equal]
\addplot [thick,draw={\colorone}, smooth,domain=0:720,samples=240]
 ({4*cos(x)-cos(4*x)},{4*sin(x)-sin(4*x)});
\end{axis}
\node [right] at (myplot.right of origin) {\scriptsize $x$};
\node [above] at (myplot.above origin) {\scriptsize $y$};
\end{tikzpicture}}{ALT-TEXT-TO-BE-DETERMINED}
  &
  \pdftooltip{\begin{tikzpicture}
   \begin{axis}[width=\scalefactor\marginparwidth,tick label style={font=\scriptsize},
                axis y line=middle,axis x line=middle,name=myplot,
                ymin=-2.1,ymax=2.1,xmin=-2.1,xmax=2.1,axis equal]
    \addplot [thick,draw={\colorone}, smooth,domain=-40:130,samples=20]
     ({3*sin(x)*cos(x)^2/((sin(x))^3+(cos(x))^3)},{3*sin(x)^2*cos(x)/((sin(x))^3+(cos(x))^3)});
   \end{axis}
   \node [right] at (myplot.right of origin) {\scriptsize $x$};
   \node [above] at (myplot.above origin) {\scriptsize $y$};
  \end{tikzpicture}}{ALT-TEXT-TO-BE-DETERMINED} \\
  \parbox{150pt}{\centering Hypotrochoid\\$x=2\cos(t)+5\cos(2t/3)$\\$y=2\sin(t)-5\sin(2t/3)$} &
  \parbox{150pt}{\centering Epicycloid\\$x=4\cos(t)-\cos(4t)$\\$y=4\sin(t)-\sin(4t)$} &
  \parbox{150pt}{\centering Folium of Descartes\\$x=3at/(1+t^3)$\\$y=3at^2/(1+t^3)$}
 \end{tabular}}
\bigskip

One might note a feature shared by three of these graphs: ``sharp corners,'' or \textbf{cusps}. We have seen graphs with cusps before and determined that such functions are not differentiable at these points. This leads us to a definition.

\begin{definition}[Smooth]\label{def:smooth}%
A curve $C$ defined by $x=f(t)$, $y=g(t)$ is \textbf{smooth} on an interval $I$ if $\fp$ and $g\primeskip'$ are continuous on $I$ and not simultaneously 0 (except possibly at the endpoints of $I$). A curve is \textbf{piecewise smooth} on $I$ if $I$ can be partitioned into subintervals where $C$ is smooth on each subinterval.
\index{curve!smooth}\index{smooth curve}\index{cusp}
\end{definition}

Consider the astroid, given by $x=\cos^3t$, $y=\sin^3t$. Taking derivatives, we have:
\[
x\primeskip' = -3\cos^2t\sin t\quad \text{and}\quad y\primeskip' = 3\sin^2t\cos t.
\]
It is clear that each is 0 when $t=0,\ \pi/2,\ \pi,\dotsc$. Thus the astroid is not smooth at these points, corresponding to the cusps seen in the figure. However, by restricting the domain of the astroid to all reals except $t = \frac{k\pi}{2}$ for $k \in\mathbb{Z}$ we have a piecewise smooth curve.

We demonstrate this once more.

\begin{example}[Determine where a curve is not smooth]\label{ex_pareq9}%
Let a curve $C$ be defined by the parametric equations $x=t^3-12t+17$ and $y=t^2-4t+8$. Determine the points, if any, where it is not smooth.
\solution
We begin by taking derivatives.
%
\mtable{Graphing the curve in \autoref{ex_pareq9}; note it is not smooth at $(1,4)$.}{fig:pareq9}{\pdftooltip{\begin{tikzpicture}
\begin{axis}[width=1.16\marginparwidth,tick label style={font=\scriptsize},
axis y line=middle,axis x line=middle,name=myplot,minor x tick num=4,
minor y tick num=1,ymin=-1,ymax=9,xmin=-1,xmax=11]
\addplot [thick,draw={\colorone}, smooth,domain=-2:3.5,samples=40]
 ({x^3-12*x+17},{x^2-4*x+8)});
\end{axis}
\node [right] at (myplot.right of origin) {\scriptsize $x$};
\node [above] at (myplot.above origin) {\scriptsize $y$};
\end{tikzpicture}}{ALT-TEXT-TO-BE-DETERMINED}}
%
\[x\primeskip' = 3t^2-12,\quad y\primeskip' = 2t-4.\]
We set each equal to 0:
\begin{align*}
 x\primeskip' =0 &\Rightarrow 3t^2-12=0 \Rightarrow t=\pm 2\\
 y\primeskip' =0 &\Rightarrow 2t-4 = 0 \Rightarrow t=2
\end{align*}
We consider only the value of $t=2$ since both $x'$ and $y'$ must be 0. Thus $C$ is not smooth at $t=2$, corresponding to the point $(1,4)$. The curve is graphed in \autoref{fig:pareq9}, illustrating the cusp at $(1,4)$.
\end{example}

If a curve is not smooth at $t=t_0$, it means that $x\primeskip'(t_0)=y\primeskip'(t_0)=0$ as defined. This, in turn, means that rate of change of $x$ (and $y$) is 0; that is, at that instant, neither $x$ nor $y$ is changing. If the parametric equations describe the path of some object, this means the object is at rest at $t_0$. An object at rest can make a ``sharp'' change in direction, whereas moving objects tend to change direction in a ``smooth'' fashion.

% cycloid courtesy of Paul Dawkins
\begin{example}[The Cycloid]\label{ex_cycloid}%
A well-known parametric curve is the cycloid.  Fix $r$, and let $x=r(t-\sin t)$, $y=r(1-\cos t)$.  This represents the path traced out by a point on a wheel of radius $r$ as it starts rolling to the right.  We can think of $t$ as the angle through which the point has rotated.
 
\noindent\begin{minipage}[t]{\linewidth}\noindent%
\captionsetup{type=figure}%
 \centering
  \pdftooltip{\begin{tikzpicture}
   \begin{axis}[axis lines=none,xmin=-1,xmax=4*pi+1,ymin=-.5,ymax=3,axis equal,
     width=\textwidth,height=\ifbool{latexml}{.3}{}\textwidth]
    \foreach \x in { 0, 2*pi/3, 4*pi/3, 2*pi, 8*pi/3, 10*pi/3, 4*pi} {
     \edef\temp{%
      \noexpand\draw [draw={\colorone}] (axis cs:\x,1) circle (1);
      \noexpand\draw [draw={\colorone},fill={\colorone}] (axis cs:\x,1)circle (1.5pt)
       -- (axis cs:{\x-sin(deg(\x))},{1-cos(deg(\x))}) circle (1.5pt);
     }\temp
    }
    \draw (axis cs:0,2.5) node {$t=0$};
    \draw (axis cs:2*pi,2.5) node {$t=2\pi$};
    \draw (axis cs:4*pi,2.5) node {$t=4\pi$};
    \foreach \t in { 2, 4, 8, 10} {
     \edef\temp{%
      \noexpand\draw (axis cs:\t*pi/3,2.5) node {$t=\frac{\t\pi}3$};
     }\temp
    }
    \addplot[draw={\colortwo},smooth,thick,domain=0:4*pi]
     ({\x-sin(deg(\x))},{1-cos(deg(\x))});
   \end{axis}
  \end{tikzpicture}}{ALT-TEXT-TO-BE-DETERMINED}
 \caption{A cycloid traced through two revolutions.}
 \label{fig_cycloid}
\end{minipage}
 
\autoref{fig_cycloid} shows a cycloid sketched out with the wheel shown at various places.  The dot on the rim is the point on the wheel that we're using to trace out the curve.
 
From this sketch we can see that one arch of the cycloid is traced out in the range $0\le t\le2\pi$.  This makes sense when you consider that the point will be back on the ground after it has rotated through an angle of $2\pi$.
\end{example}

One should be careful to note that a ``sharp corner'' does not have to occur when a curve is not smooth. For instance, one can verify that $x=t^3$, $y=t^6$ produce the familiar $y=x^2$ parabola. However, in this parameterization, the curve is not smooth. A particle traveling along the parabola according to the given parametric equations comes to rest at $t=0$, though no sharp point is created.\bigskip

Our previous experience with cusps taught us that a function was not differentiable at a cusp. This can lead us to wonder about derivatives in the context of parametric equations and the application of other calculus concepts. Given a curve defined parametrically, how do we find the slopes of tangent lines? Can we determine concavity? We explore these concepts and more in the next section.

\printexercises{exercises/09-02-exercises}

\input{text/09_Parametric_Calculus}
\section{Introduction to Polar Coordinates}\label{sec:polar}

We are generally introduced to the idea of graphing curves by relating $x$-values to $y$-values through a function $f$. That is, we set $y=f(x)$, and plot lots of point pairs $(x,y)$ to get a good notion of how the curve looks. This method is useful but has limitations, not least of which is that curves that ``fail the vertical line test'' cannot be graphed without using multiple functions.

The previous two sections introduced and studied a new way of plotting points in the $x,y$-plane. Using parametric equations, $x$ and $y$ values are computed independently and then plotted together. This method allows us to graph an extraordinary range of curves. This section introduces yet another way to plot points in the plane: using \textbf{polar coordinates}.

\subsection{Polar Coordinates}

\mtable{Illustrating polar coordinates.}{fig:polar_intro1}{%
\begin{tikzpicture}[>=latex]
	\draw[thick,->] (0,0) node [below] {$O$} -- (3,0) node [below] {initial ray} ;
	\filldraw (0,0) circle (1.5pt);
	\filldraw [rotate=55] (2,0) circle (1.5pt);
	\draw [thick,rotate=55] (0,0)-- node [rotate=55,pos=.5,above] {$r$} (2,0) node [above] {$(r,\theta)$};
	\draw [->] (.75,0) arc(0:55:.75); 
	\draw [rotate=27.5] (1,0) node {$\theta$};
\end{tikzpicture}}

Start with a point $O$ in the plane called the \textbf{pole} (we will always identify this point with the origin). From the pole, draw a ray, called the \textbf{initial ray} (we will always draw this ray horizontally, identifying it with the positive $x$-axis). A point $P$ in the plane is determined by the distance $r$ that $P$ is from $O$, and the angle $\theta$ formed between the initial ray and the segment $\overline{OP}$ (measured counter-clockwise). We record the distance and angle as an ordered pair $(r,\theta)$.%To avoid confusion with rectangular coordinates, we will denote polar coordinates with the letter $P$, as in $P(r,\theta)$. This is illustrated in \autoref{fig:polar_intro1}
\index{polar coordinates}\index{polar!coordinates}

\youtubeVideo{r0fv9V9GHdo}{Polar Coordinates --- The Basics}

Practice will make this process more clear.

\begin{example}[Plotting Polar Coordinates]\label{ex_polar1}
Plot the following polar coordinates:\index{polar coordinates!plotting points}
\[A(1,\pi/4)\quad B(1.5,\pi)\quad C(2,-\pi/3)\quad D(-1,\pi/4)\]
\solution
To aid in the drawing, a polar grid is provided
%
\parbox[t][0pt]{0pt}{%
\iflatexml
 \begin{tikzpicture}[remember picture,overlay,scale=.8]
\else
 \checkoddpage%
 \ifbool{oddpage}{%
  \begin{tikzpicture}[remember picture,overlay,scale=.8,shift=(current page.south),shift={+(.5in,1.7in)}]
 }{%
  \begin{tikzpicture}[remember picture,overlay,scale=.8,shift=(current page.south),shift={+(-.5in,1.7in)}]
 }%
\fi
 \foreach \x in { 0,30,45,60, 90,120,135,150, 180,210,225,240, 270,300,315,330 }
  { \draw [dashed,gray] (0,0) -- (\x:3.1); }
 \draw[thick,->,>=stealth] (0,0) node [below] {$O$} -- (3.5,0) ;
 \filldraw (0,0) circle (1.5pt);
 \foreach \x in {1,2,3} {
  \draw (0,0) circle (\x cm);
  \draw (\x,0) node [below right] {\x};
 }
\end{tikzpicture}}%
\ifbool{latexml}{here.}{at the bottom of this page.}%
%
\mtable{Plotting polar points in \autoref{ex_polar1}.}{fig:polar1}{\begin{tikzpicture}[scale=.75,>=latex]
	\draw[thick,->] (0,0) node [below] {$O$} -- (3.5,0) ;
	\filldraw (0,0) circle (1.5pt);
	\foreach \x in {1,2,3} {
		\draw (0,0) circle (\x cm);
		\draw (\x,0) node [below right] {\x};
	}
\filldraw [rotate=45] (1,0) circle (1.5pt) node [above right] {$A$};
\filldraw [rotate=180] (1.5,0) circle (1.5pt) node [above] {$B$};
\filldraw [rotate=-60] (2,0) circle (1.5pt) node [below right] {$C$};
\filldraw [rotate=45] (-1,0) circle (1.5pt) node [below] {$D$};
\end{tikzpicture}}
%
To place the point $A$, go out 1 unit along the initial ray (putting you on the inner circle shown on the grid), then rotate counter-clockwise $\pi/4$ radians (or $45^\circ$).  Alternately, one can consider the rotation first: think about the ray from $O$ that forms an angle of $\pi/4$ with the initial ray, then move out 1 unit along this ray (again placing you on the inner circle of the grid).

To plot $B$, go out $1.5$ units along the initial ray and rotate $\pi$ radians ($180^\circ$). 

To plot $C$, go out 2 units along the initial ray then rotate \emph{clockwise} $\pi/3$ radians, as the angle given is negative.

To plot $D$, move along the initial ray ``$-1$'' units --- in other words, ``back up'' 1 unit, then rotate counter-clockwise by $\pi/4$. The results are given in \autoref{fig:polar1}.
\end{example}

Consider the following two points: $A(1,\pi)$ and $B(-1,0)$. To locate $A$, go out 1 unit on the initial ray then rotate $\pi$ radians; to locate $B$, go out $-1$ units on the initial ray and don't rotate. One should see that $A$ and $B$ are located at the same point in the plane. We can also consider $C(1,3\pi)$, or $D(1,-\pi)$; all four of these points share the same location. 

This ability to identify a point in the plane with multiple polar coordinates is both a ``blessing'' and a ``curse.'' We will see that it is beneficial as we can plot beautiful functions that intersect themselves (much like we saw with parametric functions). The unfortunate part of this is that it can be difficult to determine when this happens. We'll explore this more later in this section.

\subsection{Polar to Rectangular Conversion}

\mtable{Converting between rectangular and polar coordinates.}{fig:polar_intro2}{\begin{tikzpicture}[scale=.75,>=latex]
	\draw [thick] (0,0) -- node [below,pos=.5] {$x$} (4,0)
	 -- node [right,pos=.5] {$y$} (4,2)
	 -- node [above,rotate=26.57,pos=.5] {$r$} (0,0);
	\draw [->,thick] (1.5,0) arc (0:26:1.5);
	\draw [rotate=13] (1.8,0) node {$\theta$};
	\filldraw (0,0) circle (1.5pt) node [below] {$O$}
			(4,2) circle (1.5pt) node [above] {$P$};
\end{tikzpicture}}

It is useful to recognize both the rectangular (or, Cartesian) coordinates of a point in the plane and its polar coordinates. \autoref{fig:polar_intro2} shows a point $P$ in the plane with rectangular coordinates $(x,y)$ and polar coordinates $(r,\theta)$. Using trigonometry, we can make the identities given in the following Key Idea.

\begin{keyidea}[Converting Between Rectangular and Polar Coordinates]\label{idea:polarconvert}
Given the polar point $P(r,\theta)$, the rectangular coordinates are determined by \[x=r\cos \theta\qquad y=r\sin \theta.\]

Given the rectangular coordinates $(x,y)$, the polar coordinates are determined by
\[r^2=x^2+y^2\qquad \tan \theta = \frac yx.\]
\end{keyidea}

\begin{example}[Converting Between Polar and Rectangular Coordinates]\label{ex_polar2}
\mbox{}\\[-2\baselineskip]\parbox[t]{\linewidth}{%
\begin{enumerate}
\item		Convert the polar coordinates $A(2,2\pi/3)$ and $B(-1,5\pi/4)$ to rectangular coordinates.
\item		Convert the rectangular coordinates $(1,2)$ and $(-1,1)$ to polar coordinates.
\end{enumerate}}
\solution
\begin{enumerate}
	\item \begin{enumerate}
		\item 
	We start with $A(2,2\pi/3)$. Using \autoref{idea:polarconvert}, we have 
	\[x= 2\cos (2\pi/3) = -1\qquad y = 2\sin (2\pi/3) = \sqrt{3}.\]
	So the rectangular coordinates are $(-1,\sqrt{3}) \approx (-1,1.732)$.
	
	\item The polar point $B(-1,5\pi/4)$ is converted to rectangular with:
	\[x=-1\cos (5\pi/4) = \sqrt{2}/2\qquad y= -1\sin (5\pi/4) = \sqrt{2}/2.\]
	So the rectangular coordinates are $(\sqrt{2}/2,\sqrt{2}/2) \approx (0.707,0.707)$.
	\end{enumerate}
	These points are plotted in \autoref{fig:polar2} (a). The rectangular coordinate system is drawn lightly under the polar coordinate system so that the relationship between the two can be seen.
	
\mtable{Plotting rectangular and polar points in \autoref{ex_polar2}.}{fig:polar2}{%
\begin{tikzpicture}[>=latex]
 \draw [thin,gray] (-2.5,-2.5) grid (2.5,2.5);
 \draw [thick,->] (0,0) node [below] {$O$} -- (2.5,0);
 \foreach \x in {1,2}{
  \draw [very thick] (0,0) circle (\x cm);
 }
 \filldraw (0,0) circle (1.5pt);
 \filldraw [rotate=120] (2,0) circle (1.5pt) node [ left] {\scriptsize $A(2,\frac{2\pi}{3})$};
 \filldraw [rotate=225] (-1,0) circle (1.5pt) node [shift={(0pt,12pt)}] {\scriptsize$B(-1,\frac{5\pi}{4})$};
\end{tikzpicture}
\smallskip\\(a)\bigskip\\
\begin{tikzpicture}[>=latex]
\draw [very thick] (-2.5,-2.5) grid (2.5,2.5);
\draw [thin,gray,->] (0,0) node [below left,black] {\scriptsize$(0,0)$} -- (2.5,0);
\foreach \x in {1,2} {
 \draw [thin,gray] (0,0) circle (\x cm);
}
\filldraw (0,0) circle (1.5pt);
\filldraw [] (1,2) circle (1.5pt) node [above left] {\scriptsize $(1,2)$};
\filldraw [] (-1,1) circle (1.5pt) node [above right] {\scriptsize$(-1,1)$};
\draw [gray,dashed] (0,0) -- ($(0,0)!2.2!(-1,1)$);
\draw [gray,->] (.5,0) arc (0:135:.5);
\draw [rotate=35,gray ] (.65,0) node [] {\scriptsize$\frac{3\pi}{4}$};
\draw [gray,dashed] (0,0) -- ($(0,0)!1.2!(1,2)$);
\draw [gray,->] (.5,0) arc (0:135:.5);
\draw [gray,->] (-.5,0) arc (180:135:.5);
\draw [rotate=35,gray ] (.65,0) node [] {\scriptsize$\frac{3\pi}{4}$};
\draw [rotate=-30,gray ] (-.65,0) node [shift={(-3pt,0pt)}] {\scriptsize$\frac{-\pi}{4}$};
\draw [gray,->] (1.5,0) arc (0:63.4:1.5);
\draw [rotate=28,gray ] (1.75,0) node [] {\scriptsize$1.11$};
\end{tikzpicture}
\smallskip\\(b)}

	\item \begin{enumerate}
		\item To convert the rectangular point $(1,2)$ to polar coordinates, we use the Key Idea to form the following two equations:
		\[1^2+2^2 = r^2 \qquad \tan \theta = \frac{2}{1}.\]
		The first equation tells us that $r=\sqrt{5}$. Using the inverse tangent function, we find
		\[\tan \theta = 2 \quad \Rightarrow \quad \theta = \tan^{-1} 2 \approx 1.11\text{ radians}\approx 63.43^\circ.\]
		Thus polar coordinates of $(1,2)$ are $(\sqrt{5},1.11)$.
		\item		To convert $(-1,1)$ to polar coordinates, we form the equations 
		\[(-1)^2+1^2=r^2 \qquad \tan \theta = \frac{1}{-1}.\]
		Thus $r=\sqrt{2}$. We need to be careful in computing $\theta$: using the inverse tangent function, we have
		\[\tan\theta = -1 \quad \Rightarrow \quad \theta = \tan^{-1}(-1) = -\pi/4.\]
		This is not the angle we desire. The range of $\tan^{-1}x $ is $(-\pi/2,\pi/2)$; that is, it returns angles that lie in the $1^\text{st}$ and $4^\text{th}$ quadrants. To find locations in the $2^\text{nd}$  and $3^\text{rd}$ quadrants, add $\pi$ to the result of $\tan^{-1}x$. So  $\pi+(-\pi/4)$ puts the angle at $3\pi/4$. Thus the polar point is $(\sqrt{2},3\pi/4)$.
		
		An alternate method is to use the angle $\theta$ given by arctangent, but change the sign of $r$. Thus we could also refer to $(-1,1)$ as\\ $(-\sqrt{2},-\pi/4)$.
	\end{enumerate}
These points are plotted in \autoref{fig:polar2} (b). The polar system is drawn lightly under the rectangular grid with rays to demonstrate the angles used.
\end{enumerate}
\end{example}

\subsection{Polar Functions and Polar Graphs}

Defining a new coordinate system allows us to create a new kind of function, a \textbf{polar function.} Rectangular coordinates lent themselves well to creating functions that related $x$ and $y$, such as $y=x^2.$ Polar coordinates allow us to create functions that relate $r$ and $\theta$. Normally these functions look like $r=f(\theta)$, although we can create functions of the form $\theta = f(r)$. The following examples introduce us to this concept.\index{polar!functions}\index{polar!functions!graphing}

\begin{example}[Introduction to Graphing Polar Functions]\label{ex_polar3}
Describe the graphs of the following polar functions.
\begin{enumerate}
	\item $r = 1.5$
	\item $\theta = \pi/4 $
\end{enumerate}
%
%\noindent$1.\ r = 1.5 \qquad 2.\ \theta = \pi/4 $
\solution
\begin{enumerate}
\item		The equation $r=1.5$ describes all points that are 1.5 units from the pole; as the angle is not specified, any $\theta$ is allowable. All points 1.5 units from the pole describes a circle of radius 1.5.

We can consider the rectangular equivalent of this equation; using $r^2=x^2+y^2$, we see that $1.5^2=x^2+y^2$, which we recognize as the equation of a circle centered at $(0,0)$ with radius 1.5. This is sketched in \autoref{fig:polar3}.

\mtable{Plotting standard polar plots.}{fig:polar3}{\begin{tikzpicture}[>=latex]
 \draw [thick,->] (0,0) node [below] {$O$} -- (2.5,0);
 \foreach \x in {1,2}
  {\draw [thin,gray] (0,0) circle (\x cm);
   \draw (\x,0) node [shift={(3pt,-4pt)}] {\scriptsize\x};
  }
 \filldraw (0,0) circle (1.5pt);
 \draw [very thick] (0,0) circle (1.5);
 \draw[rotate=90] (1.7,0) node {\scriptsize $r=1.5$};
 \draw [thick] (-2,-2)
  -- node [pos=1,left] {\scriptsize $\theta = \frac{\pi}4$} (2,2);
\end{tikzpicture}}

\item		The equation $\theta = \pi/4$ describes all points such that the line through them and the pole make an angle of $\pi/4$ with the initial ray. As the radius $r$ is not specified, it can be any value (even negative). Thus $\theta = \pi/4$ describes the line through the pole that makes an angle of $\pi/4 = 45^\circ$ with the initial ray.

We can again consider the rectangular equivalent of this equation. Combine $\tan \theta =y/x$ and $\theta =\pi/4$:
\[\tan \pi/4 = y/x \quad \Rightarrow \quad x\tan \pi/4 = y \quad \Rightarrow \quad y = x.\] 
This graph is also plotted in \autoref{fig:polar3}.
\end{enumerate}
\end{example}

The basic rectangular equations of the form $x=h$ and $y=k$ create vertical and horizontal lines, respectively; the basic polar equations $r= h$ and $\theta =\alpha$ create circles and lines through the pole, respectively. With this as a foundation, we can create more complicated polar functions of the form $r=f(\theta)$. The input is an angle; the output is a length, how far in the direction of the angle to go out.

We sketch these functions much like we sketch rectangular and parametric functions: we plot lots of points and ``connect the dots'' with curves. We demonstrate this in the following example.

\begin{example}[Sketching Polar Functions]\label{ex_polar4}
Sketch
%
\mtable[-1in]{Graph of the polar function in \autoref{ex_polar4} by plotting points.}{fig:polar4}{%
\begin{tabular}{c}
 $\begin{array}{cc}
	\theta & r=1+\cos\theta \\ \midrule
	0 & 2 \\
	\pi/6 & 1+\sqrt{3}/2 \\
	\pi/4 & 1+1/\sqrt2 \\
	\pi/3 & 3/2 \\
	\pi/2 & 1 \\
	2\pi/3 & 1/2 \\
	3\pi/4 & 1-1/\sqrt2 \\
	5\pi/6 & 1-\sqrt3/2 \\
	\pi & 0 \\
	7\pi/6 & 1-\sqrt3/2 \\
	5\pi/4 & 1-1/\sqrt2 \\
	4\pi/3 & 1/2 \\
	3\pi/2 & 1 \\
	5\pi/3 & 3/2 \\
	7 \pi /4 & 1+1/\sqrt2 \\
	11\pi/6 & 1+\sqrt{3}/2 \\
 \end{array}$ \\ \\
 \begin{tikzpicture}[scale=.95]
  \draw [dashed,gray] (-2.1,0) -- (0,0);
  \draw[thick,->,>=stealth] (0,0) node [below] {$O$} -- (2.5,0) ;
  \filldraw (0,0) circle (1.5pt);
  \foreach \x in {1,2}
   {\draw[gray] (0,0) circle (\x cm);
    \draw (\x,0) node [shift={(3pt,-4pt)}] {\scriptsize\x};
   }
  \foreach \x in {30,45,60,90,120,135,150}
   {\draw [rotate=\x,dashed,gray] (-2.3,0) -- (2.3,0);}
  \draw [thick,draw={\colorone},domain=0:360,samples=60,smooth] plot
   ({cos(\x)*(1+cos(\x))},{sin(\x)*(1+cos(\x))});
  \foreach \x in { 0,30,45,60, 90,120,135,150, 180,210,225,240, 270,300,315,330 }
   { \filldraw ({\x}:{1+cos(\x)}) circle (1pt); }
  \foreach \x/\y in { 45/{\pi/4}, 90/{\pi/2}, 135/{3\pi/4}, 180/{\pi}, 225/{5\pi/4}, 270/{3\pi/2}, 315/{7\pi/4} }
   { \draw({\x}:{2.3}) node{$\scriptstyle\y$}; }
\end{tikzpicture}
\end{tabular}}
%
the polar function $r=1+\cos \theta$ on $[0,2\pi]$ by plotting points.
\solution
A common question when sketching curves by plotting selected points is ``Which points should I plot?'' With rectangular equations, we often chose ``easy'' values --- integers, then added more if needed. When plotting polar equations, start with the ``common'' angles --- multiples of $\pi/6$ and $\pi/4$. \autoref{fig:polar4} gives a table of just a few values of $\theta$ in $[0,\pi]$. 

Consider the point $(2,0)$ determined by the first line of the table. The angle is 0 radians --- we do not rotate from the initial ray -- then we go out 2 units from the pole. When $\theta=\pi/6$, $r = 1+\sqrt{3}/2$; so rotate by $\pi/6$ radians and go out $1+\sqrt{3}/2$ units.
\end{example}

%\paragraph{Technology Note:} Plotting functions in this way can be tedious, just as it was with rectangular functions. To obtain very accurate graphs, technology is a great aid. Most graphing calculators can plot polar functions; in the menu, set the plotting mode to something like \texttt{polar} or \texttt{POL}, depending on one's calculator. As with plotting parametric functions, the viewing ``window'' no longer determines the $x$-values that are plotted, so additional information needs to be provided. Often with the ``window'' settings are the settings for  the beginning and ending $\theta$ values (often called \texttt{$\theta_{\text{min}}$} and \texttt{$\theta_{\text{max}}$}) as well as the \texttt{$\theta_{\text{step}}$} --- that is, how far apart the $\theta$ values are spaced. The smaller the \texttt{$\theta_{\text{step}}$} value, the more accurate the graph (which also increases plotting time). Using technology, we graphed the polar function $r=1+\cos \theta$ from \autoref{ex_polar4} in \autoref{fig:polar4b}.
%
%\mtable{Using  technology to graph a polar function.}{fig:polar4b}{\begin{tikzpicture}[scale=1]
%	\draw [dashed,gray] (-2.1,0) -- (0,0);
%	\draw[thick,->,>=stealth] (0,0) node [below] {$O$} -- (2.5,0) ;
%	\filldraw (0,0) circle (1.5pt);
%	\foreach \x in {1,2} {
%		\draw (0,0) circle (\x cm);
%		\draw (\x,0) node [shift={(3pt,-4pt)}] {\scriptsize\x};
%	}
%	\foreach \x in {30,45,60,90,120,135,150} {
%		\draw [rotate=\x,dashed,gray] (-2.3,0) -- (2.3,0);
%	}
%	\draw [thick,draw={\colorone},domain=0:360,samples=60,smooth]
%	 plot ({cos(\x)*(1+cos(\x))},{sin(\x)*(1+cos(\x))});
%\end{tikzpicture}}

\begin{example}[Sketching Polar Functions]\label{ex_polar5}
Sketch the polar function $r=\cos (2\theta)$ on $[0,2\pi]$ by plotting points.
\solution
We start by making a table of $\cos (2\theta)$ evaluated at common angles $\theta$, as shown in \autoref{fig:polar5table}. These points are then plotted in \autoref{fig:polar5}. This particular graph ``moves'' around quite a bit and one can easily forget which points should be connected to each other. To help us with this, we numbered each point in the table and on the graph. 

{\centering
\captionsetup{type=figure}%
$\begin{array}{ccl @{\hspace{4em}} ccl}\toprule
\text{Pt.} & \theta & \cos (2\theta) & \text{Pt.} & \theta & \cos (2\theta) \\
 \cmidrule(r{4em}){1-3}\cmidrule(l{-1em}){4-6}
 1 & 0 & \phantom{-}1 & 10 & 7 \pi /6 & \phantom{-}0.5 \\
 2 & \pi /6 & \phantom{-}0.5 & 11 & 5 \pi /4 & \phantom{-}0 \\
 3 & \pi /4 & \phantom{-}0 & 12 & 4 \pi /3 & -0.5 \\
 4 & \pi /3 & -0.5 & 13 & 3 \pi /2 & -1 \\
 5 & \pi /2 & -1 & 14 & 5 \pi /3 & -0.5 \\
 6 & 2 \pi /3 & -0.5 & 15 & 7 \pi /4 & \phantom{-}0 \\
 7 & 3 \pi /4 & \phantom{-}0 & 16 & 11 \pi /6 & \phantom{-}0.5 \\
 8 & 5 \pi /6 & \phantom{-}0.5 & 17 & 2 \pi & \phantom{-}1 \\
 9 & \pi  & \phantom{-}1 \\\bottomrule
\end{array}$
\caption{Tables of points for plotting a polar curve.}
\label{fig:polar5table}
}% end centering

\mtable[-2in]{Polar plots from \autoref{ex_polar5}.}{fig:polar5}{%
 \begin{tikzpicture}[scale=1]
  \draw [dashed,gray] (-2.1,0) -- (0,0);
  \draw[thick,->,>=latex] (0,0) node [below] {} -- (2.5,0) ;
  \filldraw (0,0) circle (1.5pt);
  \foreach \x in {1,2}
   {\draw[gray] (0,0) circle (\x cm);}
  \foreach \x in {30,45,60,90,120,135,150}
   {\draw [rotate=\x,dashed,gray] (-2.3,0) -- (2.3,0);}
%  \draw [thick,draw={\colorone}] plot coordinates {(2.,0)(0.866,0.5)(0,0)(-0.5,-0.866)(0,-2.)(0.5,-0.866)(0,0)(-0.866,0.5)(-2.,0)(-0.866,-0.5)(0,0)(0.5,0.866)(0,2.)(-0.5,0.866)(0,0)(0.866,-0.5)(2,0)};
  \foreach \x/\y/\z/\w in
   {2./0/1/{above right}, 0.866/0.5/2/above, 0/0/3/above, -0.5/-0.866/4/above,
    0/-2./5/above, 0.5/-0.866/6/above, 0/0/7/left, -0.866/0.5/8/above,
    -2./0/9/above left, -0.866/-0.5/10/above, 0/0/11/right, 0.5/0.866/12/above,
    0/2./13/above, -0.5/0.866/14/above, 0/0/15/below, 0.866/-0.5/16/above,
    2/0/17/{below right}}
   {\filldraw (\x,\y) circle (1.5pt) node [\w] {\scriptsize \z};}       
  \draw [thick,draw={\colorone},domain=0:360,samples=60,smooth]
   plot ({\x}:{2*cos(2*\x)});
\end{tikzpicture}}

%Using more points (and the aid of technology) a smoother plot can be made as shown in \autoref{fig:polar5} (b).
This plot is an example of a \emph{rose curve}.
\end{example}

It is sometimes desirable to refer to a graph via a polar equation, and other times by a rectangular equation. Therefore it is necessary to be able to convert between polar and rectangular functions, which we practice in the following example. We will make frequent use of the identities found in \autoref{idea:polarconvert}.

\begin{example}[Converting between rectangular and polar equations.]\label{ex_polar6}
\begin{minipage}[t]{.5\linewidth}
Convert from rectangular to polar.
\begin{enumerate}
	\item $y=x^2$
	\item $xy = 1$
\end{enumerate}
\end{minipage}%
\begin{minipage}[t]{.5\linewidth}
Convert from polar to rectangular.
\begin{enumerate}\addtocounter{enumi}{2}
	\item $\ds r=\frac{2}{\sin \theta-\cos\theta}$
	\item $r=2\cos \theta$
\end{enumerate}
\end{minipage}
\solution
\begin{enumerate}
	\item Replace $y$ with $r\sin\theta$ and replace $x$ with $r\cos\theta$, giving:
	\begin{align*}
	y &=x^2\\
	r\sin\theta &= r^2\cos^2\theta\\
	\frac{\sin\theta}{\cos^2\theta} &= r
	\end{align*}
	We have found that $r=\sin\theta/\cos^2\theta = \tan\theta\sec\theta$. The domain of this polar function is $(-\pi/2,\pi/2)$; plot a few points to see how the familiar parabola is traced out by the polar equation.
	
	\item		We again replace $x$ and $y$ using the standard identities and work to solve for $r$:
	\begin{align*}
	xy &= 1 \\
	r\cos\theta\cdot r\sin\theta & = 1\\
	r^2 & = \frac{1}{\cos\theta\sin\theta}\\
	r & = \frac{1}{\sqrt{\cos\theta\sin\theta}}\\
	\end{align*}
%
\mtable{Graphing $xy=1$ from \autoref{ex_polar6}.}{fig:polar6}{\begin{tikzpicture}
\begin{axis}[width=1.16\marginparwidth,tick label style={font=\scriptsize},
axis y line=middle,axis x line=middle,name=myplot,minor x tick num=4,
minor y tick num=4,ymin=-5.3,ymax=5.3,xmin=-5.3,xmax=5.3,axis equal]
\addplot [draw={\colorone},thick, smooth,domain=-5:-.1,samples=30] {1/x};
\addplot [draw={\colorone},thick, smooth,domain=.1:5,samples=30] {1/x};
\end{axis}
\node [right] at (myplot.right of origin) {\scriptsize $x$};
\node [above] at (myplot.above origin) {\scriptsize $y$};
\end{tikzpicture}}
%
	This function is valid only when the product of $\cos\theta\sin\theta$ is positive. This occurs in the first and third quadrants, meaning the domain of this polar function is $(0,\pi/2) \cup (\pi,3\pi/2)$.
	
	We can rewrite the original rectangular equation $xy=1$ as $y=1/x$. This is graphed in \autoref{fig:polar6}; note how it only exists in the first and third quadrants.
		
	\item		There is no set way to convert from polar to rectangular; in general, we look to form the products $r\cos \theta$ and $r\sin\theta$, and then replace these with $x$ and $y$, respectively. We start in this problem by multiplying both sides by $\sin\theta-\cos\theta$:
	\begin{align*}
	r &= \frac{2}{\sin\theta-\cos\theta} \\
	r(\sin\theta-\cos\theta) &= 2\\
	r\sin\theta-r\cos\theta &= 2. \qquad \text{Now replace with $y$ and $x$:}\\
	y-x &= 2\\
	y &= x+2.
	\end{align*}
	The original polar equation, $r=2/(\sin\theta-\cos\theta)$ does not easily reveal that its graph is simply a line. However, our conversion shows that it is. The upcoming gallery of polar curves gives the general equations of lines in polar form.

	\item		By multiplying both sides by $r$, we obtain both an $r^2$ term and an $r\cos\theta$ term, which we replace with $x^2+y^2$ and $x$, respectively. 
	\begin{align*}
	r &=2\cos\theta \\
	r^2 &= 2r\cos\theta \\
	x^2+y^2 &= 2x. 
	\intertext{We recognize this as a circle; by completing the square we can find its radius and center.}
	x^2-2x+y^2 &= 0 \\
	(x-1)^2 + y^2 &=1.
	\end{align*}
	The circle is centered at $(1,0)$ and has radius 1. The upcoming gallery of polar curves gives the equations of \emph{some} circles in polar form; circles with arbitrary centers have a complicated polar equation that we do not consider here.
\end{enumerate}
\end{example}

Some curves have very simple polar equations but rather complicated rectangular ones. For instance, the equation $r=1+\cos\theta$ describes a \emph{cardioid} (a shape important to the sensitivity of microphones, among other things; one is graphed in the gallery in the Limaçon section). Its rectangular form is not nearly as simple; it is the implicit equation
$x^4+y^4+2x^2y^2-2xy^2-2x^3-y^2=0.$ The conversion is not ``hard,'' but takes several steps, and is left as an exercise.

\subsection{Gallery of Polar Curves}

There are a number of basic and ``classic'' polar curves, famous for their beauty and/or applicability to the sciences. \index{polar!function!gallery of graphs} This section ends with a small gallery of some of these graphs. We encourage the reader to understand how these graphs are formed, and to investigate with technology other types of polar functions.

\newlength{\gallerywidth}
%\setlength{\gallerywidth}{.25\textwidth}
% no.  the textwidth doesn't change until \exercisegeometry in 50 lines
\setlength{\gallerywidth}{(0pt+\marginparwidth+\textwidth)/4}

\noindent
\flushinner{%
\noindent\hrulefill\bigskip\\
\begin{tabular}{p{\gallerywidth}p{\gallerywidth}p{\gallerywidth}p{\gallerywidth}}
%
\textbf{\large Lines}\smallskip\\
\textbf{Through the origin:} & \textbf{Horizontal line:} &
\textbf{Vertical line:} & \textbf{Not through origin:} \smallskip\\
$\theta = \alpha$ & $r=a\csc\theta$ &
$r=a\sec\theta$ & $\ds r=\frac{b}{\sin\theta-m\cos\theta}$ \medskip\\
\begin{tikzpicture}[scale=.9,>=stealth]
	\draw [<->,] (-2.1,0) -- (2.1,0);
	\draw [<->,] (0,-2.1) -- (0,2.1);
	\draw [thick,draw={\colorone},rotate=50] (-2.1,0) -- (2.1,0);
	\draw [->] (.5,0) arc (0:50:.5);
	\draw [rotate=25] (.7,0) node {\scriptsize $\alpha$};
\end{tikzpicture}		
&
\begin{tikzpicture}[scale=.9,>=stealth]
	\draw [<->,] (-2.1,0) -- (2.1,0);
	\draw [<->,] (0,-2.1) -- (0,2.1);
	\draw [thick,draw={\colorone}] (-2,.6) -- (2,.6);
	\draw (-.2,.3) node {\scriptsize $a\left\{\rule[-.23cm]{0pt}{.23cm}\right.$};
\end{tikzpicture}		
&
\begin{tikzpicture}[scale=.9,>=stealth]
	\draw [<->,] (-2.1,0) -- (2.1,0);
	\draw [<->,] (0,-2.1) -- (0,2.1);
	\draw [thick,draw={\colorone}] (.6,-2) -- (.6,2);
	\draw (.3,.2) node {\scriptsize $\overbrace{\makebox[.55cm]{}}$};
	\draw (.3,.4) node {\scriptsize $a$};
\end{tikzpicture}		
&
\begin{tikzpicture}[scale=.9,>=stealth]
	\draw [<->,] (-2.1,0) -- (2.1,0);
	\draw [<->,] (0,-2.1) -- (0,2.1);
	\draw [thick,draw={\colorone},shift={(0,.6)},rotate=50] (-2.1,0) -- (1.7,0)
	 node [pos=.82,rotate=50,black,shift={(0,-5pt)}] {\scriptsize slope $=m$};
	\draw (.2,.3) node {\scriptsize $\left.\rule[-.23cm]{0pt}{.23cm}\right\}b$};
\end{tikzpicture}		
%
\end{tabular}}

\clearpage

\exercisegeometry

\thispagestyle{empty}
%
\noindent\flushinner{%
\begin{tabular}{p{\gallerywidth}p{\gallerywidth}p{\gallerywidth}p{\gallerywidth}}
%
\textbf{\large Circles} &&& \textbf{\large Spiral} \\
\textbf{Centered on origin:} & $(x-\frac a2)^2+y^2=\frac{a^2}4$ & $x^2+(y-\frac a2)^2=\frac{a^2}4$ & \textbf{Archimedean spiral}\\
$r=a$ & $r=a\cos \theta$ & $r=a\sin\theta$ & $r=\theta$\\
\begin{tikzpicture}[scale=.9,>=stealth]
	\draw [<->,] (-2.1,0) -- (2.1,0);
	\draw [<->,] (0,-2.1) -- (0,2.1);
	\draw [thick,draw={\colorone}] (0,0) circle (.9);
	\draw (.45,.1) node {\scriptsize $\overbrace{\makebox[.8cm]{}}$};
	\draw (.45,.3) node {\scriptsize $a$};
\end{tikzpicture}		
&
\begin{tikzpicture}[scale=.9,>=stealth]
	\draw [<->,] (-2.1,0) -- (2.1,0);
	\draw [<->,] (0,-2.1) -- (0,2.1);
	\draw [thick,draw={\colorone}] (.9,0) circle (.9);
	\draw (.9,.1) node {\scriptsize $\overbrace{\makebox[1.7cm]{}}$};
	\draw (.9,.3) node {\scriptsize $a$};
\end{tikzpicture}		
&
\begin{tikzpicture}[scale=.9,>=stealth]
	\draw [<->,] (-2.1,0) -- (2.1,0);
	\draw [<->,] (0,-2.1) -- (0,2.1);
	\draw [thick,draw={\colorone}] (0,.9) circle (.9);
	\draw (-.2,.9) node {\scriptsize $a\left\{\rule[-.8cm]{0cm}{0.8cm}\right.$};
\end{tikzpicture}		
&
\begin{tikzpicture}[scale=.9,>=stealth]
	\draw [<->,] (-2.1,0) -- (2.1,0);
	\draw [<->,] (0,-2.1) -- (0,2.1);
	\draw [thick,draw={\colorone},domain=0:18.85,samples=100,smooth] plot
	 ({cos(\x r)*(\x/9.5)},{sin(\x r)*(\x/9.5)});
\end{tikzpicture}		
\bigskip\\
%
\textbf{\large Limaçons}\\
\makebox[0pt][l]{Symmetric about $x$-axis: $r=a\pm b\cos\theta$; \qquad
Symmetric about $y$-axis:  $r=a\pm b\sin \theta$; \qquad $a,b>0$}\\
\textbf{With inner loop:} & \textbf{Cardioid:} &
\textbf{Dimpled:} & \textbf{Convex:} \smallskip\\
$\dfrac ab < 1$ & $\dfrac ab=1$ & $1<\dfrac ab <2$ & $\dfrac ab>2$ \\
\begin{tikzpicture}[scale=.9,>=stealth]
	\draw [<->,] (-2.1,0) -- (2.1,0);
	\draw [<->,] (0,-2.1) -- (0,2.1);
	\draw [thick,draw={\colorone},domain=0:360,samples=60,smooth] plot
	 ({cos(\x)*(.6+1.2*cos(\x))},{sin(\x)*(.6+1.2*cos(\x))});
\end{tikzpicture}		
&
\begin{tikzpicture}[scale=.9,>=stealth]
	\draw [<->,] (-2.1,0) -- (2.1,0);
	\draw [<->,] (0,-2.1) -- (0,2.1);
	\draw [thick,draw={\colorone},domain=0:360,samples=60,smooth] plot
	 ({cos(\x)*(.9+.9*cos(\x))},{sin(\x)*(.9+.9*cos(\x))});
\end{tikzpicture}		
&
\begin{tikzpicture}[scale=.9,>=stealth]
	\draw [<->,] (-2.1,0) -- (2.1,0);
	\draw [<->,] (0,-2.1) -- (0,2.1);
	\draw [thick,draw={\colorone},domain=0:360,samples=60,smooth] plot
	 ({cos(\x)*(1+.8*cos(\x))},{sin(\x)*(1+.8*cos(\x))});
\end{tikzpicture}		
&
\begin{tikzpicture}[scale=.9,>=stealth]
	\draw [<->,] (-2.1,0) -- (2.1,0);
	\draw [<->,] (0,-2.1) -- (0,2.1);
	\draw [thick,draw={\colorone},domain=0:360,samples=60,smooth] plot
	 ({cos(\x)*(1.3+.6*cos(\x))},{sin(\x)*(1.3+.6*cos(\x))});
\end{tikzpicture}		
\bigskip\\
%
\textbf{\large Rose Curves}\\
\makebox[0pt][l]{Symmetric about $x$-axis: $r=a \cos(n\theta)$; \qquad
Symmetric about $y$-axis:  $r=a\sin(n\theta)$}\\
\makebox[0pt][l]{Curve contains $2n$ petals when $n$ is even and $n$ petals when $n$ is odd.}\\
$r=a\cos (2\theta)$ & $r=a\sin(2\theta)$ &
$r=a\cos (3\theta)$ & $r=a\sin (3\theta)$ \smallskip\\
\begin{tikzpicture}[scale=.9,>=stealth]
	\draw [<->,] (-2.1,0) -- (2.1,0);
	\draw [<->,] (0,-2.1) -- (0,2.1);
	\draw [thick,draw={\colorone},domain=0:360,samples=60,smooth] plot
	 ({cos(\x)*(1.9*cos(2*\x))},{sin(\x)*(1.9*cos(2*\x))});
\end{tikzpicture}		
&
\begin{tikzpicture}[scale=.9,>=stealth]
	\draw [<->,] (-2.1,0) -- (2.1,0);
	\draw [<->,] (0,-2.1) -- (0,2.1);
	\draw [thick,draw={\colorone},domain=0:360,samples=60,smooth] plot
	 ({cos(\x)*(1.9*sin(2*\x))},{sin(\x)*(1.9*sin(2*\x))});
\end{tikzpicture}		
&
\begin{tikzpicture}[scale=.9,>=stealth]
	\draw [<->,] (-2.1,0) -- (2.1,0);
	\draw [<->,] (0,-2.1) -- (0,2.1);
	\draw [thick,draw={\colorone},domain=0:360,samples=60,smooth] plot
	 ({cos(\x)*(1.9*cos(3*\x))},{sin(\x)*(1.9*cos(3*\x))});
\end{tikzpicture}		
&
\begin{tikzpicture}[scale=.9,>=stealth]
	\draw [<->,] (-2.1,0) -- (2.1,0);
	\draw [<->,] (0,-2.1) -- (0,2.1);
	\draw [thick,draw={\colorone},domain=0:360,samples=60,smooth] plot
	 ({cos(\x)*(1.9*sin(3*\x))},{sin(\x)*(1.9*sin(3*\x))});
\end{tikzpicture}		
\bigskip\\
%
\textbf{\large Special Curves} \\
\textbf{Rose curves} &  & \textbf{Lemniscate:} &
\textbf{Eight Curve:} \smallskip\\
$r=a\sin (\theta/5)$ & $r=a\sin(2\theta/5)$ &
$r^2=a^2\cos (2\theta)$ & $r^2=a^2\sec^4\theta\cos (2\theta)$ \medskip\\
\begin{tikzpicture}[scale=.9,>=stealth]
	\draw [<->,] (-2.1,0) -- (2.1,0);
	\draw [<->,] (0,-2.1) -- (0,2.1);
	\draw [thick,draw={\colorone},domain=0:900,samples=100,smooth] plot
	 ({cos(\x)*(1.9*sin(\x/5))},{sin(\x)*(1.9*sin(\x/5))});
\end{tikzpicture}		
&
\begin{tikzpicture}[scale=.9,>=stealth]
	\draw [<->,] (-2.1,0) -- (2.1,0);
	\draw [<->,] (0,-2.1) -- (0,2.1);
	\draw [thick,draw={\colorone},domain=0:1800,samples=200,smooth] plot
	 ({cos(\x)*(1.9*cos(2*\x/5))},{sin(\x)*(1.9*cos(2*\x/5))});
\end{tikzpicture}		
&
\begin{tikzpicture}[scale=.9,>=stealth]
	\draw [<->,] (-2.1,0) -- (2.1,0);
	\draw [<->,] (0,-2.1) -- (0,2.1);
	\draw [thick,draw={\colorone},domain=-45:45,samples=60,smooth] plot
	 ({cos(\x)*(1.9*sqrt(cos(2*\x)))},{sin(\x)*(1.9*sqrt(cos(2*\x)))});
	\draw [thick,draw={\colorone},domain=-45:45,samples=60,smooth] plot
	 ({-cos(\x)*(1.9*sqrt(cos(2*\x)))},{-sin(\x)*(1.9*sqrt(cos(2*\x)))});
\end{tikzpicture}		
&
\begin{tikzpicture}[scale=.9,>=stealth]
	\draw [<->,] (-2.1,0) -- (2.1,0);
	\draw [<->,] (0,-2.1) -- (0,2.1);
	\draw [thick,draw={\colorone},domain=-45:45,samples=60,smooth] plot
	 ({cos(\x)*1.9*sec(\x)*sec(\x)*sqrt(cos(2*\x))},
	  {sin(\x)*1.9*sec(\x)*sec(\x)*sqrt(cos(2*\x))});
	\draw [thick,draw={\colorone},domain=-45:45,samples=60,smooth] plot
	 ({-cos(\x)*1.9*sec(\x)*sec(\x)*sqrt(cos(2*\x))},
	  {-sin(\x)*1.9*sec(\x)*sec(\x)*sqrt(cos(2*\x))});
\end{tikzpicture}
%
\end{tabular}}

\restoregeometry

Earlier we discussed how each point in the plane does not have a unique representation in polar form. This can be a ``good'' thing, as it allows for the beautiful and interesting curves seen in the preceding gallery. However, it can also be a ``bad'' thing, as it can be difficult to determine where two curves intersect.

\begin{example}[Finding points of intersection with polar curves]\label{ex_polar7}
Determine where the graphs of the polar equations $r=1+3\cos\theta$ and $r=\cos \theta$ intersect.
\solution
As technology is generally readily available, it is usually a good idea to start with a graph. We have graphed the two functions in \autoref{fig:polar7}(a); to better discern the intersection points, part (b) of the figure zooms in around the origin.
\mtable{Graphs to help determine the points of intersection of the polar functions given in \autoref{ex_polar7}.}{fig:polar7}{%
\begin{tikzpicture}
\begin{axis}[width=1.16\marginparwidth,tick label style={font=\scriptsize},
axis y line=middle,axis x line=middle,name=myplot,minor x tick num=1,
minor y tick num=1,ymin=-2.6,ymax=2.6,xmin=-1.5,xmax=4.5,axis equal]
\addplot [draw={\colorone},thick, smooth,domain=0:360,samples=60]
 ({cos(x)*(1+3*cos(x))},{sin(x)*(1+3*cos(x))});
\draw[draw={\colortwo},thick](axis cs:.5,0)circle(.5);
\end{axis}
\node [right] at (myplot.right of origin) {\scriptsize $0$};
\node [above] at (myplot.above origin) {\scriptsize $\pi/2$};
\end{tikzpicture}
\\(a)\\
\begin{tikzpicture}
\begin{axis}[width=1.16\marginparwidth,tick label style={font=\scriptsize},
axis y line=middle,axis x line=middle,name=myplot,minor x tick num=1,
ymin=-.6,ymax=.6,xmin=-.72,xmax=.72,axis equal]
\addplot [draw={\colorone},thick, smooth,domain=0:360,samples=60]
 ({cos(x)*(1+3*cos(x))},{sin(x)*(1+3*cos(x))});
\draw[draw={\colortwo},thick](axis cs:.5,0)circle(.5);
\end{axis}
\node [right] at (myplot.right of origin) {\scriptsize $0$};
\node [above] at (myplot.above origin) {\scriptsize $\pi/2$};
\end{tikzpicture}
\\(b)}
We start by setting the two functions equal to each other and solving for $\theta$:
\begin{align*}
1+3\cos\theta &= \cos \theta \\
2\cos\theta &= -1\\
\cos\theta&= -\frac12\\
\theta &= \frac{2\pi}{3}, \frac{4\pi}{3}.
\end{align*}
(There are, of course, infinite solutions to the equation $\cos\theta=-1/2$; as the limaçon is traced out once on $[0,2\pi]$, we restrict our solutions to this interval.) 

We need to analyze this solution. When $\theta = 2\pi/3$ we obtain the point of intersection that lies in the 4\textsuperscript{th} quadrant. When $\theta = 4\pi/3$, we get the point of intersection that lies in the 1\textsuperscript{st} quadrant. There is more to say about this second intersection point, however. The circle defined by $r=\cos\theta$ is traced out once on $[0,\pi]$, meaning that this point of intersection occurs while tracing out the circle a second time. It seems strange to pass by the point once and then recognize it as a point of intersection only when arriving there a ``second time.'' The first time the circle arrives at this point is when $\theta = \pi/3$.
It is key to understand that these two points are the same: $(\cos \pi/3,\pi/3)$ and $(\cos 4\pi/3,4\pi/3)$. 

To summarize what we have done so far, we have found two points of intersection: when $\theta=2\pi/3$ and when $\theta=4\pi/3$. When referencing the circle $r=\cos \theta$, the latter point is better referenced as when $\theta=\pi/3$.

There is yet another point of intersection: the pole (or, the origin). We did not recognize this intersection point using our work above as each graph arrives at the pole at a different $\theta$ value.

A graph intersects the pole when $r=0$. Considering the circle $r=\cos\theta$, $r=0$ when $\theta = \pi/2$ (and odd multiples thereof, as the circle is repeatedly traced). The limaçon intersects the pole when $1+3\cos\theta =0$; this occurs when $\cos \theta = -1/3$, or for $\theta = \cos^{-1}(-1/3)$. This is a nonstandard angle, approximately $\theta = 1.9106\text{ radians} \approx 109.47^\circ$. The limaçon intersects the pole twice in $[0,2\pi]$; the other angle at which the limaçon is at the pole is the reflection of the first angle across the $x$-axis. That is, $\theta = 4.3726 \approx 250.53^\circ$.
\end{example}

If all one is concerned with is the $(x,y)$ coordinates at which the graphs intersect, much of the above work is extraneous. We know they intersect at $(0,0)$; we might not care at what $\theta$ value. Likewise, using $\theta =2\pi/3$ and $\theta=4\pi/3$ can give us the needed rectangular coordinates. However, in the next section we apply calculus concepts to polar functions. When computing the area of a region bounded by polar curves, understanding the nuances of the points of intersection becomes important.

\printexercises{exercises/09_04_exercises}

\section{Calculus and Polar Functions} \label{sec:polarcalc}

The previous section defined polar coordinates, leading to polar functions. We investigated plotting these functions and solving a fundamental question about their graphs, namely, where do two polar graphs intersect?

We now turn our attention to answering other questions, whose solutions require the use of calculus. A basis for much of what is done in this section is the ability to turn a polar function $r=f(\theta)$ into a set of parametric equations. Using the identities $x=r\cos \theta$ and $y=r\sin \theta$, we can create the parametric equations $x=f(\theta)\cos\theta$, $y=f(\theta)\sin\theta$ and apply the concepts of \autoref{sec:par_calc}.

\subsection*{Polar Functions and $\dfrac{dy}{dx}$}

We are interested in the lines tangent to a given graph, regardless of whether that graph is produced by rectangular, parametric, or polar equations. In each of these contexts, the slope of the tangent line is $\frac{dy}{dx}$. Given $r=f(\theta)$, we are generally \textit{not} concerned with $r\,'=\fp(\theta)$; that describes how fast $r$ changes with respect to $\theta$. Instead, we will use $x=f(\theta)\cos\theta$, $y=f(\theta)\sin\theta$ to compute $\frac{dy}{dx}$. 

Using \autoref{idea:dydxpar} we have
\[\frac{dy}{dx} = \frac{dy}{d\theta}\Big/\frac{dx}{d\theta}.\]
Each of the two derivatives on the right hand side of the equality requires the use of the Product Rule. We state the important result as a Key Idea.

\keyidea{idea:dydxpol}{Finding $\frac{dy}{dx}$ with Polar Functions}
{Let $r=f(\theta)$ be a polar function. With $x=f(\theta)\cos\theta$ and $y=f(\theta)\sin\theta$,
\index{polar!functions!finding $\frac{dy}{dx}$}\index{tangent line}
\[
 \frac{dy}{dx}
 =\frac{\frac{dy}{d\theta}}{\frac{dx}{d\theta}}
 =\frac{\fp(\theta)\sin\theta+f(\theta)\cos\theta}{\fp(\theta)\cos\theta-f(\theta)\sin\theta}.
\]
}

\youtubeVideo{QTa9OZ4iGPo}{The Slope of Tangent Lines to Polar Curves}

% todo `` answer should be ``-2'' not ``-1''. ''  whatever that means
\example{ex_polcalc1}{Finding $\frac{dy}{dx}$ with polar functions.}
{Consider the lima\c con $r=1+2\sin\theta$ on $[0,2\pi]$.
\begin{enumerate}
	\item	Find the rectangular equations of the tangent and normal lines to the graph at $\theta=\pi/4$.
	\item	Find where the graph has vertical and horizontal tangent lines.
\end{enumerate}
}
{\begin{enumerate}
	\item We start by computing $\frac{dy}{dx}$. With $\fp(\theta) = 2\cos\theta$, we have
	\begin{align*}
	\frac{dy}{dx} &= \frac{2\cos\theta\sin\theta + \cos\theta(1+2\sin\theta)}{2\cos^2\theta-\sin\theta(1+2\sin\theta)}\\
	&= \frac{\cos\theta(4\sin\theta+1)}{2(\cos^2\theta-\sin^2\theta)-\sin\theta}.
	\end{align*}
	When $\theta=\pi/4$, $\frac{dy}{dx}=-2\sqrt{2}-1$ (this requires a bit of simplification). In rectangular coordinates, the point on the graph at $\theta=\pi/4$ is $(1+\sqrt{2}/2,1+\sqrt{2}/2)$. Thus the rectangular equation of the line tangent to the lima\c con at $\theta=\pi/4$ is 
\[y=(-2\sqrt{2}-1)\big(x-(1+\sqrt{2}/2)\big)+1+\sqrt{2}/2 \approx  -3.83 x+8.24.\]
The lima\c con and the tangent line are graphed in \autoref{fig:polcalc1}. 
	
	The normal line has the opposite--reciprocal slope as the tangent line, so its equation is 
\[y \approx \frac{1}{3.83}x+1.26.\]
	
	\item		To find the horizontal lines of tangency, we find where $\frac{dy}{dx}=0$ (when the denominator does not equal 0); thus we find where the numerator of our equation for $\frac{dy}{dx}$ is 0.
	\[\cos\theta(4\sin\theta+1)=0\quad \Rightarrow \quad \cos\theta=0 \quad \text{or}\quad 4\sin\theta+1=0.\]
	On $[0,2\pi]$, $\cos\theta=0$ when $\theta=\pi/2,\ 3\pi/2$. 

Setting $4\sin\theta+1=0$ gives $\theta=\sin^{-1}(-1/4)\approx -0.2527 = -14.48^\circ$. We want the results in $[0,2\pi]$; we also recognize there are two solutions, one in the 3$^\text{rd}$ quadrant and one in the 4$^\text{th}$. Using reference angles, we have our two solutions as $\theta =3.39$ and $6.03$ radians. The four points we obtained where the lima\c con has a horizontal tangent line are given in \autoref{fig:polcalc1} with black--filled dots.\bigskip

To find the vertical lines of tangency, we determine where $\frac{dy}{dx}$ is undefined by setting the denominator of $\frac{dy}{dx}=0$ (when the numerator does not equal 0).
\begin{align*}
2(\cos^2\theta -\sin^2\theta)-\sin\theta &= 0 .
\intertext{Convert the $\cos^2\theta$ term to $1-\sin^2\theta$:}
2(1-\sin^2\theta-\sin^2\theta)-\sin\theta &= 0\\
4\sin^2\theta + \sin\theta -2 &= 0.
\intertext{Recognize this as a quadratic in the variable $\sin\theta$. Using the quadratic formula, we have}
\sin\theta &= \frac{-1\pm\sqrt{33}}{8}.
\end{align*}
%
\mtable{The lima\c con in \autoref{ex_polcalc1} with its tangent line at $\theta=\pi/4$ and points of vertical and horizontal tangency.}{fig:polcalc1}{\begin{tikzpicture}
\begin{axis}[width=1.16\marginparwidth,tick label style={font=\scriptsize},
axis y line=middle,axis x line=middle,name=myplot,minor x tick num=1,
ymin=-.35,ymax=3.2,xmin=-2.1,xmax=2.1]
\addplot [draw={\colorone},thick, smooth,domain=0:360,samples=60]
 ({cos(x)*(1+2*sin(x))},{sin(x)*(1+2*sin(x))});
\addplot [draw={\colortwo},thick, smooth,domain=0.1:2,samples=2] {-3.82843*x+8.24264};
\filldraw (axis cs: 0,3) circle (1.5pt)
		(axis cs: 0,1) circle (1.5pt)
		(axis cs: -.493,-0.125) circle (1.5pt)
		(axis cs: .483,-0.125) circle (1.5pt);
\filldraw [fill=white,draw=black,thick] (axis cs: 1.76011, 1.31048) circle (1.5pt)
		(axis cs: -1.76011, 1.31048) circle (1.5pt)
		(axis cs: 0.368977, 0.572639) circle (1.5pt)
		(axis cs: -0.369009, 0.578743) circle (1.5pt);
\end{axis}
\node [right] at (myplot.right of origin) {\scriptsize $0$};
\node [above] at (myplot.above origin) {\scriptsize $\pi/2$};
\end{tikzpicture}}%
%
We solve $\sin\theta = \frac{-1+\sqrt{33}}8$ and $\sin\theta = \frac{-1-\sqrt{33}}8$:
\begin{align*}
\sin\theta &=\frac{-1+\sqrt{33}}8 & \sin\theta &= \frac{-1-\sqrt{33}}{8}\\
\theta &= \sin^{-1}\left(\frac{-1+\sqrt{33}}8\right) & \theta &= \sin^{-1}\left(\frac{-1-\sqrt{33}}8\right)\\
\theta &\approx 0.6349 & \theta &\approx -1.0030
\end{align*}
In each of the solutions above, we only get one of the possible two solutions as $\sin^{-1}x$ only returns solutions in $[-\pi/2,\pi/2]$, the 4$^\text{th}$ and $1^\text{st}$ quadrants. Again using reference angles, we have:
\[
\sin\theta = \frac{-1+\sqrt{33}}8 \quad \Rightarrow \quad
\theta \approx 0.6349,\ 2.5067 \text{ radians}
\]
and 
\[
\sin\theta = \frac{-1-\sqrt{33}}8 \quad \Rightarrow \quad
\theta \approx 4.1446,\ 5.2802 \text{ radians.}
\]
These points are also shown in \autoref{fig:polcalc1} with white--filled dots.\eoehere
\end{enumerate}}

When the graph of the polar function $r=f(\theta)$ intersects the pole, it means that $f(\alpha) = 0$ for some angle $\alpha$. Making this substitution in the formula for $\dfrac{dy}{dx}$ given in \autoref{idea:dydxpol} we see
\[
 \frac{dy}{dx}
 =\frac{\fp(\alpha)\sin\alpha+f(\alpha)\cos\alpha}{\fp(\alpha)\cos\alpha+f(\alpha)\sin\alpha}
 =\frac{\sin\alpha}{\cos\alpha}
 =\tan\alpha.
\]
%
%When the graph of the polar function $r=f(\theta)$ intersects the pole, it means that $f(\alpha)=0$ for some angle $\alpha$. Thus the formula for $\frac{dy}{dx}$ in such instances is very simple, reducing simply to \[\frac{dy}{dx} = \tan \alpha.\]
This equation makes an interesting point. It tells us the slope of the tangent line at the pole is $\tan \alpha$; some of our previous work (see, for instance, \autoref{ex_polar3}) shows us that the line through the pole with slope $\tan \alpha$ has polar equation $\theta=\alpha$. Thus when a polar graph touches the pole at $\theta=\alpha$, the equation of the tangent line at the pole is $\theta=\alpha$.

\example{ex_polcalc2}{Finding tangent lines at the pole}{Let $r=1+2\sin\theta$, a lima\c con. Find the equations of the lines tangent to the graph at the pole.
%
\mtable{Graphing the tangent lines at the pole in \autoref{ex_polcalc2}.}{fig:polcalc2}{\begin{tikzpicture}
\begin{axis}[width=1.16\marginparwidth,tick label style={font=\scriptsize},
axis y line=middle,axis x line=middle,name=myplot,
ymin=-.5,ymax=1.37,xmin=-1.1,xmax=1.1]
\addplot [draw={\colorone},thick, smooth,domain=0:360,samples=60]
 ({cos(x)*(1+2*sin(x))},{sin(x)*(1+2*sin(x))});
\addplot [draw={\colortwo},thick, smooth,domain=-1:1,samples=2] {x/sqrt(3)};
\addplot [draw={\colortwo},thick, smooth,domain=-1:1,samples=2] {-x/sqrt(3)};
\end{axis}
\node [right] at (myplot.right of origin) {\scriptsize $0$};
\node [above] at (myplot.above origin) {\scriptsize $\pi/2$};
\end{tikzpicture}}}
{We need to know when $r=0$. 
\begin{align*}
1+2\sin\theta &= 0\\
\sin\theta &= -1/2\\
\theta &= \frac{7\pi}{6},\ \frac{11\pi}6.
\end{align*}
Thus the equations of the tangent lines, in polar, are $\theta = 7\pi/6$ and $\theta = 11\pi/6$. In rectangular form, the tangent lines are $y=\tan(7\pi/6)x=\frac x{\sqrt3}$ and $y=\tan(11\pi/6)x=-\frac x{\sqrt3}$. The full lima\c con can be seen in \autoref{fig:polcalc1}; we zoom in on the tangent lines in \autoref{fig:polcalc2}.}

\subsection*{Area}

When using rectangular coordinates, the equations $x=h$ and $y=k$ defined vertical and horizontal lines, respectively, and combinations of these lines create rectangles (hence the name ``rectangular coordinates''). It is then somewhat natural to use rectangles to approximate area as we did when learning about the definite integral.\index{polar!functions!area}

\mnote{\textbf{Note:} Recall that the area of a sector of a circle with radius $r$ subtended by an angle $\theta$ is $A = \frac12\theta r^2$.
\[
\begin{tikzpicture}[x=30pt,y=30pt,thick]
			\draw (2,0) arc (0:50:2) -- (0,0);
			\draw [] (0,0) -- (2,0) node [pos=.5,below] {$r$};
			\draw [fill=black] (0,0) circle (1pt);
			%\draw (1.95,1.0) node {$s$};
			\draw (0,0) node [shift={(15pt,8pt)}] {$\theta$};
			\end{tikzpicture}
\]}

When using polar coordinates, the equations $\theta=\alpha$ and $r=c$ form lines through the origin and circles centered at the origin, respectively, and combinations of these curves form sectors of circles. It is then somewhat natural to calculate the area of regions defined by polar functions by first approximating with sectors of circles. 

Consider \autoref{fig:polararea} (a) where a region defined by $r=f(\theta)$ on $[\alpha,\beta]$ is given. (Note how the ``sides'' of the region are the lines $\theta=\alpha$ and $\theta=\beta$, whereas in rectangular coordinates the ``sides'' of regions were often the vertical lines $x=a$ and $x=b$.)

Partition the interval $[\alpha,\beta]$ into $n$ equally spaced subintervals as $\alpha=\theta_0<\theta_1<\dotso<\theta_n=\beta$. The radian of each subinterval is $\Delta\theta = (\beta-\alpha)/n$, representing a small change in angle. The area of the region defined by the $i\,^\text{th}$ subinterval $[\theta_{i-1},\theta_i]$ can be approximated with a sector of a circle with radius $f(c_i)$, for some $c_i$ in $[\theta_{i-1},\theta_i]$. The area of this sector is $\frac12[f(c_i)]^2\Delta\theta$. This is shown in part (b) of the figure, where $[\alpha,\beta]$ has been divided into 4 subintervals. We approximate the area of the whole region by summing the areas of all sectors:
%
\mtable{Computing the area of a polar region.}{fig:polararea}{%
\begin{tikzpicture}
\begin{axis}[width=1.16\marginparwidth,tick label style={font=\scriptsize},
axis y line=middle,axis x line=middle,name=myplot,
ymin=-.1,ymax=1.1,xmin=-.1,xmax=1.1,axis equal]
\addplot
 [draw={\colorone},fill={\coloronefill},area style, smooth,domain=18:72,samples=30]
% (axis cs:x:{1+.05*cos(9*x)})
 ({cos(x)*(1+.05*cos(9*x))},{sin(x)*(1+.05*cos(9*x))})
  -- (axis cs:0,0) -- cycle;
\addplot [draw={\colorone},thick, smooth,domain=0:90,samples=30]
 ({cos(x)*(1+.05*cos(9*x))},{sin(x)*(1+.05*cos(9*x))});
\draw [thick,draw={\colorone},] (axis cs:0,0) -- (axis cs: 0.905831, 0.294322)
 node [pos=.7,below,rotate=18,black] {\scriptsize $\theta=\alpha$};
\draw [thick,draw={\colorone},] (axis cs:0,0) -- (axis cs:0.313792, 0.965751)
 node [pos=.7,above,rotate=72,black] {\scriptsize $\theta=\beta$};
\draw [thick,draw={\colorone},dashed]
 (axis cs:0,0) -- (axis cs: 0.862592, 0.528597)
  (axis cs:0,0) -- (axis cs: 0.732107, 0.732107)
   (axis cs:0,0) -- (axis cs: 0.497095, 0.811186);
\draw (axis cs:.8,.85) node {\scriptsize $r=f(\theta)$};
\end{axis}
\node [right] at (myplot.right of origin) {\scriptsize $0$};
\node [above] at (myplot.above origin) {\scriptsize $\pi/2$};
\end{tikzpicture}
\\(a)\\
\begin{tikzpicture}
\begin{axis}[width=1.16\marginparwidth,tick label style={font=\scriptsize},
axis y line=middle,axis x line=middle,name=myplot,
ymin=-.1,ymax=1.1,xmin=-.1,xmax=1.1,axis equal]
\addplot
 [draw={\coloronefill},fill={\coloronefill},area style, smooth,domain=18:72,samples=30]
 ({cos(x)*(1+.05*cos(9*x))},{sin(x)*(1+.05*cos(9*x))}) -- (axis cs:0,0) -- cycle;
\draw [thick,draw={\colorone},] (axis cs:0,0) -- (axis cs:0.313792, 0.965751)
 node [pos=.7,above,rotate=72,black] {\scriptsize $\theta=\beta$};
\addplot [draw={\colortwo},fill={\colortwofill},thick, smooth,domain=18:31.5,samples=30]
 ({.96*cos(x)},{.96*sin(x)}) -- (axis cs:0,0) -- cycle;
\addplot [draw={\colortwo},fill={\colortwofill},thick, smooth,domain=31.5:45,samples=30]
 ({1.05*cos(x)},{1.05*sin(x)}) -- (axis cs:0,0) -- cycle;
\addplot [draw={\colortwo},fill={\colortwofill},thick, smooth,domain=45:58.5,samples=30]
 ({1.0*cos(x)},{1*sin(x)}) -- (axis cs:0,0) -- cycle;
\addplot [draw={\colortwo},fill={\colortwofill},thick, smooth,domain=58.5:72,samples=30]
 ({.96*cos(x)},{.96*sin(x)}) -- (axis cs:0,0) -- cycle;
\draw (axis cs:.8,.85) node {\scriptsize $r=f(\theta)$};
\draw [thick,draw={\colortwo},] (axis cs:0,0) -- (axis cs: 0.905831, 0.294322)
 node [pos=.7,below,rotate=18,black] {\scriptsize $\theta=\alpha$};
\addplot [draw={\colorone},thick, smooth,domain=0:90,samples=30]
 ({cos(x)*(1+.05*cos(9*x))},{sin(x)*(1+.05*cos(9*x))});
\end{axis}
\node [right] at (myplot.right of origin) {\scriptsize $0$};
\node [above] at (myplot.above origin) {\scriptsize $\pi/2$};
\end{tikzpicture}
\\(b)}%
%
\[\text{Area} \approx \sum_{i=1}^n \frac12[f(c_i)]^2\Delta\theta.\]
This is a Riemann sum. By taking the limit of the sum as $n\to\infty$, we find the exact area of the region in the form of a definite integral.

\theorem{thm:polar_area}{Area of a Polar Region}
{Let $f$ be continuous and non-negative on $[\alpha,\beta]$, where $0\leq \beta-\alpha\leq 2\pi$. The area  $A$ of the region bounded by the curve $r=f(\theta)$ and the lines $\theta=\alpha$ and $\theta=\beta$ is 
\[
A=\frac12\int_\alpha^\beta[f(\theta)]^2 \ d\theta
=\frac12\int_\alpha^\beta r^{\,2} \ d\theta
\]}

The theorem states that $0\leq \beta-\alpha\leq 2\pi$. This ensures that region does not overlap itself, which would give a result that does not correspond directly to the area.

\example{ex_polcalc3}{Area of a polar region}{Find the area of the circle defined by $r=\cos \theta$. (Recall this circle has radius $1/2$.)}
{This is a direct application of \autoref{thm:polar_area}. The circle is traced out on $[0,\pi]$, leading to the integral
\mnote{\textbf{Note:} \autoref{ex_polcalc3} requires the use of the integral $\ds\int \cos^2\theta\ d\theta$. This is handled well by using the half angle formula as found in the back of this text. Due to the nature of the area formula, integrating $\cos^2\theta$ and $\sin^2\theta$ is required often. We offer here these indefinite integrals as a time--saving measure.
\begin{align*}
	\int\cos^2\theta\ d\theta &= \frac12\theta+\frac14\sin(2\theta)+C\\
	\int\sin^2\theta\ d\theta &= \frac12\theta-\frac14\sin(2\theta)+C
\end{align*}}
\begin{align*}
	\text{Area} &= \frac12\int_0^\pi \cos^2\theta\ d  \theta \\
	&= \frac12\int_0^\pi \frac{1+\cos(2\theta)}{2}\ d\theta\\
	&= \frac14\big(\theta +\frac12\sin(2\theta)\big)\Bigg|_0^\pi\\
	&= \frac\pi4.
\end{align*}
Of course, we already knew the area of a circle with radius $1/2$. We did this example to demonstrate that the area formula is correct.}

\example{ex_polcalc4}{Area of a polar region}{Find the area of the cardioid $r=1+\cos\theta$ bound between $\theta=\pi/6$ and $\theta=\pi/3$, as shown in \autoref{fig:polcalc4}.}
{This is again a direct application of \autoref{thm:polar_area}.
%
\mtable{Finding the area of the shaded region of a cardioid in \autoref{ex_polcalc4}.}{fig:polcalc4}{\begin{tikzpicture}
\begin{axis}[width=1.16\marginparwidth,tick label style={font=\scriptsize},
axis y line=middle,axis x line=middle,name=myplot,ytick={1},
ymin=-.7,ymax=1.5,xmin=-.5,xmax=2.1]
\addplot
 [draw={\coloronefill},fill={\coloronefill},area style,smooth,domain=30:60,samples=30]
 ({cos(x)*(1+cos(x))},{sin(x)*(1+cos(x))}) -- (axis cs:0,0) -- cycle;
\addplot [draw={\colorone},thick, smooth,domain=0:360,samples=60]
 ({cos(x)*(1+cos(x))},{sin(x)*(1+cos(x))});
\draw [thick,draw={\colorone}] (axis cs:0,0) -- (axis cs: 1.61603, 0.933013)
 node [pos=.6,below,rotate=30,black] {\scriptsize $\theta=\pi/6$};
\draw [thick,draw={\colorone}] (axis cs:0,0) -- (axis cs:0.75, 1.29904)
 node [pos=.6,above,rotate=60,black] {\scriptsize $\theta=\pi/3$};
\end{axis}
\node [right] at (myplot.right of origin) {\scriptsize $0$};
\node [above] at (myplot.above origin) {\scriptsize $\pi/2$};
\end{tikzpicture}}
%
\begin{align*}
	\text{Area}
	&= \frac12\int_{\pi/6}^{\pi/3} (1+\cos\theta)^2\ d\theta\\
	&= \frac12\int_{\pi/6}^{\pi/3} (1+2\cos\theta+\cos^2\theta)\ d\theta\\
	&= \frac12\left[\theta+2\sin\theta+\frac12\theta
	+\frac14\sin(2\theta)\right]_{\pi/6}^{\pi/3} \\
	&= \frac18\big(\pi+4\sqrt{3}-4\big).\eoehere
\end{align*}}

\subsubsection*{Area Between Curves}

Our study of area in the context of rectangular functions led naturally to finding area bounded between curves. We consider the same in the context of polar functions. \index{polar!functions!area between curves}

\mtable{Illustrating area bound between two polar curves.}{fig:polarea3}{\begin{tikzpicture}
\begin{axis}[width=1.16\marginparwidth,tick label style={font=\scriptsize},
axis y line=middle,axis x line=middle,name=myplot,
ymin=-.1,ymax=1.1,xmin=-.1,xmax=1.1]
\addplot [draw={\colortwo}, thick,smooth,domain=0:90,samples=30]
 ({cos(x)*(.7+.05*sin(6*x))},{sin(x)*(.7+.05*sin(6*x))});
\draw (axis cs:.8,.85) node {\scriptsize $r_2=f_2(\theta)$}
      (axis cs:.2,.8) node {\scriptsize $r_1=f_1(\theta)$};
\addplot [draw={\colorone},thick, smooth,domain=0:90,samples=30]
 ({cos(x)*(1+.05*cos(9*x))},{sin(x)*(1+.05*cos(9*x))});
\addplot[draw={\colorone},fill={\coloronefill},thick, smooth,domain=30:60,samples=30]
 ({cos(x)*(1+.05*cos(9*x))},{sin(x)*(1+.05*cos(9*x))})--(axis cs:0,0)--cycle;
\addplot[draw={\colortwo},fill=white,thick,smooth,domain=30:60,samples=30]
 ({cos(x)*(.7+.05*sin(6*x))},{sin(x)*(.7+.05*sin(6*x))})--(axis cs:0,0)--cycle;
\draw [thick,draw={\colorone},] (axis cs:0,0) -- (axis cs: 0.866025, 0.5)
 node [pos=.4,below,rotate=30,black] {\scriptsize $\theta=\alpha$};
\draw [thick,draw={\colorone},] (axis cs:0,0) -- (axis cs:0.475, 0.822724)
 node [pos=.4,above,rotate=60,black] {\scriptsize $\theta=\beta$};
\end{axis}
\node [right] at (myplot.right of origin) {\scriptsize $0$};
\node [above] at (myplot.above origin) {\scriptsize $\pi/2$};
\end{tikzpicture}}

Consider the shaded region shown in \autoref{fig:polarea3}. We can find the area of this region by computing the area bounded by $r_2=f_2(\theta)$ and subtracting the area bounded by $r_1=f_1(\theta)$ on $[\alpha,\beta]$. Thus
\[
\text{Area}
\ = \ \frac12\int_\alpha^\beta r_2^{\,2}\ d\theta
- \frac12\int_\alpha^\beta r_1^{\,2}\ d\theta
= \frac12\int_\alpha^\beta \big(r_2^{\,2}-r_1^{\,2}\big)\ d\theta.
\]

\keyidea{idea:area_between_polar}{Area Between Polar Curves}
{The area $A$ of the region bounded by $r_1=f_1(\theta)$ and $r_2=f_2(\theta)$, $\theta=\alpha$ and $\theta=\beta$, where $f_1(\theta)\leq f_2(\theta)$ on $[\alpha,\beta]$, is
\[
A = \frac{1}{2} \int_\alpha^\beta [f_2(\theta)]^2 - [f_1(\theta)]^2\ d\theta
= \frac12\int_\alpha^\beta \big(r_2^{\,2}-r_1^{\,2}\big)\ d\theta.
\]}

\mtable{Finding the area between polar curves in \autoref{ex_polcalc5}.}{fig:polcalc5}{\begin{tikzpicture}
\begin{axis}[width=1.16\marginparwidth,tick label style={font=\scriptsize},
axis y line=middle,axis x line=middle,name=myplot,axis on top,
ymin=-1.6,ymax=1.6,xmin=-.6,xmax=3.2]
\draw [fill={\coloronefill},draw={\colorone},thick, smooth,domain=0:180,samples=60]  (axis cs:1.5,0)circle(1.5);
% fill=white to cover up what we don't want
\addplot [fill=white,draw={\colortwo}, thick,smooth,domain=0:360,samples=60]
 ({cos(x)*(1+cos(x))},{sin(x)*(1+cos(x))});
% redraw the circle for the part that was covered
\draw [draw={\colorone},thick, smooth,domain=0:180,samples=60]
 (axis cs:1.5,0)circle(1.5);
\end{axis}
\node [right] at (myplot.right of origin) {\scriptsize $0$};
\node [above] at (myplot.above origin) {\scriptsize $\pi/2$};
\end{tikzpicture}}

\example{ex_polcalc5}{Area between polar curves}{Find the area bounded between the curves $r=1+\cos \theta$ and $r=3\cos\theta$, as shown in \autoref{fig:polcalc5}.}
{We need to find the points of intersection between these two functions. Setting them equal to each other, we find:
\begin{align*}
1+\cos\theta &= 3\cos \theta \\
 \cos\theta &=1/2\\
\theta &= \pm \pi/3
\end{align*}
Thus we integrate $\frac12\big((3\cos\theta)^2-(1+\cos\theta)^2\big)$ on $[-\pi/3,\pi/3]$.
\begin{align*}
	\text{Area}
	&= \frac12\int_{-\pi/3}^{\pi/3} \big((3\cos\theta)^2-(1+\cos\theta)^2\big)\ d\theta\\
	&= \frac12\int_{-\pi/3}^{\pi/3} \big( 8\cos^2\theta-2\cos\theta-1\big)\ d\theta \\
	&= \frac12\big(2\sin(2\theta) - 2\sin\theta+3\theta\big)\Bigg|_{-\pi/3}^{\pi/3}\\
	&= \pi.
\end{align*}
Amazingly enough, the area between these curves has a ``nice'' value.}

\mtable{Graphing the region bounded by the functions in \autoref{ex_polcalc6}.}{fig:polcalc6}{%
\begin{tikzpicture}
\begin{axis}[width=1.16\marginparwidth,tick label style={font=\scriptsize},
axis y line=middle,axis x line=middle,name=myplot,axis on top,
ymin=-1.15,ymax=1.15,xmin=-.5,xmax=2.2]
\addplot[draw={\coloronefill},fill={\coloronefill},area style,domain=30:45]
 ({cos(x)*(2*cos(2*x))},{sin(x)*(2*cos(2*x))})--(axis cs:{sqrt(3)/2},0)--cycle;
\addplot[draw={\coloronefill},fill={\coloronefill},area style,domain=0:30]
 ({cos(x)},{sin(x)})--(axis cs:0,0)--cycle;
\addplot [draw={\colorone},thick, smooth,domain=0:360,samples=90]
 ({cos(x)*(2*cos(2*x))},{sin(x)*(2*cos(2*x))});
\draw[draw={\colortwo},thick](axis cs:0,0)circle(1);
\end{axis}
\node [right] at (myplot.right of origin) {\scriptsize $0$};
\node [above] at (myplot.above origin) {\scriptsize $\pi/2$};
\end{tikzpicture}
\\(a)\\[10pt]
\begin{tikzpicture}
\begin{axis}[width=1.16\marginparwidth,tick label style={font=\scriptsize},
axis y line=middle,axis x line=middle,name=myplot,axis on top,
ymin=-.1,ymax=1,xmin=-.1,xmax=1.2]
\addplot[draw={\colorone},fill={\coloronefill},area style,domain=30:45]
 ({cos(x)*(2*cos(2*x))},{sin(x)*(2*cos(2*x))})--(axis cs:{sqrt(3)/2},0)--cycle;
\addplot[draw={\colortwo},fill={\colortwofill},area style,domain=0:30]
 ({cos(x)},{sin(x)})--(axis cs:0,0)--cycle;
\addplot [draw={\colorone},thick, smooth,domain=0:360,samples=90]
 ({cos(x)*(2*cos(2*x))},{sin(x)*(2*cos(2*x))});
\draw[draw={\colortwo},thick](axis cs:0,0)circle(1);
\draw [thick,dashed] (axis cs: 0,0) -- (axis cs:.866,.5);
\end{axis}
\node [right] at (myplot.right of origin) {\scriptsize $0$};
\node [above] at (myplot.above origin) {\scriptsize $\pi/2$};
\end{tikzpicture}
\\(b)}

% todo the region is also in the first petal and y>0
\example{ex_polcalc6}{Area defined by polar curves}{Find the area bounded between the polar curves $r=1$ and $r=2\cos(2\theta)$, as shown in \autoref{fig:polcalc6} (a).}
{We need to find the point of intersection between the two curves. Setting the two functions equal to each other, we have
\[
2\cos(2\theta) = 1 \quad \Rightarrow \quad \cos(2\theta) = \frac12
\quad \Rightarrow \quad 2\theta = \pi/3\quad \Rightarrow \quad \theta=\pi/6.
\]
In part (b) of the figure, we zoom in on the region and note that it is not really bounded \textit{between} two polar curves, but rather \textit{by} two polar curves, along with $\theta=0$. The dashed line breaks the region into its component parts. Below the dashed line, the region is defined by $r=1$, $\theta=0$ and $\theta = \pi/6$. (Note: the dashed line lies on the line $\theta=\pi/6$.) Above the dashed line the region is bounded by $r=2\cos(2\theta)$ and $\theta =\pi/6$. Since we have two separate regions, we find the area using two separate integrals.

Call the area below the dashed line $A_1$ and the area above the dashed line $A_2$. They are determined by the following integrals:
\[
A_1 = \frac12\int_0^{\pi/6} (1)^2\ d\theta\qquad
A_2 = \frac12\int_{\pi/6}^{\pi/4} \big(2\cos(2\theta)\big)^2\ d\theta.
\]
(The upper bound of the integral computing $A_2$ is $\pi/4$ as $r=2\cos(2\theta)$ is at the pole when $\theta=\pi/4$.)

We omit the integration details and let the reader verify that $A_1 = \pi/12$ and $A_2 = \pi/12-\sqrt{3}/8$; the total area is $A = \pi/6-\sqrt{3}/8$.}

\subsection*{Arc Length}

As we have already considered the arc length of curves defined by rectangular and parametric equations, we now consider it in the context of polar equations. Recall that the arc length $L$ of the graph defined by the parametric equations $x=f(t)$, $y=g(t)$ on $[a,b]$ is
\index{arc length}\index{polar!function!arc length}
\begin{equation}L = \int_a^b \sqrt{[\fp(t)]^2 + [g\primeskip'(t)]^2}\ dt = \int_a^b \sqrt{[x\primeskip'(t)]^2+[y\primeskip'(t)]^2}\ dt.\label{eq:polar_arclength}\end{equation}

Now consider the polar function $r=f(\theta)$. We again use the identities $x=f(\theta)\cos\theta$ and $y=f(\theta)\sin\theta$ to create parametric equations based on the polar function. We compute $x\primeskip'(\theta)$ and $y\primeskip'(\theta)$ as done before when computing $\frac{dy}{dx}$, then apply \autoeqref{eq:polar_arclength}.

The expression $[x\primeskip'(\theta)]^2+[y\primeskip'(\theta)]^2$ can be simplified a great deal; we leave this as an exercise and state that
\[
[x\primeskip'(\theta)]^2+[y\primeskip'(\theta)]^2 = [\fp(\theta)]^2+[f(\theta)]^2.
\]
This leads us to the  arc length formula.

\keyidea{idea:polar_arclength}{Arc Length of Polar Curves}
{Let  $r=f(\theta)$ be a polar function with $\fp$ continuous on an open interval $I$ containing $[\alpha,\beta]$, on which the graph traces itself only once. The arc length $L$ of the graph on $[\alpha,\beta]$ is
\[L = \int_\alpha^\beta \sqrt{[\fp(\theta)]^2+[f(\theta)]^2}\ d\theta = \int_\alpha^\beta\sqrt{(r\,')^2+ r^2}\ d\theta.\]
}

\example{ex_cardiod_length}{Arc Length of Polar Curves}{Find the arc length of the cardioid $r=1+\cos \theta$.}{With $r=1+\cos \theta$, we have $r' = -\sin \theta$. The cardioid is traced out once on $[0,2\pi]$, giving us our bounds of integration. Applying \autoref{idea:polar_arclength} we have
\begin{align*}
	L
	& = \int_0^{2\pi} \sqrt{(-\sin \theta)^2  + (1 + \cos \theta)^2}\ d\theta \\
	& = \int_0^{2\pi} \sqrt{\sin^2 \theta + (1 + 2\cos \theta + \cos \theta)}\ d\theta\\
	& = \int_0^{2\pi} \sqrt{2 + 2\cos \theta}\ d\theta\\
	& = \int_0^{2\pi} \sqrt{2 + 2\cos \theta}\ \frac{\sqrt {2-2\cos \theta}}{\sqrt {2-2\cos \theta}}\ d\theta\\
	& = \int_0^{2\pi} \frac{\sqrt{4 - 4\cos^2 \theta}}{\sqrt {2-2\cos \theta}}\ d\theta\\
	& = 2\int_0^{2\pi} \frac{\sqrt{1 - \cos^2 \theta}}{\sqrt {2-2\cos \theta}}\ d\theta\\
	& = 2\int_0^{2\pi} \frac{\abs{\sin\theta}}{\sqrt {2-2\cos \theta}}\ d\theta
\end{align*}
Since the $\sin \theta > 0$ on $[0, \pi]$ and $\sin \theta < 0$ on $[\pi, 2\pi]$ we separate the integral into two parts
\[
2\int_0^{\pi} \frac{\sin \theta}{\sqrt {2-2\cos \theta}}\ d\theta - 2\int_{\pi}^{2\pi} \frac{\sin \theta}{\sqrt {2-2\cos \theta}}\ d\theta
\]
Using the symmetry of the cardioid and $u$-substitution ($u = 2 - 2\cos \theta$) we simplify the integration to
\begin{align*}
L
& = 4\int_0^{\pi} \frac{\sin \theta}{\sqrt {2-2\cos \theta}}\ d\theta\\
& = 2\int_0^4 \frac{1}{\sqrt u}\ du\\
& = 4  u^{1/2}\biggr|_0^4 = 8.\eoehere
\end{align*}}

\example{ex_polcalc7}{Arc length of a lima\c con}{Find the arc length of the lima\c con $r=1+2\sin t$.}
{With $r=1+2\sin t$, we have $r\,' = 2\cos t$. The lima\c con is traced out once on $[0,2\pi]$, giving us our bounds of integration. Applying \autoref{idea:polar_arclength}, we have
\begin{align*}
	L
	&= \int_0^{2\pi} \sqrt{(2\cos\theta)^2+(1+2\sin\theta)^2}\ d\theta \\
	&=	\int_0^{2\pi} \sqrt{4\cos^2\theta+4\sin^2\theta +4\sin\theta+1}\ d\theta\\
	&=	\int_0^{2\pi} \sqrt{4\sin\theta+5}\ d\theta\\
	&\approx 13.3649.
\end{align*}
%
\mtable[-6\baselineskip]{The lima\c con in \autoref{ex_polcalc7} whose arc length is measured.}{fig:polcalc7}{\begin{tikzpicture}
\begin{axis}[width=1.16\marginparwidth,tick label style={font=\scriptsize},
axis y line=middle,axis x line=middle,name=myplot,axis on top,
ymin=-.5,ymax=3.2,xmin=-2.22,xmax=2.22]
\addplot [draw={\colorone},thick, smooth,domain=0:360,samples=90]
 ({cos(x)*(1+2*sin(x))},{sin(x)*(1+2*sin(x))});
\end{axis}
\node [right] at (myplot.right of origin) {\scriptsize $0$};
\node [above] at (myplot.above origin) {\scriptsize $\pi/2$};
\end{tikzpicture}}
%
The final integral cannot be solved in terms of elementary functions, so we resorted to a numerical approximation. (Simpson's Rule, with $n=4$, approximates the value with $13.0608$. Using $n=22$ gives the value above, which is accurate to 4 places after the decimal.)}

\subsection*{Surface Area}

The formula for arc length leads us to a formula for surface area. The following Key Idea is based on \autoref{idea:surface_area_parametric}.

\keyidea{idea:surface_area_polar}{Surface Area of a Solid of Revolution}
{Consider the graph of the polar equation $r=f(\theta)$, where $\fp$ is continuous on an open interval containing $[\alpha,\beta]$ on which the graph does not cross itself.
\index{surface area!solid of revolution}\index{polar!function!surface area}\index{integration!surface area}
\begin{enumerate}
	\item The surface area of the solid formed by revolving the graph about the initial ray ($\theta=0$) is:
	\[\text{Surface Area} = 2\pi\int_\alpha^\beta f(\theta)\sin\theta\sqrt{[\fp(\theta)]^2+[f(\theta)]^2}\ d\theta.\]
	\item The surface area of the solid formed by revolving the graph about the line $\theta=\pi/2$ is:
	\[\text{Surface Area} = 2\pi\int_\alpha^\beta f(\theta)\cos\theta\sqrt{[\fp(\theta)]^2+[f(\theta)]^2}\ d\theta.\]
\end{enumerate}}

\example{ex_polcalc8}{Surface area determined by a polar curve}{Find the surface area formed by revolving one petal of the rose curve $r=\cos(2\theta)$ about its central axis (see \autoref{fig:polcalc8}).}
{\mtable{Finding the surface area of a rose--curve petal that is revolved around its central axis.}{fig:polcalc8}{%
\begin{tikzpicture}
\begin{axis}[width=1.16\marginparwidth,tick label style={font=\scriptsize},
axis y line=middle,axis x line=middle,name=myplot,axis on top,xtick={-1,1},
ytick={-1,1},ymin=-1.1,ymax=1.1,xmin=-1.3,xmax=1.3]
\addplot [draw={\colorone},thick, smooth,domain=0:45,samples=30]
 ({cos(x)*(cos(2*x))},{sin(x)*(cos(2*x))});
\addplot [draw={\colortwo},thick, smooth,domain=45:360,samples=60]
 ({cos(x)*(cos(2*x))},{sin(x)*(cos(2*x))});
\end{axis}
\node [right] at (myplot.right of origin) {\scriptsize $0$};
\node [above] at (myplot.above origin) {\scriptsize $\pi/2$};
\end{tikzpicture}
\\(a)\\[10pt]
\myincludeasythree{width=\marginparwidth,
3Droll=0,
3Dortho=0.009,
3Dc2c=0.41893383860588074 -0.887881875038147 0.1901581883430481,
3Dcoo=65.85308837890625 6.341495513916016 17.98039436340332,
3Droo=85}{width=\marginparwidth}{figures/figpolcalc8a_3D}
\\(b)}
We choose, as implied by the figure, to revolve the portion of the curve that lies on $[0,\pi/4]$ about the initial ray. Using \autoref{idea:surface_area_polar} and the fact that $\fp(\theta) = -2\sin(2\theta)$, we have
\begin{align*}
\text{Surface Area}
&= 2\pi\int_0^{\pi/4} \cos(2\theta)\sin(\theta)
\sqrt{\big(-2\sin(2\theta)\big)^2+\big(\cos(2\theta)\big)^2}\ d\theta \\
&\approx 1.36707.
\end{align*}
The integral is another that cannot be evaluated in terms of elementary functions. Simpson's Rule, with $n=4$, approximates the value at $1.36751$.%; with $n=10$, the value is accurate to 4 decimal places.
}


This chapter has been about curves in the plane. While there is great mathematics to be discovered in the two dimensions of a plane, we live in a three dimensional world and hence we should also look to do mathematics in 3D --- that is, in \emph{space}. The next chapter begins our exploration into space by introducing the topic of \emph{vectors}, which are incredibly useful and powerful mathematical objects.

\printexercises{exercises/09_05_exercises}

