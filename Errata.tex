\errorcontextlines 10000

\documentclass{amsart}

\usepackage[margin=1in]{geometry}
\usepackage{comment}

\newif\ifinstructor
\instructortrue

\newcommand{\term}{Spring 2017}

\title{Errata to Apex LT \term}
\date{\today}

\newcommand{\ds}{\displaystyle}

\begin{document}

\maketitle

The following errors exist in the \term\ printed version of Apex LT:
\begin{itemize}
\item Examples are numbered within each section.  But when we refer to an example in a different section, it doesn't indicate that you should go to that section.  For example, in Section 6.3, ``Example 7'' should be ``Example 6.2.7''.
\item The prerequisite sections (1.0, 2.0, and 10.0) have the running header from the previous section.
\item Theorem 5 (Squeeze Theorem) requires the functions be defined and the squeezing to hold at the limit, which is unnecessary.
\item Example 1.3.7 should be $\ds \lim_{x\to 0} \frac{\sqrt{x+4}-2}{x}$.
\item Definition 2 (One Sided Limits) and Definition 8 (Left and Right Continuity) requires that the function be defined in an open interval around the point, whereas it only needs to be defined to the left or right.
\item Definition 9 (Continuity on Closed Intervals) has ``left'' and ``right'' reversed.
\item Example 3.1.6: ``are are $0$ and $\pm\pi$'' should be ``are $0$ and $\pm\sqrt\pi$''.
\item \S3.4 p175 ``maximizing $f'$ means finding the where'' shouldn't have ``the''.
\item Example 4.2.3 Line 1: ``an power station'' should be ``a power station''.
\item \S5.1 p222 Theorem 31 only holds on an interval.  Similarly, the paragraph after Definition 22 should begin ``Using Definitions 21 and 22, we can say that on an interval''.
\item The solution to \S7.3\#7 should be $\dfrac{1-x\ln x\ln5}{x5^x\ln5}$\smallskip
\item The solution to \S7.3\#21 should be $\dfrac{\ln\frac{245}3}{\ln3}=\dfrac{\ln245}{\ln3}-1$.
\item \S8.1 p383 ``If we had chosen\ldots $du=\sin x dx$'' should be $du=-\sin x dx$.
\item The solution to \S8.5\#1 should be $x\sin^{-1}x+\sqrt{1-x^2}+C$.  The given solution is $\ds\int x\sin^{-1}x dx$.\vspace{-.3\baselineskip}
\item The solution to \S8.5\#5 should be $\dfrac x{25\sqrt{x^2+25}}+C$.\vspace{-.5\baselineskip}
\item \S9.2 Theorem 64.  Part 1 assumes $r\neq1$.  The statement of part 2 assumes $S_n=\ds\sum_{k=0}^n ar^k$.\vspace{-\baselineskip} The proof assumes that $S_n=\ds\sum_{k=0}^{n-1} ar^k$ has $n$ terms.
\item \S9.2\#27--29: The index of summation should be $n$, not $i$.  The solutions for the latter two should be ``converges'', not a number.
\item \S9.3\#15 should be $\ds\sum_{i=3}^\infty\frac1{n\ln n[\ln(\ln n)]^p}$.
\item \S9.7\#9: The answer should be ``converges conditionally''.
\item \S9.7\#13 should be $\ds \sum_{n=1}^\infty \frac{1-\cos n}{n^3}$.
\item \S9.7\#15: The answer should be ``converges absolutely''.
\item \S9.8\#33: The desired answer is not a power series.
\item \S9.10\#15: Key Idea 32 gives the MacLaurin series for $\ln(x+1)$, and it converges on $(-1,1]$.  We can't expect students to prove convergence at the right endpoint (requires Abel's Theorem) or on $(-1,0)$ (requires Cauchy's form of the remainder) at this time.  (They can prove the series converges, but not what it converges to.)
\item The Conic Sections portion is supposed to be labeled 10.0.  The initial title is unlabeled, and the exercises are labeled 10.-1.  (Additionally, the running header is incorrect, as previously noted.)
\item The solution to \S10.5\#9(a) should use $\theta$, not $t$. (b) should have $(x-\sqrt3/4)$, not $(x+\sqrt3/4)$.
\end{itemize}

\begin{comment}
\section*{Fall 2016}

In addition to most of the previous (where applicable), the following errors exist in the Fall 2016 printed version of Apex LT:
\begin{itemize}
\item Sometimes, ``In Exercises \#--\#'' in exercise directions will go too far, and refer to exercises past the end of the intended range. This was an artifact of the way exercises were loaded.
\item p. 7, \S1.0 Exercises 11--14: The directions should be ``Graph the given f.''
\item p. 35, \S1.3 Example 1.3.7: The example should be: Evaluate $\lim_{x\to2}\frac{\sqrt{x+4}-2}x$. (But see the previous erratum in Spring 2017.)
\item p. 39, \S1.3\#33: The answer should be 1/6.
\item p. 47, \S1.4\#13: The answer to parts b and d should be 0.
\item p. 47, \S1.4\#23: $\ds\lim_{-2^+}f(x)=0$ should be $\ds\lim_{x\to-2^+}f(x)=0$.
\item p. 60, \S1.5\#31: This problem ``Reviews'' the next section, and should be there instead.
\item p. 62, \S1.6 Example 1.6.2: The floor function returns the largest integer smaller than or equal to the input.
\item p. 72, \S1.6\#41: The given answer of $a=1$, $b=−1$ is not continuous at $x=−1$. The correct answer is $a=\frac34$, $b=−\frac14$.
\item p. 76, \S2.0 Example 2.0.3: Part 2 should begin $(x^{−3})^4$. Part 3 should begin $(x^{−1/2})^{2/3}$.
\item p. 77, \S2.0 Example 2.0.5: The problem in part 1 should be $x^{7/3}−4x^{2/3}$.
\item p. 102, \S2.2 Exercise 17: From the graph, you cannot tell which function is the derivative of the other. The back of the book says this as well, but it would be better if you could tell.
\item p. 141, \S2.6 Example 2.6.4: On page 141, line 4, $2y^2$ should be $2x\cdot y^2$; line 5, $y^2$ should be $xy^2$.
\item p. 207, \S4.2\#19: The solution dimensions give a cost of 60.
\item p. 212, \S4.3\#38: ``Exercises 36'' should be singular.
\item p. 231, 232, \S5.1: Exercises 20, 21, 36, and 41 require $\int a^x dx$, although that hasn't yet been discussed. It should be removed.
\item p. 277 \S5.4\#11: Ditto.
\item pp. 304--306: The text states that moving a curve through space creates a solid, when it fact it creates a surface. To create a solid, we need to move a region through space, or look at the region enclosed by a surface. This leads to the following changes:
\begin{itemize}
\item p. 304, paragraph 2, line 2: ``a horizontal axis creates a three-dimensional solid'' should be ``a horizontal axis encloses a three-dimensional solid''.
\item p. 304, Key Idea 12: ``Let a solid be formed by revolving the curve'' should be ``Let a solid be enclosed by revolving the curve''.
\item p. 305, Example 6.2.2: The region being rotated is the one bounded by the curve $y=1/x$, $x=1$, $x=2$, and the $x$-axis.
\item p. 306, Example 6.2.3: The region being rotated is the one bounded by the curve $y=1/x$, $y=1$, $y=0.5$, and the $y$-axis.
\item p. 306: The paragraph following Example 6.2.3 states:
\begin{quote}
The previous two examples demonstrate how taking the same region and rotating it about two different axes will result in different solids and thus volumes.
\end{quote}
The examples do not have the same region. This sentence should be deleted.
\end{itemize}
\item p. 309, Figure 6.16: The label (b) should indicate the middle figure.
\end{itemize}
\end{comment}

\end{document}
