\documentclass{amsart}

\usepackage[margin=1.5in]{geometry}

\newcommand{\BBR}{\mathbb{R}}
\newcommand{\cvec}{\vec c}
\newcommand{\Fvec}{\vec F}
\newcommand{\dd}{d}
\newcommand{\dA}{\dd A}
\newcommand{\ds}{\dd s}
\newcommand{\dS}{\dd S}
\newcommand{\dt}{\dd t}
\newcommand{\du}{\dd u}
\newcommand{\dv}{\dd v}
\newcommand{\dV}{\dd V}
\newcommand{\dx}{\dd x}
\newcommand{\dy}{\dd y}
\newcommand{\dsvec}{\dd\vec s}
\newcommand{\dSvec}{\dd\vec S}
\newcommand{\grad}{\nabla}
\newcommand{\norm}[1]{\left\lVert #1\right\rVert}

\title{Vector Calculus Summary}

\begin{document}

\maketitle

Let summarize what we've learned in this chapter.

We introduced four new types of integrals, which are gathered in Table \ref{tab:ints}.
\begin{table}[!ht]
 \begin{tabular}{|l|p{.3\linewidth}|p{.4\linewidth}|}\hline
  & $\cvec:\BBR\to\BBR^3$\par parameterizes a curve
  & $\Phi:\BBR^2\to\BBR^3$\par parameterizes a surface \\\hline
  $f:\BBR^3\to\BBR$
  & scalar line integral:\par
  $\int_{c(D)} f\ds$\par
  \quad$=\int_D f(\cvec(t))\norm{\cvec~'(t)}\dt \vphantom{\dfrac11}$
  & scalar surface integral:\par
  $\iint_{\Phi(D)} f\dS$\par
  \quad$=\iint_D f(\Phi(u,v))\norm{\Phi_u\times\Phi_v}\du\dv \vphantom{\dfrac11}$
  \\\hline
  $\Fvec:\BBR^3\to\BBR^3\vphantom{\dfrac11}$
  & vector line integral:\par
  $\int_{c(D)}\Fvec\cdot\dsvec$\par
  \quad$=\int_D\Fvec(\cvec(t))\cdot\cvec~'(t)\dt \vphantom{\dfrac11}$
  & vector surface integral:\par
  $\iint_{\Phi(D)}\Fvec\cdot\dSvec$\par
  \quad$=\iint_D\Fvec(\Phi(u,v))\cdot(\Phi_u\times\Phi_v)\du\dv \vphantom{\dfrac11}$
  \\\hline
 \end{tabular}
 \caption{Summary of Integrating Parameterized Curves and Surfaces}
 \label{tab:ints}
\end{table}
If we are in the bottom left entry, where $\cvec:\BBR\to\BBR^3$ and
$\Fvec:\BBR^3\to\BBR^3$, we may be able to use the Fundamental Theorem for
Gradient Vector Fields: $\int_C\grad\phi\cdot\dsvec=\phi(Q)-\phi(P)$.

The three theorems in this chapter all relate an integral over a domain to an
integral over the boundary of the domain.  This means that we need to pay
special attention to the orientation of the domain.  We'll occasionally use the
symbol $\partial$ to indicate the boundary of something (so the boundary of $D$
would be $\partial D$).  This is the same symbol as a partial derivative, so
you'll have to look at the context to figure out which definition is being
used.

We also have four new theorems about these types of integrals.
These are similar to the Fundamental Theorem of Calculus, so we'll put all five
together into Table \ref{tab:fundThms}
\begin{table}[!ht]
 \begin{tabular}{|p{.25\linewidth}|l|p{.2\linewidth}|}\hline
  Fundamental Theorem of Calculus & \vphantom{$\dfrac11$}
  $f(b)-f(a)=\int_a^b f'(x)\dx$ & \\\hline
  Fundamental Theorem of Gradient Fields & \vphantom{$\dfrac11$}
  $\phi(Q)-\phi(P)=\int_C\grad\phi\cdot\ds$ & $C$ goes from $P$ to $Q$ \\\hline
  Green's Theorem \vphantom{$\dfrac11$} &
  $\oint_{\partial D}P\dx+Q\dy=\iint_D Q_x-P_y\dA$ &
  $\partial D$ oriented\par counterclockwise \\\hline
  Stokes' Theorem \vphantom{$\dfrac11$} &
  $\int_{\partial S}\Fvec\cdot\ds=\iint_S\grad\times\Fvec\cdot\dS$ &
  $\partial S$ oriented with\par $S$ to the left \\\hline
  Divergence (Gauss's)\par Theorem & \vphantom{$\dfrac11$}
  $\iint_{\partial W}\Fvec\cdot\dSvec=\iiint_W\grad\cdot\Fvec\dV$ &
  $\partial W$ oriented\par outwards \\\hline
 \end{tabular}
 \caption{Fundamental Theorems Relating Integrals and Domains}
 \label{tab:fundThms}
\end{table}

%\section*{Epilogue}
%
%You don't need to know the following for this course.  You may have noticed
%that the equations in Table \ref{tab:fundThms} have a similar flavor:
%the left side is the
%integral of a function over the boundary of a region (in some sense), while the
%right side is the integral of a derivative (in some sense) of the entire
%region.  There is actually a theorem unifying all of these into one equation.
%Also known as Stokes' Theorem, it says that
%$\int_{\partial M}\omega=\int_M\dd\omega$.  Of course, getting to the point
%where that equation makes sense takes a good deal of work
%(all the way to Math 432).

\end{document}