\documentclass{amsart}

\usepackage[margin=1in]{geometry}
\usepackage{booktabs}
\usepackage{url}
\usepackage{refcount}
\usepackage{comment}

\title{Errata to Apex LT}
\date{\today}

\newcommand{\ds}{\displaystyle}
\newcommand{\bracket}[1]{\left\langle#1\right\rangle}
\newcommand{\abs}[1]{\left\lvert#1\right\rvert}
\DeclareMathOperator{\sech}{sech}
\DeclareMathOperator{\divv}{div}

\newcounter{totalsI}\setcounter{totalsI}{0}
\newcounter{totalsII}\setcounter{totalsII}{0}
\newcounter{totalsIII}\setcounter{totalsIII}{0}

\makeatletter
\newcommand{\errorboxed}[2]{%
 \@ifundefined{r@#1#2plus}{%
  \@ifundefined{r@#1#2}{%
  }{% plusless defined
   \makebox[2em][r]{\ref{#1#2}\phantom{+}}%
   \addtocounter{totals#2}{\getrefnumber{#1#2}}%
  }%
 }{% plus defined
  \makebox[2em][r]{\ref{#1#2plus}{+}}%
  \addtocounter{totals#2}{\getrefnumber{#1#2plus}}%
 }%
}
% as part of a write, errortext needs to be expandable
% getrefnumber is. ref isn't
\newcommand{\errortext}[2]{%
 \@ifundefined{r@#1#2plus}{%
  \@ifundefined{r@#1#2}{%
  }{% plusless defined
   \getrefnumber{#1#2}%
  }%
 }{% plus defined
   \getrefnumber{#1#2plus}+%
 }%
}
\makeatother

\newcommand{\errorTotals}{%
 Total &
 \makebox[2em][r]{\arabic{totalsI}{+}} &
 \makebox[2em][r]{\arabic{totalsII}{+}} &
 \makebox[2em][r]{\arabic{totalsIII}{+}}%
 \write\markdown{---|---|---|---}%
 \write\markdown{Total|\arabic{totalsI}+|\arabic{totalsII}+|\arabic{totalsIII}+}%
}

\newcommand{\errorrow}[1]{%
 #1 & \errorboxed{#1}{I} & \errorboxed{#1}{II} & \errorboxed{#1}{III}%
 \write\markdown{#1|\errortext{#1}{I}|\errortext{#1}{II}|\errortext{#1}{III}}%
}

\newcommand{\startMarkdownTable}{
 \newwrite\markdown
 \openout\markdown=README.md
 \write\markdown{Errata Totals}
 \write\markdown{=====================}
 \write\markdown{}
 \write\markdown{If you are interested in a particular error, you can look through [Errata.tex](Errata.tex) (or compile it to Errata.pdf).  That will also list the total number of errors for each edition, which we repeat here:}
 \write\markdown{}
 \write\markdown{Version | Calculus I | Calculus II | Calculuc III}
 \write\markdown{---|---|---|---}
 %
 \AtEndDocument{
  \write\markdown{}
  \write\markdown{"+" indicates a systemic error.\space\space}
  \write\markdown{(The totals for a row may be higher than what's listed in [changes.md](../changes.md) due to double counting.)}
  \closeout\markdown
 }
}

\begin{document}

%\vspace*{-.6in}

\maketitle

\noindent
The following errors exist in the in June 2021 printed version of Apex LT Calculus I:
\begin{enumerate}
\item \S1.5 p57 Example 7\#3a: The limit should be $-\infty$.
\item At the back of the book, integration rule \#2 should enclose its integrand in parentheses, \#23 should have $a>0$ or use $|a|$, and \#31 should have its result in absolute values.
\label{2021-06-00I}
\end{enumerate}

\noindent
The following errors exist in the in June 2021 printed version of Apex LT Calculus II:
\begin{enumerate}
\item \S7.1 p348 \#5: The solution's line segment should go through $(5,-2)$.
\item \S7.3 Example 7.3.3\#3 p361: The $C$ disappears at the end.
\item \S7.4 Key Idea 7.4.4\#1 p371: $\cosh^{-1}$ is only defined for positive arguments, so we don't need the absolute value.
\item \S7.5 Example 7.5.1\#3 p377: We need to have a $\lim_{x\to 0}$ in front of the $\frac2{\cos x}$.
\item \S8.2 Key Idea 8.2.2\#4 p400: $\tan^{m-2}x$ is missing its $x$.
\item \S8.3 Key Idea 8.3.1 identifies its cases by 1, 2, and 3, but subsequent examples refer to a, b, and c.
\item \S8.6 p436: ``An improper integral is said to \textbf{converge} if its corresponding limit exists'' \emph{and is finite}.
\item \S8.7 p456 Theorem 1: The variable $n$ is representing two different things.  For that matter, so is $M$.
\item \S9.2 Theorem 9.2.1 p479: We need to specify that $a\neq 0$.
\item \S9.3 p490: $[n,\infty]$ should be $[n,\infty)$.
\item \S9.4 p496: ``For $n\ge N$, We're now'' should be lowercased.
\item \S9.4 Example 9.4.2 p498: The stated inequality should be for all $n\ge 2$.
\item \S9.4 Example 9.4.3 p500: ``Harmonic Sequence'' should be ``Harmonic Series''.
\item \S9.7 p523 \#6: If the terms contain a factorial, then the divergence test might work.  (The root test is also a possibility, if you believe Knewton.)
\item \S9.9 p546: ``pattern that make their formation'' should be ``makes''.
\item \S9.9 Theorem 9.9.1 p549: The theorem defines $R_n(x)$ and should be stated as such.
\item \S9.9 Example 9.9.3\#1 p550: The seventh derivative is $y=+6!/x^7$.
\item \S9.9 Example 9.9.3\#2 p550: ``[1,2] again'' should be ``[1,2] is again''.
\item \S9.9 Example 9.9.3\#2 p550: $720/5040\approx1/7$, with subsequent changes.
\item \S9.9 Example 9.9.4 p552: ``guarantee'' should be ``guarantees''.
\item \S9.9 p554 \#28: We want to approximate $\sin\pi$, not $\cos\pi$.
\item \S9.10 p557: ``not the case'' should be ``not necessarily the case''.
\item \S9.10 Example 9.10.6 p562: The Taylor expansion for $\ln x$ given in Key Idea 9.10.1 is centered at $x=0$.  We can use Theorem 9.10.2 with $h(x)=x-1$.
\item \S10.1 p580: The surface area approximation is not actually a Riemann Sum, because we use two different points in the same interval.
\item \S10.2 Example 10.2.13 p594: ``on a wheel of radius $r$ as starts rolling" should be ``as it starts rolling''.
\item \S10.3 p608: ``the parametric equation defining these curves'' should be ``equations''.
\item \S10.4 Example 10.4.3 p615: $\tan\pi/4$ should be $\tan(\pi/4)$.
\item \S10.5 Example 10.5.2 p627: ``in polar,'' should be ``in polar coordinates,''.
\item \S10.5 Example 10.5.7 p632: $1+2\cos\theta+\cos\theta$ should have $\cos^2\theta$.
\item At the back of the book, integration rule \#2 should enclose its integrand in parentheses, \#23 should have $a>0$ or use $|a|$, and \#31 should have its result in absolute values.
\label{2021-06-00II}
\end{enumerate}

\clearpage

\noindent
The following errors exist in the in June 2021 printed version of Apex LT Calculus III:
\begin{enumerate}
\item \S12.1 p717 Example 3: In the last displayed equation, ``Multiplying $\vec r(t)$ by 5'' should produce $5\vec r(t)=\langle5\cos t+t,5\sin t+\frac32t\rangle$.
\item \S13.5 p815 Example 6: In the final fraction, the ``$+3y^2$'' should be in the denominator, not the numerator.
\item \S14.7 p928, 931: ``So that each point in space that does not lie on the $z$-axis is defined uniquely'', we need $0\le\theta<2\pi$, not $0\le\theta\le2\pi$.
\item \S14.7 p931 Example 4: We need to divide by $M$.  It also turns out that we can compute these integrals exactly.
\item \S15.1 p949 line 5: We should have $\beta^{-1}$ instead of $\alpha$.
\item \S15.1 p951 Example 3: $\sqrt{1+t^2}$ should be $\sqrt{1+4t^2}$.  The final answer is correct.
\item \S15.4 p986 Example 3: In the last line, $\vec Fd\vec r$ should be $\vec F\cdot d\vec r$.
\item \S15.4\#27 p 989: $\vec f$ should be $\vec F$.
\item \S15.7 p 1021 Key Idea 15.7.1: In the $\mathbb{R}^2$ Divergence Theorem, the left hand side should have $\oint_{\partial R}$, not a double integral.  In Green's Theorem, $\vec F$ is missing its vector arrow.
\item At the back of the book, integration rule \#2 should enclose its integrand in parentheses, \#23 should have $a>0$ or use $|a|$, and \#31 should have its result in absolute values.
\label{2021-06-00III}
\end{enumerate}

\startMarkdownTable
\begin{table}[ht]
\caption{Errata Tally (``+'' indicates systemic errata)}
\begin{tabular}{lccc}\toprule
Version & Calculus I & Calculus II & Calculus III \\\midrule
\errorrow{2021-06-00} \\
\errorrow{2019-06-00} \\
\errorrow{2018-07-13} \\
\errorrow{2017-11-13} \\
\errorrow{2017-07-27} \\
\errorrow{2017-05-00} \\
\errorrow{2017-01-00} \\
\errorrow{2016-08-00} \\
\midrule
\errorTotals \\
\bottomrule
\end{tabular}
\end{table}

\noindent
The following errors exist in the in June 2019 printed version of Apex LT Calculus I:
\begin{enumerate}
\item \S1.4 p41: Theorem 1.4.1 needs to say ``except possibly at $c$''.
\item \S1.6\#14,16: To be continuous at a point, the function needs to be defined in a neighborhood of the point.  This means that the functions are not continuous at the indicated points.
\item \S3.3: A careful reading of Definition 3.3.1 shows that intervals of increasing and decreasing are usually closed within the domain.  We often had open intervals instead.
\item \S5.3 p257 Line -6: There should be a $\ds\sum_{i=1}^n$ between the limit and the parentheses.
\item \S5.4 p262: In order for Theorem 5.4.1 (FTC1) to help with the MVT later on, this needs to also include that $F$ is continuous at the endpoints.
\item In the integration formulas at the back of the book, \#13 is missing its $dx$.
\label{2019-06-00Iplus}
\end{enumerate}

\noindent
The following errors exist in the in June 2019 printed version of Apex LT Calculus II:
\begin{enumerate}
\item \S7.4 p364 Line -1: as $x\to\infty$, both $\sinh x$ and $\cosh x$ approach $e^x/2$.
\item \S7.5 p382\#40: $\ds\lim_{x\to1^+}$ should be $\ds\lim_{x\to1^-}$.
\item \S8.2 p402 Line 10: The is an extraneous $dx$ at the end of the line.
\item \S8.6 p438 Figure 8.6.3: The graph is $f(x)=1/x$, not $1/x^2$.
\item The chapter headings in \S10.2--10.5 show Chapter 9.
\item \S10.3 p608\#40: In order to be integrable, we should have $y=4e^{t/2}$.  The length is then $e^3+11-e^{-8}$.
\item In the integration formulas at the back of the book, \#13 is missing its $dx$.
\label{2019-06-00II}
\end{enumerate}

\noindent
The following errors exist in the in June 2019 printed version of Apex LT Calculus III:
\begin{enumerate}
\item \S11.2 p659 Definition 11.2.3 refers to $c\vec v$ as a scalar product, whereas most authors use ``scalar product'' as a synonym for the dot product.
\item \S11.2 p660 Line 17: This is the definition of $\vec u-\vec v$; it does not follow from anything.
\item \S11.4 p691 Example 11.4.6: The final equation is missing an equal sign: $\dotsm0(1-0)1\dotsb$ should be $\dotsm0=(1-0)1\dotsb$.
\item \S12.1 p720\#29: In the answer, the $z$-component should be $ht$.
\item \S13.5 p817\#3: We need to state how the functions are defined.
\item \S13.7 p839\#24: The equation should be $\sin(xy)+\cos(yz)=1$.
\item \S13.8 Definition 13.8.1 part 1: We need $D\subset S$.
\item \S14.5: We need to refer to the surface area of $z=f(x,y)$, not $f(x,y)$.
\item \S14.6 Example 2: When the order of integration is $dz\,dy\,dx$, the outer differential should be $dx$, not $dz$.
\item \S14.6 Example 4: The definition of $D$ should clarify that we want $0\le z\le-y$.
\item \S14.7 p944\#12: Cylindrical should be Spherical.
\item \S14.7 p944\#15 solution: $z=1$ should be $z=0$.
\item \S14.7 p945\#19--24: The differentials should be in the order $d\rho d\phi d\theta$.
\item \S15.3 p970 Line 8: We should have $\vec r'_1(t)=\bracket{1,1}$, not $\bracket{1,2}$.
\item \S15.4 p991\#31: The curve should be given by $\vec r(t)=\bracket{3t^2-2t-t^3,2(t-1)^2}$.
\item \S15.5 p1002\#9 solution: For $x=1$, the $x$-component to $\vec r$ should be 1.
\item \S15.5 p1002\#13: The vertical surface is given by $x^2+y^2/9=1$.
\item \S15.5 p1003\#17: $2\le v\le3$ should be $2\le y\le3$.
\item \S15.5 p1003\#27 solution: The integral should be $\int_{-1}^1\int_0^{1-v^2}\sqrt{27}\ du\ dv$. The exact value of the integral is $4\sqrt3\approx6.928$.
\item \S15.6 p1012\#5: The mass function should be $\delta(x,y,z)=z+10$ in order to give the book's answer and maintain a positive mass density.
\item \S15.7 p1024 solution: The flux through the plane is 22.
\item In the integration formulas at the back of the book, \#13 is missing its $dx$.
\label{2019-06-00III}
\end{enumerate}

%\newpage

\noindent
The following errors exist in the July 13, 2018 printed version of Apex LT Calculus in the Important Formulas at the end of the book:
\begin{enumerate}
\item In Algebra / Binomial Theorem, the summation should have an index of $k$.
\item In Algebra / Factoring by Grouping, $(cs+d)$ should be $(cx+d)$.
\item As Additional Formulas / Taylor Series Expansion is an infinite series, it should have $+\dotsb$ after the final term.
\label{2018-07-13}
\end{enumerate}\bigskip

In addition to 1--\ref{2018-07-13}, the following errors exist in the July 13, 2018 printed version of Apex LT Calculus I:
\begin{enumerate}\setcounter{enumi}{\getrefnumber{2018-07-13}}
\item \S1.1 p12 Line 8: $\sin(x)/x$ should be $\sin(1/x)$.
\item \S2.0 p75 Example 2.0.5 Solution part 2: $(1+2(x-3))$ should be $(x+2(x-3))$, with subsequent appropriate changes.
%\item \S2.3 p101 Line 6: ``linear function'' should be pluralized.
\item \S2.5\#35: The portion after the second $=$ is the derivative, not a rewriting of the original function.
\item \S3.1 Theorem 3.1.1: $I$ should be a \emph{finite} closed interval.
\item \S3.1\#16: The graph is $\sqrt[3]{x^4-2x^2+1}$.
\item \S3.2\#3--10: In order to match Rolle's Theorem, the final $[a,b]$ should be $(a,b)$.
\item \S3.5 Example 3.5.2 Solution Item 5: $(x^2+x-6)^2$ should be $(x^2-x-6)^2$.
\item \S4.1\#5: ``circular'' should be ``spherical''.
\item \S4.3 p205 Line -2: $\Delta y$ ended up with a spurious quote mark in front.
\item \S4.3 p207 Example 4.3.1 Solution Line 4: $\Delta y$ ended up with a spurious quote mark in front.
%\item \S4.4 p218 Line -1: ``Be able to do so'' should be ``Being able to do so''.
\item \S5.2\#9: The figure should say $f(x)=\sqrt{4-(x-2)^2}$; the expression was truncated.
\item \S5.4 p262 Line -4: ``denominator'' should be ``numerator''.
\item \S5.4 p263 Line 8: ``the Comparison Properties of Integrals'' should be ``Theorem 5.3.3''.
\item \S6.1\#6--8: The top functions' definitions were truncated.  They should be: \#6: $y=-3x^3+3x+2$; \#7 : $y=2$; and \#8: $y=\sin x+1$.
%\item \S6.1\#29--32: ``area triangle'' should be ``area of the triangle''.
\item \S6.3 Example 6.3.4 solution line 3: $h(y)=y=y^2$ should be $h(y)=y-y^2$.
%\item \S6.4\#27: ``an truncated'' should be ``a truncated''.
%
%% the rest are errors in the solutions
\item \S1.0\#8: The solution should be $(-\infty,-2)\cup(2,\infty)$.
\item \S1.0\#20: In the solution, $(-\infty,5.677)$ should be $(-\infty,5.677]$.
\item \S1.1\#17: In the solution, the limit should seem to be exactly $-7$.
\item \S2.0\#3: The solution should not have a negative sign.
\item \S2.1\#31--32: The numbers in the instructions should be letters; the second set should be d,e,f.  In order to match the solutions, the inequalities for d and e should be switched.
\item \S2.1\#33: The solution should be negated.
\item \S2.3\#39c: The solution should be $t=0$ and $t=\sqrt2$.
\item \S2.4\#27: The solution should not have parts (b) or (c) (and no label ``(a)'').
\item \S2.4\#29: The solution should be $-2x-5-\dfrac{10}{x^2}=-\dfrac{2x^3+5x^2+10}{x^2}$.
\item \S2.4\#39: The solution should use $h'$, not $f'$.
\item \S2.4\#61b: The solution should be $\frac1{18}$.
\item \S2.5\#27: In the solution, $(2x+1)10$ should be $(2x+1)^{10}$.
\item \S2.5\#41c: The solution should be $-3$.
\item \S2.6\#19 solution: The $\sin y^3$ in the denominator should be $\cos y^3$.
\item \S2.6\#25b solution: $+\sqrt[4]{108}$ should be $-\sqrt[4]{108}$.
\item \S3.2\#27: 0 should not be included in the solution.
\item \S3.3\#33 solution: It is decreasing on $(-\infty,-1)$.
\item \S4.4\#15: The given problem has a unique solution at $x=0$.  The given solution works for $g(x)=\cos x+1$.
\item \S5.1\#29 solution: $+\frac29 x^3$ should be $-\frac29 x^3$.
\item \S5.2\#23 solution: The answer should be $22$.
\item \S5.2\#25 solution: The answer should be $0$.
\item \S5.3\#49 solution: Since the curve is decreasing, we should have (a)left$>$(c)midpoint$>$(b)right, swapping (b) and (c).
\item \S5.4\#25 solution: The answer should be $63/2$.
\item \S5.5\#33 solution: For $\abs x<1$, we need $\frac12 \ln\abs{\ln \left(x^2\right)}+C$ (absolute value).
\item \S6.1\#31: The solution should be 0, or a $(3,3)$ should become $(-1,3)$.
\item \S6.1\#32 solution: The area should be $\frac{21}2$.
\item Example 6.3.4 solution line 3: ``$h(y)=y=y^2$'' should be ``$h(y)=y-y^2$''.
\item Example 6.4.1 solution line 2: ``it is'' should be ``is it''.
\item \S6.4\#5b solution: The answer is a length, so the units should just be ft.
\label{2018-07-13I}
\end{enumerate}\bigskip

%\newpage

In addition to 1--\ref{2018-07-13}, the following errors exist in the July 13, 2018 printed version of Apex LT Calculus II:
\begin{enumerate}\setcounter{enumi}{\getrefnumber{2018-07-13}}
\item It turns out that the units newton and joule are not capitalized.
\item \S7.2 Example 7.2.2 solution part 3: The derivative is $\dfrac{-\sin x}{\abs{\sin x}}$.
\item \S7.2 Example 7.2.3 solution part 2: The first integral is missing its $dx$.
\item \S7.3 Example 7.3.4 solution line -1: The tangent line should be $y=2.582(x-1.5)+1.837$.
\item In Exercise 7.3\#2, the argument to the function should be $t$, not $x$.
\item In Exercise 7.3\#6, the argument to the function should be $r$, not $x$.
\item \S7.4, page 368, line 8: $\sqrt{4y^2-4}$ should be $\sqrt{4y^2+4}$.
\item \S8.6, page 438, line -10: $\ds\lim_{t\to0^-}\left(-\dfrac1t+1\right)$ should be $\ds\lim_{t\to0^-}\left(-\dfrac1t-1\right)$.
\item \S8.6, page 439, line 9: ``power of $b$'' should be ``power of $t$''.
\item \S8.6, page 441, line -6: ``the quadratic function'' should be ``the square root of a quadratic function''.
\item \S8.6, page 442, Figure 8.6.12: The graph labels should be switched.
\item \S9.1, page 467, line 8: The numerator $n^2+4n+1$ should be $n(n+3)$.
\item \S9.1, page 468, lines 3\&4: $-10n^2+60n+55$ should be $-10n^2+60n-55$.
\item \S9.2, page 476, line 2: $\left(\dfrac12\right)^2$ should be $\left(\dfrac12\right)^n$.
\item \S9.2, page 481, line -7: The terms must approach, or the limit must be, but the limit cannot approach.
\item \S9.3, page 490, line -6: The series converges to $-\ln 2$.
\item In Exercises 9.3\#11--14, the index of summation should be $n$, not $i$.
\item \S9.4, page 492, line -4: We don't necessarily have $S_n\le T_n$, because we can't control what happened for $n<N$.  The entire proof is affected.
\item \S9.4, page 494, line 2: The series should start at $n=2$.
\item \S9.4, page 496, lines 1\&5: $\ds\lim_{n\to\infty}a_n$ should be $\ds\lim_{n\to\infty}\dfrac{a_n}{b_n}$.
\item \S9.5, page 503, line 11: ``converges to $s$'' should be ``converges to $L$''.
\item \S9.5, page 507, para 1: The part this is referring to was removed, so that this statement is now false.
\item \S9.5, page 509 line 4: The inequality with $L$ is in the wrong order.
\item In Exercises 9.5\#14\&18, the summation should begin at 2.
\item In Example 9.6.3\#2, the summation should begin at 2.
\item In Exercises 9.6\#25\&26, the summation should begin at 2.
\item \S9.7, page 518, line -4: The inequality is backwards; the series needs the Limit Comparison Test.
\item In Exercise 9.7\#18, the summation should begin at 2 and use the index $n$.
\item In Exercises 9.7\#34\&36, the summation should begin at 2.
\item \S9.8, page 534, line -6: $g(x)=\dfrac1{(1-x)^2}$.
\item \S9.9, page 541, line 5: The final numerator for $p_n(x)$ should be $(-1)^{k+1}$.
\item In Theorem 9.9.1, the first displayed equation should be
\[f(x) = \sum_{k=0}^n\frac{f\,^{(k)}(c)}{k!}(x-c)^k+R_n(x).\]
The proof needs $I$ to be an open interval.  The maximum is then over all $z$ between $x$ and $c$, and requires that $f^{(n+1)}$ be continuous.  Changing the argument for the maximum simplifies Examples 9.9.3\#1,3 and 9.9.5\#3 as well as the solutions to Exercises 9.9\#23--25.
\item In Example 9.10.4 on page 552, 3 lines from the end of the example, when $-1<k<0$, the interval of convergence is $(-1,1]$.
\item In Key Idea 9.10.1 on page 553, second entry from the end, when $-1<k<0$, the interval of convergence is $(-1,1]$.
\item \S9.10, page 555: Analyticity lets you go from a function to its power series.  To go the other direction, you need the converse theorem.  Something like: ``If a function is equal to some power series on an interval, then that power series is the Taylor series of the function.''  Eg, Example 9.10.5 gives a power series for $e^x\cos x$, but we don't have a way to know that it is the Taylor series.
\item \S10.2, page 579, lines -1, -10, \& -17: The vertex occurs at $(.5,-6.25)$.  The $y$-coordinate is incorrect all three times; the $x$-coordinate once.
\item \S10.4, page 613, para 4, line 3: 2\textsuperscript{nd} quadrant should be 1\textsuperscript{st} quadrant.
\item \S10.5, page 619, line -1: The denominator should be $f'(\alpha)\cos\alpha-f(\alpha)\sin\alpha$.
\item In Example 10.5.8, the parameter switches from $t$ to $\theta$.
% solution errata
\item \S7.2\#25: The solution should be $-\pi/6$.
\item \S7.4\#13: The solution should be $2x\sech^2(x^2)$.
\item \S8.5\#43: The solution should be $\ds-\ln x+\frac4{\sqrt[4]{x}}+4\ln\abs{1-\sqrt[4]{x}}+C$.
\item \S8.5\#45: The solution should be $\ds\frac{-x}{2(25+x^2)}+\frac1{10}\tan^{-1}\bigl(\frac x5\bigr)+C$.
\item \S9.2\#33a: The solution should have the negative with $n$ odd, not $n$ even.
\item \S9.2\#35: The solution should be (a) $S_n=\dfrac{e^{-1}-e^{-n-1}}{1-e^{-1}}=\dfrac{1-e^{-n}}{e-1}$, (b) $\dfrac{e^{-1}}{1-e^{-1}}=\dfrac1{e-1}$.
\item \S9.2\#39a: The solution should be $a_n = \frac12\left(\frac{1}{2n-1}-\frac{1}{2n+1}\right)$, so that $S_n=\frac12\left(1-\frac1{2n+1}\right) = \frac{n}{2n+1}$.
\item \S9.4\#5: The solution should specify the comparison ``for all $n\ge 3$''.
\item \S9.4\#23: The solution should specify $n\ge3$, not $n\ge2$.
\item \S9.8\#33 b and c: The first summations in the solutions should start at $n=1$, not $n=0$.
\item \S9.9\#5: The solution should be $p_3(x)=1-x+\frac12 x^2-\frac16 x^3$.
\item \S9.9\#26: The derivative is not bounded in $n$.  A correct solution: The $n^\text{th}$ derivative of $f(x)=\sqrt x$ has a maximum on $[3,4]$ of $(2n-3)!!(-1)^{n+1}3^{1/2}6^{-n}$.  Thus $\abs{R_n(3)}\leq \frac{3^{1/2}2^{-2}}{3^n n(n+1)}$. When $n=5$, this is less than $0.0001$.
\item \S10.2\#65: The solution should be $t=k\pi$ for $k\in\mathbb{Z}$.
\item \S10.4\#57: The solution should also include the origin.
\label{2018-07-13IIplus}
\end{enumerate}\bigskip

In addition to 1--\ref{2018-07-13}, the following errors exist in the July 13, 2018 printed version of Apex LT Calculus III:
\begin{enumerate}\setcounter{enumi}{\getrefnumber{2018-07-13}}
\item It turns out that the units newton and joule are not capitalized.
\item \S11.1\#29,30: The subsequent image has clipped the problem statement.\\
 \#29 should have (a) $\ds x=y^2+\frac{z^2}{9}$; (b) $\ds x=y^2+\frac{z^2}{3}$.\\
 \#30 should have (a) $\ds x^2-y^2-z^2=0$; (b) $x^2-y^2+z^2=0$
\item \S12.4: The caption in Figure 12.4.4 should refer to Example 12.4.3.
\item \S13.8 Example 13.8.7 p843 l5: These are simultaneously 0 only when $w=0$.
\item \S13.8 Example 13.8.7 p843 para4: This gives a volume of approximately 20{,}347.
\item \$15.2\#25: The problem statement should ask you to show that $\divv\,(\nabla f\times\nabla g) = 0$.
\item \$15.3 p968 line 3: ``A region is simply connected\ldots'' should be ``A connected region is simply connected\ldots''.
% solution errata
\item \S14.3\#17 solution: The answer should be $3\pi/4-9\sqrt3/16$.
\item \S14.3\#21 solution: The first answer should be $2(1-a^3)/3(1-a^2)$.
\item \S15.1\#17 solution: The answer should be $(17\sqrt{17} - 5\sqrt{5})/3$.
\label{2018-07-13IIIplus}
\end{enumerate}

\section*{Digital Math Resources}

We are pleased to announce that the 3d files are now available online, linked from the book's distribution page at \url{https://arts-sciences.und.edu/academics/math/calc-1-texts.html} (the page currently links to \url{https://sites.und.edu/timothy.prescott/apex/prc/}, but that may change in the near future).  This makes it possible to use the mobile app ``3D PDF Reader'' by Tech Soft 3D to view the files on an Android or iOS device.  Appropriate links to the apps are provided on the page.  A single 3d file is several hundred kilobytes, so this could zap your data allotment if you're not careful.

%\end{document}

%\newpage

In the July 27, 2017 and November 13, 2017 printed versions of Apex LT Calculus, there are numerous instances of ``$\lim A+B$''.  The convention seems to be that this should be ``$\lim(A+B)$''.  We believe that all such instances have been corrected for subsequent versions of the text.\bigskip

\noindent
The following errors exist in the November 13, 2017 printed version of Apex LT Calculus III:
\begin{enumerate}
\item \S11.1 p637 Example 5 Line 1: ``cylinder following cylinders'' should be ``following cylinders''.
\item \S11.2\#28--31: These should refer to the magnitude of the force.
\item The solution to \S11.7\#11 should be negated.
\item \S12.3 p737 Example 3 \# 2 Problem \& Solution: The occurrences of ``derive'' don't really fit the definition.
\item \S12.5 p757 Example 1 Lines -1 \& -5: $s(2)$ should be $\vec r(2)$.
\item \S13.5 p813 Line -2: $-1$ or ``negative'' need to be involved.
\item \S13.7 p836 Exercise 3: The scalars $f_x$ and $f_y$ should be replaced by $\ell_x$ and $\ell_y$.
\item \S14.4 p896 \#15,23,\&30: The regions are disks, not circles.
\item \S14.6 p907 Example 2 Line 3: This is the first reference to the 1\textsuperscript{st} octant, which should have been defined in \S11.1.
\item \S15.1 p940 Paragraph 1: The $C_i$ can overlap only at isolated points.
\item \S15.3 p954 4 Lines after Example 2: The reference to Exercise 8 should be Exercise 9.
\item \S15.4 Exercises 1--4, 7: $d\sigma$ should be $d\vec\sigma$.  The answer to \#7 should be $7/4$.
\label{2017-11-13IIIplus}
\end{enumerate}\vspace{1in}

The following errors exist in the July 27, 2017 printed versions of Apex LT Calculus I:
\begin{enumerate}
\item \S1.3 p33 Example 5 Solution Line 1: We would like to use Theorem 3, not 4.
\item \S1.3 p36 Example 8 Line -2: This uses Theorem 3, not 4.
\item \S1.4\#25,26: The limit should not be a part of the functions' definition.
\item \S1.6\#41: We should have $f(x)=a^2x-a$ for $x>3$.
\item \S4.1\#14: Part (c) (walking 40') occurs after part (d) (the weight reaches the pulley).
\item \S5.1\#38: $\ds \int \frac{\sin\theta+\sin\theta\tan^2\theta}{\sec^2}\ d\theta$ should be $\ds \int \frac{\sin\theta+\sin\theta\tan^2\theta}{\sec^2\theta}\ d\theta$.
\item \S5.2\#21,25: These should specify $a$ and $b$ are non-zero.  For consistency with the previous problems (and the solution), 25 should be $\ds\int_0^5 \big(as(t)+br(t)\big) \ dt=0$.
\item \S6.2 p304 Example 5: ``Figure (not yet created)'' should be ``(b)''.
\item \S6.3 p314 Example 5: $\dfrac{4\pi}3$ should be $\dfrac{8\pi}3$ units\textsuperscript3.
\item \S6.3 p314 Example 5: The washer approach at the end is incorrect.
\label{2017-07-27Iplus}
\end{enumerate}

%\clearpage

The following errors exist in the July 27, 2017 printed versions of Apex LT Calculus II:
\begin{enumerate}
\item \S7.3\#17: The answer should be $\frac{3^{x^2-1}}{2\ln3}+C$.
\item \S8.1 p385 Example 6 Line -3: The $\ln(1+x^2)$ at the end should be $\frac12\ln(1+x^2)$.
\item \S8.1\#25 Solution: The second term $x\left(\ln\abs{x+1}\right)$ should be $x\left(\ln\abs{x+1}\right)^2$.
\item \S8.2\#5 Solution: The solution should be $\frac38 x+\frac14\sin2x+\frac1{32}\sin4x+C$.
\item \S8.2 p402: Most of the definite integrals \emph{do not} ``appear in the previous set''.
\item \S8.3\#17 Solution: $\tan^{-1}\left(\frac{x+2}2\right)$ should be $\tan^{-1}\left(\frac{x+2}3\right)$.
\item \S8.4\#31 Solution: The first and third term should be negative.
\item \S8.5\#1: $\int \sin^{-1}x\ dx=x\sin^{-1}x+\sqrt{1-x^2}+C$, whereas\\
$\int x\sin^{-1}x\ dx=\frac{x^2}{2} \sin^{-1}x - \frac{1}{4} \sin^{-1}x + \frac{x}{4} \sqrt{1-x^2}+C$.
\item \S8.5\#5 Solution: The $\sqrt{x^2+25}$ should be in the denominator.
\item \S8.5\#11: $\int e^{2x}\sin^2 3x\ dx=\frac14e^{2x}-\frac1{40}e^{2x}(\cos6x+3\sin6x)$, whereas $\int e^{2x}\sin 3x\ dx=\frac{1}{13} e^{2x}(2 \sin 3x - 3\cos 3x) +C$.
\item \S8.5\#27 Solution: $C$ should be $+C$.
\item \S8.5\#33 Solution: The second term should be added, not subtracted.
\item \S8.5\#43: $\ds\int(x-\cot 3x)^2\ dx$ cannot be expressed in terms of elementary functions.  It should be $\ds\int (5- \cot 3x)^2\ dx$.
\item \S8.6\#3 Solution: Should have $\le10$ instead of $<10$.
\item \S8.6\#45: The solution should compare the integrand to $x/(x^2+1)$.
\item \S9.1\#7: All of the answers should be negated.
\item \S9.1\#13--16: $a_n$ is multiply defined.
\item \S9.2 p475: ``a telescoping series is one in which the partial sums reduce to just a finite number of terms'' is meaningless, since every partial sum has a finite number of terms.  This should be ``a telescoping series is one in which the partial sums reduce to a fixed number of terms.''
\item \S9.4\#25a Solution: $a_n/n<n$ should be $a_n/n<a_n$.
\item \S9.4\#33 Solution: Diverges; the Limit Comparison Test can be used with $\sum n^{-1/2}$.
\item \S9.6\#23 Solution: $e^2$ should be $e^{-2}$.
\item \S9.8\#25 Solution: $R=1$, interval is $[-\frac13,\frac53)$.
\item \S9.9\#7 Solution: The polynomial is the desired $p_5$, but it is labeled $p_8$.
\item \S9.9\#33: Should specify ``centered at $c=1$.''  In the solution, $(-1)^n$ should be $(-1)^{n+1}$.
\item \S9.10 p550 Example 4 Line -3: The Maclaurin series is missing the $x$'s.  It should be
\[
1+ kx + \frac{k(k-1)}{2!}x^2 + \frac{k(k-1)(k-2)}{3!}x^3 + \dotsb + \frac{k(k-1)\dotsm\big(k-(n-1)\big)}{n!}x^n+\dotsb
\]
\item \S9.10 p551 Example 4 Line 3: $\ds\lim_{n\to\infty} \abs{\frac{k-n}{n}x}$ should be $\ds\lim_{n\to\infty} \abs{\frac{k-n}{n+1}x}$.
\item \S9.10\#11 Solution: The Taylor Series should be $\frac12+\sum_{n=1}^\infty\dotsb$.
\item \S10.3\#11 solution: a: The second term in the numerator should be doubled; b: The normal line is $y=-x$.
\item \S10.3\#29 solution: The intervals should be: concave up: $(-\pi/2,0)$; concave down: $(0,\pi/2)$.
\item \S10.4\#41 solution: $x^2y^2x^2$ should be $x^2y^2$.
\item \S10.4\#53 solution: $P(-1/2,\pi/3)$ should be $P(-1/2,2\pi/3)$.
\item \S10.5\#19 solution: The area is $\frac{4\pi}{3}+2\sqrt 3$.
\label{2017-07-27IIplus}
\end{enumerate}

%\end{document}

%\clearpage

In addition to the previous, the following errors exist in the July 27, 2017 printed version of Apex LT Calculus III:
\begin{enumerate}
\item The solution to \S11.1\#19 should be $x^2+z^2=\frac1{(1+y^2)^2}$.
\item \S11.5 p696 \#31b: ``to the use the formula'' should be ``to use the formula''.
\item \S12.2 p727 Theorem 96: $\ds\int\vec r$ is not actually defined, so we should take this to be a definition, not a theorem.
\item \S12.2 p730 \#13: The function is not a vector (on the other hand, one could argue part of the problem is that realization).
\item \S13.4 p710 Definition 91: The $dz$ on the left should be $dw$.
\item \S13.6 p823 \#25--28 part (b): $\vec u$ is the unit vector in the direction of $\vec v$.
\item The solution to \S13.8\#17 should have $f(x)=\dots\pm2\sqrt{4-x^2}$ instead of $f(x)=\dots\pm\sqrt{4-x^2}$ (three locations).  Everything before and after these changes is already correct.
\item \S14.3 p878 Example 1 Solution: The region $R$ is the disk bounded by the given circle.
\item \S14.5 p899 \#17\&18: ``you result'' should be ``your result''.
\item \S14.6 Example 1 p904: ``in the 1\textsuperscript{st} octant'' should be ``where $x,y>0$''.
\label{2017-07-27III}
\end{enumerate}

%\clearpage

In addition to the previous, the following error exists in the Summer 2017 printed version of Apex LT Calculus I:
\begin{enumerate}
\item \S4.2 p204 \#4\&5: The questions should specify that the desired numbers are positive.
\label{2017-05-00I}
\end{enumerate}
\vspace{1in}

In addition to the previous, the following errors exist in the Summer 2017 printed version of Apex LT Calculus II:
\begin{enumerate}
\item \S9.6 p510 Example 1 Solution Part 3: $|sin n|/n$ should be $\{|sin n|/n^2\}$.
\item \S10.3 p592,3 Definitions 48\&49, subsequent paragraph: $f((t_0), g(t_0))$ should be $(f(t_0), g(t_0))$ (three times).
\label{2017-05-00II}
\end{enumerate}

%\clearpage

In addition to the previous, the following errors exist in the Summer 2017 printed version of Apex LT Calculus III:
\begin{enumerate}
\item The solution to \S11.1\#13 should also include where the coordinate is 0.
\item The solution to \S11.1\#19 should be $x^2+z^2=\frac1{(1+y^2)^2}$.\vspace{-.2\baselineskip}
\item The solution to \S11.2\#7 should be $4\hat\imath-4\hat\jmath$.
\item \S11.4 p688 Line 3: ``face opposite face'' should be ``opposite face''.
\item \S11.4 p690 Lines 4\&8: $\sin-15^\circ=\frac{5(1-\sqrt3)}{\sqrt2}$
\item The solution to \S11.4\#27 should be $3\sqrt{30}$.
\item The solution to \S11.4\#37 should specify ``any \emph{unit} vector''.
\item \S11.5 p699 Line 7: The vector $c$ is missing its arrow.
\item \S11.5 p701 \#29--31: Five vectors $\ell$ and one $c$ are missing their arrows.
\item \S11.6 p709 \#11--14\&17: Nine vectors $\ell$ are missing their arrows.
\item \S12.2 p723 Definition 71 Line 4: A vector $r$ is missing its arrow.
\item \S12.2 p728 Definition 74: Should also include that $\vec r'$ is continuous.
\item \S12.2 p730 Lines 2--6: All six $\vec u$ should be $\vec u'$.
\item \S12.3 p737 Example 2 Solution Paragraph 5\&6: The intervals are 1/5\textsuperscript{th} of a second apart; the ``large change in position'' is from $t=-1$ to $t=-0.8$.
\item \S12.3 p737 Example 2 Paragraph -1: ``the have the'' should be ``they have the''; ``they have they have'' should be ``they have''.
\item \S12.3 p741 Line 4: The $t$ in $\vec C = v_0\bracket{\cos t,\sin t}$ should be $\theta$.
\item \S12.3\#27: Both $t$'s in $\vec r_2(s)=\bracket{6t-6,4t-4}$ should be $s$.
\item \S12.5 Example 2 Line 4: Both $t$'s in $\vec r(s)=\bracket{3t/5-1, 4t/5+2}$ should be $s$.
\item \S12.5 Example 5 Line -6: ``explictly'' is misspelled.
\item \S13.2 p783 Example 4 Line 4: ``$2/0$'' should be ``$-2/0$''.
\item The solution to \S13.2\#17a should be ``Along $y=mx$, the limit is 0.''
\item \S13.3 p791 Example 2 Solution 3 Line 4: The first term should have a factor of $\sqrt{x^2+1}$:\\
$2xy^3e^{x^2y^3}\sqrt{x^2+1}$.
\item \S13.6\#13b: The solution should be $-2/\sqrt5$ ($\vec u =\bracket{-1/\sqrt5,-2/\sqrt5}$) (two divisions are missing).
\item \S13.7 p831 Example 3 Line -2: ``The surface $z=-x^2+y^2$'' should be ``The surface $z=-x^2-y^2+2$''.
\item \S13.7 p834 Example 6 Line -1: Ditto.
\item The solution to \S13.8\#11 should be that there are critical points when $x=0$ or $y=0$ (these are absolute minima).
\item The solution to \S13.8\#17 should be that the absolute maximum is at $(\sqrt2,\sqrt2,4+4\sqrt2)$.
\item \S14.2 p869 Example 2 Lines -2--(-3): The $-\frac{17}3$ should be $+\frac{17}3$ both times.
\item \S15.2 p947 Theorem 134, Proof, Line 3: Figure 4.2.2 should be Figure 15.7.
\item \S15.2 p948 Theorem 135 \& p950 Theorem 136: These require that $R$ be simply connected.
\item \S15.3 p965 Example 2 Solution, Line 3--4: Figure 4.4.5 should be Figure 15.16.
\item \S15.5 p975 Lines 5--6 of the Proof: Figure 4.5.4 should be Figure 15.20.
\item \S15.5 p981 Line 2: Figure 4.5.6 should be Figure 15.22.
\item \S15.6 p991 Key Idea 68 Divergence: $(\sin \phi \,f_{\theta})$ should be $(\sin \phi \,f_{\phi})$.
\item There are many places where a vector needs an arrow (because of the conversion from the previous text, many of these look like $\oint_C f\cdot\vec x\,dt$, where the first argument to the dot product is missing its arrow).
\item There's several places with a circle $x^2+y^2=r^2$ when what is really meant is the two dimensional disk bounded by that circle.  Known instances:
\begin{enumerate}
\item \S13.1 p770 Example 2 Line 6 (an ellipse instead of a circle).
\item \S14.3 p878 Example 1 Line 2.
\item \S14.5 p899 Example 2; p900 Example 3 Solution Line 3; \& p903 Exercises \# 8,13,17--19.
\item \S14.6 p911 Example 4 Solution Line 6; p912 Line 5; \& p923 Exercise \# 8.
\end{enumerate}
\label{2017-05-00IIIplus}
\end{enumerate}

%\clearpage

\newcommand{\springerrors}{%
In addition to the previous, the following errors exist in the Spring 2017 printed version of Apex LT:
\begin{enumerate}
\item Examples are numbered within each section.  But when we refer to an example in a different section, it doesn't indicate that you should go to that section.  For example, in Section 6.3, ``Example 7'' should be ``Example 6.2.7''.
\item The prerequisite sections (1.0, 2.0, and 10.0) have the running header from the previous section.
\end{enumerate}} % end \springerrors

\springerrors

In addition to the previous, the following errors exist in the Spring 2017 printed version of Apex LT Calculus I:
\begin{enumerate}
\item Theorem 5 (Squeeze Theorem) requires the functions be defined and the squeezing to hold at the limit, which is unnecessary.\vspace{-.5\baselineskip}
\item Example 1.3.7 should be $\ds \lim_{x\to 0} \frac{\sqrt{x+4}-2}{x}$.
\item Definition 2 (One Sided Limits) and Definition 8 (Left and Right Continuity) requires that the function be defined in an open interval around the point, whereas it only needs to be defined to the left or right.
\item Definition 9 (Continuity on Closed Intervals) has ``left'' and ``right'' reversed.
\item Example 3.1.6: ``are are $0$ and $\pm\pi$'' should be ``are $0$ and $\pm\sqrt\pi$''.
\item \S3.4 p175 ``maximizing $f'$ means finding the where'' shouldn't have ``the''.
\item Example 4.2.3 Line 1: ``an power station'' should be ``a power station''.
\item \S5.1 p222 Theorem 31 only holds on an interval.  Similarly, the paragraph after Definition 22 should begin ``Using Definitions 21 and 22, we can say that on an interval''.
\item \S6.1 p296 Example 5: In Figure 6.7, $x+3$ should be $x+1$.  In the example itself, the limits of integration are determined from the picture (using $x+1$), while the integrand uses $x+3$.
\label{2017-01-00Iplus}
\end{enumerate}

%\clearpage

\springerrors

In addition to the previous, the following errors exist in the Spring 2017 printed version of Apex LT Calculus II:
\begin{enumerate}
\item The solution to \S7.3\#7 should be $\dfrac{1-x\ln x\ln5}{x5^x\ln5}$\smallskip
\item The solution to \S7.3\#21 should be $\dfrac{\ln\frac{245}3}{\ln3}=\dfrac{\ln245}{\ln3}-1$.
\item \S8.1 p383 ``If we had chosen\ldots $du=\sin x dx$'' should be $du=-\sin x dx$.
\item The solution to \S8.5\#1 should be $x\sin^{-1}x+\sqrt{1-x^2}+C$.  The given solution is $\ds\int x\sin^{-1}x dx$.\vspace{-.3\baselineskip}
\item The solution to \S8.5\#5 should be $\dfrac x{25\sqrt{x^2+25}}+C$.
\item The solution to \S8.5\#19 should be $\dfrac23 (1+e^x)^{3/2}+C$ (the exponent is inverted)
\item \S9.1 p465 paragraph -2, line 2: $\ds \lim_{n\to \infty}=e$ should be $\ds \lim_{n\to \infty}b_n=e$.
\vspace{-.5\baselineskip}
\item \S9.2 Theorem 64.  Part 1 assumes $r\neq1$.  The statement of part 2 assumes $S_n=\ds\sum_{k=0}^n ar^k$.\vspace{-\baselineskip} The proof assumes that $S_n=\ds\sum_{k=0}^{n-1} ar^k$ has $n$ terms.\vspace{-.7\baselineskip}
\item \S9.2 p479 line -1, p480 line 2, and p481 line 2: $\ds\sum_{n\to\infty}$ should be $\ds\sum_{n=1}^\infty$.
\item \S9.2\#27--29: The index of summation should be $n$, not $i$.  The solutions for the latter two should be ``converges'', not a number.\vspace{-.3\baselineskip}
\item \S9.3\#15 should be $\ds\sum_{i=3}^\infty\frac1{n\ln n[\ln(\ln n)]^p}$.
\item \S9.7\#9: The answer should be ``converges conditionally''.
\item \S9.7\#13 should be $\ds \sum_{n=1}^\infty \frac{1-\cos n}{n^3}$.
\item \S9.7\#15: The answer should be ``converges absolutely''.
\item \S9.8\#33: The desired answer is not a power series.
\item \S9.10\#15: Key Idea 32 gives the MacLaurin series for $\ln(x+1)$, and it converges on $(-1,1]$.  We can't expect students to prove convergence at the right endpoint (requires Abel's Theorem) or on $(-1,0)$ (requires Cauchy's form of the remainder) at this time.  (They can prove the series converges, but not what it converges to.)
\item The Conic Sections portion is supposed to be labeled 10.0.  The initial title is unlabeled, and the exercises are labeled 10.-1.  (Additionally, the running header is incorrect, as previously noted.)
\item \S10.5 Key Idea 44: $[f_1(\theta)]^2 - [f_2(\theta)]^2$ should be $[f_2(\theta)]^2 - [f_1(\theta)]^2$.
\item The solution to \S10.5\#9(a) should use $\theta$, not $t$.
\label{2017-01-00IIplus}
\end{enumerate}

%\clearpage

In addition to most of the previous, the following errors exist in the Fall 2016 printed version of Apex LT:
\begin{enumerate}
\item Sometimes, ``In Exercises \#--\#'' in exercise directions will go too far, and refer to exercises past the end of the intended range. This was an artifact of the way exercises were loaded.
\item p. 7, \S1.0 Exercises 11--14: The directions should be ``Graph the given f.''
\item p. 35, \S1.3 Example 1.3.7: The example should be: Evaluate $\ds\lim_{x\to2}\frac{\sqrt{x+4}-2}x$. (But see the previous erratum in Spring 2017.)
\item p. 39, \S1.3\#33: The answer should be 1/6.
\item p. 47, \S1.4\#13: The answer to parts b and d should be 0.
\item p. 47, \S1.4\#23: $\ds\lim_{-2^+}f(x)=0$ should be $\ds\lim_{x\to-2^+}f(x)=0$.
\item p. 60, \S1.5\#31: This problem ``Reviews'' the next section, and should be there instead.
\item p. 62, \S1.6 Example 1.6.2: The floor function returns the largest integer smaller than or equal to the input.
\item p. 72, \S1.6\#41: The given answer of $a=1$, $b=-1$ is not continuous at $x=-1$. The correct answer is $a=\frac34$, $b=-\frac14$.
\item p. 76, \S2.0 Example 2.0.3: Part 2 should begin $(x^{-3})^4$. Part 3 should begin $(x^{-1/2})^{2/3}$.
\item p. 77, \S2.0 Example 2.0.5: The problem in part 1 should be $x^{7/3}-4x^{2/3}$.
\item p. 102, \S2.2 Exercise 17: From the graph, you cannot tell which function is the derivative of the other. The back of the book says this as well, but it would be better if you could tell.
\item p. 141, \S2.6 Example 2.6.4: On page 141, line 4, $2y^2$ should be $2x\cdot y^2$; line 5, $y^2$ should be $xy^2$.
\item p. 207, \S4.2\#19: The solution dimensions give a cost of 60.
\item p. 212, \S4.3\#38: ``Exercises 36'' should be singular.
\item p. 231, 232, \S5.1: Exercises 20, 21, 36, and 41 require $\int a^x dx$, although that hasn't yet been discussed. It should be removed.
\item p. 277 \S5.4\#11: Ditto.
\item pp. 304--306: The text states that moving a curve through space creates a solid, when it fact it creates a surface. To create a solid, we need to move a region through space, or look at the region enclosed by a surface. This leads to the following changes:
\begin{enumerate}
\item p. 304, paragraph 2, line 2: ``a horizontal axis creates a three-dimensional solid'' should be ``a horizontal axis encloses a three-dimensional solid''.
\item p. 304, Key Idea 12: ``Let a solid be formed by revolving the curve'' should be ``Let a solid be enclosed by revolving the curve''.
\item p. 305, Example 6.2.2: The region being rotated is the one bounded by the curve $y=1/x$, $x=1$, $x=2$, and the $x$-axis.
\item p. 306, Example 6.2.3: The region being rotated is the one bounded by the curve $y=1/x$, $y=1$, $y=0.5$, and the $y$-axis.
\item p. 306: The paragraph following Example 6.2.3 states:
\begin{quote}
The previous two examples demonstrate how taking the same region and rotating it about two different axes will result in different solids and thus volumes.
\end{quote}
The examples do not have the same region. This sentence should be deleted.
\end{enumerate}
\item p. 309, Figure 6.16: The label (b) should indicate the middle figure.
\label{2016-08-00Iplus}
\end{enumerate}

\end{document}
