\printconcepts

\exercise{What is wrong with the following ``definition'' of a limit?
	\begin{quote}
``The limit of $f(x)$, as $x$ approaches $a$, is $K$'' means that given any $\delta>0$ there exists $\epsilon>0$ such that whenever $\abs{f(x)-K}< \epsilon$, we have $\abs{x-a}<\delta$.
	\end{quote}
}{$\epsilon$ should be given first, and the restriction 	$\abs{x-a}<\delta$ implies $\abs{f(x)-K}< \epsilon$, not the other way around.}

\exercise{Which is given first in establishing a limit, the $x$-tolerance or the $y$-tolerance?}{The $y$-tolerance.}

\exercise{T/F: $\epsilon$ must always be positive.}{T}

\exercise{T/F: $\delta$ must always be positive.}{T}

\printproblems

\exercise{Use the graph below of $f$ to find a number $\delta$ such that if $0<\abs{x-2}<\delta$, then $\abs{f(x)-1}<0.5$.\\
\begin{tikzpicture}[>=stealth]
\begin{axis}[width=1.16\marginparwidth,tick label style={font=\scriptsize},
minor x tick num=1,axis y line=middle,axis x line=middle,name=myplot,
ymin=-.1,ymax=2.5,xmin=-.1,xmax=3,xtick={1,1.414,2,2.449},ytick={.5,1,1.5,2}]
\addplot [draw={\colorone},smooth,thick] {.25*x*x};
\draw[thin,draw=\colortwo] (axis cs:0,1.5) -- (axis cs:2.449,1.5);
\draw[thin,draw=\colortwo] (axis cs:0,.5) -- (axis cs:1.414,.5);
\draw[thin,draw={\colortwo}] (axis cs:2.449,0) -- (axis cs:2.449,1.5);
\draw[thin,draw={\colortwo}] (axis cs:1.414,0) -- (axis cs:1.414,.5);
\draw[thin,dashed] (axis cs:0,1) -- (axis cs: 2,1);
\draw[thin,dashed] (axis cs:2,0) -- (axis cs: 2,1);
\fill[draw={\colortwo}] (axis cs:2.449,1.5) circle (1pt);
\fill[draw={\colortwo}] (axis cs:1.414,.5) circle (1pt);
\fill[black](axis cs:(2,1) circle (1pt);
\end{axis}
\node [right] at (myplot.right of origin) {\scriptsize $x$};
\node [above] at (myplot.above origin) {\scriptsize $y$};
\end{tikzpicture}}{$\delta\le0.45$}

\exercise{Use the graph below of $f$ to find a number $\delta$ such that if $0<\abs{x-2}<\delta$, then $\abs{f(x)-1}<0.3$.\\
\begin{tikzpicture}[>=stealth]
\begin{axis}[width=1.16\marginparwidth,tick label style={font=\scriptsize},
minor x tick num=1,axis y line=middle,axis x line=middle,name=myplot,
ymin=-.1,ymax=1.7,xmin=-.1,xmax=3,xtick={1,1.29,2,2.95},ytick={.7,1,1.3}]
\addplot [draw={\colorone},smooth,thick] {ln(x+.718)};
\draw[thin,draw=\colortwo] (axis cs:0,1.3) -- (axis cs:2.95,1.3);
\draw[thin,draw=\colortwo] (axis cs:0,.7) -- (axis cs:1.29,.7);
\draw[thin,draw={\colortwo}] (axis cs:2.95,0) -- (axis cs:2.95,1.3);
\draw[thin,draw={\colortwo}] (axis cs:1.26,0) -- (axis cs:1.29,.7);
\draw[thin,dashed] (axis cs:0,1) -- (axis cs: 2,1);
\draw[thin,dashed] (axis cs:2,0) -- (axis cs: 2,1);
\fill[draw={\colortwo}] (axis cs:2.95,1.3) circle (1pt);
\fill[draw={\colortwo}] (axis cs:1.29,.7) circle (1pt);
\fill[black](axis cs:(2,1) circle (1pt);
%\addplot[mark=*,fill=white, only marks] coordinates {(2,1)};
%\fill[white](axis cs:(2,1) circle (3pt);
\end{axis}
\node [right] at (myplot.right of origin) {\scriptsize $x$};
\node [above] at (myplot.above origin) {\scriptsize $y$};
\end{tikzpicture}}{$\delta\le0.71$}

\input{exercises/01-02-exset-02}
