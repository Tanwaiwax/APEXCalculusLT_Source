\printconcepts

\exercise{In your own words, describe what it means for a function to be continuous.}{Answers will vary.}

\exercise{In your own words, describe what the Intermediate Value Theorem states.}{Answers will vary.}

\exercise{What is a ``root'' of a function?}{A root of a function $f$ is a value $c$ such that $f(c)=0$.}

\exercise{Given functions $f$ and $g$ on an interval $I$, how can the Bisection Method be used to find a value $c$ where $f(c) = g(c)$?}{Consider the function $h(x) = g(x) - f(x)$, and use the Bisection Method to find a root of $h$.}

\exercise{T/F:	If $f$ is defined on an open interval containing $c$, and $\ds \lim_{x\to c}f(x)$ exists, then $f$ is continuous at $c$.}{F}

\exercise{T/F: If $f$ is continuous at $c$, then $\ds \lim_{x\to c}f(x)$ exists.}{T}

\exercise{T/F: If $f$ is continuous at $c$, then $\ds \lim_{x\to c^+}f(x) = f(c)$.}{T}

\exercise{T/F: If $f$ is continuous on $[a,b]$, then $\ds\lim_{x\to a^-}f(x) = f(a)$.}{F}

\exercise{T/F: If $f$ is continuous on $[0,1)$ and $[1,2)$, then $f$ is continuous on $[0,2)$.}{F}

\exercise{T/F: The sum of continuous functions is also continuous.}{T}

\printproblems

\exercisesetinstructions{, a graph of a function $f$ is given along with a value $a$. Determine if $f$ is continuous at $a$; if it is not, state why it is not.}

% most of these are copies from the one sided limits section

\exercise{\noindent $a = 1$\\
\begin{tikzpicture}[alt={A line segment from (0,1) to a hollow dot at (1,2) then curving to (2,0).  There is also a solid dot at (1,1).}]
\begin{axis}[width=\marginparwidth,tick label style={font=\scriptsize},
axis y line=middle,axis x line=middle,name=myplot,
ymin=-.1,ymax=2.1,xmin=-.1,xmax=2.1]
\addplot [thick,draw={\colorone}] coordinates {(0.,1) (1,2)};
\addplot [draw={\colorone},smooth,thick,domain=1:2] {2*(x-2)^2};
\fill[black,draw=black] (axis cs:0,1) circle (1.5pt);
\fill[white,draw=black] (axis cs:1,2) circle (1.5pt);
\fill[black,draw=black] (axis cs:1,1) circle (1.5pt);
\fill[black,draw=black] (axis cs:2,0) circle (1.5pt);
\end{axis}
\node [right] at (myplot.right of origin) {\scriptsize $x$};
\node [above] at (myplot.above origin) {\scriptsize $y$};
\end{tikzpicture}}{No; $\ds \lim_{x\to 1} f(x) = 2$, while $f(1) = 1$.}

\exercise{\noindent $a = 1$\\
\begin{tikzpicture}[alt={A line segment from the origin to a hollow dot at (1,1).  A separate line segment goes from (1,2) to (2,0).}]
\begin{axis}[width=\marginparwidth,tick label style={font=\scriptsize},
axis y line=middle,axis x line=middle,name=myplot,
ymin=-.1,ymax=2.1,xmin=-.1,xmax=2.1]
\addplot [thick,draw={\colorone}] coordinates {(0.,0) (1,1)};
\addplot [draw={\colorone},thick] coordinates {(1,2) (2,0)};
\fill[black,draw=black] (axis cs:0,0) circle (1.5pt);
\fill[white,draw=black] (axis cs:1,1) circle (1.5pt);
\fill[black,draw=black] (axis cs:1,2) circle (1.5pt);
\fill[black,draw=black] (axis cs:2,0) circle (1.5pt);
\end{axis}
\node [right] at (myplot.right of origin) {\scriptsize $x$};
\node [above] at (myplot.above origin) {\scriptsize $y$};
\end{tikzpicture}}{No; $\ds \lim_{x\to 1} f(x)$ does not exist.}

\exercise{\noindent $a = 1$\\
\begin{tikzpicture}[alt={A curve from the origin that keeps getting larger as x approaches 1 from the left.  In a mirror image, the curve is large as x approaches 1 from the right and goes down to (2,0).}]
\begin{axis}[width=\marginparwidth,tick label style={font=\scriptsize},
axis y line=middle,axis x line=middle,name=myplot,
ymin=-.1,ymax=2.1,xmin=-.1,xmax=2.1]
\draw [thick,draw={\colorone}] (axis cs:0,0) parabola (axis cs:1,3);
\draw [thick,draw={\colorone}] (axis cs:2,0) parabola (axis cs:1,3);
\draw [draw={\colorone},dashed] (axis cs: 1,2.1) -- (axis cs:1,-.1);
\fill[black,draw=black] (axis cs:0,0) circle (1.5pt);
\fill[black,draw=black] (axis cs:2,0) circle (1.5pt);
\end{axis}
\node [right] at (myplot.right of origin) {\scriptsize $x$};
\node [above] at (myplot.above origin) {\scriptsize $y$};
\end{tikzpicture}}{No; $f(1)$ does not exist.}

\exercise{\noindent $a = 1$\\
\begin{tikzpicture}[alt={A line segment from (0,1) to a hollow dot at (1,2), then a solid dot at (1,1), and then a curve from (1,0) to (2,2).}]
\begin{axis}[width=\marginparwidth,tick label style={font=\scriptsize},
axis y line=middle,axis x line=middle,name=myplot,
ymin=-.1,ymax=2.1,xmin=-.1,xmax=2.1]
\draw [thick,draw={\colorone}] (axis cs:1,0) parabola (axis cs:2,2);
\draw [thick,draw={\colorone}] (axis cs:0,1) -- (axis cs:1,2);
\fill[black,draw=black] (axis cs:0,1) circle (1.5pt);
\fill[white,draw=black] (axis cs:1,2) circle (1.5pt);
\fill[black,draw=black] (axis cs:1,1) circle (1.5pt);
\fill[black,draw=black] (axis cs:2,2) circle (1.5pt);
\fill[white,draw=black] (axis cs:1,0) circle (1.5pt);
\end{axis}
\node [right] at (myplot.right of origin) {\scriptsize $x$};
\node [above] at (myplot.above origin) {\scriptsize $y$};
\end{tikzpicture}}{No}

\exercise{\noindent $a = 1$\\
\begin{tikzpicture}[alt={A curve from the origin to (1,2) and then a line segment from (1,2) to (2,0).}]
\begin{axis}[width=\marginparwidth,tick label style={font=\scriptsize},
axis y line=middle,axis x line=middle,name=myplot,
ymin=-.1,ymax=2.1,xmin=-.1,xmax=2.1]
\draw [thick,draw={\colorone}] (axis cs:1,2) parabola (axis cs:0,0);
\draw [thick,draw={\colorone}] (axis cs:1,2) -- (axis cs:2,0);
\fill[black,draw=black] (axis cs:0,0) circle (1.5pt);
\fill[black,draw=black] (axis cs:1,2) circle (1.5pt);
\fill[black,draw=black] (axis cs:2,0) circle (1.5pt);
\end{axis}
\node [right] at (myplot.right of origin) {\scriptsize $x$};
\node [above] at (myplot.above origin) {\scriptsize $y$};
\end{tikzpicture}}{Yes}

\exercise{\noindent $a = 2$\\
\begin{tikzpicture}[alt={A curve from (-4,-4) up to a hollow dot at (0,4), then a solid dot at the origin, and then a separate curve from (0,-4) up to (4,4).}]
\begin{axis}[width=\marginparwidth,tick label style={font=\scriptsize},
axis y line=middle,axis x line=middle,name=myplot,xtick={-4,...,-1,1,2,...,4},
ymin=-4.5,ymax=4.5,xmin=-4.5,xmax=4.5]
\addplot [thick,draw={\colorone},domain=-4:0] {4*cos(45*x)};
\addplot [thick,draw={\colorone},domain=0:4]  {-4*cos(45*x)};
\fill[black,draw=black] (axis cs:0,0) circle (1.5pt);
\fill[white,draw=black] (axis cs:0,4) circle (1.5pt);
\fill[black,draw=black] (axis cs:-4,-4) circle (1.5pt);
\fill[black,draw=black] (axis cs:4,4) circle (1.5pt);
\fill[white,draw=black] (axis cs:0,-4) circle (1.5pt);
\end{axis}
\node [right] at (myplot.right of origin) {\scriptsize $x$};
\node [above] at (myplot.above origin) {\scriptsize $y$};
\end{tikzpicture}}{Yes}

\exercise{\mbox{}\\[-2\baselineskip]\parbox[t]{\linewidth}{\begin{enumext}
\item		$a = -2$
\item		$a=0$
\item		$a=2$
\end{enumext}}
\begin{tikzpicture}[alt={A line segment from (-4,0) to a hollow dot at (-2,2), then to a solid dot at the origin, then to a hollow dot at (2,2), then to a solid dot at (4,0).  There is also a solid dot at (-2,0).}]
\begin{axis}[width=\marginparwidth,tick label style={font=\scriptsize},
axis y line=middle,axis x line=middle,name=myplot,xtick={-4,...,-1,1,2,...,4},
ymin=-4.5,ymax=4.5,xmin=-4.5,xmax=4.5]
\addplot [thick,draw={\colorone}] coordinates {(-4,0) (-2,2) (0,0) (2,2) (4,0)}; 
\fill[black,draw=black] (axis cs:0,0) circle (1.5pt);
\fill[black,draw=black] (axis cs:-4,0) circle (1.5pt);
\fill[black,draw=black] (axis cs:-2,0) circle (1.5pt);
\fill[white,draw=black] (axis cs:-2,2) circle (1.5pt);
\fill[white,draw=black] (axis cs:2,2) circle (1.5pt);
\fill[black,draw=black] (axis cs:4,0) circle (1.5pt);
\end{axis}
\node [right] at (myplot.right of origin) {\scriptsize $x$};
\node [above] at (myplot.above origin) {\scriptsize $y$};
\end{tikzpicture}}{\mbox{}\\[-2\baselineskip]\parbox[t]{\linewidth}{\begin{enumext}
\item		No; $\ds \lim_{x\to -2}f(x) \neq f(-2)$
\item		Yes
\item		No; $f(2)$ is not defined.
\end{enumext}}}

\exercise{\noindent $a = 3\pi/2$\\
\begin{tikzpicture}[alt={A wave that starts at (0,1), goes up to (π/2,2), down to (3π/2,0), and up to (2π,1).}]
\begin{axis}[width=\marginparwidth,tick label style={font=\scriptsize},
axis y line=middle,axis x line=middle,name=myplot,xtick=\empty,% 
extra x ticks={1.57,3.14,4.71,6.28},
extra x tick labels={$\pi/2$,$\pi$,$3\pi/2$,$2\pi$},%
ymin=-.1,ymax=2.1,xmin=-.5,xmax=6.5]
\addplot [draw={\colorone},smooth,thick,domain=0:2*pi] {sin(\x r)+1};
\end{axis}
\node [right] at (myplot.right of origin) {\scriptsize $x$};
\node [above] at (myplot.above origin) {\scriptsize $y$};
\end{tikzpicture}}{Yes; $\ds \lim_{x\to 3\pi/2} \sin x +1 = 0$, and $\sin(3\pi/2)+1 = 0$.}

\exercisesetend


%\questioncolumnbreak

\exercisesetinstructions{, determine if $f$ is continuous at the indicated values. If not, explain why.}

\exercise{$\ds f(x) = \begin{cases}
1 & x=0\\
\frac{\sin x}{x} & x\ne0
\end{cases}$
\begin{enumext}
\item		$x=0$
\item		$x=\pi$
\end{enumext}}{\mbox{}\\[-2\baselineskip]\parbox[t]{\linewidth}{\begin{enumext}
\item		Yes
\item		Yes
\end{enumext}}}

\exercise{$\ds f(x) = \begin{cases}
x^3-x & x<1\\
x-2 & x\geq 1
\end{cases}$
\begin{enumext}
\item		$x=0$
\item		$x=1$
\end{enumext}}{\mbox{}\\[-2\baselineskip]\parbox[t]{\linewidth}{\begin{enumext}
\item		Yes
\item		No; the left and right hand limits at 1 are not equal.
\end{enumext}}}

\exercise{$\ds f(x) = \begin{cases}
\frac{x^2+5x+4}{x^2+3x+2} &  x\neq -1\\
3 & x=-1
\end{cases}$
\begin{enumext}
\item		$x=-1$
\item		$x=10$
\end{enumext}}{\mbox{}\\[-2\baselineskip]\parbox[t]{\linewidth}{\begin{enumext}
\item		Yes
\item		Yes
\end{enumext}}}

\exercise{$\ds f(x) = \begin{cases}
\frac{x^2-64}{x^2-11 x+24} &  x\neq 8\\
5 & x=8
\end{cases}$
\begin{enumext}
\item		$x=0$
\item		$x=8$
\end{enumext}}{\mbox{}\\[-2\baselineskip]\parbox[t]{\linewidth}{\begin{enumext}
\item		Yes
\item		No. $\lim_{x\to 8} f(x) = 16/5 \neq f(8) = 5$.
\end{enumext}}}

\exercisesetend


\input{exercises/01-05-exset-03}

\exercise{Let $\ds f(x) = \begin{cases}
x^2-1&x < 3 \\
x+5&x\geq 3
\end{cases}$.\\
Is $f$ continuous everywhere?}{Yes. The only ``questionable'' place is at $x=3$, but the left and right limits agree.}

\exercise{Let $\ds f(x) = \begin{cases}
x\sin(\frac1x)&x \ne 0 \\
0&x=0
\end{cases}$.\\
Is $f$ continuous everywhere?}{Yes. The only ``questionable'' place is at $x=0$, but the Squeeze Theorem shows that the limits agree.}

\exercisesetinstructions[Exercises]{ test your understanding of the Intermediate Value Theorem.}

\exercise{Let $f$ be continuous on $[1,5]$ where $f(1) = -2$ and $f(5) = -10$. Does a value $1<c<5$ exist such that $f(c) = -9$? Why/why not?}{Yes, by the Intermediate Value Theorem.}

\exercise{Let $g$ be continuous on $[-3,7]$ where $g(0) = 0$ and $g(2) = 25$. Does a value $-3<c<7$ exist such that $g(c) = 15$? Why/why not?}{Yes, by the Intermediate Value Theorem. In fact, we can be more specific and state such a value $c$ exists in $(0,2)$, not just in $(-3,7)$.}

\exercise{Let $f$ be continuous on $[-1,1]$ where $f(-1) = -10$ and $f(1) = 10$. Does a value $-1<c<1$ exist such that $f(c) = 11$? Why/why not?}{We cannot say; the Intermediate Value Theorem only applies to function values between $-10$ and 10; as 11 is outside this range, we do not know.}

\exercise{Let $h$ be a function on $[-1,1]$ where $h(-1) = -10$ and $h(1) = 10$. Does a value $-1<c<1$ exist such that $h(c) = 0$? Why/why not?}{We cannot say; the Intermediate Value Theorem only applies to continuous functions. As we do know know if $h$ is continuous, we cannot say.}

\exercisesetend

\input{exercises/01-05-exset-05}

\exercisesetinstructions{, sketch the graph of a function that has the following properties.}

\exercise{$f$ is discontinuous at 3, but continuous from the left at 3, and continuous elsewhere.}{Answers will vary.}

\exercise{$f$ is discontinuous at -1 and 2, but continuous from the right at -1 and continuous from the left at 2, and continuous elsewhere.}{Answers will vary.}

\exercise{$f$ has a jump discontinuity at -2 and an infinite discontinuity at 4 and is continuous elsewhere.}{Answers will vary.}

\exercise{$f$ has a removable discontinuity at 2, is continuous only from the left at 5, and is continuous elsewhere.}{Answers will vary.}

\exercisesetend


\input{exercises/01-05-exset-04}

\printreview

\exercise{Let $\ds f(x)= \begin{cases}
x^2-5 & x<5 \\
5x & x \geq 5
\end{cases}$. Find
\begin{multicols}{2}
\begin{enumerate}
\item		$\ds \lim_{x\to 5^-} f(x)$
\item		$\ds \lim_{x\to 5^+} f(x)$
\item		$\ds \lim_{x\to 5} f(x)$
\item		$f(5)$\end{enumerate}
\end{multicols}}{\mbox{}\\[-2\baselineskip]\parbox[t]{\linewidth}{\begin{enumerate}
\item		20
\item		25
\item		Limit does not exist
\item		25
\end{enumerate}}}

\exercise{Numerically approximate the following limits:
\begin{enumerate}
\item	$\ds \lim_{x\to -4/5^+} \frac{x^2-8.2 x-7.2}{x^2+5.8 x+4}$
\item	$\ds \lim_{x\to -4/5^-} \frac{x^2-8.2 x-7.2}{x^2+5.8 x+4}$
\end{enumerate}}{\tagpdfsetup{table/header-rows={1}}\begin{tabular}{cc}
$x$ & $f(x)$ \\ \hline 
$-0.81\phantom{0} $& $-2.34129$ \\
$ -0.801$ & $-2.33413$ \\
$ -0.79\phantom{0} $& $-2.32542 $\\
$ -0.799$ & $-2.33254$
\end{tabular}

The top two lines give an approximation of the limit from the left: $-2.33$. The bottom two lines give an approximation from the right: $-2.33$ as well.}

\exercise{Give an example of function $f(x)$ for which $\ds \lim_{x\to 0} f(x)$ does not exist.}{Answers will vary.}
