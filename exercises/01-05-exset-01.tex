\exercisesetinstructions{, a graph of a function $f$ is given along with a value $a$. Determine if $f$ is continuous at $a$; if it is not, state why it is not.}

% most of these are copies from the one sided limits section

\exercise{\noindent $a = 1$\\
\begin{tikzpicture}[alt={A line segment from (0,1) to a hollow dot at (1,2) then curving to (2,0).  There is also a solid dot at (1,1).}]
\begin{axis}[width=\marginparwidth,tick label style={font=\scriptsize},
axis y line=middle,axis x line=middle,name=myplot,
ymin=-.1,ymax=2.1,xmin=-.1,xmax=2.1]
\addplot [thick,draw={\colorone}] coordinates {(0.,1) (1,2)};
\addplot [draw={\colorone},smooth,thick,domain=1:2] {2*(x-2)^2};
\fill[black,draw=black] (axis cs:0,1) circle (1.5pt);
\fill[white,draw=black] (axis cs:1,2) circle (1.5pt);
\fill[black,draw=black] (axis cs:1,1) circle (1.5pt);
\fill[black,draw=black] (axis cs:2,0) circle (1.5pt);
\end{axis}
\node [right] at (myplot.right of origin) {\scriptsize $x$};
\node [above] at (myplot.above origin) {\scriptsize $y$};
\end{tikzpicture}}{No; $\ds \lim_{x\to 1} f(x) = 2$, while $f(1) = 1$.}

\exercise{\noindent $a = 1$\\
\begin{tikzpicture}[alt={A line segment from the origin to a hollow dot at (1,1).  A separate line segment goes from (1,2) to (2,0).}]
\begin{axis}[width=\marginparwidth,tick label style={font=\scriptsize},
axis y line=middle,axis x line=middle,name=myplot,
ymin=-.1,ymax=2.1,xmin=-.1,xmax=2.1]
\addplot [thick,draw={\colorone}] coordinates {(0.,0) (1,1)};
\addplot [draw={\colorone},thick] coordinates {(1,2) (2,0)};
\fill[black,draw=black] (axis cs:0,0) circle (1.5pt);
\fill[white,draw=black] (axis cs:1,1) circle (1.5pt);
\fill[black,draw=black] (axis cs:1,2) circle (1.5pt);
\fill[black,draw=black] (axis cs:2,0) circle (1.5pt);
\end{axis}
\node [right] at (myplot.right of origin) {\scriptsize $x$};
\node [above] at (myplot.above origin) {\scriptsize $y$};
\end{tikzpicture}}{No; $\ds \lim_{x\to 1} f(x)$ does not exist.}

\exercise{\noindent $a = 1$\\
\begin{tikzpicture}[alt={A curve from the origin that keeps getting larger as x approaches 1 from the left.  In a mirror image, the curve is large as x approaches 1 from the right and goes down to (2,0).}]
\begin{axis}[width=\marginparwidth,tick label style={font=\scriptsize},
axis y line=middle,axis x line=middle,name=myplot,
ymin=-.1,ymax=2.1,xmin=-.1,xmax=2.1]
\draw [thick,draw={\colorone}] (axis cs:0,0) parabola (axis cs:1,3);
\draw [thick,draw={\colorone}] (axis cs:2,0) parabola (axis cs:1,3);
\draw [draw={\colorone},dashed] (axis cs: 1,2.1) -- (axis cs:1,-.1);
\fill[black,draw=black] (axis cs:0,0) circle (1.5pt);
\fill[black,draw=black] (axis cs:2,0) circle (1.5pt);
\end{axis}
\node [right] at (myplot.right of origin) {\scriptsize $x$};
\node [above] at (myplot.above origin) {\scriptsize $y$};
\end{tikzpicture}}{No; $f(1)$ does not exist.}

\exercise{\noindent $a = 1$\\
\begin{tikzpicture}[alt={A line segment from (0,1) to a hollow dot at (1,2), then a solid dot at (1,1), and then a curve from (1,0) to (2,2).}]
\begin{axis}[width=\marginparwidth,tick label style={font=\scriptsize},
axis y line=middle,axis x line=middle,name=myplot,
ymin=-.1,ymax=2.1,xmin=-.1,xmax=2.1]
\draw [thick,draw={\colorone}] (axis cs:1,0) parabola (axis cs:2,2);
\draw [thick,draw={\colorone}] (axis cs:0,1) -- (axis cs:1,2);
\fill[black,draw=black] (axis cs:0,1) circle (1.5pt);
\fill[white,draw=black] (axis cs:1,2) circle (1.5pt);
\fill[black,draw=black] (axis cs:1,1) circle (1.5pt);
\fill[black,draw=black] (axis cs:2,2) circle (1.5pt);
\fill[white,draw=black] (axis cs:1,0) circle (1.5pt);
\end{axis}
\node [right] at (myplot.right of origin) {\scriptsize $x$};
\node [above] at (myplot.above origin) {\scriptsize $y$};
\end{tikzpicture}}{No}

\exercise{\noindent $a = 1$\\
\begin{tikzpicture}[alt={A curve from the origin to (1,2) and then a line segment from (1,2) to (2,0).}]
\begin{axis}[width=\marginparwidth,tick label style={font=\scriptsize},
axis y line=middle,axis x line=middle,name=myplot,
ymin=-.1,ymax=2.1,xmin=-.1,xmax=2.1]
\draw [thick,draw={\colorone}] (axis cs:1,2) parabola (axis cs:0,0);
\draw [thick,draw={\colorone}] (axis cs:1,2) -- (axis cs:2,0);
\fill[black,draw=black] (axis cs:0,0) circle (1.5pt);
\fill[black,draw=black] (axis cs:1,2) circle (1.5pt);
\fill[black,draw=black] (axis cs:2,0) circle (1.5pt);
\end{axis}
\node [right] at (myplot.right of origin) {\scriptsize $x$};
\node [above] at (myplot.above origin) {\scriptsize $y$};
\end{tikzpicture}}{Yes}

\exercise{\noindent $a = 2$\\
\begin{tikzpicture}[alt={A curve from (-4,-4) up to a hollow dot at (0,4), then a solid dot at the origin, and then a separate curve from (0,-4) up to (4,4).}]
\begin{axis}[width=\marginparwidth,tick label style={font=\scriptsize},
axis y line=middle,axis x line=middle,name=myplot,xtick={-4,...,-1,1,2,...,4},
ymin=-4.5,ymax=4.5,xmin=-4.5,xmax=4.5]
\addplot [thick,draw={\colorone},domain=-4:0] {4*cos(45*x)};
\addplot [thick,draw={\colorone},domain=0:4]  {-4*cos(45*x)};
\fill[black,draw=black] (axis cs:0,0) circle (1.5pt);
\fill[white,draw=black] (axis cs:0,4) circle (1.5pt);
\fill[black,draw=black] (axis cs:-4,-4) circle (1.5pt);
\fill[black,draw=black] (axis cs:4,4) circle (1.5pt);
\fill[white,draw=black] (axis cs:0,-4) circle (1.5pt);
\end{axis}
\node [right] at (myplot.right of origin) {\scriptsize $x$};
\node [above] at (myplot.above origin) {\scriptsize $y$};
\end{tikzpicture}}{Yes}

\exercise{\mbox{}\\[-2\baselineskip]\parbox[t]{\linewidth}{\begin{enumext}
\item		$a = -2$
\item		$a=0$
\item		$a=2$
\end{enumext}}
\begin{tikzpicture}[alt={A line segment from (-4,0) to a hollow dot at (-2,2), then to a solid dot at the origin, then to a hollow dot at (2,2), then to a solid dot at (4,0).  There is also a solid dot at (-2,0).}]
\begin{axis}[width=\marginparwidth,tick label style={font=\scriptsize},
axis y line=middle,axis x line=middle,name=myplot,xtick={-4,...,-1,1,2,...,4},
ymin=-4.5,ymax=4.5,xmin=-4.5,xmax=4.5]
\addplot [thick,draw={\colorone}] coordinates {(-4,0) (-2,2) (0,0) (2,2) (4,0)}; 
\fill[black,draw=black] (axis cs:0,0) circle (1.5pt);
\fill[black,draw=black] (axis cs:-4,0) circle (1.5pt);
\fill[black,draw=black] (axis cs:-2,0) circle (1.5pt);
\fill[white,draw=black] (axis cs:-2,2) circle (1.5pt);
\fill[white,draw=black] (axis cs:2,2) circle (1.5pt);
\fill[black,draw=black] (axis cs:4,0) circle (1.5pt);
\end{axis}
\node [right] at (myplot.right of origin) {\scriptsize $x$};
\node [above] at (myplot.above origin) {\scriptsize $y$};
\end{tikzpicture}}{\mbox{}\\[-2\baselineskip]\parbox[t]{\linewidth}{\begin{enumext}
\item		No; $\ds \lim_{x\to -2}f(x) \neq f(-2)$
\item		Yes
\item		No; $f(2)$ is not defined.
\end{enumext}}}

\exercise{\noindent $a = 3\pi/2$\\
\begin{tikzpicture}[alt={A wave that starts at (0,1), goes up to (π/2,2), down to (3π/2,0), and up to (2π,1).}]
\begin{axis}[width=\marginparwidth,tick label style={font=\scriptsize},
axis y line=middle,axis x line=middle,name=myplot,xtick=\empty,% 
extra x ticks={1.57,3.14,4.71,6.28},
extra x tick labels={$\pi/2$,$\pi$,$3\pi/2$,$2\pi$},%
ymin=-.1,ymax=2.1,xmin=-.5,xmax=6.5]
\addplot [draw={\colorone},smooth,thick,domain=0:2*pi] {sin(\x r)+1};
\end{axis}
\node [right] at (myplot.right of origin) {\scriptsize $x$};
\node [above] at (myplot.above origin) {\scriptsize $y$};
\end{tikzpicture}}{Yes; $\ds \lim_{x\to 3\pi/2} \sin x +1 = 0$, and $\sin(3\pi/2)+1 = 0$.}

\exercisesetend
