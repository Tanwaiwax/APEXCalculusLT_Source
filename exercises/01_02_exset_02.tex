\exerciseset{In Exercises}{, prove the given limit using an $\epsilon - \delta$ proof.}{

\exercise{$\ds\lim_{x\to4}(2x+5)=13$}{Given $\epsilon>0$, choose $\delta=\frac\epsilon2$.
\begin{gather*}
\abs{x-4}<\delta=\frac\epsilon2\\
\abs{2x-8}<\epsilon\\
\abs{(2x+5)-(13)}<\epsilon.
\end{gather*}
Thus $\ds\lim_{x\to4}(2x+5)=13$.}

\exercise{$\ds\lim_{x\to 5}(3-x)= -2$}{%Let $\epsilon >0$ be given. We wish to find $\delta >0$ such that when $\abs{x-5}<\delta$, $\abs{f(x)-(-2)}<\epsilon$. 
%
%Scratch-Work:
%Consider $\abs{f(x)-(-2)}<\epsilon$:
%\begin{gather*}
%\abs{f(x) + 2 } < \epsilon \\
%\abs{(3-x) + 2 }<\epsilon \\
%\abs{ 5-x } < \epsilon \\
%-\epsilon < 5-x < \epsilon \\
%-\epsilon < x-5 < \epsilon. \\
%\end{gather*}
%This implies we can let $\delta =\epsilon$.
%
%Proof:
Given $\epsilon>0$, choose $\delta=\epsilon$.
\begin{gather*}
\abs{x-5}<\delta=\epsilon \\
\abs{5-x}<\epsilon \\
%-\delta < x-5 < \delta\\
%-\epsilon < x-5 < \epsilon\\
%-\epsilon < (x-3)-2 < \epsilon \\
%-\epsilon < (-x+3)-(-2) < \epsilon \\
\abs{3-x - (-2)} < \epsilon.
\end{gather*}
Thus $\ds\lim_{x\to 5}(3-x)= -2$.}

\exercise{$\ds\lim_{x\to 5}(4x-12)=8$}{%Let $\epsilon >0$ be given. We wish to find $\delta >0$ such that when $\abs{x-5}<\delta$, $\abs{f(x)-8}<\epsilon$. 
%
%Scratch-Work:
%Consider $\abs{f(x)-8}<\epsilon$, keeping in  mind we want to make a statement about $\abs{x-5}$:
%\begin{align*}
%\abs{f(x)-8}&<\epsilon \\
%\abs{4x-12-8}&<\epsilon \\
%\abs{4x-20}&<\epsilon\\
%4\abs{x-5}&<\epsilon \\
%\abs{x-5}&<\frac{\epsilon}{4}\\
%\end{align*}
%suggesting $\delta =\frac{\epsilon}4$.
%
%Proof:
Given $\epsilon>0$, let $\delta =\dfrac{\epsilon}4$. Then:
\begin{align*}
\abs{x - 5} &< \delta=\frac{\epsilon}{4} \\
4\abs{x - 5} &< \frac{\epsilon}{4}\cdot 4\\
\abs{4x-20}&< \epsilon \\
\abs{4x-12-8}&< \epsilon  \\
\end{align*}
Thus  $\ds \lim_{x\to 5}(4x-12)=8$.}

\exercise{$\ds \lim_{x\to 3}(5-2x)=-1$}{%Let $\epsilon >0$ be given. We wish to find $\delta >0$ such that when $\abs{x-3}<\delta$, $\abs{f(x)-(-1)}<\epsilon$. 
%
%Scratch-Work:
%Consider $\abs{f(x)-(-1)}<\epsilon$, keeping in  mind we want to make a statement about $\abs{x-3}$:
%\begin{align*}
%\abs{f(x)-(-1)}&<\epsilon \\
%\abs{5-2x+1}&<\epsilon \\
%\abs{-2x+6}&<\epsilon\\
%2\abs{x-3}&<\epsilon \\
%\abs{x-3}&<\frac{\epsilon}{2}\\
%\end{align*}
%suggesting $\delta=\frac{\epsilon}{2}$.
%
%Proof:
Given $\epsilon>0$, let $\delta =\dfrac{\epsilon}2$. Then:
\begin{align*}
\abs{x - 3} &< \delta=\frac{\epsilon}{2} \\
2\abs{x - 3} &< \frac{\epsilon}{2}\cdot 2\\
\abs{-2x+6}&< \epsilon \\
\abs{5-2x+1}&< \epsilon  \\
\end{align*}
Thus  $\ds \lim_{x\to 3}(5-2x)=-1$.}

\exercise{$\ds\lim_{x\to 3}\left(x^2-3\right)= 6$}{%Let $\epsilon >0$ be given. We wish to find $\delta >0$ such that when $\abs{x-3}<\delta$, $\abs{f(x)-6}<\epsilon$. 
%
%Scratch-Work:
%Consider $\abs{f(x)-6}<\epsilon$, keeping in  mind we want to make a statement about $\abs{x-3}$:
%\begin{gather*}
%\abs{f(x) -6 } < \epsilon \\
%\abs{x^2-3 -6 }<\epsilon \\
%\abs{ x^2-9 } < \epsilon \\
%\abs{ x-3 }\cdot\abs{x+3} < \epsilon \\
%\abs{ x-3 } < \epsilon/\abs{x+3} \\
%\end{gather*}
%Since $x$ is near 3, we can safely assume that, for instance, $2<x<4$. Thus
%\begin{gather*}
%2+3<x+3<4+3 \\
%5 < x+3 < 7 \\
%\frac{1}{7} < \frac{1}{x+3} < \frac{1}{5} \\
%\frac{\epsilon}{7} < \frac{\epsilon}{x+3} < \frac{\epsilon}{5} \\
%\end{gather*}
%This suggests $\delta =\frac{\epsilon}{7}$.
%
%Proof:
Given $\epsilon>0$, let $\delta =\frac{\epsilon}{7}$.
\begin{gather*}
\abs{x-3}<\delta=\frac{\epsilon}7\\
\abs{x-3} < \frac{\epsilon}{x+3}\\
\abs{x-3}\cdot\abs{x+3} < \frac{\epsilon}{x+3}\cdot\abs{x+3}\\
\end{gather*}
Assuming $x$ is near 3, $x+3$ is positive and we can drop the absolute value signs on the right.
\begin{gather*}
\abs{x-3}\cdot\abs{x+3} < \frac{\epsilon}{x+3}\cdot(x+3)\\
\abs{x^2-9} < \epsilon\\
\abs{(x^2-3) - 6} < \epsilon.
\end{gather*}
Thus, $\ds\lim_{x\to 3}\left(x^2-3\right)= 6$.
}

\exercise{$\ds\lim_{x\to 4}\left(x^2+x-5\right)= 15$}{%Let $\epsilon >0$ be given. We wish to find $\delta >0$ such that when $\abs{x-4}<\delta$, $\abs{f(x)-15}<\epsilon$. 
%
%Scratch-Work:
%Consider $\abs{f(x)-15}<\epsilon$, keeping in  mind we want to make a statement about $\abs{x-4}$:
%\begin{gather*}
%\abs{f(x) -15 } < \epsilon \\
%\abs{x^2+x-5 -15 }<\epsilon \\
%\abs{ x^2+x-20 } < \epsilon \\
%\abs{ x-4 }\cdot\abs{x+5} < \epsilon \\
%\abs{ x-4 } < \epsilon/\abs{x+5} \\
%\end{gather*}
%Since $x$ is near 4, we can safely assume that, for instance, $3<x<5$. Thus
%\begin{gather*}
%3+5<x+5<5+5 \\
%8 < x+5 < 10 \\
%\frac{1}{10} < \frac{1}{x+5} < \frac{1}{8} \\
%\frac{\epsilon}{10} < \frac{\epsilon}{x+5} < \frac{\epsilon}{8} \\
%\end{gather*}
%suggesting $\delta =\frac{\epsilon}{10}$.
%
%Proof:
Given $\epsilon>0$, let $\delta =\frac{\epsilon}{10}$. Then:
\begin{gather*}
\abs{x-4}<\delta=\frac{\epsilon}{10}\\
\abs{x-4} < \frac{\epsilon}{x+5}\\
\abs{x-4}\cdot\abs{x+5} < \frac{\epsilon}{x+5}\cdot\abs{x+5}\\
\end{gather*}
Assuming $x$ is near 4, $x+5$ is positive and we can drop the absolute value signs on the right.
\begin{gather*}
\abs{x-4}\cdot\abs{x+5} < \frac{\epsilon}{x+5}\cdot(x+5)\\
\abs{x^2+x-20} < \epsilon\\
\abs{(x^2+x-5) -15} < \epsilon.
\end{gather*}
Thus, $\ds\lim_{x\to 4}\left(x^2+x-5\right)= 15$.}

\exercise{$\ds\lim_{x\to1}\left(2x^2+3x+1\right)=6$}{Given $\epsilon>0$, let $\delta =\min\{1,\frac\epsilon9\}$. Then:
\begin{gather*}
\abs{x-1}<\delta \\
\abs{x-1}<\frac\epsilon9\\
\abs{x-1}<\frac\epsilon{2x+5}\\
\abs{x-1}\cdot\abs{2x+5}<\frac\epsilon{2x+5}\cdot\abs{2x+5}=\epsilon\\
%\end{gather*}
%Assuming $x$ is near 1, $2x+5$ is positive and we can drop the absolute value signs on the right.
%\begin{gather*}
%|x-1|\cdot|2x+5| < \frac{\epsilon}{2x+5}\cdot(2x+5)\\
\abs{2x^2+3x-5}<\epsilon\\
\abs{(2x^2+3x+1)-6}<\epsilon.
\end{gather*}
Thus, $\ds\lim_{x\to1}\left(2x^2+3x+1\right)=6$.}

\exercise{$\ds\lim_{x\to 2}\left(x^3-1\right)= 7$}{%Let $\epsilon >0$ be given. We wish to find $\delta >0$ such that when $\abs{x-2}<\delta$, $\abs{f(x)-7}<\epsilon$. 
%
%Scratch-Work:
%Consider $\abs{f(x)-7}<\epsilon$, keeping in  mind we want to make a statement about $\abs{x-2}$:
%\begin{gather*}
%\abs{f(x) -7 } < \epsilon \\
%\abs{x^3-1 -7 }<\epsilon \\
%\abs{ x^3-8 } < \epsilon \\
%\abs{ x-2 }\cdot\abs{x^2+2x+4} < \epsilon \\
%\abs{ x-3 } < \epsilon/\abs{x^2+2x+4} \\
%\end{gather*}
%Since $x$ is near 2, we can safely assume that, for instance, $1<x<3$. Thus
%\begin{gather*}
%1^2+2\cdot1+4<x^2+2x+4<3^2+2\cdot3+4 \\
%7 < x^2+2x+4 < 19 \\
%\frac{1}{19} < \frac{1}{x^2+2x+4} < \frac{1}{7} \\
%\frac{\epsilon}{19} < \frac{\epsilon}{x^2+2x+4} < \frac{\epsilon}{7}, \\
%\end{gather*}
%suggesting $\delta =\frac{\epsilon}{19}$.
%
%Proof:
Given $\epsilon>0$, let $\delta =\frac{\epsilon}{19}$.
\begin{gather*}
\abs{x-2}<\delta=\frac{\epsilon}{19}\\
\abs{x-2} < \frac{\epsilon}{x^2+2x+4}\\
\abs{x-2}\cdot\abs{x^2+2x+4} < \frac{\epsilon}{x^2+2x+4}\cdot\abs{x^2+2x+4}\\
\end{gather*}
Assuming $x$ is near 2, $x^2+2x+4$ is positive and we can drop the absolute value signs on the right.
\begin{gather*}
\abs{x-2}\cdot\abs{x^2+2x+4} < \frac{\epsilon}{x^2+2x+4}\cdot(x^2+2x+4)\\
\abs{x^3-8} < \epsilon\\
\abs{(x^3-1) - 7} < \epsilon,
\end{gather*}
which is what we wanted to prove.}

\exercise{$\ds\lim_{x\to 2} 5 = 5$}{Let $\epsilon >0$ be given. We wish to find $\delta >0$ such that when $\abs{x-2}<\delta$, $\abs{f(x)-5}<\epsilon$. However, since $f(x)=5$, a constant function, the latter inequality is simply $\abs{5-5}<\epsilon$, which is always true. Thus we can choose any $\delta$ we like; we arbitrarily choose $\delta =\epsilon$. }

\exercise{$\ds\lim_{x\to0}\left(e^{2x}-1\right)=0$}{%Let $\epsilon >0$ be given. We wish to find $\delta >0$ such that when $\abs{x-0}<\delta$, $\abs{f(x)-0}<\epsilon$. 
%
%Consider $\abs{f(x)-0}<\epsilon$, keeping in  mind we want to make a statement about $\abs{x-0}$ (i.e., $\abs x$):
%\begin{gather*}
%\abs{f(x)-0}< \epsilon \\
%\abs{e^{2x}-1}<\epsilon \\
%-\epsilon< e^{2x}-1 < \epsilon \\
%1-\epsilon< e^{2x} < 1+\epsilon \\
%\ln (1-\epsilon) < 2x < \ln (1+\epsilon) \\
%\frac{\ln (1-\epsilon)}{2} < x < \frac{\ln (1+\epsilon)}{2} \\
%\end{gather*}
%Let $\delta = \min\left\{\abs{\frac{\ln(1-\epsilon)}2},\frac{\ln(1+\epsilon)}{2}\right\}=\frac{\ln(1+\epsilon)}2.$
%
Given $\epsilon>0$, let $\delta=\frac{\ln(1+\epsilon)}2<\abs{\frac{\ln(1-\epsilon)}2}$.
\begin{gather*}
\abs x<\delta \\
%|x| < \min\left\{\left|\frac{\ln(1-\epsilon)}{2}\right|,\left|\frac{\ln(1+\epsilon)}{2}\right|\right\} \\
\abs x<\frac{\ln(1+\epsilon)}2<\abs{\frac{\ln(1-\epsilon)}2} \\
\frac{\ln(1-\epsilon)}2<x<\frac{\ln(1+\epsilon)}2\\
\ln(1-\epsilon)<2x<\ln(1+\epsilon)\\
1-\epsilon<e^{2x}<1+\epsilon\\
-\epsilon<e^{2x}-1<\epsilon\\
\abs{e^{2x}-1-(0)}<\epsilon,
\end{gather*}
which is what we wanted to prove.}

\exercise{$\ds\lim_{x\to1}\frac1x=1$}{Given $\epsilon>0$, let $\delta =\min\{\frac12,\frac\epsilon2\}$. Then:
\begin{gather*}
\abs{x-1}<\delta \\
\abs{x-1}<\frac\epsilon2\\
\abs{x-1}<\epsilon\cdot x\\
\abs{x-1}/x < \epsilon\\
%\end{gather*}
%Assuming $x$ is near 1, $x$ is positive and we can bring it into the absolute value signs on the left.
%\begin{gather*}
\abs{(x-1)/x} < \epsilon\\
\abs{1-1/x}< \epsilon\\
\abs{(1/x)-1}< \epsilon,
\end{gather*}
which is what we wanted to prove.}

\exercise{$\ds\lim_{x\to 0} \sin x= 0$ (Hint: use the fact that $\abs{\sin x} \leq \abs{x}$, with equality only when $x=0$.)}{Let $\epsilon >0$ be given. We wish to find $\delta >0$ such that when $\abs{x-0}<\delta$, $\abs{f(x)-0}<\epsilon$. In simpler terms, we want to show that when $\abs{x}<\delta$, $\abs{\sin x} < \epsilon$. 

Set $\delta = \epsilon$. We start with assuming that $\abs{x}<\delta$. Using the hint, we have that $\abs{\sin x } < \abs{x} < \delta = \epsilon$. Hence if $\abs{x}<\delta$, we know immediately that $\abs{\sin x} < \epsilon$.}

}
