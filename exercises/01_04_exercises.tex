\printconcepts
\exercise{What are the three ways in which a limit may fail to exist?}{The function approaches different values from the left and right; the function grows without bound; the function oscillates.}

\exercise{T/F: If $\ds \lim_{x\to 1^-} f(x) = 5$, then $\ds \lim_{x\to 1} f(x) = 5$}{F}

\exercise{T/F: If $\ds \lim_{x\to 1^-} f(x) = 5$, then $\ds \lim_{x\to 1^+} f(x) = 5$}{F}

\exercise{T/F: If $\ds \lim_{x\to 1} f(x) = 5$, then $\ds \lim_{x\to 1^-} f(x) = 5$}{T}

\printproblems
\exerciseset{In Exercises}{, evaluate each expression using the given graph of $f(x)$.}{

\exercise{
\begin{minipage}{\linewidth}\centering
\myincludegraphics[scale=.8]{figures/fig01_04_ex_05}
%\captionsetup{type=figure}%
%\caption{Setting up Integration by Parts.}\label{fig:ibp7}
\end{minipage}

\noindent\begin{minipage}[t]{.49\linewidth}
\begin{enumerate}
\item		$\ds \lim_{x\to 1^-} f(x)$
\item		$\ds \lim_{x\to 1^+} f(x)$
\item		$\ds \lim_{x\to 1} f(x)$
\end{enumerate}
\end{minipage}
\noindent\begin{minipage}[t]{.49\linewidth}
\begin{enumerate}\addtocounter{enumii}{3}
\item		$f(1)$
\item		$\ds \lim_{x\to 0^-} f(x)$
\item		$\ds \lim_{x\to 0^+} f(x)$
\end{enumerate}
\end{minipage}
}{\begin{enumerate}
\item		2
\item		2
\item		2
\item		1
\item	 	As $f$ is not defined for $x<0$, this limit is not defined.
\item		1
\end{enumerate}}

\exercise{
\noindent\begin{minipage}{\linewidth}\centering
\myincludegraphics[scale=.8]{figures/fig01_04_ex_06}
%\captionsetup{type=figure}%
%\caption{Setting up Integration by Parts.}\label{fig:ibp7}
\end{minipage}

\noindent\begin{minipage}[t]{.49\linewidth}
\begin{enumerate}
\item		$\ds \lim_{x\to 1^-} f(x)$
\item		$\ds \lim_{x\to 1^+} f(x)$
\item		$\ds \lim_{x\to 1} f(x)$
\end{enumerate}
\end{minipage}
\noindent\begin{minipage}[t]{.49\linewidth}
\begin{enumerate}\addtocounter{enumii}{3}
\item		$f(1)$
\item		$\ds \lim_{x\to 2^-} f(x)$
\item		$\ds \lim_{x\to 2^+} f(x)$
\end{enumerate}
\end{minipage}
%\ifthenelse{\boolean{printquestions}}{\columnbreak}{}
}{\begin{enumerate}
\item		1
\item		2
\item		Does not exist.
\item		2
\item		0
\item	 	As $f$ is not defined for $x>2$, this limit is not defined.
\end{enumerate}
}

%\exercise{
%\noindent\begin{minipage}{\linewidth}\centering
%\myincludegraphics[scale=.8]{figures/fig01_04_ex_07}
%%\captionsetup{type=figure}%
%%\caption{Setting up Integration by Parts.}\label{fig:ibp7}
%\end{minipage}
%
%\noindent\begin{minipage}[t]{.49\linewidth}
%\begin{enumerate}
%\item		$\ds \lim_{x\to 1^-} f(x)$
%\item		$\ds \lim_{x\to 1^+} f(x)$
%\item		$\ds \lim_{x\to 1} f(x)$
%\end{enumerate}
%\end{minipage}
%\noindent\begin{minipage}[t]{.49\linewidth}
%\begin{enumerate}\addtocounter{enumii}{3}
%\item		$f(1)$
%\item		$\ds \lim_{x\to 2^-} f(x)$
%\item		$\ds \lim_{x\to 0^+} f(x)$
%\end{enumerate}
%\end{minipage}
%}{\begin{enumerate}
%\item		Does not exist.
%\item		Does not exist.
%\item		Does not exist.
%\item		Not defined.
%\item		0
%\item	 	0
%\end{enumerate}
%}

\exercise{
\noindent\begin{minipage}{\linewidth}\centering
\myincludegraphics[scale=.8]{figures/fig01_04_ex_08}
%\captionsetup{type=figure}%
%\caption{Setting up Integration by Parts.}\label{fig:ibp7}
\end{minipage}

\noindent\begin{minipage}[t]{.49\linewidth}
\begin{enumerate}
\item		$\ds \lim_{x\to 1^-} f(x)$
\item		$\ds \lim_{x\to 1^+} f(x)$
\end{enumerate}
\end{minipage}
\noindent\begin{minipage}[t]{.49\linewidth}
\begin{enumerate}\addtocounter{enumii}{2}
\item		$\ds \lim_{x\to 1} f(x)$
\item		$f(1)$
%\item		$\ds \lim_{x\to 2^-} f(x)$
%\item		$\ds \lim_{x\to 0^+} f(x)$
\end{enumerate}
\end{minipage}
}{\begin{enumerate}
\item		2
\item		0
\item		Does not exist.
\item		1
\end{enumerate}
}

\exercise{
\noindent\begin{minipage}{\linewidth}\centering
\myincludegraphics[scale=.8]{figures/fig01_04_ex_09}
%\captionsetup{type=figure}%
%\caption{Setting up Integration by Parts.}\label{fig:ibp7}
\end{minipage}

\noindent\begin{minipage}[t]{.49\linewidth}
\begin{enumerate}
\item		$\ds \lim_{x\to 1^-} f(x)$
\item		$\ds \lim_{x\to 1^+} f(x)$
\end{enumerate}
\end{minipage}
\noindent\begin{minipage}[t]{.49\linewidth}
\begin{enumerate}\addtocounter{enumii}{2}
\item		$\ds \lim_{x\to 1} f(x)$
\item		$f(1)$
%\item		$\ds \lim_{x\to 2^-} f(x)$
%\item		$\ds \lim_{x\to 0^+} f(x)$
\end{enumerate}
\end{minipage}
}{\begin{enumerate}
\item		2
\item		2
\item		2
\item		2
\end{enumerate}
}

\exercise{
\noindent\begin{minipage}{\linewidth}\centering
\myincludegraphics[scale=.8]{figures/fig01_04_ex_10}
%\captionsetup{type=figure}%
%\caption{Setting up Integration by Parts.}\label{fig:ibp7}
\end{minipage}

\noindent\begin{minipage}[t]{.49\linewidth}
\begin{enumerate}
\item		$\ds \lim_{x\to 0^-} f(x)$
\item		$\ds \lim_{x\to 0^+} f(x)$
\end{enumerate}
\end{minipage}
\noindent\begin{minipage}[t]{.49\linewidth}
\begin{enumerate}\addtocounter{enumii}{2}
\item		$\ds \lim_{x\to 0} f(x)$
\item		$f(0)$
%\item		$\ds \lim_{x\to 2^-} f(x)$
%\item		$\ds \lim_{x\to 0^+} f(x)$
\end{enumerate}
\end{minipage}
}{\begin{enumerate}
\item		4
\item		$-4$
\item		Does not exist.
\item		0
\end{enumerate}
}

\exercise{
\noindent\begin{minipage}{\linewidth}\centering
\myincludegraphics[scale=.8]{figures/fig01_04_ex_11}
%\captionsetup{type=figure}%
%\caption{Setting up Integration by Parts.}\label{fig:ibp7}
\end{minipage}

\noindent\begin{minipage}[t]{.49\linewidth}
\begin{enumerate}
\item		$\ds \lim_{x\to -2^-} f(x)$
\item		$\ds \lim_{x\to -2^+} f(x)$
\item		$\ds \lim_{x\to -2} f(x)$
\item		$f(-2)$
\end{enumerate}
\end{minipage}
\noindent\begin{minipage}[t]{.49\linewidth}
\begin{enumerate}\addtocounter{enumii}{4}
\item		$\ds \lim_{x\to 2^-} f(x)$
\item		$\ds \lim_{x\to 2^+} f(x)$
\item		$\ds \lim_{x\to 2} f(x)$
\item		$f(2)$
\end{enumerate}
\end{minipage}
}{\begin{enumerate}
\item		2
\item		2
\item		2
\item		0
\item		2
\item		2
\item		2
\item		Not defined

\end{enumerate}
}

\exercise{
\noindent\begin{minipage}{\linewidth}\centering
\myincludegraphics[scale=.8]{figures/fig01_04_ex_12}
%\captionsetup{type=figure}%
%\caption{Setting up Integration by Parts.}\label{fig:ibp7}
\end{minipage}

Let $-3\leq a\leq 3$ be an integer.

\noindent\begin{minipage}[t]{.49\linewidth}
\begin{enumerate}
\item		$\ds \lim_{x\to a^-} f(x)$
\item		$\ds \lim_{x\to a^+} f(x)$
\end{enumerate}
\end{minipage}
\noindent\begin{minipage}[t]{.49\linewidth}
\begin{enumerate}\addtocounter{enumii}{2}
\item		$\ds \lim_{x\to a} f(x)$
\item		$f(a)$\end{enumerate}
\end{minipage}
}{\begin{enumerate}
\item		$a-1$
\item		$a$
\item		Does not exist.
\item		$a$
\end{enumerate}
}
}
\exerciseset{In Exercises}{, evaluate the given limits of the piecewise defined functions $f$.}{

\exercise{$\ds f(x) = \begin{cases}
	x+1 & x\leq 1 \\
	x^2-5 & x>1
	\end{cases}$

\noindent\begin{minipage}[t]{.49\linewidth}
\begin{enumerate}
\item		$\ds \lim_{x\to 1^-} f(x)$
\item		$\ds \lim_{x\to 1^+} f(x)$
\end{enumerate}
\end{minipage}
\noindent\begin{minipage}[t]{.49\linewidth}
\begin{enumerate}\addtocounter{enumii}{2}
\item		$\ds \lim_{x\to 1} f(x)$
\item		$f(1)$\end{enumerate}
\end{minipage}
}{\begin{enumerate}
\item		2
\item		$-4$
\item		Does not exist.
\item		2
\end{enumerate}}

\exercise{$f(x) = \begin{cases}
	2x^2+5x-1 & x<0 \\
	\sin x & x\geq 0
	\end{cases}$

\noindent\begin{minipage}[t]{.5\linewidth}
\begin{enumerate}
\item		$\ds \lim_{x\to 0^-} f(x)$
\item		$\ds \lim_{x\to 0^+} f(x)$
\end{enumerate}
\end{minipage}%
\begin{minipage}[t]{.5\linewidth}
\begin{enumerate}\addtocounter{enumii}{2}
\item		$\ds \lim_{x\to 0} f(x)$
\item		$f(0)$\end{enumerate}
\end{minipage}
}{\begin{enumerate}
\item		$-1$
\item		0
\item		Does not exist.
\item		0
\end{enumerate}}

\exercise{$\ds f(x) = \begin{cases}
	x^2-1 & x<-1 \\
	x^3+1 & -1\leq x\leq 1\\
	x^2+1 & x>1
	\end{cases}$

\noindent\begin{minipage}[t]{.49\linewidth}
\begin{enumerate}
\item		$\ds \lim_{x\to -1^-} f(x)$
\item		$\ds \lim_{x\to -1^+} f(x)$
\item		$\ds \lim_{x\to -1} f(x)$
\item		$f(-1)$
\end{enumerate}
\end{minipage}
\noindent\begin{minipage}[t]{.49\linewidth}
\begin{enumerate}\addtocounter{enumii}{4}
\item		$\ds \lim_{x\to 1^-} f(x)$
\item		$\ds \lim_{x\to 1^+} f(x)$
\item		$\ds \lim_{x\to 1} f(x)$
\item		$f(1)$
\end{enumerate}
\end{minipage}
%\ifthenelse{\boolean{printquestions}}{\columnbreak}{}
}{\begin{enumerate}
\item		0
\item		0
\item	  0
\item		0
\item		2
\item		2
\item		2
\item		2
\end{enumerate}}

\exercise{$\ds f(x) = \begin{cases}
	\cos x & x<\pi \\
	\sin x & x\geq \pi
	\end{cases}$

\noindent\begin{minipage}[t]{.49\linewidth}
\begin{enumerate}
\item		$\ds \lim_{x\to \pi^-} f(x)$
\item		$\ds \lim_{x\to \pi^+} f(x)$
%\item		$\ds \lim_{x\to -1} f(x)$
%\item		$f(-1)$
\end{enumerate}
\end{minipage}
\noindent\begin{minipage}[t]{.49\linewidth}
\begin{enumerate}\addtocounter{enumii}{2}
%\item		$\ds \lim_{x\to \pi} f(x)$
%\item		$\ds \lim_{x\to 1^+} f(x)$
\item		$\ds \lim_{x\to \pi} f(x)$
\item		$f(\pi)$
\end{enumerate}
\end{minipage}
}{\begin{enumerate}
\item		$-1$
\item		0
\item		Does not exist.
\item		0
\end{enumerate}}

\exercise{$\ds f(x) = \begin{cases}
	1-\cos^2 x & x<a \\
	\sin^2 x & x\geq a
	\end{cases},$\\
where $a$ is a real number.

\noindent\begin{minipage}[t]{.49\linewidth}
\begin{enumerate}
\item		$\ds \lim_{x\to a^-} f(x)$
\item		$\ds \lim_{x\to a^+} f(x)$
%\item		$\ds \lim_{x\to -1} f(x)$
%\item		$f(-1)$
\end{enumerate}
\end{minipage}
\noindent\begin{minipage}[t]{.49\linewidth}
\begin{enumerate}\addtocounter{enumii}{2}
%\item		$\ds \lim_{x\to \pi} f(x)$
%\item		$\ds \lim_{x\to 1^+} f(x)$
\item		$\ds \lim_{x\to a} f(x)$
\item		$f(a)$
\end{enumerate}
\end{minipage}
}{\begin{enumerate}
\item		$1-\cos^2 a = \sin^2 a$
\item		$\sin^2 a$
\item		$\sin^2 a$
\item		$\sin ^2 a$
\end{enumerate}}

\exercise{$\ds f(x) = \begin{cases}
	x+1 & x<1 \\
	1  & x=1\\
	x-1 & x>1
	\end{cases}$

\noindent\begin{minipage}[t]{.49\linewidth}
\begin{enumerate}
\item		$\ds \lim_{x\to 1^-} f(x)$
\item		$\ds \lim_{x\to 1^+} f(x)$
%\item		$\ds \lim_{x\to -1} f(x)$
%\item		$f(-1)$
\end{enumerate}
\end{minipage}
\noindent\begin{minipage}[t]{.49\linewidth}
\begin{enumerate}\addtocounter{enumii}{2}
%\item		$\ds \lim_{x\to \pi} f(x)$
%\item		$\ds \lim_{x\to 1^+} f(x)$
\item		$\ds \lim_{x\to 1} f(x)$
\item		$f(1)$
\end{enumerate}
\end{minipage}
}{\begin{enumerate}
\item		2
\item		0
\item		Does not exist
\item	  1
\end{enumerate}}

\exercise{$\ds f(x) = \begin{cases}
	x^2 & x<2 \\
	x+1  & x=2\\
	-x^2+2x+4 & x>2
	\end{cases}$

\noindent\begin{minipage}[t]{.49\linewidth}
\begin{enumerate}
\item		$\ds \lim_{x\to 2^-} f(x)$
\item		$\ds \lim_{x\to 2^+} f(x)$
%\item		$\ds \lim_{x\to -1} f(x)$
%\item		$f(-1)$
\end{enumerate}
\end{minipage}
\noindent\begin{minipage}[t]{.49\linewidth}
\begin{enumerate}\addtocounter{enumii}{2}
%\item		$\ds \lim_{x\to \pi} f(x)$
%\item		$\ds \lim_{x\to 1^+} f(x)$
\item		$\ds \lim_{x\to 2} f(x)$
\item		$f(2)$
\end{enumerate}
\end{minipage}
}{\begin{enumerate}
\item		4
\item		4
\item		4
\item	  3
\end{enumerate}}

\exercise{$\ds f(x) = \begin{cases}
	a(x-b)^2+c & x<b \\
	a(x-b)+c & x\geq b
	\end{cases},$

where $a$, $b$ and $c$ are real numbers.

\noindent\begin{minipage}[t]{.49\linewidth}
\begin{enumerate}
\item		$\ds \lim_{x\to b^-} f(x)$
\item		$\ds \lim_{x\to b^+} f(x)$
%\item		$\ds \lim_{x\to -1} f(x)$
%\item		$f(-1)$
\end{enumerate}
\end{minipage}
\noindent\begin{minipage}[t]{.49\linewidth}
\begin{enumerate}\addtocounter{enumii}{2}
%\item		$\ds \lim_{x\to \pi} f(x)$
%\item		$\ds \lim_{x\to 1^+} f(x)$
\item		$\ds \lim_{x\to b} f(x)$
\item		$f(b)$
\end{enumerate}
\end{minipage}
}{\begin{enumerate}
\item		$c$
\item		$c$
\item		$c$
\item	  $c$
\end{enumerate}}

\exercise{$\ds f(x) = \begin{cases}
	\frac{\abs x}{x} & x\neq 0 \\
	0 & x=0
	\end{cases}$

\noindent\begin{minipage}[t]{.49\linewidth}
\begin{enumerate}
\item		$\ds \lim_{x\to 0^-} f(x)$
\item		$\ds \lim_{x\to 0^+} f(x)$
%\item		$\ds \lim_{x\to -1} f(x)$
%\item		$f(-1)$
\end{enumerate}
\end{minipage}
\noindent\begin{minipage}[t]{.49\linewidth}
\begin{enumerate}\addtocounter{enumii}{2}
%\item		$\ds \lim_{x\to \pi} f(x)$
%\item		$\ds \lim_{x\to 1^+} f(x)$
\item		$\ds \lim_{x\to 0} f(x)$
\item		$f(0)$
\end{enumerate}
\end{minipage}
}{\begin{enumerate}
\item		$-1$
\item		$1$
\item		Does not exist
\item	  $0$
\end{enumerate}}

\exercise{$\ds\lim_{x\to 4}\frac{\abs{4-x}}{x-4}$

\noindent\begin{minipage}[t]{.5\linewidth}
\begin{enumerate}
\item		$\ds \lim_{x\to 4^-} f(x)$
\item		$\ds \lim_{x\to 4^+} f(x)$
\end{enumerate}
\end{minipage}%
\begin{minipage}[t]{.5\linewidth}
\begin{enumerate}\addtocounter{enumii}{2}
\item		$\ds \lim_{x\to 4} f(x)$
\item		$f(4)$\end{enumerate}
\end{minipage}}
{\begin{enumerate}
\item		$-1$
\item		$1$
\item		Does not exist.
\item		Undefined
\end{enumerate}}

\exercise{$\ds \lim_{x\to -2} \frac{x+2}{\abs{x+2}}$

\noindent\begin{minipage}[t]{.5\linewidth}
\begin{enumerate}
\item		$\ds \lim_{x\to -2^-} f(x)$
\item		$\ds \lim_{x\to -2^+} f(x)$
\end{enumerate}
\end{minipage}%
\noindent\begin{minipage}[t]{.5\linewidth}
\begin{enumerate}\addtocounter{enumii}{2}
\item		$\ds \lim_{x\to -2} f(x)$
\item		$f(-2)$\end{enumerate}
\end{minipage}}
{\begin{enumerate}
\item		$-1$
\item		$1$
\item		Does not exist.
\item		undefined
\end{enumerate}}

}

{\noindent In Exercises}
{, sketch the graph of a function $f$ that satisfies all of the given conditions.}
\exinput{exercises/01_04_ex_30}
\exinput{exercises/01_04_ex_31}
\exinput{exercises/01_04_ex_32}
\exinput{exercises/01_04_ex_33}

\printreview
\exercise{Evaluate the limit: $\ds \lim_{x\to -1} \frac{x^2+5 x+4}{x^2-3 x-4}$.
}{$-3/5$
}

\exercise{Evaluate the limit: $\ds \lim_{x\to -4} \frac{x^2-16}{x^2-4 x-32}$.
}{$2/3$
}

\exercise{Evaluate the limit: $\ds \lim_{h\to 0} \frac{\sqrt{3+h}-\sqrt{3}}{h}$.}{$\frac{1}{2\sqrt{3}}$}

%\exercise{Evaluate the limit: $\ds \lim_{x\to 2} \frac{x^2-6 x+9}{x^2-3 x}$.}{$-1/2$}

\exercise{Approximate  the limit numerically: $\ds \lim_{h\to 0} \frac{(2+h)^2-4}{h}$.}{$2$}

\exercise{Approximate  the limit numerically: $\ds \lim_{x\to 0.2} \frac{x^2+5.8 x-1.2}{x^2-4.2 x+0.8}$.}{$-1.63$}

%\exercise{Approximate  the limit numerically: $\ds \lim_{x\to -0.5} \frac{x^2-0.5 x-0.5}{x^2+6.5 x+3}$.}{$-3/11$}

%\exercise{Approximate  the limit numerically: $\ds \lim_{x\to 0.1} \frac{x^2+0.9 x-0.1}{x^2+7.9 x-0.8}$.}{$11/81$}
