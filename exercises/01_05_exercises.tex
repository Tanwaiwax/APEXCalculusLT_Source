\printconcepts

\exercise{In your own words, describe what it means for a function to be continuous.}{Answers will vary.}

\exercise{In your own words, describe what the Intermediate Value Theorem states.}{Answers will vary.}

\exercise{What is a ``root'' of a function?}{A root of a function $f$ is a value $c$ such that $f(c)=0$.}

\exercise{Given functions $f$ and $g$ on an interval $I$, how can the Bisection Method be used to find a value $c$ where $f(c) = g(c)$?}{Consider the function $h(x) = g(x) - f(x)$, and use the Bisection Method to find a root of $h$.}

\exercise{T/F:	If $f$ is defined on an open interval containing $c$, and $\ds \lim_{x\to c}f(x)$ exists, then $f$ is continuous at $c$.}{F}

\exercise{T/F: If $f$ is continuous at $c$, then $\ds \lim_{x\to c}f(x)$ exists.}{T}

\exercise{T/F: If $f$ is continuous at $c$, then $\ds \lim_{x\to c^+}f(x) = f(c)$.}{T}

\exercise{T/F: If $f$ is continuous on $[a,b]$, then $\ds\lim_{x\to a^-}f(x) = f(a)$.}{F}

\exercise{T/F: If $f$ is continuous on $[0,1)$ and $[1,2)$, then $f$ is continuous on $[0,2)$.}{F}

\exercise{T/F: The sum of continuous functions is also continuous.}{T}

\printproblems

\exerciseset{In Exercises}{, a graph of a function $f$ is given along with a value $a$. Determine if $f$ is continuous at $a$; if it is not, state why it is not.}{

\exercise{\noindent $a = 1$

\myincludegraphics[scale=.8]{figures/fig01_04_ex_05}
}{No; $\ds \lim_{x\to 1} f(x) = 2$, while $f(1) = 1$.}

\exercise{\noindent $a = 1$

\myincludegraphics[scale=.8]{figures/fig01_04_ex_06}
}{No; $\ds \lim_{x\to 1} f(x)$ does not exist.}

\exercise{\noindent $a = 1$

\myincludegraphics[scale=.8]{figures/fig01_04_ex_07}
}{No; $f(1)$ does not exist.}

\exercise{\noindent $a = 0$

\myincludegraphics[scale=.8]{figures/fig01_04_ex_08}
}{Yes}

\exercise{\noindent $a = 1$

\myincludegraphics[scale=.8]{figures/fig01_04_ex_09}
}{Yes}

\exercise{\noindent $a = 4$

\myincludegraphics[scale=.8]{figures/fig01_04_ex_10}

}{Yes}

\exercise{\begin{enumerate}
\item		$a = -2$
\item		$a=0$
\item		$a=2$
\end{enumerate}

\myincludegraphics[scale=.8]{figures/fig01_04_ex_11}
}{\begin{enumerate}
\item		No; $\ds \lim_{x\to -2}f(x) \neq f(-2)$
\item		Yes
\item		No; $f(2)$ is not defined.
\end{enumerate}
}
}

%\ifthenelse{\boolean{printquestions}}{\columnbreak}{}

\input{exercises/01_05_exset_02}

\exerciseset{In Exercises}{, give the intervals on which the given function is continuous.}{

\exercise{$f(x) = x^2-3x+9$}{$(-\infty,\infty)$}

\exercise{$\ds g(x) = \sqrt{x^2-4}$}{$(-\infty,-2]\cup [2,\infty)$}

\exercise{$\ds h(k) = \sqrt{1-k}+\sqrt{k+1}$}{$[-1,1]$}

\exercise{$\ds f(t) = \sqrt{5t^2-30}$}{$(-\infty,-\sqrt{6}]\cup [\sqrt{6},\infty)$}

\exercise{$\ds g(t) = \frac{1}{\sqrt{1-t^2}}$}{$(-1,1)$}

\exercise{$\ds g(x) = \frac{1}{1+x^2}$}{$(-\infty,\infty)$}

\exercise{$\ds f(x) = e^x$}{$(-\infty,\infty)$}

\exercise{$\ds g(s) = \ln s$}{$(0,\infty)$}

\exercise{$\ds h(t) = \cos t$}{$(-\infty,\infty)$}

\exercise{$\ds f(k) = \sqrt{1-e^k}$}{$(-\infty,0]$}

\exercise{$\ds f(x) = \sin(e^x+x^2)$}{$(-\infty,\infty)$}

\exercise{$\ds f(x) = \begin{cases} 
\frac{x+1}{x+4} & x<2 \\
x^2-3 &2\leq x\leq 5 \\
6-2x & x>5
\end{cases}$}{$(-\infty,-4)\cup (-4,2)\cup (2,5)\cup (5,\infty)$}

\exercise{$\ds f(x) = \begin{cases} 
\frac{1}{x-1} & x<0 \\
2x^2-3x-1 &0\leq x\leq 2 \\
5x^2-4x & x>2
\end{cases}$}{$(-\infty, 2)\cup(2,\infty)$}

}


\exercise{Let $\ds f(x) = \begin{cases}
x^2-1&x < 3 \\
x+5&x\geq 3
\end{cases}$.\\
Is $f$ continuous everywhere?}{Yes. The only ``questionable'' place is at $x=3$, but the left and right limits agree.}

\exercise{Let $f$ be continuous on $[1,5]$ where $f(1) = -2$ and $f(5) = -10$. Does a value $1<c<5$ exist such that $f(c) = -9$? Why/why not?}{Yes, by the Intermediate Value Theorem.}

\exercise{Let $g$ be continuous on $[-3,7]$ where $g(0) = 0$ and $g(2) = 25$. Does a value $-3<c<7$ exist such that $g(c) = 15$? Why/why not?}{Yes, by the Intermediate Value Theorem. In fact, we can be more specific and state such a value $c$ exists in $(0,2)$, not just in $(-3,7)$.}

\exercise{Let $f$ be continuous on $[-1,1]$ where $f(-1) = -10$ and $f(1) = 10$. Does a value $-1<c<1$ exist such that $f(c) = 11$? Why/why not?}{We cannot say; the Intermediate Value Theorem only applies to function values between $-10$ and 10; as 11 is outside this range, we do not know.}

\exercise{Let $h$ be a function on $[-1,1]$ where $h(-1) = -10$ and $h(1) = 10$. Does a value $-1<c<1$ exist such that $h(c) = 0$? Why/why not?}{We cannot say; the Intermediate Value Theorem only applies to continuous functions. As we do know know if $h$ is continuous, we cannot say.}

\exerciseset{In Exercises}{, find the value(s) of $a$ and $b$ so that the function is continuous on $\mathbb{R}$.}{

\exercise{$\ds g(x)= \begin{cases} 
ax^2+3x & x<2 \\
x^3-ax & x\geq 2
\end{cases}$}{$a=\frac{1}{3}$}

\exercise{$\ds f(x)= \begin{cases} 
a^2x-a & x>3 \\
4& x\leq 3
\end{cases}$}{$a=-1$ and $\frac{4}{3}$}

\exercise{$\ds f(x)=\begin{cases}
ax-b & x<-1 \\
2x^2+3ax+b &-1\leq x<1 \\
4 & x\geq 1
\end{cases}$}{$a=\frac34$ and $b=-\frac14$}

\exercise{$\ds f(x)=\begin{cases}
x^2+2x & x\leq a \\
-1 & x>a\end{cases}$}{$a=-1$}

}


{\noindent In Exercises}
{, sketch the graph of a function that has the following properties.}
\exinput{exercises/01_05_ex_50}
\exinput{exercises/01_05_ex_51}
\exinput{exercises/01_05_ex_52}
\exinput{exercises/01_05_ex_53}


%\ifthenelse{\boolean{printquestions}}{\columnbreak}{}

{\noindent In Exercises}
{, show that the functions have at least one real root.}
\exinput{exercises/01_05_ex_37}
\exinput{exercises/01_05_ex_38}
\exinput{exercises/01_05_ex_39}
\exinput{exercises/01_05_ex_40}

\printreview

\exercise{Let $\ds f(x)= \begin{cases}
x^2-5 & x<5 \\
5x & x \geq 5
\end{cases}$.

\noindent\begin{minipage}[t]{.49\linewidth}
\begin{enumerate}
\item		$\ds \lim_{x\to 5^-} f(x)$
\item		$\ds \lim_{x\to 5^+} f(x)$
\end{enumerate}
\end{minipage}
\noindent\begin{minipage}[t]{.49\linewidth}
\begin{enumerate}\addtocounter{enumii}{2}
\item		$\ds \lim_{x\to 5} f(x)$
\item		$f(5)$\end{enumerate}
\end{minipage}		
}{\begin{enumerate}
\item		20
\item		25
\item		Limit does not exist
\item		25
\end{enumerate}
}

\exercise{Numerically approximate the following limits:
\begin{enumerate}
\item	$\ds \lim_{x\to -4/5^+} \frac{x^2-8.2 x-7.2}{x^2+5.8 x+4}$
\item	$\ds \lim_{x\to -4/5^-} \frac{x^2-8.2 x-7.2}{x^2+5.8 x+4}$
\end{enumerate}
}{\begin{tabular}{cc}
$x$ & $f(x)$ \\ \hline 
$-0.81 $& $-2.34129$ \\
$ -0.801$ & $-2.33413$ \\
$ -0.79 $& $-2.32542 $\\
$ -0.799$ & $-2.33254$
\end{tabular}

The top two lines give an approximation of the limit from the left: $-2.33$. The bottom two lines give an approximation from the right: $-2.33$ as well.}

\exercise{Give an example of function $f(x)$ for which $\ds \lim_{x\to 0} f(x)$ does not exist.}{Answers will vary.}
