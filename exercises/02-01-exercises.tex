\printconcepts

\exercise{T/F: Let $f$ be a position function. The average rate of change on $[a,b]$ is the slope of the line through the points $(a, f(a))$ and $(b,f(b))$.}{T}

% cut for parity
%\exercise{T/F: The definition of the derivative of a function at a point involves taking a limit.}{T}

\exercise{In your own words, explain the difference between the average rate of change and instantaneous rate of change.}{Answers will vary.}

\exercise{In your own words, explain the difference between Definitions \ref{def:derivative_at_a_point} and \ref{def:the_derivative}.}{Answers will vary.}

\exercise{Let $y=f(x)$. Give three different notations equivalent to ``$\fp(x)$.''}{Answers will vary.}

% in Hartman, but we omit normal lines
%\exercise{If two lines are perpendicular, what is true of their slopes?}{Their product is $-1$.}

\printproblems

\exercisesetinstructions{,\begin{enumerate}
\item use the definition of the derivative to compute the derivative of the given function.
\item Find the tangent line to the graph of the given function at $x=c$.
\end{enumerate}}

\exercise{$\ds f(x) = 6$ at $x=-2$}{(a) $\fp(x) = 0$, (b) $y=6$}

\exercise{$\ds f(x) = 2x$ at $x=3$}{(a) $\fp(x) = 2$, (b) $y=2x$}

\exercise{$\ds f(x) = 4-3x$ at $x=7$}{(a) $\fp(x) = -3$, (b) $y=4-3x$}

\exercise{$\ds g(x) = x^2$ at $x=-2$}{(a) $g'(x) = 2x$, (b) $y=-4x-4$}

\exercise{$h(x)=2x-x^2$ at $x=1$}{(a) $h'(x)=2-2x$ (b) $y=1$}

\exercise{$\ds f(x) = 3x^2-x+4$ at $x=-1$}{(a) $\fp'(x) = 6x-1$, (b) $y=-7x+1$}

\exercise{$g(x)=\sqrt{x+3}$ at $x=1$}{(a) $g'(x)=\frac1{2\sqrt{x+3}}$, (b) $y%=2+\frac14(x-1)
=\frac x4+\frac74$}

\exercise{$\ds r(x) = \frac1x$ at $x=-2$}{(a) $r\primeskip'(x) = \frac{-1}{x^2}$, (b) $y=-\frac x4-1$}

\exercise{$h(x)=\dfrac3{\sqrt{x}}$ at $x=4$}{(a) $h'(x)=-\dfrac3{2x\sqrt{x}}$, (b) $y%=\frac32-\frac3{16}(x-4)
=-\dfrac{3x}{16}+\dfrac94$}

\exercise{$f(x)=\dfrac1{x-2}$ at $x=3$}{(a) $\fp(x)=\dfrac{-1}{(s-2)^2}$, (b) $y=-x+4$}

\exercisesetend


\begin{exerciseset}{In Exercises}{, each limit represents the derivative of some function, $f$, at some number $c$. State an appropriate $f$ and $c$ for each.}

\exercise{$\ds\lim_{h\to 0}\frac{\sqrt {16+h} - 4}{h}$}{$f(x)=\sqrt x$, $c=16$.}

\exercise{$\ds\lim_{h\to 0}\frac{(3+h)^4-81}{h}$}{$f(x)=x^4$, $c=3$}

\exercise{$\ds\lim_{h\to 0}\frac{\frac{1}{2+h}-\frac12}{h}$}{$f(x)=\dfrac1x$, $c=2$}

\exercise{$\ds\lim_{h\to 0}\frac{\cos (-\pi +h)+1}{h}$}{$f(x)=\cos x$, $c=-\pi$.}

\end{exerciseset}


\input{exercises/02-01-exset-01}

\exercise{The graph of $f(x)=x^2-1$ is shown. 
	\begin{enumext}[start=1]
	\item		Use the graph to approximate the slope of the tangent line to $f$ at the following points: $(-1,0)$, $(0,-1)$ and $(2,3)$. 
	\item		Using the definition, find $\fp(x)$.
	\item		Find the slope of the tangent line at the points $(-1,0)$, $(0,-1)$ and $(2,3)$.
	\end{enumext}
\begin{tikzpicture}[alt={A U shaped graph through (-1,0), (0,-1), and (1,0).}]
\begin{axis}[width=\marginparwidth,tick label style={font=\scriptsize},
axis y line=middle,axis x line=middle,name=myplot,
ytick={-1,1,2,3},ymin=-1.3,ymax=3.5,xmin=-2.1,xmax=2.1,grid=major]
\addplot [thick,draw={\colorone},smooth,domain=-2.1:2.1] {x^2-1};
\end{axis}
\node [right] at (myplot.right of origin) {\scriptsize $x$};
\node [above] at (myplot.above origin) {\scriptsize $y$};
\end{tikzpicture}}{\mbox{}\\[-2\baselineskip]\parbox[t]{\linewidth}{\begin{enumext}[start=1]
\item	Approximations will vary; they should match (c) closely.
\item	$\fp(x) = 2x$
\item	At $(-1,0)$, slope is $-2$. At $(0,-1)$, slope is 0. At $(2,3)$, slope is 4.
\end{enumext}}}

\exercise{The graph of $\ds f(x)=\frac{1}{x+1}$ is shown. 
	\begin{enumext}[start=1]
	\item		Use the graph to approximate the slope of the tangent line to $f$ at the following points: $(0,1)$ and $(1,0.5)$. 
	\item		Using the definition, find $\fp(x)$.
	\item		Find the slope of the tangent line at the points $(0,1)$ and $(1,0.5)$.
	\end{enumext}
\begin{tikzpicture}[alt={A curve starting up near (-1,5) and then moving downward toward (3,0).}]
\begin{axis}[width=\marginparwidth,tick label style={font=\scriptsize},
axis y line=middle,axis x line=middle,name=myplot,ytick={1,2,3,4,5},
ymin=-.5,ymax=5.5,xmin=-1.1,xmax=3.1,grid=major]
\addplot [thick,draw={\colorone},smooth,domain=-.9:3.1,samples=50] {1/(x+1)};
\end{axis}
\node [right] at (myplot.right of origin) {\scriptsize $x$};
\node [above] at (myplot.above origin) {\scriptsize $y$};
\end{tikzpicture}}{\mbox{}\\[-2\baselineskip]\parbox[t]{\linewidth}{\begin{enumext}[start=1]
\item	Approximations will vary; they should match (c) closely.
\item		$\fp(x) = -1/(x+1)^2$
\item		At $(0,1)$, slope is $-1$. At $(1,0.5)$, slope is $-1/4$.
\end{enumext}}}

\exercisesetinstructions{, a graph of a function $f(x)$ is given. Using the graph, sketch $\fp(x)$.}

% todo Tim alt text for the answer images.

\exercise{\noindent\begin{minipage}{\linewidth}
\begin{tikzpicture}[alt={A line segment from (-2,3) to (4,0).}]
\begin{axis}[width=\marginparwidth,tick label style={font=\scriptsize},
axis y line=middle,axis x line=middle,name=myplot,
xtick={-2,-1,1,2,3,4},ytick={-1,1,2,3},ymin=-1.5,ymax=3.5,xmin=-2.1,xmax=5.1]
%
\addplot [thick,draw={\colorone},domain=-2.1:5.1] {-.5*x+2};
%
\end{axis}
\node [right] at (myplot.right of origin) {\scriptsize $x$};
\node [above] at (myplot.above origin) {\scriptsize $y$};
\end{tikzpicture}
\end{minipage}
}{\mbox{}\\[-\baselineskip]\begin{tikzpicture}[alt={The horizontal line y=-0.5.}]
\begin{axis}[width=\marginparwidth,tick label style={font=\scriptsize},
axis y line=middle,axis x line=middle,name=myplot,xtick={-2,-1,1,2,3,4},
ytick={-1,1,2,3},ymin=-1.5,ymax=3.5,xmin=-2.1,xmax=5.1]
%
\addplot [thick,draw={\colorone},domain=-2.1:5.1] {-.5};
%
\end{axis}
\node [right] at (myplot.right of origin) {\scriptsize $x$};
\node [above] at (myplot.above origin) {\scriptsize $y$};
\end{tikzpicture}
}

\exercise{\noindent\begin{minipage}{\linewidth}
\begin{tikzpicture}[alt={U shaped graph going through (-4,-1), (-2,-3), and (0,-1).}]
\begin{axis}[width=\marginparwidth,tick label style={font=\scriptsize},
axis y line=middle,axis x line=middle,name=myplot,
ymin=-3.5,ymax=3.5,xmin=-7.1,xmax=3.1]
%
\addplot [thick,draw={\colorone},domain=-7.1:3.1,samples=40] {.5*(x+2)^2-3};
%
\end{axis}
\node [right] at (myplot.right of origin) {\scriptsize $x$};
\node [above] at (myplot.above origin) {\scriptsize $y$};
\end{tikzpicture}
\end{minipage}
}{\mbox{}\\[-\baselineskip]\begin{tikzpicture}[alt={The line y=x+2.}]
\begin{axis}[width=\marginparwidth,tick label style={font=\scriptsize},
axis y line=middle,axis x line=middle,name=myplot,
ymin=-3.5,ymax=3.5,xmin=-7.1,xmax=3.1]
%
\addplot [thick,draw={\colorone},domain=-7.1:3.1] {(x+2)};
%
\end{axis}
\node [right] at (myplot.right of origin) {\scriptsize $x$};
\node [above] at (myplot.above origin) {\scriptsize $y$};
\end{tikzpicture}
}

\exercise{\noindent\begin{minipage}{\linewidth}
\begin{tikzpicture}[alt={A curve starting near (-3,-5), going through (-2,0), turning downward near (-1,3), going through the origin, turning upward near (1,-3), going through (2,0), and then finishing near (3,5).}]
\begin{axis}[width=\marginparwidth,tick label style={font=\scriptsize},
axis y line=middle,axis x line=middle,name=myplot,xtick={-2,-1,1,2},
ymin=-5.5,ymax=5.5,xmin=-3.1,xmax=3.1]
%
\addplot [thick,draw={\colorone},domain=-3.1:3.1,samples=40] {(x-2)*x*(x+2)};
%
\end{axis}
\node [right] at (myplot.right of origin) {\scriptsize $x$};
\node [above] at (myplot.above origin) {\scriptsize $y$};
\end{tikzpicture}
\end{minipage}
}{\mbox{}\\[-\baselineskip]\begin{tikzpicture}[alt={The parabola starting near (-2,5), bottoming out at (0,-4), and finishing in a mirror image near (2,5).}]
\begin{axis}[width=\marginparwidth,tick label style={font=\scriptsize},
axis y line=middle,axis x line=middle,name=myplot,xtick={-2,-1,1,2},
ymin=-5.5,ymax=5.5,xmin=-3.1,xmax=3.1]
%
\addplot [thick,draw={\colorone},domain=-3.1:3.1,samples=40] {3*x^2-4};
%
\end{axis}
\node [right] at (myplot.right of origin) {\scriptsize $x$};
\node [above] at (myplot.above origin) {\scriptsize $y$};
\end{tikzpicture}
}

\exercise{\noindent\begin{minipage}{\linewidth}
\begin{tikzpicture}[alt={A wave of constant amplitude 1 that has its peaks at multiples of 2π and its troughs midway between.}]
\begin{axis}[width=\marginparwidth,tick label style={font=\scriptsize},
axis y line=middle,axis x line=middle,name=myplot,xtick=\empty,
extra x ticks={-6.28,-3.14,3.14,6.28},%
extra x tick labels={$-2\pi$, $-\pi$, $\pi$, $2\pi$},
ymin=-1.1,ymax=1.1,xmin=-6.9,xmax=6.9]
%
\addplot [thick,draw={\colorone},smooth,domain=-6.9:6.9,samples=50] {cos(deg(x))};
%
\end{axis}
\node [right] at (myplot.right of origin) {\scriptsize $x$};
\node [above] at (myplot.above origin) {\scriptsize $y$};
\end{tikzpicture}
\end{minipage}
}{\mbox{}\\[-\baselineskip]\begin{tikzpicture}[alt={A wave of constant amplitude 1 that has its peaks at (2k-1)π/2 and its troughs at (2k+1)π/2.}]
\begin{axis}[width=\marginparwidth,tick label style={font=\scriptsize},
axis y line=middle,axis x line=middle,name=myplot,xtick=\empty,
extra x ticks={-6.28,-3.14,3.14,6.28},
extra x tick labels={$-2\pi$, $-\pi$, $\pi$, $2\pi$},
ymin=-1.1,ymax=1.1,xmin=-6.9,xmax=6.9]
%
\addplot [thick,draw={\colorone},smooth,domain=-6.9:6.9,samples=50] {-sin(deg(x))};
%
\end{axis}
\node [right] at (myplot.right of origin) {\scriptsize $x$};
\node [above] at (myplot.above origin) {\scriptsize $y$};
\end{tikzpicture}
}

\exercise{\begin{minipage}[]{\linewidth}
\begin{tikzpicture}[alt={A line segment from (-3,-3) up to (-1,1) with a sharp corner from there to finish at (3,-1).},baseline=10pt,scale=.8]
\begin{axis}[width=1.16\marginparwidth,tick label style={font=\scriptsize},
axis y line=middle,axis x line=middle,name=myplot,
xtick={-2,-1,1,2,-3,3},ytick={-2,2,-4,4,-6,6},
ymin=-6.5,ymax=6.5,xmin=-3.5,xmax=3.5]
%
\addplot [thick,draw={\colorone},domain=-3.5:-1] {3*x+6};
\addplot [thick,draw={\colorone},domain=-1:3.5] {-x+2};
%
\end{axis}
\node [right] at (myplot.right of origin) {\scriptsize $x$};
\node [above] at (myplot.above origin) {\scriptsize $y$};
\end{tikzpicture}
\end{minipage}
}{\mbox{}\\[-\baselineskip]\begin{tikzpicture}[alt={A horizontal line y=3 from the left ending at a hollow dot at (-1,3).  A hollow dot at (-1,-1) beginning a horizontal line y=-1.},baseline=10pt,scale=.8]
\begin{axis}[width=1.16\marginparwidth,tick label style={font=\scriptsize},
axis y line=middle,axis x line=middle,name=myplot,
xtick={-2,-1,1,2,-3,3},ytick={-2,2,-4,4,-6,6},
ymin=-6.5,ymax=6.5,xmin=-3.5,xmax=3.5]
%
\addplot [thick,draw={\colorone},domain=-3.5:-1] {3};
\addplot [thick,draw={\colorone},domain=-1:3.5] {-1};
\filldraw [fill=white] (axis cs:-1,3) circle (1.5pt);
\filldraw [fill=white] (axis cs:-1,-1) circle (1.5pt);
%
\end{axis}
\node [right] at (myplot.right of origin) {\scriptsize $x$};
\node [above] at (myplot.above origin) {\scriptsize $y$};
\end{tikzpicture}
}

% cut for parity (and it matches three problems earlier
%\exercise{\begin{minipage}[]{\linewidth}
%\begin{tikzpicture}[alt={A curve starting near (-2,-5), going through (-1.2,0) and (0,5), turning downward near (0.2,5), passing through (2.5,0), turning upward near (3.8,-2), going through (4.6,0), and then finishing near (5,2).},baseline=10pt,scale=.8]
%\begin{axis}[width=1.16\marginparwidth,tick label style={font=\scriptsize},
%axis y line=middle,axis x line=middle,name=myplot,
%xtick={-2,-1,1,2,4,3,5},ytick={-2,2,-4,4,-6,6},
%ymin=-6.5,ymax=6.5,xmin=-2.5,xmax=5.5]
%%
%\addplot [thick,draw={\colorone},smooth] {x^3/3-2*x^2+x+5};
%%
%\end{axis}
%\node [right] at (myplot.right of origin) {\scriptsize $x$};
%\node [above] at (myplot.above origin) {\scriptsize $y$};
%\end{tikzpicture}
%\end{minipage}
%}{\mbox{}\\[-\baselineskip]\begin{tikzpicture}[alt={A parabola starting at (-1,6), passing through (0,1) and (0.3,0), bottoming out at (2,-3), passing through (3.7,0) to finish at (5,6).},baseline=10pt,scale=.8]
%\begin{axis}[width=1.16\marginparwidth,tick label style={font=\scriptsize},
%axis y line=middle,axis x line=middle,name=myplot,
%xtick={-2,-1,1,2,3,4,5},ytick={-2,2,-4,4,-6,6},
%ymin=-6.5,ymax=6.5,xmin=-2.5,xmax=5.5]
%%
%\addplot [thick,draw={\colorone},smooth] {x^2-4*x+1};
%\end{axis}
%\node [right] at (myplot.right of origin) {\scriptsize $x$};
%\node [above] at (myplot.above origin) {\scriptsize $y$};
%\end{tikzpicture}
%}

\exercise{\begin{minipage}[]{\linewidth}
\begin{tikzpicture}[alt={A curve starting horizontally near (-5,-4), spiking upward toward (-1,10).  A separate curve starts near (-1,-10) before turning to finish horizontally near (5,3).},baseline=10pt,scale=.8]
\begin{axis}[width=1.16\marginparwidth,tick label style={font=\scriptsize},
axis y line=middle,axis x line=middle,name=myplot,
ymin=-10.9,ymax=10.9,xmin=-5.5,xmax=5.5]
%
\addplot [thick,draw={\colorone},smooth,domain=-5.5:-1.08,samples=40] {-1/(x+1)-4};
\addplot [thick,draw={\colorone},smooth,domain=-.92:5.5,samples=40] {-1/(x+1)+3};
\draw [dashed,draw={\colorone}] (axis cs:-1,-10.5) -- (axis cs:-1,10.5);
\draw [dashed,draw={\colorone}] (axis cs:-5.5,-4) -- (axis cs:5.5,-4);
\draw [dashed,draw={\colorone}] (axis cs:-5.5,3) -- (axis cs:5.5,3);
%
\end{axis}
\node [right] at (myplot.right of origin) {\scriptsize $x$};
\node [above] at (myplot.above origin) {\scriptsize $y$};
\end{tikzpicture}
\end{minipage}
}{\mbox{}\\[-\baselineskip]\begin{tikzpicture}[alt={A curve starting near (-5,0), spiking upward toward (-1,10).  A separate curve starts near (-1,10) moving downward before turning to finish near (5,0).},baseline=10pt,scale=.8]
\begin{axis}[width=1.16\marginparwidth,tick label style={font=\scriptsize},
axis y line=middle,axis x line=middle,name=myplot,
ymin=-1.1,ymax=10.9,xmin=-5.5,xmax=5.5]
%
\addplot [thick,draw={\colorone},smooth,domain=-5.5:-1.08,samples=40] {1/((x+1)^2)};
\addplot [thick,draw={\colorone},smooth,domain=-.9:5.5,samples=60] {1/((x+1)^2)};
\draw [draw={\colorone}, dashed] (axis cs:-1,-1) -- (axis cs:-1,10.5);
%
\end{axis}
\node [right] at (myplot.right of origin) {\scriptsize $x$};
\node [above] at (myplot.above origin) {\scriptsize $y$};
\end{tikzpicture}
}

\exercisesetend


\exercisesetinstructions{, a graph of $g(x)$ is given.  Using the graph, answer the following questions.
\begin{multicols}{2}
	\begin{enumerate}%[label=(\alph*)]
	\renewcommand{\theenumi}{(\alph{enumi})}
		\item	Where is $g(x) > 0$?
		\item	Where is $g(x) < 0$?
		\item	Where is $g(x) = 0$?
		\item	Where is $g'(x) > 0$?
		\item	Where is $g'(x) < 0$?
		\item	Where is $g'(x) = 0$?
\end{enumerate}
\end{multicols}}

\exercise{\mbox{}\\[-2\baselineskip]\begin{tikzpicture}[alt={A curve starting near (-3,-5), passing through (-2,0), going up to (-1,1), turning downward to the origin and down to (1,-1), turning back upward to (2,0) and finishing near (3,5).}]
\begin{axis}[width=\marginparwidth,tick label style={font=\scriptsize},
axis y line=middle,axis x line=middle,name=myplot,xtick={-2,-1,1,2},
ymin=-5.5,ymax=5.5,xmin=-3.1,xmax=3.1]
\addplot [thick,draw={\colorone},domain=-3.1:3.1,samples=40] {(x-2)*x*(x+2)};
\end{axis}
\node [right] at (myplot.right of origin) {\scriptsize $x$};
\node [above] at (myplot.above origin) {\scriptsize $y$};
\end{tikzpicture}}{\mbox{}\\[-2\baselineskip]\parbox[t]{\linewidth}{\begin{enumext}[start=1]
\item		Approximately on $(-2,0)$ and $(2,\infty)$.
\item		Approximately on $(-\infty,-2)$ and $(0,2)$.
\item		Approximately at $x=0,\ \pm 2$.
\item		Approximately on $(-\infty,-1)$ and $(1,\infty)$.
\item		Approximately on $(-1,1)$.
\item		Approximately at $x=\pm 1$.
\end{enumext}}}

\exercise{\mbox{}\\[-2\baselineskip]\begin{tikzpicture}[alt={A curve starting near (-2,-6), going through (-1.5,0), up to (-1,4), turning at that point downward to (0,1), turning at that point up to (1,4), turning back down to (1.5,0), and finishing near (2,-6).},scale=.8]
\begin{axis}[width=\marginparwidth,tick label style={font=\scriptsize},
axis y line=middle,axis x line=middle,name=myplot,xtick={-2,-1,1,2},
ymin=-7.9,ymax=5.9,xmin=-2.5,xmax=2.5]
\addplot [thick,draw={\colorone},smooth,domain=-2.3:2.3,samples=50] {(-10)*(x^4/4-x^2/2)+1};
\end{axis}
\node [right] at (myplot.right of origin) {\scriptsize $x$};
\node [above] at (myplot.above origin) {\scriptsize $y$};
\end{tikzpicture}}{\mbox{}\\[-2\baselineskip]\parbox[t]{\linewidth}{\begin{enumext}[start=1]
	\item	Approximately on $(-1.5,1.5)$.
	\item	Approximately on $(-\infty,-1.5) \cup (1.5,\infty)$.
	\item	Approximately at $x=\pm 1.5$.
	\item	On $(-\infty,-1) \cup (0,1)$.
	\item	On $(-1,0) \cup (1,\infty)$.
	\item	At $x=\pm 1$ and $x=0$.
\end{enumext}}}

\exercisesetend

\exercise{Suppose\label{diffimpliescont} that $f(x)$ is defined on an open interval containing the number $c$. Suppose that the limit
\[\lim_{h\to0}\frac{f(c+h)-f(c)}h\]
exists. Show that $f(x)$ is continuous at $x=c$. This shows that we could drop the assumption that $f(x)$ is continuous at $x=c$ in the definition of $\fp(c)$.}{}

% these problems should be much later
%\exerciseset{In Exercises}{, a function $f(x)$ is given, along with its domain and derivative. Determine if $f(x)$ is differentiable on its domain.}{
%
%\exercise{$\ds f(x) = \sqrt{x^5(1-x)}$, domain = $[0,1]$, $\ds\fp(x) = \frac{(5-6x)x^{3/2}}{2\sqrt{1-x}}$}{$\lim_{h\to0^+}\frac{f(0+h)-f(0)}h=0$; note also that $\lim_{x\to0^+}\fp(x)=0$. So $f$ is differentiable at $x=0$.\\
%$\lim_{h\to0^-}\frac{f(1+h)-f(1)}h=-\infty$; note also that $\lim_{x\to1^-}\fp(x) = -\infty$. So $f$ is not differentiable at $x=1$.\\
%$f$ is differentiable on $[0,1)$, not its entire domain.}
%
%\exercise{$\ds f(x) = \cos\big(\sqrt{x}\big)$,\ domain = $[0,\infty)$,\ $\ds\fp(x) = -\frac{\sin\big(\sqrt{x}\big)}{2\sqrt{x}}$}{The limit of the difference quotient is difficult to evaluate. Using \autoref{thm:special_limits}, we can determine $\lim_{x\to0^+}\fp(x) = -1/2$.\\
%Since $\fp$ is defined on $(0,\infty)$, we conclude $f$ is differentiable on $[0,\infty)$.}
%
%}

\printreview

\exercise{Approximate $\ds \lim_{x\to 5}\frac{x^2+2 x-35}{x^2-10.5 x+27.5}$.}{Approximately $-24$.}

\exercise{Use the Bisection Method to approximate, accurate to two decimal places, the root of $g(x) = x^3+x^2+x-1$ on $[0.5,0.6]$.}{Approximately $0.54$.}

\exercise{Give intervals on which each of the following functions are continuous.
\begin{multicols}{2}
\begin{enumext}[start=1]
\item		$\ds \frac{1}{e^x+1}$
\item		$\ds \frac{1}{x^2-1}$
\item		$\ds \sqrt{5-x}\phantom{\frac{1}{e^x+1}}$
\item		$\ds \sqrt{5-x^2}\phantom{\frac{1}{x^2-1}}$
\end{enumext}
\end{multicols}}{\mbox{}\\[-2\baselineskip]\parbox[t]{\linewidth}{\begin{enumext}[start=1]
\item		$(-\infty,\infty)$
\item		$(-\infty,-1)\cup (-1,1) \cup (1,\infty)$
\item		$(-\infty,5]$
\item		$[-\sqrt5,\sqrt5]$
\end{enumext}}}

\exercise{Use the graph of $f(x)$ provided to answer the following.
\begin{multicols}{2}
\begin{enumext}[start=1]
\item		$\ds \lim_{x\to-3^-} f(x) = $?
\item		$\ds \lim_{x\to-3^+} f(x) = $?
\item		$\ds \lim_{x\to-3} f(x) = $?
\item		Where is $f$ continuous?
\end{enumext}
\end{multicols}
\begin{tikzpicture}[alt={A curve starting near (-5,3) and finishing at a hollow dot at (-3,1), then a solid dot at (-3,2), and then another curve starting from (-3,3) and finishing near (-1,-1).}]
\begin{axis}[width=\marginparwidth,tick label style={font=\scriptsize},
axis y line=middle,axis x line=middle,name=myplot,
ymin=-1.1,ymax=3.1,xmin=-5.1,xmax=0.5]
\addplot [thick,draw={\colorone},smooth,domain=-5:-3] {(x+3)^2+1};
\addplot [thick,draw={\colorone},smooth,domain=-3:0.5] {-(x+3)^2+3};
\filldraw [fill=white] (axis cs:-3,1) circle (1.5pt);
\filldraw [fill=white] (axis cs:-3,3) circle (1.5pt);
\filldraw [fill=black] (axis cs:-3,2) circle (1.5pt);
\end{axis}
\node [right] at (myplot.right of origin) {\scriptsize $x$};
\node [above] at (myplot.above origin) {\scriptsize $y$};
\end{tikzpicture}}{\mbox{}\\[-2\baselineskip]\parbox[t]{\linewidth}{\begin{enumext}[start=1]
\item		1
\item		3
\item		Does not exist
\item		$(-\infty,-3)\cup (3,\infty)$
\end{enumext}}}
