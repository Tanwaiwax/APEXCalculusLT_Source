\exercisesetinstructions{, graphs of functions $f(x)$ and $g(x)$ are given. Identify which function is the derivative of the other.}

\exercise{\begin{minipage}[t]{\linewidth}
\begin{tikzpicture}[alt={The curve f is a line segment from (-3,-2) to (3,1).  The curve g is U shaped, with the bottom of the U near (1,2).},baseline={([yshift={-\baselineskip}]current bounding box.north)},scale=.8]
\begin{axis}[width=1.16\marginparwidth,tick label style={font=\scriptsize},
axis y line=middle,axis x line=middle,name=myplot,
ymin=-5.1,ymax=5.1,xmin=-4.5,xmax=4.3]
\addplot [thick,draw={\colortwo},smooth,domain=-4.5:4.5] {.25*(x-1)^2+2};
\addplot [thick,draw={\colorone},smooth,domain=-4.5:4.5] {.5*(x-1)};
\draw (axis cs:3.9,2) node {\scriptsize $f(x)$};
\draw (axis cs:3,4) node[] {\scriptsize $g(x)$};
\end{axis}
\node [right] at (myplot.right of origin) {\scriptsize $x$};
\node [above] at (myplot.above origin) {\scriptsize $y$};
\end{tikzpicture}
\end{minipage}}{$f(x) = g'(x)$}

\exercise{\begin{minipage}[t]{\linewidth}
\begin{tikzpicture}[alt={The curve f starts at (-1,-6), goes through the origin, peaks near (.5,.5), goes through (1,0), bottoms out near (1.5,-.5), goes through (2,0), and finishes near (3,6).  The curve g is U shaped, going through (.5,0), bottoming out near (1,-1), and then through (1.5,0).},baseline={([yshift={-\baselineskip}]current bounding box.north)},scale=.8]
\begin{axis}[width=1.16\marginparwidth,tick label style={font=\scriptsize},
axis y line=middle,axis x line=middle,name=myplot,
ymin=-5.1,ymax=5.1,xmin=-1.5,xmax=3.5]
\addplot [thick,draw={\colorone},smooth,domain=-1.5:3.5] {x*(x-1)*(x-2)};
\addplot [thick,draw={\colortwo},smooth,domain=-1.5:3.5] {3*x^2-6*x+2};
\draw (axis cs:3,3) node {\scriptsize $f(x)$};
\draw (axis cs:1.85,3) node[] {\scriptsize $g(x)$};
\end{axis}
\node [right] at (myplot.right of origin) {\scriptsize $x$};
\node [above] at (myplot.above origin) {\scriptsize $y$};
\end{tikzpicture}
\end{minipage}}{$g(x) = \fp(x)$}

\exercise{\begin{minipage}[t]{\linewidth}
 \begin{tikzpicture}[alt={The curve f starts near (-5,2), turns toward negative y values as x approaches 0 from the left, restarts at positive y values as x approaches 0 from the right, and turns to finish near (5,2).  The curve g starts near (-5,0), turns toward negative y values as x approaches 0 from the left, restarts at negative y values as x approaches 0 from the right, and turns to finish near (5,0).},baseline={([yshift={-\baselineskip}]current bounding box.north)}]
  \begin{axis}[width=1.16\marginparwidth,tick label style={font=\scriptsize},
    axis y line=middle,axis x line=middle,name=myplot,
    ymin=-5.5,ymax=5.5,xmin=-5.5,xmax=5.5]
   \addplot [thick,draw={\colorone},smooth,domain=-5.5:-.1,samples=40] {2+1/x};
   \addplot [thick,draw={\colorone},smooth,domain=.1:5.5,samples=40] {2+1/x}
    node[above,pos=.8,color=black]{\scriptsize$f(x)$};
   \addplot [thick,draw={\colortwo},smooth,domain=-5.5:-.1,samples=40] {-1/(x*x)};
   \addplot [thick,draw={\colortwo},smooth,domain=.4:5.5,samples=40] {-1/(x*x)}
    node[below,pos=.7,color=black]{\scriptsize$g(x)$};
  \end{axis}
  \node [right] at (myplot.right of origin) {\scriptsize $x$};
  \node [above] at (myplot.above origin) {\scriptsize $y$};
 \end{tikzpicture}
\end{minipage}}{$g(x) = \fp(x)$}

\exercise{\begin{minipage}[t]{\linewidth}
\begin{tikzpicture}[alt={The curve f is a wave that is zero at even integers, has its peaks at integers one greater than multiples of 4, and its troughs at integers one less than multiples of 4.  The curve g is a wave that is zero at odd integers, has its peaks at multiples of 4, and its troughs at even integers that are not multiples of 4.},baseline={([yshift={-\baselineskip}]current bounding box.north)},scale=.8]
\begin{axis}[width=1.16\marginparwidth,tick label style={font=\scriptsize},
axis y line=middle,axis x line=middle,name=myplot, axis equal image,
ymin=-2.5,ymax=2.5,xmin=-4.5,xmax=4.5]
\addplot [thick,draw={\colorone},smooth,domain=-4.5:4.5,samples=40] {sin(deg(3.14*x/2))};
\addplot [thick,draw={\colortwo},smooth,domain=-4.5:4.5,samples=40] {cos(deg(3.14*x/2))};
\draw (axis cs:1,1) node[above] {\scriptsize $f(x)$};
\draw (axis cs:2,-1) node[below] {\scriptsize $g(x)$};
\end{axis}
\node [right] at (myplot.right of origin) {\scriptsize $x$};
\node [above] at (myplot.above origin) {\scriptsize $y$};
\end{tikzpicture}
\end{minipage}}{$g(x) = \fp(x)$}

\exercisesetend
