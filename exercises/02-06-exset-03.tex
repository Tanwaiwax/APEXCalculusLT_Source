\begin{exerciseset}{In Exercises}{, find the equation of the tangent line to  the graph of the  implicitly defined function at the indicated points. As a visual aid, each function is graphed.}

\exercise{$x^{2/5}+y^{2/5}=1$
\begin{enumerate}
\item	At $(1,0)$.
\item	At $(0.1,0.281)$ (which does not \emph{exactly} lie on the curve, but is very close).
\end{enumerate}
\pdftooltip{\begin{tikzpicture}[baseline=10pt]
\begin{axis}[width=\marginparwidth,tick label style={font=\scriptsize},axis y line=middle,axis x line=middle,name=myplot,axis equal,
ymin=-1.1,ymax=1.1,xmin=-1.1,xmax=1.1]
\addplot [thick,draw={\colorone},smooth,domain=0:360,samples=40] ({(cos(x))^5},{(sin(x))^5});
\filldraw [] (axis cs:.1,.28) node [above right] {\scriptsize $(0.1,0.281)$} circle (1pt);
\end{axis}
\node [right] at (myplot.right of origin) {\scriptsize $x$};
\node [above] at (myplot.above origin) {\scriptsize $y$};
\end{tikzpicture}}{A diamond shape with the sides pushed far in, so that the corners are pointed.}}{\mbox{}\\[-2\baselineskip]\parbox[t]{\linewidth}{\begin{enumerate}
\item	$y=0$
\item	$y = -1.859(x-0.1)+0.281$
\end{enumerate}}}

\exercise{$x^4+y^4=1$
\begin{enumerate}
\item	At $(1,0)$.
\item	At $(\sqrt{0.6},\sqrt{0.8})$.
\item	At $(0,1)$.
\end{enumerate}
\pdftooltip{\begin{tikzpicture}[baseline=10pt]
\begin{axis}[width=\marginparwidth,tick label style={font=\scriptsize},
axis y line=middle,axis x line=middle,name=myplot,axis equal,
ymin=-1.1,ymax=1.1,xmin=-1.1,xmax=1.1]
\addplot[thick,draw={\colorone},smooth,domain=0:90]({sqrt(cos(x))},{sqrt(sin(x))});
\addplot[thick,draw={\colorone},smooth,domain=0:90]({-sqrt(cos(x))},{sqrt(sin(x))});
\addplot[thick,draw={\colorone},smooth,domain=0:90]({-sqrt(cos(x))},{-sqrt(sin(x))});
\addplot[thick,draw={\colorone},smooth,domain=0:90]({sqrt(cos(x))},{-sqrt(sin(x))});
%\addplot [thick,draw={\colorone},smooth] coordinates {(1.,0) (0.98901,0.45597) (0.9558,0.63776) (0.89945,0.76667) (0.818,0.86206) (0.70711,0.9306) (0.55589,0.97522) (0.32331,0.99726) (-0.32331,0.99726) (-0.55589,0.97522) (-0.70711,0.9306) (-0.818,0.86206) (-0.89945,0.76667) (-0.9558,0.63776) (-0.98901,0.45597) (-1.,0) (-1.,0) (-0.98901,-0.45597) (-0.9558,-0.63776) (-0.89945,-0.76667) (-0.818,-0.86206)(-0.70711,-0.9306) (-0.55589,-0.97522) (-0.32331,-0.99726)(0.32331,-0.99726) (0.55589,-0.97522) (0.70711,-0.9306)(0.818,-0.86206) (0.89945,-0.76667) (0.9558,-0.63776)(0.98901,-0.45597) (1.,0) };
\filldraw [] (axis cs:.775,.894) node [below left] {\scriptsize $(\sqrt{0.6},\sqrt{0.8})$} circle (1pt);
\end{axis}
\node [right] at (myplot.right of origin) {\scriptsize $x$};
\node [above] at (myplot.above origin) {\scriptsize $y$};
\end{tikzpicture}}{A squarish shape, with corners that have been rounded off.}}{\mbox{}\\[-2\baselineskip]\parbox[t]{\linewidth}{\begin{enumerate}
\item	$x=1$
\item	$y = -\frac{3\sqrt{3}}{8}(x-\sqrt{.6})+\sqrt{.8} \approx -0.65(x-0.775)+0.894$
\item	$y=1$
\end{enumerate}}}

\exercise{$(x^2+y^2-4)^3 = 108y^2$
\begin{enumerate}
\item	At $(0,4)$.
\item	At $(2,-\sqrt[4]{108})$.
\end{enumerate}
\pdftooltip{\begin{tikzpicture}[baseline=10pt]
\begin{axis}[width=\marginparwidth,tick label style={font=\scriptsize},
axis y line=middle,axis x line=middle,name=myplot,axis equal,
ymin=-4.5,ymax=4.5,xmin=-4.5,xmax=4.5]
\addplot[thick,draw={\colorone},smooth,domain=-4:4,samples=60]({sqrt((108*x*x)^(1/3)+4-x*x)},x);
\addplot[thick,draw={\colorone},smooth,domain=-4:4,samples=60]({-sqrt((108*x*x)^(1/3)+4-x*x)},x);
\addplot[thick,draw={\colorone},smooth]coordinates{(-2.0396,3.1825) (-1.6513,3.5) (-1.2857,3.7088) (-1.0714,3.8007)(-0.78571,3.8938) (-0.52476,3.9533) (-0.28571,3.9855)(-0.070563,3.9991)(0,4)(0.27289,3.9872) (0.64411,3.9298) (1.082,3.7963)(1.4286,3.6354) (1.7143,3.4554) (1.9675,3.2532) (2.1511,3.0714)};
\addplot[thick,draw={\colorone},smooth]coordinates{(-2.0396,-3.1825) (-1.6513,-3.5) (-1.2857,-3.7088) (-1.0714,-3.8007)(-0.78571,-3.8938) (-0.52476,-3.9533) (-0.28571,-3.9855)(-0.070563,-3.9991)(0,-4)(0.27289,-3.9872) (0.64411,-3.9298) (1.082,-3.7963)(1.4286,-3.6354) (1.7143,-3.4554) (1.9675,-3.2532) (2.1511,-3.0714)};
%coordinates{(2,3.22)(0,4)(-2,3.22)};
%\addplot [thick,draw={\colorone},smooth] coordinates {(-0.22282,-3.9915) (-0.57143,-3.9441) (-1.,-3.8244) (-1.2959,-3.7041)(-1.5714,-3.551) (-1.8453,-3.3571) (-2.0412,-3.1841) (-2.2149,-3.)(-2.3706,-2.7991) (-2.4727,-2.6429) (-2.5832,-2.4403)(-2.7019,-2.1552) (-2.783,-1.8571) (-2.8275,-1.4582)(-2.8002,-1.0859) (-2.7143,-0.74808) (-2.5958,-0.5)(-2.4708,-0.31491) (-2.3571,-0.19279) (-2.2143,-0.081938)(-2.0659,-0.005515) (-2.2078,0.077938) (-2.4036,0.23922)
%(-2.5608,0.4392) (-2.6877,0.68771) (-2.786,1.) (-2.8262,1.3571)(-2.8065,1.7143) (-2.7075,2.136) (-2.5114,2.5714) (-2.2471,2.9614)(-2.0396,3.1825) (-1.6513,3.5) (-1.2857,3.7088) (-1.0714,3.8007)(-0.78571,3.8938) (-0.52476,3.9533) (-0.28571,3.9855)
%(-0.070563,3.9991) (0.27289,3.9872) (0.64411,3.9298) (1.082,3.7963)(1.4286,3.6354) (1.7143,3.4554) (1.9675,3.2532) (2.1511,3.0714)(2.3284,2.8571) (2.4288,2.7143) (2.5351,2.5351) (2.6283,2.3426)(2.7079,2.1365) (2.7939,1.7939) (2.8255,1.3969) (2.8017,1.0874)(2.7481,0.85714) (2.6429,0.58296) (2.552,0.42857) (2.438,0.27631)
%(2.3214,0.16103) (2.1786,0.06028) (2.1143,-0.028574)(2.2857,-0.13046) (2.4439,-0.28571) (2.5958,-0.5) (2.7143,-0.74808)(2.8002,-1.0859) (2.827,-1.4286) (2.7984,-1.773) (2.7079,-2.1365)(2.6052,-2.3948) (2.4292,-2.7137) (2.2453,-2.9596) (2.,-3.2231)(1.7143,-3.4554) (1.4682,-3.611) (1.1839,-3.7553) (0.86968,-3.8697)
%(0.57143,-3.9441) (0.33765,-3.9805) (0.072336,-3.9991) (-0.22282,-3.9915)};
\filldraw [] (axis cs:2,-3.22) node [shift={(15pt,-7pt)}] {\scriptsize $(2,-\sqrt[4]{108})$} circle (1pt);
\end{axis}
\node [right] at (myplot.right of origin) {\scriptsize $x$};
\node [above] at (myplot.above origin) {\scriptsize $y$};
\end{tikzpicture}}{An oval oriented along the y-axis, with its sides pinched in toward the origin.}}{\mbox{}\\[-2\baselineskip]\parbox[t]{\linewidth}{\begin{enumerate}
\item	$y=4$
\item	$y = 0.93(x-2)-\sqrt[4]{108}$
\end{enumerate}}}

\exercise{$(x^2+y^2+x)^2 = x^2+y^2$
\begin{enumerate}
\item	At $(0,1)$.
\item	At $\ds \left(-\frac34, \frac{3 \sqrt{3}}{4}\right)$.
\end{enumerate}
\pdftooltip{\begin{tikzpicture}[baseline=10pt]
\begin{axis}[width=\marginparwidth,tick label style={font=\scriptsize},
axis y line=middle,axis x line=middle,name=myplot,axis equal,
ymin=-1.5,ymax=1.5,xmin=-2.5,xmax=0.5]
\addplot[thick,draw={\colorone},smooth,domain=0:360,data cs=polar](x,{1-cos(x)});
%\addplot [thick,draw={\colorone},smooth] coordinates {(-2.,0) (-1.9547,-0.34466) (-1.8227,-0.66341) (-1.616,-0.93301)(-1.3529,-1.1352) (-1.056,-1.2584) (-0.75,-1.299) (-0.459,-1.2611)(-0.2038,-1.1558) (0,-1.) (0.14349,-0.8138) (0.22504,-0.6183)(0.25,-0.43301) (0.22961,-0.27364) (0.17922,-0.15038)(0.11603,-0.066987) (0.05667,-0.020626) (0.014961,-0.0026381) (0,0)
%(0.014961,0.0026381) (0.05667,0.020626) (0.11603,0.066987)(0.17922,0.15038) (0.22961,0.27364) (0.25,0.43301) (0.22504,0.6183)(0.14349,0.8138) (0,1.) (-0.2038,1.1558) (-0.459,1.2611)(-0.75,1.299) (-1.056,1.2584) (-1.3529,1.1352) (-1.616,0.93301)(-1.8227,0.66341) (-1.9547,0.34466) (-2.,0)};
\filldraw [] (axis cs:-.75,1.3) node [shift={(0pt,-10pt)}] {\scriptsize $\left(-\frac{3}{4},\frac{3\sqrt{3}}{4}\right)$} circle (1pt);
\end{axis}
\node [right] at (myplot.right of origin) {\scriptsize $x$};
\node [above] at (myplot.above origin) {\scriptsize $y$};
\end{tikzpicture}}{A round shape mostly to the left of the y-axis.  To the right of the y-axis, the roundness is interrupted by the curve pinching in to a cusp at the origin.}}{\mbox{}\\[-2\baselineskip]\parbox[t]{\linewidth}{\begin{enumerate}
\item	$y=-1/3x+1$
\item	$y = 3\sqrt{3}/4$
\end{enumerate}}}

\exercise{$(x-2)^2+(y-3)^2=9$
\begin{enumerate}
\item	At $\ds \left(\frac72,\frac{6+3\sqrt{3}}{2}\right)$.
\item	At $\ds\left(\frac{4+3\sqrt{3}}{2}, \frac32\right)$.
\end{enumerate}
\pdftooltip{\begin{tikzpicture}[baseline=10pt]
\begin{axis}[width=\marginparwidth,tick label style={font=\scriptsize},
axis y line=middle,axis x line=middle,name=myplot,axis equal,
ymin=-.5,ymax=6.5,xmin=-1.5,xmax=6.5]
\addplot [thick,draw={\colorone},smooth,domain=0:360] ({(2+3*cos(x))},{(3+3*sin(x))});
\filldraw [] (axis cs:4.6,1.5) node [left] {\scriptsize $\left(\frac{4+3\sqrt{3}}{2},1.5\right)$} circle (1pt);
\filldraw [] (axis cs:3.5,5.6) node [below left] {\scriptsize $\left(3.5,\frac{6+3\sqrt{3}}{2}\right)$} circle (1pt);
\end{axis}
\node [right] at (myplot.right of origin) {\scriptsize $x$};
\node [above] at (myplot.above origin) {\scriptsize $y$};
\end{tikzpicture}}{A circle centered at (2,3) with radius 3.}}{\mbox{}\\[-2\baselineskip]\parbox[t]{\linewidth}{\begin{enumerate}
\item	$y=-\frac{1}{\sqrt{3}}(x-\frac72)+\frac{6+3\sqrt{3}}{2}$
\item	$y = \sqrt{3}(x-\frac{4+3\sqrt{3}}2)+\frac32$
\end{enumerate}}}

\exercise{$x^2+2xy-y^2+x=2$
\begin{enumerate}
\item	At $(-2,0)$.
\item	At $(1,2)$.
\end{enumerate}
\pdftooltip{\begin{tikzpicture}%[baseline=10pt]
\begin{axis}[width=.7\linewidth,tick label style={font=\scriptsize},axis y line=middle,axis x line=middle,name=myplot,%
                        ymin=-2,ymax=3,%
                        xmin=-3,xmax=2,%
]
\addplot [thick,draw={\colorone},smooth,domain=-89:89]
		({sec(x)*sqrt(17)/4-.25},{(sec(x)+tan(x)*sqrt(2))*sqrt(17)/4-.25});
\addplot [thick,draw={\colorone},smooth,domain=91:269]
		({sec(x)*sqrt(17)/4-.25},{(sec(x)+tan(x)*sqrt(2))*sqrt(17)/4-.25});
\filldraw (axis cs:-2,0) node [above right] {\scriptsize $(-2,0)$} circle (1pt);
\filldraw (axis cs:1,2) node [below right] {\scriptsize $(1,2)$} circle (1pt);
\end{axis}
\node [right] at (myplot.right of origin) {\scriptsize $x$};
\node [above] at (myplot.above origin) {\scriptsize $y$};
\end{tikzpicture}}{A curve starting in the second quadrant near the x-axis, crossing the x-axis at x=-2, turning downward and then fading away from the y-axis as it continues downward.  A second curve mirrors the first with 180 degree symmetry.}}{\mbox{}\\[-2\baselineskip]\parbox[t]{\linewidth}{\begin{enumerate}
\item	$y=-\dfrac{3x}4-\dfrac32$
\item	$y=\dfrac72x-\dfrac32$
\end{enumerate}}}

\end{exerciseset}
