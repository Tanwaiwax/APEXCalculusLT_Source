\printconcepts

\exercise{T/F: Let $f$ be a position function. The average rate of change on $[a,b]$ is the slope of the line through the points $(a, f(a))$ and $(b,f(b))$.}{T}

\exercise{T/F: The definition of the derivative of a function at a point involves taking a limit.}{T}

\exercise{In your own words, explain the difference between the average rate of change and instantaneous rate of change.}{Answers will vary.}

\exercise{In your own words, explain the difference between Definitions \ref{def:derivative_at_a_point} and \ref{def:the_derivative}.}{Answers will vary.}

\exercise{Let $y=f(x)$. Give three different notations equivalent to ``$\fp(x)$.''}{Answers will vary.}

\printproblems

{\noindent In Exercises}
{,\begin{enumerate}
\item[(a)] use the definition of the derivative to compute the derivative of the given function.
\item[(b)] Find the tangent line to the graph of the given function at $x=c$.
\end{enumerate}
}
\exinput{exercises/02_01_ex_10}
\exinput{exercises/02_01_ex_11}
\exinput{exercises/02_01_ex_12}
\exinput{exercises/02_01_ex_13}
\exinput{exercises/02_01_ex_35}
\exinput{exercises/02_01_ex_14}
\exinput{exercises/02_01_ex_36}
\exinput{exercises/02_01_ex_15}
\exinput{exercises/02_01_ex_37}
\exinput{exercises/02_01_ex_16}


{\noindent In Exercises}
{, each limit represents the derivative of some function, $f$, at some number $c$. State an appropriate $f$ and $c$ for each.
}
\exinput{exercises/02_01_ex_17}
\exinput{exercises/02_01_ex_18}
\exinput{exercises/02_01_ex_19}
\exinput{exercises/02_01_ex_20}


\begin{exerciseset}{In Exercises}{, a function $f$ and an $x$-value $a$ are given. Approximate the equation of the tangent line to the graph of $f$ at $x=a$ by numerically approximating $\fp(a)$, using $h=0.1$.}

\exercise{$f(x) = x^2+2x+1$, $x=3$}{$y=8.1(x-3)+16$}

\exercise{$\ds f(x) = \sqrt x$, $x=4$}{$y%=2+.248(x-4)
=.248x+1.006$}

\exercise{$\ds f(x) = \frac{10}{x+1}$, $x=9$}{$y=-0.099(x-9)+1$}

\exercise{$\ds f(x) = e^x$, $x=2$}{$y=7.77(x-2)+e^2$, or $y = 7.77(x-2)+7.39$}

\exercise{$\ds f(x) = \ln x$, $x=2$}{$y=.49(x-2)+\ln2$}

\exercise{$\ds f(x) = \cos x$, $x=0$}{$y=-0.05x+1$}

\end{exerciseset}


\exercise{The graph of $f(x)=x^2-1$ is shown. 
	\begin{enumerate}
	\item		Use the graph to approximate the slope of the tangent line to $f$ at the following points: $(-1,0)$, $(0,-1)$ and $(2,3)$. 
	\item		Using the definition, find $\fp(x)$.
	\item		Find the slope of the tangent line at the points $(-1,0)$, $(0,-1)$ and $(2,3)$.
	\end{enumerate}
\begin{tikzpicture}[scale=.8]
\begin{axis}[width=\marginparwidth+25pt,tick label style={font=\scriptsize},
axis y line=middle,axis x line=middle,name=myplot,
ytick={-1,1,2,3},ymin=-1.3,ymax=3.5,xmin=-2.1,xmax=2.1,grid=major]

\addplot [thick,draw={\colorone},smooth,domain=-2.1:2.1] {x^2-1};

\end{axis}
\node [right] at (myplot.right of origin) {\scriptsize $x$};
\node [above] at (myplot.above origin) {\scriptsize $y$};
\end{tikzpicture}}{\mbox{}\\[-2\baselineskip]\begin{enumerate}
\item	Approximations will vary; they should match (c) closely.
\item	$\fp(x) = 2x$
\item	At $(-1,0)$, slope is $-2$. At $(0,-1)$, slope is 0. At $(2,3)$, slope is 4.
\end{enumerate}}

\exercise{The graph of $\ds f(x)=\frac{1}{x+1}$ is shown. 
	\begin{enumerate}
	\item		Use the graph to approximate the slope of the tangent line to $f$ at the following points: $(0,1)$ and $(1,0.5)$. 
	\item		Using the definition, find $\fp(x)$.
	\item		Find the slope of the tangent line at the points $(0,1)$ and $(1,0.5)$.
	\end{enumerate}
\begin{tikzpicture}[scale=.8]
\begin{axis}[width=\marginparwidth+25pt,tick label style={font=\scriptsize},
axis y line=middle,axis x line=middle,name=myplot,ytick={1,2,3,4,5},
ymin=-.5,ymax=5.5,xmin=-1.1,xmax=3.1,grid=major]

\addplot [thick,draw={\colorone},smooth,domain=-.9:3.1,samples=50] {1/(x+1)};

\end{axis}
\node [right] at (myplot.right of origin) {\scriptsize $x$};
\node [above] at (myplot.above origin) {\scriptsize $y$};
\end{tikzpicture}}{\mbox{}\\[-2\baselineskip]\begin{enumerate}
\item	Approximations will vary; they should match (c) closely.
\item		$\fp(x) = -1/(x+1)^2$
\item		At $(0,1)$, slope is $-1$. At $(1,0.5)$, slope is $-1/4$.
\end{enumerate}
}

\exerciseset{In Exercises}{, a graph of a function $f(x)$ is given. Using the graph, sketch $\fp(x)$.}{

\exercise{\noindent\begin{minipage}{\linewidth}
\myincludegraphics[scale=.8]{figures/fig02_01_ex_26}
\end{minipage}
}{\mbox{}\\[-\baselineskip]\myincludegraphics[scale=.8]{figures/fig02_01_ex_26ans}}

\exercise{\noindent\begin{minipage}{\linewidth}
\myincludegraphics[scale=.8]{figures/fig02_01_ex_27}
\end{minipage}
}{\mbox{}\\[-\baselineskip]\myincludegraphics[scale=.8]{figures/fig02_01_ex_27ans}}

\exercise{\noindent\begin{minipage}{\linewidth}
\myincludegraphics[scale=.8]{figures/fig02_01_ex_28}
\end{minipage}
}{\mbox{}\\[-\baselineskip]\myincludegraphics[scale=.8]{figures/fig02_01_ex_28ans}}

\exercise{\noindent\begin{minipage}{\linewidth}
\myincludegraphics[scale=.8]{figures/fig02_01_ex_29}
\end{minipage}
}{\mbox{}\\[-\baselineskip]\myincludegraphics[scale=.8]{figures/fig02_01_ex_29ans}}

}


\exercise{Using the graph of $g(x)$ below, answer the following questions.\\
\begin{minipage}[t]{.5\linewidth}
	\begin{enumerate}
		\item	Where is $g(x) > 0$?
		\item	Where is $g(x) < 0$?
		\item	Where is $g(x) = 0$?
	\end{enumerate}
\end{minipage}%
\begin{minipage}[t]{.5\linewidth}
	\begin{enumerate}\addtocounter{enumii}{3}
		\item	Where is $g'(x) < 0$?
		\item	Where is $g'(x) > 0$?
		\item	Where is $g'(x) = 0$?
\end{enumerate}
\end{minipage}

\begin{tikzpicture}[scale=.8]
\begin{axis}[width=\marginparwidth+25pt,tick label style={font=\scriptsize},
axis y line=middle,axis x line=middle,name=myplot,xtick={-2,-1,1,2},
ymin=-7.9,ymax=5.9,xmin=-2.5,xmax=2.5]

\addplot [thick,draw={\colorone},smooth,domain=-2.3:2.3,samples=50] {(-10)*(x^4/4-x^2/2)+1};

\end{axis}
\node [right] at (myplot.right of origin) {\scriptsize $x$};
\node [above] at (myplot.above origin) {\scriptsize $y$};
\end{tikzpicture}}{\mbox{}\\[-2\baselineskip]\begin{enumerate}
	\item	Approximately on $(-1.5,1.5)$.
	\item	Approximately on $(-\infty,-1.5) \cup (1.5,\infty)$.
	\item	Approximately at $x=\pm 1.5$.
	\item	On $(-\infty,-1) \cup (0,1)$.
	\item	On $(-1,0) \cup (1,\infty)$.
	\item	At $x=\pm 1$.
\end{enumerate}}

\printreview

\exercise{Approximate $\ds \lim_{x\to 5}\frac{x^2+2 x-35}{x^2-10.5 x+27.5}$.}{Approximately 24.}

\exercise{Use the Bisection Method to approximate, accurate to two decimal places, the root of $g(x) = x^3+x^2+x-1$ on $[0.5,0.6]$.}{Approximately $0.54$.}

\exercise{Give intervals on which each of the following functions are continuous.

\noindent\begin{minipage}[t]{.49\linewidth}
\begin{enumerate}
\item		$\ds \frac{1}{e^x+1}$
\item		$\ds \frac{1}{x^2-1}$
\end{enumerate}
\end{minipage}
\begin{minipage}[t]{.49\linewidth}
\begin{enumerate}\addtocounter{enumii}{2}
\item		$\ds \sqrt{5-x}\phantom{\frac{1}{e^x+1}}$
\item		$\ds \sqrt{5-x^2}\phantom{\frac{1}{x^2-1}}$
\end{enumerate}
\end{minipage}}{\mbox{}\\[-2\baselineskip]\begin{enumerate}
\item		$(-\infty,\infty)$
\item		$(-\infty,-1)\cup (-1,1) \cup (1,\infty)$
\item		$(-\infty,5]$
\item		$[-\sqrt5,\sqrt5]$
\end{enumerate}}

\exercise{Use the graph of $f(x)$ provided to answer the following.

\noindent\begin{minipage}[t]{.49\linewidth}
\begin{enumerate}
\item		$\ds \lim_{x\to-3^-} f(x) = $?
\item		$\ds \lim_{x\to-3^+} f(x) = $?
\end{enumerate}
\end{minipage}
\begin{minipage}[t]{.49\linewidth}
\begin{enumerate}\addtocounter{enumii}{2}
\item		$\ds \lim_{x\to-3} f(x) = $?
\item		Where is $f$ continuous?
\end{enumerate}
\end{minipage}

\begin{tikzpicture}[scale=.8]
\begin{axis}[width=\marginparwidth+25pt,tick label style={font=\scriptsize},
axis y line=middle,axis x line=middle,name=myplot,
ymin=-1.1,ymax=3.1,xmin=-5.1,xmax=0.5]

\addplot [thick,draw={\colorone},smooth,domain=-5:-3] {(x+3)^2+1};
\addplot [thick,draw={\colorone},smooth,domain=-3:0.5] {-(x+3)^2+3};
\filldraw [fill=white] (axis cs:-3,1) circle (1.5pt);
\filldraw [fill=white] (axis cs:-3,3) circle (1.5pt);
\filldraw [fill=black] (axis cs:-3,2) circle (1.5pt);

\end{axis}
\node [right] at (myplot.right of origin) {\scriptsize $x$};
\node [above] at (myplot.above origin) {\scriptsize $y$};
\end{tikzpicture}}{\mbox{}\\[-2\baselineskip]\begin{enumerate}
\item		1
\item		3
\item		Does not exist
\item		$(-\infty,-3)\cup (3,\infty)$
\end{enumerate}}
