\printconcepts

%\exercise{T/F: Every function has an inverse.}{F}

%\exercise{In your own words explain what it means for a function to be ``one to one.''}{Answers will vary.}

%\exercise{If $(1,10)$ lies on the graph of $y=f(x)$, what can be said about the graph of $y=f^{-1}(x)$?}{The point $(10,1)$ lies on the graph of $y=f^{-1}(x)$ (assuming $f$ is invertible).}

\exercise{If $(1,10)$ lies on the graph of $y=f(x)$ and $\fp(1) = 5$, what can be said about $y=f^{-1}(x)$?}{The point $(10,1)$ lies on the graph of $y=f^{-1}(x)$ (assuming $f$ is invertible) and $(f^{-1})'(10) = 1/5$.}

\printproblems

%\begin{exerciseset}{In Exercises}{, verify that the given functions are inverses.}

\exercise{$f(x) = 2x+6$ and $g(x) = \frac12x-3$}{Compose $f(g(x))$ and $g(f(x))$ to confirm that each equals $x$.}

\exercise{$f(x) = x^2+6x+11$, $x\geq-3$ and $g(x) = \sqrt{x-2}-3$, $x\geq 2$}{Compose $f(g(x))$ and $g(f(x))$ to confirm that each equals $x$.}

% cut for parity
%\exercise{$f(x)=x^2+6x+11$, $x\le 3$ and $g(x)=-\sqrt{x-2}-3$, $x\ge 2$.}{Compose $f(g(x))$ and $g(f(x))$ to confirm that each equals $x$.}

\exercise{$\ds f(x) = \frac{3}{x-5}$, $x\neq 5$ and  $\ds g(x) = \frac{3+5x}{x}$, $x\neq 0$}{Compose $f(g(x))$ and $g(f(x))$ to confirm that each equals $x$.}

\exercise{$\ds f(x) = \frac{x+1}{x-1}$, $x\neq 1$ and $g(x) = f(x)$}{Compose $f(g(x))$ and $g(f(x))$ to confirm that each equals $x$.}

\end{exerciseset}


\input{exercises/02_07_exset_02}

\exerciseset{In Exercises}{, compute the derivative of the given function.}{

\exercise{$h(t) = \sin^{-1} (2t)$}{$h'(t) = \frac{2}{\sqrt{1-4t^2}}$}

\exercise{$f(t) = \sec^{-1} (2t)$}{$\fp(t) = \frac{1}{\abs t\sqrt{4t^2+1}}$}

\exercise{$g(x)=\tan^{-1} (2x)$}{$g'(x) = \frac{2}{1+4x^2}$}

\exercise{$f(x) =x\sin^{-1}x$}{$\fp(x) = \frac{x}{\sqrt{1-x^2}}+\sin ^{-1}(x)$}

\exercise{$g(t) = \sin t\cos^{-1}t$}{$g'(t) = \cos ^{-1}(t) \cos (t)-\frac{\sin (t)}{\sqrt{1-t^2}}$}

\exercise{$\ds h(x) = \frac{\sin^{-1}x}{\cos^{-1}x}$}{$h'(x)=\frac{\sin ^{-1}(x)+\cos ^{-1}(x)}{\sqrt{1-x^2} \cos ^{-1}(x)^2}$}

\exercise{$g(x) = \tan^{-1}(\sqrt{x})$}{$g'(x) = \frac{1}{\sqrt{x} (2 x+2)}$}

\exercise{$f(x) =\sec^{-1}(1/x)$}{$\fp(x) = -\frac{1}{\sqrt{1-x^2}}$}

\exercise{$f(x) =\sin (\sin^{-1}x)$}{\begin{enumerate}
\item		$f(x) = x$, so $\fp(x) = 1$
\item		$\fp(x) = \cos(\sin^{-1}x)\frac{1}{\sqrt{1-x^2}} = 1$.
\end{enumerate}}

}


\input{exercises/02_07_exset_04}

\input{exercises/02_07_exset_05}

%\begin{exerciseset}{In Exercises}{, find the requested derivative.}

\exercise{$f(x) = x\sin x$; find $\fpp(x)$.}{$\fpp(x) = 2\cos x - x\sin x$}

\exercise{$f(x) = x\sin x$; find $f\,^{(4)}(x)$.}{$f^{(4)}(x) = -4\cos x+ x\sin x$}

\exercise{$f(x) = \csc x$; find $\fpp(x)$.}{$\fpp(x) = \cot^2x\csc x + \csc^3 x$}

\exercise{$f(x) = (x^3-5x+2)(x^2+x-7)$; find $f\,^{(8)}(x)$.}{$f^{(8)} = 0$}

\end{exerciseset}


%\exerciseset{In Exercises}{, use the graph of $f(x)$ to sketch $\fp(x)$.}{

\exercise{\begin{minipage}[]{\linewidth}
\myincludegraphics[scale=.8]{figures/fig02_04_ex_43}
\end{minipage}
}{\mbox{}\\[-\baselineskip]\myincludegraphics[scale=.8]{figures/fig02_04_ex_43a}
}

\exercise{\begin{minipage}[]{\linewidth}
\myincludegraphics[scale=.8]{figures/fig02_04_ex_44}
\end{minipage}
}{\mbox{}\\[-\baselineskip]\myincludegraphics[scale=.8]{figures/fig02_04_ex_44a}
}

\exercise{\begin{minipage}[]{\linewidth}
\myincludegraphics[scale=.8]{figures/fig02_04_ex_45}
\end{minipage}
}{\mbox{}\\[-\baselineskip]\myincludegraphics[scale=.8]{figures/fig02_04_ex_45a}
}

}


\exercise{A regulation hockey goal is 6 feet wide. If a player is skating towards the end line on a line perpendicular to the end line and 10 feet from the imaginary line joining the center of one goal to the center of the other, the angle between the player and the goal first increases and then begins to decrease. In order to maximize this angle, how far from the end line should the player be when they shoot the puck?

\begin{tikzpicture}[scale=.8]
\draw[dotted] (0,0) -- (8,0);
\draw (0,-3) -- node[pos=.9,yshift=12,scale=.2]{\psBill} (5,-3);
\draw (5,-3) -- node[pos=.3,above]{$\theta$}(8,-1) -- node[pos=.5,right]{6 ft} (8,1) -- cycle;
%\draw (5.5,-2.6) node[above right]{$\theta$};
\draw (1,0) -- node[pos=.5,right]{10 ft} (1,-3);
\end{tikzpicture}}{$\sqrt{91}\approx 9.54$ feet}

\exerciseset{In Exercises}{, compute the indicated integral.}{

\exercise{$\ds\int_{1/\sqrt 2}^{1/2} \frac{2}{\sqrt{1-x^2}}\,dx$}{$\pi/2$}

\exercise{$\ds\int_{0}^{\sqrt 3}  \frac{4}{9+x^2}\,dx$}{$2\pi/9$}

\exercise{$\ds\int\frac{1-t}{1+t^2}\,dt$}{$\tan^{-1}t -\frac12\ln(1+t^2)+C$}

\exercise{$\ds\int\frac{\sin^{-1}r}{\sqrt{1-r^2}}\,dr$}{$\frac12\left(\sin^{-1}r\right)^2+C$}

\exercise{$\ds\int\frac{1}{(1+x^2)\tan^{-1}x}\,dx$}{$\ln\abs{\tan^{-1}x}+C$}

\exercise{$\ds\int\frac{x^3}{4+x^8}\,dx$}{$\frac18\tan^{-1}(x^4/2)+C$}

\exercise{$\ds\int\frac{e^t}{\sqrt{10-e^{2t}}}\,dt$}{$\sin^{-1}(e^t/\sqrt{10})+C$}

\exercise{$\ds\int\frac{1}{\sqrt{3-x^2+2x}}\,dx$}{$\sin^{-1}\left(\frac{x-1}2 \right)+C$}

\exercise{$\ds\int\frac{1}{\sqrt x(1+x)}\,dx$}{$2\tan^{-1}(\sqrt x)+C$}

}

\printreview

\exercise{Find $\dfrac{dy}{dx}$, where $x^2y-y^2x = 1$.}{ $\frac{dy}{dx} = \frac{y (y-2 x)}{x (x-2 y)}$}

\exercise{Find the derivative of %equation of the line tangent to the graph of\\
$x^2+y^2+xy=7$ at the point $(1,2)$.}{$-\frac45$}%{$y=-\frac45(x-1)+2$}

%\exercise{Let $f(x) = x^3+x$. Evaluate $\ds \lim_{s\to 0} \frac{f(x+s)-f(x)}{s}$.}{ $3x^2+1$}

%\exercise{Approximate the value of $(3.01)^4$ using the tangent line to $f(x) = x^4$ at $x=3$.}{The tangent line to $f(x) = x^4$ at $x=3$ is $y=108(x-3)+81$; thus $(3.01)^4 \approx y(3.01) = 108(.01)+81 = 82.08$. }
