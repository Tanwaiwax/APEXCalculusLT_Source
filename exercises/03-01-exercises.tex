\printconcepts

\exercise{Describe what an ``extreme value'' of a function is in your own words.}{Answers will vary.}

\exercise{Sketch the graph of a function $f$ on $(-1,1)$ that has both a maximum and minimum value.}{Answers will vary.}

\exercise{Describe the difference between absolute and relative maxima in your own words.}{Answers will vary.}

\exercise{Sketch the graph of a function $f$ where $f$ has a relative maximum at $x=1$ and $\fp(1)$ is undefined.}{Answers will vary.}

\exercise{T/F: If $c$ is a critical value of a function $f$, then $f$ has either a relative maximum or relative minimum at $x=c$. }{F}

\exercise{Fill in the blanks: The critical points of a function $f$ are found where $\fp(x)$ is equal to \underline{\hskip.5in} or where $\fp(x)$ is \underline{\hskip.5in}.}{Where $\fp(x)$ is equal to \underline{0} or where $\fp(x)$ is \underline{undefined}.}

\printproblems

\input{exercises/03-01-exset-01}

\exercisesetinstructions{, evaluate $\fp(x)$ at the points indicated in the graph.}

\exercise{$\ds f(x) = \frac{2}{x^2+1}$\\
\pdftooltip{\begin{tikzpicture}[baseline=10pt]
\begin{axis}[width=\marginparwidth,tick label style={font=\scriptsize},
axis y line=middle,axis x line=middle,name=myplot,
ymin=-.5,ymax=2.5,xmin=-5.5,xmax=5.5]
\addplot [thick,draw={\colorone},smooth] {2/((x^2+1))};
\draw [draw={\colorone},fill=\colorone]    (axis cs: 0,2) node [above right,black] {\scriptsize $(0,2)$} circle (1.5pt);
\end{axis}
\node [right] at (myplot.right of origin) {\scriptsize $x$};
\node [above] at (myplot.above origin) {\scriptsize $y$};
\end{tikzpicture}}{A curve starting above the x-axis near (-5,0), curving up to peak as it crosses the y-axis at a solid dot at y=2, and then continuing in a mirror image across the y-axis to finish above the x-axis near (5,0).}}{$\fp(0) = 0$}

\exercise{$\ds f(x) = x^2\sqrt{6-x^2}$\\
\pdftooltip{\begin{tikzpicture}[baseline=10pt]
\begin{axis}[width=\marginparwidth,tick label style={font=\scriptsize},
axis y line=middle,axis x line=middle,name=myplot,
ymin=-.9,ymax=6.5,xmin=-3.5,xmax=3.5]
\addplot [thick,draw={\colorone},smooth,domain=-2.44948:2.449489743,samples=100] {x^2*sqrt(6-x^2)};
\draw [draw={\colorone},fill=\colorone]    (axis cs: 0,0) node [below right,black] {\scriptsize $(0,0)$} circle (1.5pt);
\draw [draw={\colorone},fill=\colorone]    (axis cs: 2,5.66) node [left,black] {\scriptsize $(2,4\sqrt{2})$} circle (1.5pt);
\end{axis}
\node [right] at (myplot.right of origin) {\scriptsize $x$};
\node [above] at (myplot.above origin) {\scriptsize $y$};
\end{tikzpicture}}{A curve starting near (-2.4,0) and going steeply up to peak near (-2,5.7) before turning around to descend and bottom out at the origin.  It then continues in a mirror image across the y-axis to finish near (2.4,0).  There is a solid dot at the origin and near (2,5.7).}}{$\fp(0) = 0$; $\fp(2) = 0$}

\exercise{$\ds f(x) = \sin x$\\
\pdftooltip{\begin{tikzpicture}[baseline=10pt]
\begin{axis}[width=\marginparwidth,tick label style={font=\scriptsize},
axis y line=middle,axis x line=middle,name=myplot,
ymin=-1.5,ymax=1.5,xmin=-.5,xmax=6.5]
\addplot [thick,draw={\colorone},smooth,domain=0:6.28] {sin(deg(x))};
\draw [draw={\colorone},fill=\colorone]    (axis cs: 1.57,1) node [above,black] {\scriptsize $(\pi/2,1)$} circle (1.5pt);
\draw [draw={\colorone},fill=\colorone]    (axis cs: 4.71,-1) node [below,black] {\scriptsize $(3\pi/2,-1)$} circle (1.5pt);
\end{axis}
\node [right] at (myplot.right of origin) {\scriptsize $x$};
\node [above] at (myplot.above origin) {\scriptsize $y$};
\end{tikzpicture}}{A wave that starts at the origin, increases to peak at a solid dot at (π/2,1), decreases to cross the origin at x=π, bottoms out at a solid dot at (3π/2,-1), and increases to finish at (2π,0).}}{$\fp(\pi/2) = 0$; $\fp(3\pi/2) = 0$}

\exercise{$\ds f(x) = x^2\sqrt{4-x}$\\
\pdftooltip{\begin{tikzpicture}[baseline=10pt]
\begin{axis}[width=\marginparwidth,tick label style={font=\scriptsize},
axis y line=middle,axis x line=middle,name=myplot,
ymin=-1.5,ymax=12,xmin=-2.5,xmax=4.5]
\addplot [thick,draw={\colorone},smooth,domain=-2:4] {x^2*sqrt(4-x)};
\draw [draw={\colorone},fill=\colorone]    (axis cs: 0,0) node [below right,black] {\scriptsize $(0,0)$} circle (1.5pt);
\draw [draw={\colorone},fill=\colorone]    (axis cs: 3.2,9.16) node [above,black] {\scriptsize $\left(\frac{16}{5},\frac{512}{25\sqrt{5}}\right)$} circle (1.5pt);
\draw [draw={\colorone},fill=\colorone]    (axis cs: 4,0) node [above left,black] {\scriptsize $(4,0)$} circle (1.5pt);
\end{axis}
\node [right] at (myplot.right of origin) {\scriptsize $x$};
\node [above] at (myplot.above origin) {\scriptsize $y$};
\end{tikzpicture}}{A curve starting near (-2,5.7), decreases to flatten out at a solid dot at the the origin, increases to a solid dot at (16/5,512/25√5), and then turns back downward to finish at a solid dot at (4,0).}}{$\fp(0) = 0$; $\fp(3.2) = 0$; $\fp(4)$ is undefined}

\exercise{$\ds f(x) = \begin{cases} x^2 & x\leq 0 \\ x^5 & x> 0 \end{cases}$\\
\pdftooltip{\begin{tikzpicture}[baseline=10pt]
\begin{axis}[width=\marginparwidth,tick label style={font=\scriptsize},
axis y line=middle,axis x line=middle,name=myplot,
ymin=-.5,ymax=1.1,xmin=-1.1,xmax=1.1]
\addplot [thick,draw={\colorone},smooth,domain=-1:0] {x^2};
\addplot [thick,draw={\colorone},smooth,domain=0:1] {x^5};
\draw [draw={\colorone},fill=\colorone]    (axis cs: 0,0) node [below right,black] {\scriptsize $(0,0)$} circle (1.5pt);
\end{axis}
\node [right] at (myplot.right of origin) {\scriptsize $x$};
\node [above] at (myplot.above origin) {\scriptsize $y$};
\end{tikzpicture}}{A curve beginning at (-1,1), decreasing downward to be flat at a solid dot at the origin.  It is even flatter after the origin, but curves upward to finish at (1,1).}}{$\fp(0) = 0$}

\exercise{$\ds f(x) = \begin{cases}x^2 & x\leq 0 \\ x & x> 0 \end{cases}$\\
\pdftooltip{\begin{tikzpicture}[baseline=10pt]
\begin{axis}[width=\marginparwidth,tick label style={font=\scriptsize},
axis y line=middle,axis x line=middle,name=myplot,
ymin=-.5,ymax=1.1,xmin=-1.1,xmax=1.1]
\addplot [thick,draw={\colorone},smooth,domain=-1:0] {x^2};
\addplot [thick,draw={\colorone},smooth,domain=0:1] {x};
\draw [draw={\colorone},fill=\colorone]    (axis cs: 0,0) node [below right,black] {\scriptsize $(0,0)$} circle (1.5pt);
\end{axis}
\node [right] at (myplot.right of origin) {\scriptsize $x$};
\node [above] at (myplot.above origin) {\scriptsize $y$};
\end{tikzpicture}}{A curve starting at (-1,1), decreasing to downward to be flat at a solid dot at the origin.  After the origin, it is a straight line to finish at (1,1).}}{$\fp(0)$ is not defined}

\exercise{$\ds f(x) = \frac{(x-2)^{2/3}}{x}+1$\\
\pdftooltip{\begin{tikzpicture}[baseline=10pt]
\begin{axis}[width=\marginparwidth,tick label style={font=\scriptsize},
axis y line=middle,axis x line=middle,name=myplot,
ymin=-.5,ymax=6.5,xmin=-1,xmax=11]
\addplot [thick,draw={\colorone},domain=.2:11,samples=100] {(((x-2)^2)^(1/3))/x+1};
\draw [draw={\colorone},fill=\colorone]    (axis cs: 2,1) node [below,black] {\scriptsize $(2,1)$} circle (1.5pt);
\draw [draw={\colorone},fill=\colorone]    (axis cs: 6,1.42) node [above,black] {\scriptsize $\left(6,1+\frac{\sqrt[3]{2}}{3}\right)$} circle (1.5pt);
\end{axis}
\node [right] at (myplot.right of origin) {\scriptsize $x$};
\node [above] at (myplot.above origin) {\scriptsize $y$};
\end{tikzpicture}}{A curve starting near (0,6), but with a positive x value, decreasing to a solid dot at a cusp at the point (2,1).  After the cusp, it increases to a solid dot the point (6,1+cuberoot(2)/3).  The curve is relatively flat near and after that point.}}{$\fp(2)$ is not defined; $\fp(6) = 0$}

\exercise{$\ds f(x) = \sqrt[3]{x^4-2x^2+1}$\\
\pdftooltip{\begin{tikzpicture}[baseline=10pt]
\begin{axis}[width=\marginparwidth,tick label style={font=\scriptsize},
axis y line=middle,axis x line=middle,name=myplot,xtick={-2,-1,1,2},
ymin=-.5,ymax=3.5,xmin=-2.5,xmax=2.5]
\addplot [thick,draw={\colorone},domain=-2:-1,samples=25,smooth] {(x^4-2*x^2+1)^(1/3)};
\addplot [thick,draw={\colorone},domain=-1:1,samples=50,smooth] {(x^4-2*x^2+1)^(1/3)};
\addplot [thick,draw={\colorone},domain=1:2,samples=25,smooth] {(x^4-2*x^2+1)^(1/3)};
\draw [draw={\colorone},fill={\colorone}]    (axis cs: 1,0) node [above right,black] {\scriptsize $(1,0)$} circle (1.5pt);
\draw [draw={\colorone},fill={\colorone}]    (axis cs: -1,0) node [above left,black] {\scriptsize $(-1,0)$} circle (1.5pt);
\end{axis}
\node [right] at (myplot.right of origin) {\scriptsize $x$};
\node [above] at (myplot.above origin) {\scriptsize $y$};
\end{tikzpicture}}{A curve starting near (-2,2.1), going downward to a cusp at a solid dot at (-1,0), bouncing back up to a semi-circle shape that crosses the y-axis at y=1.  After crossing the y-axis, the curve is a mirror image across the y-axis, finishing near (2,2.1).}}{Both $\fp(-1)$ and $\fp(1)$ are undefined.}

\exercisesetend


\exercisesetinstructions{, find the extreme values of the function on the given interval.}

\exercise{$\ds f(x) = x^2+x+4$\quad on \quad $[-1,2]$.}{min: $(-0.5,3.75)$

max: $(2,10)$}

\exercise{$\ds f(x) = x^3-\frac92x^2-30x+3$\quad  on \quad $[0,6]$.}{min: $(5,-134.5)$

max: $(0,3)$}

\exercise{$\ds f(x) = 3\sin x$\quad  on \quad $[\pi/4,2\pi/3]$.}{min: $(\pi/4,3\sqrt{2}/2)$

max: $(\pi/2,3)$}

\exercise{$\ds f(x) = x^2\sqrt{4-x^2}$\quad  on \quad $[-2,2]$.}{min: $(0,0)$ and $(\pm 2,0)$

max: $(\pm 2\sqrt{2/3},16\sqrt{3}/9)$}

\exercise{$\ds f(x) = x+\frac3x$\quad  on \quad $[1,5]$.}{min: $(\sqrt3,2\sqrt3)$

max: $(5,28/5)$}

\exercise{$\ds f(x) = \frac{x^2}{x^2+5}$\quad  on \quad $[-3,5]$.}{min: $(0,0)$

max: $(5,5/6)$}

\exercise{$\ds f(x) = e^x\cos x$\quad  on \quad $[0,\pi]$.}{min: $(\pi,-e^\pi)$

max: $(\pi/4,\frac{\sqrt{2}e^{\pi/4}}{2})$}

\exercise{$\ds f(x) = e^x\sin x$\quad  on \quad $[0,\pi]$.}{min: $(0,0)$ and $(\pi,0)$

max: $(3\pi/4,\frac{\sqrt{2}e^{3\pi/4}}{2})$}

\exercise{$\ds f(x) = \frac{\ln x}{x}$\quad  on \quad $[1,4]$.}{min: $(1,0)$

max: $(e,1/e)$}

\exercise{$\ds f(x) = x^{2/3}-x$\quad  on \quad $[0,2]$.}{min: $(2,2^{2/3}-2)$

max: $(8/27,4/27)$}

\exercisesetend


\exercise{\mbox{}\\[-2.5\baselineskip]\parbox[t]{\linewidth}{\begin{enumext}
\item Sketch the graph of a function that has a local minimum at 3 and is differentiable at 3.
\item Sketch the graph of a function that has a local minimum at 3 and is continuous but not differentiable at 3.
\item Sketch the graph of a function that has a local minimum at 3 and is not continuous at 3.
\end{enumext}}}{Answers will vary.}

\exercise{Show that 4 is a critical number of $f(x)=(x-4)^3+7$ but $f$ does not have a relative extreme value at 4.}{$\fp(x)=3(x-4)^2$, so $\fp(4)=0$.  But $f(x)>7$ for $x>4$ and $f(x)<7$ for $x<4$.}

\exercise{A cubic function is a polynomial of degree 3; that is, it has the form $ax^3+bx^2+cx+d$, where $a\neq 0$.
\begin{enumext}
\item Show that a cubic function can have 2, 1, or 0 critical numbers.  Give examples and sketches to illustrate the 3 possibilities.
\item How many local extreme values can a cubic function have?
\end{enumext}}{\mbox{}\\[-2\baselineskip]\parbox[t]{\linewidth}{\begin{enumext}\item $x^3-x$, $x^3$, and $x^3+x$ have 2, 1, and 0 critical numbers respectively.  Because the derivative is a quadratic with at most 2 roots, a cubic cannot have 3 or more critical numbers.
\item A cubic can only have 2 or 0 extreme values.
\end{enumext}}}

\exercise{Suppose that $a$ and $b$ are positive numbers.  Find the extreme values of $f(x)=x^a(1-x)^b$ on $[0,1]$.}{min: $(0,0)$ and $(1,0)$\\ max: $(\frac a{a+b},\frac{a^ab^b}{(a+b)^{a+b}})$}

\printreview

\exercise{Find $\dfrac{\dd y}{\dd x}$, where $x^2y-y^2x = 1$.}{$\frac{\dd y}{\dd x} = \frac{y (y-2 x)}{x (x-2 y)}$}

\exercise{Find the equation of the line tangent to the graph of $x^2+y^2+xy=7$ at the point $(1,2)$.}{$y=-\frac45(x-1)+2$}

\exercise{Let $f(x) = x^3+x$. 

Evaluate $\ds \lim_{s\to 0} \frac{f(x+s)-f(x)}{s}$.}{$3x^2+1$}
