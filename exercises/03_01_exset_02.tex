\begin{exerciseset}{In Exercises}{, evaluate $\fp(x)$ at the points indicated in the graph.}

\exercise{$\ds f(x) = \frac{2}{x^2+1}$\\
\begin{tikzpicture}[baseline=10pt]
\begin{axis}[width=\marginparwidth,tick label style={font=\scriptsize},
axis y line=middle,axis x line=middle,name=myplot,
ymin=-.5,ymax=2.5,xmin=-5.5,xmax=5.5]
\addplot [thick,draw={\colorone},smooth] {2/((x^2+1))};
\draw [draw={\colorone},fill=\colorone]    (axis cs: 0,2) node [above right,black] {\scriptsize $(0,2)$} circle (1.5pt);
\end{axis}
\node [right] at (myplot.right of origin) {\scriptsize $x$};
\node [above] at (myplot.above origin) {\scriptsize $y$};
\end{tikzpicture}}{$\fp(0) = 0$}

\exercise{$\ds f(x) = x^2\sqrt{6-x^2}$\\
\begin{tikzpicture}[baseline=10pt]
\begin{axis}[width=\marginparwidth,tick label style={font=\scriptsize},
axis y line=middle,axis x line=middle,name=myplot,
ymin=-.9,ymax=6.5,xmin=-3.5,xmax=3.5]
\addplot [thick,draw={\colorone},smooth,domain=-2.44948:2.449489743,samples=100] {x^2*sqrt(6-x^2)};
\draw [draw={\colorone},fill=\colorone]    (axis cs: 0,0) node [below right,black] {\scriptsize $(0,0)$} circle (1.5pt);
\draw [draw={\colorone},fill=\colorone]    (axis cs: 2,5.66) node [left,black] {\scriptsize $(2,4\sqrt{2})$} circle (1.5pt);
\end{axis}
\node [right] at (myplot.right of origin) {\scriptsize $x$};
\node [above] at (myplot.above origin) {\scriptsize $y$};
\end{tikzpicture}}{$\fp(0) = 0$; $\fp(2) = 0$}

\exercise{$\ds f(x) = \sin x$\\
\begin{tikzpicture}[baseline=10pt]
\begin{axis}[width=\marginparwidth,tick label style={font=\scriptsize},
axis y line=middle,axis x line=middle,name=myplot,
ymin=-1.5,ymax=1.5,xmin=-.5,xmax=6.5]
\addplot [thick,draw={\colorone},smooth,domain=0:6.28] {sin(deg(x))};
\draw [draw={\colorone},fill=\colorone]    (axis cs: 1.57,1) node [above,black] {\scriptsize $(\pi/2,1)$} circle (1.5pt);
\draw [draw={\colorone},fill=\colorone]    (axis cs: 4.71,-1) node [below,black] {\scriptsize $(3\pi/2,-1)$} circle (1.5pt);
\end{axis}
\node [right] at (myplot.right of origin) {\scriptsize $x$};
\node [above] at (myplot.above origin) {\scriptsize $y$};
\end{tikzpicture}}{$\fp(\pi/2) = 0$; $\fp(3\pi/2) = 0$}

\exercise{$\ds f(x) = x^2\sqrt{4-x}$\\
\begin{tikzpicture}[baseline=10pt]
\begin{axis}[width=\marginparwidth,tick label style={font=\scriptsize},
axis y line=middle,axis x line=middle,name=myplot,
ymin=-1.5,ymax=11,xmin=-2.5,xmax=4.5]
\addplot [thick,draw={\colorone},smooth,domain=-2:4] {x^2*sqrt(4-x)};
\draw [draw={\colorone},fill=\colorone]    (axis cs: 0,0) node [below right,black] {\scriptsize $(0,0)$} circle (1.5pt);
\draw [draw={\colorone},fill=\colorone]    (axis cs: 3.2,9.16) node [above,black] {\scriptsize $\left(\frac{16}{5},\frac{512}{25\sqrt{5}}\right)$} circle (1.5pt);
\draw [draw={\colorone},fill=\colorone]    (axis cs: 4,0) node [above left,black] {\scriptsize $(4,0)$} circle (1.5pt);
\end{axis}
\node [right] at (myplot.right of origin) {\scriptsize $x$};
\node [above] at (myplot.above origin) {\scriptsize $y$};
\end{tikzpicture}}{$\fp(0) = 0$; $\fp(3.2) = 0$; $\fp(4)$ is undefined}

\exercise{$\ds f(x) = \begin{cases} x^2 & x\leq 0 \\ x^5 & x> 0 \end{cases}$\\
\begin{tikzpicture}[baseline=10pt]
\begin{axis}[width=\marginparwidth,tick label style={font=\scriptsize},
axis y line=middle,axis x line=middle,name=myplot,
ymin=-.5,ymax=1.1,xmin=-1.1,xmax=1.1]
\addplot [thick,draw={\colorone},smooth,domain=-1:0] {x^2};
\addplot [thick,draw={\colorone},smooth,domain=0:1] {x^5};
\draw [draw={\colorone},fill=\colorone]    (axis cs: 0,0) node [below right,black] {\scriptsize $(0,0)$} circle (1.5pt);
\end{axis}
\node [right] at (myplot.right of origin) {\scriptsize $x$};
\node [above] at (myplot.above origin) {\scriptsize $y$};
\end{tikzpicture}}{$\fp(0) = 0$}

\exercise{$\ds f(x) = \begin{cases}x^2 & x\leq 0 \\ x & x> 0 \end{cases}$\\
\begin{tikzpicture}[baseline=10pt]
\begin{axis}[width=\marginparwidth,tick label style={font=\scriptsize},
axis y line=middle,axis x line=middle,name=myplot,
ymin=-.5,ymax=1.1,xmin=-1.1,xmax=1.1]
\addplot [thick,draw={\colorone},smooth,domain=-1:0] {x^2};
\addplot [thick,draw={\colorone},smooth,domain=0:1] {x};
\draw [draw={\colorone},fill=\colorone]    (axis cs: 0,0) node [below right,black] {\scriptsize $(0,0)$} circle (1.5pt);
\end{axis}
\node [right] at (myplot.right of origin) {\scriptsize $x$};
\node [above] at (myplot.above origin) {\scriptsize $y$};
\end{tikzpicture}}{$\fp(0)$ is not defined}

\exercise{$\ds f(x) = \frac{(x-2)^{2/3}}{x}+1$\\
\begin{tikzpicture}[baseline=10pt]
\begin{axis}[width=\marginparwidth,tick label style={font=\scriptsize},
axis y line=middle,axis x line=middle,name=myplot,
ymin=-.5,ymax=6.5,xmin=-1,xmax=11]
\addplot [thick,draw={\colorone},domain=.2:11,samples=100] {(((x-2)^2)^(1/3))/x+1};
\draw [draw={\colorone},fill=\colorone]    (axis cs: 2,1) node [below,black] {\scriptsize $(2,1)$} circle (1.5pt);
\draw [draw={\colorone},fill=\colorone]    (axis cs: 6,1.42) node [above,black] {\scriptsize $\left(6,1+\frac{\sqrt[3]{2}}{3}\right)$} circle (1.5pt);
\end{axis}
\node [right] at (myplot.right of origin) {\scriptsize $x$};
\node [above] at (myplot.above origin) {\scriptsize $y$};
\end{tikzpicture}}{$\fp(2)$ is not defined; $\fp(6) = 0$}

\exercise{$\ds f(x) = \sqrt[3]{x^4-2x^2+1}$\\
\begin{tikzpicture}[baseline=10pt]
\begin{axis}[width=\marginparwidth,tick label style={font=\scriptsize},
axis y line=middle,axis x line=middle,name=myplot,xtick={-2,-1,1,2},
ymin=-.5,ymax=3.5,xmin=-2.5,xmax=2.5]
\addplot [thick,{\colorone},domain=-2:-1,samples=25,smooth] {(x^4-2*x^2+1)^(1/3)};
\addplot [thick,{\colorone},domain=-1:1,samples=50,smooth] {(x^4-2*x^2+1)^(1/3)};
\addplot [thick,{\colorone},domain=1:2,samples=25,smooth] {(x^4-2*x^2+1)^(1/3)};
\draw [{\colorone},fill={\colorone}]    (axis cs: 1,0) node [above right,black] {\scriptsize $(1,0)$} circle (1.5pt);
\draw [{\colorone},fill={\colorone}]    (axis cs: -1,0) node [above left,black] {\scriptsize $(-1,0)$} circle (1.5pt);
\end{axis}
\node [right] at (myplot.right of origin) {\scriptsize $x$};
\node [above] at (myplot.above origin) {\scriptsize $y$};
\end{tikzpicture}}{Both $\fp(-1)$ and $\fp(1)$ are undefined.}

\end{exerciseset}
