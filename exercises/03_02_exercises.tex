\printconcepts

\exercise{Explain in your own words what the Mean Value Theorem states.}{Answers will vary.}

\exercise{Explain in your own words what Rolle's Theorem states.}{Answers will vary.}

\printproblems

\exerciseset{In Exercises}{, a function $f(x)$ and interval $[a,b]$ are given. Check if Rolle's Theorem can be applied to $f$ on $[a,b]$; if so, find $c$ in $(a,b)$ such that $\fp(c)=0$.}{

\exercise{$f(x) = 6$ on $[-1,1]$.}{Any $c$ in $(-1,1)$ is valid.}

\exercise{$f(x) = 6x$ on $[-1,1]$.}{Rolle's Thm. does not apply.}

\exercise{$f(x) = x^2+x-6$ on $[-3,2]$.}{$c=-1/2$}

\exercise{$f(x) = x^2+x-2$ on $[-3,2]$.}{$c=-1/2$}

\exercise{$f(x) = x^2+x$ on $[-2,2]$.}{Rolle's Thm. does not apply.}

\exercise{$f(x) = \sin x$ on $[\pi/6,5\pi/6]$.}{$c=\pi/2$}

\exercise{$f(x) = \cos x$ on $[0,\pi]$.}{Rolle's Thm. does not apply.}

\exercise{$\ds f(x) = \frac{1}{x^2-2x+1}$ on $[0,2]$.}{Rolle's Thm. does not apply.}

}


\exerciseset{In Exercises}{, a function $f(x)$ and interval $[a,b]$ are given. Check if the Mean Value Theorem can be applied to $f$ on $[a,b]$; if so, find a value $c$ in $[a,b]$ guaranteed by the Mean Value Theorem.}{

\exercise{$\ds f(x) = x^2+3x-1$ on $[-2,2]$.}{$c=0$}

\exercise{$\ds f(x) = 5x^2-6x+8$ on $[0,5]$.}{$c=5/2$}

\exercise{$\ds f(x) = \sqrt{9-x^2}$ on $[0,3]$.}{$c=3/\sqrt{2}$}

\exercise{$\ds f(x) = \sqrt{25-x}$ on $[0,9]$.}{$c=19/4$}

\exercise{$\ds f(x) =\frac{x^2-9}{x^2-1}$ on $[0,2]$.}{The Mean Value Theorem does not apply.}

\exercise{$\ds f(x) = \ln x$ on $[1,5]$.}{$c=4/\ln 5$}

\exercise{$\ds f(x) = \tan x$ on $[-\pi/4,\pi/4]$.}{$c=\pm \sec^{-1} (2/\sqrt{\pi})$}

\exercise{$\ds f(x) = x^3-2x^2+x+1$ on $[-2,2]$.}{$c=-2/3$}

\exercise{$\ds f(x) = 2x^3-5x^2+6x+1$ on $[-5,2]$.}{$c=\frac{5\pm7\sqrt{7}}{6}$}

%\exercise{$\ds f(x) = \sin^{-1} x$ on $[-1,1]$.}{$c=\frac{\pm\sqrt{\pi^2-4}}{\pi}$}

}


\exercise{Suppose that $f$ is continuous on $[1,4]$ and differentiable on $(1,4)$. If $f(1)=10$ and $\fp(x)\ge2$ for $1\le x\le 4$, how small can $f(4)$ possibly be?}{With $c$ given by the Mean Value Theorem, $f(4)=f(1)+\fp(c)(4-1)=10+3\fp(c)\ge 16$.}

\exercise{Does there exist a function $f$ such that $f(0)=-1$, $f(2)=4$, and $\fp(x)\le2$ for all $x$?}{No.  Otherwise, with $c$ given by the Mean Value Theorem, $\dfrac{4--1}{2-0}=\fp(c)\le2$, a contradiction.}

\exercise{Show that the equation $1+2x+x^3+4x^5=0$ has exactly one real root.}{$f(-1)<0<f(0)$, so it has at least one root. $\fp=2+3x^2+20x^4\ge2$, so more than one root would contradict Rolle's Theorem.}

\exercise{Show that a polynomial of degree 3 has at most 3 real roots.}{If $f$ has more than 3 real roots, then Rolle's Theorem implies $\fp$ is a quadratic with more than $2$ real roots.}

\exercise{\begin{enumerate}
\item Suppose that $f$ is differentiable everywhere and has 2 roots.  Show that $\fp$ has at least one real root.
\item Suppose that $f$ is twice differentiable everywhere and has 3 roots.  Show that $\fpp$ has at least one real root.
\end{enumerate}}{(a) is Rolle's Theorem.  For (b), applying Rolle's Theorem to roots 1 and 2 and roots 2 and 3 shows that $\fp$ has two roots, and we can then apply (a).}

\exercise{Let $p$, $q$, and $r$ be constants, and define $f(x)=px^2+qx+r$.  Show that the Mean Value Theorem applied to $f$ for the interval $[a,b]$ is always satisfied at the midpoint of the interval.}{$2pc+q=\fp(c)=\frac{f(b)-f(a)}{b-a}=\frac{pb^2+qb+r-pa^2-qa-r}{b-a}=\frac{p(b^2-a^2)+q(b-a)}{b-a}=p(b+a)+q$ implies that $c=\frac{a+b}2$.}

\printreview

\exercise{Find the extreme values of $f(x) =x^2-3x+9$ on $[-2,5]$.}{Max value of 19 at $x=-2$ and $x=5$; min value of 6.75 at $x=1.5$.}

\exercise{Describe the critical points of $f(x) = \cos x$.}{They are the odd, integer valued multiples of $\pi/2$ (such as $0,\pm\pi/2, \pm 3\pi/2, \pm5\pi/2$, etc.)}

\exercise{Describe the critical points of $f(x) = \tan x$.}{They are the odd, integer valued multiples of $\pi/2$ (such as $0, \pm\pi/2, \pm 3\pi/2, \pm5\pi/2$, etc.)}
