\printconcepts

\exercise{T/F: Given a function $f(x)$, Newton's Method produces an exact solution to $f(x) = 0$.}{F}

\exercise{T/F: In order to get a solution to $f(x)=0$ accurate to $d$ places after the decimal, at least $d+1$ iterations of Newton's Method must be used.}{F}

\printproblems

\input{exercises/04-01-exset-01}

\input{exercises/04-01-exset-02}

\exercisesetinstructions{, use Newton's Method to approximate when the given functions are equal, accurate to 3 places after the decimal. Use technology to obtain good initial approximations.}

\exercise{$f(x) = x^2$, $g(x) = \cos x$}{$x=\pm 0.824$,}

\exercise{$f(x) = x^2-1$, $g(x) = \sin x$}{$x=-0.637$, $x=1.410$}

\exercise{$f(x) = e^{x^2}$, $g(x) = \cos x+1$}{$x=\pm 0.743$}

\exercise{$f(x) = x$, $g(x) = \tan x$ on $[-6,6]$}{$x=\pm 4.493$, $x=0$}

\exercisesetend


\exercise{Why does Newton's Method fail in finding a root of $f(x) = x^3-3x^2+x+3$ when $x_0=1$?}{The approximations alternate between $x=1$ and $x=2$.}

\exercise{Why does Newton's Method fail in finding a root of $f(x) = -17x^4+130x^3-301x^2+156x+156$ when $x_0=1$?}{The approximations alternate between $x=1$, $x=2$  and $x=3$.}

\exercisesetinstructions{, use Newton's Method to approximate the given value.}

\exercise{$\sqrt{16.5}$.}{$f(x)=x^2-16.5$ and $x_0=4$ yield $x_1=\frac{65}{16}=4.0625$ and $x_2=\frac{8449}{2080}\approx4.0620192$.}

\exercise{$\sqrt{24}$.}{$f(x)=x^2-24$ and $x_0=5$ yield $x_1=\frac{49}{10}=4.9$ and $x_2=\frac{4801/980}\approx4.898980$.}

\exercise{$\sqrt[3]{63}$.}{$f(x)=x^3-63$ and $x_0=4$ yield $x_1=\frac{191}{48}\approx3.97916667$ and $x_2\approx3.9790572$.}

\exercise{$\sqrt[3]{8.5}$.}{$f(x)=x^3-8.5$ and $x_0=2$ yield $x_1=\frac{49}{24}\approx2.0416667$ and $x_2\approx2.0408279$.}

\exercisesetend


\exercise{Show graphically what happens when Newton's Method is used at different $x_0$ for the function shown.\\
\begin{minipage}[t]{.25\linewidth}
\begin{enumerate}
\item $x_0=0$
\item $x_0=1$
\item $x_0=3$
\item $x_0=4$
\item $x_0=5$
\end{enumerate}
\end{minipage}\hfill\begin{minipage}[t]{.65\linewidth}
\begin{tikzpicture}[baseline=(current bounding box.north)]
\begin{axis}[width=\marginparwidth, tick label style={font=\scriptsize},
			minor x tick num=1, axis y line=middle, axis x line=middle, ymin=-2,
			ymax=2, xmin=-2, xmax=7, name=myplot]
\addplot [draw={\colorone},thick,smooth] coordinates {(-2,.2)(-1,.5)(0,1)(1,1.5)(2,0)(4,-1)(6,0)(7,2)};
%\draw [draw={\colorone},thick] (axis cs:0,1) parabola [bend at end] (axis cs:1,1.5);
%\draw [draw={\colorone},thick] (axis cs:1,1.5) parabola (axis cs:2,0);
%\draw [draw={\colorone},thick] (axis cs:2,0) parabola [bend at end] (axis cs:4,-3);
%\draw [draw={\colorone},thick] (axis cs:4,-3) parabola (axis cs:7,3.75);
%\filldraw [black] (axis cs:2,86) circle (1pt);
%\filldraw [black] (axis cs:3,6) circle (1pt);
\end{axis}
\node [right] at (myplot.right of origin) {\scriptsize $x$};
\node [above] at (myplot.above origin) {\scriptsize $y$};
\end{tikzpicture}
\end{minipage}}{\mbox{}\\[-2\baselineskip]\parbox[t]{\linewidth}{\begin{enumerate}
\item $x_n\to-\infty$
\item $x_1$ is undefined
\item $x_n\to2$
\item $x_1$ is undefined
\item $x_n\to6$
\end{enumerate}}}

\exercise{If we need to calculate $c^{-1/2}$ quickly (for example, in doing computer graphics), one possible approach is to use Newton's Method. Show that $c^{-1/2}$ is a root of $f(x)=x^{-2}-c$.  According to Newton's Method, what is $x_{n+1}$ in terms of $x_n$ and $c$ for this $f$? (You can read the Wikipedia article on \href{https://en.wikipedia.org/wiki/Fast_inverse_square_root}{Fast Inverse Square Root} for even more details.)}{Substituting, we find that $f(c^{-1/2})=(c^{-1/2})^{-2}-c=c-c=0$, so that $c^{-1/2}$ is a root of $f$. Since $f'(x)=-2x^{-3}$, Newton's Method shows that
 \[
  x_{n+1}=x_n-\frac{x_n^{-2}-c}{-2x^{-3}}
  =x_n+\frac{x_n-x_n^3c}2=x_n\left(\frac32-x_n^2\frac c2\right).
 \]}

%\printreview

%\exercise{Consider $f(x) = x^2-3x+5$ on $[-1,2]$; find $c$ guaranteed by the Mean Value Theorem.}{$c=1/2$}

%\exercise{Consider $f(x) = \sin x$ on $[-\pi/2,\pi/2]$; find $c$ guaranteed by the Mean Value Theorem.}{$c=\pm \cos^{-1}(2/\pi)$}
