\printconcepts

\exercise{T/F: Implicit differentiation is often used when solving ``related rates'' type problems.}{T}

\exercise{T/F: A study of related rates is part of the standard police officer training.}{F}

\printproblems

\exercise{The area of a square is increasing at a rate of 42 ft\textsuperscript{2}/min. How fast is the side length increasing when the length is 7 ft?}{3 ft/min}
% A=s^2 ; A'=2ss' ; 42=2*7*s' ; 6=2s' ; s'=3

\exercise{Water flows onto a flat surface at a rate of 5cm$^3$/s forming a circular puddle 10mm deep. How fast is the radius growing when the radius is:
	\begin{enumerate}
	\item		1 cm?
	\item		10 cm?
	\item		100 cm?
	\end{enumerate}}{\mbox{}\\[-2\baselineskip]\parbox[t]{\linewidth}{\begin{enumerate}
\item	$5/(2\pi) \approx 0.796$cm/s
\item $1/(4\pi)\approx 0.0796$ cm/s
\item $1/(40\pi)\approx 0.00796$ cm/s
\end{enumerate}}}

\exercise{A spherical balloon is inflated with air flowing at a rate of 10cm$^3$/s. How fast is the radius of the balloon increasing when the radius is:
	\begin{enumerate}
	\item		1 cm?
	\item		10 cm?
	\item		100 cm?
	\end{enumerate}}{\mbox{}\\[-2\baselineskip]\parbox[t]{\linewidth}{\begin{enumerate}
\item	$5/(2\pi)\approx 0.796$cm/s
\item $1/(40\pi)\approx 0.00796$ cm/s
\item $1/(4000\pi)\approx 0.0000796$ cm/s
\end{enumerate}}}

\exercise{Consider the traffic situation introduced in \autoref{ex_rr3}. How fast is the ``other car'' traveling if the officer and the other car are each 1/2 mile from the intersection, the other car is traveling \textit{due west}, the officer is traveling north at 50mph, and the radar reading is $-80$mph?}{$63.14$mph}

\exercise{Consider the traffic situation introduced in \autoref{ex_rr3}. Calculate how fast the ``other car'' is traveling in each of the following situations.
	\begin{enumerate}
	\item The officer is traveling due north at 50mph and is 1/2 mile from the intersection, while the other car is 1 mile from the intersection traveling west and the radar reading is $-80$mph?
	\item The officer is traveling due north at 50mph and is 1 mile from the intersection, while the other car is 1/2 mile from the intersection traveling west and the radar reading is $-80$mph?
	\end{enumerate}}{\mbox{}\\[-2\baselineskip]\parbox[t]{\linewidth}{\begin{enumerate}
	\item $64.44$ mph
	\item	$78.89$ mph
\end{enumerate}}}

\exercise{An\label{exer:04_02_ex_07} F-22 aircraft is flying at 500mph with an elevation of 10,000ft on a straight-line path that will take it directly over an anti-aircraft gun. 

\noindent\begin{minipage}{\linewidth}
\centering
\begin{tikzpicture}[alt={ALT-TEXT-TO-BE-DETERMINED},>=latex]
 \begin{scope}
  \clip (0,0) rectangle (120pt,50pt);
  \draw [inner color={\colorone},draw = white] (-10,-25pt) rectangle (180pt,70pt);
 \end{scope}
 \draw [top color=brown, bottom color=white,draw=white] (0,0) rectangle (120pt,-15pt);
 \draw (0,0) -- (120pt,0);
 \draw [dashed] (12.5pt,0pt) node [xshift=22pt,yshift=4pt] {\scriptsize $\theta$} -- (105pt,35pt);
 \draw [ultra thick] (12.5pt,0) -- (18pt,4pt);
 \filldraw [fill=white] (10pt,-2.5pt) rectangle (15pt,2.5pt);
 \draw [<->] (15pt,-10pt) -- (110pt,-10pt) node [below,pos=.5] {\scriptsize $x$};
 \draw [<->] (120pt,0pt) -- (120pt,35pt) node [right, pos=.5] {\scriptsize 10,000 ft};
 \begin{scope}[shift={(105pt,35pt)}]
  \draw [fill=black] (0,0) rectangle (10pt,2pt);
  \draw [very thick] (3pt,1pt) -- (6pt,-1.5pt) -- (5pt,0pt);
  \draw [very thick] (9pt,2pt) -- (9pt,4pt) -- (7pt,2pt);
 \end{scope}
\end{tikzpicture}
\end{minipage}

How fast must the gun be able to turn to accurately track the aircraft when the plane is:
\begin{enumerate}
\item	1 mile away?
\item	1/5 mile away?
\item	Directly overhead?
\end{enumerate}}{Due to the height of the plane, the gun does not have to rotate very fast.
\begin{enumerate}
\item  $0.0573$ rad/s
\item	$0.0725$ rad/s
\item In the limit, rate goes to $0.0733$ rad/s
\end{enumerate}}

%5280/5

\exercise{An F-22 aircraft is flying at 500mph with an elevation of  100ft on a straight-line path that will take it directly over an anti-aircraft gun as in \autoref{exer:04_02_ex_07} (note the lower elevation here).

How fast must the gun be able to turn to accurately track the aircraft when the plane is:
\begin{enumerate}
\item	1000 feet away?
\item	100 feet away?
\item	Directly overhead?
\end{enumerate}}{Due to the height of the plane, the gun does not have to rotate very fast.
\begin{enumerate}
\item  $0.073$ rad/s
\item	$3.66$ rad/s (about 1/2 revolution/sec)
\item In the limit, rate goes to $7.33$ rad/s (more than 1 revolution/sec)
\end{enumerate}}

\exercise{A 24ft. ladder is leaning against a house while the base is pulled away at a constant rate of 1ft/s.

\noindent\begin{minipage}{\linewidth}
\centering
\begin{tikzpicture}[alt={ALT-TEXT-TO-BE-DETERMINED},>=latex]
\draw [top color=brown, bottom color=brown!50!white,draw=black] (0,0) rectangle (20pt,30pt);
\draw [fill=black] (0,30pt) -- (10pt,38pt) -- (20pt,30pt) -- cycle;
\draw [fill=white] (7pt,15pt) rectangle (13pt,25pt);
\draw (10pt,15pt) -- (10pt,25pt) (7pt,20pt) -- (13pt,20pt);
\draw [thick] (20pt,25pt) -- node [pos=.5,rotate=-67,yshift=5pt] {\scriptsize 24 ft} (30pt,0);
\draw [thick] (0,0) -- (35pt,0);
\draw [->,>=latex] (34pt,2pt) -- node [above right,pos=.5] {\scriptsize 1 ft/s} (50pt,2pt);
\end{tikzpicture}
\end{minipage}

At what rate is the top of the ladder sliding down the side of the house when the base is:
\begin{enumerate}
\item	1 foot from the house?
\item	10 feet from the house?
\item	23 feet from the house?
\item	24 feet from the house?
\end{enumerate}}{\mbox{}\\[-2\baselineskip]\parbox[t]{\linewidth}{\begin{enumerate}
\item  $0.04$ ft/s
\item	 $0.458$ ft/s
\item  $3.35$ ft/s
\item	Not defined; as the distance approaches 24, the rates approaches $\infty$.	
\end{enumerate}}}

\exercise{A boat is being pulled into a dock at a constant rate of 30ft/min by a winch located 10ft above the deck of the boat.

\noindent\begin{minipage}{\linewidth}
\centering
\begin{tikzpicture}[alt={ALT-TEXT-TO-BE-DETERMINED},>=latex]
\draw [top color=blue!60!white, bottom color=blue!10!white,draw=blue] (-5pt,5pt) rectangle (100pt,-5pt);
\draw [thick,fill=brown] (-10pt,20pt) rectangle (8pt,22pt);
\draw [top color=brown!30!black, bottom color=brown!70!white,draw=black] (0,0) rectangle (5pt,30pt);
\draw [fill=white] (50pt,0pt) -- (90pt,0pt) -- (90pt, 15pt) -- (45pt,15pt) -- cycle;
\draw [fill=black] (65pt,15pt) rectangle (67pt,20pt);
\draw [fill=yellow] (55pt,20pt) -- (80pt,20pt) -- (65pt,40pt) -- cycle;
\draw [fill=black] (8pt,35pt) circle (2pt);
\draw (45pt,15pt) [dashed] -- (8pt,15pt) -- (8pt,35pt) node [pos=.4,xshift=7pt] {\scriptsize 10ft};
\draw [,>=latex,thick] (8pt,35pt) -- (45pt,15pt); 
\end{tikzpicture}
\end{minipage}

At what rate is the boat approaching the dock when the boat is:
\begin{enumerate}
\item	50 feet out?
\item	15 feet out?
\item	1 foot from the dock?
\item	What happens when the length of rope pulling in the boat is less than 10 feet long?
\end{enumerate}}{\mbox{}\\[-2\baselineskip]\parbox[t]{\linewidth}{\begin{enumerate}
\item  $30.59$ ft/min
\item	 $36.1$ ft/min
\item  $301$ ft/min
\item	 The boat no longer floats as usual, but is being pulled up by the winch (assuming it has the power to do so).	
\end{enumerate}}}

\exercise{An inverted cylindrical cone, 20ft deep and 10ft across at the top, is being filled with water at a rate of 10ft$^3$/min. At what rate is the water rising in the tank when the depth of the water is:
\begin{enumerate}
\item	1 foot?
\item	10 feet?
\item	19 feet?
\end{enumerate}
How long will the tank take to fill when starting at empty?}{\mbox{}\\[-2\baselineskip]\parbox[t]{\linewidth}{\begin{enumerate}
\item  $50.92$ ft/min
\item	 $0.509$ ft/min
\item  $0.141$ ft/min
\end{enumerate}}
As the tank holds about 523.6ft$^3$, it will take about 52.36 minutes.}

\exercise{A\label{exer:04_02_ex_12} rope, attached to a weight, goes up through a pulley at the ceiling and back down to a worker. The man holds the rope at the same height as the connection point between rope and weight.

\noindent\begin{minipage}{\linewidth}
\centering
\begin{tikzpicture}[alt={ALT-TEXT-TO-BE-DETERMINED},>=latex,x=1pt,y=1pt,scale=1.25]
\draw [thick](0,40) -- (55,40); % top line
\draw [thick] (0,0) -- (55,0); % bottom line
\draw (2,0) -- (12,0) -- (9,5) -- (5,5)--cycle; % weight
\draw (7,5) -- (7,35) -- (33,5); % rope
\draw (7,35) -- (7,40); % pulley rope
\draw [fill=gray] (7,35) circle (2pt); % pulley
\node [scale=.13] at (35,6) {\psBill};
%\draw (35,10) circle (2pt); % head
%\draw (35,8) -- (35,4) -- (33,0) (35,4) -- (37,0) (33,6) -- (37,6); % body, left, right, arms
\draw (-4,5) -- (-1,5)  (-2.5,5) -- (-2.5,35) node [pos=.5,draw,fill=white,draw=white,rotate=90] {\scriptsize 30 ft}
       (-4,35) -- (-1,35);
\draw [thick,->](40,5) --(55,5) node[pos=.5,above] {\scriptsize 2 ft/s};
\end{tikzpicture}
\end{minipage}

Suppose the man stands directly next to the weight (i.e., a total rope length of 60 ft) and begins to walk away at a rate of 2ft/s. How fast is the weight rising when the man has walked:
\begin{enumerate}
\item	10 feet?
\item	40 feet?
\end{enumerate}
How far must the man walk to raise the weight all the way to the pulley?}{\mbox{}\\[-2\baselineskip]\parbox[t]{\linewidth}{\begin{enumerate}
\item  $0.63$ ft/sec
\item	 $1.6$ ft/sec
\end{enumerate}}
About 52 ft.}

\exercise{Consider the situation described in \autoref{exer:04_02_ex_12}. Suppose the man starts 40ft from the weight and begins to walk away at a rate of 2ft/s. 
\begin{enumerate}
\item	How long is the rope?
\item	How fast is the weight rising after the man has walked 10 feet?
\item	How fast is the weight rising after the man has walked 30 feet?
\item	How far must the man walk to raise the weight all the way to the pulley?
\end{enumerate}}{\mbox{}\\[-2\baselineskip]\parbox[t]{\linewidth}{\begin{enumerate}
\item The rope is 80ft long.
\item $1.71$ ft/sec
\item $1.84$ ft/sec
\item About 34 feet.
\end{enumerate}}}

\exercise{A hot air balloon lifts off from ground rising vertically. From 100 feet away, a 5' woman tracks the path of the balloon. When her sightline with the balloon makes a 45$^\circ$ angle with the horizontal, she notes the angle is increasing at about 5$^\circ$/min. 
\begin{enumerate}
\item		What is the elevation of the balloon?
\item		How fast is it rising?
\end{enumerate}}{\mbox{}\\[-2\baselineskip]\parbox[t]{\linewidth}{\begin{enumerate}
\item		The balloon is 105ft in the air.
\item  The balloon is rising at a rate of 17.45ft/min. (Hint: convert all angles to radians.)
\end{enumerate}}}

\exercise{A company that produces landscaping materials is dumping sand into a conical pile. The sand is being poured at a rate of 5ft$^3$/sec; the physical properties of the sand, in conjunction with gravity, ensure that the cone's height is roughly 2/3 the length of the diameter of the circular base. 

How fast is the cone rising when it has a height of 30 feet?}{The cone is rising at a rate of $0.003$ft/s.}

%\printreview

%\exercise{Consider $f(x) = x^2-3x+5$ on $[-1,2]$; find $c$ guaranteed by the Mean Value Theorem.}{$c=1/2$}

%\exercise{Consider $f(x) = \sin x$ on $[-\pi/2,\pi/2]$; find $c$ guaranteed by the Mean Value Theorem.}{$c=\pm \cos^{-1}(2/\pi)$}
