\exercisesetinstructions[Exercises]{ explore some issues related to surveying in which distances are approximated using other measured distances and measured angles. \textit{(Hint: Convert all angles to radians before computing.)}}

\exercise{The\label{exer:04_04_ex_35} length $l$ of a long wall is to be approximated. The angle $\theta$, as shown in the diagram (not to scale), is measured to be $85.2^\circ$, accurate to $1^\circ$. Assume that the triangle formed is a right triangle.\\
\begin{minipage}{\linewidth}
\centering
\begin{tikzpicture}[alt={A right triangle with base 25 feet and adjacent angle 𝜃.  The unknown height is labeled 𝑙.}]
\draw [ultra thick] (1,-1) -- node [pos=.5,right] {\scriptsize $l=$?}(1,1);
\draw [dashed] (1,1) -- (-1,-1) node [xshift=10pt,yshift=5pt] {\scriptsize $\theta$} -- node [pos=.5,below] {\scriptsize $25'$} (1,-1);
%\draw (-.5,0) -- node [pos=.5,draw=white,fill=white] {\scriptsize $50'$} (1,0);
\end{tikzpicture}
\end{minipage}
\begin{enumext}
\item		What is the measured length $l$ of the wall?
\item		What is the propagated error? 
\item		What is the percent error?
\end{enumext}}{\mbox{}\\[-2\baselineskip]\parbox[t]{\linewidth}{\begin{enumext}
\item		297.8 feet
\item		$\pm 62.3$ ft
\item		$\pm 20.9$\% 
\end{enumext}}}

\exercise{Answer\label{exer:04_04_ex_36} the questions of \autoref{exer:04_04_ex_35}, but with a measured angle of $71.5^\circ$, accurate to $1^\circ$, measured from a point $100'$ from the wall.}{\mbox{}\\[-2\baselineskip]\parbox[t]{\linewidth}{\begin{enumext}
\item		298.8 feet
\item		$\pm 17.3$ ft
\item		$\pm 5.8$\% 
\end{enumext}}}

\exercise{The\label{exer:04_04_ex_34} length $l$ of a long wall is to be calculated by measuring the angle $\theta$ shown in the diagram (not to scale). Assume the formed triangle is an isosceles triangle. The measured angle is $143^\circ$, accurate to $1^\circ$.\\
\begin{minipage}{\linewidth}
\centering
\begin{tikzpicture}[alt={A triangle with unknown side length 𝑙.  50 feet away is the angle 𝜃.}]
\draw [ultra thick] (1,-1) -- node [pos=.5,right] {\scriptsize $l=$?}(1,1);
\draw [dashed] (1,1) -- (-1,0) node [xshift=10pt] {\scriptsize $\theta$} -- (1,-1);
\draw (-.5,0) -- node [pos=.5,draw=white,fill=white] {\scriptsize $50'$} (1,0);
\end{tikzpicture}
\end{minipage}
\begin{enumext}
\item		What is the measured length of the wall?
\item		What is the propagated error? 
\item		What is the percent error?
%\item		What is a key assumption about the location where the angle is measured? 
\end{enumext}}{\mbox{}\\[-2\baselineskip]\parbox[t]{\linewidth}{\begin{enumext}
\item		298.9 feet
\item		$\pm 8.67$ ft
\item		$\pm 2.9$\%
%\item		Is is assumed that the location is halfway along the wall. 
\end{enumext}}}

% this requires all three problems to be assigned.  also cut for parity
%\exercise{The length of the walls in Exercises~\ref{exer:04_04_ex_35}--\ref{exer:04_04_ex_34} are essentially the same. Which setup gives the most accurate result?}{The isosceles triangle setup works the best with the smallest percent error.}

\exercise{Consider the setup in \autoref{exer:04_04_ex_34}. This time, assume the angle measurement of $143^\circ$ is exact but the measured $50'$ from the wall is accurate to $6''$. What is the approximate percent error?}{1\%}

\exercisesetend
