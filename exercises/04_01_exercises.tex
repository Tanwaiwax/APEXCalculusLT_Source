\printconcepts

\exercise{T/F: Given a function $f(x)$, Newton's Method produces an exact solution to $f(x) = 0$.}{F}

\exercise{T/F: In order to get a solution to $f(x)=0$ accurate to $d$ places after the decimal, at least $d+1$ iterations of Newton's Method must be used.}{F}

\printproblems

\begin{exerciseset}{In Exercises}{, the roots of $f(x)$ are known or are easily found. Use 5 iterations of Newton's Method with the given initial approximation to approximate the root. Compare it to the known value of the root.}

\exercise{$f(x) = \cos x$, $x_0=1.5$}{$x_0=1.5$, $x_1=1.5709148$, $x_2=1.5707963$, $x_3=1.5707963$, $x_4=1.5707963$, $x_5=1.5707963$}

\exercise{$f(x) = \sin x$, $x_0=1$}{$x_0=1$, $x_1=-0.55740772$, $x_2=0.065936452$, $x_3=-0.000095721919$, $x_4=2.9235662*10^{-13}$, $x_5=0$}

\exercise{$f(x) = x^2+x-2$, $x_0=0$}{$x_0=0$, $x_1=2$, $x_2=1.2$, $x_3=1.0117647$, $x_4=1.0000458$, $x_5=1$}

\exercise{$f(x) = x^2-2$, $x_0=1.5$}{$x_0=1.5$, $x_1=1.4166667$, $x_2=1.4142157$, $x_3=1.4142136$, $x_4=1.4142136$, $x_5=1.4142136$}

\exercise{$f(x) = \ln x$, $x_0=2$}{$x_0=2$, $x_1=0.6137056389$, $x_2=0.9133412072$, $x_3=0.9961317034$, $x_4=0.9999925085$, $x_5=1$}

\exercise{$f(x) = x^3-x^2+x-1$, $x_0=1$}{$x_0=1$, $x_1=1$, $x_2=1$, $x_3=1$, $x_4=1$, $x_5=1$}

\end{exerciseset}


\begin{exerciseset}{In Exercises}{, use Newton's Method to approximate all roots of the given functions accurate to 3 places after the decimal. If an interval is given, find only the roots that lie in that interval. Use technology to obtain good initial approximations.}

\exercise{$f(x) = x^3+5x^2-x-1$}{roots are: $x=-5.156$, $x=-0.369$ and $x=0.525$}

\exercise{$f(x) = x^4+2x^3-7x^2-x+5$}{roots are: $x=-3.714$, $x=-0.857$, $x=1$ and $x=1.571$}

\exercise{$f(x) = x^{17}-2x^{13}-10x^8+10$ on $(-2,2)$}{roots are: $x=-1.013$, $x=0.988$, and $x=1.393$}

\exercise{$f(x) = x^2\cos x + (x-1)\sin x$ on $(-3,3)$}{roots are: $x=-2.165$, $x=0$, $x=0.525$ and $x=1.813$}

\end{exerciseset}


\input{exercises/04_01_exset_03}

\exercise{Why does Newton's Method fail in finding a root of $f(x) = x^3-3x^2+x+3$ when $x_0=1$?}{The approximations alternate between $x=1$ and $x=2$.}

\exercise{Why does Newton's Method fail in finding a root of $f(x) = -17x^4+130x^3-301x^2+156x+156$ when $x_0=1$?}{The approximations alternate between $x=1$, $x=2$  and $x=3$.}

{\noindent In Exercises}
{, use Newton's Method to approximate the given value.
}
\exinput{exercises/04_01_ex_19}
\exinput{exercises/04_01_ex_20}
\exinput{exercises/04_01_ex_21}
\exinput{exercises/04_01_ex_22}


\exercise{Show graphically what happens when Newton's Method is used at different $x_0$ for the function shown.
\iflatexml\begin{enumerate}\else\begin{enumerate*}\fi
\item $x_0=0$
\item $x_0=1$
\item $x_0=3$
\item $x_0=4$
\item $x_0=5$
\iflatexml\end{enumerate}\else\end{enumerate*}\fi

\begin{tikzpicture}
\begin{axis}[width=\marginparwidth, tick label style={font=\scriptsize},
			minor x tick num=1, axis y line=middle, axis x line=middle, ymin=-2,
			ymax=2, xmin=-2, xmax=7, name=myplot]
\addplot [draw={\colorone},thick,smooth] coordinates {(-2,.2)(-1,.5)(0,1)(1,1.5)(2,0)(4,-1)(6,0)(7,2)};
%\draw [draw={\colorone},thick] (axis cs:0,1) parabola [bend at end] (axis cs:1,1.5);
%\draw [draw={\colorone},thick] (axis cs:1,1.5) parabola (axis cs:2,0);
%\draw [draw={\colorone},thick] (axis cs:2,0) parabola [bend at end] (axis cs:4,-3);
%\draw [draw={\colorone},thick] (axis cs:4,-3) parabola (axis cs:7,3.75);
%\filldraw [black] (axis cs:2,86) circle (1pt);
%\filldraw [black] (axis cs:3,6) circle (1pt);
\end{axis}

\node [right] at (myplot.right of origin) {\scriptsize $x$};
\node [above] at (myplot.above origin) {\scriptsize $y$};
\end{tikzpicture}}{\mbox{}\\[-2\baselineskip]\begin{enumerate}
\item $x_n\to-\infty$
\item $x_1$ is undefined
\item $x_n\to2$
\item $x_1$ is undefined
\item $x_n\to6$
\end{enumerate}}

%\printreview

%\exercise{Consider $f(x) = x^2-3x+5$ on $[-1,2]$; find $c$ guaranteed by the Mean Value Theorem.}{$c=1/2$}

%\exercise{Consider $f(x) = \sin x$ on $[-\pi/2,\pi/2]$; find $c$ guaranteed by the Mean Value Theorem.}{$c=\pm \cos^{-1}(2/\pi)$}
