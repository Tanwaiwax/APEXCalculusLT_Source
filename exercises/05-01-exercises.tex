\printconcepts

\exercise{Define the term ``antiderivative'' in your own words.}{Answers will vary.}

\exercise{Is it more accurate to refer to ``the'' antiderivative of $f(x)$ or ``an'' antiderivative of $f(x)$?}{``an''}

\exercise{Use your own words to define the indefinite integral of $f(x)$.}{Answers will vary.}

\exercise{Fill in the blanks: ``Inverse operations do the \underline{\hskip .5in} things in the \underline{\hskip .5in} order.''}{opposite; opposite}

\exercise{What is an ``initial value problem''?}{Answers will vary.}

\exercise{The derivative of a position function is a \underline{\hskip .5in} function.}{velocity}

\exercise{\raggedright The antiderivative of an acceleration function is a \underline{\hskip .5in} function.}{velocity}

\exercise{If $F(x)$ is an antiderivative of $f(x)$, and $G(x)$ is an antiderivative of $g(x)$, give an antiderivative of $f(x)+g(x)$.}{$F(x)+G(x)$}

\printproblems

\input{exercises/05-01-exset-01}

\exercisesetinstructions{, find $f(x)$ described by the given initial value problem.}

\exercise{$\fp(x) = \sin x$ and $f(0)= 2$}{$-\cos x+3$}

\exercise{$\fp(x) = 5e^x$ and $f(0)= 10$}{$5e^x+5$}

\exercise{$\fp(x) = 4x^3-3x^2$ and $f(-1)= 9$}{$x^4-x^3+7$}

\exercise{$\fp(x) = \sec^2 x$ and $f(\pi/4)= 5$}{$\tan x + 4$}

%\exercise{$\fp(x) = 7^x$ and $f(2)= 1$}{$7^x/\ln 7 + 1-49/\ln 7$}

% cut for parity
%\exercise{$\fpp(x) = 5$ and $\fp(0)= 7$, $f(0) = 3$}{$5x^2/2+7x+3$}

\exercise{$\fpp(x) = 7x$ and $\fp(1)= -1$, $f(1) = 10$}{$\frac{7 x^3}{6}-\frac{9 x}{2}+\frac{40}{3}$}

\exercise{$\fpp(x) = 5e^x$ and $\fp(0)= 3$, $f(0) = 5$}{$5e^x-2x$}

\exercise{$\fpp(\theta) = \sin \theta$ and $\fp(\pi)= 2$, $f(\pi) = 4$}{$\theta-\sin (\theta)-\pi +4$}

%\exercise{$\fpp(x) = 24x^2+2^x-\cos x$ and $\fp(0)= 5$, $f(0) = 0$}{$\frac{2 x^4 \ln ^2(2)+2^x+x \ln 2(\ln 32-1)+\ln^2(2) \cos (x)-1-\ln ^2(2)}{\ln ^2(2)}$}

\exercise{$\fpp(x) = 0$ and $\fp(1)= 3$, $f(1) = 1$}{$3x-2$}

\exercise{$\fp(x)=\dfrac{-2}{x^3}$ and $f(1)= 2$}{$x^{-2}+1$}

\exercise{$\fp(x)=\dfrac1{\sqrt x}$ and $f(4)= 0$}{$2\sqrt x-4$}

\exercisesetend


\exercise{This problem investigates why \autoref{thm:indef_alg} states that $\ds \int \frac1x\dd x = \ln\abs x + C$.
\begin{enumext}
	\item		What is the domain of $y = \ln x$?
	\item		Find $\frac{\dd}{\dd x}\big(\ln x\big)$. 
	\item		What is the domain of $y = \ln (-x)$?
	\item		Find $\frac{\dd}{\dd x}\big(\ln (-x)\big)$. 
	\item		You should find that $1/x$ has two types of antiderivatives, depending on whether $x>0$ or $x<0$. In one expression, give a formula for $\ds \int \frac{1}{x}\dd x$ that takes these different domains into account, and explain your answer.
	\end{enumext}
}{\begin{enumext}
\item		$x>0$
\item		$1/x$
\item		$x<0$
\item		$1/x$
\item		$\ln \abs x+C$. Explanations will vary.
\end{enumext}}

\exercise{An object is moving so that its velocity at time $t$ is given by $v(t)=3\sqrt t$. If the object was at the origin at time $t=0$, find it's position $s(t)$ at time $t$.}{$s(t)=2t^{3/2}$.}

\exercise{A nickel dropped from the top of the North Dakota State Capital Building has acceleration $a(t)=-32$ ft/sec$^2$ (ignoring air resistance), initial velocity $v(0)=0$, and initial height $s(0)=241.67$ ft. How long will it take the nickel to hit the ground?}{$s(t)=241.67-16t^2$ ft, so $s(t)=0$ at $t=3.89$sec.}

\exercise{Given the graph of $f$ below, sketch the graph of the antiderivative $F$ of $f$ that passes through the origin. What do the graphs of the other antiderivatives of $f$ look like?

\begin{tikzpicture}[alt={A line from (0,2) to (1,1) to (5,1).}]
\begin{axis}[width=1.16\marginparwidth,tick label style={font=\scriptsize},minor x tick num=1, minor y tick num=1, axis y line=middle,axis x line=middle,ymin=-1,ymax=5.9,xmin=-1,xmax=5.9,name=myplot]
\draw[draw={\colorone},thick](axis cs:0,2)--(axis cs:1,1)--(axis cs:5,5);
%\addplot [draw={\colorone},domain=0:5,thick] {abs(x-1)+1};
\end{axis}
\node [right] at (myplot.right of origin) {\scriptsize $x$};
\node [above] at (myplot.above origin) {\scriptsize $y$};
\end{tikzpicture}}{\mbox{}\\[-2\baselineskip]\begin{tikzpicture}[alt={A curve from the origin up to (1,1.5) and then increasing even faster afterward.}]
\begin{axis}[width=1.16\marginparwidth,tick label style={font=\scriptsize},minor x tick num=1, minor y tick num=1, axis y line=middle,axis x line=middle,ymin=-1,ymax=5.9,xmin=-1,xmax=5.9,name=myplot]
\addplot [draw={\colorone},domain=0:1,thick] {2*x-(x^2)/2};
\addplot [draw={\colorone},domain=1:5,thick] {(x^2)/2+1};
\end{axis}
\node [right] at (myplot.right of origin) {\scriptsize $x$};
\node [above] at (myplot.above origin) {\scriptsize $y$};
\end{tikzpicture}\\
Other antiderivatives are vertical shifts of this one.}

\exercise{Given the graph of $f$ below, sketch the graph of the antiderivative $F$ of $f$ that passes through the origin. What do the graphs of the other antiderivatives of $f$ look like?

\begin{tikzpicture}[alt={A curve starting from (0,2) through (1.5,0) to (2,-2).  It makes a sharp point before increasing through (3.5,0) toward (5,4).}]
\begin{axis}[width=1.16\marginparwidth,tick label style={font=\scriptsize},minor x tick num=1, minor y tick num=1, axis y line=middle,axis x line=middle,ymin=-3,ymax=4.9,xmin=-1,xmax=5.9,name=myplot]
\addplot [draw={\colorone},domain=0:2,thick] {2-x*x};
\addplot [draw={\colorone},domain=2:5,thick] {((x*x)/4)-3};
\end{axis}
\node [right] at (myplot.right of origin) {\scriptsize $x$};
\node [above] at (myplot.above origin) {\scriptsize $y$};
\end{tikzpicture}}{\mbox{}\\[-2\baselineskip]\begin{tikzpicture}[alt={A curve starting from the origin increasing to flatten out at (1.5,2), decreasing to (3,0), bottoming out around (3.5,-0.5) increasing through (4,0) to finish near (5,2).}]
\begin{axis}[width=1.16\marginparwidth,tick label style={font=\scriptsize},minor x tick num=1, minor y tick num=1, axis y line=middle,axis x line=middle,ymin=-3,ymax=4.9,xmin=-1,xmax=5.9,name=myplot]
\addplot [draw={\colorone},domain=0:2,thick] {2*x-x*x*x/3}; % @x=2: =4/3
\addplot [draw={\colorone},domain=2:5,thick] {((x*x*x)/12)-3*x+20/3}; % @x=2: -16/3
\end{axis}
\node [right] at (myplot.right of origin) {\scriptsize $x$};
\node [above] at (myplot.above origin) {\scriptsize $y$};
\end{tikzpicture}\\
Other antiderivatives are vertical shifts of this one.}

\printreview

\exercise{Use information gained from the first and second derivatives to sketch $\ds f(x)=\frac1{e^x+1}$.}{Use technology to verify}

\exercise{Given $y = x^2e^x\cos x$, find $\dd y$.}{$\dd y = (2xe^x\cos x+ x^2e^x\cos x- x^2e^x\sin x)\dd x$}
