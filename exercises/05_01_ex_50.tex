{Given the graph of $f$ below, sketch the graph of the antiderivative $F$ of $f$ that passes through the origin. What do the graphs of the other antiderivatives of $f$ look like?

\begin{tikzpicture}
\begin{axis}[width=\marginparwidth+25pt,tick label style={font=\scriptsize},minor x tick num=1, minor y tick num=1, axis y line=middle,axis x line=middle,ymin=-1,ymax=5.9,xmin=-1,xmax=5.9,name=myplot]
\addplot [{\colorone},domain=0:5,thick] {abs(x-1)+1};
\end{axis}

\node [right] at (myplot.right of origin) {\scriptsize $x$};
\node [above] at (myplot.above origin) {\scriptsize $y$};
\end{tikzpicture}}
{~\\\begin{tikzpicture}
\begin{axis}[width=\marginparwidth+25pt,tick label style={font=\scriptsize},minor x tick num=1, minor y tick num=1, axis y line=middle,axis x line=middle,ymin=-1,ymax=5.9,xmin=-1,xmax=5.9,name=myplot]
\addplot [{\colorone},domain=0:1,thick] {2*x-(x^2)/2};
\addplot [{\colorone},domain=1:5,thick] {(x^2)/2+1};
\end{axis}

\node [right] at (myplot.right of origin) {\scriptsize $x$};
\node [above] at (myplot.above origin) {\scriptsize $y$};
\end{tikzpicture}\\
Other antiderivatives are vertical shifts of this one.}
