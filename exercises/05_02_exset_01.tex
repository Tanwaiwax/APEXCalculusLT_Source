\begin{exerciseset}{In Exercises}{, a graph of a function $f(x)$ is given. Using the geometry of the graph, evaluate the definite integrals.}

\exercise{\noindent
\begin{minipage}{\linewidth}
\begin{tikzpicture}
\begin{axis}[width=\marginparwidth,tick label style={font=\scriptsize},
axis y line=middle,axis x line=middle,name=myplot,axis on top,
ymin=-4.5,ymax=4.5,xmin=-.9,xmax=4.5]
\addplot [thick, draw={\colorone},domain=0:4] {-2*x+4};
\draw (axis cs:2,2.5) node {\scriptsize $y = -2x+4$};
\end{axis}
\node [right] at (myplot.right of origin) {\scriptsize $x$};
\node [above] at (myplot.above origin) {\scriptsize $y$};
\end{tikzpicture}
\end{minipage}
\begin{multicols}{2}
\begin{enumerate}
\item		$\ds \int_0^1 (-2x+4)\ dx$
\item		$\ds \int_0^2 (-2x+4)\ dx$
\item		$\ds \int_0^3 (-2x+4)\ dx$
\item		$\ds \int_1^3 (-2x+4)\ dx$
\item		$\ds \int_2^4 (-2x+4)\ dx$
\item		$\ds \int_0^1 (-6x+12)\ dx$
\end{enumerate}
\end{multicols}}{\mbox{}\\[-2\baselineskip]\parbox[t]{\linewidth}{\begin{enumerate}
\item		3
\item		4
\item		3
\item		0
\item		$-4$
\item		9
\end{enumerate}}}

\exercise{\noindent
\begin{minipage}{\linewidth}
\begin{tikzpicture}
\begin{axis}[width=\marginparwidth,tick label style={font=\scriptsize},
axis y line=middle,axis x line=middle,name=myplot,axis on top,
xtick={1,2,3,4,5},ymin=-2.2,ymax=2.2,xmin=-.9,xmax=5.5]
\addplot [thick, draw={\colorone},domain=0:2] {-2};
\addplot [thick, draw={\colorone},domain=2:3] {2*x-6};
\addplot [thick, draw={\colorone},domain=3:5] {x-3};
\draw (axis cs:3.2,-1.5) node {\scriptsize $y = f(x)$};
\end{axis}
\node [right] at (myplot.right of origin) {\scriptsize $x$};
\node [above] at (myplot.above origin) {\scriptsize $y$};
\end{tikzpicture}
\end{minipage}
\begin{multicols}{2}
\begin{enumerate}
\item		$\ds \int_0^2 f(x)\ dx$
\item		$\ds \int_0^3 f(x)\ dx$
\item		$\ds \int_0^5 f(x)\ dx$
\item		$\ds \int_2^5 f(x)\ dx$
\item		$\ds \int_5^3 f(x)\ dx$
\item		$\ds \int_0^3 -2f(x)\ dx$
\end{enumerate}
\end{multicols}}{\mbox{}\\[-2\baselineskip]\parbox[t]{\linewidth}{\begin{enumerate}
\item		$-4$
\item		$-5$
\item		$-3$
\item		1
\item		$-2$
\item		10
\end{enumerate}}}

\exercise{\noindent
\begin{minipage}{\linewidth}
\begin{tikzpicture}
\begin{axis}[width=\marginparwidth,tick label style={font=\scriptsize},
axis y line=middle,axis x line=middle,name=myplot,axis on top,
xtick={1,2,3,4,5},ymin=-.5,ymax=4.5,xmin=-.9,xmax=4.5]
\addplot [thick, draw={\colorone},domain=0:1] {4*x};
\addplot [thick, draw={\colorone},domain=1:2] {-4*(x-2)};
\addplot [thick, draw={\colorone},domain=2:3] {2*(x-2)};
\addplot [thick, draw={\colorone},domain=3:4] {-2*(x-4)};
\draw (axis cs:3,2.5) node {\scriptsize $y = f(x)$};
\end{axis}
\node [right] at (myplot.right of origin) {\scriptsize $x$};
\node [above] at (myplot.above origin) {\scriptsize $y$};
\end{tikzpicture}
\end{minipage}
\begin{multicols}{2}
\begin{enumerate}
\item		$\ds \int_0^2 f(x)\ dx$
\item		$\ds \int_2^4 f(x)\ dx$
\item		$\ds \int_2^4 2f(x)\ dx$
\item		$\ds \int_0^1 4x\ dx$
\item		$\ds \int_2^3 (2x-4)\ dx$
\item		$\ds \int_2^3 (4x-8)\ dx$
\end{enumerate}
\end{multicols}}{\mbox{}\\[-2\baselineskip]\parbox[t]{\linewidth}{\begin{enumerate}
\item		$4$
\item		$2$
\item		$4$
\item		2
\item		$1$
\item		2
\end{enumerate}}}

\exercise{\noindent
\begin{minipage}{\linewidth}
\begin{tikzpicture}
\begin{axis}[width=\marginparwidth,tick label style={font=\scriptsize},
axis y line=middle,axis x line=middle,name=myplot,axis on top,
xtick={1,2,3,4,5},ytick={-1,1,2,3},ymin=-1.5,ymax=3.5,xmin=-.9,xmax=4.5]
\addplot [thick, draw={\colorone},domain=0:4] {x-1};
\draw (axis cs:2,2.5) node {\scriptsize $y = x-1$};
\end{axis}
\node [right] at (myplot.right of origin) {\scriptsize $x$};
\node [above] at (myplot.above origin) {\scriptsize $y$};
\end{tikzpicture}
\end{minipage}
\begin{multicols}{2}
\begin{enumerate}
\item		$\ds \int_0^1 (x-1)\ dx$
\item		$\ds \int_0^2 (x-1)\ dx$
\item		$\ds \int_0^3 (x-1)\ dx$
\item		$\ds \int_2^3 (x-1)\ dx$
\item		$\ds \int_1^4 (x-1)\ dx$
\item		$\ds \int_1^4 \big((x-1)+1\big)\ dx$
\end{enumerate}
\end{multicols}}{\mbox{}\\[-2\baselineskip]\parbox[t]{\linewidth}{\begin{enumerate}
\item		$-1/2$
\item		$0$
\item		$3/2$
\item		$3/2$
\item		$9/2$
\item		$15/2$
\end{enumerate}}}

\exercise{\noindent
\begin{minipage}{\linewidth}
\begin{tikzpicture}
\begin{axis}[width=\marginparwidth,tick label style={font=\scriptsize},
axis y line=middle,axis x line=middle,name=myplot,axis on top,
xtick={1,2,3,4},ytick={1,2,3},ymin=-.5,ymax=3.5,xmin=-.9,xmax=4.5]
\addplot [thick, draw={\colorone},smooth,samples=40,domain=0:3.14159] ({2*cos(deg(x))+2}, {2*sin(deg(x))});
\draw (axis cs:2,2.4) node {\scriptsize $f(x) = \sqrt{4-(x-2)^2}$};
\end{axis}
\node [right] at (myplot.right of origin) {\scriptsize $x$};
\node [above] at (myplot.above origin) {\scriptsize $y$};
\end{tikzpicture}
\end{minipage}
\begin{multicols}{2}
\begin{enumerate}
\item		$\ds \int_0^2 f(x)\ dx$
\item		$\ds \int_2^4 f(x)\ dx$
\item		$\ds \int_0^4 f(x)\ dx$
\item		$\ds \int_0^4 5f(x)\ dx$
\end{enumerate}
\end{multicols}}{\mbox{}\\[-2\baselineskip]\parbox[t]{\linewidth}{\begin{enumerate}
\item		$\pi$
\item		$\pi$
\item		$2\pi$
\item		$10\pi$
\end{enumerate}}}

\exercise{\noindent
\begin{minipage}{\linewidth}
\begin{tikzpicture}
\begin{axis}[width=\marginparwidth,tick label style={font=\scriptsize},
axis y line=middle,axis x line=middle,name=myplot,axis on top,
ytick={1,2,3},ymin=-.5,ymax=3.5,xmin=-1,xmax=10.9]
\addplot [thick,draw={\colorone},smooth,samples=2,domain=0:10] ({x}, {3});
\draw (axis cs:5,3.4) node {\scriptsize $f(x) = 3$};
\end{axis}
\node [right] at (myplot.right of origin) {\scriptsize $x$};
\node [above] at (myplot.above origin) {\scriptsize $y$};
\end{tikzpicture}
\end{minipage}
\begin{multicols}{2}
\begin{enumerate}
\item           $\ds \int_0^5 f(x)\ dx$
\item           $\ds \int_3^7 f(x)\ dx$
\item           $\ds \int_0^0 f(x)\ dx$
\item           $\ds \int_a^b f(x)\ dx$, where\\
$0\leq a\leq b\leq 10$
\end{enumerate}
\end{multicols}}{\mbox{}\\[-2\baselineskip]\parbox[t]{\linewidth}{\begin{enumerate}
\item           $15$
\item           $12$
\item           $0$
\item           $3(b-a)$
\end{enumerate}}}

\end{exerciseset}
