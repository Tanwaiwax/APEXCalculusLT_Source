\printconcepts

\exercise{T/F: Simpson's Rule is a method of approximating antiderivatives.}{F}

\exercise{What are the two basic situations where approximating the value of a definite integral is necessary?}{When the antiderivative cannot be computed and when the integrand is unknown.}

\exercise{Why are the Left and Right Hand Rules rarely used?}{They are superseded by the Trapezoidal Rule; it takes an equal amount of work and is generally more accurate.}

\exercise{Why is the Midpoint Rule rarely used?}{It is superseded by the Trapezoidal Rule; it is about as accurate, but takes more work.}

\printproblems

\exerciseset{In Exercises}{, a definite integral is given. 
\begin{enumerate}
\item	[(a)] Approximate the definite integral with the Trapezoidal Rule and $n=4$.
\item	[(b)] Approximate the definite integral with Simpson's Rule and $n=4$.
\item	[(c)] Find the exact value of the integral.
\end{enumerate}}{

\exercise{$\ds \int_{-1}^1 x^2\ dx$}{\mbox{}\\[-2\baselineskip]\begin{enumerate}
\item		$3/4$
\item		$2/3$
\item		$2/3$
\end{enumerate}}

\exercise{$\ds \int_{0}^{10} 5x\ dx$}{\mbox{}\\[-2\baselineskip]\begin{enumerate}
\item		$250$
\item		$250$
\item		$250$
\end{enumerate}}

\exercise{$\ds \int_{0}^{\pi} \sin x\ dx$}{\mbox{}\\[-2\baselineskip]\begin{enumerate}
\item		$\frac14(1+\sqrt2)\pi\approx 1.896$
\item		$\frac1{6}(1+2\sqrt{2})\pi\approx 2.005$
\item		$2$
\end{enumerate}}

\exercise{$\ds \int_{0}^{4} \sqrt x\ dx$}{\mbox{}\\[-2\baselineskip]\begin{enumerate}
\item		$2+\sqrt2+\sqrt3\approx 5.15$
\item		$2/3(3+\sqrt2+2\sqrt3)\approx 5.25$
\item		$16/3\approx 5.33$
\end{enumerate}}

\exercise{$\ds \int_{0}^{3} (x^3+2x^2-5x+7)\ dx$}{\mbox{}\\[-2\baselineskip]\begin{enumerate}
\item		$38.5781$
\item		$147/4\approx 36.75$
\item		$147/4\approx 36.75$
\end{enumerate}}

\exercise{$\ds \int_{0}^{1} x^4\ dx$}{\mbox{}\\[-2\baselineskip]\begin{enumerate}
\item		$0.2207$
\item		$0.2005$
\item		$1/5$
\end{enumerate}}

\exercise{$\ds \int_{0}^{2\pi} \cos x\ dx$}{\mbox{}\\[-2\baselineskip]\begin{enumerate}
\item		$0$
\item		$0$
\item		$0$
\end{enumerate}}

\exercise{$\ds \int_{-3}^{3} \sqrt{9-x^2} \ dx$}{\mbox{}\\[-2\baselineskip]\begin{enumerate}
\item		$9(1+\sqrt3)/2\approx 12.294$
\item		$3+6\sqrt3\approx 13.392$
\item		$9\pi/2\approx 14.137$
\end{enumerate}}

}


\begin{exerciseset}{In Exercises}{, approximate the definite integral with the Trapezoidal Rule and Simpson's Rule, with $n=6$.}

\exercise{$\ds \int_{0}^{1} \cos \big(x^2\big) \ dx$}{Trapezoidal Rule: 	$0.9006$

Simpson's Rule: $0.90452$}

\exercise{$\ds \int_{-1}^{1} e^{x^2} \ dx$}{Trapezoidal Rule: 	$3.0241$

Simpson's Rule: $2.9315$}

\exercise{$\ds \int_{0}^{5} \sqrt{x^2+1} \ dx$}{Trapezoidal Rule: 	$13.9604$

Simpson's Rule: $13.9066$}

\exercise{$\ds \int_{0}^{\pi} x\sin x \ dx$}{Trapezoidal Rule: 	$3.0695$

Simpson's Rule: $3.14295$}

\exercise{$\ds \int_{0}^{\pi/2} \sqrt{\cos x} \ dx$}{Trapezoidal Rule: 	$1.1703$

Simpson's Rule: $1.1873$}

\exercise{$\ds \int_{1}^{4} \ln x \ dx$}{Trapezoidal Rule: 	$2.52971$

Simpson's Rule: $2.5447$}

\exercise{$\ds \int_{-1}^{1} \frac{1}{\sin x+2} \ dx$}{Trapezoidal Rule: 	$1.0803$

Simpson's Rule: $1.077$}

\exercise{$\ds \int_{0}^{6} \frac{1}{\sin x+2} \ dx$}{Trapezoidal Rule: 	$3.5472$

Simpson's Rule: $3.6133$}

\end{exerciseset}


\exerciseset{In Exercises}{, find $n$ such that the error in approximating the given definite integral is less than $0.0001$ when using:
\begin{enumerate}
\item [(a)] the Trapezoidal Rule
\item [(b)] Simpson's Rule
\end{enumerate}
}{

\exercise{$\ds \int_{0}^{\pi} \sin x \ dx$}{\begin{enumerate}
\item		$n=161$ (using $\max\big(\fpp(x)\big)=1$)
\item		$n=12$	(using $\max\big(f\,^{(4)}(x)\big)=1$)
\end{enumerate}
}
\exercise{$\ds \int_{1}^{4} \frac{1}{\sqrt x} \ dx$}{\begin{enumerate}
\item		$n=150$ (using $\max\big(\fpp(x)\big)=1$)
\item		$n=18$	(using $\max\big(f\,^{(4)}(x)\big)=7$)
\end{enumerate}
}
\exercise{$\ds \int_{0}^{\pi} \cos \big(x^2\big) \ dx$}{\begin{enumerate}
\item		$n=1004$ (using $\max\big(\fpp(x)\big)=39$)
\item		$n=62$	(using $\max\big(f\,^{(4)}(x)\big)=800$)
\end{enumerate}
}
\exercise{$\ds \int_{0}^{5} x^4 \ dx$}{\begin{enumerate}
\item		$n=5591$ (using $\max\big(\fpp(x)\big)=300$)
\item		$n=46$	(using $\max\big(f\,^{(4)}(x)\big)=24$)
\end{enumerate}
}
}

\exerciseset{In Exercises}{, a region is given. Find the area of the region using Simpson's Rule:
\begin{enumerate}
\item	[(a)] where the measurements are in centimeters, taken in 1 cm increments, and
\item	[(b)] where the measurements are in hundreds of yards, taken in 100 yd increments.
\end{enumerate}
}{

\exercise{\begin{minipage}{\linewidth}\myincludegraphics{figures/fig05_05_ex_01}\end{minipage}}{\begin{enumerate}
\item		Area is $30.8667$ cm$^2$.
\item		Area is $308,667$ yd$^2$.
\end{enumerate}
}
\exercise{\begin{minipage}{\linewidth}\myincludegraphics{figures/fig05_05_ex_02}\end{minipage}
}{\begin{enumerate}
\item		Area is 25.0667 cm$^2$
\item		Area is 250,667 yd$^2$
\end{enumerate}
}}

\exercise{Let $f$ be the quadratic function that goes through the points $(x_1,y_1)$, $(x_1+\Delta x,y_2)$ and $(x_1+2\Delta x,y_3)$.  Show that $\ds\int_{x_1}^{x_1+2\Delta x}f(x)\dd x=\frac{\Delta x}3(y_1+4y_2+y_3)$.}{Let $f(x)=a(x-x_1)^2+b(x-x_1)+c$, so that $f(x_1)=c=y_1$, $f(x_1+\Delta x)=a\Delta x^2+b\Delta x+c=y_2$, and $f(x_1+2\Delta x)=4a\Delta x^2+2b\Delta x+c=y_3$.  Therefore, $a=\frac{y_1-2y_2+y_3}{2(\Delta x)^2}$ and $b=\frac{4y_2-y_3-3y_1}{2\Delta x}$, and $\ds\int_{x_1}^{x_1+2\Delta x}a(x-x_1)^2+b(x-x_1)+c\dd x=\frac{a(2\Delta x)^3}3+\frac{b(2\Delta x)^2}2+c(2\Delta x)=\frac{4(y_1-2y_2+y_3)\Delta x}3+(4y_2-y_3-3y_1)\Delta x+2y_1\Delta x=\frac{\Delta x}3(4y_1-8y_2+4y_3+12y_2-3y_3-9y_1+6y_1)=\frac{\Delta x}3(y_1+4y_2+y_3)$.}
