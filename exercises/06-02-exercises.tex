\printconcepts

\exercise{T/F: Integration by Parts is useful in evaluating integrands that contain products of functions.}{T}

\exercise{T/F: Integration by Parts can be thought of as the ``opposite of the Chain Rule.''}{F}

% we're not a fan of this question
%\exercise{For what is ``LIATE'' useful?}{Determining which functions in the integrand to set equal to ``$u$'' and which to set equal to ``$dv$''.}

% cut for parity
%\exercise{T/F: If the integral that results from Integration by Parts appears to also need Integration by Parts, then a mistake was made in the original choice of ``u''.}{F}

\printproblems

\exercisesetinstructions{, evaluate the given indefinite integral.}

\exercise{\mbox{}\label{ibp_prob:4}$\ds\int x\sin x\dd x$}{$\sin x-x\cos x+C$}

\exercise{$\ds\int xe^{-x}\dd x$}{$-e^{-x}-xe^{-x}+C$}

\exercise{$\ds\int x^2\sin x\dd x$}{$-x^2\cos x+2x\sin x+2\cos x+C$}

\exercise{$\ds\int x^3\sin x\dd x$}{$-x^3\cos x+3x^2\sin x+6x\cos x-6\sin x+C$}

\exercise{$\ds\int xe^{x^2}\dd x$}{$1/2e^{x^2}+C$}

\exercise{$\ds\int x^3e^{x}\dd x$}{$x^3e^x-3x^2e^x+6xe^x-6e^x+C$}

\exercise{$\ds\int xe^{-2x}\dd x$}{$-\frac12 xe^{-2x}-\frac{e^{-2 x}}4+C$}

\exercise{$\ds\int e^x\sin x\dd x$}{$\frac12 e^x(\sin x-\cos x)+C$}

\exercise{$\ds\int e^{2x}\cos x\dd x$}{$1/5e^{2x}(\sin x+2\cos x)+C$}

\exercise{\mbox{}\label{ibp_prob:12}$\ds\int e^{2x}\sin (3x)\dd x$}{$\frac1{13}e^{2x}(2\sin(3x)-3\cos(3x))+C$}

\exercise{$\ds\int e^{5x}\cos (5x)\dd x$}{$\frac1{10}e^{5x}(\sin(5x)+\cos(5x))+C$}

\exercise{$\ds\int\sin x\cos x\dd x$}{$-\frac12\cos^2x+C$}

\exercise{$\ds\int\sin^{-1}x\dd x$}{$\sqrt{1-x^2}+x\sin^{-1}(x)+C$}

\exercise{$\ds\int\tan^{-1}(2x)\dd x$}{$x\tan^{-1}(2x)-\frac14\ln\abs{4 x^2+1}+C$}

\exercise{$\ds\int x\tan^{-1} x\dd x$}{$\frac12 x^2\tan^{-1}(x)-\frac x2+\frac12\tan^{-1}(x)+C$}

\exercise{$\ds\int\cos^{-1}x\dd x$}{$x\cos^{-1}x-\sqrt{1-x^2}+C$}

\exercise{$\ds\int x\ln x\dd x$}{$\frac12 x^2\ln\abs x-\frac{x^2}4+C$}

\exercise{$\ds\int(x-2)\ln x\dd x$}{$-\frac{x^2}4+\frac12 x^2\ln\abs x-2x\ln\abs x+2x+C$}

\exercise{$\ds\int x\ln (x-1)\dd x$}{$-\frac{x^2}4+\frac12 x^2 \ln\abs{x-1}-\frac x2-\frac12\ln\abs{x-1}+C$}

\exercise{$\ds\int x\ln (x^2)\dd x$}{$\frac12 x^2\ln\left(x^2\right)-\frac{x^2}2+C$}

\exercise{$\ds\int x^2\ln x\dd x$}{$\frac13 x^3\ln\abs x-\frac{x^3}9+C$}

\exercise{$\ds\int\left(\ln x\right)^2\dd x$}{$2x+x\left(\ln\abs x\right)^2-2x\ln\abs x+C$}

\exercise{$\ds\int\left(\ln (x+1)\right)^2\dd x$}{$2x+x\left(\ln\abs{x+1}\right)^2+\left(\ln\abs{x+1}\right)^2-2x\ln\abs{x+1}
-2\ln\abs{x+1}+2+C$}

\exercise{$\ds\int x\sec^2x\dd x$}{$x\tan x+\ln\abs{\cos x}+C$}

\exercise{$\ds\int x\csc^2x\dd x$}{$\ln\abs{\sin x}-x\cot x+C$}

\exercise{$\ds\int x\sqrt{x-2}\dd x$}{$\frac25 (x-2)^{5/2}+\frac43 (x-2)^{3/2}+C$}

\exercise{$\ds\int x\sqrt{x^2-2}\dd x$}{$\frac13 (x^2-2)^{3/2}+C$}

\exercise{$\ds\int\sec x\tan x\dd x$}{$\sec x+C$}

\exercise{$\ds\int x\sec x\tan x\dd x$}{$x\sec x-\ln\abs{\sec x+\tan x}+C$}

\exercise{$\ds\int x\csc x\cot x\dd x$}{$-x\csc x-\ln\abs{\csc x+\cot x}+C$}

\exercise{$\ds\int x\cosh x\dd x$}{$x\sinh x-\cosh x+C$}

\exercise{$\ds\int x\sinh x\dd x$}{$x\cosh x-\sinh x+C$}

\exercise{$\ds\int\sinh^{-1}x\dd x$}{$x\sinh^{-1}x-\sqrt{x^2+1}+C$}

\exercise{$\ds\int\tanh^{-1}x\dd x$}{$x\tanh^{-1}x+\frac12 \ln\abs{x^2-1}+C$}

\exercisesetend


\input{exercises/06-02-exset-02}

\exercisesetinstructions{, evaluate the definite integral. Note: the corresponding indefinite integrals appear in Exercises~\ref{ibp_prob:4}--\ref{ibp_prob:12}.}

\exercise{$\ds\int_0^\pi x\sin x\dd x$}{$\pi$}

\exercise{$\ds\int_{-1}^1 xe^{-x}\dd x$}{$-2/e$}

\exercise{$\ds\int_{-\pi/4}^{\pi/4} x^2\sin x\dd x$}{$0$}

\exercise{$\ds\int_{-\pi/2}^{\pi/2} x^3\sin x\dd x$}{$\frac{3\pi^2}{2}-12$}

\exercise{$\ds\int_0^{\sqrt{\ln 2}} xe^{x^2}\dd x$}{$1/2$}

\exercise{$\ds\int_0^1 x^3e^{x}\dd x$}{$6-2e$}

\exercise{$\ds\int_1^2 xe^{-2x}\dd x$}{$\frac3{4e^2}-\frac5{4e^4}$}

\exercise{$\ds\int_0^{\pi} e^x\sin x\dd x$}{$\frac12+\frac{e^\pi}2$}

\exercise{$\ds\int_{-\pi/2}^{\pi/2} e^{2x}\cos x\dd x$}{$\frac15\left(e^\pi+e^{-\pi}\right)$}

\exercise{$\ds\int_0^{\pi/3}e^{2x}\sin (3x)\dd x$}{$\frac3{13}(1+e^{2\pi/3})+C$}

\exercisesetend


\exercise{\mbox{}\\[-2\baselineskip]\parbox[t]{\linewidth}{\begin{enumerate}
 \item  For $n\geq2$ show that
 \[
  \int_0^{\pi/2}\sin^n x\dd x = \frac{n-1}n \int_0^{\pi/2}\sin^{n-2} x\dd x.
 \]
 Hint:  Begin by writing $\sin^n x$ as $(\sin^{n-1} x) \sin x$ and using Integration by Parts.
 \item  For $k \geq 1$ show that
 \begin{align*}
  \int_0^{\pi/2}\sin^{2k}x\dd x&=\frac{1\cdot3\cdot5\dotsm(2k-1)}{2\cdot4\cdot6\dotsm(2k)}\frac\pi2\qquad\text{and}\\
  \int_0^{\pi/2}\sin^{2k+1}x\dd x&=\frac{2\cdot4\cdot6\dotsm(2k)}{1\cdot3\cdot5\cdot7\dotsm(2k+1)}.
 \end{align*}
\end{enumerate}}}{}

\exercise{Find the volume of the solid of revolution obtained by rotating the region bounded by $y=0$, $y=\ln x$, $x=1$, and $x=e$:
\begin{enumerate}
 \item About the $x$-axis, using the disk method.
 \item About the $y$-axis, using the shell method.
\end{enumerate}}{\mbox{}\\[-2\baselineskip]\parbox[t]{\linewidth}{\begin{enumerate}
\item $\pi(e-2)$
\item $\frac\pi2(e^2+1)$
\end{enumerate}}}

\exercise{Let $f(x)=x$ for $-\pi\le x<\pi$ and extend this function so that it is periodic with period $2\pi$.  This function is known as a \emph{sawtooth wave}\index{sawtooth wave} and looks like
\begin{minipage}{\linewidth}\centering
\pdftooltip{\begin{tikzpicture}
 \begin{axis}[width=\textwidth,tick label style={font=\scriptsize},
  axis y line=middle, axis x line=middle, name=myplot, axis on top,
  xmin=-7,xmax=7,ymin=-5,ymax=5,axis equal image]
  \draw[thick,draw={\colorone}](axis cs:-9.42,-3.14)--(axis cs:-3.14,3.14)
    (axis cs:-3.14,-3.14)--(axis cs:3.14,3.14)
    (axis cs:3.14,-3.14)--(axis cs:9.42,3.14);
  \draw(axis cs:-3.14,-3.14)circle[\colorone,fill={\colorone},radius=.1];
  \draw(axis cs:3.14,-3.14)circle[\colorone,fill={\colorone},radius=.1];
  \draw(axis cs:-3.14,3.14)circle[\colorone,fill={\colorone},radius=.1];
  \draw(axis cs:3.14,3.14)circle[\colorone,fill={\colorone},radius=.1];
 \end{axis}
 \node [right] at (myplot.right of origin) {\scriptsize $x$};
 \node [above] at (myplot.above origin) {\scriptsize $y$};
\end{tikzpicture}}{ALT-TEXT-TO-BE-DETERMINED}
\end{minipage}
For a positive integer $n$, define $\ds b_n=\frac1\pi\int_{-\pi}^\pi f(x)\sin(nx)\dd x$.
\begin{enumerate}
\item Find $b_n$.
\item Graph $\ds\sum_{n=1}^N b_n\sin(nx)$ for various values of $N$.  What do you observe?
\end{enumerate}}{\mbox{}\\[-2\baselineskip]\parbox[t]{\linewidth}{\begin{enumerate}
\item $b_n=(-1)^{n+1} 2/n$
\item answers will vary
\end{enumerate}}}

\exercise{Let $f(x)=\begin{cases}-x-\pi&-\pi\le x<-\frac\pi2\\x&-\frac\pi2\le x<\phantom{-}\frac\pi2\\\pi-x&\phantom{-}\frac\pi2\le x<\phantom{-}\pi\end{cases}$ and extend this function so that it is periodic with period $2\pi$.  This function is known as a \emph{triangle wave}\index{triangle wave} and looks like
\begin{minipage}{\linewidth}\centering
\pdftooltip{\begin{tikzpicture}
 \begin{axis}[width=\textwidth,tick label style={font=\scriptsize},
  axis y line=middle, axis x line=middle, name=myplot, axis on top,
  xmin=-7,xmax=7,ymin=-3,ymax=3,axis equal image]
  \draw[thick,draw={\colorone}](axis cs:-9.42,0)--(axis cs:-7.85,-1.57)
    --(axis cs:-4.71, 1.57)--(axis cs:-1.57,-1.57)--(axis cs:1.57,1.57)
    --(axis cs: 4.71,-1.57)--(axis cs: 7.85, 1.57)--(axis cs:9.42,0);
 \end{axis}
 \node [right] at (myplot.right of origin) {\scriptsize $x$};
 \node [above] at (myplot.above origin) {\scriptsize $y$};
\end{tikzpicture}}{ALT-TEXT-TO-BE-DETERMINED}
\end{minipage}
For a positive integer $n$, define $\ds b_n=\frac1\pi\int_{-\pi}^\pi f(x)\sin(nx)\dd x$.
\begin{enumerate}
\item Find $b_n$.
\item Graph $\ds\sum_{n=1}^N b_n\sin(nx)$ for various values of $N$.  What do you observe?
\end{enumerate}}{\mbox{}\\[-2\baselineskip]\parbox[t]{\linewidth}{\begin{enumerate}
\item $b_n=(-1)^{(n-1)/2}4/\pi n^2$ for odd $n$ and $b_n=0$ for even $n$
\item answers will vary
\end{enumerate}}}

%\printreview

%\exercisesetinstructions{, use the Fundamental Theorem of Calculus Part 1 to find $F'(x)$.}

\exercise{$\ds F(x) = \int_2^{x^3+x} \frac{1}{t}\dd t$}{$F'(x) = (3x^2+1)\frac{1}{x^3+x}$}

\exercise{$\ds F(x) = \int_{x^3}^{0} t^3\dd t$}{$F'(x) = -3x^{11}$}

\exercise{$\ds F(x) = \int_{x}^{x^2} (t+2)\dd t$}{$F'(x) = 2x(x^2+2)-(x+2)$}

\exercise{$\ds F(x) = \int_{\ln x}^{e^x} \sin t\dd t$}{$F'(x) = e^x\sin (e^x) - \frac1x \sin(\ln x)$}

\exercise{$\ds F(x)=\int_1^{x} \frac{\ln t+4}{t^2+7}\dd t$}{$F'(x)=\frac{\ln x+4}{x^2+7}$}

\exercise{$\ds F(x)= \int_2^{\sin x} \cos^3 t+3\tan^3 t\dd t$}{$F'(x)=[\cos^3(\sin x)+3\tan^3(\sin x)]\cos x $}

\exercise{$\ds F(x)= \int_{5x^3}^4 \frac{\sqrt{\cos t+5}}{t^2+e^t}\dd t$}{$F'(x)=-\frac{15x^2\sqrt{\cos (5x^3)+5}}{25x^6+e^{5x^3}}$}

\exercise{$\ds  F(x)=\int_{\tan^2 x}^{10} \ln t +e^{t^2-7}\dd t$}{$F'(x)= -2\tan x\sec^2 x[\ln (\tan^2x) +e^{\tan^4x-7}]$}

\exercisesetend

