\printconcepts

\exercise{Fill in the blank: Partial Fraction Decomposition is a method of rewriting \underline{\hskip .5in} functions.}{rational}

\exercise{T/F: It is sometimes necessary to use polynomial division before using Partial Fraction Decomposition.}{T}

\exercise{Decompose $\ds \frac{1}{x^2-3x}$ without solving for the coefficients, as done in \autoref{ex_pf1}.}{$\frac{A}{x} + \frac{B}{x-3}$}

\exercise{Decompose $\ds \frac{7-x}{x^2-9}$ without solving for the coefficients, as done in \autoref{ex_pf1}.}{$\frac{A}{x-3} + \frac{B}{x+3}$}

\exercise{Decompose $\ds \frac{x-3}{x^2-7}$ without solving for the coefficients, as done in \autoref{ex_pf1}.}{$\frac{A}{x-\sqrt{7}} + \frac{B}{x+\sqrt{7}}$}

\exercise{Decompose $\ds \frac{2x+5}{x^3+7x}$ without solving for the coefficients, as done in \autoref{ex_pf1}.}{$\frac{A}{x} + \frac{Bx+C}{x^2+7}$}

\printproblems

\input{exercises/06-04-exset-01}

\input{exercises/06-04-exset-02}

\exercise{Recall that
 \[\frac\dd{\dd x}\sinh^{-1}x=\frac1{\sqrt{x^2+1}}\]
 Now use a trigonometric substitution to evaluate the indefinite integral
 \[\int\frac1{\sqrt{x^2+1}}\dd x\]
 and show that
 \[\sinh^{-1}x=\ln(x+\sqrt{x^2+1}).\]}{}

%\printreview

%\exercisesetinstructions{, use the Fundamental Theorem of Calculus Part 1 to find $F'(x)$.}

\exercise{$\ds F(x) = \int_2^{x^3+x} \frac{1}{t}\dd t$}{$F'(x) = (3x^2+1)\frac{1}{x^3+x}$}

\exercise{$\ds F(x) = \int_{x^3}^{0} t^3\dd t$}{$F'(x) = -3x^{11}$}

\exercise{$\ds F(x) = \int_{x}^{x^2} (t+2)\dd t$}{$F'(x) = 2x(x^2+2)-(x+2)$}

\exercise{$\ds F(x) = \int_{\ln x}^{e^x} \sin t\dd t$}{$F'(x) = e^x\sin (e^x) - \frac1x \sin(\ln x)$}

\exercise{$\ds F(x)=\int_1^{x} \frac{\ln t+4}{t^2+7}\dd t$}{$F'(x)=\frac{\ln x+4}{x^2+7}$}

\exercise{$\ds F(x)= \int_2^{\sin x} \cos^3 t+3\tan^3 t\dd t$}{$F'(x)=[\cos^3(\sin x)+3\tan^3(\sin x)]\cos x $}

\exercise{$\ds F(x)= \int_{5x^3}^4 \frac{\sqrt{\cos t+5}}{t^2+e^t}\dd t$}{$F'(x)=-\frac{15x^2\sqrt{\cos (5x^3)+5}}{25x^6+e^{5x^3}}$}

\exercise{$\ds  F(x)=\int_{\tan^2 x}^{10} \ln t +e^{t^2-7}\dd t$}{$F'(x)= -2\tan x\sec^2 x[\ln (\tan^2x) +e^{\tan^4x-7}]$}

\exercisesetend


% todo write a partial fraction exercise with a repeated quadratic and one with two distinct quadratics
