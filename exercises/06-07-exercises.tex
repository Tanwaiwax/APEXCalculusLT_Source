\printconcepts

\exercise{The definite integral was defined with what two stipulations?}{The interval of integration is finite, and the integrand is continuous on that interval.}

\exercise{If $\ds \lim_{b\to \infty} \int_0^b f(x)\dd x$ exists, then the integral $\ds \int_0^\infty f(x)\dd x$ is said to \underline{\hskip 1in}.}{converge}

\exercise{If $\ds \int_1^\infty f(x)\dd x=10$, and $0\leq g(x)\leq f(x)$ for all $x$, then we know that $\ds \int_1^\infty g(x)\dd x$  \underline{\hskip 1in}.}{converges; could also state $\le10$.}

\exercise{For what values of $p$ will $\ds \int_1^\infty \frac1{x^p}\dd x$ converge?}{$p>1$}

\exercise{For what values of $p$ will $\ds \int_{10}^\infty \frac1{x^p}\dd x$ converge?}{$p>1$}

\exercise{For what values of $p$ will $\ds \int_{0}^1 \frac1{x^p}\dd x$ converge?}{$p<1$}

\printproblems

\exercisesetinstructions{, evaluate the given improper integral.}

\exercise{$\ds \int_0^\infty e^{5-2x}\dd x$}{$e^5/2$}

\exercise{$\ds \int_1^\infty \frac{1}{x^3}\dd x$}{$1/2$}

\exercise{$\ds \int_1^\infty x^{-4}\dd x$}{$1/3$}

\exercise{$\ds \int_{-\infty}^\infty \frac{1}{x^2+9}\dd x$}{$\pi/3$}

\exercise{$\ds \int_{-\infty}^0 2^x\dd x$}{$1/\ln 2$}

\exercise{$\ds \int_{-\infty}^0 \left(\frac12\right)^x\dd x$}{diverges}

\exercise{$\ds \int_{-\infty}^\infty\frac{x}{x^2+1}\dd x$}{diverges}

\exercise{$\ds \int_{-\infty}^\infty\frac{x}{x^2+4}\dd x$}{diverges}

\exercise{$\ds \int_{2}^\infty\frac{1}{(x-1)^2}\dd x$}{$1$}

\exercise{$\ds \int_{1}^2\frac{1}{(x-1)^2}\dd x$}{diverges}

\exercise{$\ds \int_{2}^\infty\frac{1}{x-1}\dd x$}{diverges}

\exercise{$\ds \int_{1}^2\frac{1}{x-1}\dd x$}{diverges}

\exercise{$\ds \int_{0}^3\frac1x\dd x$}{diverges}

\exercise{$\ds \int_{-1}^1 \frac 1x\dd x$}{diverges}

\exercise{$\ds \int_{2}^5 \frac{\dd x}{\sqrt{x-2}}$}{$2\sqrt3$}

\exercise{$\ds \int_{1}^9 \frac{\dd x}{\sqrt[3]{9-x}}$}{$6$}

\exercise{$\ds \int_{1}^3\frac{1}{x-2}\dd x$}{diverges}

\exercise{$\ds \int_{0}^\pi \sec^2 x\dd x$}{diverges}

\exercise{$\ds \int_{0}^{\frac\pi2}\sec x\dd x$}{diverges}

\exercise{$\ds \int_{-2}^1 \frac{1}{\sqrt{\abs x}}\dd x$}{$2+2\sqrt{2}$}

\exercise{$\ds \int_{0}^\infty xe^{-x}\dd x$}{$1$}

\exercise{$\ds \int_{0}^\infty xe^{-x^2}\dd x$}{$1/2$}

\exercise{$\ds \int_{-\infty}^\infty xe^{-x^2}\dd x$}{$0$}

\exercise{$\ds \int_{-\infty}^\infty \frac{1}{e^x+e^{-x}}\dd x$}{$\pi/2$}

\exercise{$\ds \int_{0}^1 x\ln x\dd x$}{$-1/4$}

\exercise{$\ds \int_{1}^\infty \frac{\ln x}{x}\dd x$}{diverges}

\exercise{$\ds \int_{0}^1 \ln x\dd x$}{$-1$}

\exercise{$\ds \int_{1}^\infty \frac{\ln x}{x^2}\dd x$}{$1$}

\exercise{$\ds \int_{1}^\infty \frac{\ln x}{\sqrt{x}}\dd x$}{diverges}

\exercise{$\ds \int_{0}^\infty e^{-x}\sin x\dd x$}{$1/2$}

% cut for parity
%\exercise{$\ds \int_{0}^\infty e^{-x}\cos x\dd x$}{$1/2$}

\exercisesetend


\input{exercises/06-07-exset-02}

%\input{exercises/06-05-exset-03}
%\exercisesetinstructions{, evaluate the given indefinite integral.}

\exercise{$\ds \int \tanh (2x)\dd x$}{$\frac12\ln (\cosh(2x))+C$}

\exercise{$\ds \int \cosh (3x-7)\dd x$}{$\frac13\sinh(3x-7)+C$}

\exercise{$\ds \int \sinh x\cosh x\dd x$}{$\frac12\sinh^2x+C$ or $1/2\cosh^2x+C$}

\exercise{$\ds \int \frac{1}{9-x^2}\dd x$}{$\begin{cases}\frac13\tanh^{-1}\left(\frac x3\right)+C & x^2<9 \\
\frac13\coth^{-1}\left(\frac x3\right)+C & 9<x^2 \end{cases}
= \frac12\ln\abs{x+1} - \frac12\ln\abs{x-1}+C$}

\exercise{$\ds \int \frac{2x}{\sqrt{x^4-4}}\dd x$}{$\cosh^{-1} (x^2/2) + C = \ln (x^2+\sqrt{x^4-4})+C$}

\exercise{$\ds \int \frac{\sqrt{x}}{\sqrt{1+x^3}}\dd x$}{$2/3\sinh^{-1} x^{3/2} + C = 2/3\ln (x^{3/2}+\sqrt{x^3+1})+C$}

\exercise{$\ds \int \frac{e^x}{e^{2x}+1}\dd x$}{$\tan^{-1}(e^x)+C$}

\exercise{$\ds \int \sech x\dd x$ \quad(Hint: multiply by $\frac{\cosh x}{\cosh x}$; set $u = \sinh x$.)}{$\tan^{-1}(\sinh x)+C$}

\exercisesetend

%\exercisesetinstructions{, evaluate the given definite integral.}

\exercise{$\ds \int_{-1}^1 \sinh x\dd x$}{$0$}

\exercise{$\ds \int_{-\ln 2}^{\ln 2} \cosh x\dd x$}{$3/2$}

\exercisesetend

%\printreview
%\exercisesetinstructions{, use the Fundamental Theorem of Calculus Part 1 to find $F'(x)$.}

\exercise{$\ds F(x) = \int_2^{x^3+x} \frac{1}{t}\dd t$}{$F'(x) = (3x^2+1)\frac{1}{x^3+x}$}

\exercise{$\ds F(x) = \int_{x^3}^{0} t^3\dd t$}{$F'(x) = -3x^{11}$}

\exercise{$\ds F(x) = \int_{x}^{x^2} (t+2)\dd t$}{$F'(x) = 2x(x^2+2)-(x+2)$}

\exercise{$\ds F(x) = \int_{\ln x}^{e^x} \sin t\dd t$}{$F'(x) = e^x\sin (e^x) - \frac1x \sin(\ln x)$}

\exercise{$\ds F(x)=\int_1^{x} \frac{\ln t+4}{t^2+7}\dd t$}{$F'(x)=\frac{\ln x+4}{x^2+7}$}

\exercise{$\ds F(x)= \int_2^{\sin x} \cos^3 t+3\tan^3 t\dd t$}{$F'(x)=[\cos^3(\sin x)+3\tan^3(\sin x)]\cos x $}

\exercise{$\ds F(x)= \int_{5x^3}^4 \frac{\sqrt{\cos t+5}}{t^2+e^t}\dd t$}{$F'(x)=-\frac{15x^2\sqrt{\cos (5x^3)+5}}{25x^6+e^{5x^3}}$}

\exercise{$\ds  F(x)=\int_{\tan^2 x}^{10} \ln t +e^{t^2-7}\dd t$}{$F'(x)= -2\tan x\sec^2 x[\ln (\tan^2x) +e^{\tan^4x-7}]$}

\exercisesetend


\exercise{In probability theory, the lifetimes of certain devices (e.g. certain types of fuses and light bulbs) are modeled by an \emph{Exponential Distribution}.
\begin{enumext}[start=1]
\item The probability that a device lasts more than $a$ (time units) is $\int_a^\infty\lambda e^{-\lambda x}\dd x$ where $\lambda$ is a parameter that depends on the type of device. Evaluate this integral.
\item The expected lifetime of the device is given by $\int_0^\infty x\lambda e^{-\lambda x}\dd x$. Evaluate this integral.
\item What is the probability that a given device lasts more than the expected lifetime for such devices?
\end{enumext}}{\mbox{}\\[-2\baselineskip]\parbox[t]{\linewidth}{\begin{enumext}[start=1]
\item $e^{-\lambda a}$
\item $\frac1\lambda$
\item $e^{-1}$
\end{enumext}}}

\exercise{For $n>0$, the gamma function\index{gamma function} is defined by $\ds\Gamma(n)=\int_0^\infty x^{n-1}e^{-x}\dd x$.
\begin{enumext}[start=1]
\item Show that $\Gamma(1)=1$.
\item Show that $\Gamma(n+1)=n\Gamma(n)$ for $n>1$.
\item Conclude that $\Gamma(n+1)=n!$ for integers $n$ such that $n\ge 1$.
\item Show that this converges for $0<n<1$.
\end{enumext}}{}
