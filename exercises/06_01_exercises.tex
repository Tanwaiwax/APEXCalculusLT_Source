\printconcepts

\exercise{Substitution ``undoes'' what derivative rule?}{Chain Rule.}

\exercise{T/F: One can sometimes use algebra to rewrite the integrand of an integral to make it easier to evaluate.}{T}

\printproblems

\exerciseset{In Exercises}{, evaluate the indefinite integral.}{

% was set 1
\exercise{$\ds \int 3 x^2 \left(x^3-5\right)^7\ dx$}{$\frac18(x^3-5)^8+C$}

\exercise{$\ds \int (2 x-5) \left(x^2-5 x+7\right)^3\ dx$}{$\frac14(x^2-5x+7)^4+C$}
\exercise{$\ds \int x \left(x^2+1\right)^8\ dx$}{$\frac{1}{18} \left(x^2+1\right)^9+C$}
\exercise{$\ds \int (12 x+14) \left(3 x^2+7 x-1\right)^5\ dx$}{$\frac13(3 x^2+7 x-1)^6+C$}
\exercise{$\ds \int \frac{1}{2 x+7}\ dx$}{$\frac{1}{2} \ln |2 x+7|+C$}
\exercise{$\ds \int \frac{1}{\sqrt{2 x+3}}\ dx$}{$\sqrt{2 x+3}+C$}
\exercise{$\ds \int \frac{x}{\sqrt{x+3}}\ dx$}{$\frac23(x+3)^{3/2}-6(x+3)^{1/2}+C = \frac{2}{3} (x-6) \sqrt{x+3}+C$}
\exercise{$\ds \int \frac{x^3-x}{\sqrt{x}}\ dx$}{$\frac{2}{21} x^{3/2} \left(3 x^2-7\right)+C$}
\exercise{$\ds \int \frac{e^{\sqrt{x}}}{\sqrt{x}}\ dx$}{$2 e^{\sqrt{x}}+C$}
\exercise{$\ds \int \frac{x^4}{\sqrt{x^5+1}}\ dx$}{$\frac{2 \sqrt{x^5+1}}{5}+C$}
\exercise{$\ds \int \frac{\frac{1}{x}+1}{x^2}\ dx$}{$-\frac{1}{2 x^2}-\frac{1}{x}+C$}
\exercise{$\ds \int \frac{\ln (x)}{x}\ dx$}{$\frac{\ln ^2(x)}{2}+C$}

% was set 2
\exercise{$\ds \int \sin ^2(x) \cos (x)\ dx$}{$\frac{\sin ^3(x)}{3}+C$}
\exercise{$\ds \int \cos (3-6 x)\ dx$}{$-\frac16\sin(3-6x)+C$}
\exercise{$\ds \int \sec ^2(4-x)\ dx$}{$-\tan (4-x)+C$}
\exercise{$\ds \int \sec (2 x)\ dx$}{$\frac{1}{2} \ln |\sec (2x)+\tan(2x)|+C$}
\exercise{$\ds \int \tan ^2(x) \sec ^2(x)\ dx$}{$\frac{\tan ^3(x)}{3}+C$}
\exercise{$\ds \int x \cos \left(x^2\right)\ dx$}{$\frac{\sin \left(x^2\right)}{2}+C$}
\exercise{$\ds \int \cot x\ dx$. Do not just refer to \autoref{thm:triganti} for the answer; justify it through Substitution.}{The key is to rewrite $\cot x$ as $\cos x/\sin x$, and let $u=\sin x$.}

\exercise{$\ds \int \csc x\ dx$. Do not just refer to \autoref{thm:triganti} for the answer; justify it through Substitution.}{The key is to multiply $\csc x$ by 1 in the form $(\csc x+\cot x)/(\csc x+\cot x)$.}

% was set 3
\exercise{$\ds \int e^{3 x-1}\ dx$}{$\frac{1}{3} e^{3 x-1}+C$}
\exercise{$\ds \int e^{x^3} x^2\ dx$}{$\frac{e^{x^3}}{3}+C$}
\exercise{$\ds \int e^{x^2-2 x+1} (x-1)\ dx$}{$\frac{1}{2} e^{(x-1)^2}+C$}
\exercise{$\ds \int \frac{e^x+1}{e^x}\ dx$}{$x-e^{-x}+C$}
\exercise{$\ds \int \frac{e^x-e^{-x}}{e^{2x}}\ dx$}{$\frac{e^{-3 x}}{3}-e^{-x}+C$}

% was set 4
\exercise{$\ds \int \frac{\ln x}{x}\ dx$}{$\frac12\ln^2(x)+C$}
\exercise{$\ds \int \frac{\big(\ln x\big)^2}{x}\ dx$}{$\frac{\big(\ln x\big)^3}{3}+C$}
\exercise{$\ds \int \frac{\ln \left(x^3\right)}{x}\ dx$}{$\frac{1}{6} \ln ^2\left(x^3\right)+C$}
\exercise{$\ds \int \frac{1}{x \ln \left(x^2\right)}\ dx$}{$\frac{1}{2} \ln \left(\ln \left(x^2\right)\right)+C$}

% was set 7
\exercise{$\ds \int \frac{x^2}{\left(x^3+3\right)^2}\ dx$}{$-\frac{1}{3 \left(x^3+3\right)}+C$}
\exercise{$\ds \int \left(3 x^2+2 x\right) \left(5 x^3+5 x^2+2\right)^8\ dx$}{$\frac{1}{45}(5x^3+5x^2+2)^9+C$}
\exercise{$\ds \int \frac{x}{\sqrt{1-x^2}}\ dx$}{$-\sqrt{1-x^2}+C$}
\exercise{$\ds \int x^2 \csc ^2\left(x^3+1\right)\ dx$}{$-\frac{1}{3} \cot \left(x^3+1\right)+C$}
\exercise{$\ds \int \sin (x) \sqrt{\cos (x)}\ dx$}{$-\frac{2}{3} \cos ^{\frac{3}{2}}(x)+C$}
\exercise{$\ds \int \frac{1}{x-5}\ dx$}{$\ln |x-5|+C$}
\exercise{$\ds \int \frac{7}{3 x+2}\ dx$}{$\frac{7}{3} \ln |3 x+2|+C$}
\exercise{$\ds \int \frac{2 x+7}{x^2+7 x+3}\ dx$}{$\ln \left|x^2+7 x+3\right|+C$}
\exercise{$\ds \int \frac{9 (2 x+3)}{3 x^2+9 x+7}\ dx$}{$3 \ln \left|3 x^2+9 x+7\right|+C$}
\exercise{$\ds \int \frac{3 x-3}{\sqrt{x^2-2 x-6}}\ dx $}{$3 \sqrt{x^2-2 x-6}+C$}
\exercise{$\ds \int \frac{x-3}{\sqrt{x^2-6 x+8}}\ dx $}{$\sqrt{x^2-6 x+8}+C$}
\exercise{$\ds\int \frac{\cos\sqrt x}{\sqrt x}\ dx$}{$2\sin\sqrt x+C$}
\exercise{$\ds\int \sec^2\theta\tan\theta\ d\theta$}{$\frac12\sec^2\theta +C$ or $\frac12\tan^2\theta+C$}

\exercise{$\ds\int x\sqrt{2x+3}\ dx$}{$\frac1{10}(2x+3)^{5/2}-\frac12(2x+3)^{3/2}+C$}
\exercise{$\ds\int\frac{x^3}{(x^2+1)^3}\ dx$}{$-\frac1{2(x^2+1)}+\frac1{4(x^2+1)^2}+C$}
\exercise{$\ds\int\frac{2x^5}{x^2+1}\ dx$}{$\frac12(x^2+1)^2-2(x^2+1)+\ln(x^2+1)+C$}
\exercise{$\ds\int3x^8(x^3+2)^8\ dx$}{$\frac1{11}(x^3+2)^{11}-\frac25(x^3+2)^{10}+\frac49(x^3+2)^9+C$}

\exercise{$\ds\int\sin\bigl(\frac{x}{3}\bigr)\ dx$}{$-3 \cos(\frac{x}{3}) + C$}

\exercise{$\ds\int\sin^5\bigl(\frac{x}{4}\bigr)\cos\bigl(\frac{x}{4}\bigr)\ dx$}{$\frac{2}{3}\sin^6(\frac{x}{4}) + C$}

\exercise{$\ds\int x^{1/2}\cos(x^{3/2}+1)\ dx$}{$\frac23\sin(x^{3/2}+1)+C$}

}


%\exerciseset{Exercises}{ were removed from 6.1 and are free for the taking (commented ones have already been taken?).}{

\exercise{$\ds \int e^{3 x-1}\dd x $}{$\frac{1}{3} e^{3 x-1}+C$}
\exercise{$\ds \int e^{x^3} x^2\dd x $}{$\frac{e^{x^3}}{3}+C$}
\exercise{$\ds \int e^{x^2-2 x+1} (x-1)\dd x $}{$\frac{1}{2} e^{(x-1)^2}+C$}
\exercise{$\ds \int \frac{e^x+1}{e^x}\dd x $}{$x-e^{-x}+C$}
\exercise{$\ds \int \frac{e^x-e^{-x}}{e^{2x}}\dd x $}{$\frac{e^{-3 x}}{3}-e^{-x}+C$}
%\exercise{$\ds \int 3^{3 x}\dd x $}{$\frac{27^x}{\ln 27}+C$}
%\exercise{$\ds \int 4^{2 x}\dd x $}{$\frac{16^x}{\ln (16)}+C$}
}

%\exerciseset{In Exercises}{, use Substitution to evaluate the indefinite integral involving logarithmic functions.}{

\exercise{$\ds \int \frac{\ln x}{x} dx $}{$\frac12\ln^2(x)+C$}
\exercise{$\ds \int \frac{\big(\ln x\big)^2}{x} dx $}{$\frac{\big(\ln x\big)^3}{3}+C$}
\exercise{$\ds \int \frac{\ln \left(x^3\right)}{x} dx $}{$\frac{1}{6} \ln ^2\left(x^3\right)+C$}
\exercise{$\ds \int \frac{1}{x \ln \left(x^2\right)} dx $}{$\frac{1}{2} \ln \left(\ln \left(x^2\right)\right)+C$}}

%%\exerciseset{In Exercises}{, use Substitution to evaluate the indefinite integral involving rational functions.}{

\exercise{$\ds \int \frac{x^2+3 x+1}{x} dx $}{$\frac{x^2}{2}+3 x+\ln |x|+C$}
\exercise{$\ds \int \frac{x^3+x^2+x+1}{x} dx $}{$\frac{x^3}{3}+\frac{x^2}{2}+x+\ln |x|+C$}
\exercise{$\ds \int \frac{x^3-1}{x+1} dx $}{$\frac{x^3}{3}-\frac{x^2}{2}+x-2 \ln |x+1|+C$}
\exercise{$\ds \int \frac{x^2+2 x-5}{x-3} dx $}{$\frac{1}{2} \left(x^2+10 x+20 \ln |x-3|\right)+C$}
\exercise{$\ds \int \frac{3 x^2-5 x+7}{x+1} dx $}{$\frac{3}{2}x^2-8x+15 \ln |x+1|+C$}
\exercise{$\ds \int \frac{x^2+2 x+1}{x^3+3 x^2+3 x} dx $}{$\frac{1}{3} \ln \left|x^2+3 x+3\right|+\frac{\ln |x|}{3}+C$}}

%%\exerciseset{In Exercises}{, use Substitution to evaluate the indefinite integral involving inverse trigonometric functions.}{

\exercise{$\ds \int \frac{7}{x^2+7} dx $}{$\sqrt{7} \tan ^{-1}\left(\frac{x}{\sqrt{7}}\right)+C$}
\exercise{$\ds \int \frac{3}{\sqrt{9-x^2}} dx $}{$3 \sin ^{-1}\left(\frac{x}{3}\right)+C$}
\exercise{$\ds \int \frac{14}{\sqrt{5-x^2}} dx $}{$14 \sin ^{-1}\left(\frac{x}{\sqrt{5}}\right)+C$}
\exercise{$\ds \int \frac{2}{x \sqrt{x^2-9}} dx $}{$\frac23\sec^{-1}(|x|/3)+C$}
\exercise{$\ds \int \frac{5}{\sqrt{x^4-16 x^2}} dx $}{$\frac54\sec^{-1}(|x|/4)+C$}
\exercise{$\ds \int \frac{x}{\sqrt{1-x^4}} dx $}{$\frac{1}{2} \sin ^{-1}\left(x^2\right)+C$}
\exercise{$\ds \int \frac{1}{x^2-2 x+8} dx $}{$\frac{\tan ^{-1}\left(\frac{x-1}{\sqrt{7}}\right)}{\sqrt{7}}+C$}
\exercise{$\ds \int \frac{2}{\sqrt{-x^2+6 x+7}} dx $}{$2 \sin ^{-1}\left(\frac{x-3}{4}\right)+C$}
\exercise{$\ds \int \frac{3}{\sqrt{-x^2+8 x+9}} dx $}{$3 \sin ^{-1}\left(\frac{x-4}{5}\right)+C$}
\exercise{$\ds \int \frac{5}{x^2+6 x+34} dx $}{$\tan ^{-1}\left(\frac{x+3}{5}\right)+C$}}

%\exerciseset{In Exercises}{, evaluate the indefinite integral.}{

\exercise{$\ds \int \frac{x^2}{\left(x^3+3\right)^2} dx $}{$-\frac{1}{3 \left(x^3+3\right)}+C$}
\exercise{$\ds \int \left(3 x^2+2 x\right) \left(5 x^3+5 x^2+2\right)^8 dx $}{$\frac{1}{45}(5x^3+5x^2+2)^9+C$}
\exercise{$\ds \int \frac{x}{\sqrt{1-x^2}} dx $}{$-\sqrt{1-x^2}+C$}
\exercise{$\ds \int x^2 \csc ^2\left(x^3+1\right) dx $}{$-\frac{1}{3} \cot \left(x^3+1\right)+C$}
\exercise{$\ds \int \sin (x) \sqrt{\cos (x)} dx $}{$-\frac{2}{3} \cos ^{\frac{3}{2}}(x)+C$}
\exercise{$\ds \int \frac{1}{x-5} dx $}{$\ln |x-5|+C$}
\exercise{$\ds \int \frac{7}{3 x+2} dx $}{$\frac{7}{3} \ln |3 x+2|+C$}
%\exercise{$\ds \int \frac{3 x^3+4 x^2+2 x-22}{x^2+3 x+5} dx $}{$\frac{3 x^2}{2}+\ln \left|x^2+3 x+5\right|-5 x+C$}
\exercise{$\ds \int \frac{2 x+7}{x^2+7 x+3} dx $}{$\ln \left|x^2+7 x+3\right|+C$}
\exercise{$\ds \int \frac{9 (2 x+3)}{3 x^2+9 x+7} dx $}{$3 \ln \left|3 x^2+9 x+7\right|+C$}
%\exercise{$\ds \int \frac{-x^3+14 x^2-46 x-7}{x^2-7 x+1} dx $}{$-\frac{x^2}{2}+2 \ln \left|x^2-7 x+1\right|+7 x+C$}
%\exercise{$\ds \int \frac{x}{x^4+81} dx $}{$\frac{1}{18} \tan ^{-1}\left(\frac{x^2}{9}\right)+C$}
%\exercise{$\ds \int \frac{2}{4 x^2+1} dx $}{$\tan ^{-1}(2 x)+C$}
%\exercise{$\ds \int \frac{1}{x \sqrt{4 x^2-1}} dx $}{$\sec^{-1}(|2x|)+C$}
%\exercise{$\ds \int \frac{1}{\sqrt{16-9 x^2}} dx $}{$\frac{1}{3} \sin ^{-1}\left(\frac{3 x}{4}\right)+C$}
%\exercise{$\ds \int \frac{3 x-2}{x^2-2 x+10} dx $}{$\frac{3}{2} \ln \left|x^2-2 x+10\right|+\frac{1}{3} \tan ^{-1}\left(\frac{x-1}{3}\right)+C$}
%\exercise{$\ds \int \frac{7-2 x}{x^2+12 x+61} dx $}{$\frac{19}{5} \tan ^{-1}\left(\frac{x+6}{5}\right)-\ln \left|x^2+12 x+61\right|+C$}
%\exercise{$\ds \int \frac{x^2+5 x-2}{x^2-10 x+32} dx $}{$\frac{15}{2} \ln \left|x^2-10 x+32\right|+x+\frac{41 \tan ^{-1}\left(\frac{x-5}{\sqrt{7}}\right)}{\sqrt{7}}+C$}
%\exercise{$\ds \int \frac{x^3}{x^2+9} dx $}{$\frac{x^2}{2}-\frac{9}{2} \ln \left|x^2+9\right|+C$}
%\exercise{$\ds \int \frac{x^3-x}{x^2+4 x+9} dx $}{$\frac{x^2}{2}+3 \ln \left|x^2+4 x+9\right|-4 x+\frac{24 \tan ^{-1}\left(\frac{x+2}{\sqrt{5}}\right)}{\sqrt{5}}+C$}
%\exercise{$\ds \int \frac{\sin (x)}{\cos ^2(x)+1} dx $}{$-\tan ^{-1}(\cos (x))+C$}
%\exercise{$\ds \int \frac{\cos (x)}{\sin ^2(x)+1} dx $}{$\tan ^{-1}(\sin (x))+C$}
%\exercise{$\ds \int \frac{\cos (x)}{1-\sin ^2(x)} dx $}{$\ln|\sec x+\tan x|+C$ (integrand simplifies to $\sec x$)}
\exercise{$\ds \int \frac{3 x-3}{\sqrt{x^2-2 x-6}} dx $}{$3 \sqrt{x^2-2 x-6}+C$}
\exercise{$\ds \int \frac{x-3}{\sqrt{x^2-6 x+8}} dx $}{$\sqrt{x^2-6 x+8}+C$}}

{\noindent In Exercises}
{, evaluate the definite integral.}
\exinput{exercises/06_01_ex_74}
\exinput{exercises/06_01_ex_75}
\exinput{exercises/06_01_ex_76}
\exinput{exercises/06_01_ex_77}
\exinput{exercises/06_01_ex_78}
\exinput{exercises/06_01_ex_86}
\exinput{exercises/06_01_ex_87}
\exinput{exercises/06_01_ex_88}
\exinput{exercises/06_01_ex_89}
\exinput{exercises/06_01_ex_90}
\exinput{exercises/06_01_ex_91}


%\printreview

%\exerciseset{In Exercises}{, use the Fundamental Theorem of Calculus Part 1 to find $F'(x)$.
}{

\exercise{$\ds F(x) = \int_2^{x^3+x} \frac{1}{t}\ dt$
}{$F'(x) = (3x^2+1)\frac{1}{x^3+x}$
}

\exercise{$\ds F(x) = \int_{x^3}^{0} t^3\ dt$
}{$F'(x) = 3x^{11}$
}

\exercise{$\ds F(x) = \int_{x}^{x^2} (t+2)\ dt$
}{$F'(x) = 2x(x^2+2)-(x+2)$
}

\exercise{$\ds F(x) = \int_{\ln x}^{e^x} \sin t\ dt$
}{$F'(x) = e^x\sin (e^x) - 1/x\sin(\ln x)$
}
}

% set 2 is the rejects from u-sub (trig int, trig sub, part frac, )

%\exerciseset{Exercises}{ were removed from 6.1 and are free for the taking.}{

\exercise{$\ds \int \tan ^2(x) dx $}{$\tan (x)-x+C$}
\exercise{$\ds \int 3^{3 x} dx $}{$\frac{27^x}{\ln 27}+C$}
\exercise{$\ds \int 4^{2 x} dx $}{$\frac{16^x}{\ln (16)}+C$}
\exercise{$\ds \int \frac{x^2+3 x+1}{x} dx $}{$\frac{x^2}{2}+3 x+\ln |x|+C$}
\exercise{$\ds \int \frac{x^3+x^2+x+1}{x} dx $}{$\frac{x^3}{3}+\frac{x^2}{2}+x+\ln |x|+C$}
\exercise{$\ds \int \frac{x^3-1}{x+1} dx $}{$\frac{x^3}{3}-\frac{x^2}{2}+x-2 \ln |x+1|+C$}
\exercise{$\ds \int \frac{x^2+2 x-5}{x-3} dx $}{$\frac{1}{2} \left(x^2+10 x+20 \ln |x-3|\right)+C$}
\exercise{$\ds \int \frac{3 x^2-5 x+7}{x+1} dx $}{$\frac{3}{2}x^2-8x+15 \ln |x+1|+C$}
\exercise{$\ds \int \frac{x^2+2 x+1}{x^3+3 x^2+3 x} dx $}{$\frac{1}{3} \ln \left|x^2+3 x+3\right|+\frac{\ln |x|}{3}+C$}
\exercise{$\ds \int \frac{7}{x^2+7} dx $}{$\sqrt{7} \tan ^{-1}\left(\frac{x}{\sqrt{7}}\right)+C$}
\exercise{$\ds \int \frac{3}{\sqrt{9-x^2}} dx $}{$3 \sin ^{-1}\left(\frac{x}{3}\right)+C$}
\exercise{$\ds \int \frac{14}{\sqrt{5-x^2}} dx $}{$14 \sin ^{-1}\left(\frac{x}{\sqrt{5}}\right)+C$}
\exercise{$\ds \int \frac{2}{x \sqrt{x^2-9}} dx $}{$\frac23\sec^{-1}(|x|/3)+C$}
\exercise{$\ds \int \frac{5}{\sqrt{x^4-16 x^2}} dx $}{$\frac54\sec^{-1}(|x|/4)+C$}
\exercise{$\ds \int \frac{x}{\sqrt{1-x^4}} dx $}{$\frac{1}{2} \sin ^{-1}\left(x^2\right)+C$}
\exercise{$\ds \int \frac{1}{x^2-2 x+8} dx $}{$\frac{\tan ^{-1}\left(\frac{x-1}{\sqrt{7}}\right)}{\sqrt{7}}+C$}
\exercise{$\ds \int \frac{2}{\sqrt{-x^2+6 x+7}} dx $}{$2 \sin ^{-1}\left(\frac{x-3}{4}\right)+C$}
\exercise{$\ds \int \frac{3}{\sqrt{-x^2+8 x+9}} dx $}{$3 \sin ^{-1}\left(\frac{x-4}{5}\right)+C$}
\exercise{$\ds \int \frac{5}{x^2+6 x+34} dx $}{$\tan ^{-1}\left(\frac{x+3}{5}\right)+C$}
\exercise{$\ds \int \frac{3 x^3+4 x^2+2 x-22}{x^2+3 x+5} dx $}{$\frac{3 x^2}{2}+\ln \left|x^2+3 x+5\right|-5 x+C$}
\exercise{$\ds \int \frac{-x^3+14 x^2-46 x-7}{x^2-7 x+1} dx $}{$-\frac{x^2}{2}+2 \ln \left|x^2-7 x+1\right|+7 x+C$}
\exercise{$\ds \int \frac{x}{x^4+81} dx $}{$\frac{1}{18} \tan ^{-1}\left(\frac{x^2}{9}\right)+C$}
\exercise{$\ds \int \frac{2}{4 x^2+1} dx $}{$\tan ^{-1}(2 x)+C$}
\exercise{$\ds \int \frac{1}{x \sqrt{4 x^2-1}} dx $}{$\sec^{-1}(|2x|)+C$}
\exercise{$\ds \int \frac{1}{\sqrt{16-9 x^2}} dx $}{$\frac{1}{3} \sin ^{-1}\left(\frac{3 x}{4}\right)+C$}
\exercise{$\ds \int \frac{3 x-2}{x^2-2 x+10} dx $}{$\frac{3}{2} \ln \left|x^2-2 x+10\right|+\frac{1}{3} \tan ^{-1}\left(\frac{x-1}{3}\right)+C$}
\exercise{$\ds \int \frac{7-2 x}{x^2+12 x+61} dx $}{$\frac{19}{5} \tan ^{-1}\left(\frac{x+6}{5}\right)-\ln \left|x^2+12 x+61\right|+C$}
\exercise{$\ds \int \frac{x^2+5 x-2}{x^2-10 x+32} dx $}{$\frac{15}{2} \ln \left|x^2-10 x+32\right|+x+\frac{41 \tan ^{-1}\left(\frac{x-5}{\sqrt{7}}\right)}{\sqrt{7}}+C$}
\exercise{$\ds \int \frac{x^3}{x^2+9} dx $}{$\frac{x^2}{2}-\frac{9}{2} \ln \left|x^2+9\right|+C$}
\exercise{$\ds \int \frac{x^3-x}{x^2+4 x+9} dx $}{$\frac{x^2}{2}+3 \ln \left|x^2+4 x+9\right|-4 x+\frac{24 \tan ^{-1}\left(\frac{x+2}{\sqrt{5}}\right)}{\sqrt{5}}+C$}
\exercise{$\ds \int \frac{\sin (x)}{\cos ^2(x)+1} dx $}{$-\tan ^{-1}(\cos (x))+C$}
\exercise{$\ds \int \frac{\cos (x)}{\sin ^2(x)+1} dx $}{$\tan ^{-1}(\sin (x))+C$}
\exercise{$\ds \int \frac{\cos (x)}{1-\sin ^2(x)} dx $}{$\ln|\sec x+\tan x|+C$ (integrand simplifies to $\sec x$)}
\exercise{$\ds \int_{-1}^{1} \frac{1}{1+x^2}\ dx $}{$\pi/2$}
\exercise{$\ds \int_{2}^{4} \frac{1}{x^2-6x+10}\ dx $}{$\pi/2$}
\exercise{$\ds \int_{1}^{\sqrt{3}} \frac{1}{\sqrt{4-x^2}}\ dx $}{$\pi/6$}
}

% the Beta function problems were in Jacobian
% but they're not
% but they're improper, which is several sections later
% so we'll cut them
%
%\exercise{Show that the \emph{Beta function}\index{Beta function}, defined by
%\[B(x,y) = \int_0^1 t^{x-1} (1-t)^{y-1} \dd t ,\quad\text{for $x > 0$, $y > 0$,}\]
%satisfies the relation $B(y,x)=B(x,y)$ for $x,y>0$.}{}
%
%\exercise{Using the substitution $t=u/(u+1)$, show that the Beta function can be written as
%\[ B(x,y)=\int_0^\infty\frac{u^{x-1}}{(u+1)^{x+y}}\,du ,\quad\text{for $x,y>0$.}\]}{}
