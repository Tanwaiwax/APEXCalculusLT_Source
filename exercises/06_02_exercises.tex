\printconcepts

\exercise{T/F: Integration by Parts is useful in evaluating integrands that contain products of functions.}{T}

\exercise{T/F: Integration by Parts can be thought of as the ``opposite of the Chain Rule.''}{F}

% we're not a fan of this question
%\exercise{For what is ``LIATE'' useful?}{Determining which functions in the integrand to set equal to ``$u$'' and which to set equal to ``$dv$''.}

% cut for parity
%\exercise{T/F: If the integral that results from Integration by Parts appears to also need Integration by Parts, then a mistake was made in the original choice of ``u''.}{F}

\printproblems

\exerciseset{In Exercises}{, evaluate the given indefinite integral.}{

\exercise{\label{ibp_prob:4}$\ds \int x\sin x\ dx$}{$\sin x - x\cos x+C$
}

\exercise{$\ds \int xe^{-x}\ dx$}{$-e^{-x}-xe^{-x}+C$
}

\exercise{$\ds \int x^2\sin x\ dx$}{$-x^2\cos x+2x\sin x+2\cos x+C$
}

\exercise{$\ds \int x^3\sin x\ dx$}{$-x^3\cos x + 3x^2\sin x+6x\cos x-6\sin x+C$
}

\exercise{$\ds \int xe^{x^2}\ dx$}{$1/2e^{x^2}+C$
}

\exercise{$\ds \int x^3e^{x}\ dx$}{$x^3e^x-3x^2e^x+6xe^x-6e^x+C$
}

\exercise{$\ds \int xe^{-2x}\ dx$}{$-\frac{1}{2}x e^{-2 x}-\frac{e^{-2 x}}{4}+C$
}

\exercise{$\ds \int e^x\sin x\ dx$}{$1/2e^x(\sin x-\cos x)+C$
}

\exercise{\label{ibp_prob:12}$\ds \int e^{2x}\cos x\ dx$}{$1/5e^{2x}(\sin x+2\cos x)+C$
}

\exercise{$\ds \int e^{2x}\sin (3x)\ dx$}{$1/13e^{2x}(2\sin (3x)-3\cos (3x))+C$
}

\exercise{$\ds \int e^{5x}\cos (5x)\ dx$}{$1/10e^{5x}(\sin (5x)+\cos (5x))+C$
}

\exercise{$\ds \int \sin x\cos x\ dx$}{$-1/2\cos^2x+C$
}

\exercise{$\ds \int \sin^{-1} x\ dx$}{$\sqrt{1-x^2}+x \sin ^{-1}(x)+C$
}

\exercise{$\ds \int \tan^{-1} (2x)\ dx$}{$x \tan ^{-1}(2 x)-\frac{1}{4} \ln \left|4 x^2+1\right|+C$
}

\exercise{$\ds \int x\tan^{-1} x\ dx$}{$\frac{1}{2} x^2 \tan ^{-1}(x)-\frac{x}{2}+\frac{1}{2}
   \tan ^{-1}(x)+C$
}

\exercise{$\ds \int \sin^{-1} x\ dx$}{$\sqrt{1-x^2}+x\sin^{-1}x+C$
}

\exercise{$\ds \int x\ln x\ dx$}{$\frac{1}{2} x^2 \ln |x|-\frac{x^2}{4}+C$
}

\exercise{$\ds \int (x-2)\ln x\ dx$}{$-\frac{x^2}{4}+\frac{1}{2} x^2 \ln |x|+2 x-2 x \ln |x|+C$
}

\exercise{$\ds \int x\ln (x-1)\ dx$}{$-\frac{x^2}{4}+\frac{1}{2} x^2 \ln |x-1|-\frac{x}{2}-\frac{1}{2} \ln |x-1|+C$
}

\exercise{$\ds \int x\ln (x^2)\ dx$}{$\frac{1}{2} x^2 \ln \left(x^2\right)-\frac{x^2}{2}+C$
}

\exercise{$\ds \int x^2\ln x\ dx$}{$\frac{1}{3} x^3 \ln |x|-\frac{x^3}{9}+C$
}

\exercise{$\ds \int \left(\ln x\right)^2\ dx$}{$2 x+x \left(\ln |x|\right)^2-2 x \ln |x|+C$
}

\exercise{$\ds \int \left(\ln (x+1)\right)^2\ dx$}{$2 x+x \left(\ln |x+1|\right)+\left(\ln |x+1|\right)^2-$ $2 x \ln |x+1|-2 \ln |x+1|+2+C$
}

\exercise{$\ds \int x\sec^2x\ dx$}{$x \tan (x)+\ln |\cos (x)|+C$
}

\exercise{$\ds \int x\csc^2x\ dx$}{$\ln |\sin (x)|-x \cot (x)+C$
}

\exercise{$\ds \int x\sqrt{x-2}\ dx$}{$\frac 25(x-2)^{5/2}+\frac43 (x-2)^{3/2}+C$
}

\exercise{$\ds \int x\sqrt{x^2-2}\ dx$}{$\frac13(x^2-2)^{3/2}+C$
}

\exercise{$\ds \int \sec x\tan x\ dx$}{$\sec x+C$
}

\exercise{$\ds \int x\sec x\tan x\ dx$}{$x\sec x-\ln |\sec x+\tan x|+C$
}

\exercise{$\ds \int x\csc x\cot x\ dx$}{$-x\csc x-\ln |\csc x+\cot x|+C$
}
}

\begin{exerciseset}{In Exercises}{, evaluate the indefinite integral after first making a substitution.}

\exercise{$\ds \int \sin(\ln x)\dd x$}{$1/2x\big(\sin(\ln x)-\cos (\ln x)\big)+C$}

\exercise{$\ds \int \sin(\sqrt{x})\dd x$}{$2 \sin \left(\sqrt{x}\right)-2 \sqrt{x} \cos \left(\sqrt{x}\right)+C$}

\exercise{$\ds \int \ln(\sqrt{x})\dd x$}{$\frac{1}{2} x \ln |x|-\frac{x}{2}+C$}

\exercise{$\ds \int e^{\sqrt{x}}\dd x$}{$2\sqrt{x} e^{\sqrt{x}} -2 e^{\sqrt{x}}+C$}

\exercise{$\ds \int e^{\ln x}\dd x$}{$1/2x^2+C$}

\exercise{$\ds \int x^3 e^{x^2}\dd x$}{$\frac12 e^{x^2} (x^2-1)+C$}

% commented for parity
%\exercise{$\ds \int \frac{\sin(\frac{1}{x})}{x^3}\dd x$}{$\dfrac{\cos(\frac1x)}x-\sin(\frac1x)+C$}

\end{exerciseset}


\exerciseset{In Exercises}{, evaluate the definite integral. Note: the corresponding indefinite integrals appear in Exercises \ref{ibp_prob:4} -- \ref{ibp_prob:12}.}{

\exercise{$\ds \int_0^\pi x\sin x\ dx$}{$\pi$}

\exercise{$\ds \int_{-1}^1 xe^{-x}\ dx$}{$-2/e$}

\exercise{$\ds \int_{-\pi/4}^{\pi/4} x^2\sin x\ dx$}{$0$}

\exercise{$\ds \int_{-\pi/2}^{\pi/2} x^3\sin x\ dx$}{$\frac{3 \pi ^2}{2}-12$}

\exercise{$\ds \int_0^{\sqrt{\ln 2}} xe^{x^2}\ dx$}{$1/2$}

\exercise{$\ds \int_0^1 x^3e^{x}\ dx$}{$6-2e$}

\exercise{$\ds \int_1^2 xe^{-2x}\ dx$}{$\frac{3}{4 e^2}-\frac{5}{4 e^4}$}

\exercise{$\ds \int_0^{\pi} e^x\sin x\ dx$}{$\frac{1}{2}+\frac{e^{\pi }}{2}$}

\exercise{$\ds \int_{-\pi/2}^{\pi/2} e^{2x}\cos x\ dx$}{$1/5\left(e^{\pi}+e^{-\pi}\right)$}

% maybe
%\exercise{$\ds \int_{0}^{1} \tanh^{-1} x \ dx$}{$2$}

}


\exercise{\mbox{}\\[-2\baselineskip]\begin{enumerate}
 \item  For $n\geq2$ show that
 \[
  \int_0^{\pi/2}\sin^n x\ dx = \frac{n-1}n \int_0^{\pi/2}\sin^{n-2} x\,dx.
 \]
 Hint:  Begin by writing $\sin^n x$ as $(\sin^{n-1} x) \sin x$ and using Integration by Parts.
 \item  For $k \geq 1$ show that
 \begin{align*}
  \int_0^{\pi/2}\sin^{2k+1}x\ dx&=\frac{2\cdot4\cdot6\dotsm(2k)}{1\cdot3\cdot5\cdot7\dotsm(2k+1)}\qquad\text{and}\\
  \int_0^{\pi/2}\sin^{2k}x\ dx&=\frac{1\cdot3\cdot5\dotsm(2k-1)}{2\cdot4\cdot6\dotsm(2k)}\frac\pi2.
 \end{align*}
\end{enumerate}}{}

\exercise{Find the volume of the solid of revolution obtained by rotating the region bounded by $y=0$, $y=\ln x$, $x=1$, and $x=e$:
\begin{enumerate}
 \item About the $x$-axis, using the disk method.
 \item About the $y$-axis, using the shell method.
\end{enumerate}}{\mbox{}\\[-2\baselineskip]\begin{enumerate}\item $\pi(e-2)$ \item $\frac\pi2(e^2+1)$\end{enumerate}}

%\printreview

%\exerciseset{In Exercises}{, use the Fundamental Theorem of Calculus Part 1 to find $F'(x)$.
}{

\exercise{$\ds F(x) = \int_2^{x^3+x} \frac{1}{t}\ dt$
}{$F'(x) = (3x^2+1)\frac{1}{x^3+x}$
}

\exercise{$\ds F(x) = \int_{x^3}^{0} t^3\ dt$
}{$F'(x) = 3x^{11}$
}

\exercise{$\ds F(x) = \int_{x}^{x^2} (t+2)\ dt$
}{$F'(x) = 2x(x^2+2)-(x+2)$
}

\exercise{$\ds F(x) = \int_{\ln x}^{e^x} \sin t\ dt$
}{$F'(x) = e^x\sin (e^x) - 1/x\sin(\ln x)$
}
}
