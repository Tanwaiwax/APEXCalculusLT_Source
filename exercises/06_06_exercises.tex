\printconcepts

\exercise{List the different indeterminate forms described in this section.}{$0/0, \infty/\infty, 0\cdot\infty,\infty-\infty,0^0,1^\infty,\infty^0$}

\exercise{T/F: L'Hôpital's Rule provides a faster method of computing derivatives.}{F}

\exercise{T/F: L'Hôpital's Rule states that $\ds \frac{\dd}{\dd x}\left[\frac{f(x)}{g(x)}\right] = \frac{\fp(x)}{g'(x)}.$}{F}

\exercise{Explain what the indeterminate form ``$1^\infty$'' means.}{The base of an expression is approaching 1 while its power is growing without bound.}

\exercise{Fill in the blanks: The Quotient Rule is applied to $\ds \frac{f(x)}{g(x)}$ when taking \underline{\hskip .5in}; L'Hôpital's Rule is applied when taking certain \underline{\hskip .5in}.}{derivatives; limits}

\exercise{Create (but do not evaluate!) a limit that returns ``$\infty^0$''.}{Answers will vary.}

\exercise{Create a function $f(x)$ such that $\ds \lim_{x\to 1}f(x)$ returns ``$0^0$''.}{Answers will vary.}

\exercise{Create a function $f(x)$ such that $\ds \lim_{x\to \infty}f(x)$ returns ``$0\cdot\infty$''.}{Answers will vary.}

\printproblems

% todo Create exercises for L'Hopital's rule with the indeterminate form infinity-infinity
% todo Create exercises for L'Hopital's rule with limits other than 0, 1, or infinity

\exerciseset{In Exercises}{, evaluate the given limit.}{

\exercise{$\ds \lim_{x\to 1} \frac{x^2+x-2}{x-1}$}{3}

\exercise{$\ds \lim_{x\to 2} \frac{x^2+x-6}{x^2-7x+10}$}{$-5/3$}

\exercise{$\ds \lim_{x\to \pi} \frac{\sin x}{x-\pi}$}{$-1$}

\exercise{$\ds \lim_{x\to \pi/4} \frac{\sin x-\cos x}{\cos (2x)}$}{$-\sqrt{2}/2$}

\exercise{$\ds \lim_{x\to 0} \frac{\sin (5x)}{x}$}{$5$}

\exercise{$\ds \lim_{x\to 0} \frac{\sin (2x)}{x+2}$}{$0$}

%\exercise{$\ds \lim_{x\to 0} \frac{\sin (2x)}{\sin (3x)}$}{$2/3$}

\exercise{$\ds \lim_{x\to 0} \frac{\sin (ax)}{\sin (bx)}$}{$a/b$}

\exercise{$\ds \lim_{x\to 0^+} \frac{e^x-1}{x^2}$}{$\infty$}

\exercise{$\ds \lim_{x\to 0^+} \frac{e^x-x-1}{x^2}$}{$1/2$}

\exercise{$\ds \lim_{x\to 0^+} \frac{x-\sin x}{x^3-x^2}$}{$0$}

\exercise{$\ds \lim_{x\to \infty} \frac{x^4}{e^x}$}{$0$}

\exercise{$\ds \lim_{x\to \infty} \frac{\sqrt{x}}{e^x}$}{$0$}

\exercise{$\ds \lim_{x\to \infty} \frac{e^x}{\sqrt{x}}$}{$\infty$}

\exercise{$\ds \lim_{x\to \infty} \frac{e^x}{2^x}$}{$\infty$}

\exercise{$\ds \lim_{x\to \infty} \frac{e^x}{3^x}$}{$0$}

\exercise{$\ds \lim_{x\to 3} \frac{x^3-5x^2+3x+9}{x^3-7x^2+15x-9}$}{$2$}

\exercise{$\ds \lim_{x\to -2} \frac{x^3+4x^2+4x}{x^3+7x^2+16x+12}$}{$-2$}

\exercise{$\ds \lim_{x\to \infty} \frac{\ln x}{x}$}{$0$}

\exercise{$\ds \lim_{x\to \infty} \frac{\ln (x^2)}{x}$}{$0$}

\exercise{$\ds \lim_{x\to \infty} \frac{\Big(\ln x\Big)^2}{x}$}{$0$}

\exercise{$\ds \lim_{x\to 0^+} x\ln x$}{$0$}

\exercise{$\ds \lim_{x\to 0^+} \sqrt{x}\ln x$}{$0$}

\exercise{$\ds \lim_{x\to 0^+} xe^{1/x}$}{$\infty$}

\exercise{$\ds \lim_{x\to \infty} \left(x^3-x^2\right)$}{$\infty$}

\exercise{$\ds \lim_{x\to \infty} \left(\sqrt{x}-\ln x\right)$}{$\infty$}

\exercise{$\ds \lim_{x\to -\infty} xe^x$}{$0$}

\exercise{$\ds \lim_{x\to 0^+} \frac{1}{x^2}e^{-1/x}$}{$0$}

\exercise{$\ds \lim_{x\to 0^+} (1+x)^{1/x}$}{$e$}

\exercise{$\ds \lim_{x\to 0^+} (2x)^{x}$}{$1$}

\exercise{$\ds \lim_{x\to 0^+} (2/x)^{x}$}{$1$}

\exercise{$\ds \lim_{x\to 0^+} (\sin x)^{x}$}{$1$} %{ \quad Hint: use the Squeeze Theorem.}

\exercise{$\ds \lim_{x\to 1^+} (1-x)^{1-x}$}{$1$}

\exercise{$\ds \lim_{x\to \infty} (x)^{1/x}$}{$1$}

\exercise{$\ds \lim_{x\to \infty} (1/x)^{x}$}{$0$}

\exercise{$\ds \lim_{x\to 1^+} (\ln x)^{1-x}$}{$1$}

\exercise{$\ds \lim_{x\to \infty} (1+x)^{1/x}$}{$1$}

\exercise{$\ds \lim_{x\to \infty} (1+x^2)^{1/x}$}{$1$}

\exercise{$\ds \lim_{x\to \pi/2} \tan x\cos x$}{$1$}

\exercise{$\ds \lim_{x\to \pi/2} \tan x\sin (2x)$}{$2$}

\exercise{$\ds \lim_{x\to 1^+} \left(\frac{1}{\ln x} - \frac{1}{x-1}\right)$}{$1/2$}

\exercise{$\ds \lim_{x\to 3^+} \left(\frac{5}{x^2-9} - \frac{x}{x-3}\right)$}{$-\infty$}

\exercise{$\ds \lim_{x\to \infty} x\tan (1/x)$}{$1$}

\exercise{$\ds \lim_{x\to \infty} \frac{(\ln x)^3}{x}$}{$0$}

\exercise{$\ds \lim_{x\to 1} \frac{x^2+x-2}{\ln x}$}{$3$}

}

%\input{exercises/06_05_exset_02}
%\begin{exerciseset}{In Exercises}{, find the equation of the line tangent to the function at the given $x$-value.}

\exercise{$f(x) = \sinh x$ at $x=0$}{$y=x$}

\exercise{$f(x) = \cosh x$ at $x=\ln 2$}{$y=3/4(x-\ln 2)+5/4$}

\exercise{$f(x) = \tanh x$ at $x=-\ln 3$}{$y=\frac9{25}(x+\ln 3)-\frac45$}

\exercise{$f(x) = \sech^2 x$ at $x=\ln 3$}{$y=-72/125(x-\ln 3)+9/25$}

\exercise{$f(x) = \sinh^{-1} x$ at $x=0$}{$y=x$}

\exercise{$f(x) = \cosh^{-1} x$ at $x=\sqrt 2$}{$y=(x-\sqrt{2})+\cosh^{-1}(\sqrt{2}) \approx (x-1.414)+0.881$}

\end{exerciseset}

%\exerciseset{In Exercises}{, evaluate the given indefinite integral.}{

\exercise{$\ds \int \tanh (2x)\ dx$}{$\frac12\ln (\cosh(2x))+C$}

\exercise{$\ds \int \cosh (3x-7)\ dx$}{$\frac13\sinh(3x-7)+C$}

\exercise{$\ds \int \sinh x\cosh x\ dx$}{$\frac12\sinh^2x+C$ or $1/2\cosh^2x+C$}

\exercise{$\ds \int \frac{1}{9-x^2}\ dx$}{$\begin{cases}\frac13\tanh^{-1}\left(\frac x3\right)+C & x^2<9 \\
\frac13\coth^{-1}\left(\frac x3\right)+C & 9<x^2 \end{cases}\\
= \frac12\ln\abs{x+1} - \frac12\ln\abs{x-1}+C$}

\exercise{$\ds \int \frac{2x}{\sqrt{x^4-4}}\ dx$}{$\cosh^{-1} (x^2/2) + C = \ln (x^2+\sqrt{x^4-4})+C$}

\exercise{$\ds \int \frac{\sqrt{x}}{\sqrt{1+x^3}}\ dx$}{$2/3\sinh^{-1} x^{3/2} + C = 2/3\ln (x^{3/2}+\sqrt{x^3+1})+C$}

\exercise{$\ds \int \frac{e^x}{e^{2x}+1}\ dx$}{$\tan^{-1}(e^x)+C$}

\exercise{$\ds \int \sech x \ dx$ \quad(Hint: multiply by $\frac{\cosh x}{\cosh x}$; set $u = \sinh x$.)}{$\tan^{-1}(\sinh x)+C$}

}

%\exerciseset{In Exercises}{, evaluate the given definite integral.}{

\exercise{$\ds \int_{-1}^1 \sinh x \ dx$}{$0$}

\exercise{$\ds \int_{-\ln 2}^{\ln 2} \cosh x \ dx$}{$3/2$}

\exercise{$\ds \int_{0}^{1} \tanh^{-1} x \ dx$}{$2$}
}
%\printreview
%\exerciseset{In Exercises}{, use the Fundamental Theorem of Calculus Part 1 to find $F'(x)$.
}{

\exercise{$\ds F(x) = \int_2^{x^3+x} \frac{1}{t}\ dt$
}{$F'(x) = (3x^2+1)\frac{1}{x^3+x}$
}

\exercise{$\ds F(x) = \int_{x^3}^{0} t^3\ dt$
}{$F'(x) = 3x^{11}$
}

\exercise{$\ds F(x) = \int_{x}^{x^2} (t+2)\ dt$
}{$F'(x) = 2x(x^2+2)-(x+2)$
}

\exercise{$\ds F(x) = \int_{\ln x}^{e^x} \sin t\ dt$
}{$F'(x) = e^x\sin (e^x) - 1/x\sin(\ln x)$
}
}

\exercise{Following the guidelines in \autoref{sec:sketch}, and using L'Hô\-pi\-tal's rule where appropriate, neatly sketch the graph of $f(x)=\frac{\ln(x)}{\ln(2x)}$. Check your answer using a graphing utility (and be careful near 0).}{Use technology to verify sketch.}
