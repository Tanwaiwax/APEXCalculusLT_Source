\printconcepts

\exercise{The definite integral was defined with what two stipulations?}{The interval of integration is finite, and the integrand is continuous on that interval.}

\exercise{If $\ds \lim_{b\to \infty} \int_0^b f(x)\ dx$ exists, then the integral $\ds \int_0^\infty f(x)\ dx$ is said to \underline{\hskip 1in}.}{converge}

\exercise{If $\ds \int_1^\infty f(x)\ dx=10$, and $0\leq g(x)\leq f(x)$ for all $x$, then we know that $\ds \int_1^\infty g(x)\ dx$  \underline{\hskip 1in}.}{converges; could also state $<10$.}

\exercise{For what values of $p$ will $\ds \int_1^\infty \frac1{x^p}\ dx$ converge?}{$p>1$}

\exercise{For what values of $p$ will $\ds \int_{10}^\infty \frac1{x^p}\ dx$ converge?}{$p>1$}

\exercise{For what values of $p$ will $\ds \int_{0}^1 \frac1{x^p}\ dx$ converge?}{$p<1$}

\printproblems

\exerciseset{In Exercises}{, evaluate the given improper integral.}{

\exercise{$\ds \int_0^\infty e^{5-2x}\ dx$}{$e^5/2$}

\exercise{$\ds \int_1^\infty \frac{1}{x^3}\ dx$}{$1/2$}

\exercise{$\ds \int_1^\infty x^{-4}\ dx$}{$1/3$}

\exercise{$\ds \int_{-\infty}^\infty \frac{1}{x^2+9}\ dx$}{$\pi/3$}

\exercise{$\ds \int_{-\infty}^0 2^x\ dx$}{$1/\ln 2$}

\exercise{$\ds \int_{-\infty}^0 \left(\frac12\right)^x\ dx$}{diverges}

\exercise{$\ds \int_{-\infty}^\infty\frac{x}{x^2+1}\ dx$}{diverges}

\exercise{$\ds \int_{-\infty}^\infty\frac{x}{x^2+4}\ dx$}{$\pi/2$}

\exercise{$\ds \int_{2}^\infty\frac{1}{(x-1)^2}\ dx$}{$1$}

\exercise{$\ds \int_{1}^2\frac{1}{(x-1)^2}\ dx$}{diverges}

\exercise{$\ds \int_{2}^\infty\frac{1}{x-1}\ dx$}{diverges}

\exercise{$\ds \int_{1}^2\frac{1}{x-1}\ dx$}{diverges}

\exercise{$\ds \int_{-1}^1 \frac 1x \ dx$}{diverges}

\exercise{$\ds \int_{1}^3\frac{1}{x-2}\ dx$}{diverges}

\exercise{$\ds \int_{0}^\pi \sec^2 x\ dx$}{diverges}

\exercise{$\ds \int_{-2}^1 \frac{1}{\sqrt{|x|}} \ dx$}{$2+2\sqrt{2}$}

\exercise{$\ds \int_{0}^\infty xe^{-x} \ dx$}{$1$}

\exercise{$\ds \int_{0}^\infty xe^{-x^2} \ dx$}{$1/2$}

\exercise{$\ds \int_{-\infty}^\infty xe^{-x^2} \ dx$}{$0$}

\exercise{$\ds \int_{-\infty}^\infty \frac{1}{e^x+e^{-x}} \ dx$}{$\pi/2$}

\exercise{$\ds \int_{0}^1 x\ln x \ dx$}{$-1/4$}

\exercise{$\ds \int_{1}^\infty \frac{\ln x}{x} \ dx$}{diverges}

\exercise{$\ds \int_{0}^1 \ln x \ dx$}{$-1$}

\exercise{$\ds \int_{1}^\infty \frac{\ln x}{x^2} \ dx$}{$1$}

\exercise{$\ds \int_{1}^\infty \frac{\ln x}{\sqrt{x}} \ dx$}{diverges}

\exercise{$\ds \int_{0}^\infty e^{-x}\sin x \ dx$}{$1/2$}

\exercise{$\ds \int_{0}^\infty e^{-x}\cos x \ dx$}{$1/2$}
}
\ifthenelse{\boolean{printquestions}}{\columnbreak}{}
\exerciseset{In Exercises}{, use the Direct Comparison Test or the Limit Comparison Test to determine whether the given definite integral converges or diverges. Clearly state what test is being used and what function the integrand is being compared to.}{

\exercise{$\ds \int_{10}^\infty \frac{3}{\sqrt{3x^2+2x-5}} \ dx$}{diverges; Limit Comparison Test with $1/x$.}

\exercise{$\ds \int_{2}^\infty \frac{4}{\sqrt{7x^3-x}} \ dx$}{converges; Limit Comparison Test with $1/x^{3/2}$.}

\exercise{$\ds \int_{0}^\infty \frac{\sqrt{x+3}}{\sqrt{x^3-x^2+x+1}} \ dx$}{diverges; Limit Comparison Test with $1/x$.}

\exercise{$\ds \int_{1}^\infty e^{-x}\ln x \ dx$}{converges; Direct Comparison Test with $xe^{-x}$.}

\exercise{$\ds \int_{5}^\infty e^{-x^2+3x+1} \ dx$}{converges; Direct Comparison Test with $e^{-x}$.}

\exercise{$\ds \int_{0}^\infty \frac{\sqrt{x}}{e^x} \ dx$}{converges; Direct Comparison Test with $xe^{-x}$.}

\exercise{$\ds \int_{2}^\infty \frac{1}{x^2+\sin x} \ dx$}{converges; Direct Comparison Test with $1/(x^2-1)$.}

\exercise{$\ds \int_{0}^\infty \frac{x}{x^2+\cos x} \ dx$}{diverges; Direct Comparison Test with $x/(x^2+\cos x)$.}

\exercise{$\ds \int_{0}^\infty \frac{1}{x+e^x} \ dx$}{converges; Direct Comparison Test with $1/e^x$.}

\exercise{$\ds \int_{0}^\infty \frac{1}{e^x-x} \ dx$}{converges; Limit Comparison Test with $1/e^x$.}

}
%\begin{exerciseset}{In Exercises}{, find the equation of the line tangent to the function at the given $x$-value.}

\exercise{$f(x) = \sinh x$ at $x=0$}{$y=x$}

\exercise{$f(x) = \cosh x$ at $x=\ln 2$}{$y=3/4(x-\ln 2)+5/4$}

\exercise{$f(x) = \tanh x$ at $x=-\ln 3$}{$y=\frac9{25}(x+\ln 3)-\frac45$}

\exercise{$f(x) = \sech^2 x$ at $x=\ln 3$}{$y=-72/125(x-\ln 3)+9/25$}

\exercise{$f(x) = \sinh^{-1} x$ at $x=0$}{$y=x$}

\exercise{$f(x) = \cosh^{-1} x$ at $x=\sqrt 2$}{$y=(x-\sqrt{2})+\cosh^{-1}(\sqrt{2}) \approx (x-1.414)+0.881$}

\end{exerciseset}

%\exerciseset{In Exercises}{, evaluate the given indefinite integral.}{

\exercise{$\ds \int \tanh (2x)\ dx$}{$\frac12\ln (\cosh(2x))+C$}

\exercise{$\ds \int \cosh (3x-7)\ dx$}{$\frac13\sinh(3x-7)+C$}

\exercise{$\ds \int \sinh x\cosh x\ dx$}{$\frac12\sinh^2x+C$ or $1/2\cosh^2x+C$}

\exercise{$\ds \int \frac{1}{9-x^2}\ dx$}{$\begin{cases}\frac13\tanh^{-1}\left(\frac x3\right)+C & x^2<9 \\
\frac13\coth^{-1}\left(\frac x3\right)+C & 9<x^2 \end{cases}\\
= \frac12\ln\abs{x+1} - \frac12\ln\abs{x-1}+C$}

\exercise{$\ds \int \frac{2x}{\sqrt{x^4-4}}\ dx$}{$\cosh^{-1} (x^2/2) + C = \ln (x^2+\sqrt{x^4-4})+C$}

\exercise{$\ds \int \frac{\sqrt{x}}{\sqrt{1+x^3}}\ dx$}{$2/3\sinh^{-1} x^{3/2} + C = 2/3\ln (x^{3/2}+\sqrt{x^3+1})+C$}

\exercise{$\ds \int \frac{e^x}{e^{2x}+1}\ dx$}{$\tan^{-1}(e^x)+C$}

\exercise{$\ds \int \sech x \ dx$ \quad(Hint: multiply by $\frac{\cosh x}{\cosh x}$; set $u = \sinh x$.)}{$\tan^{-1}(\sinh x)+C$}

}

%\exerciseset{In Exercises}{, evaluate the given definite integral.}{

\exercise{$\ds \int_{-1}^1 \sinh x \ dx$}{$0$}

\exercise{$\ds \int_{-\ln 2}^{\ln 2} \cosh x \ dx$}{$3/2$}

\exercise{$\ds \int_{0}^{1} \tanh^{-1} x \ dx$}{$2$}
}
%\printreview
%\exerciseset{In Exercises}{, use the Fundamental Theorem of Calculus Part 1 to find $F'(x)$.
}{

\exercise{$\ds F(x) = \int_2^{x^3+x} \frac{1}{t}\ dt$
}{$F'(x) = (3x^2+1)\frac{1}{x^3+x}$
}

\exercise{$\ds F(x) = \int_{x^3}^{0} t^3\ dt$
}{$F'(x) = 3x^{11}$
}

\exercise{$\ds F(x) = \int_{x}^{x^2} (t+2)\ dt$
}{$F'(x) = 2x(x^2+2)-(x+2)$
}

\exercise{$\ds F(x) = \int_{\ln x}^{e^x} \sin t\ dt$
}{$F'(x) = e^x\sin (e^x) - 1/x\sin(\ln x)$
}
}