\printconcepts

\exercise{Trigonometric Substitution works on the same principles as Integration by Substitution, though it can feel ``\underline{\hskip .5in}''.}{backwards}

\exercise{If one uses Trigonometric Substitution on an integrand containing $\sqrt{25-x^2}$, then one should set $x=$\underline{\hskip .5in}.}{$5\sin\theta$}

\exercise{Consider the Pythagorean Identity $\sin^2\theta+\cos^2\theta = 1$.
		\begin{enumerate}
			\item What identity is obtained when both sides are divided by $\cos^2\theta$?
			\item	Use the new identity to simplify $9\tan^2\theta + 9.$
		\end{enumerate}
}{\begin{enumerate}
	\item $\tan^2\theta + 1 = \sec^2\theta$
	\item	$9\sec^2\theta$.
\end{enumerate}}

\exercise{Why does \autoref{idea:trigsub}(a) state that $\sqrt{a^2-x^2} = a\cos\theta$, and not $|a\cos\theta|$?}{Because we are considering $a>0$ and $x=a\sin \theta$, which means $\theta = \sin^{-1}(x/a)$. The arcsine function has a domain of $-\pi/2\leq \theta \leq \pi/2$; on this domain, $\cos \theta \geq 0$, so $a\cos\theta$ is always non-negative, allowing us to drop the absolute value signs.}

\printproblems

\exerciseset{In Exercises}{, apply Trigonometric Substitution to evaluate the indefinite integrals.}{

% why did we cut some of these?

\exercise{$\ds \int \sqrt{x^2+1}\ dx$\label{06_08_ex_05}}{$\frac12\left(x\sqrt{x^2+1}+\ln\abs{\sqrt{x^2+1}+x}\right) + C$}

%\exercise{$\ds \int \sqrt{x^2+4}\ dx$}{$2\left(\frac x4\sqrt{x^2+4}+\ln\abs{\frac{\sqrt{x^2+1}}2+\frac x2}\right) + C$}

%\exercise{$\ds \int \sqrt{1-x^2}\ dx$}{$\frac12\left(\sin^{-1}x+x\sqrt{1-x^2}\right)+C$}

%\exercise{$\ds \int \sqrt{9-x^2}\ dx$}{$\frac12\left(9\sin^{-1}(x/3)+x\sqrt{9-x^2}\right)+C$}

\exercise{$\ds \int \sqrt{x^2-1}\ dx$}{$\frac12x\sqrt{x^2-1}-\frac12\ln\abs{x+\sqrt{x^2-1}}+C$}

%\exercise{$\ds \int \sqrt{x^2-16}\ dx$}{$\frac12x\sqrt{x^2-16}-8\ln\abs{\frac x4+\frac{\sqrt{x^2-16}}4}+C$}

\exercise{$\ds \int \sqrt{4x^2+1}\ dx$}{$x\sqrt{x^2+1/4}+\frac14\ln\abs{2\sqrt{x^2+1/4}+2x} + C\\
= \frac12x\sqrt{4x^2+1}+\frac14\ln\abs{\sqrt{4x^2+1}+2x}+C$}

\exercise{$\ds \int \sqrt{1-9x^2}\ dx$}{$\frac16\sin^{-1}(3x)+\frac32\sqrt{1/9-x^2}+C=\frac16\sin^{-1}(3x)+\frac12\sqrt{1-9x^2}+C$}

\exercise{$\ds \int \sqrt{16x^2-1}\ dx$}{$4\left(\frac12x\sqrt{x^2-1/16}-\frac1{32}\ln\abs{4x+4\sqrt{x^2-1/16}}\right)+C = \frac12x\sqrt{16x^2-1}-\frac18\ln\abs{4x+\sqrt{16x^2-1}}+C$}

\exercise{$\ds \int \frac8{\sqrt{x^2+2}}\ dx$}{$8\ln\abs{\frac{\sqrt{x^2+2}}{\sqrt{2}}+\frac x{\sqrt{2}}}+C$; with \autoref{sec:hyperbolic}, we can state the answer as $8\sinh^{-1}(x/\sqrt{2})+C$.}

\exercise{$\ds \int \frac3{\sqrt{7-x^2}}\ dx$}{$3\sin^{-1}\left(\frac{x}{\sqrt{7}}\right)+C$ (Trig. Subst. is not needed)}

\exercise{$\ds \int \frac5{\sqrt{x^2-8}}\ dx$\label{06_08_ex_16}}{$5\ln\abs{\frac{x}{\sqrt{8}}+\frac{\sqrt{x^2-8}}{\sqrt{8}}}+C$}

}


\ifthenelse{\boolean{printquestions}}{\columnbreak}{}

\exerciseset{In Exercises}{,  evaluate the indefinite integrals. Some may be evaluated without Trigonometric Substitution.}{

\exercise{$\ds \int \frac {\sqrt{x^2-11}}x\ dx$}{$\sqrt{x^2-11}-\sqrt{11}\sec^{-1}(x/\sqrt{11})+C$}

%\exercise{$\ds \int \frac {1}{(x^2+1)^2}\ dx$}{$\frac12\left(\tan^{-1}x+\frac x{x^2+1}\right)+C$}

\exercise{$\ds \int \frac x{\sqrt{x^2-3}}\ dx$}{$\sqrt{x^2-3}+C$ (Trig. Subst. is not needed)}

%\exercise{$\ds \int x^2\sqrt{1-x^2}\ dx$}{$\frac18\sin^{-1}x-\frac18x\sqrt{1-x^2}(1-2x^2)+C$}

\exercise{$\ds \int \frac {x}{(x^2+9)^{3/2}}\ dx$}{$-\frac1{\sqrt{x^2+9}}+C$\quad (Trig. Subst. is not needed)}

\exercise{$\ds \int \frac {5x^2}{\sqrt{x^2-10}}\ dx$}{$\frac52x\sqrt{x^2-10}+25\ln\left|\frac{x}{\sqrt{10}}+\frac{\sqrt{x^2-10}}{\sqrt{10}}\right|+C$}

\exercise{$\ds \int \frac {1}{(x^2+4x+13)^2}\ dx$}{$\frac1{18}\frac{x+2}{x^2+4x+13}+\frac1{54}\tan^{-1}\left(\frac{x+2}2\right)+C$}

\exercise{$\ds \int x^2(1-x^2)^{-3/2}\ dx$}{$\frac x{\sqrt{1-x^2}} - \sin^{-1}x+C$}

\exercise{$\ds \int \frac{\sqrt{5-x^2}}{7x^2}\ dx$}{$\frac17\left(-\frac{\sqrt{5-x^2}}x-\sin^{-1}(x/\sqrt{5})\right)+C$}

\exercise{$\ds \int \frac{x^2}{\sqrt{x^2+3}}\ dx$}{$\frac12x\sqrt{x^2+3}-\frac32\ln\left|\frac{\sqrt{x^2+3}}{\sqrt{3}}+\frac x{\sqrt{3}}\right|+C$}

}


\exerciseset{In Exercises}{,  evaluate the definite integrals by making the proper trigonometric substitution \emph{and} changing the bounds of integration.% (Note: each of the corresponding indefinite integrals has appeared previously in this Exercise set.)
}{

\exercise{$\ds \int_{-1}^1 \sqrt{1-x^2}\ dx$}{$\pi/2$}

\exercise{$\ds \int_{4}^8 \sqrt{x^2-16}\ dx$}{$16\sqrt3-8\ln(2+\sqrt3)$}

\exercise{$\ds \int_{0}^2 \sqrt{x^2+4}\ dx$}{$2\sqrt2+2\ln(1+\sqrt2)$}

\exercise{$\ds \int_{-1}^1 \frac1{(x^2+1)^2}\ dx$}{$\pi/4+1/2)$}

\exercise{$\ds \int_{-1}^1 \sqrt{9-x^2}\ dx$}{$9\sin^{-1}(1/3)+\sqrt{8}$\quad Note: the new bounds of integration are $\sin^{-1}(-1/3)<\theta<\sin^{-1}(1/3)$. The final answer comes with recognizing that $\sin^{-1}(-1/3)=-\sin^{-1}(1/3)$ and that $\cos\big(\sin^{-1}(1/3)\big)=\cos\big(\sin^{-1}(-1/3)\big) = \sqrt{8}/3$.}

\exercise{$\ds \int_{-1}^1 x^2\sqrt{1-x^2}\ dx$}{$\pi/8$}

}

