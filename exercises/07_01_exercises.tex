\printconcepts

\exercise{T/F: The area between curves is always positive.}{T}

\exercise{T/F: Calculus can be used to find the area of basic geometric shapes.}{T}

\exercise{In your own words, describe how to find the total area enclosed by $y=f(x)$ and $y=g(x)$.}{Answers will vary.}

\printproblems

\exerciseset{In Exercises}{, find the area of the shaded region in the given graph.}{

\exercise{\begin{minipage}{\linewidth}\centering
\begin{tikzpicture}
\begin{axis}[width=\marginparwidth+25pt,tick label style={font=\scriptsize},
axis y line=middle,axis x line=middle,name=myplot,axis on top,
xtick=\empty,extra x ticks={3.14,6.28},extra x tick labels={$\pi$,$2\pi$},
ymin=-.2,ymax=6.5,xmin=-1,xmax=7]

\addplot [{\coloronefill},domain=0:6.28,stack plots=y,samples=40] {.5*cos(deg(x))+1};
\addplot [{\coloronefill},thick,fill={\coloronefill},area style,domain=0:6.28,stack plots=y,samples=40] {.5*x+3-(.5*cos(deg(x))+1)} \closedcycle;
\addplot [smooth,thick, {\colorone},domain=-.9:6.5,samples=40] {.5*cos(deg(x))+1} node [shift={(-25pt,7pt)} ,black] {\scriptsize $y=\frac12\cos x+1$};
\addplot [smooth,thick, {\colorone},domain=-.9:6.5,samples=40] {.5*x+3}node [shift={(-30pt,0pt)} ,black] {\scriptsize $y=\frac12x+3$};

\end{axis}

\node [right] at (myplot.right of origin) {\scriptsize $x$};
\node [above] at (myplot.above origin) {\scriptsize $y$};
\end{tikzpicture}
\end{minipage}}{$4\pi+\pi^2\approx 22.436$}

\exercise{\begin{minipage}{\linewidth}\centering
\begin{tikzpicture}
\begin{axis}[width=\marginparwidth+25pt,tick label style={font=\scriptsize},
axis y line=middle,axis x line=middle,name=myplot,axis on top,
xtick={-1,1},ytick={-1,1,2,3},ymin=-1.5,ymax=3.5,xmin=-1.1,xmax=1.1]

\addplot [{\coloronefill},domain=-1:1,stack plots=y,samples=40] {x^2+x-1};
\addplot [{\coloronefill},thick,fill={\coloronefill},area style,domain=-1:1,stack plots=y,samples=40] {-3*x^3+3*x+2-(x^2+x-1)} \closedcycle;

\addplot [smooth,thick, {\colorone},domain=-1.05:1.05,samples=40] {x^2+x-1} node [shift={(-23pt,-40pt)} ,black] {\scriptsize $y=x^2+x-1$};

\addplot [smooth,thick, {\colorone},domain=-1.05:1.05,samples=40] {-3*x^3+3*x+2}node [shift={(-95pt,13pt)} ,black] {\scriptsize $y=-3x^3+3x+2$};

\end{axis}

\node [right] at (myplot.right of origin) {\scriptsize $x$};
\node [above] at (myplot.above origin) {\scriptsize $y$};
\end{tikzpicture}
\end{minipage}
%\ifthenelse{\boolean{printquestions}}{\columnbreak}{}
}{$16/3$}

\exercise{\begin{minipage}{\linewidth}\centering
\begin{tikzpicture}
\begin{axis}[width=\marginparwidth+25pt,tick label style={font=\scriptsize},
axis y line=middle,axis x line=middle,name=myplot,axis on top,
xtick=\empty,extra x ticks={3.14,1.57},extra x tick labels={$\pi$,$\pi/2$},
ytick={-1,1,2,3},ymin=-.5,ymax=2.2,xmin=-.1,xmax=3.5]

\addplot [{\coloronefill},stack plots=y,samples=40,domain=0:3.14] {1};
\addplot [{\coloronefill},thick,fill={\coloronefill},area style,stack plots=y,samples=40,domain=0:3.14] {1} \closedcycle;

\addplot [smooth,thick, {\colorone},samples=40] {1} node [shift={(-70pt,-5pt)} ,black] {\scriptsize $y=1$};

\addplot [smooth,thick, {\colorone},samples=40] {2} node [shift={(-70pt,5pt)} ,black] {\scriptsize $y=2$};

\end{axis}

\node [right] at (myplot.right of origin) {\scriptsize $x$};
\node [above] at (myplot.above origin) {\scriptsize $y$};
\end{tikzpicture}
\end{minipage}
\ifthenelse{\boolean{printquestions}}{\columnbreak}{}
}{$\pi$}

\exercise{\begin{minipage}{\linewidth}\centering
\begin{tikzpicture}
\begin{axis}[width=\marginparwidth+25pt,tick label style={font=\scriptsize},
axis y line=middle,axis x line=middle,name=myplot,axis on top,
xtick=\empty,extra x ticks={3.14,1.57},extra x tick labels={$\pi$,$\pi/2$},
ytick={-1,1,2,3},ymin=-.5,ymax=2.2,xmin=-.1,xmax=3.5]

\addplot [{\coloronefill},stack plots=y,samples=40,domain=0:3.14] {sin(deg(x))};
\addplot [{\coloronefill},thick,fill={\coloronefill},area style,stack plots=y,samples=40,domain=0:3.14] {sin(deg(x))+1-(sin(deg(x)))} \closedcycle;

\addplot [smooth,thick, {\colorone},samples=40] {sin(deg(x))} node [shift={(-100pt,60pt)} ,black] {\scriptsize $y=\sin x$};

\addplot [smooth,thick, {\colorone},samples=40] {sin(deg(x))+1} node [shift={(-75pt,80pt)} ,black] {\scriptsize $y=\sin x+1$};

\end{axis}

\node [right] at (myplot.right of origin) {\scriptsize $x$};
\node [above] at (myplot.above origin) {\scriptsize $y$};
\end{tikzpicture}
\end{minipage}
%\ifthenelse{\boolean{printquestions}}{\columnbreak}{}
}{$\pi$}

\exercise{\begin{minipage}{\linewidth}\centering
\begin{tikzpicture}
\begin{axis}[width=\marginparwidth+25pt,tick label style={font=\scriptsize},
axis y line=middle,axis x line=middle,name=myplot,axis on top,
xtick=\empty,extra x ticks={.785,.392},extra x tick labels={$\pi/4$,$\pi/8$},
ytick={-1,1,2,3},ymin=-.2,ymax=2.2,xmin=-.1,xmax=1]

\addplot [{\coloronefill},stack plots=y,samples=40,domain=0:.785] {sin(deg(4*x))};
\addplot [{\coloronefill},thick,fill={\coloronefill},area style,stack plots=y,samples=40,domain=0:.785] {sec(deg(x))^2-(sin(deg(4*x)))} \closedcycle;

\addplot [smooth,thick, {\colorone},domain=-.1:.9,samples=60] {sin(deg(4*x))} node [shift={(-65pt,40pt)} ,black] {\scriptsize $y=\sin (4x)$};

\addplot [smooth,thick, {\colorone},domain=-.1:.9,samples=60] {sec(deg(x))^2} node [shift={(-35pt,-20pt)} ,black] {\scriptsize $y=\sec^2 x$};

\end{axis}

\node [right] at (myplot.right of origin) {\scriptsize $x$};
\node [above] at (myplot.above origin) {\scriptsize $y$};
\end{tikzpicture}
\end{minipage}}{$1/2$}

\exercise{\begin{minipage}{\linewidth}\centering
\begin{tikzpicture}
\begin{axis}[width=\marginparwidth+25pt,tick label style={font=\scriptsize},
axis y line=middle,axis x line=middle,name=myplot,axis on top,
xtick=\empty,extra x ticks={.785,1.57,2.36,3.14,3.92},
extra x tick labels={$\pi/4$,$\pi/2$,$3\pi/4$,$\pi$,$5\pi/4$},
ymin=-1.1,ymax=1.1,xmin=-.1,xmax=4.1]

\addplot [{\coloronefill},stack plots=y,samples=40,domain=.785:3.92] {cos(deg(x))};
\addplot [{\coloronefill},thick,fill={\coloronefill},area style,stack plots=y,samples=40,domain=.785:3.92] {sin(deg(x))-(cos(deg(x)))} \closedcycle;

\addplot [smooth,thick, {\colorone},domain=-.1:4.1,samples=40] {sin(deg(x))} node [shift={(-25pt,70pt)} ,black] {\scriptsize $y=\sin x$};

\addplot [smooth,thick, {\colorone},domain=-.1:4.1,samples=40] {cos(deg(x))} node [shift={(-75pt,0pt)} ,black] {\scriptsize $y=\cos x$};


\end{axis}

\node [right] at (myplot.right of origin) {\scriptsize $x$};
\node [above] at (myplot.above origin) {\scriptsize $y$};
\end{tikzpicture}
\end{minipage}}{$2\sqrt{2}$}

%\exercise{\begin{minipage}{\linewidth}\centering
%\begin{tikzpicture}
%\begin{axis}[width=\marginparwidth+25pt,tick label style={font=\scriptsize},
%axis y line=middle,axis x line=middle,name=myplot,axis on top,
%ymin=-.1,ymax=4.1,xmin=-.1,xmax=1.1]
%
%\addplot [{\coloronefill},stack plots=y,samples=40,domain=0:1] {2^x};
%\addplot [{\coloronefill},thick,fill={\coloronefill},area style,stack plots=y,samples=40,domain=0:1] {4^x-(2^x)} \closedcycle;
%
%\addplot [smooth,thick, {\colorone},domain=-.1:1.1,samples=40] {2^x} node [shift={(-25pt,-20pt)} ,black] {\scriptsize $y=2^x$};
%
%\addplot [smooth,thick, {\colorone},domain=-.1:1.1,samples=40] {4^x} node [shift={(-40pt,-20pt)} ,black] {\scriptsize $y=4^x$};
%
%\end{axis}
%
%\node [right] at (myplot.right of origin) {\scriptsize $x$};
%\node [above] at (myplot.above origin) {\scriptsize $y$};
%\end{tikzpicture}
%\end{minipage}}{$1/\ln 4$}

}


%\ifthenelse{\boolean{printquestions}}{\clearpage}{}

\begin{exerciseset}{In Exercises}{, find the area of the region bounded by the given curves.}

\exercise{$f(x) = 2x^2+5x-3$, $g(x) = x^2+4x-1$}{$4.5$}

\exercise{$f(x) = x^2-3x+2$, $g(x) = -3x+3$}{$4/3$}

\exercise{$f(x) = \sin x$, $g(x) = 2x/\pi$}{$2-\pi/2$}

\exercise{$f(x) = x^3-4x^2+x-1$, $g(x) = -x^2+2x-4$}{$8$}

\exercise{$f(x) = x$, $g(x) = \sqrt{x}$}{$1/6$}

\exercise{$f(x) = -x^3+5x^2+2x+1$, $g(x) = 3x^2+x+3$}{$37/12$}

\exercise{$x=2y^2$,\quad $x+y=1$}{$\frac{9}{8}$}

\exercise{$x=y^2-1$,\quad $x=1-y^2$}{$\frac{8}{3}$}

\exercise{$4x+y^2=12$,\quad $x=y$}{$\frac{64}{3}$}

\exercise{$x=y^2-4y$,\quad $x=2y-y^2$}{$9$}

\end{exerciseset}


\exercise{The functions $f(x) = \cos (2x)$ and $g(x) = \sin x$ intersect infinitely many times, forming an infinite number of repeated, enclosed regions. Find the areas of these regions.}{On regions such as $[\pi/6,5\pi/6]$, the area is $3\sqrt{3}/2$. On regions such as $[-\pi/2,\pi/6]$, the area is $3\sqrt{3}/4$. }

%\ifthenelse{\boolean{printquestions}}{\clearpage}{}

\exerciseset{In Exercises}{, find the area of the enclosed region in two ways:
		\begin{enumerate}
		\item		by treating the boundaries as functions of $x$, and
		\item		by treating the boundaries as functions of $y$.
		\end{enumerate}
}{

\exercise{\begin{minipage}{\linewidth}\centering\myincludegraphics{figures/fig07_01_ex_18}\end{minipage}}{$1$}

\exercise{\begin{minipage}{\linewidth}\centering\myincludegraphics{figures/fig07_01_ex_19}\end{minipage}}{$5/3$}

\exercise{\begin{minipage}{\linewidth}\centering\myincludegraphics{figures/fig07_01_ex_20}\end{minipage}}{$9/2$}

\exercise{\begin{minipage}{\linewidth}\centering\myincludegraphics{figures/fig07_01_ex_21}\end{minipage}}{$9/4$}

\exercise{\begin{minipage}{\linewidth}\centering\myincludegraphics{figures/fig07_01_ex_22}\end{minipage}}{$1/12(9-2\sqrt{2})\approx 0.514$}
}

\exerciseset{In Exercises}{, find the area triangle formed by the given three points.
}{

\exercise{$(1,1)$,\quad $(2,3)$,\quad and \quad $(3,3)$}{1}

\exercise{$(-1,1)$,\quad $(1,3)$,\quad and \quad $(2,-1)$}{5}

\exercise{$(1,1)$,\quad $(3,3)$,\quad and \quad $(3,3)$}{4}

\exercise{$(0,0)$,\quad $(2,5)$,\quad and \quad $(5,2)$}{133/20}
}

%\ifthenelse{\boolean{printquestions}}{\clearpage}{}

%\exercise{Use the Trapezoidal Rule to approximate the area of the pictured lake whose lengths, in hundreds of feet, are measured in 100-foot increments.\label{07_01_ex_29}
%
%\begin{center}
%\begin{tikzpicture}[yscale=.6]
%
%\draw [{\colorone},thick,fill=\coloronefill,smooth] plot coordinates {(0,1.)(0.1013,1.899)(0.3567,2.785)(0.6934,3.42)(1.039,3.569)(1.355,3.224)(1.646,2.705)(1.925,2.355)(2.2,2.473)(2.475,2.968)(2.75,3.529)(3.025,3.844)(3.3,3.709)(3.575,3.249)(3.85,2.651)(4.125,2.098)(4.389,1.661)(4.624,1.285)(4.807,0.9071)(4.921,0.4743)(4.958,0)(4.921,-0.4743)(4.807,-0.9071)(4.624,-1.279)(4.389,-1.625)(4.125,-1.986)(3.85,-2.401)(3.575,-2.843)(3.3,-3.233)(3.025,-3.487)(2.75,-3.536)(2.475,-3.397)(2.2,-3.111)(1.925,-2.718)(1.646,-2.263)(1.355,-1.792)(1.039,-1.352)(0.6934,-0.9844)(0.3567,-0.6744)(0.1013,-0.3653)(0,0)(0,1)};
%
%\draw (1,3.56) -- (1,-1.35) node [shift={(-3pt,0pt)},rotate=90,pos=.5] {\scriptsize 4.9};
%
%\draw (2,2.37) -- (2,-2.8) node [shift={(-3pt,0pt)},rotate=90,pos=.5] {\scriptsize 5.2};
%
%\draw (3,3.8) -- (3,-3.5) node [shift={(-3pt,0pt)},rotate=90,pos=.5] {\scriptsize 7.3};
%
%\draw (4,2.35) -- (4,-2.15) node [shift={(-3pt,0pt)},rotate=90,pos=.5] {\scriptsize 4.5};
%\end{tikzpicture}
%\end{center}}{219,000 ft$^2$}

%\exercise{Use Simpson's Rule to approximate the area of the pictured lake whose lengths, in hundreds of feet, are measured in 200-foot increments.\label{07_01_ex_30}
%
%\begin{center}
%\begin{tikzpicture}[yscale=.6]
%
%\draw [{\colorone},thick,fill=\coloronefill,smooth] plot coordinates {(0,2.)(0.1013,2.717)(0.3567,3.238)(0.6934,3.594)(1.039,3.818)(1.355,3.948)(1.646,4.035)(1.925,4.132)(2.2,4.28)(2.475,4.428)(2.75,4.471)(3.025,4.307)(3.305,3.88)(3.607,3.287)(3.952,2.652)(4.361,2.098)(4.815,1.661)(5.253,1.285)(5.614,0.9071)(5.843,0.4705)(5.938,-0.04167)(5.919,-0.6295)(5.807,-1.293)(5.624,-2.009)(5.389,-2.703)(5.125,-3.29)(4.849,-3.692)(4.562,-3.892)(4.243,-3.929)(3.871,-3.846)(3.432,-3.674)(2.956,-3.409)(2.484,-3.028)(2.057,-2.511)(1.692,-1.869)(1.363,-1.168)(1.039,-0.4819)(0.6934,0.1248)(0.3567,0.679)(0.1013,1.273)(0,2.)};
%
%\draw (1,3.8) -- (1,-.45) node [shift={(-3pt,0pt)},rotate=90,pos=.5] {\scriptsize 4.25};
%
%\draw (2,4.15) -- (2,-2.45) node [shift={(-3pt,0pt)},rotate=90,pos=.5] {\scriptsize 6.6};
%
%\draw (3,4.3) -- (3,-3.4) node [shift={(-3pt,0pt)},rotate=90,pos=.5] {\scriptsize 7.7};
%
%\draw (4,2.6) -- (4,-3.85) node [shift={(-3pt,0pt)},rotate=90,pos=.5] {\scriptsize 6.45};
%
%\draw (5,1.45) -- (5,-3.45) node [shift={(-3pt,0pt)},rotate=90,pos=.5] {\scriptsize 4.9};
%
%\end{tikzpicture}
%\end{center}}{623,333 ft$^2$}

%\exerciseset{In Exercises}{, use the Direct Comparison Test or the Limit Comparison Test to determine whether the given definite integral converges or diverges. Clearly state what test is being used and what function the integrand is being compared to.}{

\exercise{$\ds \int_{10}^\infty \frac{3}{\sqrt{3x^2+2x-5}} \ dx$}{diverges; Limit Comparison Test with $1/x$.}

\exercise{$\ds \int_{2}^\infty \frac{4}{\sqrt{7x^3-x}} \ dx$}{converges; Limit Comparison Test with $1/x^{3/2}$.}

\exercise{$\ds \int_{0}^\infty \frac{\sqrt{x+3}}{\sqrt{x^3-x^2+x+1}} \ dx$}{diverges; Limit Comparison Test with $1/x$.}

\exercise{$\ds \int_{1}^\infty e^{-x}\ln x \ dx$}{converges; Direct Comparison Test with $xe^{-x}$.}

\exercise{$\ds \int_{5}^\infty e^{-x^2+3x+1} \ dx$}{converges; Direct Comparison Test with $e^{-x}$.}

\exercise{$\ds \int_{0}^\infty \frac{\sqrt{x}}{e^x} \ dx$}{converges; Direct Comparison Test with $xe^{-x}$.}

\exercise{$\ds \int_{2}^\infty \frac{1}{x^2+\sin x} \ dx$}{converges; Direct Comparison Test with $1/(x^2-1)$.}

\exercise{$\ds \int_{0}^\infty \frac{x}{x^2+\cos x} \ dx$}{diverges; Direct Comparison Test with $x/(x^2+\cos x)$.}

\exercise{$\ds \int_{0}^\infty \frac{1}{x+e^x} \ dx$}{converges; Direct Comparison Test with $1/e^x$.}

\exercise{$\ds \int_{0}^\infty \frac{1}{e^x-x} \ dx$}{converges; Limit Comparison Test with $1/e^x$.}

}

%\begin{exerciseset}{In Exercises}{, find the equation of the line tangent to the function at the given $x$-value.}

\exercise{$f(x) = \sinh x$ at $x=0$}{$y=x$}

\exercise{$f(x) = \cosh x$ at $x=\ln 2$}{$y=3/4(x-\ln 2)+5/4$}

\exercise{$f(x) = \tanh x$ at $x=-\ln 3$}{$y=\frac9{25}(x+\ln 3)-\frac45$}

\exercise{$f(x) = \sech^2 x$ at $x=\ln 3$}{$y=-72/125(x-\ln 3)+9/25$}

\exercise{$f(x) = \sinh^{-1} x$ at $x=0$}{$y=x$}

\exercise{$f(x) = \cosh^{-1} x$ at $x=\sqrt 2$}{$y=(x-\sqrt{2})+\cosh^{-1}(\sqrt{2}) \approx (x-1.414)+0.881$}

\end{exerciseset}


%\exerciseset{In Exercises}{, evaluate the given indefinite integral.}{

\exercise{$\ds \int \tanh (2x)\ dx$}{$\frac12\ln (\cosh(2x))+C$}

\exercise{$\ds \int \cosh (3x-7)\ dx$}{$\frac13\sinh(3x-7)+C$}

\exercise{$\ds \int \sinh x\cosh x\ dx$}{$\frac12\sinh^2x+C$ or $1/2\cosh^2x+C$}

\exercise{$\ds \int \frac{1}{9-x^2}\ dx$}{$\begin{cases}\frac13\tanh^{-1}\left(\frac x3\right)+C & x^2<9 \\
\frac13\coth^{-1}\left(\frac x3\right)+C & 9<x^2 \end{cases}\\
= \frac12\ln\abs{x+1} - \frac12\ln\abs{x-1}+C$}

\exercise{$\ds \int \frac{2x}{\sqrt{x^4-4}}\ dx$}{$\cosh^{-1} (x^2/2) + C = \ln (x^2+\sqrt{x^4-4})+C$}

\exercise{$\ds \int \frac{\sqrt{x}}{\sqrt{1+x^3}}\ dx$}{$2/3\sinh^{-1} x^{3/2} + C = 2/3\ln (x^{3/2}+\sqrt{x^3+1})+C$}

\exercise{$\ds \int \frac{e^x}{e^{2x}+1}\ dx$}{$\tan^{-1}(e^x)+C$}

\exercise{$\ds \int \sech x \ dx$ \quad(Hint: multiply by $\frac{\cosh x}{\cosh x}$; set $u = \sinh x$.)}{$\tan^{-1}(\sinh x)+C$}

}


%\exerciseset{In Exercises}{, evaluate the given definite integral.}{

\exercise{$\ds \int_{-1}^1 \sinh x \ dx$}{$0$}

\exercise{$\ds \int_{-\ln 2}^{\ln 2} \cosh x \ dx$}{$3/2$}

\exercise{$\ds \int_{0}^{1} \tanh^{-1} x \ dx$}{$2$}
}

%\printreview

%\exerciseset{In Exercises}{, use the Fundamental Theorem of Calculus Part 1 to find $F'(x)$.
}{

\exercise{$\ds F(x) = \int_2^{x^3+x} \frac{1}{t}\ dt$
}{$F'(x) = (3x^2+1)\frac{1}{x^3+x}$
}

\exercise{$\ds F(x) = \int_{x^3}^{0} t^3\ dt$
}{$F'(x) = 3x^{11}$
}

\exercise{$\ds F(x) = \int_{x}^{x^2} (t+2)\ dt$
}{$F'(x) = 2x(x^2+2)-(x+2)$
}

\exercise{$\ds F(x) = \int_{\ln x}^{e^x} \sin t\ dt$
}{$F'(x) = e^x\sin (e^x) - 1/x\sin(\ln x)$
}
}
