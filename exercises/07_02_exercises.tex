\printconcepts

\exercise{T/F: A solid of revolution is formed by revolving a shape around an axis.}{T}

\exercise{In your own words, explain how the Disk and Washer Methods are related.}{Answers will vary.}

\exercise{Explain the how the units of volume are found in the integral of \autoref{thm:volume_by_cross_section}: if $A(x)$ has units of in$^2$, how does $\int A(x)\ dx$ have units of in$^3$?}{Recall that ``$dx$'' does not just ``sit there;'' it is multiplied by $A(x)$ and represents the thickness of a small slice of the solid. Therefore $dx$ has units of in, giving $A(x)\ dx$ the units of in$^3$.}

\exercise{A fundamental principle of this section is ``\underline{\hskip .5in} can be found by integrating an area function.''}{volume}

\printproblems

\exerciseset{In Exercises}{, a region of the Cartesian plane is shaded. Use the Disk/Washer Method to find the volume of the solid of revolution formed by revolving the region about the $x$-axis.}{

\exercise{\begin{minipage}{\linewidth}\centering\myincludegraphics{figures/fig07_02_ex_05}\end{minipage}}{$48\pi\sqrt{3}/5$ units$^3$}

\exercise{\begin{minipage}{\linewidth}\centering\myincludegraphics{figures/fig07_02_ex_06}\end{minipage}}{$175\pi/3$ units$^3$}

%\exercise{\begin{minipage}{\linewidth}\centering\myincludegraphics{figures/fig07_02_ex_07}\end{minipage}
%\ifthenelse{\boolean{printquestions}}{\columnbreak}{}
%}{$\pi^2/4$ units$^3$}

\exercise{\begin{minipage}{\linewidth}\centering\myincludegraphics{figures/fig07_02_ex_04}\end{minipage}}{$\pi/6$ units$^3$}

\exercise{\begin{minipage}{\linewidth}\centering
\begin{tikzpicture}
 \begin{axis}[width=\marginparwidth+25pt,tick label style={font=\scriptsize},
   axis y line=middle,axis x line=middle,name=myplot,axis on top,
   ymin=-.1,ymax=8.5,xmin=-.1,xmax=2.5]
  \addplot [ {\coloronefill},fill={\coloronefill}, samples=50,domain=0:2]
   {x^3}--(axis cs:0,8);
  \addplot[smooth,thick,{\colorone}, samples=50, domain=-.1:2.1] {8};
  \addplot [smooth,thick, {\colorone}, samples=50,domain=0:2.1] {x^3}
   node [pos=.5,below right, black] {\scriptsize $y=x^3$};
 \end{axis}
 \node [right] at (myplot.right of origin) {\scriptsize $x$};
 \node [above] at (myplot.above origin) {\scriptsize $y$};
\end{tikzpicture}\end{minipage}}{$\dfrac{768\pi}{7}$}

}


\exerciseset{In Exercises}{, a region of the Cartesian plane is shaded. Use the Disk/Washer Method to find the volume of the solid of revolution formed by revolving the region about the $y$-axis.}{

\exercise{\begin{minipage}{\linewidth}\centering\myincludegraphics{figures/fig07_02_ex_09}\end{minipage}}{$9\pi/2$ units$^3$}

\exercise{\begin{minipage}{\linewidth}\centering\myincludegraphics{figures/fig07_02_ex_10}\end{minipage}}{$35\pi/3$ units$^3$}

%\exercise{\begin{minipage}{\linewidth}\centering\myincludegraphics{figures/fig07_02_ex_11}\end{minipage}
%
%(Hint: Integration By Parts will be necessary, twice. First let $u = \arccos^2x$, then let $u=\arccos x$.)}{$\pi^2-2\pi$ units$^3$}

\exercise{\begin{minipage}{\linewidth}\centering\myincludegraphics{figures/fig07_02_ex_08}\end{minipage}}{$2\pi/15$ units$^3$}

\exercise{\begin{minipage}{\linewidth}\centering
\begin{tikzpicture}
 \begin{axis}[width=\marginparwidth+25pt,tick label style={font=\scriptsize},
   axis y line=middle,axis x line=middle,name=myplot,axis on top,
   ymin=-.1,ymax=8.5,xmin=-.1,xmax=2.5]
  \addplot [ {\coloronefill},fill={\coloronefill}, samples=50,domain=0:2]
   {x^3}--(axis cs:0,8);
  \addplot[smooth,thick,{\colorone}, samples=50, domain=-.1:2.1] {8};
  \addplot [smooth,thick, {\colorone}, samples=50,domain=0:2.1] {x^3}
   node [pos=.5,below right, black] {\scriptsize $y=x^3$};
 \end{axis}
 \node [right] at (myplot.right of origin) {\scriptsize $x$};
 \node [above] at (myplot.above origin) {\scriptsize $y$};
\end{tikzpicture}\end{minipage}}{$\dfrac{96\pi}{5}$}

}


\exerciseset{In Exercises}{, a region of the Cartesian plane is described. Use the Disk/Washer Method to find the volume of the solid of revolution formed by rotating the region about each of the given axes.
}{

\exercise{Region bounded by: $y=\sqrt{x}$, $y=0$ and $x=1$.\\
Rotate about:\\
\begin{minipage}[t]{.5\linewidth}
\begin{enumerate}
\item		the $x$-axis
\item		$y=1$
\end{enumerate}
\end{minipage}%
\begin{minipage}[t]{.5\linewidth}
\begin{enumerate}\addtocounter{enumii}{2}
\item		the $y$-axis
\item		$x=1$
\end{enumerate}
\end{minipage}}{\mbox{}\\[-2\baselineskip]\begin{enumerate}
\item $\pi/2$
\item $5\pi/6$
\item $4\pi/5$
\item $8\pi/15$
\end{enumerate}}

\exercise{Region bounded by: $y=4-x^2$ and $y=0$.\\
Rotate about:\\
\begin{minipage}[t]{.5\linewidth}
\begin{enumerate}
\item		the $x$-axis
\item		$y=4$
\end{enumerate}
\end{minipage}%
\begin{minipage}[t]{.5\linewidth}
\begin{enumerate}\addtocounter{enumii}{2}
\item		$y=-1$
\item		$x=2$
\end{enumerate}
\end{minipage}}{\mbox{}\\[-2\baselineskip]\begin{enumerate}
\item $512\pi/15$
\item $256\pi/5$
\item $832\pi/15$
\item $128\pi/3$
\end{enumerate}}

\exercise{The triangle with vertices $(1,1)$, $(1,2)$ and $(2,1)$.\\
Rotate about:\\
\begin{minipage}[t]{.5\linewidth}
\begin{enumerate}
\item		the $x$-axis
\item		$y=2$
\end{enumerate}
\end{minipage}%
\begin{minipage}[t]{.5\linewidth}
\begin{enumerate}\addtocounter{enumii}{2}
\item		the $y$-axis
\item		$x=1$
\end{enumerate}
\end{minipage}}{\mbox{}\\[-2\baselineskip]\begin{enumerate}
\item $4\pi/3$
\item $2\pi/3$
\item $4\pi/3$
\item $\pi/3$
\end{enumerate}
}

\exercise{Region bounded by $y=x^2-2x+2$ and $y=2x-1$.\\
Rotate about:\\
\begin{minipage}[t]{.5\linewidth}
\begin{enumerate}
\item		the $x$-axis
\item		$y=1$
\end{enumerate}
\end{minipage}%
\begin{minipage}[t]{.5\linewidth}
\begin{enumerate}\addtocounter{enumii}{2}
%\item		the $y$-axis
\item		$y=5$
\end{enumerate}
\end{minipage}}{\mbox{}\\[-2\baselineskip]\begin{enumerate}
\item $104\pi/15$
\item $64\pi/15$
\item $32\pi/5$
%\item $\pi/3$
\end{enumerate}}

\exercise{Region bounded by $y=2x$, $y=x$ and $x=2$.\\
Rotate about:\\
\begin{minipage}[t]{.5\linewidth}
\begin{enumerate}
\item		the $x$-axis
\item		$y=4$
\end{enumerate}
\end{minipage}%
\begin{minipage}[t]{.5\linewidth}
\begin{enumerate}\addtocounter{enumii}{2}
\item		the $y$-axis
\item		$x=2$
\end{enumerate}
\end{minipage}}{\mbox{}\\[-2\baselineskip]\begin{enumerate}
\item $8\pi$
\item $8\pi$
\item $16\pi/3$
\item $8\pi/3$
\end{enumerate}}

\exercise{Region bounded by $y=\cos x$, $x=0$, $x=\dfrac\pi4$ and the $x$-axis.\\
Rotate about:\\
\begin{minipage}[t]{.5\linewidth}
\begin{enumerate}
\item		the $x$-axis % \pi\int_0^{\pi/4}\cos^2 x dx
\item		$y=1$ % \pi\int_0^{\pi/4} (1-\cos x)^2 dx
\end{enumerate}
\end{minipage}%
\begin{minipage}[t]{.5\linewidth}
\begin{enumerate}\addtocounter{enumii}{2}
%\item		the $y$-axis
\item		$y=-1$ % \pi\int_0^{\pi/4}(\cos x+1)^2-1 dx
\end{enumerate}
\end{minipage}}{\mbox{}\\[-2\baselineskip]\begin{enumerate}
\item $\frac{\pi^2}8+\frac\pi4$
\item $\frac{3\pi^2}8+\frac\pi4-\pi\sqrt2$
\item $\frac{\pi^2}8+\frac\pi4+\pi\sqrt2$
\end{enumerate}}

}


%\ifthenelse{\boolean{printquestions}}{\columnbreak}{}

\input{exercises/07_02_exset_04}
