\exercisesetinstructions{, a convergent alternating series is given along with its sum and a value of $\epsilon$. Use \autoref{thm:alt_series_approx} to find $n$ such that the $n^\text{th}$ partial sum of the series is within $\epsilon$ of the sum of the series.}

\exercise{$\ds \sum_{n=1}^\infty\label{pi_fourth} \frac{(-1)^{n+1}}{n^4} = \frac{7\pi^4}{720}$, \quad $\epsilon = 0.001$}{$n=5$}

\exercise{$\ds \sum_{n=0}^\infty \frac{(-1)^{n}}{n!} = \frac1e$, \quad $\epsilon = 0.0001$}{$n=7$}

\exercise{$\ds \sum_{n=0}^\infty\label{pi_alt} \frac{(-1)^{n}}{2n+1}=\frac{\pi}4$, \quad $\epsilon = 0.001$}{Using the theorem, we find $n=499$ guarantees the sum is within $0.001$ of $\pi/4$. (Convergence is actually faster, as the sum is within $\epsilon$ of $\pi/24$ when  $n\geq 249$.)}

\exercise{$\ds \sum_{n=0}^\infty \frac{(-1)^{n}}{(2n)!}=\cos 1$, \quad $\epsilon = 10^{-8}$}{$n=5$ ($(2n)!>10^8$ when $n\geq 6$)}

\exercisesetend
