\begin{exerciseset}{Exercises}{ ask for an $n$ to be found such that $p_n(x)$ approximates $f(x)$ within a certain bound of accuracy.}

\exercise{Find $n$ such that the  Maclaurin polynomial of degree $n$ of $f(x)= e^x$ approximates $e$ within $0.0001$ of the actual value.}{The $n^\text{th}$ derivative of $f(x)=e^x$ is bounded by $e$ on $[0,1]$. Thus $\abs{R_n(1)}\leq \frac{e}{(n+1)!}1^{(n+1)}$. When $n=7$, this is less than $0.0001$. }

\exercise{Find $n$ such that the  Taylor polynomial of degree $n$ of $f(x)= \sqrt x$, centered at $x=4$, approximates $\sqrt 3$ within $0.0001$ of the actual value.}{The $n^\text{th}$ derivative of $f(x)=\sqrt x$ has a maximum on $[3,4]$ of $(2n-3)!!(-1)^{n+1}3^{1/2}6^{-n}$.  Thus $\abs{R_n(3)}\leq \frac{3^{1/2}2^{-2}}{3^n n(n+1)}$. When $n=5$, this is less than $0.0001$.}

\exercise{Find $n$ such that the  Maclaurin polynomial of degree $n$ of $f(x)= \cos x$ approximates $\cos \pi/3$ within $0.0001$ of the actual value.}{The $n^\text{th}$ derivative of $f(x)=\cos x$ is bounded by $1$ on intervals containing $0$ and $\pi/3$. Thus $\abs{R_n(\pi/3)}\leq \frac{1}{(n+1)!}(\pi/3)^{(n+1)}$. When $n=7$, this is less than $0.0001$. Since the Maclaurin polynomial of $\cos x$ only uses even powers, we can actually just use $n=6$.}

\exercise{Find $n$ such that the  Maclaurin polynomial of degree $n$ of $f(x)= \sin x$ approximates $\cos \pi$ within $0.0001$ of the actual value.}{The $n^\text{th}$ derivative of $f(x)=\sin x$ is bounded by $1$ on intervals containing $0$ and $\pi$. Thus $\abs{R_n(\pi)}\leq \frac{1}{(n+1)!}(\pi)^{(n+1)}$. When $n=12$, this is less than $0.0001$. Since the Maclaurin polynomial of $\sin x$ only uses odd powers, we can actually just use $n=11$.}

\end{exerciseset}
