\begin{exerciseset}{\autoref{idea:common_taylor} gives the $n^\text{th}$ term of the Taylor series of common functions. In Exercises}{, verify the formula given in the Key Idea by finding the first few terms of the Taylor series of the given function and identifying a pattern.}

\exercise{$f(x) = e^x$;\quad $c=0$}{All derivatives of $e^x$ are $e^x$ which evaluate to 1 at $x=0$.\\
The Taylor series starts $1+x+\frac12x^2+\frac{1}{3!}x^3+\frac{1}{4!}x^4+\dotsb$;\\
the Taylor series is $\ds \sum_{n=0}^\infty \frac{x^n}{n!}$}

\exercise{$f(x) = \sin x$;\quad $c=0$}{All derivatives of $\sin x$ are either $\pm\cos x$ or $\pm \sin x$, which evaluate to $\pm 1$ or $0$ at $x=0$. The Taylor series starts $0+x+0x^2-\frac16x^3+0x^4+\frac1{120}x^5$;\\
the Taylor series is $\ds \sum_{n=0}^\infty (-1)^n\frac{x^{2n+1}}{(2n+1)!}$}

\exercise{$f(x) = 1/(1-x)$;\quad $c=0$}{The $n^\text{th}$ derivative of $1/(1-x)$ is $f\,^{(n)}(x) = (n)!/(1-x)^{n+1}$, which evaluates to $n!$ at $x=0$.\\
The Taylor series starts $1+x+x^2+x^3+\dotsb$;\\
the Taylor series is $\ds \sum_{n=0}^\infty x^n$}

\exercise{$f(x) = \tan^{-1}x$;\quad $c=0$}{The derivative of $\tan^{-1}x$ is $1/(1+x^2)$. Taking successive derivatives using the Quotient Rule, the derivatives of $\tan^{-1}x$ fall into two categories in terms of their evaluation at $x=0$. \\
When $n$ is even, $\ds f\,^{(n)}(x) = (-1)^{(n-1)/2}\frac{p(x)}{(1+x^2)^n}$, where $p(x)$ is a polynomial such that $p(0) = 0$. Hence $f\,^{(n)}(0) = 0$ when $n$ is even.\\
When $n$ is odd, $\ds f\,^{(n)}(x) = (-1)^{(n-1)/2}\frac{p(x)}{(1+x^2)^n}$, where $p(x)$ is a polynomial such that $p(0) = (n-1)!$. Hence $f\,^{(n)}(0) = (-1)^{(n-1)/2}(n-1)!$ when $n$ is odd. (The unusual power of $(-1)$ is such that every other odd term is negative.)\\ 
The Taylor series starts $x-\frac13x^3+\frac15x^5+\dotsb$; by reindexing to only obtain odd powers of $x$, we get\\
the Taylor series is $\ds \sum_{n=0}^\infty (-1)^n\frac{x^{2n+1}}{2n+1}$.}

\exercisesetend
