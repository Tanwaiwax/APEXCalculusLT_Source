\exercisesetinstructions{, show that the Taylor series for $f(x)$, as given in \autoref{idea:common_taylor}, is equal to $f(x)$ by applying \autoref{thm:function_series_equality}; that is, show $\ds \lim_{n\to\infty}R_n(x) =0$.}


\exercise{$f(x) = e^x$}{Given a value $x$, the magnitude of the error term $R_n(x)$ is bounded by
\[\abs{R_n(x)}\leq\frac{\max\abs{f\,^{(n+1)}(z)}}{(n+1)!}\abs{x^{(n+1)}},\]
where $z$ is between $0$ and $x$. 

If $x>0$, then $z<x$ and $f\,^{(n+1)}(z)=e^z<e^x$. If $x<0$, then $x<z<0$ and $f\,^{(n+1)}(z)=e^z<1$. So given a fixed $x$ value, let $M = \max\{e^x,1\}$; $f\,^{(n)}(z)<M.$ This allows us to state
\[\abs{R_n(x)}\leq\frac{M}{(n+1)!}\abs{x^{(n+1)}}.\]
For any $x$, $\ds\lim_{n\to\infty} \frac{M}{(n+1)!}\abs{x^{(n+1)}}=0$. Thus by the Squeeze Theorem, we conclude that $\ds \lim_{n\to\infty} R_n(x) = 0$ for all $x$, and hence
\[e^x=\sum_{n=0}^\infty \frac{x^{n}}{n!}\quad \text{for all $x$}.\]}


\exercise{$f(x) = \sin x$}{The following argument is essentially the same as that given for $f(x) = \cos x$ in \autoref{ex_ts3}.

Given a value $x$, the magnitude of the error term $R_n(x)$ is bounded by
\[\abs{R_n(x)}\leq\frac{\max\abs{f\,^{(n+1)}(z)}}{(n+1)!}\abs{x^{(n+1)}}.\]
Since all derivatives of $\sin x$ are $\pm \cos x$ or $\pm\sin x$, whose magnitudes are bounded by $1$, we can state
\[\abs{R_n(x)}\leq\frac{1}{(n+1)!}\abs{x^{(n+1)}}.\]
For any $x$, $\ds\lim_{n\to\infty} \frac{x^{n+1}}{(n+1)!} = 0$. Thus by the Squeeze Theorem, we conclude that $\ds \lim_{n\to\infty} R_n(x) = 0$ for all $x$, and hence
\[\sin x=\sum_{n=0}^\infty(-1)^{n}\frac{x^{2n+1}}{(2n+1)!}\quad \text{for all $x$}.\]}

% (-1,0) requires Cauchy form of the remainder. see
% http://math.stackexchange.com/q/307077/147357
% x=1 requies Abel's theorem
\exercise{$f(x) = \ln(x+1)$ (show equality only on $(0,1)$).}{Given a value $x$, the magnitude of the error term $R_n(x)$ is bounded by
\[\abs{R_n(x)}\leq\frac{\max\abs{f\,^{(n+1)}(z)}}{(n+1)!}\abs{x^{(n+1)}},\]
where $z$ is between $0$ and $x$. Since $\abs{f\,^{(n+1)}(z)}=\frac{n!}{(z+1)^{n+1}}$, 
\[\abs{R_n(x)}\leq\frac1{n+1}\left(\frac{\abs x}{\min z+1}\right)^{n+1}.\]

%We consider the cases when $x>0$ and when $x<0$ separately.

If $0<x<1$, then $0<z<x$ and $f\,^{(n+1)}(z) =\frac{n!}{(z+1)^{n+1}}<n!$. Thus
\[\abs{R_n(x)}\leq\frac{n!}{(n+1)!}\abs{x^{(n+1)}}=\frac{x^{n+1}}{n+1}.\]
For a fixed $x<1$,
\[\lim_{n\to\infty} \frac{x^{n+1}}{n+1}=0.\]
%
%If $-1<x<0$, then $x<z<0$ and $f\,^{(n+1)}(z) =\frac{n!}{z^{n+1}}<\frac{n!}{x^{n+1}}$. Thus
%\[\abs{R_n(x)}\leq\frac{n!/x^{n+1}}{(n+1)!}\abs{x^{(n+1)}}=\frac{x^{n+1}}{n+1}(-x)^{n+1}.\]
%Since $-1<x<0$, $(-x)^{n+1}<1$. We can then extend the inequality from above to state
%\[\abs{R_n(x)}\leq\frac{x^{n+1}}{n+1}(-x)^{n+1}<\frac1{n+1}.\]
%
%As $n\to\infty$, $1/(n+1)\to0$. Thus by the Squeeze Theorem, we conclude that $\ds \lim_{n\to\infty} R_n(x) = 0$ for all $x$, and hence
%\[\ln(x+1)=\sum_{n=1}^\infty(-1)^{n+1}\frac{x^n}{n}\quad \text{for all $-1<x<1$}.\]
}

\exercise{$f(x) = 1/(1-x)$ (show equality only on $(-1,0)$)}{Given a value $x$, the magnitude of the error term $R_n(x)$ is bounded by
\[\abs{R_n(x)}\leq\frac{\max\abs{f\,^{(n+1)}(z)}}{(n+1)!}\abs{x^{(n+1)}},\]
where $z$ is between $0$ and $x$.

Note that $\abs{f\,^{(n+1)}(x)}=\frac{(n+1)!}{(1-x)^{n+2}}$. 

%We consider the cases when $x>0$ and when $x<0$ separately.

%If $0<x<1$, then $0<z<x$ and $f\,^{(n+1)}(z) =\frac{(n+1)!}{(1-z)^{n+2}}<\frac{(n+1)!}{(1-x)^{n+2}}$.
%Thus
%\[\abs{R_n(x)}\leq\frac{(n+1)!}{(1-x)^{n+2}}\frac1{(n+1)!}\abs{x^{n+1}}= \frac{(x-1)^{n+1}}{n+1}.\]
%For a fixed $x$,
%\[\lim_{n\to\infty} \frac{(x-1)^{n+1}}{n+1}=0,\]
%hence
%\[\frac1{1-x}=\sum_{n=0}^\infty x^n \text{ on } (-1,0).\]
 
If $-1<x<0$, then $x<z<0$ and $f\,^{(n+1)}(z) =\frac{(n+1)!}{(1-z)^{n+2}}<(n+1)!$. 
Thus
\[\abs{R_n(x)}\leq\frac{(n+1)!}{(n+1)!}\abs{x^{n+1}}=\abs{x^{n+1}}.\]
For a fixed $x$,
\[\lim_{n\to\infty}\abs{x^{n+1}}=0 \text{ as }\abs{x}<1.\]
}

\exercisesetend
