\printconcepts

\exercise{Use your own words to define a \emph{sequence.}}{Answers will vary.}

\exercise{The domain of a sequence is the \underline{\hskip .5in} numbers.}{natural}

\exercise{Use your own words to describe the \emph{range} of a sequence.}{Answers will vary.}

\exercise{Describe what it means for a sequence to be \emph{bounded}.}{Answers will vary.}

\printproblems

\input{exercises/08_01_exset_01}

\exerciseset{In Exercises}{, determine the $n^\text{th}$ term of the given sequence.}{

\exercise{4, 7, 10, 13, 16, $\dotsc$}{$a_n = 3n+1$}

\exercise{$\ds 3,\ -\frac32,\ \frac34,\ -\frac38,\ \dotsc$}{$a_n = (-1)^{n+1}\frac{3}{2^{n-1}}$}

\exercise{$10,\ 20,\ 40,\ 80,\ 160,\ \dotsc$}{$a_n = 10\cdot 2^{n-1}$}

\exercise{$\ds 1, 1,\ \frac12,\ \frac16,\ \frac1{24},\ \frac1{120},\ \dotsc$}{$a_n = 1/(n-1)!$}

}


\input{exercises/08_01_exset_06}

\exerciseset{In Exercises}{, determine whether the sequence converges or diverges. If convergent, give the limit of the sequence.}{

\exercise{$\ds\{a_n\} = \left\{(-1)^n\frac{n}{n+1}\right\}$}{diverges}

\exercise{$\ds\{a_n\} = \left\{\frac{4n^2-n+5}{3n^2+1}\right\}$}{converges to $4/3$}

\exercise{$\ds\{a_n\} = \left\{\frac{4^n}{5^n}\right\}$}{converges to $0$}

\exercise{$\ds\{a_n\} = \left\{\frac{(n-3)!}{(n+1)!}\right\}$}{converges to 0}

\exercise{$\ds\{a_n\} = \left\{\frac{n-1}{n}-\frac{n}{n-1}\right\}$, $n\geq 2$}{converges to $0$}

\exercise{$\ds\{a_n\} = \left\{\frac{6^{n+3}}{8^n}\right\}$}{converges to 0}

\exercise{$\ds\{a_n\} = \left\{\ln (n)\right\}$}{diverges}

\exercise{$\ds\{a_n\} = \left\{\frac{3n}{\sqrt{n^2+1}}\right\}$}{converges to 3}

% todo Tim 9.1#25 was mentioned 5 pages prior, just after Theorem 9.1.5
\exercise{$\ds\{a_n\} = \left\{\left(1+\frac1n\right)^n\right\}$}{converges to $e$}

\exercise{$\ds\{a_n\} = \left\{\frac{(2n+1)!}{(2n-1)!}\right\}$}{diverges}

\exercise{$\ds\{a_n\} = \left\{5-\frac1n\right\}$}{converges to 5}

\exercise{$\ds\{a_n\} = \left\{\frac{(-1)^{n+1}}{n}\right\}$}{converges to 0}

\exercise{$\ds\{a_n\} = \left\{\frac{1.1^n}{n}\right\}$}{diverges}

\exercise{$\ds\{a_n\} = \left\{\frac{2n}{n+1}\right\}$}{converges to 2}

\exercise{$\ds\{a_n\} = \left\{(-1)^n\frac{n^2}{2^n-1}\right\}$}{converges to 0}

\exercise{$\ds\{a_n\} = \left\{2+\frac{9^n}{8^n}\right\}$}{diverges}

\exercise{$\ds\{a_n\} = \left\{\frac{(n-1)!}{(n+1)!}\right\}$}{converges to 0}

\exercise{$\ds\{a_n\} = \{\ln(3n+2)-\ln n\}$}{converges to $\ln3$}

\exercise{$\ds\{a_n\} = \{\ln(2n^2+3n+1)-\ln(n^2+1)\}$}{converges to $\ln2$}

\exercise{$\ds\{a_n\} = \left\{n\sin\biggl(\frac{1}{n}\biggr)\right\}$}{converges to $1$}

% cut for parity
%\exercise{$\ds \{a_n\}=\left\{\frac{\cos^2 n}{2^n}\right\}$}{converges to $0$}

\exercise{$\ds \{a_n\}=\left\{\frac{e^n+e^{-n}}{e^{2n}-1}\right\}$}{converges to $0$}

\exercise{$\ds \{a_n\}=\left\{\frac{\ln n}{\ln 2n}\right\}$}{converges to $1$}

}


\input{exercises/08_01_exset_04}

\ifthenelse{\boolean{printquestions}}{\columnbreak}{}

\input{exercises/08_01_exset_05}

\exercise{Prove Theorem \ref{thm:abs_val_seq}; that is, 
		use the definition of the limit of a sequence to show that if $\ds\lim_{n\to\infty} \abs{a_n} = 0$, then $\ds \lim_{n\to\infty} a_n = 0$.
}{Let $\{a_n\}$ be given such that $\ds\lim_{n\to\infty} \abs{a_n} = 0$. By the definition of the limit of a sequence, given any $\epsilon >0$, there is a $m$ such that for all $n>m, \abs{\ \abs{a_n} - 0} <\epsilon$. Since $\abs{\ \abs{a_n}-0} = \abs{a_n - 0}$, this directly implies that for all $n>m$, $\abs{a_n - 0} < \epsilon$, meaning that $\ds\lim_{n\to\infty} a_n = 0$.
}

\exercise{Let $\{a_n\}$ and $\{b_n\}$ be sequences such that $\ds\lim_{n\to\infty} a_n = L$ and $\ds\lim_{n\to\infty} b_n = K$. 
		\begin{enumerate}
		\item		Show that if $a_n<b_n$ for all $n$, then $L\leq K$. 
		\item		Give an example where $L = K$.
		\end{enumerate}
}{\begin{enumerate}
\item		Left to reader
\item		$a_n = 1/3^n$ and $b_n = 1/2^n$
\end{enumerate}
}

\exercise{Prove the Squeeze Theorem for sequences: Let $\{a_n\}$ and $\{b_n\}$  be such that $\ds\lim_{n\to\infty} a_n = L$ and $\ds\lim_{n\to\infty} b_n = L$, and let $\{c_n\}$ be such that $a_n\leq c_n\leq b_n$ for all $n$. Then $\ds\lim_{n\to\infty} c_n = L$}{Left to reader}
