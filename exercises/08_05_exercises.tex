\printconcepts

\exercise{Why is $\ds\sum_{n=1}^\infty \sin n$ not an alternating series?}{The signs of the terms do not alternate; in the given series, some terms are negative and the others positive, but they do not necessarily alternate.}

\exercise{A series  $\ds\sum_{n=1}^\infty (-1)^na_n$ converges when $\{a_n\}$ is \underlinespace, \underlinespace\ and $\ds\lim_{n\to\infty}a_n = $\underlinespace.\vspace{-.5\baselineskip}}{positive, decreasing, 0}

\exercise{Give an example of a series where\vspace{-.5\baselineskip} $\ds \sum_{n=0}^\infty a_n$ converges but $\ds \sum_{n=0}^\infty\abs{a_n}$ does not.}{Many examples exist; one common example is $a_n = (-1)^n/n$.}

\exercise{The sum of a \underlinespace\ convergent series can be changed by rearranging the order of its terms.}{conditionally}

\printproblems

\exerciseset{In Exercises}{, an alternating series $\ds \sum_{n=i}^\infty a_n$ is given.
\begin{enumerate}
	\item [(a)] Determine if the series converges or diverges.
	\item [(b)] Determine if $\ds \sum_{n=0}^\infty\abs{a_n}$ converges or diverges.\vspace{-.5\baselineskip}
	\item [(c)] If $\ds \sum_{n=0}^\infty a_n$ converges, determine if the convergence is conditional or absolute.
\end{enumerate}}{

\exercise{$\ds \sum_{n=1}^\infty \frac{(-1)^{n+1}}{n^2}$}{\mbox{}\\[-2\baselineskip]\begin{enumerate}
	\item converges
	\item	converges ($p$-Series)
	\item absolute
\end{enumerate}}

%\exercise{$\ds \sum_{n=1}^\infty \frac{(-1)^{n+1}}{\sqrt{n!}}$}{\mbox{}\\[-2\baselineskip]\begin{enumerate}
%	\item converges
%	\item	converges (use Ratio Test in next section)
%	\item absolute
%\end{enumerate}}

\exercise{$\ds \sum_{n=0}^\infty (-e)^{-n}$}{\mbox{}\\[-2\baselineskip]\begin{enumerate}
	\item converges
	\item	converges (Geometric Series with $r=1/e$)
	\item absolute
\end{enumerate}}

\exercise{$\ds \sum_{n=0}^\infty (-1)^{n}\frac{n+5}{3n-5}$}{\mbox{}\\[-2\baselineskip]\begin{enumerate}
	\item diverges (limit of terms is not 0)
	\item	diverges
	\item n/a; diverges
\end{enumerate}}

\exercise{$\ds \sum_{n=1}^\infty (-1)^{n}\frac{2^n}{n^2}$}{\mbox{}\\[-2\baselineskip]\begin{enumerate}
	\item diverges (limit of terms is not 0)
	\item	diverges
	\item n/a; diverges
\end{enumerate}}

\exercise{$\ds \sum_{n=0}^\infty (-1)^{n+1}\frac{3n+5}{n^2-3n+1}$}{\mbox{}\\[-2\baselineskip]\begin{enumerate}
	\item converges
	\item	diverges (Limit Comparison Test with $1/n$)
	\item conditional
\end{enumerate}}

\exercise{$\ds \sum_{n=1}^\infty \frac{(-1)^{n}}{\ln n+1}$}{\mbox{}\\[-2\baselineskip]\begin{enumerate}
	\item converges
	\item	diverges (Direct Comparison Test with $1/n$)
	\item conditional
\end{enumerate}}

\exercise{$\ds \sum_{n=2}^\infty (-1)^n\frac{n}{\ln n }$}{\mbox{}\\[-2\baselineskip]\begin{enumerate}
	\item diverges (limit of terms is not 0)
	\item	diverges 
	\item n/a; diverges
\end{enumerate}}

\exercise{$\ds \sum_{n=1}^\infty \frac{(-1)^{n+1}}{1+3+5+\cdots+(2n-1) }$}{\mbox{}\\[-2\baselineskip]\begin{enumerate}
	\item converges
	\item	converges (the sum in the denominator is $n^2$)
	\item absolute
\end{enumerate}}

\exercise{$\ds \sum_{n=1}^\infty \cos \big(\pi n\big)$}{\mbox{}\\[-2\baselineskip]\begin{enumerate}
	\item diverges (terms oscillate between $\pm 1$)
	\item	diverges
	\item n/a; diverges
\end{enumerate}}

\exercise{$\ds \sum_{n=1}^\infty \frac{\sin\big((n+1/2)\pi\big)}{n\ln n}$}{\mbox{}\\[-2\baselineskip]\begin{enumerate}
	\item converges
	\item	diverges (Integral Test)
	\item conditional
\end{enumerate}}

\exercise{$\ds \sum_{n=0}^\infty \left(-\frac23\right)^n$}{\mbox{}\\[-2\baselineskip]\begin{enumerate}
	\item converges
	\item	converges (Geometric Series with $r=2/3$)
	\item absolute
\end{enumerate}}

%\exercise{$\ds \sum_{n=0}^\infty \frac{(-1)^nn^2}{n!}$}{\mbox{}\\[-2\baselineskip]\begin{enumerate}
%	\item converges
%	\item	converges (Ratio Test in next section)
%	\item absolute
%\end{enumerate}}

\exercise{$\ds \sum_{n=0}^\infty (-1)^n2^{-n^2}$}{\mbox{}\\[-2\baselineskip]\begin{enumerate}
	\item converges
	\item converges (Direct Comparison to $2^{-n}$)
	\item absolute
\end{enumerate}}

\exercise{$\ds \sum_{n=1}^\infty \frac{(-1)^n}{\sqrt{n}}$}{\mbox{}\\[-2\baselineskip]\begin{enumerate}
	\item converges
	\item	diverges ($p$-Series Test with $p=1/2$)
	\item conditional
\end{enumerate}}

%\exercise{$\ds \sum_{n=1}^\infty \frac{(-1000)^n}{n!}$}{\mbox{}\\[-2\baselineskip]\begin{enumerate}
%	\item converges
%	\item	converges (Ratio Test in next section)
%	\item absolute
%\end{enumerate}}

\exercise{$\ds \sum_{n=1}^\infty \frac{(-1)^n}{n(\ln n)^2}$}{\mbox{}\\[-2\baselineskip]\begin{enumerate}
	\item converges
	\item converges (Integral Test)
	\item absolute
\end{enumerate}}

}


\exerciseset{Let $S_n$ be the $n^\text{th}$ partial sum of a series. In Exercises}{, a convergent alternating series is given and a value of $n$. Compute $S_n$ and $S_{n+1}$ and use these values to find bounds on the sum of the series.   
}{

\exercise{$\ds \sum_{n=1}^\infty \frac{(-1)^n}{\ln (n+1)}$, \quad $n=5$
}{$S_5 = -1.1906$; $S_{6} = -0.6767$;

$\ds -1.1906 \leq \sum_{n=1}^\infty \frac{(-1)^n}{\ln (n+1)} \leq -0.6767$
}
\exercise{$\ds \sum_{n=1}^\infty \frac{(-1)^{n+1}}{n^4}$, \quad $n=4$
}{$S_4 = 0.9459$; $S_5 = 0.9475$;

$\ds 0.9459 \leq \sum_{n=1}^\infty \frac{(-1)^n}{n^4} \leq 0.9475$
}
\exercise{$\ds \sum_{n=0}^\infty \frac{(-1)^{n}}{n!}$, \quad $n=6$
}{$S_6 = 0.3681$; $S_7 = 0.3679$;

$\ds 0.3681 \leq \sum_{n=0}^\infty \frac{(-1)^{n}}{n!} \leq 0.3679$
}
\exercise{$\ds \sum_{n=0}^\infty \left(-\frac12\right)^n$, \quad $n=9$
}{$S_9 = 0.666016$; $S_{10} = 0.666992$;

$\ds 0.666016 \leq \sum_{n=0}^\infty \left(-\frac12\right)^n \leq 0.666992$
}}

\input{exercises/08_05_exset_03}

\exercise{The partial sums in problems \ref{pi_fourth} and \ref{pi_alt} can be used to approximate $\pi$. Using the values of $n$ from these problems, compute the respective partial sums and then use them to approximate $\pi$. Which gives a better estimate of $\pi$?}{Using 5 terms, the series in 23 gives $\pi\approx3.142013$. Using 499 terms, the series in 25 gives $\pi\approx3.143597$. The series in 23 gives the better approximation, and requires many fewer terms.}

\exercise{The book shows a rearrangement of the Alternating Harmonic Series,
\[
 \left(1-\frac12\right)-\frac14+
 \left(\frac13-\frac16\right)-\frac18+
 \left(\frac15-\frac1{10}\right)-\frac1{12}+\dots
\]
which gives the sum $\frac12 \ln 2$. Note that the terms in parentheses are positive, so if we simplified those terms we would have an alternating series. Without actually simplifying, show that the same scheme of rearranging terms so that each positive term is followed by two successive negative terms yields
\begin{multline*}
 \left(1-\frac12\right)-\frac14-\frac18+
 \left(\frac13-\frac16\right)-\frac1{12}-\frac1{16}+\\
 \left(\frac15-\frac1{10}\right)-\frac1{20}-\frac1{24}+\dots
\end{multline*}
and then show that this new rearrangement of the Alternating Harmonic Series has the sum $\frac14 \ln 2$.\\
Hint: move each right parenthesis one term to the right and then simplify inside the parentheses.}{}
