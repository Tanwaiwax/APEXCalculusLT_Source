\printconcepts

\exercise{What is the difference between a Taylor polynomial and a Maclaurin polynomial?}{The Maclaurin polynomial is a special case of Taylor polynomials. Taylor polynomials are centered at a specific $x$-value; when that $x$-value is 0, it is a Maclaurin polynomial.}

\exercise{T/F: In general, $p_n(x)$ approximates $f(x)$ better and better as $n$ gets larger.}{T}

\exercise{For some function $f(x)$, the Maclaurin polynomial of degree 4 is $p_4(x) = 6+3x-4x^2+5x^3-7x^4$. What is $p_2(x)$?}{$p_2(x) = 6+3x-4x^2$.}

\exercise{For some function $f(x)$, the Maclaurin polynomial of degree 4 is $p_4(x) = 6+3x-4x^2+5x^3-7x^4$. What is $\fpp'(0)$?}{$\fpp'(0)=30$}

\printproblems

\input{exercises/08_07_exset_01}

\exerciseset{In Exercises}{, find the Taylor polynomial of degree $n$, at $x=c$, for the given function.
}{

\exercise{$\ds f(x) = \sqrt x, \quad n=4, \quad c=1$
}{$p_4(x) = 1+\frac{1}{2} (-1+x)-\frac{1}{8} (-1+x)^2+\frac{1}{16}
   (-1+x)^3-\frac{5}{128} (-1+x)^4$
}
\exercise{$\ds f(x) = \ln (x+1), \quad n=4, \quad c=1$
}{$p_4(x) = \ln (2)+\frac{1}{2} (-1+x)-\frac{1}{8}
   (-1+x)^2+\frac{1}{24} (-1+x)^3-\frac{1}{64} (-1+x)^4$
}
\exercise{$\ds f(x) = \cos x, \quad n=6, \quad c=\pi/4$
}{$p_6(x) = \frac{1}{\sqrt{2}}-\frac{-\frac{\pi
   }{4}+x}{\sqrt{2}}-\frac{\left(-\frac{\pi
   }{4}+x\right)^2}{2 \sqrt{2}}+\frac{\left(-\frac{\pi
   }{4}+x\right)^3}{6 \sqrt{2}}+\frac{\left(-\frac{\pi
   }{4}+x\right)^4}{24 \sqrt{2}}-\frac{\left(-\frac{\pi
   }{4}+x\right)^5}{120 \sqrt{2}}-\frac{\left(-\frac{\pi
   }{4}+x\right)^6}{720 \sqrt{2}}$
}
\exercise{$\ds f(x) = \sin x, \quad n=5, \quad c=\pi/6$
}{$p_5(x) = \frac{1}{2}+\frac{1}{2} \sqrt{3} \left(-\frac{\pi
   }{6}+x\right)-\frac{1}{4} \left(-\frac{\pi
   }{6}+x\right)^2-\frac{\left(-\frac{\pi }{6}+x\right)^3}{4
   \sqrt{3}}+\frac{1}{48} \left(-\frac{\pi
   }{6}+x\right)^4+\frac{\left(-\frac{\pi
   }{6}+x\right)^5}{80 \sqrt{3}}$
}
\exercise{$\ds f(x) = \frac1x, \quad n=5, \quad c=2$
}{$p_5(x) = \frac{1}{2}-\frac{x-2}{4}+\frac{1}{8} (x-2)^2-\frac{1}{16}
   (x-2)^3+\frac{1}{32} (x-2)^4-\frac{1}{64} (x-2)^5$
}
\exercise{$\ds f(x) = \frac{1}{x^2}, \quad n=8, \quad c=1$
}{$p_8(x) = 1-2 (-1+x)+3 (-1+x)^2-4 (-1+x)^3+5 (-1+x)^4-6 (-1+x)^5+7
   (-1+x)^6-8 (-1+x)^7+9 (-1+x)^8$
}
\exercise{$\ds f(x) = \frac{1}{x^2+1}, \quad n=3, \quad c=-1$
}{$p_3(x) =\frac{1}{2}+\frac{1+x}{2}+\frac{1}{4} (1+x)^2$
}
\exercise{$\ds f(x) = x^2\cos x, \quad n=2, \quad c=\pi$
}{$p_2(x) =-\pi ^2-2 \pi  (x-\pi)+\frac{1}{2} \left(\pi ^2-2\right)
   (x-\pi)^2$
}
}

\exerciseset{In Exercises}{, approximate the function value with the indicated Taylor polynomial and give approximate bounds on the error.}{

\exercise{Approximate $\sin 0.1$ with the Maclaurin polynomial of degree 3.}{$p_3(x) =x-\frac{x^3}{6}$; $p_3(0.1) = 0.09983$. Error is bounded by $\frac{1}{4!}\cdot0.1^4 \approx 0.000004167$.}

\exercise{Approximate $\cos 1$ with the Maclaurin polynomial of degree 4.}{$p_4(x) =1-\frac{x^2}{2}+\frac{x^4}{24}$; $p_4(1) = 13/24\approx 0.54167$. Error is bounded by $\frac{1}{5!}\cdot1^5 \approx 0.00833$.}

\exercise{Approximate $\sqrt{10}$ with the Taylor polynomial of degree 2 centered at $x=9$.}{$p_2(x) =3+\frac{1}{6} (-9+x)-\frac{1}{216} (-9+x)^2$; $p_2(10) =  3.16204$. The third derivative of $f(x) =\sqrt x$ is bounded on $[9,10]$ by $0.0015$. Error is bounded by $\frac{0.0015}{3!}\cdot1^3 = 0.0003$.}

\exercise{Approximate $\ln1.5$ with the Taylor polynomial of degree 3 centered at $x=1$.}{$p_3(x) =-1+x-\frac{1}{2} (-1+x)^2+\frac{1}{3} (-1+x)^3$; $p_3(1.5) =  0.41667$. The absolute value of the fourth derivative of $f(x) =\ln x$ is bounded on $[1,1.5]$ by $6$. Error is bounded by $\frac{6}{4!}\cdot.5^4 = 0.016$.}

}


\exerciseset{Exercises}{ ask for an $n$ to be found such that $p_n(x)$ approximates $f(x)$ within a certain bound of accuracy.}{

\exercise{Find $n$ such that the  Maclaurin polynomial of degree $n$ of $f(x)= e^x$ approximates $e$ within $0.0001$ of the actual value.}{The $n^\text{th}$ derivative of $f(x)=e^x$ is bounded by $3$ on intervals containing $0$ and 1. Thus $\abs{R_n(1)}\leq \frac{3}{(n+1)!}1^{(n+1)}$. When $n=7$, this is less than $0.0001$. }

\exercise{Find $n$ such that the  Taylor polynomial of degree $n$ of $f(x)= \sqrt x$, centered at $x=4$, approximates $\sqrt 3$ within $0.0001$ of the actual value.}{The $n^\text{th}$ derivative of $f(x)=\sqrt x$ is bounded by $0.1$ on intervals containing $3$ and $4$. Thus $\abs{R_n(\pi)}\leq \frac{0.1}{(n+1)!}(1)^{(n+1)}$. When $n=4$, this is less than $0.0001$.}

\exercise{Find $n$ such that the  Maclaurin polynomial of degree $n$ of $f(x)= \cos x$ approximates $\cos \pi/3$ within $0.0001$ of the actual value.}{The $n^\text{th}$ derivative of $f(x)=\cos x$ is bounded by $1$ on intervals containing $0$ and $\pi/3$. Thus $\abs{R_n(\pi/3)}\leq \frac{1}{(n+1)!}(\pi/3)^{(n+1)}$. When $n=7$, this is less than $0.0001$. Since the Maclaurin polynomial of $\cos x$ only uses even powers, we can actually just use $n=6$.}

\exercise{Find $n$ such that the  Maclaurin polynomial of degree $n$ of $f(x)= \sin x$ approximates $\cos \pi$ within $0.0001$ of the actual value.}{The $n^\text{th}$ derivative of $f(x)=\sin x$ is bounded by $1$ on intervals containing $0$ and $\pi$. Thus $\abs{R_n(\pi)}\leq \frac{1}{(n+1)!}(\pi)^{(n+1)}$. When $n=12$, this is less than $0.0001$. Since the Maclaurin polynomial of $\sin x$ only uses odd powers, we can actually just use $n=11$.}

}


\exerciseset{In Exercises}{, find the $x^n$ term of the indicated Taylor polynomial.}{

\exercise{Find a formula for the $x^n$ term of the Maclaurin polynomial for $f(x)=e^x$.}{$\frac{1}{n!}x^n$}

\exercise{Find a formula for the $x^n$ term of the Maclaurin polynomial for $f(x)=\cos x$.}{When $n$ is even,  $\frac{(-1)^{n/2}}{n!}x^n$; when $n$ is odd, $0$.}

\exercise{Find a formula for the $x^n$ term of the Maclaurin polynomial for $\ds f(x)=\frac{1}{1-x}$.}{$x^n$}

\exercise{Find a formula for the $x^n$ term of the Maclaurin polynomial for $\ds f(x)=\frac{1}{1+x}$.}{$(-1)^nx^n$}

\exercise{Find a formula for the $x^n$ term of the Taylor polynomial for $\ds f(x)=\ln x$ centered at $c=1$.}{$(-1)^{n+1}\frac{(x-1)^n}{n}$}

}


\exerciseset{In Exercises}{, approximate the solution to the given differential equation with a degree 4 Maclaurin polynomial.}{

\exercise{$y'=y$, \qquad $y(0) = 1$}{$\ds 1+x+\frac12x^2+\frac16x^3+\frac1{24}x^4$}

% cut for parity
%\exercise{$y'=5y$,\qquad $y(0) = 3$}{$\ds 3+15x+\frac{75}{2}x^2+\frac{375}{6}x^3+\frac{1875}{24}x^4$}

\exercise{$\ds y'=\frac2y$,\qquad $y(0) = 1$}{$\ds 1+2x-2x^2+4x^3-10x^4$}

}

