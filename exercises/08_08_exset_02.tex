\exerciseset{In Exercises}{, find a formula for the $n^\text{th}$ term of the Taylor series of $f(x)$, centered at $c$, by finding the coefficients of the first few powers of $x$ and looking for a pattern. (The formulas for several of these are found in Key Idea \ref{idea:common_taylor}; show work verifying these formula.)
}{

\exercise{$f(x) = \cos x$;\quad $c=\pi/2$
}{The Taylor series starts $0-(x-\pi/2)+0x^2+\frac16(x-\pi/2)^3+0x^4-\frac1{120}(x-\pi/2)^5$; 

the Taylor series is $\ds \sum_{n=0}^\infty (-1)^{n+1}\frac{(x-\pi/2)^{2n+1}}{(2n+1)!}$
}
\exercise{$f(x) = 1/x$;\quad $c=1$
}{The Taylor series starts $1-(x-1)+(x-1)^2-(x-1)^3+(x-1)^4-(x-1)^5$; 

the Taylor series is $\ds \sum_{n=0}^\infty (-1)^{n}(x-1)^n$
}
\exercise{$f(x) = e^{-x}$;\quad $c=0$
}{$f\,^{(n)}(x) = (-1)^ne^{-x}$; at $x=0$, $f\,^{(n)}(0)=-1$ when $n$ is odd and $f\,^{(n)}(0)=1$ when $n$ is even.

The Taylor series starts $1-x+\frac12x^2-\frac1{3!}x^3+\cdots$; 

the Taylor series is $\ds \sum_{n=0}^\infty (-1)^n\frac{x^n}{n!}$.
}
\exercise{$f(x) = \ln(1+x)$;\quad $c=0$
}{$f\,^{(n)}(x) = (-1)^{n+1}\frac{(n-1)!}{(1+x)^n}$; at $x=0$, $f\,^{(n)}(0)=(-1)^{n+1}(n-1)!$

The Taylor series starts $x-\frac{x^2}2+\frac{x^3}3-\frac{x^4}4+\cdots$; 

the Taylor series is $\ds \sum_{n=1}^\infty (-1)^{n+1}\frac{x^n}{n}$.
}
\exercise{$f(x) = x/(x+1)$;\quad $c=1$
}{$f\,^{(n)}(x) = (-1)^{n+1}\frac{n!}{(x+1)^{n+1}}$; at $x=1$, $f\,^{(n)}(1)=(-1)^{n+1}\frac{n!}{2^{n+1}}$

The Taylor series starts $\frac12+\frac14(x-1)-\frac18(x-1)^2+\frac1{16}(x-1)^3\cdots$; 

the Taylor series is $\ds \sum_{n=0}^\infty (-1)^{n+1}\frac{(x-1)^n}{2^{n+1}}$.
}
\exercise{$f(x) = \sin x$;\quad $c=\pi/4$
}{The derivatives of $\sin x$ are $\pm \cos x$ and $\pm \sin x$; at $x=\pi/4$, these derivatives evaluate to $\pm \sqrt{2}/2$. 

The Taylor series starts $\frac{\sqrt{2}}2+\frac{\sqrt{2}}2(x-\pi/4) - \frac{\sqrt{2}}2\frac{(x-\pi/4)^2}{2}-\frac{\sqrt{2}}2\frac{(x-\pi/4)^3}{3!}+\frac{\sqrt{2}}2\frac{(x-\pi/4)^4}{4!}+\frac{\sqrt{2}}2\frac{(x-\pi/4)^5}{5!}\cdots$. Note how the signs are ``even, even, odd, odd, even, even, odd, odd,$\ldots$ We saw signs like these in Example \ref{ex_seq1} of Section \ref{sec:sequences}; one way of producing such signs is to raise $(-1)$ to a special quadratic power. While many possibilities exist, 
one such quadratic is $(n+3)(n+4)/2$. 

Thus the Taylor series is $\ds \sum_{n=0}^\infty (-1)^{\frac{(n+3)(n+4)}{2}}\frac{\sqrt2}{2}\frac{(x-\pi/4)^n}{n!}$.
}}