\printconcepts

\exercise{T/F: When sketching the graph of parametric equations, the $x$ and $y$ values are found separately, then plotted together.}{T}

\exercise{The direction in which a graph is ``moving'' is called the \underline{\hskip 15pt} of the graph.}{orientation}

\exercise{An equation written as $y=f(x)$ is written in \underline{\hskip 15pt} form.}{rectangular}

\exercise{Create parametric equations $x=f(t)$, $y=g(t)$ and sketch their graph. Explain any interesting features of your graph based on the functions $f$ and $g$.}{Answers will vary.}

\printproblems

\exercisesetinstructions{, sketch the graph of the given parametric equations \textbf{by hand}, making a table of points to plot. Be sure to indicate the orientation of the graph.}

\exercise{$x=t^2+t$,\quad $y=1-t^2$,\quad $-3\leq t\leq 3$}{\mbox{}\\[-\baselineskip]\begin{minipage}{\linewidth}
\begin{tikzpicture}[alt={ALT-TEXT-TO-BE-DETERMINED}]
\begin{axis}[width=1.16\marginparwidth,tick label style={font=\scriptsize},
axis y line=middle,axis x line=middle,name=myplot,axis equal,
minor x tick num=4,minor y tick num=4,ymin=-9,ymax=2,xmin=-1,xmax=13]
\addplot [thick, smooth,domain=-3:3,samples=30,\colorone] ({x^2+x},{1-x^2});
\draw [>=latex,->,thick] (axis cs: 6,-3) -- (axis cs:6.05,-3.04);
\end{axis}
\node [right] at (myplot.right of origin) {\scriptsize $x$};
\node [above] at (myplot.above origin) {\scriptsize $y$};
\end{tikzpicture}
\end{minipage}}

\exercise{$x=1$,\quad $y=5\sin t$,\quad $-\pi/2\leq t\leq \pi/2$}{\mbox{}\\[-\baselineskip]\begin{minipage}{\linewidth}
\begin{tikzpicture}[alt={ALT-TEXT-TO-BE-DETERMINED}]
\begin{axis}[width=1.16\marginparwidth,tick label style={font=\scriptsize},
axis y line=middle,axis x line=middle,name=myplot,axis equal,
minor y tick num=4,ymin=-5.5,ymax=5.5,xmin=-.5,xmax=1.6]
\addplot [thick, smooth,domain=-5:5,samples=30,\colorone] ({1},{x});
\draw [>=latex,->,thick] (axis cs: 1,1) -- (axis cs:1,1.1);
\end{axis}
\node [right] at (myplot.right of origin) {\scriptsize $x$};
\node [above] at (myplot.above origin) {\scriptsize $y$};
\end{tikzpicture}
\end{minipage}}

\exercise{$x=t^2$,\quad $y=2$,\quad $-2\leq t\leq 2$}{\mbox{}\\[-\baselineskip]\begin{minipage}{\linewidth}
\begin{tikzpicture}[alt={ALT-TEXT-TO-BE-DETERMINED}]
\begin{axis}[width=1.16\marginparwidth,tick label style={font=\scriptsize},
axis y line=middle,axis x line=middle,name=myplot,axis equal,
minor x tick num=4,minor y tick num=4,ymin=-.5,ymax=2.5,xmin=-.5,xmax=2.5]
\addplot [thick, smooth,domain=-2:2,samples=30,\colorone] ({x^2},{2});
\draw [>=latex,<->,thick] (axis cs: .9,2) -- (axis cs:1.5,2);
\end{axis}
\node [right] at (myplot.right of origin) {\scriptsize $x$};
\node [above] at (myplot.above origin) {\scriptsize $y$};
\end{tikzpicture}
\end{minipage}}

\exercise{$x=t^3-t+3$,\quad $y=t^2+1$,\quad $-2\leq t\leq 2$}{\mbox{}\\[-\baselineskip]\begin{minipage}{\linewidth}
\begin{tikzpicture}[alt={ALT-TEXT-TO-BE-DETERMINED}]
\begin{axis}[width=1.16\marginparwidth,tick label style={font=\scriptsize},
axis y line=middle,axis x line=middle,name=myplot,axis equal,
ymin=-.5,ymax=4.5,xmin=-.5,xmax=5.5]
\addplot [thick, smooth,domain=-2:2,samples=30,\colorone] ({x^3-x+3},{x^2+1});
\draw [>=latex,->,thick] (axis cs: 3.897,2.69) -- (axis cs:3.938,2.7161);
\end{axis}
\node [right] at (myplot.right of origin) {\scriptsize $x$};
\node [above] at (myplot.above origin) {\scriptsize $y$};
\end{tikzpicture}
\end{minipage}}

\exercisesetend


\exercisesetinstructions{, sketch the graph of the given parametric equations; using a graphing utility is advisable. Be sure to indicate the orientation of the graph.}

\exercise{$x=t^3-2t^2$,\quad $y=t^2$,\quad $-2\leq t \leq 3$}{\mbox{}\\[-\baselineskip]\begin{minipage}{\linewidth}
\begin{tikzpicture}[alt={ALT-TEXT-TO-BE-DETERMINED}]
\begin{axis}[width=1.167\marginparwidth,tick label style={font=\scriptsize},axis y line=middle,axis x line=middle,name=myplot,
ymin=0.1,ymax=9,xmin=-11,xmax=11]
\addplot [thick, smooth,domain=-2:1,samples=30,->,\colorone] ({x^3-2*x^2},{x^2});
\addplot [thick, smooth,domain=1:3,samples=30,\colorone] ({x^3-2*x^2},{x^2});
\end{axis}
\node [right] at (myplot.right of origin) {\scriptsize $x$};
\node [above] at (myplot.above origin) {\scriptsize $y$};
\end{tikzpicture}
\end{minipage}}

\exercise{$x=1/t$,\quad $y=\sin t$,\quad $0< t \leq 10$}{\mbox{}\\[-\baselineskip]\begin{minipage}{\linewidth}
\begin{tikzpicture}[alt={ALT-TEXT-TO-BE-DETERMINED}]
\begin{axis}[width=1.167\marginparwidth,tick label style={font=\scriptsize},axis y line=middle,axis x line=middle,name=myplot,
ymin=-1.1,ymax=1.1,xmin=-0.5,xmax=1.5]
\addplot [thick, smooth,domain=0.1:3,samples=60,->,\colorone] ({1/x},{sin(deg(x))});
\addplot [thick, smooth,domain=3:10,samples=60,\colorone] ({1/x},{sin(deg(x))});
\end{axis}
\node [right] at (myplot.right of origin) {\scriptsize $x$};
\node [above] at (myplot.above origin) {\scriptsize $y$};
\end{tikzpicture}
\end{minipage}}

\exercise{$x=3\cos t$,\quad $y=5\sin t$,\quad $0\leq t \leq 2\pi$}{\mbox{}\\[-\baselineskip]\begin{minipage}{\linewidth}
\begin{tikzpicture}[alt={ALT-TEXT-TO-BE-DETERMINED}]
\begin{axis}[width=1.167\marginparwidth,tick label style={font=\scriptsize},axis y line=middle,axis x line=middle,name=myplot,
ymin=-5.5,ymax=5.5,xmin=-5.5,xmax=5.5]
\addplot [thick, smooth,domain=0:60,samples=10,->,\colorone] ({3*cos(x)},{5*sin(x)});
\addplot [thick, smooth,domain=60:360,samples=60,\colorone] ({3*cos(x)},{5*sin(x)});
\end{axis}
\node [right] at (myplot.right of origin) {\scriptsize $x$};
\node [above] at (myplot.above origin) {\scriptsize $y$};
\end{tikzpicture}
\end{minipage}}

\exercise{$x=3\cos t+2$,\quad $y=5\sin t+3$,\quad $0\leq t \leq 2\pi$}{\mbox{}\\[-\baselineskip]\begin{minipage}{\linewidth}
\begin{tikzpicture}[alt={ALT-TEXT-TO-BE-DETERMINED}]
\begin{axis}[width=1.167\marginparwidth,tick label style={font=\scriptsize},axis y line=middle,axis x line=middle,name=myplot,
minor x tick num=4,minor y tick num=4,
ymin=-5.5,ymax=8.5,xmin=-5.5,xmax=7.5]
\addplot [thick, smooth,domain=0:60,samples=10,->,\colorone] ({3*cos(x)+2},{5*sin(x)+3});
\addplot [thick, smooth,domain=60:360,samples=60,\colorone] ({3*cos(x)+2},{5*sin(x)+3});
\end{axis}
\node [right] at (myplot.right of origin) {\scriptsize $x$};
\node [above] at (myplot.above origin) {\scriptsize $y$};
\end{tikzpicture}
\end{minipage}}

\exercise{$x=\cos t$,\quad $y=\cos(2t)$,\quad $0\leq t \leq \pi$}{\mbox{}\\[-\baselineskip]\begin{minipage}{\linewidth}
\begin{tikzpicture}[alt={ALT-TEXT-TO-BE-DETERMINED}]
\begin{axis}[width=1.167\marginparwidth,tick label style={font=\scriptsize},axis y line=middle,axis x line=middle,name=myplot,
ymin=-1.1,ymax=1.1,xmin=-1.1,xmax=1.1]
\addplot [thick, smooth,domain=0:30,samples=10,->,\colorone] ({cos(x)},{cos(x*2)});
\addplot [thick, smooth,domain=30:180,samples=60,\colorone] ({cos(x)},{cos(x*2)});
\end{axis}
\node [right] at (myplot.right of origin) {\scriptsize $x$};
\node [above] at (myplot.above origin) {\scriptsize $y$};
\end{tikzpicture}
\end{minipage}}

\exercise{$x=\cos t$,\quad $y=\sin(2t)$,\quad $0\leq t \leq 2\pi$}{\mbox{}\\[-\baselineskip]\begin{minipage}{\linewidth}
\begin{tikzpicture}[alt={ALT-TEXT-TO-BE-DETERMINED}]
\begin{axis}[width=1.167\marginparwidth,tick label style={font=\scriptsize},axis y line=middle,axis x line=middle,name=myplot,
%minor x tick num=4,minor y tick num=4,
ymin=-1.1,ymax=1.1,xmin=-1.1,xmax=1.1]
\addplot [thick, smooth,domain=0:60,samples=10,->,\colorone] ({cos(x)},{sin(x*2)});
\addplot [thick, smooth,domain=60:360,samples=60,\colorone] ({cos(x)},{sin(x*2)});
\end{axis}
\node [right] at (myplot.right of origin) {\scriptsize $x$};
\node [above] at (myplot.above origin) {\scriptsize $y$};
\end{tikzpicture}
\end{minipage}}

\exercise{$x=2\sec t$,\quad $y=3\tan t$,\quad $-\pi/2< t < \pi/2$}{\mbox{}\\[-\baselineskip]\begin{minipage}{\linewidth}
\begin{tikzpicture}[alt={ALT-TEXT-TO-BE-DETERMINED}]
\begin{axis}[width=1.167\marginparwidth,tick label style={font=\scriptsize},axis y line=middle,axis x line=middle,name=myplot,
minor x tick num=4,minor y tick num=4,
ymin=-16.,ymax=16,xmin=-1,xmax=11]
\addplot [thick, smooth,domain=-80:30,samples=60,->,\colorone] ({2*sec(x)},{3*tan(x)});
\addplot [thick, smooth,domain=30:80,samples=60,\colorone] ({2*sec(x)},{3*tan(x)});
\end{axis}
\node [right] at (myplot.right of origin) {\scriptsize $x$};
\node [above] at (myplot.above origin) {\scriptsize $y$};
\end{tikzpicture}
\end{minipage}}

\exercise{$x=\cosh t$,\quad $y=\sinh t$,\quad $-2\leq t \leq 2$}
{\mbox{}\\[-\baselineskip]\begin{minipage}{\linewidth}
\begin{tikzpicture}[alt={ALT-TEXT-TO-BE-DETERMINED}]
\begin{axis}[width=\marginparwidth+25pt,tick label style={font=\scriptsize},
axis y line=middle,axis x line=middle,name=myplot,
ymin=-4.5,ymax=4.5,xmin=-.1,xmax=4.5]
\addplot [{\colorone},thick, smooth,domain=-2:2,samples=60] ({cosh(x)},{sinh(x)});
\draw[{\colorone},>=latex,->,thick] (axis cs:2.35241, -2.12928)-- (axis cs:2.33123, -2.10586);
\end{axis}
\node [right] at (myplot.right of origin) {\scriptsize $x$};
\node [above] at (myplot.above origin) {\scriptsize $y$};
\end{tikzpicture}
\end{minipage}}

\exercise{$x=\cos t+\frac14\cos(8t)$,\quad $y=\sin t+\frac14\sin(8t)$,\quad $0\leq t \leq 2\pi$}{\mbox{}\\[-\baselineskip]\begin{minipage}{\linewidth}
\begin{tikzpicture}[alt={ALT-TEXT-TO-BE-DETERMINED}]
\begin{axis}[width=1.167\marginparwidth,tick label style={font=\scriptsize},axis y line=middle,axis x line=middle,name=myplot,
%minor x tick num=4,minor y tick num=4,
ymin=-1.5,ymax=1.5,xmin=-1.5,xmax=1.5]
\addplot [thick, smooth,domain=0:60,samples=20,->,\colorone] ({cos(x)+cos(8*x)/4},{sin(x)+sin(8*x)/4});
\addplot [thick, smooth,domain=60:360,samples=120,\colorone] ({cos(x)+cos(8*x)/4},{sin(x)+sin(8*x)/4});
\end{axis}
\node [right] at (myplot.right of origin) {\scriptsize $x$};
\node [above] at (myplot.above origin) {\scriptsize $y$};
\end{tikzpicture}
\end{minipage}}

\exercise{$x=\cos t+\frac14\sin(8t)$,\quad $y=\sin t+\frac14\cos(8t)$,\quad $0\leq t \leq 2\pi$}{\mbox{}\\[-\baselineskip]\begin{minipage}{\linewidth}
\begin{tikzpicture}[alt={ALT-TEXT-TO-BE-DETERMINED}]
\begin{axis}[width=1.167\marginparwidth,tick label style={font=\scriptsize},axis y line=middle,axis x line=middle,name=myplot,
%minor x tick num=4,minor y tick num=4,
ymin=-1.5,ymax=1.5,xmin=-1.5,xmax=1.5]
\addplot [thick, smooth,domain=0:60,samples=20,->,\colorone] ({cos(x)+sin(8*x)/4},{sin(x)+cos(8*x)/4});
\addplot [thick, smooth,domain=60:360,samples=120,\colorone] ({cos(x)+sin(8*x)/4},{sin(x)+cos(8*x)/4});
\end{axis}
\node [right] at (myplot.right of origin) {\scriptsize $x$};
\node [above] at (myplot.above origin) {\scriptsize $y$};
\end{tikzpicture}
\end{minipage}}

\exercisesetend


\exercisesetinstructions{, four sets of parametric equations are given. Describe how their graphs are similar and different. Be sure to discuss orientation and ranges.}

\exercise{\mbox{}\\[-2\baselineskip]\parbox[t]{\linewidth}{\begin{enumerate}
\item		$x=t$\quad $y=t^2$, \quad $-\infty< t< \infty$
\item		$x=\sin t$\quad $y=\sin^2t$, \quad $-\infty< t< \infty$
\item		$x=e^t$\quad $y=e^{2t}$, \quad $-\infty< t< \infty$
\item		$x=-t$\quad $y=t^2$, \quad $-\infty< t< \infty$
\end{enumerate}}}{\mbox{}\\[-2\baselineskip]\parbox[t]{\linewidth}{\begin{enumerate}
	\item Traces the parabola $y=x^2$, moves from left to right.
	\item	Traces the parabola $y=x^2$, but only from $-1\leq x\leq 1$; traces this portion back and forth infinitely.
	\item	Traces the parabola $y=x^2$, but only for $0<x$. Moves left to right.
	\item	Traces the parabola $y=x^2$, moves from right to left.
\end{enumerate}}}

\exercise{\mbox{}\\[-2\baselineskip]\parbox[t]{\linewidth}{\begin{enumerate}
\item		$x=\cos t$\quad $y=\sin t$, \quad $0\leq t\leq 2\pi$
\item		$x=\cos (t^2)$\quad $y=\sin(t^2)$, \quad $0\leq t\leq 2\pi$
\item		$x=\cos (1/t)$\quad $y=\sin(1/t)$, \quad $0<t<1$
\item		$x=\cos(\cos t)$\quad $y=\sin(\cos t)$, \quad $0\leq t\leq 2\pi$
\end{enumerate}}}{\mbox{}\\[-2\baselineskip]\parbox[t]{\linewidth}{\begin{enumerate}
	\item Traces a circle of radius $1$ counterclockwise once.
	\item	Traces a circle of radius $1$ counterclockwise over $6$ times.
	\item	Traces a circle of radius $1$ clockwise infinite times.
	\item	Traces an arc of a circle of radius $1$, from an angle of $-1$ radians to $1$ radian, twice.
\end{enumerate}}}

\exercisesetend


\exercisesetinstructions{, find a parameterization for the curve.}

\exercise{$y=9-4x$}{Possible Answer: $x=t$, $y=9-4t$}

\exercise{$4x-y^2=5$}{Possible Answer: $x=\frac{5+t^2}{4}$, $y=t$}

\exercise{$(x+9)^2 + (y-4)^2 =49$}{Possible Answer: $x=-9+7\cos t$, $y=4+7\sin t$}

\exercise{$(x-2)^2 - (y+3)^2 =25$}{Possible Answer: $x=2+5\sec t$, $y=-3+5\tan t$}

\exercisesetend

\input{exercises/09-02-exset-10}

\exercisesetinstructions{, find parametric equations for the given rectangular equation using the parameter $\ds t=\frac{\dd y}{\dd x}$. Verify that at $t=1$, the point on the graph has a tangent line with slope of 1.}

\exercise{$y=3x^2-11x+2$}{$x=(t+11)/6$, $y=(t^2-97)/12$. At $t=1$, $x=2$, $y=-8$.

$y'=6x-11$; when $x=2$, $y'=1$.}

\exercise{$y=e^x$}{$x=\ln t$, $y=t$. At $t=1$, $x=0$, $y=1$.

$y'=e^x$; when $x=0$, $y'=1$.}

\exercise{$y=\sin x$ on $[0,\pi]$}{$x=\cos^{-1} t$, $y=\sqrt{1-t^2}$. At $t=1$, $x=0$, $y=0$.

$y'=\cos x$; when $x=0$, $y'=1$.}

\exercise{$y=\sqrt{x}$ on $[0,\infty)$}{$x=1/(4t^2)$, $y=1/(2t)$. At $t=1$, $x=1/4$, $y=1/2$.

$y'=1/(2\sqrt{x})$; when $x=1/4$, $y'=1$.}

\exercisesetend


\exercise{Find parametric equations and a parameter interval for the motion of a particle that starts at $(1, 0)$ and traces the circle $x^2 + y^2 =1$
\begin{multicols}{2}
\begin{enumext}\raggedright
\item once clockwise
\item once counter-clockwise
\item twice clockwise
\item twice counter-clockwise
\end{enumext}
\end{multicols}}{Possible answers:
\begin{enumext}
\item $x=\sin t$, $y=\cos t$, $[\pi/2, 5\pi/2]$
\item $x=\cos t$, $y=\sin t$, $[0, 2\pi]$
\item $x=\sin t$, $y=\cos t$, $[\pi/2, 9\pi/2]$
\item $x=\cos t$, $y=\sin t$, $[0, 4\pi]$
\end{enumext}}

\exercise{Find parametric equations and a parameter interval for the motion of a particle that starts at $(a, 0)$ and traces the ellipse $\frac{x^2}{a^2} + \frac{y^2}{b^2} =1$
\begin{multicols}{2}
\begin{enumext}\raggedright
\item once clockwise
\item once counter-clockwise
\item twice clockwise
\item twice counter-clockwise
\end{enumext}
\end{multicols}}{Possible Answers:
\begin{enumext}
\item $x=a\sin t, y=b \cos t, [\pi/2, 5\pi/2]$
\item $x=a\cos t, y=b \sin t, [0, 2\pi]$
\item $x=a\sin t, y=b \cos t, [\pi/2, 9\pi/2]$
\item $x=a\cos t, y=b \sin t, [0, 4\pi]$
\end{enumext}}

\input{exercises/09-02-exset-09}

\input{exercises/09-02-exset-04}

\input{exercises/09-02-exset-05}

\input{exercises/09-02-exset-07}

\exercisesetinstructions{, find the value(s) of $t$ where the curve defined by the parametric equations is not smooth.}

\exercise{$x=t^3+t^2-t$,\quad $y=t^2+2t+3$}{$t=-1$}

\exercise{$x=t^2-4t$,\quad $y=t^3-2t^2-4t$}{$t=2$}

\exercise{$x=\cos t$,\quad $y=2\cos t$}{$t=k\pi$ for integer values of $k$}

\exercise{$x=2\cos t-\cos(2t)$,\quad $y=2\sin t-\sin(2t)$}{$t=\dotsc 0,\ 2\pi,\ 4\pi,\ \dotsc$}

\exercisesetend

