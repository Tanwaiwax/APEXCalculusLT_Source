\printconcepts

\exercise{Given polar equation $r=f(\theta)$, how can one create parametric equations of the same curve?}{Using $x=r\cos\theta$ and $y=r\sin\theta$, we can write $x=f(\theta)\cos\theta$, $y=f(\theta)\sin\theta$.}

\exercise{With rectangular coordinates, it is natural to approximate area with \underline{\hskip .5in}; with polar coordinates, it is natural to approximate area with \underline{\hskip .5in}.}{rectangles; sectors of circles}

\printproblems

\input{exercises/09-05-exset-01}

\begin{exerciseset}{In Exercises}{, find the values of $\theta$ in the given interval where the graph of the polar function has horizontal and vertical tangent lines.}

\exercise{$\ds r=3$;\quad $[0,2\pi]$}{horizontal: $\theta=\pi/2,3\pi/2$;

vertical: $\theta = 0,\pi,2\pi$}

\exercise{$\ds r=2\sin \theta$;\quad $[0,\pi]$}{horizontal: $\theta=0,\pi/2,\pi$;

vertical: $\theta = \pi/4,3\pi/4$}

\exercise{$\ds r=\cos(2\theta)$;\quad $[0,2\pi]$}{horizontal: $\theta=\tan^{-1}(1/\sqrt{5}),\ \pi/2,$\\
$\pi-\tan^{-1}(1/\sqrt{5}),\ \pi+\tan^{-1}(1/\sqrt{5}),\ 3\pi/2,\ 2\pi-\tan^{-1}(1/\sqrt{5})$;

vertical: $\theta = 0,\ \tan^{-1}(\sqrt{5}),\ \pi-\tan^{-1}(\sqrt{5}),\ \pi,\ \pi+\tan^{-1}(\sqrt{5}),\ 2\pi-\tan^{-1}(\sqrt{5})$}

\exercise{$\ds r=1+\cos\theta$;\quad $[0,2\pi]$}{horizontal: $\theta=\pi/3,\ 5\pi/3$;

vertical: $\theta = 0,\ 2\pi/3,\ 4\pi/3,\ 2\pi$

At $\theta=\pi$, $\frac{\dd y}{\dd x} = 0/0$; apply L'Hôpital's Rule to find that $\frac{\dd y}{\dd x}\rightarrow 0$ as $\theta\rightarrow \pi$.}

\end{exerciseset}


\begin{exerciseset}{In Exercises}{, find the equation of the lines tangent to the graph at the pole.}

\exercise{$\ds r=\sin\theta$;\quad $[0,\pi]$}{In polar: $\theta = 0 \ \cong \ \theta = \pi$

In rectangular: $y=0$}

\exercise{$r=\cos 3\theta$;\quad $[0,\pi]$}{In polar: $\theta=\frac\pi6$, $\theta=\frac\pi2$, and $\theta=-\frac\pi6$\\
In rectangular: $y=\sqrt 3 x$, $x=0$, and $y=-\sqrt 3 x$.}

\exercise{$r=\cos 2\theta$;\quad $[0,2\pi]$}{In polar: $\theta=\frac\pi4$ and $\theta=-\frac\pi4$\\
In rectangular: $y=x$ and $y=-x$.}

\exercise{$r=\sin 2\theta$;\quad $[0,2\pi]$}{In polar: $\theta=0\ \cong\ \theta=\pi$ and $\theta=\frac\pi2$\\
In rectangular: $y=0$ and $x=0$}

\end{exerciseset}


\exercisesetinstructions{, find the area of the described region.}

\exercise{Enclosed by the circle: $r=4\sin\theta$,\quad $\frac\pi3\leq\theta\leq\frac{2\pi}3$}{area = $\frac{4\pi}{3}+2\sqrt 3$}

\exercise{Enclosed by the circle $\ds r=5$}{area = $25\pi$}

\exercise{Enclosed by one petal of $\ds r=\sin(3\theta)$}{area = $\pi/12$}

\exercise{Enclosed by one petal of the rose curve $r=\cos (n\ \theta)$, where $n$ is a positive integer.}{area = $\pi/(4n)$}

\exercise{Enclosed by the cardioid $\ds r=1-\sin\theta$}{area = $3\pi/2$}

\exercise{Enclosed by the inner loop of the lima\c con $\ds r=1+2\cos\theta$}{area = $\pi-3\sqrt{3}/2$}

\exercise{Enclosed by the outer loop of the lima\c con $\ds r=1+2\cos\theta$ (including area enclosed by the inner loop)}{area = $2\pi+3\sqrt{3}/2$}

\exercise{Enclosed between the inner and  outer loop of the lima\c con $\ds r=1+2\cos\theta$}{area = $\pi+3\sqrt{3}$}

\exercise{Enclosed by $r=2\cos \theta$ and $r=2\sin\theta$, as shown:

\begin{tikzpicture}[alt={The shaded region is that inside a circle centered at (1,0) of radius 1 above the x axis, but outside a circle centered at (0,1) of radius 1.},>=stealth]
\begin{axis}[width=\marginparwidth,tick label style={font=\scriptsize},
axis y line=middle,axis x line=middle,name=myplot,axis on top,
ymin=-1.1,ymax=2.1,xmin=-1.4,xmax=2.4]
\addplot [thick,draw={\coloronefill},fill=\coloronefill, smooth,domain=0:45,samples=20] ({cos(x)*2*cos(x)},{sin(x)*2*cos(x)}) -- (axis cs:0,0);
\addplot [thick,draw={\coloronefill},fill=white, smooth,domain=0:50,samples=20] ({cos(x)*2*sin(x)},{sin(x)*2*sin(x)});
\addplot [thick,draw={\colorone}, smooth,domain=0:180,samples=60] ({cos(x)*2*cos(x)},{sin(x)*2*cos(x)});
\addplot [thick,draw={\colortwo}, smooth,domain=0:180,samples=40] ({cos(x)*2*sin(x)},{sin(x)*2*sin(x)});
\end{axis}
\node [right] at (myplot.right of origin) {\scriptsize $x$};
\node [above] at (myplot.above origin) {\scriptsize $y$};
\end{tikzpicture}
}{area = $1$}

\exercise{Enclosed by $r=\cos(3 \theta)$ and $r=\sin(3\theta)$, as shown:

\begin{tikzpicture}[alt={The shaded region is inside z=sin(3θ) for 0≤θ≤π/3 but outside z=cos(3θ) for -π/6≤θ≤π/6.},>=stealth]
\begin{axis}[width=\marginparwidth,tick label style={font=\scriptsize},
axis y line=middle,axis x line=middle,name=myplot,axis on top,
xtick={1},ytick={.5},ymin=-.25,ymax=.7,xmin=-.08,xmax=1.1]
\addplot [thick,draw={\coloronefill},fill=\coloronefill, smooth,domain=15:60,samples=20] ({cos(x)*sin(3*x)},{sin(x)*sin(3*x)}) -- (axis cs:0,0);
% draw over the previous one
\addplot [thick,draw=white,fill=white, smooth,domain=0:30,samples=20] ({cos(x)*cos(3*x)},{sin(x)*cos(3*x)});
\addplot [thick,draw={\colorone}, smooth,domain=0:60,samples=60] ({cos(x)*sin(3*x)},{sin(x)*sin(3*x)});
\addplot [thick,draw={\colortwo}, smooth,domain=-30:30,samples=60] ({cos(x)*cos(3*x)},{sin(x)*cos(3*x)});
\end{axis}
\node [right] at (myplot.right of origin) {\scriptsize $x$};
\node [above] at (myplot.above origin) {\scriptsize $y$};
\end{tikzpicture}
}{area = $\ds \int_{\pi/12}^{\pi/3} \frac12 \sin^2(3\theta)\dd\theta - \int_{\pi/12}^{\pi/6}\frac12\cos^2(3\theta)\dd\theta = \frac1{12}+\frac{\pi}{24}$}

\exercise{Enclosed by $r=\cos \theta$ and $r=\sin(2\theta)$, as shown:

\begin{tikzpicture}[alt={The shaded region is that inside z=sin(2θ) for 0≤θ≤π/2 and also inside a circle of radius 1 centered at (1,0).},>=stealth]
\begin{axis}[width=\marginparwidth,tick label style={font=\scriptsize},
axis y line=middle,axis x line=middle,name=myplot,axis on top,
xtick={1},ytick={1},ymin=-.1,ymax=1.1,xmin=-.1,xmax=1.34]
\addplot [thick,draw={\coloronefill},fill=\coloronefill, smooth,domain=90:30,samples=20] ({cos(x)*cos(x)},{sin(x)*cos(x)}) -- (axis cs:0,0);
\addplot [thick,draw={\coloronefill},fill=\coloronefill, smooth,domain=0:30,samples=20] ({cos(x)*sin(2*x)},{sin(x)*sin(2*x)}) -- (axis cs:0,0);
\addplot [thick,draw={\colorone}, smooth,domain=0:90,samples=60] ({cos(x)*cos(x)},{sin(x)*cos(x)});
\addplot [thick,draw={\colortwo}, smooth,domain=0:90,samples=40] ({cos(x)*sin(2*x)},{sin(x)*sin(2*x)});
\end{axis}
\node [right] at (myplot.right of origin) {\scriptsize $x$};
\node [above] at (myplot.above origin) {\scriptsize $y$};
\end{tikzpicture}
}{area = $\frac{1}{32}(4\pi-3\sqrt{3})$}

\exercise{Enclosed by $r=\cos\theta$ and $r=1-\cos\theta$, as shown:

\begin{tikzpicture}[alt={r=1-cosθ is a cardioid mostly lying in x<0.  It intersects a circle centered at (1,0) with radius 1 in two narrow regions lying along y=±x.  The shaded region is the top one.},>=stealth]
\begin{axis}[width=\marginparwidth,tick label style={font=\scriptsize},
axis y line=middle,axis x line=middle,name=myplot,axis on top,
ymin=-1.4,ymax=1.4,xmin=-2.2,xmax=1.2]
\addplot [thick,draw={\coloronefill},fill=\coloronefill, smooth,domain=60:90,samples=60] ({cos(x)*cos(x)},{sin(x)*cos(x)});
\addplot [thick,draw={\coloronefill},fill=\coloronefill, smooth,domain=0:60,samples=40] ({cos(x)*(1-cos(x))},{sin(x)*(1-cos(x))});
\addplot [thick,draw={\colorone}, smooth,domain=0:180,samples=60] ({cos(x)*cos(x)},{sin(x)*cos(x)});
\addplot [thick,draw={\colortwo}, smooth,domain=0:360,samples=60] ({cos(x)*(1-cos(x))},{sin(x)*(1-cos(x))});
\end{axis}
\node [right] at (myplot.right of origin) {\scriptsize $x$};
\node [above] at (myplot.above origin) {\scriptsize $y$};
\end{tikzpicture}
}{area = $\ds \int_{0}^{\pi/3} \frac12 (1-\cos\theta)^2\dd\theta +\int_{\pi/3}^{\pi/2} \frac12 (\cos\theta)^2\dd\theta =\frac{7\pi}{24}-\frac{\sqrt{3}}2\approx 0.0503$}

\exercisesetend


\input{exercises/09-05-exset-05}

\exercisesetinstructions{, answer the questions involving surface area.}

\exercise{Use \autoref{idea:surface_area_polar} to find the surface area of the sphere formed by revolving the circle $r=2$ about the initial ray.}{$SA = 16\pi$}

\exercise{Use \autoref{idea:surface_area_polar} to find the surface area of the sphere formed by revolving the circle $r=2\cos\theta$ about the initial ray.}{$SA = 4\pi$}

\exercise{Find the surface area of the solid formed by revolving the cardioid $r=1+\cos\theta$ about the initial ray.}{$SA = 32\pi/5$}

\exercise{Find the surface area of the solid formed by revolving the circle $r=2\cos\theta$ about the line $\theta=\pi/2$.}{$SA = 4\pi^2$}

\exercise{Find the surface area of the solid formed by revolving the line $r=3\sec\theta$, $-\pi/4\leq\theta\leq\pi/4$, about the line $\theta=\pi/2$.}{$SA = 36\pi$}

\exercise{Find the surface area of the solid formed by revolving the line $r=3\sec\theta$, $0\leq\theta\leq\pi/4$, about the initial ray.}{$SA = 9\pi$}

\exercisesetend

