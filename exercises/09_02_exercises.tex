\printconcepts

\exercise{T/F: When sketching the graph of parametric equations, the $x$ and $y$ values are found separately, then plotted together.}{T}

\exercise{The direction in which a graph is ``moving'' is called the \underline{\hskip 15pt} of the graph.}{orientation}

\exercise{An equation written as $y=f(x)$ is written in \underline{\hskip 15pt} form.}{rectangular}

\exercise{Create parametric equations $x=f(t)$, $y=g(t)$ and sketch their graph. Explain any interesting features of your graph based on the functions $f$ and $g$.}{Answers will vary.}

\printproblems

\exerciseset{In Exercises}{, sketch the graph of the given parametric equations \textbf{by hand}, making a table of points to plot. Be sure to indicate the orientation of the graph.
}{

\exercise{$x=t^2+t$,\quad $y=1-t^2$,\quad $-3\leq t\leq 3$
}{\begin{minipage}{\linewidth}
\myincludegraphics{figures/fig09_02_ex_05}
\end{minipage}
}

\exercise{$x=1$,\quad $y=5\sin t$,\quad $-\pi/2\leq t\leq \pi/2$
}{\begin{minipage}{\linewidth}
\myincludegraphics{figures/fig09_02_ex_06}
\end{minipage}
}

\exercise{$x=t^2$,\quad $y=2$,\quad $-2\leq t\leq 2$
}{\begin{minipage}{\linewidth}
\myincludegraphics{figures/fig09_02_ex_07}
\end{minipage}
}

\exercise{$x=t^3-t+3$,\quad $y=t^2+1$,\quad $-2\leq t\leq 2$
}{\begin{minipage}{\linewidth}
\myincludegraphics{figures/fig09_02_ex_08}
\end{minipage}
}
}

\exerciseset{In Exercises}{, sketch the graph of the given parametric equations; using a graphing utility is advisable. Be sure to indicate the orientation of the graph.
}{

\exercise{$x=t^3-2t^2$,\quad $y=t^2$,\quad $-2\leq t \leq 3$}{\begin{minipage}{\linewidth}
\myincludegraphics{figures/fig09_02_ex_09}\end{minipage}
}
\exercise{$x=1/t$,\quad $y=\sin t$,\quad $0< t \leq 10$}{\begin{minipage}{\linewidth}
\myincludegraphics{figures/fig09_02_ex_10}
\end{minipage}
}
\exercise{$x=3\cos t$,\quad $y=5\sin t$,\quad $0\leq t \leq 2\pi$}{\begin{minipage}{\linewidth}
\myincludegraphics{figures/fig09_02_ex_11}
\end{minipage}
}
\exercise{$x=3\cos t+2$,\quad $y=5\sin t+3$,\quad $0\leq t \leq 2\pi$}{\begin{minipage}{\linewidth}
\myincludegraphics{figures/fig09_02_ex_12}
\end{minipage}
}
\exercise{$x=\cos t$,\quad $y=\cos(2t)$,\quad $0\leq t \leq \pi$}{\begin{minipage}{\linewidth}
\myincludegraphics{figures/fig09_02_ex_13}
\end{minipage}
}
\exercise{$x=\cos t$,\quad $y=\sin(2t)$,\quad $0\leq t \leq 2\pi$}{\begin{minipage}{\linewidth}
\myincludegraphics{figures/fig09_02_ex_14}
\end{minipage}
}
\exercise{$x=2\sec t$,\quad $y=3\tan t$,\quad $-\pi/2< t < \pi/2$}{\begin{minipage}{\linewidth}
\myincludegraphics{figures/fig09_02_ex_15}
\end{minipage}
}
\exercise{$x=\cos t+\frac14\cos(8t)$,\quad $y=\sin t+\frac14\sin(8t)$,\quad $0\leq t \leq 2\pi$}{\begin{minipage}{\linewidth}
\myincludegraphics{figures/fig09_02_ex_16}
\end{minipage}
}
\exercise{$x=\cos t+\frac14\sin(8t)$,\quad $y=\sin t+\frac14\cos(8t)$,\quad $0\leq t \leq 2\pi$}{\begin{minipage}{\linewidth}
\myincludegraphics{figures/fig09_02_ex_17}
\end{minipage}
}}

\input{exercises/09_02_exset_03}

\begin{exerciseset}{In Exercises}{, find a parameterization for the curve.}

\exercise{$y=9-4x$}{Possible Answer: $x=t$, $y=9-4t$}

\exercise{$4x-y^2=5$}{Possible Answer: $x=\frac{5+t^2}{4}$, $y=t$}

\exercise{$(x+9)^2 + (y-4)^2 =49$}{Possible Answer: $x=-9+7\cos t$, $y=4+7\sin t$}

\exercise{$(x-2)^2 - (y+3)^2 =25$}{Possible Answer: $x=2+5\sec t$, $y=-3+5\tan t$}

\end{exerciseset}

\exerciseset{In Exercises}{, find a parameterization for the curve.}{

\exercise{$y=9-4x$}{Possible Answer: $x=t$, $y=9-4t$}

\exercise{$4x-y^2=5$}{Possible Answer: $x=\frac{5+t^2}{4}$, $y=t$}

\exercise{$(x+9)^2 + (y-4)^2 =49$}{Possible Answer: $x=-9+7\cos t$, $y=4+7\sin t$}

}

\exerciseset{In Exercises}{, find a parametric equation and a parameter interval.}{

\exercise{The line segment with endpoints $(-1, -3)$ and $(4,1)$}{Possible Answer: $x=\frac{5}{4}t+\frac{11}{4}$, $y=t$, $[-3,1]$}

\exercise{The line segment with endpoints $(-1, 3)$ and $(3,-2)$}{Possible Answer: $x=-1+4t$, $y=3-5t$, $[0,1]$}

\exercise{The left half of the parabola $y=x^2 + 2x$}{Possible Answer: $x=t$, $y=t^2+2t$, $(-\infty,-1]$}

\exercise{The lower half of the parabola $x=1-y^2$}{Possible Answer: $x=2t-t^2$, $y=1-t$, $[1,\infty)$}

}


\input{exercises/09_02_exset_06}

\exercise{Find parametric equations and a parameter interval for the motion of a particle that starts at $(1, 0)$ and traces the circle $x^2 + y^2 =1$\\
\begin{minipage}{.5\linewidth}
\begin{enumerate}\raggedright
\item once clockwise
\item once counter-clockwise
\end{enumerate}
\end{minipage}%
\begin{minipage}{.5\linewidth}
\begin{enumerate}\addtocounter{enumii}{2}\raggedright
\item twice clockwise
\item twice counter-clockwise
\end{enumerate}
\end{minipage}}{Possible answers:
\begin{enumerate}
\item $x=\sin t$, $y=\cos t$, $[\pi/2, 5\pi/2]$
\item $x=\cos t$, $y=\sin t$, $[0, 2\pi]$
\item $x=\sin t$, $y=\cos t$, $[\pi/2, 9\pi/2]$
\item $x=\cos t$, $y=\sin t$, $[0, 4\pi]$
\end{enumerate}}

\exercise{Find parametric equations and a parameter interval for the motion of a particle that starts at $(a, 0)$ and traces the ellipse $\frac{x^2}{a^2} + \frac{y^2}{b^2} =1$\\
\begin{minipage}{.5\linewidth}
\begin{enumerate}\raggedright
\item once clockwise
\item once counter-clockwise
\end{enumerate}
\end{minipage}%
\begin{minipage}{.5\linewidth}
\begin{enumerate}\addtocounter{enumii}{2}\raggedright
\item twice clockwise
\item twice counter-clockwise
\end{enumerate}
\end{minipage}}{Possible Answers:
\begin{enumerate}
\item $x=a\sin t, y=b \cos t, [\pi/2, 5\pi/2]$
\item $x=a\cos t, y=b \sin t, [0, 2\pi]$
\item $x=a\sin t, y=b \cos t, [\pi/2, 9\pi/2]$
\item $x=a\cos t, y=b \sin t, [0, 4\pi]$
\end{enumerate}}

\exerciseset{In Exercises}{, find parametric equations that describe the given situation.}{

\exercise{A projectile is fired from a height of 0 ft, landing 16 ft away in 4 s.}{$x=4t$, $y=-16t^2 + 64t$}

\exercise{A projectile is fired from a height of 0 ft, landing 200 ft away in 4 s.}{$x=50t$, $y=-16t^2 + 64t$}

\exercise{A projectile is fired from a height of 0 ft, landing 200 ft away in 20 s.}{$x=10t$, $y=-16t^2 + 320t$}

\exercise{A circle of radius 2, centered at the origin, that is traced clockwise once on $[0,2\pi]$.}{$x=2\cos t$, $y=-2\sin t$; other answers possible}

\exercise{A circle of radius 3, centered at $(1,1)$, that is traced once counter-clockwise on $[0,1]$.}{$x=3\cos (2\pi t)+1$, $y=3\sin (2\pi t)+1$; other answers possible}

\exercise{An ellipse centered at $(1,3)$ with vertical major axis of length 6 and minor axis of length 2.}{$x=\cos t+1$, $y=3\sin t +3$; other answers possible}

\exercise{An ellipse with foci at $(\pm 1,0)$ and vertices at $(\pm 5,0)$.}{$x=5\cos t$, $y=\sqrt{24}\sin t$; other answers possible}

\exercise{A hyperbola with foci at $(5,-3)$ and $(-1,-3)$, and with vertices at $(1,-3)$ and $(3,-3)$.}{$x=\pm\sec t+2$, $y=\sqrt{8}\tan t-3$; other answers possible}

\exercise{A hyperbola with vertices at $(0,\pm 6)$ and asymptotes $y=\pm 3x$.}{$x=2\tan t$, $y=\pm 6\sec t$; other answers possible}

\exercise{A lug nut that is 2'' from the center of a car tire.  The tire is 18'' in diameter and rolling at a speed of 10''/sec.}{$x=10t-2\sin t$, $y=10-2\cos t$; other answers possible}

}


\exerciseset{In Exercises}{, eliminate the parameter in the given parametric equations.
}{

\exercise{$\ds x=2t+5$, \quad $\ds y=-3t+1$
}{$y=-1.5x+8.5$
}
\exercise{$x=\sec t$, \quad $y=\tan t$
}{$x^2-y^2=1$
}
\exercise{$x=4\sin t+1$, \quad $y=3\cos t-2$
}{$\frac{(x-1)^2}{16}+\frac{(y+2)^2}{9}=1$
}
\exercise{$x=t^2$, \quad $y=t^3$
}{$y=x^{3/2}$
}
\exercise{$\ds x=\frac{1}{t+1}$, \quad $\ds y=\frac{3t+5}{t+1}$
}{$y=2x+3$
}
\exercise{$\ds x=e^t$, \quad $\ds y=e^{3t}-3$
}{$y=x^3-3$
}
\exercise{$\ds x=\ln t$, \quad $\ds y=t^2-1$
}{$y=e^{2x}-1$
}
\exercise{$\ds x=\cot t$, \quad $\ds y=\csc t$
}{$y^2-x^2=1$
}
\exercise{$\ds x=\cosh t$, \quad $\ds y=\sinh t$
}{$x^2-y^2=1$
}
\exercise{$\ds x=\cos(2t)$, \quad $\ds y=\sin t$
}{$x=1-2y^2$
}}

\exerciseset{In Exercises}{, eliminate the parameter in the given parametric equations. Describe the curve defined by the parametric equations based on its rectangular form.
}{

\exercise{$\ds x=at+x_0$, \quad $\ds y=bt+y_0$
}{$y=\frac{b}{a}(x-x_0)+y_0$; line through $(x_0,y_0)$ with slope $b/a$.
}
\exercise{$\ds x=r\cos t$, \quad $\ds y=r\sin t$
}{$x^2+y^2=r^2$; circle centered at $(0,0)$ with radius $r$.
}
\exercise{$\ds x=a\cos t+h$, \quad $\ds y=b\sin t+k$
}{$\frac{(x-h)^2}{a^2}+\frac{(y-k)^2}{b^2}=1$; ellipse centered at $(h,k)$ with horizontal axis of length $2a$ and vertical axis of length $2b$.
}
\exercise{$\ds x=a\sec t+h$, \quad $\ds y=b\tan t+k$
}{$\frac{(x-h)^2}{a^2}-\frac{(y-k)^2}{b^2}=1$; hyperbola centered at $(h,k)$ with horizontal transverse axis and asymptotes with slope $b/a$. The parametric equations only give half of the hyperbola. When $a>0$, the right half; when $a<0$, the left half.
}}

\exerciseset{In Exercises}{, find the values of $t$ where the graph of the parametric equations crosses itself.
}{

\exercise{$x=t^3-t+3$,\quad $y=t^2-3$
}{$t=\pm 1$
}
\exercise{$x=t^3-4t^2+t+7$,\quad $y=t^2-t$
}{$t=-1,\ 2$
}
\exercise{$x=\cos t$,\quad $y=\sin (2t)$ on $[0,2\pi]$
}{$t=\pi/2, 3\pi/2$
}
\exercise{$x=\cos t\cos (3t)$,\quad $y=\sin t\cos(3t)$ on $[0,\pi]$
}{$t=\pi/6, \pi/2, 5\pi/6$
}
}

\exerciseset{In Exercises}{, find the value(s) of $t$ where the curve defined by the parametric equations is not smooth.}{

\exercise{$x=t^3+t^2-t$,\quad $y=t^2+2t+3$}{$t=-1$}

\exercise{$x=t^2-4t$,\quad $y=t^3-2t^2-4t$}{$t=2$}

\exercise{$x=\cos t$,\quad $y=2\cos t$}{$t=\dotsc \pi/2,\ 3\pi/2,\ 5\pi/2,\ \dotsc$}

\exercise{$x=2\cos t-\cos(2t)$,\quad $y=2\sin t-\sin(2t)$}{$t=\dotsc 0,\ 2\pi,\ 4\pi,\ \dotsc$}

}

