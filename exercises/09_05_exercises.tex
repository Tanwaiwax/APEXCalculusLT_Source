\printconcepts

\exercise{Given polar equation $r=f(\theta)$, how can one create parametric equations of the same curve?}{Using $x=r\cos\theta$ and $y=r\sin\theta$, we can write $x=f(\theta)\cos\theta$, $y=f(\theta)\sin\theta$.}

\exercise{With rectangular coordinates, it is natural to approximate area with \underline{\hskip .5in}; with polar coordinates, it is natural to approximate area with \underline{\hskip .5in}.}{rectangles; sectors of circles}

\printproblems

\exerciseset{In Exercises}{, find:
\begin{enumerate}
	\item [(a)]  $\ds \frac{dy}{dx}$
	\item [(b)]  the equation of the tangent and normal lines to the curve at the indicated $\theta$--value.
\end{enumerate}
}{

\exercise{$r=1$;\quad $\theta = \pi/4$}{\begin{enumerate}
	\item $\frac{dy}{dx} = -\cot\theta$
	\item	tangent line: $y=-(x-\sqrt{2}/2)+\sqrt{2}/2$; normal line: $y=x$
\end{enumerate}}

\exercise{$r=\cos\theta$;\quad $\theta = \pi/4$}{\begin{enumerate}
	\item $\frac{dy}{dx} = 1/2(\tan \theta-\cot\theta)$
	\item	tangent line: $y=1/2$; normal line: $x=1/2$
\end{enumerate}}

\exercise{$r=1+\sin\theta$;\quad $\theta = \pi/6$}{\begin{enumerate}
	\item $\frac{dy}{dx} = \frac{\cos\theta(1+2\sin\theta)}{\cos^2\theta-\sin\theta(1+\sin\theta)}$
	\item	tangent line: $x=3\sqrt{3}/4$; normal line: $y=3/4$
\end{enumerate}}

\exercise{$\ds r=1-3\cos\theta$;\quad $\theta = 3\pi/4$}{\begin{enumerate}
	\item $\frac{dy}{dx} = \frac{3 \sin ^2(t)+(1-3 \cos (t)) \cos (t)}{3 \sin (t) \cos
   (t)-\sin (t) (1-3 \cos (t))}$
	\item	tangent line: $y=\frac{1}{1+3\sqrt{2}}(x+(1/\sqrt{2}+3/2))+1/\sqrt{2}+3/2 \approx y=0.19(x+2.21)+2.21$; normal line: $y=-(1+3\sqrt{2})(x+(1/\sqrt{2}+3/2))+1/\sqrt{2}+3/2$
\end{enumerate}}

\exercise{$r=\theta$;\quad $\theta = \pi/2$}{\begin{enumerate}
	\item $\frac{dy}{dx} = \frac{\theta\cos\theta+\sin\theta}{\cos\theta-\theta\sin\theta}$
	\item	tangent line: $y=-2/\pi x+\pi/2$; normal line: $y=\pi/2x+\pi/2$
\end{enumerate}}

\exercise{$r=\cos(3\theta)$;\quad $\theta = \pi/6$}{\begin{enumerate}
	\item $\frac{dy}{dx} = \frac{\cos\theta\cos(3\theta)-3\sin\theta\sin(3\theta)}{-\cos(3\theta)\sin\theta-3\cos\theta\sin(3\theta)}$
	\item	tangent line: $y=x/\sqrt{3}$; normal line: $y=-\sqrt{3}x$
\end{enumerate}}

\exercise{$r=\sin(4\theta)$;\quad $\theta = \pi/3$}{\begin{enumerate}
	\item $\frac{dy}{dx}=\frac{4\sin(\theta)\cos(4\theta)+\sin(4\theta)\cos(\theta)}{4\cos(\theta)\cos(4\theta)-\sin(\theta)\sin(4\theta)}$
	\item	tangent line: $y=5\sqrt{3}(x+\sqrt{3}/4)-3/4$; normal line: $y=-1/5\sqrt{3}(x+\sqrt{3}/4)-3/4$
\end{enumerate}}

\exercise{$\ds r=\frac1{\sin\theta-\cos\theta}$;\quad $\theta = \pi$}{\begin{enumerate}
	\item $\frac{dy}{dx} = 1$
	\item	tangent line: $y=x+1$; normal line: $y=-x-1$
\end{enumerate}}
}

\input{exercises/09_05_exset_02}

\exerciseset{In Exercises}{, find the equation of the lines tangent to the graph at the pole.
}{

\exercise{$\ds r=\sin\theta$;\quad $[0,\pi]$}{In polar: $\theta = 0 \ \cong \ \theta = \pi$

In rectangular: $y=0$
}

\exercise{$\ds r=\sin(3\theta)$;\quad $[0,\pi]$}{In polar: $\theta = 0,\ \theta = \pi/3,\ \theta = 2\pi/3$.

In rectangular: $y=0$, $y=\sqrt{3}x$, and $y= -\sqrt{3}x$.
}
}

\ifthenelse{\boolean{printquestions}}{\columnbreak}{}

\exerciseset{In Exercises}{, find the area of the described region.}{

\exercise{Enclosed by the circle: $r=4\sin\theta$,\quad $\frac\pi3\leq\theta\leq\frac{2\pi}3$}{area = $\frac{8\pi}{3}+4\sqrt 3$}

\exercise{Enclosed by the circle $\ds r=5$}{area = $25\pi$}

\exercise{Enclosed by one petal of $\ds r=\sin(3\theta)$}{area = $\pi/12$}

\exercise{Enclosed by the cardioid $\ds r=1-\sin\theta$}{area = $3\pi/2$}

\exercise{Enclosed by the inner loop of the lima\c con $\ds r=1+2\cos\theta$}{area = $\pi-3\sqrt{3}/2$}

\exercise{Enclosed by the outer loop of the lima\c con $\ds r=1+2\cos\theta$ (including area enclosed by the inner loop)}{area = $2\pi+3\sqrt{3}/2$}

\exercise{Enclosed between the inner and  outer loop of the lima\c con $\ds r=1+2\cos\theta$}{area = $\pi+3\sqrt{3}$}

\exercise{Enclosed by $r=2\cos \theta$ and $r=2\sin\theta$, as shown:

\begin{tikzpicture}[>=stealth,scale=.75]
\begin{axis}[width=\marginparwidth+25pt,tick label style={font=\scriptsize},
axis y line=middle,axis x line=middle,name=myplot,axis on top,
ymin=-1.1,ymax=2.1,xmin=-1.4,xmax=2.4]

\addplot [thick,{\coloronefill},fill={\coloronefill}, smooth,domain=0:45,samples=20] ({cos(x)*2*cos(x)},{sin(x)*2*cos(x)}) -- (axis cs:0,0);

\addplot [thick,{\coloronefill},fill=white, smooth,domain=0:50,samples=20] ({cos(x)*2*sin(x)},{sin(x)*2*sin(x)});

\addplot [thick,{\colorone}, smooth,domain=0:180,samples=60] ({cos(x)*2*cos(x)},{sin(x)*2*cos(x)});

\addplot [thick,{\colortwo}, smooth,domain=0:180,samples=40] ({cos(x)*2*sin(x)},{sin(x)*2*sin(x)});

\end{axis}

\node [right] at (myplot.right of origin) {\scriptsize $x$};
\node [above] at (myplot.above origin) {\scriptsize $y$};
\end{tikzpicture}
}{area = $1$}

\exercise{Enclosed by $r=\cos(3 \theta)$ and $r=\sin(3\theta)$, as shown:

\begin{tikzpicture}[>=stealth,scale=.75]
\begin{axis}[width=\marginparwidth+25pt,tick label style={font=\scriptsize},
axis y line=middle,axis x line=middle,name=myplot,axis on top,
xtick={1},ytick={.5},ymin=-.25,ymax=.7,xmin=-.08,xmax=1.1]

\addplot [thick,{\coloronefill},fill={\coloronefill}, smooth,domain=15:60,samples=20] ({cos(x)*sin(3*x)},{sin(x)*sin(3*x)}) -- (axis cs:0,0);

\addplot [thick,white,fill=white, smooth,domain=0:30,samples=20] ({cos(x)*cos(3*x)},{sin(x)*cos(3*x)});

\addplot [thick,{\colorone}, smooth,domain=0:60,samples=60] ({cos(x)*sin(3*x)},{sin(x)*sin(3*x)});

\addplot [thick,{\colortwo}, smooth,domain=-30:30,samples=60] ({cos(x)*cos(3*x)},{sin(x)*cos(3*x)});

\end{axis}

\node [right] at (myplot.right of origin) {\scriptsize $x$};
\node [above] at (myplot.above origin) {\scriptsize $y$};
\end{tikzpicture}
}{area = $\ds \int_{\pi/12}^{\pi/3} \frac12 \sin^2(3\theta)\ d\theta - \int_{\pi/12}^{\pi/6}\frac12\cos^2(3\theta)\ d\theta = \frac1{12}+\frac{\pi}{24}$}

\exercise{Enclosed by $r=\cos \theta$ and $r=\sin(2\theta)$, as shown:

\begin{tikzpicture}[>=stealth,scale=.75]
\begin{axis}[width=\marginparwidth+25pt,tick label style={font=\scriptsize},
axis y line=middle,axis x line=middle,name=myplot,axis on top,
xtick={1},ytick={1},ymin=-.1,ymax=1.1,xmin=-.1,xmax=1.34]

\addplot [thick,{\coloronefill},fill={\coloronefill}, smooth,domain=90:30,samples=20] ({cos(x)*cos(x)},{sin(x)*cos(x)}) -- (axis cs:0,0);

\addplot [thick,{\coloronefill},fill={\coloronefill}, smooth,domain=0:30,samples=20] ({cos(x)*sin(2*x)},{sin(x)*sin(2*x)}) -- (axis cs:0,0);

\addplot [thick,{\colorone}, smooth,domain=0:90,samples=60] ({cos(x)*cos(x)},{sin(x)*cos(x)});

\addplot [thick,{\colortwo}, smooth,domain=0:90,samples=40] ({cos(x)*sin(2*x)},{sin(x)*sin(2*x)});

\end{axis}

\node [right] at (myplot.right of origin) {\scriptsize $x$};
\node [above] at (myplot.above origin) {\scriptsize $y$};
\end{tikzpicture}
}{area = $\frac{1}{32}(4\pi-3\sqrt{3})$}

\exercise{Enclosed by $r=\cos\theta$ and $r=1-\cos\theta$, as shown:

\begin{tikzpicture}[>=stealth,scale=.75]
\begin{axis}[width=\marginparwidth+25pt,tick label style={font=\scriptsize},
axis y line=middle,axis x line=middle,name=myplot,axis on top,
ymin=-1.4,ymax=1.4,xmin=-2.2,xmax=1.2]

\addplot [thick,{\coloronefill},fill={\coloronefill}, smooth,domain=60:90,samples=60] ({cos(x)*cos(x)},{sin(x)*cos(x)});

\addplot [thick,{\coloronefill},fill={\coloronefill}, smooth,domain=0:60,samples=40] ({cos(x)*(1-cos(x))},{sin(x)*(1-cos(x))});

\addplot [thick,{\colorone}, smooth,domain=0:180,samples=60] ({cos(x)*cos(x)},{sin(x)*cos(x)});

\addplot [thick,{\colortwo}, smooth,domain=0:360,samples=60] ({cos(x)*(1-cos(x))},{sin(x)*(1-cos(x))});

\end{axis}

\node [right] at (myplot.right of origin) {\scriptsize $x$};
\node [above] at (myplot.above origin) {\scriptsize $y$};
\end{tikzpicture}
}{area = $\ds \int_{0}^{\pi/3} \frac12 (1-\cos\theta)^2\ d\theta +\int_{\pi/3}^{\pi/2} \frac12 (\cos\theta)^2\ d\theta =\frac{7\pi}{24}-\frac{\sqrt{3}}2\approx 0.0503$}

}


\exerciseset{In Exercises}{, answer the questions involving arc length.}{

\exercise{Let $x(\theta) = f(\theta)\cos\theta$ and $y(\theta)=f(\theta)\sin\theta$. Show, as suggested by the text, that 
\[x\,'(\theta)^2+y\,'(\theta)^2 = \fp(\theta)^2+f(\theta)^2.\]}{$x\,'(\theta) = \fp(\theta)\cos\theta -f(\theta)\sin\theta$, $y\,'(\theta) = \fp(\theta)\sin\theta + f(\theta)\cos\theta$. Square each and add; applying the Pythagorean Theorem twice achieves the result.}

\exercise{Use the arc length formula to compute the arc length of the circle $r=2$.}{$4\pi$}

\exercise{Use the arc length formula to compute the arc length of the circle $r=4\sin\theta$.}{$4\pi$}

\exercise{Use the arc length formula to compute the arc length of $r=\cos\theta+\sin\theta$.}{area = $\pi\sqrt2$}

\exercise{Approximate the arc length of one petal of the rose curve $r=\sin(3\theta)$ with Simpson's Rule and $n=4.$}{$L\approx 2.2592$; (actual value $L=2.22748$)}

\exercise{Approximate the arc length of the cardioid $r=1+\cos\theta$ with Simpson's Rule and $n=6.$% exact value possible: (Hint: apply the formula, simplify, then use a Power–Reducing Formula to convert $1+\cos\theta$ into a square.)
}{$L\approx 7.62933$; (actual value $L=8$)}

}


\exerciseset{In Exercises}{, answer the questions involving surface area.
}{

\exercise{Use Key Idea \ref{idea:surface_area_polar} to find the surface area of the sphere formed by revolving the circle $r=2$ about the initial ray.
}{$SA = 16\pi$
}

\exercise{Use Key Idea \ref{idea:surface_area_polar} to find the surface area of the sphere formed by revolving the circle $r=2\cos\theta$ about the initial ray.
}{$SA = 4\pi$
}

\exercise{Find the surface area of the solid formed by revolving the cardiod $r=1+\cos\theta$ about the initial ray.
}{$SA = 32\pi/5$
}

\exercise{Find the surface area of the solid formed by revolving the circle $r=2\cos\theta$ about the line $\theta=\pi/2$.
}{$SA = 4\pi^2$
}

\exercise{Find the surface area of the solid formed by revolving the line $r=3\sec\theta$, $-\pi/4\leq\theta\leq\pi/4$, about the line $\theta=\pi/2$.
}{$SA = 36\pi$
}
}
