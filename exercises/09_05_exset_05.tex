\exerciseset{In Exercises}{, answer the questions involving arc length.}{

\exercise{Let $x(\theta) = f(\theta)\cos\theta$ and $y(\theta)=f(\theta)\sin\theta$. Show, as suggested by the text, that 
\[x\,'(\theta)^2+y\,'(\theta)^2 = \fp(\theta)^2+f(\theta)^2.\]}{$x\,'(\theta) = \fp(\theta)\cos\theta -f(\theta)\sin\theta$, $y\,'(\theta) = \fp(\theta)\sin\theta + f(\theta)\cos\theta$. Square each and add; applying the Pythagorean Theorem twice achieves the result.}

\exercise{Use the arc length formula to compute the arc length of the circle $r=2$.}{$4\pi$}

\exercise{Use the arc length formula to compute the arc length of the circle $r=4\sin\theta$.}{$4\pi$}

\exercise{Use the arc length formula to compute the arc length of $r=\cos\theta+\sin\theta$.}{area = $\pi\sqrt2$}

\exercise{Approximate the arc length of one petal of the rose curve $r=\sin(3\theta)$ with Simpson's Rule and $n=4.$}{$L\approx 2.2592$; (actual value $L=2.22748$)}

\exercise{Approximate the arc length of the cardioid $r=1+\cos\theta$ with Simpson's Rule and $n=6.$% exact value possible: (Hint: apply the formula, simplify, then use a Power–Reducing Formula to convert $1+\cos\theta$ into a square.)
}{$L\approx 7.62933$; (actual value $L=8$)}

}
