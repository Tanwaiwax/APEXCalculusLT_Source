\printconcepts

\exercise{Axes drawn in space must conform to the \underline{\hskip .5in} \underline{\hskip .5in} rule.}{right hand}

\exercise{In the plane, the equation $x=2$ defines a \underline{\hskip.5in}; in space, $x=2$ defines a \underline{\hskip .5in}.}{line; plane}

\exercise{In the plane, the equation $y=x^2$ defines a \underline{\hskip.5in}; in space, $y=x^2$ defines a \underline{\hskip .5in}.}{curve (a parabola); surface (a cylinder)}

\exercise{Which quadric surface looks like a Pringles\textregistered\ chip?}{a hyperbolic paraboloid}

\exercise{Consider the hyperbola $x^2-y^2=1$ in the plane. If this hyperbola is rotated about the $x$-axis, what quadric surface is formed?}{a hyperboloid of two sheets}

\exercise{Consider the hyperbola $x^2-y^2=1$ in the plane. If this hyperbola is rotated about the $y$-axis, what quadric surface is formed?}{a hyperboloid of one sheet}

\printproblems

\exercise{The points $A=(1,4,2)$, $B=(2,6,3)$ and $C=(4,3,1)$ form a triangle in space. Find the distances between each pair of points and determine if the triangle is a right triangle.}{$\norm{\overline{AB}} = \sqrt{6}$; $\norm{\overline{BC}} = \sqrt{17}$; $\norm{\overline{AC}} = \sqrt{11}$. Yes, it is a right triangle as $\norm{\overline{AB}}^2+\norm{\overline{AC}}^2=\norm{\overline{BC}}^2$.}

\exercise{The points $A=(3,2,0)$, $B=(1,0,1)$, and $C=(-1,1,3)$ form a triangle in space.  Find the distance between each pair of points and determine if the triangle is isosceles.}{$\norm{\overline{AB}}=3$, $\norm{\overline{AC}}=\sqrt{26}$, $\norm{\overline{BC}}=3$. The triangle is isosceles.}

\exercise{Explain why three points lie on a line if and only if the distance between two of the points is equal to the sum of the distances from each of these points to the third.}{}

\exercise{Determine whether or not the points $A=(2,0,3)$, $B=(0,-3,4)$, and $C=(4,3,2)$ lie on a line.}{$\norm{\overline{AB}}=\sqrt{14}$, $\norm{\overline{AC}}=\sqrt{14}$, $\norm{\overline{BC}}=2\sqrt{14}$.  The points lie on a line.}

\exercise{Determine whether or not the points $A=(4,5,3)$, $B=(6,8,7)$, and $C=(0,0,-5)$ lie on a line.}{$\norm{\overline{AB}}=\sqrt{29}$, $\norm{\overline{AC}}=\sqrt{105}$, $\norm{\overline{BC}}=2\sqrt{61}$.  The points do not lie on a line.}

\exercise{The points $A=(1,1,3)$, $B=(3,2,7)$, $C=(2,0,8)$ and $D = (0,-1,4)$ form a quadrilateral $ABCD$ in space. Is this a parallelogram?}{Yes, as opposite sides have equal length. $\norm{\overline{AB}} = \sqrt{21}=\norm{\overline{CD}}$; $\norm{\overline{BC}} = \sqrt{6}=\norm{\overline{AD}}$.}

\exercise{Find the center and radius of the sphere defined by\\
$x^2-8x+y^2+2y+z^2+8=0.$}{Center at $(4,-1,0)$; radius = 3}

\exercise{Find the center and radius of the sphere defined by\\
$x^2+y^2+z^2+4x-2y-4z+4=0.$}{Center at $(-2,1,2)$; radius = $\sqrt{5}$}

\exercise{Show that the point $A=(2,1,5)$ is inside the sphere given by
\[x^2+y^2+z^2-2x+4y-6z=11.\]
Is $A$ closer to the center of the sphere or to the surface of the sphere?}{closer to the surface}
% (1,-2,3) radius=5
% d( center, A ) = sqrt(14) ~ 3.7

\exercise{Let $P = (x_1, y_1, z_1)$ and $Q = (x_2, y_2, z_2)$ be points in space.  Show that the midpoint of $\overline{PQ}$ is 
\[\left(\frac{x_1+x_2}2,\frac{y_1+y_2}2,\frac{z_1+z_2}2\right).\]}{}

\exerciseset{In Exercises}{, describe the region in space defined by the inequalities.
}{

\exercise{$x^2+y^2+z^2<1$
}{Interior of a sphere with radius 1 centered at the origin.
}

\exercise{$0\leq x\leq 3$
}{Region bounded between the planes $x=0$ (the $y-z$ coordinate plane) and $x=3$.
}

\exercise{$x\geq 0,\ y\geq0, \ z\geq0$
}{The first octant of space; all points $(x,y,z)$ where each of $x$, $y$ and $z$ are positive. (Analogous to the first quadrant in the plane.)
}

\exercise{$y\geq 3$
}{All points in space where the $y$ value is greater than 3; viewing space as often depicted in this text, this is the region ``to the right'' of the plane $y=3$ (which is parallel to the $x-z$ coordinate plane.)
}
}

\exerciseset{In Exercises}{, sketch the cylinder in space.}{

\exercise{$z=x^3$
}{\begin{minipage}[m]{\linewidth}
\myincludegraphicsthree{width=125pt,3Dmenu,activate=onclick,deactivate=pageinvisible,
3Droll=0,
3Dortho=0.0045,
3Dc2c=.7 .55 .43,
3Dcoo=0 0 0,
3Droo=150,
3Dlights=Headlamp,add3Djscript=asylabels.js}{scale=1.25}{figures/fig10_01_ex_15}
\end{minipage}
}

\exercise{$y=\cos z$
}{\begin{minipage}[m]{\linewidth}
\myincludegraphicsthree{width=125pt,3Dmenu,activate=onclick,deactivate=pageinvisible,
3Droll=0,
3Dortho=0.0045,
3Dc2c=.7 .55 .43,
3Dcoo=0 0 0,
3Droo=150,
3Dlights=Headlamp,add3Djscript=asylabels.js}{scale=1.25}{figures/fig10_01_ex_16}
\end{minipage}
}

\exercise{$\ds \frac{x^2}{4}+\frac{y^2}{9}=1$
}{\begin{minipage}[m]{\linewidth}
\myincludegraphicsthree{width=125pt,3Dmenu,activate=onclick,deactivate=pageinvisible,
3Droll=0,
3Dortho=0.0045,
3Dc2c=.7 .55 .43,
3Dcoo=0 0 0,
3Droo=150,
3Dlights=Headlamp,add3Djscript=asylabels.js}{scale=1.25}{figures/fig10_01_ex_17}
\end{minipage}
}

\exercise{$\ds y=\frac1x$
}{\begin{minipage}[m]{\linewidth}
\myincludegraphicsthree{width=125pt,3Dmenu,activate=onclick,deactivate=pageinvisible,
3Droll=0,
3Dortho=0.0045,
3Dc2c=.7 .55 .43,
3Dcoo=0 0 0,
3Droo=150,
3Dlights=Headlamp,add3Djscript=asylabels.js}{scale=1.25}{figures/fig10_01_ex_18}
\end{minipage}
}
}

\exerciseset{In Exercises}{, give the equation of the surface of revolution described.}{

\exercise{Revolve $\ds z=\frac1{1+y^2}$ about the $y$-axis.}{$x^2+z^2=\frac1{(1+y^2)^2}$}

\exercise{Revolve $\ds y=x^2$ about the $x$-axis.}{$y^2+z^2=x^4$}

\exercise{Revolve $\ds z=x^2$ about the $z$-axis.}{$z=(\sqrt{x^2+y^2})^2=x^2+y^2$}

\exercise{Revolve $\ds z=1/x$ about the $z$-axis.}{$z=\frac{1}{\sqrt{x^2+y^2}}$}

}


\exerciseset{In Exercises}{, a quadric surface is sketched. Determine which of the given equations best fits the graph.
}{

\exercise{\begin{minipage}[m]{\linewidth}
\centering
\myincludegraphicsthree{width=100pt,3Dmenu,activate=onclick,deactivate=pageinvisible,
3Droll=0,
3Dortho=0.0045,
3Dc2c=.7 .55 .43,
3Dcoo=0 0 0,
3Droo=150,
3Dlights=Headlamp,add3Djscript=asylabels.js}{scale=1.25}{figures/fig10_01_ex_19}
%\myincludegraphics{figures/fig10_01_ex_19}

(a)\quad $\ds x=y^2+\frac{z^2}{9}$ \qquad\qquad (b)\quad $\ds x=y^2+\frac{z^2}{3}$
\end{minipage}
}{(a)\quad $\ds x=y^2+\frac{z^2}{9}$
}

\exercise{\begin{minipage}[m]{\linewidth}
\centering
\myincludegraphicsthree{width=100pt,3Dmenu,activate=onclick,deactivate=pageinvisible,
3Droll=0,
3Dortho=0.0045,
3Dc2c=.7 .55 .43,
3Dcoo=0 0 0,
3Droo=150,
3Dlights=Headlamp,add3Djscript=asylabels.js}{scale=1.25}{figures/fig10_01_ex_20}
%\myincludegraphics{figures/fig10_01_ex_20}

(a)\quad $\ds x^2-y^2-z^2=0$ \qquad\qquad (b)\quad $x^2-y^2+z^2=0$
\end{minipage}
}{(b)\quad $x^2-y^2+z^2=0$
}

\exercise{\begin{minipage}[m]{\linewidth}
\centering
\myincludegraphicsthree{width=100pt,3Dmenu,activate=onclick,deactivate=pageinvisible,
3Droll=0,
3Dortho=0.0045,
3Dc2c=.7 .55 .43,
3Dcoo=0 0 0,
3Droo=150,
3Dlights=Headlamp,add3Djscript=asylabels.js}{scale=1.25}{figures/fig10_01_ex_21}
%\myincludegraphics{figures/fig10_01_ex_21}

(a)\quad $\ds x^2+\frac{y^2}3+\frac{z^2}2=1$ \qquad\qquad (b)\quad $\ds x^2+\frac{y^2}9+\frac{z^2}4=1$
\end{minipage}
}{(b)\quad $\ds x^2+\frac{y^2}9+\frac{z^2}4=1$
}

\exercise{\begin{minipage}[m]{\linewidth}
\centering
\myincludegraphicsthree{width=100pt,3Dmenu,activate=onclick,deactivate=pageinvisible,
3Droll=0,
3Dortho=0.0045,
3Dc2c=.7 .55 .43,
3Dcoo=0 0 0,
3Droo=150,
3Dlights=Headlamp,add3Djscript=asylabels.js}{scale=1.25}{figures/fig10_01_ex_22}
%\myincludegraphics{figures/fig10_01_ex_22}

(a)\quad $y^2-x^2-z^2=1$ \qquad\qquad (b)\quad $y^2+x^2-z^2=1$
\end{minipage}
}{(a)\quad $y^2-x^2-z^2=1$
}
}

\exerciseset{In Exercises}{, sketch the quadric surface.}{

\exercise{$\ds z-y^2+x^2=0$}{\mbox{}\\[-\baselineskip]
\myincludeasythree{width=.7\linewidth,
3Droll=0,
3Dortho=0.0045,
3Dc2c=.7 .55 .43,
3Dcoo=0 0 0,
3Droo=150}{width=.7\linewidth}{figures/fig10_01_ex_28_3D}}

\exercise{$\ds z^2=x^2+\frac{y^2}4$}{\mbox{}\\[-\baselineskip]
\myincludeasythree{width=.7\linewidth,
3Droll=0,
3Dortho=0.0045,
3Dc2c=.7 .55 .43,
3Dcoo=0 0 0,
3Droo=150}{width=.7\linewidth}{figures/fig10_01_ex_24_3D}}

\exercise{$x=-y^2-z^2$}{\mbox{}\\[-\baselineskip]
\myincludeasythree{width=.7\linewidth,
3Droll=0,
3Dortho=0.0045,
3Dc2c=.7 .55 .43,
3Dcoo=0 0 0,
3Droo=150}{width=.7\linewidth}{figures/fig10_01_ex_23_3D}}

\exercise{$\ds 16x^2-16y^2-16z^2=1$}{\mbox{}\\[-\baselineskip]
\myincludeasythree{width=.7\linewidth,
3Droll=0,
3Dortho=0.0045,
3Dc2c=.7 .55 .43,
3Dcoo=0 0 0,
3Droo=150}{width=.7\linewidth}{figures/fig10_01_ex_27_3D}}

\exercise{$\ds \frac{x^2}9-y^2+\frac{z^2}{25}=1$}{\mbox{}\\[-\baselineskip]
\myincludeasythree{width=.7\linewidth,
3Droll=0,
3Dortho=0.0045,
3Dc2c=.7 .55 .43,
3Dcoo=0 0 0,
3Droo=150}{width=.7\linewidth}{figures/fig10_01_ex_26_3D}}

\exercise{$\ds 4x^2+2y^2+z^2=4$}{\mbox{}\\[-\baselineskip]
\myincludeasythree{width=.7\linewidth,
3Droll=0,
3Dortho=0.0045,
3Dc2c=.7 .55 .43,
3Dcoo=0 0 0,
3Droo=150}{width=.7\linewidth}{figures/fig10_01_ex_25_3D}}

}

