\printconcepts

\exercise{The cross product of two vectors is a \underline{\hskip .5in}, not a scalar.}{vector}

\exercise{One can visualize the direction  of $\vec u\times\vec v$ using the\\ \underline{\hskip .5in} \underline{\hskip .5in} \underline{\hskip .5in}.}{right hand rule}

\exercise{Give a synonym for ``orthogonal.''}{``Perpendicular'' is one answer.}

\exercise{T/F: A fundamental principle of the cross product is that $\vec u\times\vec v$ is orthogonal to $\vec u$ and $\vec v$.}{T}

\exercise{\underline{\hskip .5in} is a measure of the turning force applied to an object.}{Torque}

\exercise{T/F: If $\vec u$ and $\vec v$ are parallel, then $\vec u\times\vec v=\vec 0$.}{T}

\printproblems

\exercise{State whether each expression is meaningful.
 If not, explain why.  If so, state whether it is a vector or a scalar.

\begin{minipage}{.5\linewidth}
\begin{enumerate}
 \item $\vec a\cdot(\vec b\times\vec c)$
 \item $\vec a\times(\vec b\times\vec c)$
 \item $(\vec a\cdot\vec b)\times(\vec c\cdot\vec d)$
\end{enumerate}
\end{minipage}%
\begin{minipage}{.5\linewidth}
\begin{enumerate}\setcounter{enumii}{3}
 \item $\vec a\times(\vec b\cdot\vec c)$
 \item $(\vec a\times\vec b)(\vec c\times\vec d)$
 \item $(\vec a\times\vec b)\cdot(\vec c\times\vec d)$
\end{enumerate}
\end{minipage}}{\mbox{}\\[-2\baselineskip]
 \begin{enumerate}
  \item $\vec a\cdot(\vec b\times\vec c)
  =\vec a\cdot(\text{vector})=\text{scalar}$
  \item $\vec a\times(\vec b\times\vec c)
  =\vec a\times(\text{vector})=\text{vector}$
  \item $(\vec a\cdot\vec b)\times(\vec c\cdot\vec d)
  =(\text{scalar})\times(\text{scalar})=\text{not meaningful}$
  \item $\vec a\times(\vec b\cdot\vec c)
  =\vec a\cdot(\text{scalar})=\text{not meaningful}$
  \item $(\vec a\times\vec b)(\vec c\times\vec d)
  =(\text{vector})(\text{vector})=\text{not meaningful}$
  \item $(\vec a\times\vec b)\cdot(\vec c\times\vec d)
  =(\text{vector})\cdot(\text{vector})=\text{scalar}$
 \end{enumerate}}

% todo the solution to 11.4#8
\exercise{For any $\vec{u}$ and $\vec{v}$ in $\mathbb{R}^3$ we have 
\[
\norm{\vec{u}}^2\norm{\vec{v}}^2
= (\vec{u} \cdot \vec{v})^2 + \norm{\vec{u} \times \vec{v}}^2.
\]
(This can be verified directly by a somewhat messy algebraic computation.)  Use this formula to show that 
\[\norm{\vec{u}\times\vec{v}}=\norm{\vec{u}}\norm{\vec{v}}\sin \theta\]
where $\theta$ is the angle between $\vec{u}$ and $\vec{v}$.}{}

\exerciseset{In Exercises}{, calculate the determinant.}{

\exercise{$\begin{vmatrix}4 & 1 \\2 & 5\end{vmatrix}$}{18}

\exercise{$\begin{vmatrix}-1 & 2 \\\frac{1}{2} & 4\end{vmatrix}$}{$-5$}

\exercise{$\begin{vmatrix}1 & 2 & 3 \\4 & 5 & 6 \\7 & 8 & 9\end{vmatrix}$}{0}

\exercise{$\begin{vmatrix}2 & 1 & 3 \\4 & -1 & -5 \\1 & 2 & 1\end{vmatrix}$}{36}

}

\begin{exerciseset}{In Exercises}{, vectors $\vec u$ and $\vec v$ are given. Compute $\vec u\times\vec v$ and show this is orthogonal to both $\vec u$ and $\vec v$.}

\exercise{$\vec u =\bracket{3,2,-2}$,\quad $\vec v =\bracket{0,1,5}$}{$\vec u\times\vec v =\bracket{12,-15,3}$}

\exercise{$\vec u =\bracket{5,-4,3}$,\quad $\vec v =\bracket{2,-5,1}$}{$\vec u\times\vec v =\bracket{11,1,-17}$}

\exercise{$\vec u =\bracket{4,-5,-5}$,\quad $\vec v =\bracket{3,3,4}$}{$\vec u\times\vec v =\bracket{-5,-31,27}$}

\exercise{$\vec u =\bracket{-4,7,-10}$,\quad $\vec v =\bracket{4,4,1}$}{$\vec u\times\vec v =\bracket{47,-36,-44}$}

\exercise{$\vec u =\bracket{1,0,1}$,\quad $\vec v =\bracket{5,0,7}$}{$\vec u\times\vec v =\bracket{0,-2,0}$}

\exercise{$\vec u =\bracket{1,5,-4}$,\quad $\vec v =\bracket{-2,-10,8}$}{$\vec u\times\vec v =\bracket{0,0,0}$}

\exercise{$\vec u = \veci$,\quad $\vec v = \vecj$}{$\veci\times\vecj = \vec k$}

\exercise{$\vec u = \veci$,\quad $\vec v = \vec k$}{$\veci\times\vec k = -\vecj$}

%\exercise{$\vec u = \vecj$,\quad $\vec v = \vec k$}{$\vecj\times\vec k = \veci$}

\end{exerciseset}


\exercise{Pick any vectors $\vec u$, $\vec v$ and $\vec w$ in $\mathbb{R}^3$ and show that $\vec u \times (\vec v+\vec w) = \vec u\times \vec v+\vec u\times \vec w$.}{Answers will vary.}

\exercise{Pick any vectors $\vec u$, $\vec v$ and $\vec w$ in $\mathbb{R}^3$ and show that $\vec u \cdot (\vec v\times\vec w) = (\vec u\times \vec v)\cdot \vec w$.}{Answers will vary.}

\input{exercises/10_04_exset_02}

\exerciseset{In Exercises}{, find the area of the parallelogram defined by the given vectors.
}{

\exercise{$\vec u = \la 1,1,2\ra$,\quad $\vec v = \la 2,0,3\ra$
}{$\sqrt{14}$
}

\exercise{$\vec u = \la -2,1,5\ra$,\quad $\vec v = \la -1,3,1\ra$
}{$\sqrt{230}$
}

\exercise{$\vec u = \la 1,2\ra$,\quad $\vec v = \la 2,1\ra$
}{$3$
}

\exercise{$\vec u = \la 2,0\ra$,\quad $\vec v = \la 0,3\ra$
}{$6$
}
}

\exerciseset{In Exercises}{, find the area of the triangle with the given vertices.
}{

\exercise{Vertices: $(0,0,0)$, $(1,3,-1)$ and $(2,1,1)$.
}{$5\sqrt{2}/2$
}

\exercise{Vertices: $(5,2,-1)$, $(3,6,2)$ and $(1,0,4)$.
}{$3\sqrt{30}/2$
}

\exercise{Vertices: $(1,1)$, $(1,3)$ and $(2,2)$.
}{$1$
}

\exercise{Vertices: $(3,1)$, $(1,2)$ and $(4,3)$.
}{$5/2$
}
}

\begin{exerciseset}{In Exercises}{, find the area of the quadrilateral with the given vertices. (Hint: break the quadrilateral into 2 triangles.)}

\exercise{Vertices: $(0,0)$, $(1,2)$, $(3,0)$ and $(4,3)$.}{$7$}

\exercise{Vertices: $(0,0,0)$, $(2,1,1)$, $(-1,2,-8)$ and $(1,-1,5)$.}{$8\sqrt{7/2}$}

\end{exerciseset}


\begin{exerciseset}{In Exercises}{, find the volume of the parallelepiped defined by the given vectors.}

\exercise{$\vec u =\bracket{1,1,1}$,\quad $\vec v=\bracket{1,2,3}$, \quad $\vec w =\bracket{1,0,1}$}{$2$}

\exercise{$\vec u =\bracket{-1,2,1}$,\quad $\vec v=\bracket{2,2,1}$, \quad $\vec w =\bracket{3,1,3}$}{$15$}

\end{exerciseset}


\exerciseset{In Exercises}{, find a unit vector orthogonal to both $\vec u$ and $\vec v$.
}{

\exercise{$\vec u = \la 1,1,1\ra$,\quad $\vec v=\la 2,0,1\ra$
}{$\pm\frac{ 1}{\sqrt{6}}\la 1,1,-2\ra $
}

\exercise{$\vec u = \la 1,-2,1\ra$,\quad $\vec v=\la 3,2,1\ra$
}{$\pm\frac{ 1}{\sqrt{21}}\la -2,1,4\ra $
}

\exercise{$\vec u = \la 5,0,2\ra$,\quad $\vec v=\la -3,0,7\ra$
}{$\la 0,\pm 1,0 \ra $
}

\exercise{$\vec u = \la 1,-2,1\ra$,\quad $\vec v=\la -2,4,-2\ra$
}{any vector orthogonal to $\vec u$ works (such as $\frac{1}{\sqrt{2}}\la 1,0,-1\ra$).
}
}

\exercise{A bicycle rider applies 150lb of force, straight down, onto a pedal that extends 7in horizontally from the crankshaft. Find the magnitude of the torque applied to the crankshaft.}{$87.5$ft-lb}

\exercise{A bicycle rider applies 150lb of force, straight down, onto a pedal that extends 7in from the crankshaft, making a $30^\circ$ angle with the horizontal. Find the magnitude of the torque applied to the crankshaft.}{$43.75\sqrt{3}\approx 75.78$ft-lb}

\exercise{To turn a stubborn bolt, 80lb of force is applied to a 10in wrench. What is the maximum amount of torque that can be applied to the bolt?}{$200/3\approx 66.67$ft-lb}

\exercise{To turn a stubborn bolt, 80lb of force is applied to a 10in wrench in a confined space, where the direction of applied force makes a $10^\circ$ angle with the wrench. How much torque is subsequently applied to the wrench?}{$11.58$ft-lb}

\exercise{Show,\label{pr_cross_perp} using the definition of the Cross Product, that $\vec u\cdot(\vec u\times\vec v)=0$; that is, that $\vec u$ is orthogonal to the cross product of $\vec u$ and $\vec v$.}{With $\vec u =\bracket{u_1,u_2,u_3}$ and $\vec v =\bracket{v_1,v_2,v_3}$, we have
\begin{align*}
	\vec u\cdot(\vec u\times\vec v)
	&=\bracket{u_1,u_2,u_3}\cdot\\
	&\qquad(\bracket{u_2v_3-u_3v_2,-(u_1v_3-u_3v_1),u_1v_2-u_2v_1}) \\
	&= u_1(u_2v_3-u_3v_2) - u_2(u_1v_3-u_3v_1) \\
	&\qquad\qquad{}+u_3(u_1v_2-u_2v_1)\\
	&=0.
\end{align*}}

\exercise{Show,\label{cross_self} using the definition of the Cross Product, that $\vec u\times\vec u=\vec 0$.}{With $\vec u =\bracket{u_1,u_2,u_3}$, we have
\begin{align*}
	\vec u\times\vec u
	&=\bracket{u_2u_3-u_3u_2,-(u_1u_3-u_3u_1),u_1u_2-u_2u_1}) \\
	&=\bracket{0,0,0}\\
	&=\vec 0.
\end{align*}}

% todo the solutions to 11.4#49-60
\exercise{Suppose $\vec{a}$, $\vec{v}$, and $\vec{w}$ are vectors in $\mathbb{R}^3$ with $\vec{a} \neq \vec{0}$.  Show that if both $\vec{a} \cdot \vec{v} = \vec{a} \cdot \vec{w}$ and $\vec{a} \times \vec{v} = \vec{a} \times \vec{w}$ then $\vec{v} = \vec{w}$.}{}

\exercise{Show that $\begin{vmatrix}
a & b \\
c & d
\end{vmatrix} = 0$ if and only if $\bracket{a,b}$ and $\bracket{c,d}$ are parallel.}{}

\exercise{Show that $\vec{v} \times \vec{w} = \vec{0}$ if and only if $\vec{v}$ and $\vec{w}$ are parallel in $\mathbb{R}^3$.}{}

\exercise{Show that $\begin{vmatrix}
c & d \\
a & b
\end{vmatrix} = - \begin{vmatrix}
a & b \\
c & d
\end{vmatrix}$.}{}

\exercise{Show that $\vec{w} \times \vec{v} = -(\vec{v} \times \vec{w})$ for any $\vec{v}$ and $\vec{w}$ in $\mathbb{R}^3$.}{}

\exercise{Show that for any real numbers $s$ and $t$ we have $\begin{vmatrix}
sa & sb \\
tc & td
\end{vmatrix} = st \begin{vmatrix}
a & b \\
c & d
\end{vmatrix}$.}{}

\exercise{Show that $c (\vec{v} \times \vec{w}) = \vec{v} \times (c \vec{w})$ for any scalar $c$ and for any $\vec{v}$ and $\vec{w}$ in $\mathbb{R}^3$.}{}

\exercise{Show that 
$\begin{vmatrix}
a & b \\
c + s & d + t
\end{vmatrix} = \begin{vmatrix}
a & b \\
c & d
\end{vmatrix}
+ \begin{vmatrix}
a & b \\
s & t
\end{vmatrix}$.}{}

\exercise{Show that $\vec{u} \times (\vec{v} + \vec{w}) = \vec{u} \times \vec{v} + \vec{u} \times \vec{w}$ for any $\vec{u}$, $\vec{v}$, and $\vec{w}$ in $\mathbb{R}^3$.}{}

\exercise{Let $\vec{u} =\bracket{u_1, u_2, u_3 }$, $\vec{v} =\bracket{v_1, v_2, v_3 }$, and $\vec{w} =\bracket{w_1, w_2, w_3 }$.  Show that
\[
\vec{u} \cdot ( \vec{v} \times \vec{w} ) = \begin{vmatrix}
u_1 & u_2 & u_3 \\
v_1 & v_2 & v_3 \\
w_1 & w_2 & w_3
\end{vmatrix} 
\]}{}

\exercise{We have seen that if we swap two rows of a $2 \times 2$ determinant the determinant changes sign.  This is true for $3 \times 3$ determinants as well.  Using this fact show that $\vec{u} \cdot ( \vec{v} \times \vec{w} ) = ( \vec{u} \times \vec{v} ) \cdot \vec{w}$ for any vectors $\vec{u}$, $\vec{v}$ and $\vec{w}$ in $\mathbb{R}^3$.}{}

\exercise{We have seen that if two rows of a $2 \times 2$ determinant are the same the determinant is zero.  This is true for $3 \times 3$ determinants as well.  Using this fact show that $\vec{v} \times \vec{w}$ is orthogonal to both $\vec{v}$ and $\vec{w}$ for any vectors $\vec{v}$ and $\vec{w}$ in $\mathbb{R}^3$.}{}
