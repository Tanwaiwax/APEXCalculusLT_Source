\printconcepts

\exercise{In order to find the equation of a plane, what two pieces of information must one have?}{A point in the plane and a normal vector (i.e., a direction orthogonal to the plane).}

\exercise{What is the relationship between a plane and one of its normal vectors?}{A normal vector is orthogonal to the plane.}

\printproblems

\exerciseset{In Exercises}{, give any two points in the given plane.
}{

\exercise{$2x-4y+7z=2$
}{Answers will vary.
}

\exercise{$3(x+2)+5(y-9)-4z=0$
}{Answers will vary.
}

\exercise{$x=2$
}{Answers will vary.
}

\exercise{$4(y+2)-(z-6)=0$
}{Answers will vary.
}
}

\exerciseset{In Exercises}{, give the equation of the described plane in standard and general forms.}{

\exercise{Passes through $(2,3,4)$ and has normal vector\\ $\vec n=\bracket{3,-1,7}$.}{Standard form: $3(x-2)-(y-3)+7(z-4)=0$\\
general form: $3x-y+7z=31$}

\exercise{Passes through $(1,3,5)$ and has normal vector\\ $\vec n=\bracket{0,2,4}$.}{Standard form: $2(y-3)+4(z-5)=0$\\
general form: $2y+4z=26$}

\exercise{Passes through the points $(1,2,3)$, $(3,-1,4)$ and $(1,0,1)$.}{Answers may vary;\\
Standard form: $8(x-1)+4(y-2)-4(z-3)=0$\\
general form: $8x+4y-4z=4$}

\exercise{Passes through the points $(5,3,8)$, $(6,4,9)$ and $(3,3,3)$.}{Answers may vary;\\
Standard form: $-5(x-5)+3(y-3)+2(z-8)=0$\\
general form: $-5x+3y+2z=0$}

\exercise{Contains the intersecting lines\\
$\vec\ell_1(t) =\bracket{2,1,2}+ t\bracket{1,2,3}$ and \\
$\vec\ell_2(t) =\bracket{2,1,2}+ t\bracket{2,5,4}$.}{Answers may vary;\\
Standard form: $-7(x-2)+2(y-1)+(z-2)=0$\\
general form: $-7x+2y+z=-10$}

\exercise{Contains the intersecting lines\\
$\vec\ell_1(t) =\bracket{5,0,3}+ t\bracket{-1,1,1}$ and \\
$\vec\ell_2(t) =\bracket{1,4,7}+ t\bracket{3,0,-3}$.}{Answers may vary;\\
Standard form: $3(x-5)+3(z-3)=0$\\
general form: $3x+3z=24$}

\exercise{Contains the parallel lines\\
$\vec\ell_1(t) =\bracket{1,1,1}+ t\bracket{1,2,3}$ and \\
$\vec\ell_2(t) =\bracket{1,1,2}+ t\bracket{1,2,3}$.}{Answers may vary;\\
Standard form: $2(x-1)-(y-1)=0$\\
general form: $2x-y=1$}

\exercise{Contains the parallel lines\\
$\vec\ell_1(t) =\bracket{1,1,1}+ t\bracket{4,1,3}$ and \\
$\vec\ell_2(t) =\bracket{2,2,2}+ t\bracket{4,1,3}$.}{Answers may vary;\\
Standard form: $2(x-1)+(y-1)-3(z-1)=0$\\
general form: $2x+y-3z=0$}

\exercise{Contains the point $(2,-6,1)$ and the line\\
$\ell(t) = \begin{cases}x=2+5t\\y=2+2t\\z=-1+2t\end{cases}$}{Answers may vary;\\
Standard form: $2(x-2)-(y+6)-4(z-1)=0$\\
general form: $2x-y-4z=6$}

\exercise{Contains the point $(5,7,3)$ and the line\\
$\ell(t) = \begin{cases}x=t\\y=t\\z=t\end{cases}$}{Answers may vary;\\
Standard form: $4(x-5)-2(y-7)-2(z-3)=0$\\
general form: $4x-2y-2z=0$}

\exercise{Contains the point $(5,7,3)$ and is orthogonal to the line\\
$\vec\ell(t) =\bracket{4,5,6}+ t\bracket{1,1,1}$.}{Answers may vary;\\
Standard form: $(x-5)+(y-7)+(z-3)=0$\\
general form: $x+y+z=15$}

\exercise{Contains the point $(4,1,1)$ and is orthogonal to the line\\
$\ell(t) = \begin{cases}x=4+4t\\y=1+1t\\z=1+1t\end{cases}$}{Answers may vary;\\
Standard form: $4(x-4)+(y-1)+(z-1)=0$\\
general form: $4x+y+z=18$}

\exercise{Contains the point $(-4,7,2)$ and is parallel to the plane\\
$ 3(x-2)+8(y+1) -10z=0$.}{Answers may vary;\\
Standard form: $3(x+4)+8(y-7)-10(z-2)=0$\\
general form: $3x+8y-10z=24$}

\exercise{Contains the point $(1,2,3)$ and is parallel to the plane\\
$x=5$.}{Standard form: $x-1=0$\\
general form: $x=1$}

}


\begin{exerciseset}{In Exercises}{, give the equation of the line that is the intersection of the given planes.}

\exercise{$p1:\ 3 (x - 2) + (y - 1) + 4 z=0$, and \\
$p2:\ 2 (x - 1) - 2 (y + 3) + 6 (z - 1)=0$.}{Answers may vary:\\
$\ell = \left\{\begin{aligned} x &= 14t\\
y &= -1-10t\\
z&= 2-8t\end{aligned} \right.$}

\exercise{$p1:\ 5 (x - 5) + 2 (y + 2) + 4 (z - 1)=0$, and \\
$p2:\ 3 x - 4 (y - 1) + 2 (z - 1)=0$.}{Answers may vary:\\
$\ell = \left\{\begin{aligned} x &=1+20t\\
y &= 3+2t\\
z&= 3.5-26t\end{aligned} \right.$}

\end{exerciseset}


\exerciseset{In Exercises}{, find the point of intersection between the line and the plane.
}{

\exercise{line: $\la 5,1,-1\ra + t\la 2,2,1\ra$,\\
plane: $5x-y-z=-3$
}{$(-3,-7,-5)$
}

\exercise{line: $\la 4,1,0\ra + t\la 1,0,-1\ra$,\\
plane: $3x+y-2z=8$
}{$(3,1,1)$
}

\exercise{line: $\la 1,2,3\ra + t\la 3,5,-1\ra$,\\
plane: $3x-2y-z=4$
}{No point of intersection; the plane and line are parallel.
}

\exercise{line: $\la 1,2,3\ra + t\la 3,5,-1\ra$,\\
plane: $3x-2y-z=-4$
}{The plane contains the line, so every point on the line is a ``point of intersection.''
}
}

\exerciseset{In Exercises}{, find the given distances.
}{

\exercise{The distance from the point $(1,2,3)$ to the plane\\
$3(x-1)+(y-2)+5(z-2)=0$.
}{$\sqrt{5/7}$
}

\exercise{The distance from the point $(2,6,2)$ to the plane\\
$2(x-1)-y+4(z+1)=0$.
}{$8/\sqrt{21}$
}

\exercise{The distance between the parallel planes\\
$x+y+z=0$ and \\
$(x-2)+(y-3)+(z+4)=0$

}{$1/\sqrt{3}$
}

\exercise{The distance between the parallel planes\\
$2(x-1)+2(y+1)+(z-2)=0$ and \\
$2(x-3)+2(y-1)+(z-3)=0$

}{$3$
}
}

\exercise{Show why if the point $Q$ lies in a plane, then the distance formula correctly gives the distance from the point to the plane as $0$.}{If $P$ is any point in the plane, and $Q$ is also in the plane, then $\vv{PQ}$ lies parallel to the plane and is orthogonal to $\vec n$, the normal vector. Thus $\vec n\cdot \vv{PQ}=0$, giving the distance as 0.}

\exercise{How is Exercise \ref{int_lines_dist} in \autoref{sec:lines} easier to answer once we have an understanding of planes?}{The intersecting lines define a plane with normal vector $\vec n = \vec c = \vec d_1\times \vec d_2$. Since points $P_1$ and $P_2$ lie in the plane, $\vec c$ is orthogonal to $\vv{P_1P_2}$, hence $\vv{P_1P_2}\cdot\vec c = 0$, giving a distance of 0. Knowing the principles of planes, especially their normal vectors, makes this simpler.}
