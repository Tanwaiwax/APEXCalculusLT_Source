\printproblems

\exerciseset{For Exercises }{1-4, find the (a) cylindrical and (b) spherical coordinates of the point whose Cartesian coordinates are given.}{

\exercise{$(2,2\sqrt{3},-1)$}{(a) $(4,\frac{\pi}{3},-1)$; (b) $(\sqrt{17},\frac{\pi}{3},1.816)$}

\exercise{$(-5,5,6)$}{}

\exercise{$(\sqrt{21},-\sqrt{7},0)$}{(a) $(2\sqrt{7},\frac{11\pi}{6},0)$; (b) $(2\sqrt{7},\frac{11\pi}{6},\frac{\pi}{2})$}

\exercise{$(0,\sqrt{2},2)$}{}

}

\exerciseset{For Exercises }{5-7, write the given equation in (a) cylindrical and (b) spherical coordinates.}{

\exercise{$x^2 + y^2 + z^2 = 25$}{(a) $r^2 + z^2 = 25$; (b) $\rho = 5$}

\exercise{$x^2 + y^2 = 2y$}{}

\exercise{$x^2 + y^2 + 9z^2 = 36$}{(a) $r^2 + 9z^2 = 36$; (b) $\rho^2 ( 1 + 8 \cos^2 \phi ) = 36$}

}

\exercise{Describe the intersection of the surfaces whose equations in spherical coordinates are $\theta = \frac{\pi}{2}$ and $\phi = \frac{\pi}{4}$.}{}

\exercise{Show that for $a \ne 0$, the equation $\rho = 2a \sin \phi \, \cos \theta$ in spherical coordinates describes a sphere centered at $(a,0,0)$ with radius $\abs{a}$.}{}

\exercise{Let $P = (a,\theta,\phi)$ be a point in spherical coordinates, with $a > 0$ and $0 < \phi < \pi$. Then $P$ lies on the sphere $\rho = a$. Since $0 < \phi < \pi$, the line segment from the origin to $P$ can be extended to intersect the cylinder given by $r = a$ (in cylindrical coordinates). Find the cylindrical coordinates of that point of intersection.}{$(a,\theta,a \cot \phi )$}

\exercise{Let $\ssub{P}{1}$ and $\ssub{P}{2}$ be points whose spherical coordinates are $( \ssub{\rho}{1},\ssub{\theta}{1},\ssub{\phi}{1} )$ and $( \ssub{\rho}{2},\ssub{\theta}{2},\ssub{\phi}{2} )$, respectively. Let $\ssub{\textbf{v}}{1}$ be the vector from the origin to $\ssub{P}{1}$, and let $\ssub{\textbf{v}}{2}$ be the vector from the origin to $\ssub{P}{2}$. For the angle $\gamma$ between $\ssub{\textbf{v}}{1}$ and $\ssub{\textbf{v}}{2}$, show that
  \[
   \cos \gamma = \cos \ssub{\phi}{1} \, \cos \ssub{\phi}{2} + \sin \ssub{\phi}{1} \, \sin \ssub{\phi}{2} \,
    \cos ( \, \ssub{\theta}{2} - \ssub{\theta}{1} \, ) .
  \]
  This formula is used in electrodynamics to prove the addition theorem for spherical harmonics, which provides a general expression for the electrostatic potential at a point due to a unit charge. See pp.\ 100-102 in \cite{jac}.}{}

\exercise{Show that the distance $d$ between the points $\ssub{P}{1}$ and $\ssub{P}{2}$ with cylindrical coordinates $( \ssub{r}{1},\ssub{\theta}{1},\ssub{z}{1} )$ and $( \ssub{r}{2},\ssub{\theta}{2},\ssub{z}{2} )$, respectively, is
  \[
   d = \sqrt{\ssub{r}{1}^2 + \ssub{r}{2}^2 - 2 \ssub{r}{1}\,\ssub{r}{2} \cos (\, \ssub{\theta}{2} - \ssub{\theta}{1} \,)
   + ( \ssub{z}{2} - \ssub{z}{1} )^2} \, .
  \]}{Hint: Use the distance formula for Cartesian coordinates.}

\exercise{Show that the distance $d$ between the points $\ssub{P}{1}$ and $\ssub{P}{2}$ with spherical coordinates $( \ssub{\rho}{1},\ssub{\theta}{1},\ssub{\phi}{1} )$ and $( \ssub{\rho}{2},\ssub{\theta}{2},\ssub{\phi}{2} )$, respectively, is
  \[
   d = \sqrt{\ssub{\rho}{1}^2 + \ssub{\rho}{2}^2 - 2 \ssub{\rho}{1}\,\ssub{\rho}{2} [ \sin \ssub{\phi}{1} \,
    \sin \ssub{\phi}{2} \,\cos ( \, \ssub{\theta}{2} - \ssub{\theta}{1} \, ) +
    \cos \ssub{\phi}{1} \, \cos \ssub{\phi}{2} ]} \, .
  \]}{}
