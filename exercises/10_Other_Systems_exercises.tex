\printproblems

\exerciseset{In Exercises}{, find the (a) cylindrical and (b) spherical coordinates of the point whose Cartesian coordinates are given.}{

\exercise{$(2,2\sqrt{3},-1)$}{(a) $(4,\frac{\pi}{3},-1)$; (b) $(\sqrt{17},\frac{\pi}{3},1.816)$}

\exercise{$(-5,5,6)$}{}

\exercise{$(\sqrt{21},-\sqrt{7},0)$}{(a) $(2\sqrt{7},\frac{11\pi}{6},0)$; (b) $(2\sqrt{7},\frac{11\pi}{6},\frac{\pi}{2})$}

\exercise{$(0,\sqrt{2},2)$}{}

}

\exerciseset{In Exercises}{, write the given equation in (a) cylindrical and (b) spherical coordinates.}{

\exercise{$x^2 + y^2 + z^2 = 25$}{(a) $r^2 + z^2 = 25$; (b) $\rho = 5$}

\exercise{$x^2 + y^2 = 2y$}{}

\exercise{$x^2 + y^2 + 9z^2 = 36$}{(a) $r^2 + 9z^2 = 36$; (b) $\rho^2 ( 1 + 8 \cos^2 \phi ) = 36$}

}

\exercise{Describe the intersection of the surfaces whose equations in spherical coordinates are $\theta = \frac{\pi}{2}$ and $\phi = \frac{\pi}{4}$.}{}

\exercise{Show that for $a \ne 0$, the equation $\rho = 2a \sin \phi \, \cos \theta$ in spherical coordinates describes a sphere centered at $(a,0,0)$ with radius $\abs{a}$.}{}

\exercise{Let $P = (a,\theta,\phi)$ be a point in spherical coordinates, with $a > 0$ and $0 < \phi < \pi$. Then $P$ lies on the sphere $\rho = a$. Since $0 < \phi < \pi$, the line segment from the origin to $P$ can be extended to intersect the cylinder given by $r = a$ (in cylindrical coordinates). Find the cylindrical coordinates of that point of intersection.}{$(a,\theta,a \cot \phi )$}

\exercise{Let $P_1$ and $P_2$ be points whose spherical coordinates are $( \rho_1,\theta_1,\phi_1 )$ and $( \rho_2,\theta_2,\phi_2 )$, respectively. Let $\textbf{v}_1$ be the vector from the origin to $P_1$, and let $\textbf{v}_2$ be the vector from the origin to $P_2$. For the angle $\gamma$ between $\textbf{v}_1$ and $\textbf{v}_2$, show that
  \[
   \cos \gamma = \cos \phi_1 \, \cos \phi_2 + \sin \phi_1 \, \sin \phi_2 \,
    \cos ( \, \theta_2 - \theta_1 \, ) .
  \]
  This formula is used in electrodynamics to prove the addition theorem for spherical harmonics, which provides a general expression for the electrostatic potential at a point due to a unit charge.% See pp.\ 100-102 in \cite{jac}.
}{}

\exercise{Show that the distance $d$ between the points $P_1$ and $P_2$ with cylindrical coordinates $( r_1,\theta_1,z_1 )$ and $( r_2,\theta_2,z_2 )$, respectively, is
  \[
   d = \sqrt{r_1^2 + r_2^2 - 2 r_1\,r_2 \cos (\, \theta_2 - \theta_1 \,)
   + ( z_2 - z_1 )^2} \, .
  \]}{Hint: Use the distance formula for Cartesian coordinates.}

\exercise{Show that the distance $d$ between the points $P_1$ and $P_2$ with spherical coordinates $( \rho_1,\theta_1,\phi_1 )$ and $( \rho_2,\theta_2,\phi_2 )$, respectively, is
  \[
   d = \sqrt{\rho_1^2 + \rho_2^2 - 2 \rho_1\,\rho_2 [ \sin \phi_1 \,
    \sin \phi_2 \,\cos ( \, \theta_2 - \theta_1 \, ) +
    \cos \phi_1 \, \cos \phi_2 ]} \, .
  \]}{}
