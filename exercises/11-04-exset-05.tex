\exercisesetinstructions{, find $a_{\text{T}}$ and $a_{\text{N}}$ given $\vrt$. Sketch \vrt\ on the indicated interval, and comment on the relative sizes of $a_{\text{T}}$ and $a_{\text{N}}$ at the indicated $t$ values.}

\exercise{$\vrt =\bracket{t,t^2}$ on $[-1,1]$; consider $t=0$ and $t=1$.}{$a_{\text{T}} = \frac{4t}{\sqrt{1+4t^2}}$ and $a_{\text{N}} = \sqrt{4-\frac{16t^2}{1+4t^2}}$\\
At $t=0$, $a_{\text{T}} = 0$ and $a_{\text{N}} = 2$;\\
At $t=1$, $a_{\text{T}} = 4/\sqrt{5}$ and $a_{\text{N}} = 2/\sqrt{5}$.\\
At $t=0$, all acceleration comes in the form of changing the direction of velocity and not the speed; at $t=1$, more acceleration comes in changing the speed than in changing direction.}

\exercise{$\vrt =\bracket{t,1/t}$ on $(0,4]$; consider $t=1$ and $t=2$.}{$a_{\text{T}} = \frac{-2/t^5}{\sqrt{1+1/t^4}}$ and $a_{\text{N}} = \sqrt{\frac{4}{t^6}-\frac{4/t^{10}}{1+1/t^4}}$\\
At $t=1$, $a_{\text{T}} = \sqrt{2}$ and $a_{\text{N}} = -\sqrt{2}$;\\
At $t=2$, $a_{\text{T}} = -\frac{1}{4\sqrt{17}}$ and $a_{\text{N}} = \frac{1}{\sqrt{17}}$.\\
At $t=1$, acceleration comes from changing speed and changing direction in ``equal measure;'' at $t=2$, acceleration is nearly $\vec 0$ as it is; the low value of $a_{\text{T}}$ shows that the speed is nearly constant and the low value of $a_{\text{N}}$ shows the direction is not changing quickly.}

\exercise{$\vrt =\bracket{2\cos t,2\sin t}$ on $[0,2\pi]$; consider $t=0$ and $t=\pi/2$.}{$a_{\text{T}} = 0$ and $a_{\text{N}} = 2$\\
At $t=0$, $a_{\text{T}} = 0$ and $a_{\text{N}} = 2$;\\
At $t=\pi/2$, $a_{\text{T}} = 0$ and $a_{\text{N}} = 2$.\\
The object moves at constant speed, so all acceleration comes from changing direction, hence $a_{\text{T}}=0$. $\vat$ is always parallel to $\vec N(t)$, but twice as long, hence $a_{\text{N}}=2$.}

\exercise{$\vrt =\bracket{\cos (t^2),\sin (t^2)}$ on $(0,2\pi]$; consider $t=\sqrt{\pi/2}$ and $t=\sqrt{\pi}$.}{$a_{\text{T}} = 2$ and $a_{\text{N}} = 4t^2$\\
At $t=\sqrt{\pi/2}$, $a_{\text{T}} = 2$ and $a_{\text{N}} = 2\pi$;\\
At $t=\sqrt{\pi}$, $a_{\text{T}} = 2$ and $a_{\text{N}} = 4\pi$.\\
The object moves at increasing speed (increasing at a constant rate of acceleration), hence $a_{\text{T}}=2$. Since the object is increasing speed yet always traveling in a circle of radius 1, the direction must change more quickly; the amount of acceleration that changes direction increases over time. }

\exercise{$\vrt =\bracket{a\cos t,a\sin t, bt}$ on $[0,2\pi]$, where $a,b>0$; consider $t=0$ and $t=\pi/2$.}{$a_{\text{T}} = 0$ and $a_{\text{N}} = a$\\
At $t=0$, $a_{\text{T}} = 0$ and $a_{\text{N}} = a$;\\
At $t=\pi/2$, $a_{\text{T}} = 0$ and $a_{\text{N}} = a$.\\
The object moves at constant speed, meaning that $a_{\text{T}}$ is always 0. The object ``rises'' along the $z$-axis at a constant rate, so all acceleration comes in the form of changing direction circling the $z$-axis. The greater the radius of this circle the greater the acceleration, hence $a_{\text{N}}=a$.}

\exercise{$\vrt =\bracket{5\cos t,4\sin t, 3\sin t}$ on $[0,2\pi]$; consider $t=0$ and $t=\pi/2$.}{$a_{\text{T}} = 0$ and $a_{\text{N}} = 5$\\
At $t=0$, $a_{\text{T}} = 0$ and $a_{\text{N}} = 5$;\\
At $t=\pi/2$, $a_{\text{T}} = 0$ and $a_{\text{N}} = 5$.\\
The object moves at constant speed, meaning that $a_{\text{T}}$ is always 0. Acceleration is thus always perpendicular to the direction of travel; in this particular case, it is always 5 times the unit vector pointing orthogonal to the direction of travel.}

\exercisesetend
