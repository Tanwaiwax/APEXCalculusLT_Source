\exercisesetinstructions{, a curve $C$ is described along with 2 points on $C$. 
\begin{enumerate}
\item Using a sketch, determine at which of these points the curvature is greater. 
\item Find the curvature $\kappa$ of $C$, and evaluate $\kappa$ at each of the 2 given points.
\end{enumerate}}

\exercise{$C$ is defined by $y = x^3-x$; points given at $x=0$ and $x=1/2$. }{$\kappa = \frac{\abs{6x}}{\left(1+(3x^2-1)^2\right)^{3/2}}$;

$\kappa(0) = 0$, $\kappa(1/2) = \frac{192}{17\sqrt{17}} \approx 2.74$.}

\exercise{$C$ is defined by $\ds y = \frac1{x^2+1}$; points given at $x=0$ and $x=2$. }{$\kappa = \frac{\abs{\frac{6x^2-2}{(x^2+1)^3}}}{\left(1+\frac{4x^2}{(x^2+1)^4}\right)^{3/2}}$;

$\kappa(0) = 2$, $\kappa(2) = \frac{2750}{641\sqrt{641}} \approx 0.169$.}

\exercise{$C$ is defined by $\ds y = \cos x$; points given at $x=0$ and $x=\pi/2$. }{$\kappa = \frac{\abs{\cos x}}{\left(1+\sin^2x\right)^{3/2}}$;

$\kappa(0) = 1$, $\kappa(\pi/2) = 0$}

\exercise{$C$ is defined by $\ds y = \sqrt{1-x^2}$ on $(-1,1)$; points given at $x=0$ and $x=1/2$. }{$\kappa = 1$;

$\kappa(0) = 1$, $\kappa(1/2) = 1$}

\exercise{$C$ is defined by $\ds \vrt =\bracket{\cos t, \sin (2t)}$; points given at $t=0$ and $t=\pi/4$. }{$\kappa = \frac{\abs{2\cos t\cos(2t)+4\sin t\sin(2t)}}{\left(4\cos^2(2t)+\sin^2t\right)^{3/2}}$;

$\kappa(0) = 1/4$, $\kappa(\pi/4) = 8$}

\exercise{$C$ is defined by $\ds \vrt =\bracket{\cos^2 t, \sin t\cos t}$; points given at $t=0$ and $t=\pi/3$. }{$\kappa = 2$;

$\kappa(0) = 2$, $\kappa(\pi/3) = 2$}

\exercise{$C$ is defined by $\ds \vrt =\bracket{t^2-1,t^3-t}$; points given at $t=0$ and $t=5$. }{$\kappa = \frac{\abs{6t^2+2}}{\left(4t^2+(3t^2-1)^2\right)^{3/2}}$;

$\kappa(0) = 2$, $\kappa(5) = \frac{19}{1394\sqrt{1394}}\approx 0.0004$}

\exercise{$C$ is defined by $\ds \vrt =\bracket{\tan t,\sec t}$; points given at $t=0$ and $t=\pi/6$. }{$\kappa = \frac{\abs{\sec^3t}}{\left(\sec^4t+\sec^2t\tan^2t\right)^{3/2}}$;

$\kappa(0) = 1$, $\kappa(\pi/6) = \frac{3\sqrt{3}}{5\sqrt{5}}\approx 0.465$}

\exercise{$C$ is defined by $\ds \vrt =\bracket{4t+2,3t-1,2t+5}$; points given at $t=0$ and $t=1$. }{$\kappa = 0$;

$\kappa(0) = 0$, $\kappa(1) = 0$}

\exercise{$C$ is defined by $\ds \vrt =\bracket{t^3-t,t^3-4,t^2-1}$; points given at $t=0$ and $t=1$. }{$\kappa = \frac{2 \sqrt{18 t^4+15t^2+1}}{\left(18 t^4-2t^2+1\right)^{3/2}}$;

$\kappa(0) = 2$, $\kappa(1) = 2\sqrt{2}/17\approx 0.166378$}

\exercise{$C$ is defined by $\ds \vrt =\bracket{3\cos t,3\sin t, 2t}$; points given at $t=0$ and $t=\pi/2$. }{$\kappa = \frac{3}{13}$;

$\kappa(0) = 3/13$, $\kappa(\pi/2) = 3/13$}

\exercise{$C$ is defined by $\ds \vrt =\bracket{5\cos t,13\sin t, 12\cos t}$; points given at $t=0$ and $t=\pi/2$. }{$\kappa = \frac{1}{13}$;

$\kappa(0) = 1/13$, $\kappa(\pi/2) = 1/13$}

\exercisesetend
