\exerciseset{In Exercises}{, create a vector--valued function whose graph matches the given description.
}{

\exercise{A circle of radius 2, centered at $(1,2)$, traced counter--clockwise once on $[0,2\pi]$.
}{
Answers may vary, though most direct solution is\\
$\vec r(t) = \la 2\cos t+1,2\sin t+2\ra$. 
}

\exercise{A circle of radius 3, centered at $(5,5)$, traced clockwise once on $[0,2\pi]$.
}{
Answers may vary; three solutions are\\
$\vec r(t) = \la 3\sin t+5,3\cos t+5\ra$, \\
$\vec r(t) = \la -3\cos t+5,3\sin t+5\ra$ and \\
$\vec r(t) = \la 3\cos t+5,-3\sin t+5\ra$. 
}

\exercise{An ellipse, centered at $(0,0)$ with vertical major axis of length 10 and minor axis of length 3, traced once counter--clockwise on $[0,2\pi]$. 
}{
Answers may vary, though most direct solution is\\
$\vec r(t) = \la 1.5\cos t,5\sin t\ra$.

}

\exercise{An ellipse, centered at $(3,-2)$ with horizontal major axis of length 6 and minor axis of length 4, traced once clockwise on $[0,2\pi]$. 
}{
Answers may vary, though most direct solutions are\\
$\vec r(t) = \la -3\cos t+3,2\sin t-2\ra$,\\
$\vec r(t) = \la 3\cos t+3,-2\sin t-2\ra$ and \\
$\vec r(t) = \la 3\sin t+3,2\cos t-2\ra$.

}

\exercise{A line through $(2,3)$ with a slope of 5. 
}{
Answers may vary, though most direct solutions are \\
$\vec r(t) = \la t,5(t-2)+3\ra $ and \\
$\vec r(t) = \la t+2,5t+3\ra$.

}

\exercise{A line through $(1,5)$ with a slope of $-1/2$. 
}{
Answers may vary, though most direct solutions are \\
$\vec r(t) = \la t,-1/2(t-1)+5\ra $,\\
$\vec r(t) = \la t+1,-1/2t+5\ra $, \\
$\vec r(t) = \la -2t+1,t+5\ra$ and \\
$\vec r(t) = \la 2t+1,-t+5\ra$.

}

\exercise{A vertically oriented helix with radius of 2 that starts at $(2,0,0)$ and ends at $(2,0,4\pi)$ after 1 revolution on $[0,2\pi]$. 
}{
Answers may vary, though most direct solution is\\
$\vec r(t) = \la 2\cos t,2\sin t,2t\ra $.

}

\exercise{A vertically oriented helix with radius of 3 that starts at $(3,0,0)$ and ends at $(3,0,3)$ after 2 revolutions on $[0,1]$. 
}{
Answers may vary, though most direct solution is\\
$\vec r(t) = \la 3\cos (4\pi t),3\sin (4\pi t),3t\ra $.

}
}