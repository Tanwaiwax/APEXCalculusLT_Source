\printconcepts

\exercise{Limits, derivatives and integrals of vector-valued functions are all evaluated \underline{\hskip .5in}-wise.}{component}

\exercise{The definite integral of a rate of change function gives \underline{\hskip .5in}.}{displacement}

\exercise{Why is it generally not useful to graph both $\vec r(t)$ and $\vrp(t)$ on the same axes?}{It is difficult to identify the points on the graphs of $\vec r(t)$ and $\vrp(t)$ that correspond to each other.}

\exercise{\autoref{thm:vvf_deriv_prop} contains three product rules. What are the three different types of products used in these rules?}{A scalar-vector product, a dot product and a cross product.}

\printproblems

\exerciseset{In Exercises}{, evaluate the given limit.
}{

\exercise{$\ds \lim_{t\to 5} \la 2t+1,3t^2-1,\sin t\ra$
}{
$\la 11,74,\sin 5\ra$
}

\exercise{$\ds \lim_{t\to 3} \la e^t,\frac{t^2-9}{t+3}\ra$
}{
$\la e^3,0\ra$
}

\exercise{$\ds \lim_{t\to 0} \la \frac{t}{\sin t}, (1+t)^{\frac1t}\ra$
}{
$\la 1,e\ra$
}

\exercise{$\ds \lim_{h\to 0} \frac{\vec r(t+h)-\vec r(t)}{h}$, where 
$\vec r(t) = \la t^2,t,1\ra$.
}{
$\la2t,1,0\ra$
}
}

\exerciseset{In Exercises}{, identify the interval(s) on which $\vec r(t)$ is continuous.}{

\exercise{$\vec r(t) = \la t^2,1/t\ra$}{$(-\infty,0) \bigcup (0,\infty)$}

\exercise{$\vec r(t) = \la \cos t, e^t, \ln t\ra$}{$(0,\infty)$}

}


\exerciseset{In Exercises}{, find the derivative of the given function. 
}{

\exercise{$\vec r(t) = \la \cos t, e^t, \ln t\ra$
}{
$\vrp(t) = \la -\sin t, e^t, 1/t\ra$
}

\exercise{$\ds \vec r(t) = \la \frac 1t, \frac {2t-1}{3t+1}, \tan t\ra$
}{
$\vrp(t) = \la -1/t^2, 5/(3t+1)^2, \sec^2 t\ra$
}

\exercise{$\ds \vec r(t) = (t^2)\la \sin t, 2t+5\ra$
}{
$\vrp(t) = (2t)\la \sin t,2t+5\ra + (t^2) \la \cos t, 2\ra = \la2t\sin t + t^2\cos t, 6t^2+10t\ra$
}

\exercise{$\ds \vec r(t) = \la t^2+1, t-1\ra\cdot \la \sin t, 2t+5\ra$
}{
$\vrp(t) = \la 2t,1\ra \cdot \la \sin t,2t+5\ra + \la t^2+1,t-1\ra\cdot\la \cos t, 2\ra =$\\
 $(t^2+1)\cos t+2t\sin t + 4t+3$
}

\exercise{$\ds \vec r(t) = \la t^2+1, t-1,1\ra\times \la \sin t, 2t+5,1\ra$
}{
$\vrp(t) = \la 2t,1,0\ra \times \la \sin t,2t+5,1\ra + \la t^2+1,t-1,1\ra\times\la \cos t, 2,0\ra =$\\
$\la -1, \cos t - 2t, 6t^2+10t+2+\cos t - \sin t - t\cos t \ra$
}
}

\exerciseset{In Exercises}{, find $\vrp (t)$. Sketch $\vec r(t)$ and $\vrp(1)$, with the initial point of $\vrp(1)$ at $\vec r(1)$.  
}{

\exercise{$\ds \vec r(t) = \la t^2+t, t^2-t\ra$
}{
\begin{minipage}{\linewidth}
\myincludegraphics{figures/fig11_02_ex_15}\\
$\vrp(t) = \la 2t+1,2t-1\ra$
\end{minipage}
}

\exercise{$\ds \vec r(t) = \la t^2-2t+2,t^3-3t^2+2t\ra$
}{
\begin{minipage}{\linewidth}
\myincludegraphics{figures/fig11_02_ex_16}\\
$\vrp(t) = \la 2t-2,3t^2-6t+2\ra$
\end{minipage}
}

\exercise{$\ds \vec r(t) = \la t^2+1,t^3-t\ra$
}{
\begin{minipage}{\linewidth}
\myincludegraphics{figures/fig11_02_ex_17}\\
$\vrp(t) = \la 2t,3t^2-1\ra$
\end{minipage}
}

\exercise{$\ds \vec r(t) = \la t^2-4t+5,t^3-6t^2+11t-6\ra$
}{
\begin{minipage}{\linewidth}
\myincludegraphics{figures/fig11_02_ex_18}\\
$\vrp(t) = \la 2t-4,3t^2-12t+11\ra$
\end{minipage}
}
}

\exerciseset{In Exercises}{, give the equation of the line tangent to the graph of $\vec r(t)$ at the given $t$ value.  
}{

\exercise{$\ds \vec r(t) = \la t^2+t, t^2-t\ra$ at $t=1$.
}{
$\ell(t) = \la 2,0\ra + t\la 3,1\ra$
}

\exercise{$\ds \vec r(t) = \la 3\cos t,\sin t\ra$ at $t=\pi/4$.
}{
$\ell(t) = \la 3\sqrt{2}/2,\sqrt{2}/2\ra + t\la-3\sqrt{2}/2,\sqrt{2}/2\ra$
}

\exercise{$\ds \vec r(t) = \la 3\cos t,3\sin t,t\ra$ at $t=\pi$.
}{
$\ell(t) = \la -3,0,\pi\ra + t\la0,-3,1\ra$
}

\exercise{$\ds \vec r(t) = \la e^t,\tan t,t\ra$ at $t=0$.
}{
$\ell(t) = \la 1,0,0 \ra + t\la 1,1,1\ra$
}
}

\exerciseset{In Exercises}{, find the value(s) of $t$ for which $\vec r(t)$ is not smooth.  
}{

\exercise{$\ds \vec r(t) = \la \cos t,\sin t - t\ra$
}{
$t=2n\pi$, where $n$ is an integer; so $t = \ldots-4\pi,-2\pi,0,2\pi,4\pi,\ldots$
}

\exercise{$\ds \vec r(t) = \la t^2-2t+1,t^3+t^2-5t+3\ra$
}{
$t=1$
}

\exercise{$\ds \vec r(t) = \la \cos t-\sin t, \sin t - \cos t,\cos(4t)\ra$
}{
$\vec r(t)$ is not smooth at $t=3\pi/4+n\pi$, where $n$ is an integer
}

\exercise{$\ds \vec r(t) = \la t^3-3t+2, -\cos(\pi t),\sin^2(\pi t) \ra$
}{
$t=\pm 1$
}
}

\exerciseset{Exercises}{ ask you to verify parts of \autoref{thm:vvf_deriv_prop}. In each let $f(t) = t^3$, $\vec r(t) =\la t^2,t-1,1\ra$ and $\vec s(t) = \la \sin t, e^t,t\ra$. Compute the various derivatives as indicated.
}{

\exercise{Simplify $f(t)\vec r(t)$, then find its derivative; show this is the same as $\fp(t)\vec r(t) + f(t)\vrp(t)$.
}{
Both derivatives return $\la 5t^4,4t^3-3t^2,3t^2\ra$.
}

\exercise{Simplify $\vec r(t)\cdot\vec s(t)$, then find its derivative; show this is the same as $\vrp(t)\cdot\vec s(t) + \vec r(t)\cdot\vec s\,'(t)$.
}{
Both derivatives return $2\sin t+t^2\cos t + te^t+1$.
}

\exercise{Simplify $\vec r(t)\times\vec s(t)$, then find its derivative; show this is the same as $\vrp(t)\times\vec s(t) + \vec r(t)\times\vec s\,'(t)$.
}{
Both derivatives return $\la 2t-e^t-1,\cos t-3t^2,(t^2+2t)e^t-(t-1)\cos t-\sin t\ra$.
}
}

\begin{exerciseset}{In Exercises}{, evaluate the given definite or indefinite integral.}

\exercise{$\ds \int\bracket{t^3,\cos t, te^t}\ dt$}{$\bracket{\frac14t^4,\sin t,te^t-e^t}+ \vec C$}

\exercise{$\ds \int\bracket{\frac{1}{1+t^2},\sec^2 t}\ dt$}{$\bracket{\tan^{-1} t,\tan t}+ \vec C$}

\exercise{$\ds \int_0^{\pi}\bracket{-\sin t,\cos t}\ dt$}{$\bracket{-2,0}$}

\exercise{$\ds \int_{-2}^{2}\bracket{2t+1,2t-1}\ dt$}{$\bracket{4,-4}$}

\end{exerciseset}


\exerciseset{In Exercises}{, solve the given initial value problems.}{

\exercise{Find $\vec r(t)$, given that $\vrp(t) =\bracket{t,\sin t}$ and $\vec r(0) =\bracket{2,2}$.}{$\vec r(t) =\bracket{\frac12t^2+2,-\cos t+3}$}

\exercise{Find $\vec r(t)$, given that $\vrp(t) =\bracket{1/(t+1),\tan t}$ and \\ $\vec r(0) =\bracket{1,2}$.}{$\vec r(t) =\bracket{\ln\abs{t+1}+ 1, -\ln\abs{\cos t}+ 2}$}

\exercise{Find $\vec r(t)$, given that $\vrp'(t) =\bracket{t^2,t,1}$,\\ $\vrp(0) =\bracket{1,2,3}$ and $\vec r(0) =\bracket{4,5,6}$.}{$\vec r(t) =\bracket{t^4/12+t+4,\ t^3/6+2t+5,\ t^2/2+3t+6}$}

\exercise{Find $\vec r(t)$, given that $\vrp'(t) =\bracket{\cos t,\sin t,e^t}$,\\  $\vrp(0) =\bracket{0,0,0}$ and $\vec r(0) =\bracket{0,0,0}$.}{$\vec r(t) =\bracket{-\cos t+1,t-\sin t,e^t-t-1}$}

}


\exerciseset{In Exercises}{, find the arc length of $\vec r(t)$ on the indicated interval.}{

\exercise{$\vec r(t) =\bracket{2\cos t, 2\sin t, 3t}$ on $[0,2\pi]$.}{$2\sqrt{13}\pi$}

\exercise{$\vec r(t) =\bracket{5\cos t, 3\sin t, 4\sin t}$ on $[0,2\pi]$.}{$10\pi$}

\exercise{$\vec r(t) =\bracket{t^3,t^2,t^3}$ on $[0,1]$.}{$\frac1{54}\left((22)^{3/2}-8\right)$}

\exercise{$\vec r(t) =\bracket{e^{-t}\cos t,e^{-t}\sin t}$ on $[0,1]$.}{$\sqrt{2}(1-e^{-1})$}

\exercise{$\vec r(t) =\bracket{t,\frac{t^2}{\sqrt2},\frac{t^3}3}$ on $[0,3]$.}{12}

\exercise{$\vec r(t) =\bracket{t \cos t, t \sin t, t}$ on $[0,4\pi]$.}{$2\pi\sqrt{2+16\pi^2}+\ln(2\sqrt2\pi+\sqrt{1+8\pi^2})$}

}


\exercise{Prove\label{pr_const_length} \autoref{thm:vects_of_constant_length}; that is, show if $\vec r(t)$ has constant length and is differentiable, then $\vec r(t)\cdot \vrp(t)=0$. (Hint: use the Product Rule to compute $\frac{\dd}{\dd t}\big(\vec r(t)\cdot\vec r(t)\big)$.)}{As $\vec r(t)$ has constant length, $\vec r(t)\cdot\vec r(t)=c^2$ for some constant $c$. Thus
\begin{align*}
\vec r(t)\cdot\vec r(t) &= c^2\\
\frac{\dd}{\dd t}\big(\vec r(t)\cdot\vec r(t)\big) &= \frac{\dd}{\dd t}\big(c^2\big)\\
\vrp(t)\cdot \vec r(t)+\vec r(t)\cdot \vrp(t) &= 0\\
2\vec r(t)\cdot \vrp(t) &=0\\
\vec r(t)\cdot \vrp(t) &=0.
\end{align*}}

\exercise{The graph of 
\[\vec{r}(t) =\bracket{R \cos(2\pi Nt), R\sin(2\pi Nt), ht}\]
for $0 \leq t \leq 1$ is a helix of radius $R$ which completes $N$ turns and has height $h$.  Find the length of this curve.}{$\sqrt{h^2+4\pi^2R^2N^2}$}
