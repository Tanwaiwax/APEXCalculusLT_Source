\printconcepts

\exercise{How is \textit{velocity} different from \textit{speed}?}{Velocity is a vector, indicating an objects direction of travel and its rate of distance change (i.e., its speed). Speed is a scalar.}

\exercise{What is the difference between \textit{displacement} and \textit{distance traveled}?}{Displacement is a vector, indicating the difference between the starting and ending positions of an object. Distance traveled is a scalar, indicating the arc length of the path followed.}

\exercise{What is the difference between \textit{average velocity} and \textit{average speed}?}{The average velocity is found by dividing the displacement by the time traveled --- it is a vector. The average speed is found by dividing the distance traveled by the time traveled --- it is a scalar. }

\exercise{\textit{Distance traveled} is the same as \underline{\hskip .5in} \underline{\hskip .5in}, just viewed in a different context.}{arc length}

\exercise{Describe a scenario where an object's average speed is a large number, but the magnitude of the average velocity is not a large number.}{One example is traveling at a constant speed $s$ in a circle, ending at the starting position. Since the displacement is $\vec 0$, the average velocity is $\vec 0$, hence $\vnorm 0=0$. But traveling at constant speed $s$ means the average speed is also $s >0$.}

\exercise{Explain why it is not possible to have an average velocity with a large magnitude but a small average speed.}{Distance traveled is always greater than or equal to the magnitude of displacement, therefore average speed will always be at least as large as the magnitude of the average velocity.}

\printproblems

\exerciseset{In Exercises}{ , a position function \vrt\ is given. Find $\vvt$ and $\vat$.
}{

\exercise{$\vrt = \la 2t+1, 5t-2, 7\ra$
}{
$\vvt = \la 2,5,0\ra$, $\vat = \la 0,0,0\ra$}

\exercise{$\vrt = \la 3t^2-2t+1, -t^2+t+14\ra$
}{
$\vvt = \la 6t-2,-2t+1\ra$, $\vat = \la6,-2 \ra$}

\exercise{$\vrt = \la \cos t,\sin t\ra$
}{
$\vvt = \la -\sin t, \cos t\ra$, $\vat = \la -\cos t, -\sin t \ra$}

\exercise{$\vrt = \la t/10,-\cos t,\sin t\ra$
}{
$\vvt = \la 1/10, \sin t, \cos t\ra$, $\vat = \la 0,\cos t, -\sin t \ra$}
}

\exerciseset{In Exercises}{ , a position function $\vrt$ is given. Sketch \vrt\ on the indicated interval. Find \vvt\ and \vat, then add $\vec v(t_0)$ and $\vec a(t_0)$ to your sketch, with their initial points at $\vec r(t_0)$, for the given value of $t_0$.}{

\exercise{$\ds \vrt = \la t,\sin t \ra$ on $[0,\pi/2]$; $t_0= \pi/4$}{$\vvt = \la 1,\cos t\ra$, $\vat = \la 0,-\sin t\ra$

\myincludegraphics{figures/fig11_03_ex_11}}

\exercise{$\ds \vrt = \la t^2,\sin t^2 \ra$ on $[0,\pi/2]$; $t_0=\sqrt{\pi/4}$}{$\vvt = \la 2 t,2 t \cos \left(t^2\right)\ra$, $\vat = \la 2,2 \left(\cos \left(t^2\right)-2 t^2 \sin \left(t^2\right)\right)\ra$

\myincludegraphics{figures/fig11_03_ex_12}}

\exercise{$\ds \vrt = \la t^2+t,-t^2+2t \ra$ on $[-2,2]$; $t_0=1$}{$\vvt = \la 2t+1,-2t+2\ra$, $\vat = \la 2,-2\ra$

\begin{tikzpicture}[>=stealth]
 \begin{axis}[width=\marginparwidth+25pt,tick label style={font=\scriptsize},
   axis y line=middle,axis x line=middle,name=myplot,
   ytick={2,-2,-4,-6,-8},ymin=-8.8,ymax=2.9,xmin=-.5,xmax=6.9]
  \addplot [thick,{\colorone}, smooth,domain=-2:2] ({x^2+x},{-x^2+2*x});
  \draw [thick,{\colortwo},->] (axis cs:2,1) -- (axis cs:5,1)
   node [black,above] {\scriptsize $\vec v(1)$};
  \draw [thick,{\colortwo},->] (axis cs: 2,1) -- (axis cs:4,-1)
   node [black,below] {\scriptsize $\vec a(1)$};
 \end{axis}
 \node [right] at (myplot.right of origin) {\scriptsize $x$};
 \node [above] at (myplot.above origin) {\scriptsize $y$};
\end{tikzpicture}}

\exercise{$\ds \vrt = \la \frac{2t+3}{t^2+1},t^2\ra$ on $[-1,1]$; $t_0= 0$}{$\vvt = \la -\frac{2 \left(t^2+3t-1\right)}{\left(t^2+1\right)^2},2 t\ra$, $\vat = \la \frac{2 \left(2 t^3+9 t^2-6 t-3\right)}{\left(t^2+1\right)^3},2\ra$

\myincludegraphics{figures/fig11_03_ex_10}}

}


\exerciseset{In Exercises}{ , a position function $\vrt$ of an object is given. Find the speed of the object in terms of $t$, and find where the speed is minimized/maximized on the indicated interval.
}{

\exercise{$\ds \vrt = \la t^2,t \ra$ on $[-1,1]$
}{
$\norm{\vvt} = \sqrt{4t^2+1}$. \\
Min at $t=0$; Max at $t=\pm 1$. 
}
\exercise{$\ds \vrt = \la t^2,t^2-t^3 \ra$ on $[-1,1]$
}{
$\norm{\vvt} = |t|\sqrt{9t^2-12t+8}$. \\
min: $t=0$; max: $t=-1$
}
\exercise{$\ds \vrt = \la 5\cos t,5\sin t \ra$ on $[0,2\pi]$
}{
$\norm{\vvt} = 5$. \\
Speed is constant, so there is no difference between min/max
}
\exercise{$\ds \vrt = \la 2\cos t,5\sin t \ra$ on $[0,2\pi]$
}{
$\norm{\vvt} = \sqrt{4\sin^2t+25\cos^2t}$. \\
min: $t=\pi/2,\ 3\pi/2$; max: $t=0,\ 2\pi$
}
\exercise{$\ds \vrt = \la \sec t,\tan t \ra$ on $[0,\pi/4]$
}{
$\norm{\vvt} = |\sec t|\sqrt{\tan^2t+\sec^2t}$. \\
min: $t=0$; max: $t=\pi/4$
}
\exercise{$\ds \vrt = \la t+\cos t,1-\sin t \ra$ on $[0,2\pi]$
}{
$\norm{\vvt} = \sqrt{2-2\sin t}$. \\
min: $t=\pi/2$; max: $t=3\pi/2$
}
\exercise{$\ds \vrt = \la 12t,5\cos t,5\sin t \ra$ on $[0,4\pi]$
}{
$\norm{\vvt} = 13$. \\
speed is constant, so there is no difference between min/max
}
\exercise{$\ds \vrt = \la t^2-t,t^2+t,t \ra$ on $[0,1]$
}{
$\norm{\vvt} = \sqrt{8t^2+3}$. \\
min: $t=0$; max: $t=1$
}
\exercise{$\ds \vrt = \la t,t^2,\sqrt{1-t^2}\ra$ on $[-1,1]$
}{
$\norm{\vvt} = \sqrt{4t^2+1+t^2/(1-t^2)}$. \\
min: $t=0$; max: there is no max; speed approaches $\infty$ as $t\to\pm 1$
}
\exercise{\textbf{Projectile Motion:}
 $\ds \vrt = \la (v_0\cos \theta)t,-\frac12gt^2+(v_0\sin\theta)t \ra$ on $\ds \left[0,\frac{2v_0\sin\theta}g\right]$
}{
$\norm{\vvt} = \sqrt{g^2t^2-(2gv_0\sin\theta)t+v_0^2}$. \\
min: $t=(v_0\sin\theta)/g$; max: $t=0$, $t=(2v_0\sin\theta)/g$
}
}

\exerciseset{In Exercises}{ , position functions $\vec r_1(t)$ and $\vec r_2(s)$ for two objects are given that follow the same path on the respective intervals.
	\begin{enumerate}
		\item [(a)] Show that the positions are the same at the indicated $t_0$ and $s_0$ values; i.e., show $\vec r_1(t_0) = \vec r_2(s_0).$
		\item	[(b)] Find the velocity, speed and acceleration of the two objects at $t_0$ and $s_0$, respectively.
	\end{enumerate}
}{

\exercise{$\vec r_1(t) = \la t,t^2\ra$ on $[0,1]$; $t_0 = 1$\\
$\vec r_2(s) = \la s^2,s^4\ra$ on $[0,1]$; $s_0 = 1$
}{
\begin{enumerate}
	\item $\vec r_1(1) = \la 1,1\ra$; $\vec r_2(1) = \la 1,1\ra$
	\item	$\vec v_1(1) = \la 1,2\ra$; $\norm{\vec v_1(1)} = \sqrt{5}$; $\vec a_1(1) = \la 0,2\ra$\\
			$\vec v_2(1) = \la 2,4\ra$; $\norm{\vec v_2(1)} = 2\sqrt{5}$; $\vec a_2(1) = \la 2,12\ra$
\end{enumerate}
}
\exercise{$\vec r_1(t) = \la 3\cos t,3\sin t\ra$ on $[0,2\pi]$; $t_0 = \pi/2$\\
$\vec r_2(s) = \la 3\cos (4s),3\sin(4s)\ra$ on $[0,\pi/2]$; $s_0 = \pi/8$
}{
\begin{enumerate}
	\item $\vec r_1(\pi/2) = \la0,3\ra$; $\vec r_2(\pi/8) = \la 0,3\ra$
	\item	$\vec v_1(\pi/2) = \la -3,0\ra$; $\norm{\vec v_1(\pi/2)} = 3$; $\vec a_1(\pi/2) = \la 0,-3\ra$\\
			$\vec v_2(\pi/8) = \la -12,0\ra$; $\norm{\vec v_2(\pi/8)} = 12$; $\vec a_2(\pi/8) = \la 0,-48\ra$
\end{enumerate}
}
\exercise{$\vec r_1(t) = \la 3t,2t\ra$ on $[0,2]$; $t_0 = 2$\\
$\vec r_2(s) = \la 6t-6,4t-4\ra$ on $[1,2]$; $s_0 = 2$
}{
\begin{enumerate}
	\item $\vec r_1(2) = \la6,4\ra$; $\vec r_2(2) = \la 6,4\ra$
	\item	$\vec v_1(2) = \la 3,2\ra$; $\norm{\vec v_1(2)} = \sqrt{13}$; $\vec a_1(2) = \la 0,0\ra$\\
			$\vec v_2(2) = \la 6,4\ra$; $\norm{\vec v_2(2)} = 2\sqrt{13}$; $\vec a_2(2) = \la 0,0\ra$
\end{enumerate}
}
\exercise{$\vec r_1(t) = \la t,\sqrt{t}\ra$ on $[0,1]$; $t_0 = 1$\\
$\vec r_2(s) = \la \sin t,\sqrt{\sin t}\ra$ on $[0,\pi/2]$; $s_0 = \pi/2$
}{
\begin{enumerate}
	\item $\vec r_1(1) = \la 1,1\ra$; $\vec r_2(\pi/2) = \la 1,1\ra$
	\item	$\vec v_1(1) = \la 1,1/2\ra$; $\norm{\vec v_1(1)} = \sqrt{5}/2$; $\vec a_1(1) = \la 0,-1/4\ra$\\
			$\vec v_2(\pi/2) = \la 0,0\ra$; $\norm{\vec v_2(\pi/2)} = 0$; $\vec a_2(\pi/2) = \la -1,-1/2\ra$
\end{enumerate}
}
}

\exerciseset{In Exercises}{ , find the position function of an object given its acceleration and initial velocity and position.
}{

\exercise{$\vat = \la 2,3\ra$;\quad $\vec v(0) = \la 1,2\ra$,\quad $\vec r(0) = \la 5,-2\ra$
}{$\vvt = \la 2t+1,3t+2\ra$, $\vrt = \la t^2+t+5,3t^2/2+2t-2\ra$
}
\exercise{$\vat = \la 2,3\ra$;\quad $\vec v(1) = \la 1,2\ra$,\quad $\vec r(1) = \la 5,-2\ra$
}{$\vvt = \la 2t-1,3t-1\ra$, $\vrt = \la t^2-t+5,3t^2/2-t-5/2\ra$
}
\exercise{$\vat = \la \cos t,-\sin t\ra$;\quad $\vec v(0) = \la 0,1\ra$,\quad $\vec r(0) = \la 0,0\ra$
}{$\vvt = \la \sin t,\cos t\ra$, $\vrt = \la 1-\cos t,\sin t\ra$
}
\exercise{$\vat = \la 0,-32\ra$;\quad $\vec v(0) = \la 10,50\ra$,\quad $\vec r(0) = \la 0,0\ra$
}{$\vvt = \la 10,-32t+50\ra$, $\vrt = \la 10t,-16t^2+50t\ra$
}
}

\exerciseset{In Exercises}{ , find the displacement, distance traveled, average velocity and average speed of the described object on the given interval.
}{

\exercise{An object with position function $\vrt = \la 2\cos t,2\sin t, 3t\ra$, where distances are measured in feet and time is in seconds, on $[0,2\pi]$.
}{Displacement: $\la 0,0,6\pi\ra$; distance traveled: $2\sqrt{13}\pi \approx 22.65$ft; average velocity: $\la 0,0,3\ra$; average speed: $\sqrt{13} \approx 3.61$ft/s
}
\exercise{An object with position function $\vrt = \la 5\cos t,-5\sin t\ra$, where distances are measured in feet and time is in seconds, on $[0,\pi]$.
}{Displacement: $\la -10,0\ra$; distance traveled: $5\pi \approx 15.71$ft; average velocity: $\la -10/\pi,0\ra\approx \la -3.18,0\ra$; average speed: $5$ft/s
}
\exercise{An object with velocity function $\vvt = \la \cos t,\sin t\ra$, where distances are measured in feet and time is in seconds, on $[0,2\pi]$.
}{Displacement: $\la 0,0\ra$; distance traveled: $2\pi \approx 6.28$ft; average velocity: $\la 0,0\ra$; average speed: $1$ft/s
}
\exercise{An object with velocity function $\vvt = \la 1,2,-1\ra$, where distances are measured in feet and time is in seconds, on $[0,10]$.
}{Displacement: $\la 10,20,-20\ra$; distance traveled: $30$ft; average velocity: $\la 1,2,-2\ra$; average speed: $3$ft/s
}
}

\input{exercises/11_03_exset_07}
