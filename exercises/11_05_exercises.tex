\printconcepts

\exercise{It is common to describe position in terms of both \underline{\hskip .5in} and/or \underline{\hskip .5in}.}{time and/or distance}

\exercise{A measure of the ``curviness'' of a curve is \underline{\hskip .5in}.}{curvature}

\exercise{Give two shapes with constant curvature.}{Answers may include lines, circles, helixes}

\exercise{Describe in your own words what an ``osculating circle'' is. }{Answers will vary; they should mention the circle is tangent to the curve and has the same curvature as the curve at that point.}

\exercise{Complete the identity: $\vec T\,'(s) = \underline{\hskip .5in}\vec N(s)$.}{$\kappa$}

\exercise{Given a position function \vrt, how are $a_{\text{T}}$ and $a_{\text{N}}$ affected by the curvature?}{$a_{\text{T}}$ is not affected by curvature; the greater the curvature, the larger $a_{\text{N}}$ becomes.}

\printproblems

\begin{exerciseset}{In Exercises}{, a position function \vrt\ is given, where $t=0$ corresponds to the initial position. Find the arc length parameter $s$, and rewrite \vrt\ in terms of $s$; that is, find $\vec r(s)$.}

\exercise{$\vrt =\bracket{2t, t, -2t}$}{$s = 3t$, so $\vec r(s)  =\bracket{2s/3, s/3, -2s/3}$}

\exercise{$\vrt =\bracket{7\cos t,7\sin t}$}{$s = 7t$, so $\vec r(s)  =\bracket{7\cos (s/7), 7\sin (s/7)}$}

\exercise{$\vrt =\bracket{3\cos t,3\sin t, 2t}$}{$s = \sqrt{13}t$, so\\
$\vec r(s)  =\bracket{3\cos (s/\sqrt{13}), 3\sin (s/\sqrt{13}), 2s/\sqrt{13}}$}

\exercise{$\vrt =\bracket{5\cos t,13\sin t, 12\cos t}$}{$s = 13t$, so\\
$\vec r(s)  =\bracket{5\cos (s/13), 13\sin (s/13), 12\cos (s/13)}$}

\end{exerciseset}


\input{exercises/11_05_exset_02}

\exerciseset{In Exercises}{ , find the value of $x$ or $t$ where curvature is maximized.
}{

\exercise{$\ds y=\frac16x^3$ 
}{maximized at $x=\pm \frac{\sqrt{2}}{\sqrt[4]{5}}$
}
\exercise{$\ds y=\sin x$ 
}{maximized at $x=\ldots -3\pi/2, -\pi/2, \pi/2, \ldots$
}
\exercise{$\vrt = \la t^2+2t,3t-t^2\ra$ 
}{maximized at $t=1/4$
}
\exercise{$\vrt = \la t, 4/t, 3/t\ra$ 
}{maximized at $t= \pm \sqrt{5}$
}}

\exerciseset{In Exercises}{, find the radius of curvature at the indicated value.}{

\exercise{$y=\tan x$, at $x=\pi/4$}{radius of curvature is $5\sqrt{5}/4$.}

\exercise{$y=x^2+x-3$, at $x=\pi/4$}{radius of curvature is $5\sqrt{10}$.}

\exercise{$\vrt =\bracket{\cos t, \sin (3t)}$, at $t=0$}{radius of curvature is $9$.}

\exercise{$\vrt =\bracket{5\cos (3 t), t}$, at $t=0$}{radius of curvature is $1/45$.}

}


\exerciseset{In Exercises}{ , find the equation of the osculating circle to the curve at the indicated $t$-value.
}{

\exercise{$\vrt = \la t,t^2\ra$, at $t=0$
}{$x^2+(y-1/2)^2 = 1/4$, or $\vec c(t) = \la 1/2\cos t, 1/2\sin t+ 1/2\ra$ 
}
\exercise{$\vrt = \la 3\cos t, \sin t\ra$, at $t=0$
}{$(x-8/3)^2+y^2 = 1/9$, or $\vec c(t) = \la \frac13\cos t+\frac83, \frac13\sin t\ra$ 
}
\exercise{$\vrt = \la 3\cos t, \sin t\ra$, at $t=\pi/2$
}{$x^2+(y+8)^2 = 81$, or $\vec c(t) = \la 9\cos t, 9\sin t-8\ra$ 
}
\exercise{$\vrt = \la t^2-t,t^2+t\ra$, at $t=0$
}{$(x-1/2)^2+(y-1/2)^2 = 1/2$, or $\vec c(t) = \la \frac{\sqrt{2}}{2}\cos t+\frac12, \frac{\sqrt{2}}{2}\sin t+\frac12\ra$ 
}}

%\input{exercises/11_04_exset_05}

\exercise{For \autoref{thm:curvature_formulas}, use part 3 to prove part 2: If $\vec r(t)=\bracket{x(t), y(t)}$ is a vector-valued function in the plane, then
\[\kappa = \frac{\abs{x\primeskip'y\primeskip''-x\primeskip''y\primeskip'}}{\big((x\primeskip')^2+(y\primeskip')^2\big)^{3/2}}.\]}{Let $\vec r(t)=\bracket{x(t),y(t),0}$ and apply the second formula of part 3.}

\exercise{For \autoref{thm:curvature_formulas}, use part 2 or 3 to prove part 1: If $y=f(x)$, then
\[\kappa = \frac{\abs{\fpp(x)}}{\big(1+(\fp(x))^2\big)^{3/2}}.\]}{Let $\vec r(t)=\bracket{t,f(t),0}$ and apply part 2 or the second formula of part 3.}
