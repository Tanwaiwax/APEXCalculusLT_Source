\exercisesetinstructions{, a limit is given. Evaluate the limit along the paths given, then state why these results show the given limit does not exist.}

\exercise{$\ds \lim_{(x,y)\to(0,0)} \frac{x^2-y^2}{x^2+y^2}$
\begin{enumext}[start=1]
	\item Along the path $y=0$.
	\item	Along the path $x=0$.
\end{enumext}}{\mbox{}\\[-2\baselineskip]\parbox[t]{\linewidth}{\begin{enumext}[start=1]
	\item Along $y=0$, the limit is 1.
	\item	Along $x=0$, the limit is $-1$.
\end{enumext}}
Since the above limits are not equal, the limit does not exist.}

\exercise{$\ds \lim_{(x,y)\to(0,0)} \frac{x+y}{x-y}$
\begin{enumext}[start=1]
	\item Along the path $y=mx$.
\end{enumext}}{\mbox{}\\[-2\baselineskip]\parbox[t]{\linewidth}{\begin{enumext}[start=1]
	\item Along $y=mx$, the limit is $\frac{m+1}{m-1}$.
\end{enumext}}
Since the above limit varies according to what $m$ is used, each limit is different, meaning the overall limit does not exist.}

\exercise{$\ds \lim_{(x,y)\to(0,0)} \frac{xy-y^2}{y^2+x}$
\begin{enumext}[start=1]
	\item Along the path $y=mx$.
	\item Along the path $x=0$.
\end{enumext}}{\mbox{}\\[-2\baselineskip]\parbox[t]{\linewidth}{\begin{enumext}[start=1]
	\item Along $y=mx$, the limit is $0$. %$\ds\frac{mx(1-m)}{m^2x+1}$.
	\item	Along $x=0$, the limit is $-1$.
\end{enumext}}
Since the above limits are not equal, the limit does not exist.}

\exercise{$\ds \lim_{(x,y)\to(0,0)} \frac{\sin(x^2)}{y}$
\begin{enumext}[start=1]
	\item Along the path $y=mx$.
	\item Along the path $y=x^2$.
\end{enumext}}{\mbox{}\\[-2\baselineskip]\parbox[t]{\linewidth}{\begin{enumext}[start=1]
	\item Along $y=mx$, the limit is: 
	\begin{align*}
		\lim_{(x,y)\to(0,0)} \frac{\sin(x^2)}{y}
		& =  \lim_{x\to 0} \frac{\sin(x^2)}{mx}
		\intertext{apply L'Hôpital's Rule}
		&= \lim_{x\to 0} \frac{2x\cos(x^2)}{m}\\
		&= 0.
	\end{align*}
	\item	Along $x=0$, the limit is:
   \[\lim_{(x,y)\to(0,0)}\frac{\sin(x^2)}y=\lim_{x\to 0}\frac{\sin(x^2)}{x^2}.\]
	This can be evaluated with L'Hôpital's Rule or from known limits; it is 1.
\end{enumext}}
Since the limits along the lines $y=mx$ are not the same as the limit along $y=x^2$, the overall limit does not exist.}

\exercise{$\ds \lim_{(x,y)\to(1,2)} \frac{x+y-3}{x^2-1}$
\begin{enumext}[start=1]
	\item Along the path $y=2$.
	\item Along the path $y=x+1$.
\end{enumext}}{\mbox{}\\[-2\baselineskip]\parbox[t]{\linewidth}{\begin{enumext}[start=1]
	\item Along $y=2$, the limit is: 
	\begin{align*}
		\lim_{(x,y)\to(1,2)} \frac{x+y-3}{x^2-1}
		& =  \lim_{x\to 1} \frac{x-1}{x^2-1}\\
		&= \lim_{x\to 1} \frac{1}{x+1}\\
		&= 1/2.
	\end{align*}
	\item	Along $y=x+1$, the limit is:
	\begin{align*}
		\lim_{(x,y)\to(1,2)} \frac{x+y-3}{x^2-1}
		& =  \lim_{x\to 1} \frac{2(x-1)}{x^2-1}\\
		&= \lim_{x\to 1} \frac{2}{x+1}\\
		&= 1.
	\end{align*}
\end{enumext}}
Since the limits along the lines $y=2$ and $y=x+1$ differ, the overall limit does not exist.}

\exercise{$\ds \lim_{(x,y)\to(\pi,\pi/2)} \frac{\sin x}{\cos y}$
\begin{enumext}[start=1]
	\item Along the path $x=\pi$.
	\item Along the path $y=x-\pi/2$.
\end{enumext}}{\mbox{}\\[-2\baselineskip]\parbox[t]{\linewidth}{\begin{enumext}[start=1]
	\item Along $x=\pi$, the limit is: 
	\begin{align*}
		\lim_{(x,y)\to(\pi,\pi/2)} \frac{\sin x}{\cos y} & =  \lim_{y\to \pi/2} \frac{0}{\cos y}\\
		&= 0.
	\end{align*}
	\item	Along $y=x-\pi/2$, the limit is:
	\begin{align*}
	   \lim_{(x,y)\to(\pi,\pi/2)} \frac{\sin x}{\cos y} & =  \lim_{x\to \pi} \frac{\sin x}{\cos(x-\pi/2)}\intertext{Apply L'Hôpital's Rule:}
		&= \lim_{x\to \pi} \frac{\cos x}{\sin(x-\pi/2)}\\
		&= -1.
	\end{align*}
\end{enumext}}
Since the limits along the lines $x=\pi$ and $y=x-\pi$ differ, the overall limit does not exist.}

\exercisesetend
