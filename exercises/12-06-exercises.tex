\printconcepts

\exercise{Explain how the vector $\vec v=\bracket{1,0,3}$ can be thought of as having a ``slope'' of 3.}{Answers will vary. The displacement of the vector is one unit in the $x$-direction and 3 units in the $z$-direction, with no change in $y$. Thus along a line parallel to $\vec v$, the change in $z$ is 3 times the change in $x$ --- i.e., a ``slope'' of 3. Specifically, the line in the $x$-$z$ plane parallel to $z$ has a slope of 3.}

\exercise{Explain how the vector $\vec v=\bracket{0.6,0.8, -2}$ can be thought of as having a ``slope'' of $-2$.}{Answers will vary. Let $\vec u =\bracket{0.6,0.8}$; this is a unit vector. The displacement of the vector is one unit in the $\vec u$-direction and $-2$ units in the $z$-direction. In the plane containing the $z$-axis and the vector $\vec u$, the line parallel to $\vec v$ has slope $-2$.}

\exercise{T/F: Let $z=f(x,y)$ be differentiable at $P$. If $\vec n$ is a normal vector to the tangent plane of $f$ at $P$, then $\vec n$ is orthogonal to $\ell_x$ and $\ell_y$ at $P$.}{T}

\exercise{Explain in your own words why we do not refer to \textit{the} tangent line to a surface at a point, but rather to \textit{directional} tangent line\textit{s} to a surface at a point.}{On a surface through a point, there are many different smooth curves, each with a tangent line at the point. Each of these tangent lines is also ``tangent'' to the surface. There is not just one tangent line, but many, each in a different direction. Therefore we refer to directional tangent lines, not just \textit{the} tangent line.}

\printproblems

\exercisesetinstructions{, a function $z=f(x,y)$, a vector $\vec v$ and a point $P$ are given. Give the parametric equations of the following directional tangent lines to the graph of $f$ at $P$:
\begin{enumext}
	\item $\ell_x(t)$
	\item $\ell_y(t)$
	\item $\ell_{\vec u\,}(t)$, where $\vec u$ is the unit vector in the direction of $\vec v$.
\end{enumext}}

\exercise{$f(x,y) = 2x^2y-4xy^2$,  $\vec v =\bracket{1,3}$, $P=(2,3)$.\label{12_06_ex_05}}{\mbox{}\\[-2\baselineskip]\parbox[t]{\linewidth}{\begin{enumext}
	\item $\ell_x(t) = \begin{cases}x=2+t\\ y=3 \\ z = -48-12t\end{cases}$
	\item $\ell_y(t) = \begin{cases}x=2\\ y=3+t \\ z = -48-40t\end{cases}$
	\item $\ell_{\vec u\,}(t) = \begin{cases}x=2+t/\sqrt{10}\\ y=3+3t/\sqrt{10} \\ z = -48-66\sqrt{2/5}t\end{cases}$
\end{enumext}}}

\exercise{$f(x,y) = 3\cos x\sin y$,  $\vec v =\bracket{1,2}$, $P=(\pi/3, \pi/6)$.}{\mbox{}\\[-2\baselineskip]\parbox[t]{\linewidth}{\begin{enumext}
	\item $\ell_x(t) = \begin{cases}x = \pi/3+t\\ y = \pi/6 \\ z = 3/4 -\frac{3\sqrt{3}}{4}t \end{cases}$ 
	\item $\ell_y(t) = \begin{cases}x = \pi/3\\ y = \pi/6 +t\\ z = 3/4 +\frac{3\sqrt{3}}{4}t \end{cases}$
	\item $\ell_{\vec u\,}(t) = \begin{cases}x = \pi/3+t/\sqrt{5}\\ y = \pi/6+2t/\sqrt{5} \\ z = 3/4 +\frac{3\sqrt{3/5}}{4}t \end{cases}$
\end{enumext}}}

\exercise{$f(x,y) = 3x-5y$,  $\vec v =\bracket{1,1}$, $P=(4,2)$.}{\mbox{}\\[-2\baselineskip]\parbox[t]{\linewidth}{\begin{enumext}
	\item $\ell_x(t) = \begin{cases}x = 4+t\\ y = 2 \\ z = 2 + 3t \end{cases}$ 
	\item $\ell_y(t) = \begin{cases}x = 4\\ y = 2+t\\ z = 2-5t \end{cases}$
	\item $\ell_{\vec u\,}(t) = \begin{cases}x = 4+t/\sqrt{2}\\ y = 2+t/\sqrt{2} \\ z = 2 -\sqrt{2}t \end{cases}$
\end{enumext}}}

\exercise{$f(x,y) = x^2-2x-y^2+4y$,  $\vec v =\bracket{1,1}$, $P=(1,2)$.\label{12_06_ex_08}}{\mbox{}\\[-2\baselineskip]\parbox[t]{\linewidth}{\begin{enumext}
	\item $\ell_x(t) = \begin{cases}x = 1+t\\ y = 2 \\ z = 3 \end{cases}$ 
	\item $\ell_y(t) = \begin{cases}x = 1\\ y = 2+t\\ z = 3 \end{cases}$
	\item $\ell_{\vec u\,}(t) = \begin{cases}x = 1+t/\sqrt{2}\\ y = 2+t/\sqrt{2} \\ z = 3 \end{cases}$
\end{enumext}}}

\exercisesetend


\input{exercises/12-06-exset-02}

\begin{exerciseset}{In Exercises}{, a function $z=f(x,y)$ and a point $P$ are given. Find the two points that are 2 units from the surface of the graph of $f$ at $P$. Note: these are the same functions as in Exercises~\ref{12_06_ex_05}--\ref{12_06_ex_08}.}

\exercise{$f(x,y) = 2x^2y-4xy^2$, $P=(2,3)$.\label{12_06_ex_13}}{$(1.425, 1.085, -48.078)$, $(2.575, 4.915, -47.952)$}

\exercise{$f(x,y) = 3\cos x\sin y$, $P=(\pi/3, \pi/6)$.}{$(-0.195,1.766,-0.206)$ and $(2.289,-0.719, 1.706)$ }

\exercise{$f(x,y) = 3x-5y$, $P=(4,2)$.}{$(5.014, 0.31, 1.662)$ and $(2.986, 3.690, 2.338)$ }

\exercise{$f(x,y) = x^2-2x-y^2+4y$, $P=(1,2)$.}{$(1,2,1)$ and $(1,2,5)$ }

\end{exerciseset}


\input{exercises/12-06-exset-04}

\input{exercises/12-06-exset-05}

% todo include some exercises finding the distance from a surface to a point as in Example 13.7.4
