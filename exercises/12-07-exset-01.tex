\exercisesetinstructions{, find the critical points of the given function. Use the Second Derivative Test to determine if each critical point corresponds to a relative maximum, minimum, or saddle point.}

\exercise{$f(x,y) = \frac12x^2+2y^2-8y+4x$}{One critical point at $(-4,2)$; $f_{xx} = 1$ and $D = 4$, so this point corresponds to a relative minimum.}

\exercise{$f(x,y) = x^2+4x+y^2-9y+3xy$}{One critical point at $(7,-6)$; $D = -5$, so this point corresponds to a saddle point.}

\exercise{$f(x,y) = x^2+3y^2-6y+4xy$}{One critical point at $(6,-3)$; $D = -4$, so this point corresponds to a saddle point.}

\exercise{$\ds f(x,y) = \frac1{x^2+y^2+1}$}{One critical point at $(0,0)$; $f_{xx} = -2$ and $D = 4$, so this point corresponds to a relative maximum.}

\exercise{$\ds f(x,y) = x^2+y^3-3y+1$}{Two critical points: at $(0,-1)$; $f_{xx} = 2$ and $D = -12$, so this point corresponds to a saddle point;\\
at $(0,1)$, $f_{xx} = 2$ and $D = 12$, so this corresponds to a relative minimum.}

\exercise{$\ds f(x,y) = \frac13x^3-x+\frac13y^3-4y$}{There are 4 critical points: \\
$(-1,-2)$, rel. max; $(1,-2)$, saddle point;\\
$(-1,2)$, saddle point; $(1,2)$, rel.\ min.,\\
where $f_{xx} = 2x$ and $D = 4xy$.}

\exercise{$\ds f(x,y) = x^2y^2$}{Critical points when $x$ or $y$ are $0$. $D = -12x^2y^2$, so the test is inconclusive. (Some elementary thought shows that these are absolute minima.)}

\exercise{$\ds f(x,y) = x^4-2x^2+y^3-27y-15$}{Six critical points: $f_x = 0$ when $x=-1,0$ and 1; $f_y = 0$ when $y = -3,3$. Together, we get the points:\\
$(-1,-3)$ saddle point;		$(-1,3)$ rel.\ min\\
$(0,-3)$	rel. max;	$(0,3)$ saddle point\\
$(1,-3)$ saddle point; 		$(1,3)$ relative min\\
where $f_{xx} = 12x^2-4$ and $D = 24y(3x^2-1)$.}

\exercise{$\ds f(x,y) = \sqrt{16-(x-3)^2-y^2}$}{One critical point: $f_x = 0$ when $x=3$; $f_y = 0$ when $y = 0$, so one critical point at $(3,0)$, which is a relative maximum, where 
$f_{xx} = \frac{y^2-16}{(16-(x-3)^2-y^2)^{3/2}}$ and $D = \frac{16}{(16-(x-3)^2-y^2)^{2}}$.\\
Both $f_x$ and $f_y$ are undefined along the circle $(x-3)^2+y^2=16$; at any point along this curve, $f(x,y)=0$, the absolute minimum of the function.}

\exercise{$\ds f(x,y) = \sqrt{x^2+y^2}$}{One critical point: $f_x = 0$ when $x=0$; $f_y = 0$ when $y = 0$, so one critical point at $(0,0)$ (although it should be noted that at $(0,0)$, both $f_x$ and $f_y$ are undefined.) The Second Derivative Test fails at $(0,0)$, with $D=0$. A graph, or simple calculation, shows that $(0,0)$ is the absolute minimum of $f$.}

% maintain parity
%\exercise{$f(x,y)=x^3-6xy+y^3$}{Saddle point at $(0, 0)$; rel.\ min at $(2, 2)$.}

\exercise{$f(x,y)=x^3+3xy^2-3x^2-3y^2+4$}{rel.\ max at $(0,0)$; rel.\ min at $(2,0)$; saddle points at $(1,\pm1)$.}

\exercise{$f(x,y)=y^3+x^2-6xy+3x+6y-7$}{rel.\ min at $(27/2,5)$; saddle point at $(3/2,1)$.}

\exercise{$f(x,y)=3x^2y+x^2-6x-3y-2$}{saddle points at $(1,2/3)$ and $(-1,-4/3)$.}

\exercise{$f(x,y)=(x^2+3y^2)e^{-2x}$}{rel.\ min at $(0,0)$, saddle point at $(1,0)$.}

\exercisesetend
