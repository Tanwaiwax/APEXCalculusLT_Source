\exercisesetinstructions{, find the absolute maximum and minimum of the function subject to the given constraint.}

% todo Tim should the solutions be less verbose?

\exercise{$\ds f(x,y) = x^2+y^2+y+1$, constrained to the triangle with vertices $(0,1)$, $(-1,-1)$ and $(1,-1)$.}{The triangle is bound by the lines $y=-1$, $y=2x+1$ and $y=-2x+1$.\\
Along $y=-1$, there is a critical point at $(0,-1)$.\\
Along $y=2x+1$, there is a critical point at $(-3/5,-1/5)$.\\
Along $y=-2x+1$, there is a critical point at $(3/5,-1/5)$.\\
The function $f$ has one critical point, irrespective of the constraint, at $(0,-1/2)$.\\
Checking the value of $f$ at these four points, along with the three vertices of the triangle, we find the absolute maximum is at $(0,1,3)$ and the absolute minimum is at $(0,-1/2,3/4)$.}

\exercise{$\ds f(x,y) = 5x-7y$, constrained to the region bounded by $y=x^2$ and $y=1$.}{The region has two ``corners'' at $(1,1)$ and $(-1,1)$.\\
Along $y=1$, there is no critical point.\\
Along $y=x^2$, there is a critical point at $(5/14,25/196)\approx (0.357,0.128) $.\\
The function $f$ itself has no critical points.\\
Checking the value of $f$ at the corners $(1,1)$, $(-1,1)$ and the critical point $(5/14,25/196)$, we find the absolute maximum is at $(5/14,25/196,25/28) \approx (0.357,0.128,0.893)$ and the absolute minimum is at $(-1,1,-12)$.}

\exercise{$\ds f(x,y) = x^2+2x+y^2+2y$, constrained to the region bounded by the circle $x^2+y^2=4$.}{The region has no ``corners'' or ``vertices,'' just a smooth edge.\\
To find critical points along the circle $x^2+y^2=4$, we solve for $y^2$: $y^2=4-x^2$. We can go further and state $y=\pm\sqrt{4-x^2}$. \\
We can rewrite $f$ as $f(x)=x^2+2x + (4-x^2) + 2\sqrt{4-x^2} = 2x+4+2\sqrt{4-x^2}$. (We will return and use $-\sqrt{4-x^2}$ later.) Solving $f\,'(x)=0$, we get $x=\sqrt{2} \Rightarrow y=\sqrt{2}$. $f\,'(x)$ is also undefined at $x=\pm 2$, where $y=0$.\\
Using $y=-\sqrt{4-x^2}$, we rewrite $f(x,y)$ as $f(x) = 2x+4-2\sqrt{4-x^2}$. Solving $\fp(x) =0$, we get $x=-\sqrt{2},\ y=-\sqrt{2}$.\\
The function $f$ itself has a critical point at $(-1,-1)$.\\
Checking the value of $f$ at $(-1,-1)$, $(\sqrt{2},\sqrt{2})$, $(-\sqrt{2},-\sqrt{2})$, $(2,0)$ and $(-2,0)$, we find the absolute maximum is at $(\sqrt2,\sqrt2,4+4\sqrt2)$ and the absolute minimum is at $(-1,-1,-2)$.}

\exercise{$\ds f(x,y) = 3y-2x^2$, constrained to the region bounded by the parabola $y=x^2+x-1$ and the line $y=x$.}{The region has two ``corners'' at $(-1,-1)$ and $(1,1)$.\\
Along the line $y=x$, $f(x,y)$ becomes $f(x) = 3x-2x^2$. Along this line, we have a critical point at $(3/4,3/4)$.\\
Along the curve $y=x^2+x-1$, $f(x,y)$ becomes $f(x) =x^2+3x-3$. There is a critical point along this curve at $(-3/2, -1/4)$. Since $x=-3/2$ lies outside our bounded region, we ignore this critical point.\\
The function $f$ itself has no critical points. \\
Checking the value of $f$ at $(-1,-1)$, $(1,1)$, $(3/4,3/4)$, we find the absolute maximum is at $(3/4,3/4,9/8)$ and the absolute minimum is at $(-1,-1,-5)$.}

\exercise{$f(x,y)=4(x+y)-(2x^2+y^2)$, constrained to the square with vertices $(0,0)$, $(3,0)$, $(3,3)$, and $(0,3)$.}{abs max is $(1,2,6)$, abs min is $(3,0,-6)$.}

\exercise{$f(x,y)=x^2+xy+y^2$, constrained to the region bounded by the circle $x^2+y^2=9$.}{abs max is $(\sqrt2/2,\sqrt2/2,3/2)$ and $(-\sqrt2/2,-\sqrt2/2,3/2)$, abs min is $(0,0,0)$.}

\exercisesetend
