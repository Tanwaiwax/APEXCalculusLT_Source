\printconcepts

\exercise{Give two examples (other than those given in the text) of ``real world'' functions that require more than one input.}{Answers will vary.}

\exercise{The graph of a function of two variables is a \underline{\hskip .5in}.}{surface}

\exercise{Most people are familiar with the concept of level curves in the context of \underline{\hskip .5in} maps.}{topographical}

\exercise{T/F: Along a level curve, the output of a function does not change.}{T}

\exercise{The analogue of a level curve for functions of three variables is a level \underline{\hskip .5in}.}{surface}

\exercise{What does it mean when level curves are close together? Far apart?}{When level curves are close together, it means the function is changing $z$-values rapidly. When far apart, it changes $z$-values slowly.}

\printproblems

\input{exercises/12_01_exset_01}

\exerciseset{In Exercises}{, describe in words and sketch the level curves for the function and given $c$ values.}{

\exercise{$\ds f(x,y) = 3x-2y$; $c = -2,0,2$}{Level curves are lines $y = (3/2)x-c/2$.\\
\myincludegraphics{figures/fig12_01_ex_15}
}
\exercise{$\ds f(x,y) = x^2-y^2$; $c = -1,0,1$}{Level curves are hyperbolas $\frac{x^2}{c}-\frac{y^2}{c}=1$, except for $c=0$, where the level curve is the pair of lines $y=x$, $y=-x$.\\
\myincludegraphics{figures/fig12_01_ex_16}
}
\exercise{$\ds f(x,y) = x-y^2$; $c = -2,0,2$}{Level curves are parabolas $x=y^2+c$.\\
\myincludegraphics{figures/fig12_01_ex_17}
}
\exercise{$\ds f(x,y) = \frac{1-x^2-y^2}{2y-2x}$; $c = -2,0,2$}{Level curves are hyperbolas $(x-c)^2-(y-c)^2=1$, drawn in graph in different styles to differentiate the curves.\\
\myincludegraphics{figures/fig12_01_ex_18}
}
\exercise{$\ds f(x,y) = \frac{2x-2y}{x^2+y^2+1}$; $c = -1,0,1$}{Level curves are circles, centered at $(1/c,-1/c)$ with radius $\sqrt{2/c^2-1}$. When $c=0$, the level curve is the line $y=x$.\\
\myincludegraphics{figures/fig12_01_ex_19}
}
\exercise{$\ds f(x,y) = \frac{y-x^3-1}{x}$; $c = -3,-1,0,1,3$}{Level curves are cubics of the form $y=x^3+cx+1$. Note how each curve passes through $(0,1)$ and that the function is not defined at $x=0$.\\
\myincludegraphics{figures/fig12_01_ex_20}
}
\exercise{$\ds f(x,y) = \sqrt{x^2+4y^2}$; $c = 1,2,3,4$\label{12_01_ex_21}}{Level curves are ellipses of the form $\frac{x^2}{c^2}+\frac{y^2}{c^2/4}=1$, i.e., $a=c$ and $b=c/2$.\\
\myincludegraphics{figures/fig12_01_ex_21}
}
\exercise{$\ds f(x,y) = x^2+4y^2$; $c = 1,2,3,4$\label{12_01_ex_22}}{Level curves are ellipses of the form $\frac{x^2}{c}+\frac{y^2}{c/4}=1$, i.e., $a=\sqrt{c}$ and $b=\sqrt{c}/2$.\\
\myincludegraphics{figures/fig12_01_ex_22}
}}

\input{exercises/12_01_exset_03}

\input{exercises/12_01_exset_04}

\exercise{Compare the level curves of Exercises \ref{12_01_ex_21} and \ref{12_01_ex_22}. How are they similar, and how are they different? Each surface is a quadric surface; describe how the level curves are consistent with what we know about each surface.}{The level curves for each surface are similar; for $z=\sqrt{x^2+4y^2}$ the level curves are ellipses of the form $\frac{x^2}{c^2}+\frac{y^2}{c^2/4}=1$, i.e., $a=c$ and $b=c/2$; whereas for $z=x^2+4y^2$ the level curves are ellipses of the form $\frac{x^2}{c}+\frac{y^2}{c/4}=1$, i.e., $a=\sqrt{c}$ and $b=\sqrt{c}/2$. The first set of ellipses are spaced evenly apart, meaning the function grows at a constant rate; the second set of ellipses are more closely spaced together as $c$ grows, meaning the function grows faster and faster as $c$ increases.

The function $z=\sqrt{x^2+4y^2}$ can be rewritten as $z^2=x^2+4y^2$, an elliptic cone; the function $z=x^2+4y^2$ is a paraboloid, each matching the description above.}
