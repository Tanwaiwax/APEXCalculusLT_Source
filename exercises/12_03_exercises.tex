\printconcepts

\exercise{What is the difference between a constant and a coefficient?}{A constant is a number that is added or subtracted in an expression; a coefficient is a number that is being multiplied by a nonconstant function. }

\exercise{Given a function $z=f(x,y)$, explain in your own words how to compute $f_x$.}{Answers will vary; each should include something about treating $y$ as a constant or a coefficient.}

\exercise{In the mixed partial fraction $f_{xy}$, which is computed first, $f_x$ or $f_y$?}{$f_x$}

\exercise{In the mixed partial fraction $\ds \frac{\partial^2f}{\partial x\partial y}$, which is computed first, $f_x$ or $f_y$?}{$f_y$}

\printproblems

\exerciseset{In Exercises}{, evaluate $f_x(x,y)$ and $f_y(x,y)$ at the indicated point.}{

\exercise{$f(x,y) = x^2y-x+2y+3$ at $(1,2)$
}{$f_x=2xy-1$, $f_y=x^2+2$\\
$f_x(1,2) = 3$, $f_y(1,2) = 3$
}
\exercise{$f(x,y) = x^3-3x+y^2-6y$ at $(-1,3)$
}{$f_x=3x^2-3$, $f_y=2y-6$\\
$f_x(-1,3) = 0$, $f_y(-1,3) = 0$
}
\exercise{$f(x,y) = \sin y\cos x$ at $(\pi/3,\pi/3)$
}{$f_x=-\sin x\sin y$, $f_y=\cos x\cos y$\\
$f_x(\pi/3,\pi/3) = -3/4$, $f_y(\pi/3,\pi/3) = 1/4$
}
\exercise{$f(x,y) = \ln(xy)$ at $(-2,-3)$
}{$f_x=1/x$, $f_y=1/y$\\
$f_x(-2,-3) = -1/2$, $f_y(-2,-3) = -1/3$
}}

\input{exercises/12_03_exset_02}

\exerciseset{In Exercises}{, form a function $z=f(x,y)$ such that $f_x$ and $f_y$ match those given.}{

\exercise{$f_x = \sin y+1$,\quad $f_y = x\cos y$}{$f(x,y) = x\sin y+ x+ C$, where $C$ is any constant.
}
\exercise{$f_x = x+y$,\quad $f_y = x+ y$}{$f(x,y) = \frac12x^2+xy+\frac12y^2+ C$, where $C$ is any constant.
}
\exercise{$f_x = 6xy-4y^2$,\quad $f_y = 3x^2-8xy+2$}{$f(x,y) = 3x^2y-4xy^2+2y +C$, where $C$ is any constant.
}
\exercise{$\ds f_x = \frac{2x}{x^2+y^2}$,\quad $\ds f_y = \frac{2y}{x^2+y^2}$}{$f(x,y) = \ln (x^2+y^2)+ C$, where $C$ is any constant.
}}

\input{exercises/12_03_exset_04}
