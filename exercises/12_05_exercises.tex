\printconcepts

\exercise{What is the difference between a directional derivative and a partial derivative?}{A partial derivative is essentially a special case of a directional derivative; it is the directional derivative in the direction of $x$ or $y$, i.e., $\bracket{1,0}$ or $\bracket{0,1}$.}

\exercise{For what $\vec u$ is $D_{\vec u\,} f = f_x$?}{$\vec u =\bracket{1,0}$}

\exercise{For what $\vec u$ is $D_{\vec u\,} f = f_y$?}{$\vec u =\bracket{0,1}$}

\exercise{The gradient is \underline{\hskip .5in} to level curves.}{orthogonal}

\exercise{The gradient points in the direction of \underline{\hskip .5in} increase.}{maximal, or greatest}

\exercise{It is generally more informative to view the directional derivative not as the result of a limit, but rather as the result of a \underline{\hskip .5in} product.}{dot}

\printproblems

% todo Tim make the directions more consistent.  maybe combine sets and make more parts?

\begin{exerciseset}{In Exercises}{, a function $z=f(x,y)$ is given. Find $\nabla f$.}

\exercise{$f(x,y) = -x^2y+xy^2+xy$\label{12_05_ex_07}}{$\nabla f =\bracket{-2xy+y^2+y, -x^2+2xy+x}$}

\exercise{$\ds f(x,y) = \sin x\cos y$}{$\nabla f =\bracket{\cos x\cos y, -\sin x\sin y}$}

\exercise{$\ds f(x,y) = \frac{1}{x^2+y^2+1}$}{$\nabla f =\bracket{\frac{-2x}{(x^2+y^2+1)^2}, \frac{-2y}{(x^2+y^2+1)^2}}$}

\exercise{$\ds f(x,y) = -4x+3y$\label{12_05_ex_10}}{$\nabla f =\bracket{-4,3}$}

\exercise{$\ds f(x,y) = x^2+2y^2-xy-7x$}{$\nabla f =\bracket{2x-y-7,4y-x}$}

\exercise{$\ds f(x,y) = x^2y^3-2x$\label{12_05_ex_22}}{$\nabla f =\bracket{2xy^3-2,3x^2y^2}$}

\end{exerciseset}


\exerciseset{In Exercises}{, a function $z=f(x,y)$ and a point $P$ are given. Find the directional derivative of $f$ in the indicated directions. Note: these are the same functions as in Exercises \ref{12_05_ex_07} -- \ref{12_05_ex_22}.
}{

\exercise{$f(x,y) = -x^2y+xy^2+xy$, $P= (2,1)$ \label{12_05_ex_11}
\begin{enumerate}
	\item In the direction of $\vec v = \la 3,4\ra$
	\item In the direction toward the point $Q = (1,-1)$.
\end{enumerate}
}{$\nabla f = \la -2xy+y^2+y, -x^2+2xy+x\ra$; $\nabla f(2,1) = \la -2,2\ra$.
Be sure to change all directions to unit vectors.
\begin{enumerate}
	\item $2/5$ ($\vec u = \la 3/5,4/5\ra$)
	\item	$-2\sqrt{5}$ ($\vec u = \la -1/\sqrt{5},-2\sqrt{5}\ra$)
\end{enumerate}
}
\exercise{$\ds f(x,y) = \sin x\cos y$, $\ds P = \left(\frac{\pi}{4},\frac{\pi}{3}\right)$
\begin{enumerate}
	\item In the direction of $\vec v=\la 1,1\ra$.
	\item	In the direction toward the point $Q = (0,0)$.
\end{enumerate}
}{$\nabla f = \la \cos x\cos y, -\sin x\sin y\ra$; $\nabla f(\frac{\pi}{4},\frac{\pi}{3}) = \la \frac{1}{2\sqrt{2}},-\frac12\sqrt{\frac32}\ra$. Be sure to change all directions to unit vectors. 
\begin{enumerate}
	\item $\frac14(1-\sqrt{3})$ ($\vec u = \la 1/\sqrt{2},1/\sqrt{2}\ra$)
	\item	$\frac{4\sqrt{3}-1}{10\sqrt{2}}$ ($\vec u = \la -3/5,-4/5\ra$)
\end{enumerate}
}
\exercise{$\ds f(x,y) = \frac{1}{x^2+y^2+1}$, $P = (1,1)$.
\begin{enumerate}
	\item In the direction of $\vec v = \la 1,-1\ra$.
	\item In the direction toward the point $Q = (-2,-2)$.
\end{enumerate}
}{$\nabla f = \la \frac{-2x}{(x^2+y^2+1)^2}, \frac{-2y}{(x^2+y^2+1)^2}\ra$; $\nabla f(1,1) = \la -2/9,-2/9\ra$. Be sure to change all directions to unit vectors.
\begin{enumerate}
	\item 0 ($\vec u = \la 1/\sqrt{2},-1/\sqrt{2}\ra$)
	\item	$2\sqrt{2}/9$ ($\vec u = \la -1/\sqrt{2},-1/\sqrt{2}\ra$)
\end{enumerate}
}
\exercise{$\ds f(x,y) = -4x+3y$, $P = (5,2)$\label{12_05_ex_14}
\begin{enumerate}
	\item In the direction of $\vec v = \la 3,1\ra.$
	\item In the direction toward the point $Q = (2,7)$.
\end{enumerate}
}{$\nabla f = \la -4,3\ra$; $\nabla f(5,2) = \la -4,3\ra$. Be sure to change all directions into unit vectors.
\begin{enumerate}
	\item $-9/\sqrt{10}$ ($\vec u = \la 3/\sqrt{10},1/\sqrt{10}\ra$)
	\item	$27/\sqrt{34}$ ($\vec u = \la -3/\sqrt{34},5/\sqrt{34}\ra$)
\end{enumerate}
}
\exercise{$\ds f(x,y) = x^2+2y^2-xy-7x$, $P = (4,1)$
\begin{enumerate}
	\item In the direction of $\vec v = \la -2,5\ra$
	\item	In the direction toward the point $Q = (4,0)$.
\end{enumerate}
}{$\nabla f = \la 2x-y-7,4y-x\ra$; $\nabla f(4,1) = \la 0,0\ra$.
\begin{enumerate}
	\item 0
	\item	0
\end{enumerate}
}
\exercise{$\ds f(x,y) = x^2y^3-2x$, $P = (1,1)$\label{12_05_ex_23}
\begin{enumerate}
	\item In the direction of $\vec v = \la 3,3\ra$
	\item	In the direction toward the point $Q = (1,2)$.
\end{enumerate}
}{$\nabla f = \la 2xy^3-2,3x^2y^2\ra$; $\nabla f(1,1) = \la 0,3\ra$ Be sure to change all directions to unit vectors.
	\begin{enumerate}
	\item	$3/\sqrt{2}$; ($\vec u = \la 1/\sqrt{2},1/\sqrt{2}\ra$)
	\item	$3$
\end{enumerate}
}}

\exerciseset{In Exercises}{, a function $z=f(x,y)$ and a point $P$ are given. 
\begin{enumerate}
	\item [(a)] Find the direction of maximal increase of $f$ at $P$.
	\item [(b)] What is the maximal value of $D_{\vec u\,}f$ at $P$?
	\item [(c)] Find the direction of minimal increase of $f$ at $P$.
	\item	[(d)] Give a direction $\vec u$ such that $D_{\vec u\,}f=0$ at $P$.
\end{enumerate}
Note: these are the same functions and points as in Exercises \ref{12_05_ex_11} through \ref{12_05_ex_23}.
}{

\exercise{$f(x,y) = -x^2y+xy^2+xy$, $P= (2,1)$
}{$\nabla f = \la -2xy+y^2+y, -x^2+2xy+x\ra$
\begin{enumerate}
	\item $\nabla f(2,1) = \la -2,2\ra$
	\item	$\norm{\nabla f(2,1)} = \norm{\la -2,2\ra} = \sqrt{8}$
	\item	$\la 2,-2\ra$
	\item	$\la 1/\sqrt{2},1/\sqrt{2}\ra$
\end{enumerate}
}
\exercise{$\ds f(x,y) = \sin x\cos y$, $\ds P = \left(\frac{\pi}{4},\frac{\pi}{3}\right)$
}{$\nabla f = \la \cos x\cos y, -\sin x\sin y\ra$
\begin{enumerate}
	\item $\nabla f(\frac{\pi}{4},\frac{\pi}{3}) = \la \frac{1}{2\sqrt{2}},-\frac12\sqrt{\frac32}\ra$
	\item	$\norm{\nabla f(\frac{\pi}{4},\frac{\pi}{3})} = \norm{\la \frac{1}{2\sqrt{2}},-\frac12\sqrt{\frac32}\ra} = 1/\sqrt{2} $
	\item $\la -\frac{1}{2\sqrt{2}},\frac12\sqrt{\frac32}\ra$
	\item $\la \frac12\sqrt{\frac32},\frac{1}{2\sqrt{2}}\ra$
\end{enumerate}
}
\exercise{$\ds f(x,y) = \frac{1}{x^2+y^2+1}$, $P = (1,1)$.
}{$\nabla f = \la \frac{-2x}{(x^2+y^2+1)^2}, \frac{-2y}{(x^2+y^2+1)^2}\ra$
\begin{enumerate}
	\item  $\nabla f(1,1) = \la -2/9,-2/9\ra$.
	\item		$\norm{\nabla f(1,1)} = \norm{\la -2/9,-2/9\ra}=2\sqrt{2}/9$
	\item		$\la 2/9,2/9\ra$
	\item		$\la 1/\sqrt{2},-1/\sqrt{2}\ra$
\end{enumerate}
}
\exercise{$\ds f(x,y) = -4x+3y$, $P = (5,4)$.
}{$\nabla f = \la -4,3\ra$
\begin{enumerate}
	\item  $\nabla f(5,4) = \la -4,3\ra$.
	\item		$\norm{\nabla f(5,4)} = \norm{\la -4,3\ra}=5$
	\item		$\la 4,-3\ra$
	\item		$\la 3/5,4/5\ra$
\end{enumerate}
}
\exercise{$\ds f(x,y) = x^2+2y^2-xy-7x$, $P = (4,1)$
}{$\nabla f = \la 2x-y-7,4y-x\ra$
\begin{enumerate}
	\item $\nabla f(4,1) = \la 0,0\ra$
	\item	0
	\item	$\la 0,0\ra$
	\item	All directions give a directional derivative of 0.
\end{enumerate}
}
\exercise{$\ds f(x,y) = x^2y^3-2x$, $P = (1,1)$
}{$\nabla f = \la 2xy^3-2,3x^2y^2\ra$
\begin{enumerate}
	\item $\nabla f(1,1) = \la 0,3\ra$
	\item	3
	\item	$\la 0,-3\ra$
	\item	$\vec u = \la 1,0\ra$
\end{enumerate}
}}

\exerciseset{In Exercises}{, a function $w=F(x,y,z)$, a vector $\vec v$ and a point $P$ are given. 
\begin{enumerate}
	\item [(a)] Find $\nabla F(x,y,z)$.
	\item [(b)] Find $D_{\vec u\,}F$ at $P$.
\end{enumerate}
}{

\exercise{$\ds F(x,y,z) = 3x^2z^3+4xy-3z^2$, $\vec v = \la 1,1,1\ra$, $P = (3,2,1)$
}{
\begin{enumerate}
	\item $\nabla F(x,y,z) = \la 6xz^3+4y, 4x, 9x^2z^2-6z\ra$
	\item	$113/\sqrt{3}$
\end{enumerate}
}
\exercise{$\ds F(x,y,z) = \sin(x)\cos(y)e^z$, $\vec v = \la 2,2,1\ra$, $P = (0,0,0)$
}{
\begin{enumerate}
	\item $\nabla F(x,y,z) = \la \cos x\cos ye^z, -\sin x\sin ye^z, \sin x\cos ye^z\ra$
	\item	$2/3$
\end{enumerate}
}
\exercise{$\ds F(x,y,z) = x^2y^2-y^2z^2$, $\vec v = \la -1,7,3\ra$, $P = (1,0,-1)$
}{
\begin{enumerate}
	\item $\nabla F(x,y,z) = \la 2xy^2, 2y(x^2-z^2), -2y^2z\ra$
	\item	$0$
\end{enumerate}
}
\exercise{$\ds F(x,y,z) = \frac{2}{x^2+y^2+z^2}$, $\vec v = \la 1,1,-2\ra$, $P = (1,1,1)$
}{
\begin{enumerate}
	\item $\nabla F(x,y,z) = \la -\frac{4x}{(x^2+y^2+z^2)^2},-\frac{4y}{(x^2+y^2+z^2)^2},-\frac{4z}{(x^2+y^2+z^2)^2}\ra$
	\item	$0$
\end{enumerate}
}}

\begin{exerciseset}{In Exercises}{, a function $w=F(x,y,z)$ and a point $P$ are give. Find the direction of maximum increase and the maximal value of the directional derivative for $F$ at $P$}

\exercise{$F(x,y,z)=x^2+y^2-z^2+xyz$ at $P=(2,1,3)$}{In the direction $\bracket{7,8,-4}$ with maximal value $\sqrt{129}$.}

\exercise{$F(x,y,z)=\dfrac1{x^2+2y^2+3z^2}$ at $P=(1,0,1)$}{In the direction $\bracket{-1,0,-3}$ with maximal value $\frac{\sqrt{10}}8$.}

\end{exerciseset}
