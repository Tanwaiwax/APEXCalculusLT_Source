\printconcepts
\exercise{Explain how the vector $\vec v=\la 1,0,3\ra$ can be thought of as having a ``slope'' of 3.
}{
Answers will vary. The displacement of the vector is one unit in the $x$-direction and 3 units in the $z$-direction, with no change in $y$. Thus along a line parallel to $\vec v$, the change in $z$ is 3 times the change in $x$ -- i.e., a ``slope'' of 3. Specifically, the line in the $x$-$z$ plane parallel to $z$ has a slope of 3.
}
\exercise{Explain how the vector $\vec v=\la 0.6,0.8, -2\ra$ can be thought of as having a ``slope'' of $-2$.
}{
Answers will vary. Let $\vec u = \la 0.6,0.8\ra$; this is a unit vector. The displacement of the vector is one unit in the $\vec u$-direction and $-2$ units in the $z$-direction. In the plane containing the $z$-axis and the vector $\vec u$, the line parallel to $\vec v$ has slope $-2$.
}
\exercise{T/F: Let $z=f(x,y)$ be differentiable at $P$. If $\vec n$ is a normal vector to the tangent plane of $f$ at $P$, then $\vec n$ is orthogonal to $f_x$ and $f_y$ at $P$.
}{T
}
\exercise{Explain in your own words why we do not refer to \textit{the} tangent line to a surface at a point, but rather to \textit{directional} tangent line\textit{s} to a surface at a point.
}{On a surface through a point, there are many different smooth curves, each with a tangent line at the point. Each of these tangent lines is also ``tangent'' to the surface. There is not just one tangent line, but many, each in a different direction. Therefore we refer to directional tangent lines, not just \textit{the} tangent line.
}
\printproblems
\exerciseset{In Exercises}{, a function $z=f(x,y)$, a vector $\vec v$ and a point $P$ are given. Give the parametric equations of the following directional tangent lines to $f$ at $P$:
\begin{enumerate}
	\item [(a)] $\ell_x(t)$
	\item [(b)] $\ell_y(t)$
	\item [(c)] $\ell_{\vec u\,}(t)$, where $\vec u$ is the unit vector in the direction of $\vec v$.
\end{enumerate}
}{

\exercise{$f(x,y) = 2x^2y-4xy^2$,  $\vec v = \la 1,3\ra$, $P=(2,3)$.\label{12_06_ex_05}
}{\begin{enumerate}
	\item $\ell_x(t) = \left\{\begin{array}{l} x=2+t\\ y=3 \\ z = -48-12t\end{array}\right.$
	
	\item $\ell_y(t) = \left\{\begin{array}{l} x=2\\ y=3+t \\ z = -48-40t\end{array}\right.$
	
	\item $\ell_{\vec u\,}(t) = \left\{\begin{array}{l} x=2+t/\sqrt{10}\\ y=3+3t/\sqrt{10} \\ z = -48-66\sqrt{2/5}t\end{array}\right.$
\end{enumerate}
}
\exercise{$f(x,y) = 3\cos x\sin y$,  $\vec v = \la 1,2\ra$, $P=(\pi/3, \pi/6)$.
}{
\begin{enumerate}
	\item $\ell_x(t) = \left\{\begin{array}{l} x = \pi/3+t\\ y = \pi/6 \\ z = 3/4 -\frac{3\sqrt{3}}{4}t \end{array} \right.$ 
	
	\item $\ell_y(t) = \left\{\begin{array}{l} x = \pi/3\\ y = \pi/6 +t\\ z = 3/4 +\frac{3\sqrt{3}}{4}t \end{array} \right.$
	
	\item $\ell_{\vec u\,}(t) = \left\{\begin{array}{l} x = \pi/3+t/\sqrt{5}\\ y = \pi/6+2t/\sqrt{5} \\ z = 3/4 +\frac{3\sqrt{3/5}}{4}t \end{array} \right.$
\end{enumerate}
}
\exercise{$f(x,y) = 3x-5y$,  $\vec v = \la 1,1\ra$, $P=(4,2)$.
}{
\begin{enumerate}
	\item $\ell_x(t) = \left\{\begin{array}{l} x = 4+t\\ y = 2 \\ z = 2 + 3t \end{array} \right.$ 
	
	\item $\ell_y(t) = \left\{\begin{array}{l} x = 4\\ y = 2+t\\ z = 2-5t \end{array} \right.$
	
	\item $\ell_{\vec u\,}(t) = \left\{\begin{array}{l} x = 4+t/\sqrt{2}\\ y = 2+t/\sqrt{2} \\ z = 2 -\sqrt{2}t \end{array} \right.$
\end{enumerate}
}
\exercise{$f(x,y) = x^2-2x-y^2+4y$,  $\vec v = \la 1,1\ra$, $P=(1,2)$.\label{12_06_ex_08}
}{
\begin{enumerate}
	\item $\ell_x(t) = \left\{\begin{array}{l} x = 1+t\\ y = 2 \\ z = 3 \end{array} \right.$ 
	
	\item $\ell_y(t) = \left\{\begin{array}{l} x = 1\\ y = 2+t\\ z = 3 \end{array} \right.$
	
	\item $\ell_{\vec u\,}(t) = \left\{\begin{array}{l} x = 1+t/\sqrt{2}\\ y = 2+t/\sqrt{2} \\ z = 3 \end{array} \right.$
\end{enumerate}
}}
\exerciseset{In Exercises}{, a function $z=f(x,y)$ and a point $P$ are given. Find the equation of the normal line to the graph of $f$ at $P$. Note: these are the same functions as in  Exercises~\ref{12_06_ex_05}--\ref{12_06_ex_08}.}{

\exercise{$f(x,y) = 2x^2y-4xy^2$,   $P=(2,3)$.\label{12_06_ex_09}}{$\ell_{\vec n}(t) = \begin{cases}x=2-12t\\ y=3-40t \\ z = -48-t\end{cases}$}

\exercise{$f(x,y) = 3\cos x\sin y$,   $P=(\pi/3, \pi/6)$.}{$\ell_{\vec n}(t) = \begin{cases}x = \pi/3-\frac{3\sqrt{3}}{4}t\\ y = \pi/6+\frac{3\sqrt{3}}{4}t \\ z = 3/4 -t \end{cases}$ }

\exercise{$f(x,y) = 3x-5y$,   $P=(4,2)$.}{$\ell_{\vec n}(t) = \begin{cases}x = 4+3t\\ y = 2-5t \\ z = 2 -t \end{cases}$ }

\exercise{$f(x,y) = x^2-2x-y^2+4y$,   $P=(1,2)$.}{$\ell_{\vec n}(t) = \begin{cases}x = 1\\ y = 2 \\ z = 3 -t \end{cases}$ }

}

\input{exercises/12_06_exset_03}
\exerciseset{In Exercises}{, a function $z=f(x,y)$ and  a point $P$ are given. Find the equation of the tangent plane to $f$ at $P$. Note: these are the same functions as in  Exercises~\ref{12_06_ex_05}--\ref{12_06_ex_08}.}{

\exercise{$f(x,y) = 2x^2y-4xy^2$, $P=(2,3)$.\label{12_06_ex_17}}{$-12(x-2)-40(y-3) -(z+48) = 0$}

\exercise{$f(x,y) = 3\cos x\sin y$, $P=(\pi/3, \pi/6)$.}{$-\frac{3\sqrt{3}}4(x-\pi/3)+\frac{3\sqrt{3}}4(y-\pi/6) - (z-3/4) = 0$}

\exercise{$f(x,y) = 3x-5y$, $P=(4,2)$.}{$3(x-4)-5(y-2) - (z-2) = 0$ (Note that this tangent plane is the same as the original function, a plane.)}

\exercise{$f(x,y) = x^2-2x-y^2+4y$, $P=(1,2)$.}{$- (z-3) = 0$, or $z=3$}

}

\exerciseset{In Exercises}{, an implicitly defined function of $x$, $y$ and $z$ is given along with a point $P$ that lies on the surface. Use the gradient $\nabla F$ to:
\begin{enumerate}[label=(\alph*)]
	\item find the equation of the normal line to the surface at $P$, and
	\item find the equation of the plane tangent  to the surface at $P$.
\end{enumerate}}{

\exercise{$\ds \frac{x^2}{8}+\frac{y^2}4+\frac{z^2}{16}=1$, at $P = (1,\sqrt{2},\sqrt{6})$}{$\nabla F =\bracket{x/4, y/2, z/8}$; at $P$, $\nabla F =\bracket{1/4, \sqrt{2}/2, \sqrt{6}/8}$
\begin{enumerate}
	\item $\ell_{\vec n}(t) = \begin{cases}x= 1+ t/4 \\ y = \sqrt{2}+ \sqrt{2}t/2\\ z = \sqrt{6} + \sqrt{6}t/8 \end{cases}$
	\item		$\frac14(x-1) + \frac{\sqrt{2}}{2}(y-\sqrt{2}) + \frac{\sqrt{6}}8(z-\sqrt{6}) = 0$.
\end{enumerate}}

\exercise{$\ds z^2-\frac{x^2}{4} - \frac{y^2}9=0$, at $P = (4,-3,\sqrt{5})$}{$\nabla F =\bracket{-\frac x2, -\frac{2y}9,2z}$; at $P$, $\nabla F =\bracket{-2, 2/3, 2\sqrt{5}}$
\begin{enumerate}
	\item $\ell_{\vec n}(t) = \begin{cases}x= 4-2t \\ y = -3+ 2t/3\\ z = \sqrt{5} + 2\sqrt{5}t \end{cases}$
	\item		$-2(x-4) + \frac23(y+3) + 2\sqrt{5}(z-\sqrt{5}) = 0$.
\end{enumerate}}

\exercise{$\ds xy^2-xz^2=0$, at $P = (2,1,-1)$}{$\nabla F =\bracket{y^2-z^2, 2xy, -2xz}$; at $P$, $\nabla F =\bracket{0, 4, 4}$
\begin{enumerate}
	\item $\ell_{\vec n}(t) = \begin{cases}x= 2 \\ y = 1+4t\\ z = -1+4t \end{cases}$
	\item		$4(y-1) + 4(z+1) = 0$.
\end{enumerate}}

\exercise{$\ds \sin(xy)+\cos(yz)=1$, at $P = (2, \pi/12, 4)$}{$\nabla F =\bracket{y\cos(xy), x\cos(xy)-z\sin(yz), -y\sin(yz)}$; at $P$, $\nabla F =\bracket{\frac{\pi}{8\sqrt{3}}, -\sqrt{3},-\frac{\pi}{8\sqrt{3}}}$
\begin{enumerate}
	\item $\ell_{\vec n}(t) = \begin{cases}x= 2 +\frac{\pi}{8\sqrt{3}}t\\ y = \frac{\pi}{12}-\sqrt{3}t\\ z = 4-\frac{\pi}{8\sqrt{3}}t \end{cases}$
	\item		$\frac{\pi}{8\sqrt{3}}(x-2) -\sqrt{3}(y-\frac{\pi}{12}) -\frac{\pi}{8\sqrt{3}}(z-4) = 0$.
\end{enumerate}}

}
