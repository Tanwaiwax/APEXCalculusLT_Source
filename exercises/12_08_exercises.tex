\printconcepts

\exercise{Let a level curve of $z=f(x,y)$ be described by $x=g(t)$, $y = h(t)$. Explain why $\frac{dz}{dt}=0$.}{Because the parametric equations describe a level curve, $z$ is constant for all $t$. Therefore $\frac{dz}{dt}=0$.}

\exercise{Fill in the blank: The single variable Chain Rule states $\ds\frac{d}{dx}\Big(f\big(g(x)\big)\Big) = \fp\big(g(x)\big)\cdot$\underline{\hskip.5in}.}{$g'(x)$}

\exercise{Fill in the blank: The Multivariable Chain Rule states \\[5pt]
$\ds\frac{df}{dt}=\frac{\partial f}{\partial x}\cdot\text{\underline{\hskip.5in}}+\text{\underline{\hskip.5in}}\cdot\frac{dy}{dt}$.}{$\frac{dx}{dt}$, and $\frac{\partial f}{\partial y}$}

\exercise{If $z=f(x,y)$, where $x=g(t)$ and $y=h(t)$, we can substitute and write $z$ as an explicit function of $t$.\\
T/F: Using the Multivariable Chain Rule to find $\frac{dz}{dt}$ is  sometimes easier than  first substituting and then taking the derivative.}{T}

\exercise{T/F: The  Multivariable Chain Rule is only useful when all the related functions are known explicitly.}{F}

\exercise{The Multivariable Chain Rule allows us to compute implicit derivatives easily by just computing two \underline{\hskip.5in} derivatives.}{partial}

\printproblems

\exerciseset{In Exercises} 
{,  functions $z=f(x,y)$, $x=g(t)$ and $y=h(t)$ are given.
\begin{enumerate}
	\item[(a)] Use the Multivariable Chain Rule to compute $\ds \frac{dz}{dt}$.
	\item[(b)] Evaluate $\ds \frac{dz}{dt}$ at the indicated $t$-value.
\end{enumerate}
}{

\exercise{$z=3x+4y$,\qquad $x=t^2$,\qquad $y=2t$;\qquad $t=1$\label{12_08_ex_07}}{\begin{enumerate}
	\item $\frac{dz}{dt} = 3(2t)+4(2) = 6t+8$.
	\item At $t=1$, $\frac{dz}{dt} = 14$.
\end{enumerate}}

\exercise{$\ds z=x^2-y^2$,\qquad $x=t$,\qquad $y=t^2-1$;\qquad $t=1$
}{\begin{enumerate}
	\item $\frac{dz}{dt} = 2x(1)-2y(2t) = 2x-4yt$
	
	\item		At $t=1$, $x=1$, $y=0$ and $\frac{dz}{dt} = 2$.
\end{enumerate}
}

\exercise{$\ds z=5x+2y$,\qquad $x=2\cos t+1$,\qquad $y=\sin t-3$;\qquad $t=\pi/4$
}{\begin{enumerate}
	\item $\frac{dz}{dt} = 5(-2\sin t)+2(\cos t) = -10\sin t+2\cos t$
	
	\item		At $t=\pi/4$, $\frac{dz}{dt} = -4\sqrt{2}$.
\end{enumerate}
}

\exercise{$\ds z=\frac{x}{y^2+1}$,\qquad $x=\cos t$,\qquad $y=\sin t$;\qquad $t=\pi/2$}{\begin{enumerate}
	\item $\ds\frac{dz}{dt} = \frac{1}{1+y^2}(-\sin t)-\frac{2xy}{(y^2+1)^2}(\cos t)$.
	\item At $t=\pi/2$, $x=0$, $y=1$, and $\frac{dz}{dt} = -1/2$.
\end{enumerate}}

\exercise{$\ds z=x^2+2y^2$,\qquad $x=\sin t$,\qquad $y=3\sin t$;\qquad $t=\pi/4$\label{12_08_ex_10}}{\begin{enumerate}
	\item $\ds\frac{dz}{dt} = 2x(\cos t) + 4y(3\cos t)$.
	\item At $t=\pi/4$, $x=\sqrt{2}/2$, $y=3\sqrt{2}/2$, and $\frac{dz}{dt} = 19$.
\end{enumerate}}

\exercise{$z=\cos x\sin y$,\qquad $x=\pi t$,\qquad $y=2\pi t+\pi/2$;\qquad $t=3$\label{12_08_ex_08}}{\begin{enumerate}
	\item $\frac{dz}{dt} = -\sin x\sin y(\pi) + \cos x\cos y(2\pi)$.
	\item At $t=3$, $x=3\pi$, $y=13\pi/2$, and $\frac{dz}{dt} = 0$.
\end{enumerate}}
}

\exerciseset{In Exercises}{,  functions $z=f(x,y)$, $x=g(t)$ and $y=h(t)$ are given. Find the values of $t$ where $\frac{dz}{dt}=0$. Note: these are the same surfaces/curves as found in Exercises~\ref{12_08_ex_07}--\ref{12_08_ex_08}.}{

\exercise{$\ds z=3x+4y$,\qquad $x=t^2$,\qquad $y=2t$}{$t=-4/3$; this corresponds to a minimum}

\exercise{$\ds z=x^2-y^2$,\qquad $x=t$,\qquad $y=t^2-1$}{$t=0, \pm\sqrt{3/2}$}

\exercise{$\ds z=5x+2y$,\qquad $x=2\cos t+1$,\qquad $y=\sin t-3$}{$t=\tan^{-1}(1/5) +n\pi$, where $n$ is an integer}

\exercise{$\ds z=\frac{x}{y^2+1}$,\qquad $x=\cos t$,\qquad $y=\sin t$}{We find that
\[\frac{dz}{dt} = -\frac{2\cos^2t\sin t}{(1+\sin^2t)^2}-\frac{\sin t}{1+\sin^2t}.\]
Setting this equal to 0, finding a common denominator and factoring out $\sin t$, we get
\[\sin t\left(\frac{2\cos^2t+\sin^2t+1}{(1+\sin^2t)^2}\right)=0.\]
We have $\sin t= 0$ when $t = \pi n$, where $n$ is an integer. The expression in the parenthesis above is always positive, and hence never equal 0. So all solutions are 
$t=\pi n$, n is an integer.}

\exercise{$\ds z=x^2+2y^2$,\qquad $x=\sin t$,\qquad $y=3\sin t$}{We find that
\[\frac{dz}{dt} = 38\cos t\sin t.\]
Thus $\frac{dz}{dt} = 0$ when $t=\pi n$ or $\pi n+\pi/2$, where $n$ is any integer.}

\exercise{$\ds z=\cos x\sin y$,\qquad $x=\pi t$,\qquad $y=2\pi t+\pi/2$}{We find that
\[\frac{dz}{dt} = -\pi\sin(\pi t)\sin(2\pi t+\pi/2)+2\pi\cos(\pi t)\cos(2\pi t+\pi/2).\]

One can ``easily'' see that when $t$ is an integer, $\sin(\pi t) =0$ and $\cos(2\pi t+\pi/2)=0$, hence $\frac{dz}{dt}=0$ when $t$ is an integer. There are other places where $z$ has a relative max/min that require more work. First, verify that $\sin(2\pi t+\pi/2) = \cos(2\pi t)$, and $\cos(2\pi t+\pi/2) = -\sin(2\pi t)$. This lets us rewrite $\frac{dz}{dt} = 0$ as
\[-\sin(\pi t)\cos(2\pi t)-2\cos(\pi t)\sin(2\pi t)=0.\]
By bringing one term to the other side of the equality then dividing, we can get
\[2\tan(2\pi t) = -\tan(\pi t).\]
Using the angle sum/difference formulas found in the back of the book, we know 
\[\tan(2\pi t) = \tan(\pi t)+\tan(\pi t) = \frac{\tan(\pi t)+\tan(\pi t)}{1-\tan^2(\pi t)}.\]
Thus we write
\[2\frac{\tan(\pi t)+\tan(\pi t)}{1-\tan^2(\pi t)} = -\tan(\pi t).\]
Solving for $\tan^2(\pi t)$, we find
\[\tan^2(\pi t) = 5 \quad \Rightarrow \quad \tan(\pi t) = \pm\sqrt{5},\]
and so
\[\pi t = \tan^{-1}(\pm\sqrt{5}) = \pm\tan^{-1}(\sqrt{5}).\]
Since the period of tangent is $\pi$, we can adjust our answer to be
\[\pi t = \pm\tan^{-1}(\sqrt{5})+ n\pi,\text{ where $n$ is an integer.}\]
Dividing by $\pi$, we find 
\[t = \pm\frac1\pi\tan^{-1}(\sqrt{5})+ n,\text{ where $n$ is an integer.}\]}

}


\input{exercises/12_08_exset_03}

\exerciseset{In Exercises} 
{,  find $\ds \frac{dy}{dx}$ using Implicit Differentiation and \autoref{thm:implicit_deriv_chain}.
}{

\exercise{$x^2\tan y = 50$}{$f_x = 2x\tan y$, $f_y = x^2\sec^2y$;\\
$\ds\frac{dy}{dx} = -\frac{2\tan y}{x\sec^2y}$
}

\exercise{$(3x^2+2y^3)^4=2$}{$f_x = 4(3x^2+2y^3)^3(6x)$, $f_y = 4(3x^2+2y^3)^3(6y^2)$;\\
$\ds\frac{dy}{dx} = -\frac{x}{y^2}$
}

\exercise{$\ds \frac{x^2+y}{x+y^2}=17$}{$\ds f_x = \frac{(x+y^2)(2x)-(x^2+y)(1)}{(x+y^2)^2}$, $\ds f_y = \frac{(x+y^2)(1)-(x^2+y)(2y)}{(x+y^2)^2}$;\\
$\ds\frac{dy}{dx} = -\frac{2x(x+y^2)-(x^2+y)}{x+y^2-2y(x^2+y)}$
}

\exercise{$\ds \ln(x^2+xy+y^2)=1$}{$\ds f_x = \frac{2x+y}{x^2+xy+y^2}$, $\ds f_y = \frac{x+2y}{x^2+xy+y^2}$;\\
$\ds\frac{dy}{dx} = -\frac{2x+y}{2y+x}$
}
}

\begin{exerciseset}{In Exercises} 
{,  find $\ds \frac{dz}{dt}$, or $\ds \frac{\partial z}{\partial s}$ and $\ds \frac{\partial z}{\partial t}$, using the supplied information. }

\exercise{$\ds\frac{\partial z}{\partial x} = 2$,\quad $\ds\frac{\partial z}{\partial y} = 1$,\quad $\ds\frac{dx}{dt} = 4$,\quad $\ds\frac{dy}{dt} = -5$}{$\frac{dz}{dt} = 2(4)+1(-5) = 3$.}

\exercise{$\ds\frac{\partial z}{\partial x} = 1$,\quad $\ds\frac{\partial z}{\partial y} = -3$,\quad $\ds\frac{dx}{dt} = 6$,\quad $\ds\frac{dy}{dt} = 2$}{$\frac{dz}{dt} = 1(6)+(-3)(2) = 0$.}

\exercise{$\ds\frac{\partial z}{\partial x} = -4$,\quad $\ds\frac{\partial z}{\partial y} = 9$,

$\ds\frac{\partial x}{\partial s} = 5$,\quad $\ds\frac{\partial x}{\partial t} = 7$,\quad $\ds \frac{\partial y}{\partial s} = -2$,\quad $\ds \frac{\partial y}{\partial t} = 6$}{$\frac{\partial z}{\partial s} = -4(5)+9(-2) = -38$, 

$\frac{\partial z}{\partial t} = -4(7)+9(6) = 26$.}

\exercise{$\ds\frac{\partial z}{\partial x} = 2$,\quad $\ds\frac{\partial z}{\partial y} = 1$,

$\ds\frac{\partial x}{\partial s} = -2$,\quad $\ds\frac{\partial x}{\partial t} = 3$,\quad $\ds \frac{\partial y}{\partial s} = 2$,\quad $\ds \frac{\partial y}{\partial t} = -1$}{$\frac{\partial z}{\partial s} = 2(-2)+1(2) = -2$, 

$\frac{\partial z}{\partial t} = 2(3)+1(-1) = 5$.}

\end{exerciseset}


\exercise{Suppose $z=f(x,y)$ is differentiable.  Express $(x,y)$ in polar coordinates as $x=r\cos\theta$, $y=r\sin\theta$.  Calculate $\dfrac{\partial z}{\partial\theta}$.}{$z_\theta=-f_x(x,y)r\sin\theta+f_y(x,y)r\cos\theta$}

\exercise{Suppose $w=g(x,y,z)$ is differentiable.  Express $(x,y,z)$ in spherical coordinates as $x=\rho\sin\phi\cos\theta$, $y=\rho\sin\phi\sin\theta$, $z=\rho\cos\phi$.  Calculate $\dfrac{\partial w}{\partial\phi}$.}{$w_\phi=f_x(x,y,z)\rho\cos\phi\cos\theta+f_y(x,y,z)\rho\cos\phi\sin\theta-f_z(x,y,z)\rho\sin\phi$}

\exercise{Suppose the radius of a circular cylinder is increasing at the constant rate of $1/2$ cm/sec while its height is decreasing at the rate of $1/3$ cm/sec.  How is the volume of the cylinder changing when the radius $4$ cm and the height is $10$ cm?}{It is increasing at $104\pi/3$ cm\textsuperscript3/sec}
