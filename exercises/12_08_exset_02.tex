\exerciseset{In Exercises}{,  functions $z=f(x,y)$, $x=g(t)$ and $y=h(t)$ are given. Find the values of $t$ where $\frac{dz}{dt}=0$. Note: these are the same surfaces/curves as found in Exercises~\ref{12_08_ex_07}--\ref{12_08_ex_08}.}{

\exercise{$\ds z=3x+4y$,\qquad $x=t^2$,\qquad $y=2t$}{$t=-4/3$; this corresponds to a minimum}

\exercise{$\ds z=x^2-y^2$,\qquad $x=t$,\qquad $y=t^2-1$}{$t=0, \pm\sqrt{3/2}$}

\exercise{$\ds z=5x+2y$,\qquad $x=2\cos t+1$,\qquad $y=\sin t-3$}{$t=\tan^{-1}(1/5) +n\pi$, where $n$ is an integer}

\exercise{$\ds z=\frac{x}{y^2+1}$,\qquad $x=\cos t$,\qquad $y=\sin t$}{We find that
\[\frac{dz}{dt} = -\frac{2\cos^2t\sin t}{(1+\sin^2t)^2}-\frac{\sin t}{1+\sin^2t}.\]
Setting this equal to 0, finding a common denominator and factoring out $\sin t$, we get
\[\sin t\left(\frac{2\cos^2t+\sin^2t+1}{(1+\sin^2t)^2}\right)=0.\]
We have $\sin t= 0$ when $t = \pi n$, where $n$ is an integer. The expression in the parenthesis above is always positive, and hence never equal 0. So all solutions are 
$t=\pi n$, n is an integer.}

\exercise{$\ds z=x^2+2y^2$,\qquad $x=\sin t$,\qquad $y=3\sin t$}{We find that
\[\frac{dz}{dt} = 38\cos t\sin t.\]
Thus $\frac{dz}{dt} = 0$ when $t=\pi n$ or $\pi n+\pi/2$, where $n$ is any integer.}

\exercise{$\ds z=\cos x\sin y$,\qquad $x=\pi t$,\qquad $y=2\pi t+\pi/2$}{We find that
\[\frac{dz}{dt} = -\pi\sin(\pi t)\sin(2\pi t+\pi/2)+2\pi\cos(\pi t)\cos(2\pi t+\pi/2).\]

One can ``easily'' see that when $t$ is an integer, $\sin(\pi t) =0$ and $\cos(2\pi t+\pi/2)=0$, hence $\frac{dz}{dt}=0$ when $t$ is an integer. There are other places where $z$ has a relative max/min that require more work. First, verify that $\sin(2\pi t+\pi/2) = \cos(2\pi t)$, and $\cos(2\pi t+\pi/2) = -\sin(2\pi t)$. This lets us rewrite $\frac{dz}{dt} = 0$ as
\[-\sin(\pi t)\cos(2\pi t)-2\cos(\pi t)\sin(2\pi t)=0.\]
By bringing one term to the other side of the equality then dividing, we can get
\[2\tan(2\pi t) = -\tan(\pi t).\]
Using the angle sum/difference formulas found in the back of the book, we know 
\[\tan(2\pi t) = \tan(\pi t)+\tan(\pi t) = \frac{\tan(\pi t)+\tan(\pi t)}{1-\tan^2(\pi t)}.\]
Thus we write
\[2\frac{\tan(\pi t)+\tan(\pi t)}{1-\tan^2(\pi t)} = -\tan(\pi t).\]
Solving for $\tan^2(\pi t)$, we find
\[\tan^2(\pi t) = 5 \quad \Rightarrow \quad \tan(\pi t) = \pm\sqrt{5},\]
and so
\[\pi t = \tan^{-1}(\pm\sqrt{5}) = \pm\tan^{-1}(\sqrt{5}).\]
Since the period of tangent is $\pi$, we can adjust our answer to be
\[\pi t = \pm\tan^{-1}(\sqrt{5})+ n\pi,\text{ where $n$ is an integer.}\]
Dividing by $\pi$, we find 
\[t = \pm\frac1\pi\tan^{-1}(\sqrt{5})+ n,\text{ where $n$ is an integer.}\]}

}
