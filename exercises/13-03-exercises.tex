\printconcepts

\exercise{When evaluating $\ds\iint_R f(x,y)\dd A$ using polar coordinates, $f(x,y)$ is replaced with \underline{\hskip .5in} and $\dd A$ is replaced with \underline{\hskip.5in}.}{$f\big(r\cos\theta,r\sin\theta\big)$, $r\dd r\dd\theta$}

\exercise{Why would one be interested in evaluating a double integral with polar coordinates?}{Some regions in the $x$-$y$ plane are easier to describe using polar coordinates than using rectangular coordinates. Also, some integrals are easier to evaluate one the polar substitutions have been made.}

\printproblems

\begin{exerciseset}{In Exercises}{, a function $f(x,y)$ is given and a region $R$ of the $x$-$y$ plane is described. Set up and evaluate $\iint_Rf(x,y)\dd A$ using polar coordinates.}

\exercise{$f(x,y) = 3x-y+4$; $R$ is the region enclosed by the circle $x^2+y^2=1$.}{$\ds \int_0^{2\pi}\int_0^1 \big(3r\cos\theta-r\sin\theta+4\big)r\dd r\dd \theta = 4\pi$}

\exercise{$f(x,y) = 4x+4y$; $R$ is the region enclosed by the circle $x^2+y^2=4$.}{$\ds \int_0^{2\pi}\int_0^2 \big(4r\cos\theta+4r\sin\theta\big)r\dd r\dd \theta = 0$}

\exercise{$f(x,y) = 8-y$; $R$ is the region enclosed by the circles with polar equations $r=\cos\theta$ and $r=3\cos\theta$.}{$\ds \int_0^{\pi}\int_{\cos\theta}^{3\cos\theta} \big(8-r\sin\theta\big)r\dd r\dd \theta = 16\pi$}

\exercise{$f(x,y) = 4$; $R$ is the region enclosed by the petal of the rose curve $r=\sin(2\theta)$ in the first quadrant.}{$\ds \int_0^{\pi/2}\int_{0}^{\sin(2\theta)} \big(4\big)r\dd r\dd \theta = \pi/2$}

\exercise{$f(x,y) = \ln\big(x^2+y^2)$; $R$ is the annulus enclosed by the circles $x^2+y^2=1$ and $x^2+y^2=4$.}{$\ds \int_0^{2\pi}\int_{1}^{2} \big(\ln(r^2)\big)r\dd r\dd \theta = 2\pi\big(\ln16-3/2\big)$}

\exercise{$f(x,y) = 1-x^2-y^2$; $R$ is the region enclosed by the circle $x^2+y^2=1$.}{$\ds \int_0^{2\pi}\int_{0}^{1} \big(1-r^2\big)r\dd r\dd \theta = \pi/2$}

\exercise{$f(x,y) = x^2-y^2$; $R$ is the region enclosed by the circle $x^2+y^2=36$ in the first and fourth quadrants.}{$\ds \int_{-\pi/2}^{\pi/2}\int_{0}^{6} \big(r^2\cos^2\theta-r^2\sin^2\theta\big)r\dd r\dd \theta
= \int_{-\pi/2}^{\pi/2}\int_{0}^{6} \big(r^2\cos(2\theta)\big)r\dd r\dd \theta= 0$}

\exercise{$f(x,y) = (x-y)/(x+y)$; $R$ is the region enclosed by the lines $y=x$, $y=0$ and the circle $x^2+y^2=1$ in the first quadrant.}{$\ds \int_{0}^{\pi/4}\int_{0}^{1} \left(\frac{\cos\theta-\sin\theta}{\cos\theta+\sin\theta}\right)r\dd r\dd \theta = \ln 2$}

\end{exerciseset}


\exercisesetinstructions{, an iterated integral in rectangular coordinates is given. Rewrite the integral using polar coordinates and evaluate the new double integral.}

\exercise{$\ds \int_{0}^{5}\int_{-\sqrt{25-x^2}}^{\sqrt{25-x^2}} \sqrt{x^2+y^2}\dd y\dd x$}{$\ds \int_{-\pi/2}^{\pi/2}\int_{0}^{5} \big(r^2\big)\dd r\dd \theta=125\pi/3$}

\exercise{$\ds \int_{-4}^{4}\int_{-\sqrt{16-y^2}}^{0} \big(2y-x\big)\dd x\dd y$}{$\ds \int_{\pi/2}^{3\pi/2}\int_{0}^{4} \big(2r\sin\theta-r\cos\theta\big)r\dd r\dd \theta=128/3$}

\exercise{$\ds \int_{0}^{2}\int_{y}^{\sqrt{8-y^2}} \big(x+y\big)\dd x\dd y$}{$\ds \int_{0}^{\pi/4}\int_{0}^{\sqrt8} \big(r\cos\theta+r\sin\theta\big)r\dd r\dd \theta=16\sqrt2/3$}

\exercise{$\ds \int_{-2}^{-1}\int_{0}^{\sqrt{4-x^2}} \big(x+5\big)\dd y\dd x+\int_{-1}^{1}\int_{\sqrt{1-x^2}}^{\sqrt{4-x^2}} \big(x+5\big)\dd y\dd x +\int_{1}^{2}\int_{0}^{\sqrt{4-x^2}} \big(x+5\big)\dd y\dd x$\\[5pt]
\textbf{Hint:} draw the region of each integral carefully and see how they all connect.}{$\ds \int_{0}^{\pi}\int_{1}^{2} \big(r\cos\theta+5\big)r\dd r\dd \theta=15\pi/2$}

\exercisesetend


\exercisesetinstructions{, special double integrals are presented that are especially well suited for evaluation in polar coordinates.}

\exercise{Consider $\ds \iint_R e^{-(x^2+y^2)}\dd A$.
\begin{enumext}[start=1]
	\item Why is this integral difficult to evaluate in rectangular coordinates, regardless of the region $R$?
	\item	Let $R$ be the region bounded by the circle of radius $a$ centered at the origin. Evaluate the double integral using polar coordinates.
	\item	Take the limit of your answer from (b), as $a\to\infty$. What does this imply about the volume under the surface of $\ds e^{-(x^2+y^2)}$ over the entire $x$-$y$ plane?
	\item Use your answer to (c) to argue that $\ds\int_{-\infty}^\infty e^{-t^2}\dd t=\sqrt\pi$.
\end{enumext}}{\mbox{}\\[-2\baselineskip]\parbox[t]{\linewidth}{\begin{enumext}[start=1]
	\item This is impossible to integrate with rectangular coordinates as $e^{-(x^2+y^2)}$ does not have an antiderivative in terms of elementary functions.
	\item	$\ds \int_0^{2\pi}\int_0^a re^{r^2}\dd r\dd\theta = \pi(1-e^{-a^2})$.
	\item	$\ds \lim_{a\to\infty} \pi(1-e^{-a^2})=\pi$. This implies that there is a finite volume under the surface $e^{-(x^2+y^2)}$ over the entire $x$-$y$ plane.
	\item If $R=\mathbb{R}^2$, we can write the original integral as $\ds\left(\int_{-\infty}^\infty e^{-t^2}\dd t\right)^2=\pi$.
\end{enumext}}}

\exercise{The surface of a right circular cone with height $h$ and base radius $a$ can be described by the equation $\ds f(x,y) = h-h\sqrt{\frac{x^2}{a^2}+\frac{y^2}{a^2}}$, where the tip of the cone lies at $(0,0,h)$ and the circular base lies in the $x$-$y$ plane, centered at the origin.\bigskip

Confirm that the volume of a right circular cone with height $h$ and base radius $a$ is $\ds V = \frac13\pi a^2h$ by evaluating $\ds \iint_R f(x,y)\dd A$ in polar coordinates.}{\mbox{}\\[-2\baselineskip]\parbox[t]{\linewidth}{\begin{align*}
	\iint_R & f(x,y)\dd A\\
	&= \int_0^{2\pi}\int_0^a \left(h-h\sqrt{\frac{r^2\cos^2\theta}{a^2}+\frac{r^2\sin^2\theta}{a^2}}\right)r\dd r\dd\theta \\
	&= \int_0^{2\pi}\int_0^a \left(hr-h\frac{r^2}{a}\right)\dd r\dd\theta \\
	&= \int_0^{2\pi}\left.\left(\frac12hr^2-\frac{h}{3a}r^3\right)\right|_0^a\dd\theta \\
	&= \int_0^{2\pi} \left(\frac16a^2h\right)\dd\theta\\
	&= \frac13\pi a^2h.
\end{align*}}}

\exercisesetend


\exercise{Use a double integral and polar coordinates to find the area of the region which lies inside the circle $r=\frac32$ and to the right of the line $4r\cos\theta=3$.}{$3\pi/4-9\sqrt3/16$}

\exercise{Use a double integral and polar coordinates to find the area of the region that lies inside the circle $r=3\cos\theta$ and outside the circle $r=\cos\theta$. Check your answer using geometry.}{$2\pi$}

\exercise{Evaluate $\ds\iint_R\frac1x\dd A$ where $R$ is the region that lies inside the circle $x^2+y^2=1$ and to the right of the parabola $2x+y^2=1$.}{$2$}

\exercise{Evaluate $\ds\iint_R(1+x^2+y^2)^{-3/2}\dd A$ where $R$ is that part of the disk $x^2+y^2\le1$ in the first quadrant.}{$(1-1/\sqrt2)\pi/2$}

\exercise{Let $R$ be the annular region $a^2\le x^2+y^2\le1$. Find the average distance of points in $R$ to the origin. What is the limit of the average distance as $a\to0$? As $a\to1$?}{$2(1-a^3)/3(1-a^2)$; $2/3$; $1$}
