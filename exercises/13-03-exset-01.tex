\begin{exerciseset}{In Exercises}{, a function $f(x,y)$ is given and a region $R$ of the $x$-$y$ plane is described. Set up and evaluate $\iint_Rf(x,y)\dd A$ using polar coordinates.}

\exercise{$f(x,y) = 3x-y+4$; $R$ is the region enclosed by the circle $x^2+y^2=1$.}{$\ds \int_0^{2\pi}\int_0^1 \big(3r\cos\theta-r\sin\theta+4\big)r\dd r\dd \theta = 4\pi$}

\exercise{$f(x,y) = 4x+4y$; $R$ is the region enclosed by the circle $x^2+y^2=4$.}{$\ds \int_0^{2\pi}\int_0^2 \big(4r\cos\theta+4r\sin\theta\big)r\dd r\dd \theta = 0$}

\exercise{$f(x,y) = 8-y$; $R$ is the region enclosed by the circles with polar equations $r=\cos\theta$ and $r=3\cos\theta$.}{$\ds \int_0^{\pi}\int_{\cos\theta}^{3\cos\theta} \big(8-r\sin\theta\big)r\dd r\dd \theta = 16\pi$}

\exercise{$f(x,y) = 4$; $R$ is the region enclosed by the petal of the rose curve $r=\sin(2\theta)$ in the first quadrant.}{$\ds \int_0^{\pi/2}\int_{0}^{\sin(2\theta)} \big(4\big)r\dd r\dd \theta = \pi/2$}

\exercise{$f(x,y) = \ln\big(x^2+y^2)$; $R$ is the annulus enclosed by the circles $x^2+y^2=1$ and $x^2+y^2=4$.}{$\ds \int_0^{2\pi}\int_{1}^{2} \big(\ln(r^2)\big)r\dd r\dd \theta = 2\pi\big(\ln16-3/2\big)$}

\exercise{$f(x,y) = 1-x^2-y^2$; $R$ is the region enclosed by the circle $x^2+y^2=1$.}{$\ds \int_0^{2\pi}\int_{0}^{1} \big(1-r^2\big)r\dd r\dd \theta = \pi/2$}

\exercise{$f(x,y) = x^2-y^2$; $R$ is the region enclosed by the circle $x^2+y^2=36$ in the first and fourth quadrants.}{$\ds \int_{-\pi/2}^{\pi/2}\int_{0}^{6} \big(r^2\cos^2\theta-r^2\sin^2\theta\big)r\dd r\dd \theta
= \int_{-\pi/2}^{\pi/2}\int_{0}^{6} \big(r^2\cos(2\theta)\big)r\dd r\dd \theta= 0$}

\exercise{$f(x,y) = (x-y)/(x+y)$; $R$ is the region enclosed by the lines $y=x$, $y=0$ and the circle $x^2+y^2=1$ in the first quadrant.}{$\ds \int_{0}^{\pi/4}\int_{0}^{1} \left(\frac{\cos\theta-\sin\theta}{\cos\theta+\sin\theta}\right)r\dd r\dd \theta = \ln 2$}

\end{exerciseset}
