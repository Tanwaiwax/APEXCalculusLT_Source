\exercisesetinstructions{, a solid is described along with its density function. Find the mass of the solid using spherical coordinates. %(Note: these are the same solids and density functions as found in Exercises~\ref{ex:13_07_ex_23}--\ref{ex:13_07_ex_26}.) 
}

\exercise{The upper half of the unit ball, bounded between $z= 0$ and $z=\sqrt{1-x^2-y^2}$, with density function $\delta(x,y,z) =1$.\label{ex:13_07_ex_31}}{In spherical coordinates, the density is $\delta(\rho,\theta,\varphi) = 1$. Thus mass is
\[
\int_0^{\pi/2}\int_0^{2\pi}\int_0^1\rho^2\sin(\varphi)\dd\rho\dd\theta\dd\varphi
= 2\pi/3.
\]

%We find $M_{yz} = 0$, $M_{xz} = 0$, and $M_{xy} = \pi/4$, placing the center of mass at $(0,0,3/8)$.
}

\exercise{The spherical shell bounded between $x^2+y^2+z^2=16$ and $x^2+y^2+z^2=25$ with density function $\delta(x,y,z) = \sqrt{x^2+y^2+z^2}$.}{In spherical coordinates, the density is $\delta(\rho,\theta,\varphi) = \rho$. Thus mass is
\[
\int_0^\pi\int_0^{2\pi}\int_4^5(\rho)\rho^2\sin(\varphi)\dd\rho\dd\theta\dd\varphi
= 369\pi.
\]

%We find $M_{yz} = 0$, $M_{xz} = 0$, and $M_{xy} = 0$, placing the center of mass at $(0,0,0)$.
}

\exercise{The conical region bounded above $z=\sqrt{x^2+y^2}$ and below the sphere $x^2+y^2+z^2=1$ with density function $\delta(x,y,z) = z$.}{In spherical coordinates, the density is $\delta(\rho,\theta,\varphi) = \rho\cos\varphi$. Thus mass is
\[
\int_0^{\pi/4}\int_0^{2\pi}\int_0^1
\big(\rho\cos(\varphi)\big)\rho^2\sin(\varphi)\dd\rho\dd\theta\dd\varphi = \pi/8.
\]

%We find $M_{yz} = 0$, $M_{xz} = 0$, and $M_{xy} = (4-\sqrt{2})\pi/30$, placing the center of mass at $(0,0,4(4-\sqrt2)/15)$.
}

\exercise{The cone bounded above $z=\sqrt{x^2+y^2}$ and below the plane $z=1$ with density function $\delta(x,y,z) = z$.\label{ex:13_07_ex_34}}{In spherical coordinates, the density is $\delta(\rho,\theta,\varphi) = \rho\cos\varphi$. Thus mass is
\[
\int_0^{\pi/4}\int_0^{2\pi}\int_{0}^{\sec(\varphi)}
\big(\rho\cos(\varphi)\big)\rho^2\sin(\varphi)\dd\rho\dd\theta\dd\varphi = \pi/4.
\]

%We find $M_{yz} = 0$, $M_{xz} = 0$, and $M_{xy} = \pi/5$, placing the center of mass at $(0,0,4/5)$.
}

\exercisesetend
