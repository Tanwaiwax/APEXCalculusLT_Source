\printconcepts

\exercise{An integral can be interpreted as giving the signed area over an interval; a double integral can be interpreted as giving the signed \underline{\hskip .5in} over a region.}{volume}

\exercise{Explain why the following statement is false:  ``Fubini's Theorem states that
\[\ds \int_a^b\int_{g_1(x)}^{g_2(x)} f(x,y)\dd y\dd x = \int_a^b\int_{g_1(y)}^{g_2(y)} f(x,y)\dd x\dd y.''\]}{When switching the order of integration, the bounds integrals must change to reflect the bounds of the region of integration. You cannot merely change the letters $x$ and $y$ in a few places.}

\exercise{Explain why if $f(x,y)>0$ over a region $R$, then\\
$\ds\iint_Rf(x,y)\dd A>0$.}{The double integral gives the signed volume under the surface. Since the surface is always positive, it is always above the $x$-$y$ plane and hence produces only ``positive'' volume.}

\exercise{If $\ds\iint_R f(x,y)\dd A = \iint_R g(x,y)\dd A$, does this imply $f(x,y) = g(x,y)$?}{No. It means that there is the same amount of signed volume under $f$ and $g$ over $R$, but the functions could be very different.}

\printproblems

\begin{exerciseset}{In Exercises}{, 
\begin{enumerate}
\item Evaluate the given iterated integral, and
\item rewrite the integral using the other order of integration.
\end{enumerate}}

\exercise{$\ds \int_1^2\int_{-1}^1\left(\frac xy+3\right)\ dx\ dy$\label{13_02_ex_05}}{6; $\ds \int_{-1}^1\int_{1}^2\left(\frac xy+3\right)\ dy\ dx$}

\exercise{$\ds \int_{-\pi/2}^{\pi/2}\int_{0}^\pi\left(\sin x\cos y\right)\ dx\ dy$}{4; $\ds \int_{0}^\pi\int_{-\pi/2}^{\pi/2}\left(\sin x\cos y\right)\ dy\ dx$}

\exercise{$\ds \int_{0}^{4}\int_{0}^{-x/2+2}\left(3x^2-y+2\right)\ dy\ dx$}{112/3; $\ds \int_{0}^{2}\int_{0}^{4-2y}\left(3x^2-y+2\right)\ dx\ dy$}

\exercise{$\ds \int_{1}^{3}\int_{y}^{3}\left(x^2y-xy^2\right)\ dx\ dy$\label{13_02_ex_08}}{$76/15$; $\ds \int_{1}^{3}\int_{1}^{x}\left(x^2y-xy^2\right)\ dy\ dx$}

\exercise{$\ds \int_{0}^{1}\int_{-\sqrt{1-y}}^{\sqrt{1-y}}\left(x+y+2\right)\ dx\ dy$}{16/5; $\ds \int_{-1}^{1}\int_{0}^{1-x^2}\left(x+y+2\right)\ dy\ dx$}

\exercise{$\ds \int_{0}^{9}\int_{y/3}^{\sqrt{y}}\left(xy^2\right)\ dx\ dy$}{6561/40; $\ds \int_{0}^{3}\int_{x^2}^{3x}\left(xy^2\right)\ dy\ dx$}

\end{exerciseset}


\exerciseset{In Exercises}{: 
\begin{enumerate}
\item [(a)] Sketch the region $R$ given by the problem.
\item	[(b)] Set up the iterated integrals, in both orders, that evaluate the given double integral for the described region $R$. 
\item [(c)] Evaluate one of the iterated integrals to find the signed volume under the surface $z=f(x,y)$ over the region $R$.
\end{enumerate}}{

\exercise{$\ds \iint_R x^2y\ dA$, where $R$ is bounded by $y=\sqrt{x}$ and $y=x^2$.}{\mbox{}\\[-2\baselineskip]\begin{enumerate}
\item \noindent \begin{minipage}{.9\linewidth}
\begin{tikzpicture}
\begin{axis}[width=1.16\marginparwidth,tick label style={font=\scriptsize},
axis y line=middle,axis x line=middle,name=myplot,axis on top,
xtick={1},ytick={1},ymin=-.1,ymax=1.5,xmin=-.1,xmax=1.5]
\addplot [very thick,draw={\colorone},smooth,domain=-.1:1.1,samples=20] ({x},{x^2});
\addplot [very thick,draw={\colorone},smooth,domain=-.1:1.1,samples=20] ({x^2},{x});
\draw (axis cs: .5,.5) node {$R$}
      (axis cs: .5,.9) node {\scriptsize$y=\sqrt{x}$}
                        (axis cs: .85,.4) node {\scriptsize$y=x^2$};
\end{axis}
\node [right] at (myplot.right of origin) {\scriptsize $x$};
\node [above] at (myplot.above origin) {\scriptsize $y$};
\end{tikzpicture}
\end{minipage}
\item	$\ds  \int_0^1\int_{x^2}^{\sqrt{x}}x^2y\ dy\ dx = \int_0^1\int_{y^2}^{\sqrt{y}}x^2y\ dx\ dy$.
\item	$\frac3{56}$
\end{enumerate}}

\exercise{$\ds \iint_R x^2y\ dA$, where $R$ is bounded by $y=\sqrt[3]{x}$ and $y=x^3$.}{\mbox{}\\[-2\baselineskip]\begin{enumerate}
\item \noindent \begin{minipage}{.9\linewidth}
\begin{tikzpicture}
\begin{axis}[width=1.16\marginparwidth,tick label style={font=\scriptsize},
axis y line=middle,axis x line=middle,name=myplot,axis on top,
xtick={1},ytick={1},ymin=-1.5,ymax=1.5,xmin=-1.5,xmax=1.5]
\addplot [very thick,draw={\colorone},smooth,domain=-1.1:1.1,samples=20] ({x},{x^3});
\addplot [very thick,draw={\colorone},smooth,domain=-1.1:1.1,samples=20] ({x^3},{x});
\draw (axis cs: -.5,.5)  node (A) {$R$}
      (axis cs: .5,1.1) node {\scriptsize$y=\sqrt[3]{x}$}
                        (axis cs: 1.05,.4) node {\scriptsize$y=x^3$};
\draw [->,>=stealth] (A) -- (axis cs: .5,.5);
\draw [->,>=stealth] (A) -- (axis cs: -.5,-.5);
\end{axis}
\node [right] at (myplot.right of origin) {\scriptsize $x$};
\node [above] at (myplot.above origin) {\scriptsize $y$};
\end{tikzpicture}
\end{minipage}
\item	$\ds \int_0^1\int_{x^3}^{\sqrt[3]{x}}x^2y\ dy\ dx + \int_{-1}^0\int_{\sqrt[3]{x}}^{x^3}x^2y\ dy\ dx$ $\ds = \int_0^1\int_{y^3}^{\sqrt[3]{y}}x^2y\ dx\ dy +\int_{-1}^0\int_{\sqrt[3]{y}}^{y^3}x^2y\ dx\ dy$.
\item 0
\end{enumerate}}

\exercise{$\ds \iint_R x^2-y^2\ dA$, where $R$ is the rectangle with corners $(-1,-1)$, $(1,-1)$, $(1,1)$ and $(-1,1)$.}{\mbox{}\\[-2\baselineskip]\begin{enumerate}
\item \noindent \begin{minipage}{.9\linewidth}
\begin{tikzpicture}
\begin{axis}[width=1.16\marginparwidth,tick label style={font=\scriptsize},
axis y line=middle,axis x line=middle,name=myplot,axis on top,
xtick={-1,1},ytick={1,-1},ymin=-1.1,ymax=1.1,xmin=-1.1,xmax=1.1]
\draw [very thick,draw={\colorone}] (axis cs:-1,-1) -- (axis cs:1,-1) --(axis cs:1,1) --(axis cs:-1,1) --cycle;
\draw (axis cs: .5,.5) node {$R$};
\end{axis}
\node [right] at (myplot.right of origin) {\scriptsize $x$};
\node [above] at (myplot.above origin) {\scriptsize $y$};
\end{tikzpicture}
\end{minipage}
\item	$\ds \int_{-1}^1\int_{-1}^{1}x^2-y^2\ dy\ dx = \int_{-1}^1\int_{-1}^{1}x^2-y^2\ dx\ dy$.
\item 0
\end{enumerate}}

\exercise{$\ds \iint_R ye^x\ dA$, where $R$ is bounded by $x=0$, $x=y^2$ and $y=1$. }{\mbox{}\\[-2\baselineskip]\begin{enumerate}
\item \noindent \begin{minipage}{.9\linewidth}
\begin{tikzpicture}
\begin{axis}[width=1.16\marginparwidth,tick label style={font=\scriptsize},
axis y line=middle,axis x line=middle,name=myplot,axis on top,
xtick={1},ytick={1},ymin=-.5,ymax=1.2,xmin=-.5,xmax=1.2]
\addplot [very thick,draw={\colorone},smooth,domain=-.25:1.1,samples=20] ({x},{1});
\addplot [very thick,draw={\colorone},smooth,domain=-.25:1.1,samples=20] ({x^2},{x});
\addplot [very thick,draw={\colorone},smooth,domain=-.25:1.1,samples=20] ({0},{x});
\draw (axis cs: .25,.75) node {$R$}
                        (axis cs: .5,.5) node {\scriptsize $x=y^2$};
\end{axis}
\node [right] at (myplot.right of origin) {\scriptsize $x$};
\node [above] at (myplot.above origin) {\scriptsize $y$};
\end{tikzpicture}
\end{minipage}
\item	$\ds \int_{0}^1\int_{0}^{y^2}ye^x\ dx\ dy = \int_{0}^1\int_{\sqrt{x}}^{1}ye^x\ dy\ dx$.
\item $e/2-1$
\end{enumerate}}

% todo solution to 14.2#15b, 16b
\exercise{$\ds \iint_R \big(6-3x-2y\big)\ dA$, where $R$ is bounded by $x=0$, $y=0$ and $3x+2y=6$.}{\mbox{}\\[-2\baselineskip]\begin{enumerate}
\item \noindent \begin{minipage}{.9\linewidth}
\begin{tikzpicture}
\begin{axis}[width=1.16\marginparwidth,tick label style={font=\scriptsize},
axis y line=middle,axis x line=middle,name=myplot,axis on top,
xtick={1,2},ytick={1,2,3},ymin=-.5,ymax=3.5,xmin=-.5,xmax=2.5]
\addplot [very thick,draw={\colorone},smooth,domain=-.25:2.25,samples=20] ({x},{0});
\addplot [very thick,draw={\colorone},smooth,domain=-.25:2.25,samples=20] ({x},{3-3/2*x});
\addplot [very thick,draw={\colorone},smooth,domain=-.25:3.25,samples=20] ({0},{x});
\draw (axis cs: .75,1) node {$R$}
                        (axis cs: 1,2) node [rotate=-40] {\scriptsize $3x+2y=6$};
\end{axis}
\node [right] at (myplot.right of origin) {\scriptsize $x$};
\node [above] at (myplot.above origin) {\scriptsize $y$};
\end{tikzpicture}
\end{minipage}
\item 
\item $\ds \int_{0}^2\int_{0}^{3-3/2x}\big(6-3x-2y\big)\ dy\ dx = \int_{0}^3\int_{0}^{2-2/3y}\big(6-3x-2y\big)\ dx\ dy$.
\item 6
\end{enumerate}}

\exercise{$\ds \iint_R e^y\ dA$, where $R$ is bounded by $y=\ln x$ and\par $\ds y=\frac{1}{e-1}(x-1)$.}{\mbox{}\\[-2\baselineskip]\begin{enumerate}
\item \noindent \begin{minipage}{.9\linewidth}
\begin{tikzpicture}
\begin{axis}[width=1.16\marginparwidth,tick label style={font=\scriptsize},axis y line=middle,axis x line=middle,name=myplot,axis on top,
xtick={1,2,3},ytick={1,2,3},ymin=-.5,ymax=1.25,xmin=-.5,xmax=3.25]
\addplot [very thick,draw={\colorone},smooth,domain=.5:3.25,samples=20] ({x},{ln(x)});
\addplot [very thick,draw={\colorone},smooth,domain=-.25:3.25,samples=20] ({x},{1/1.71828*(x-1)});
\draw (axis cs: 1,.5) node (A) {$R$}
                        (axis cs: 2,.9) node [rotate=40] {\scriptsize $y=\ln x$}
                        (axis cs: 2.5,.3) node [rotate=0] {\scriptsize $y=\frac{1}{1-e}(x-1)$};
\draw[->,>=stealth] (A) -- (axis cs:1.8,.5);
\end{axis}
\node [right] at (myplot.right of origin) {\scriptsize $x$};
\node [above] at (myplot.above origin) {\scriptsize $y$};
\end{tikzpicture}
\end{minipage}
\item	
\item $\ds  \int_{1}^e\int_{\frac{x-1}{e-1}}^{\ln x}e^y\ dy\ dx = \int_{0}^1\int_{e^y}^{y(e-1)+1}e^y\ dx\ dy$.
\item $-\frac12e^2+2e-\frac32$
\end{enumerate}}

\exercise{$\ds \iint_R \big(x^3y-x\big)\ dA$, where $R$ is the half disk $x^2+y^2\le9$ in the first and second quadrants.}{\mbox{}\\[-2\baselineskip]\begin{enumerate}
\item \noindent \begin{minipage}{.9\linewidth}
\begin{tikzpicture}
\begin{axis}[width=1.16\marginparwidth,tick label style={font=\scriptsize},
axis y line=middle,axis x line=middle,name=myplot,axis on top,
xtick={-3,3},ytick={-3,3},ymin=-3.1,ymax=3.1,xmin=-3.7,xmax=3.7]
\addplot [very thick,draw={\colorone},smooth,domain=0:180,samples=30] ({3*cos(x)},{3*sin(x)})--cycle;
\draw (axis cs: 1,1) node (A) {$R$};
\end{axis}
\node [right] at (myplot.right of origin) {\scriptsize $x$};
\node [above] at (myplot.above origin) {\scriptsize $y$};
\end{tikzpicture}
\end{minipage}
\item	$\ds \int_{-3}^3\int_{0}^{\sqrt{9-x^2}}\big(x^3y-x\big)\ dy\ dx = \int_{0}^3\int_{-\sqrt{9-y^2}}^{\sqrt{9-y^2}}\big(x^3y-x\big)\ dx\ dy$.
\item 0
\end{enumerate}}

\exercise{$\ds \iint_R \big(4-3y\big)\ dA$, where $R$ is bounded by $y=0$, $y=x/e$ and $y=\ln x$.}{\mbox{}\\[-2\baselineskip]\begin{enumerate}
\item \noindent \begin{minipage}{.9\linewidth}
\begin{tikzpicture}
\begin{axis}[width=1.16\marginparwidth,tick label style={font=\scriptsize},
axis y line=middle,axis x line=middle,name=myplot,axis on top,
xtick={1,2},extra x ticks={2.718},extra x tick labels={$e$},ytick={1},
ymin=-.5,ymax=1.2,xmin=-.5,xmax=3.5]
\addplot [very thick,draw={\colorone},smooth,domain=.5:3.25,samples=30] ({x},{ln(x)});
\addplot [very thick,draw={\colorone},smooth,domain=-.5:3.25,samples=2] ({x},{0});
\addplot [very thick,draw={\colorone},smooth,domain=-.5:3.25,samples=20] ({x},{x/2.71828});
\draw (axis cs: .8,.1) node (A) {$R$};
\filldraw (axis cs:2.718,1) circle (1.5pt);
\draw (axis cs:2.718,1) node [below right] {\scriptsize $(e,1)$};
\end{axis}
\node [right] at (myplot.right of origin) {\scriptsize $x$};
\node [above] at (myplot.above origin) {\scriptsize $y$};
\end{tikzpicture}
\end{minipage}
\item	$\ds   \int_{0}^1\int_{ey}^{e^y}\big(4-3y\big)\ dx\ dy = \int_{0}^1\int_{0}^{x/e}\big(4-3y\big)\ dy\ dx+\int_{1}^e\int_{\ln x}^{x/e}\big(4-3y\big)\ dy\ dx$.
\item $3e-7$
\end{enumerate}}

}


\begin{exerciseset}{In Exercises}{, state why it is difficult/impossible to integrate the iterated integral in the given order of integration. Change the order of integration and evaluate the new iterated integral.}

\exercise{$\ds \int_0^4\int_{y/2}^2 e^{x^2}\ dx\ dy$}{Integrating $e^{x^2}$ with respect to $x$ is not possible in terms of elementary functions. $\ds  \int_0^2\int_0^{2x}e^{x^2}\ dy\ dx = e^4-1$.}

\exercise{$\ds \int_0^{\sqrt{\pi/2}}\int_{x}^{\sqrt{\pi/2}} \cos\big(y^2\big)\ dy\ dx$}{Integrating $\cos(y^2)$ with respect to $y$ is not possible in terms of elementary functions. $\ds  \int_0^{\sqrt{\pi/2}}\int_0^{y}\cos(y^2)\ dx\ dy = \frac12$.}

\exercise{$\ds \int_0^{1}\int_y^{1} \frac{2y}{x^2+y^2}\ dx\ dy$}{Integrating $\ds\int_y^{1}\frac{2y}{x^2+y^2}\ dx$ gives $\tan^{-1}(1/y)-\pi/4$; integrating $\tan^{-1}(1/y)$ is hard.

$\ds\int_0^1\int_0^x\frac{2y}{x^2+y^2}\ dy \ dx = \ln 2$.}

\exercise{$\ds \int_{-1}^{1}\int_1^{2} \frac{x\tan^2y}{1+\ln y}\ dy\ dx$}{Integrating in the order shown is hard / impossible. By changing the order of integration, we have $\ds \int_{1}^{2}\int_{-1}^{1} \frac{x\tan^2y}{1+\ln y}\ dx\ dy = 0$, since the integrand is an odd function with respect to $x$. Thus the iterated integral evaluates to 0.}

\end{exerciseset}


\input{exercises/13_02_exset_04}

% todo add exercises with Riemann sums
