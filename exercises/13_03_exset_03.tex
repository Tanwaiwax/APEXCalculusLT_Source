\exerciseset{In Exercises}{, special double integrals are presented that are especially well suited for evaluation in polar coordinates.}{

\exercise{Consider $\ds \iint_R e^{-(x^2+y^2)}\ dA$.
\begin{enumerate}
	\item Why is this integral difficult to evaluate in rectangular coordinates, regardless of the region $R$?
	\item	Let $R$ be the region bounded by the circle of radius $a$ centered at the origin. Evaluate the double integral using polar coordinates.
	\item	Take the limit of your answer from (b), as $a\to\infty$. What does this imply about the volume under the surface of $\ds e^{-(x^2+y^2)}$ over the entire $x$-$y$ plane?
	\item Use your answer to (c) to argue that $\ds\int_{-\infty}^\infty e^{-t^2}\ dt=\sqrt\pi$.
\end{enumerate}}{\mbox{}\\[-2\baselineskip]\begin{enumerate}
	\item This is impossible to integrate with rectangular coordinates as $e^{-(x^2+y^2)}$ does not have an antiderivative in terms of elementary functions.
	\item	$\ds \int_0^{2\pi}\int_0^a re^{r^2}\ dr\ d\theta = \pi(1-e^{-a^2})$.
	\item	$\ds \lim_{a\to\infty} \pi(1-e^{-a^2})=\pi$. This implies that there is a finite volume under the surface $e^{-(x^2+y^2)}$ over the entire $x$-$y$ plane.
	\item If $R=\mathbb{R}^2$, we can write the original integral as $\ds\left(\int_{-\infty}^\infty e^{-t^2}\ dt\right)^2=\pi$.
\end{enumerate}}

\exercise{The surface of a right circular cone with height $h$ and base radius $a$ can be described by the equation $\ds f(x,y) = h-h\sqrt{\frac{x^2}{a^2}+\frac{y^2}{a^2}}$, where the tip of the cone lies at $(0,0,h)$ and the circular base lies in the $x$-$y$ plane, centered at the origin.\\

Confirm that the volume of a right circular cone with height $h$ and base radius $a$ is $\ds V = \frac13\pi a^2h$ by evaluating $\ds \iint_R f(x,y)\ dA$ in polar coordinates.}{\mbox{}\\[-2\baselineskip]\begin{align*}
	\iint_R & f(x,y)\ dA\\
	&= \int_0^{2\pi}\int_0^a \left(h-h\sqrt{\frac{r^2\cos^2\theta}{a^2}+\frac{r^2\sin^2\theta}{a^2}}\right)r\ dr\ d\theta \\
	&= \int_0^{2\pi}\int_0^a \left(hr-h\frac{r^2}{a}\right)\ dr\ d\theta \\
	&= \int_0^{2\pi}\left.\left(\frac12hr^2-\frac{h}{3a}r^3\right)\right|_0^a d\theta \\
	&= \int_0^{2\pi} \left(\frac16a^2h\right)\ d\theta\\
	&= \frac13\pi a^2h.
\end{align*}}

}
