\printconcepts

\exercise{``Surface area'' is analogous to what previously studied concept?}{arc length}

\exercise{To approximate the area of a small portion of a surface, we computed the area of its \underline{\hskip .5in} plane.}{tangent}

\exercise{We interpret $\ds \iint_R\ dS$ as ``sum up lots of little \underline{\hskip .5in} \underline{\hskip .5in}.''}{surface areas}

\exercise{Why is it important to know how to set up a double integral to compute surface area, even if the resulting integral is hard to evaluate?}{Technology makes good approximations accessible, if not exact answers.}

\exercise{Why do $z=f(x,y)$ and $z=g(x,y)=f(x,y)+h$, for some real number $h$, have the same surface area over a region $R$? }{Intuitively, adding $h$ to $f$ only shifts $f$ up (i.e., parallel to the $z$-axis) and does not change its shape. Therefore it will not change the surface area over $R$. 

Analytically, $f_x = g_x$ and $f_y=g_y$; therefore, the surface area of each is computed with identical double integrals.}

\exercise{Let $z=f(x,y)$ and $z=g(x,y)=2f(x,y)$. Why is the surface area of $g$ over a region $R$ not twice the surface area of $f$ over $R$?}{Analytically, $g_x = 2f_x$ and $g_y=2f_y$. The double integral to compute the surface area of $f$ over $R$ is $\ds \iint_R \sqrt{1+f_x^2+f_y^2}\ dA$; the double integral to compute the surface area of $g$ over $R$ is $\ds \iint_R\sqrt{1+4f_x^2+4f_y^2}\ dA$, which is \textit{not} twice the double integral used to calculate the surface area of $f$.}

\printproblems

\begin{exerciseset}{In Exercises}{, \textit{set up} the iterated integral that computes the surface area of the surface $z=f(x,y)$ over the region $R$.}

\exercise{$f(x,y) = \sin x\cos y$;\quad $R$ is the rectangle with bounds $0\leq x\leq 2\pi$, \quad $0\leq y\leq2\pi$.\\
\myincludeasythree{width=.7\linewidth,
3Droll=0,
3Dortho=0.0049752527847886086,
3Dc2c=0.40672242641448975 0.6828556656837463 0.6068649291992188,
3Dcoo=30.6619873046875 3.983891248703003 -25.03248405456543,
3Droo=150.00000343149}{width=.7\linewidth}{figures/fig13_05_ex_05_3D}}{$\ds S = \int_0^{2\pi}\int_0^{2\pi} \sqrt{1+ \cos^2x\cos^2y+\sin^2x\sin^2y}\ dx\ dy$}

\exercise{$\ds f(x,y) = \frac{1}{x^2+y^2+1}$;\quad $R$ is the disk $x^2+y^2\le9$.\\
\myincludeasythree{width=.7\linewidth,
3Droll=0,
3Dortho=0.005094848107546568,
3Dc2c=0.5450001358985901 0.6626530885696411 0.5136786103248596,
3Dcoo=-12.860339164733887 -20.208675384521484 36.5958251953125,
3Droo=150.00000281692286}{width=.7\linewidth}{figures/fig13_05_ex_06_3D}}{$\ds S = \int_{-3}^{3}\int_{-\sqrt{9-x^2}}^{\sqrt{9-x^2}} \sqrt{1+ \frac{4x^2+4y^2}{(1+x^2+y^2)^4}}\ dx\ dy$\\
Polar offers simpler bounds:\\
$\ds S = \int_0^{2\pi}\int_0^3 r\sqrt{1+\frac{4r^2}{(1+r^2)^4}}\ dr\ d\theta$}

\exercise{$\ds f(x,y) = x^2-y^2$;\quad $R$ is the rectangle with opposite corners $(-1,-1)$ and $(1,1)$.\\
\myincludeasythree{width=.7\linewidth,
3Droll=0,
3Dortho=0.004975249990820885,
3Dc2c=0.4553585350513458 0.7132656574249268 0.532823383808136,
3Dcoo=-0.20323395729064941 3.399019956588745 -5.396617889404297,
3Droo=150.00000127208068}{width=.7\linewidth}{figures/fig13_05_ex_07_3D}}{$\ds S = \int_{-1}^{1}\int_{-1}^{1} \sqrt{1+ 4x^2+4y^2}\ dx\ dy$}

\exercise{$\ds f(x,y) = \frac{1}{e^{x^2}+1}$;\quad $R$ is the rectangle bounded by\\[5pt]
$-5\leq x\leq 5$ and $0\leq y\leq 1$.\\
\myincludeasythree{width=.7\linewidth,
3Droll=0,
3Dortho=0.004975249990820885,
3Dc2c=0.3090496361255646 0.6646339893341064 0.680257260799408,
3Dcoo=-18.87652587890625 -2.228297710418701 8.584968566894531,
3Droo=149.99999722540937}{width=.7\linewidth}{figures/fig13_05_ex_08_3D}}{$\ds S = \int_{-5}^{5}\int_{0}^{1} \sqrt{1+ \frac{4x^2e^{2x^2}}{\big(1+e^{x^2}\big)^4}}\ dy\ dx$}

\end{exerciseset}


\begin{exerciseset}{In Exercises}{, find the area of the surface of $z=f(x,y)$ over the region $R$.}

\exercise{$f(x,y)=3x-7y+2$; $R$ is the rectangle with opposite corners $(-1,0)$ and $(1,3)$.}{$\ds S=\int_0^3\int_{-1}^1\sqrt{1+9+49}\ dx\ dy=6\sqrt{59}\approx46.09$}

\exercise{$f(x,y) = 2x+2y+2$; $R$ is the triangle with corners $(0,0)$, $(1,0)$ and $(0,1)$.}{$\ds S = \int_{0}^{1}\int_{0}^{1-x} \sqrt{1+ 4+4}\ dy\ dx = 18$}

\exercise{$f(x,y) = x^2+y^2+10$; $R$ is the disk $x^2+y^2\le16$.}{This is easier in polar:
\begin{align*}
	S &= \int_{0}^{2\pi}\int_{0}^{4} r\sqrt{1+ 4r^2\cos^2t+4r^2\sin^2t}\ dr\ d\theta\\
	&= \int_0^{2\pi}\int_0^4r\sqrt{1+4r^2}\ dr\ d\theta \\
	&= \frac{\pi}{6}\big(65\sqrt{65}-1\big) \approx 273.87
\end{align*}}

\exercise{$f(x,y) = -2x+4y^2+7$ over $R$, the triangle bounded by $y=-x$, $y=x$, $0\leq y\leq 1$.}{\mbox{}\\[-2\baselineskip]\parbox[t]{\linewidth}{\begin{align*}
	S &= \int_{0}^{1}\int_{-y}^{y} \sqrt{1+ 4+64y^2}\ dx\ dy\\
	&= \int_0^{1}\big(2y\sqrt{5+64y^2}\big)\ dy \\
	&= \frac1{96}\big(69\sqrt{69}-5\sqrt{5}\big)\approx 5.85
\end{align*}}}

% cut for parity
%\exercise{$f(x,y) = x^2+y$ over $R$, the triangle bounded by $y=2x$, $y=0$ and $x=2$.}{\mbox{}\\[-2\baselineskip]\begin{align*}
%	S &= \int_{0}^{2}\int_{0}^{2x} \sqrt{1+ 1+4x^2}\ dy\ dx\\
%	&= \int_0^{2}\big(2x\sqrt{2+4x^2}\big)\ dx \\
%	&= \frac{26}{3}\sqrt{2}\approx 12.26
%\end{align*}}

\exercise{$f(x,y) = \frac23x^{3/2}+2y^{3/2}$ over $R$, the rectangle with opposite corners $(0,0)$ and $(1,1)$.}{\mbox{}\\[-2\baselineskip]\parbox[t]{\linewidth}{\begin{align*}
	S &= \int_{0}^{1}\int_{0}^{1} \sqrt{1+ x+9y}\ dx\ dy\\
	&= \int_0^{1}\frac23\Big((9y+2)^{3/2}-(9y+1)^{3/2}\Big)\ dy \\
	&= \frac{4}{135}\big(121\sqrt{11}-100\sqrt{10}-4\sqrt{2}+1\big)\approx 2.383
\end{align*}}}

\exercise{$f(x,y) = 10-2\sqrt{x^2+y^2}$ over $R$, the disk $x^2+y^2\le25$. (This is the cone with height 10 and base radius 5; be sure to compare your result with the known formula.)}{This is easier in polar:
\begin{align*}
	S &= \int_{0}^{2\pi}\int_{0}^{5} r\sqrt{1+ \frac{4r^2\cos^2t+4r^2\sin^2t}{r^2\sin^2t+r^2\cos^2t}}\ dr\ d\theta\\
	&= \int_0^{2\pi}\int_0^5r\sqrt{5}\ dr\ d\theta \\
	&= 25\pi\sqrt{5}\approx 175.62
\end{align*}}

\exercise{Find the surface area of the sphere with radius 5 by doubling the surface area of $f(x,y) = \sqrt{25-x^2-y^2}$ over $R$, the disk $x^2+y^2\le25$. (Be sure to compare your result with the known formula.)}{This is easier in polar:
\begin{align*}
	S &= 2\int_{0}^{2\pi}\int_{0}^{5} r\sqrt{1+ \frac{r^2\cos^2t+r^2\sin^2t}{25-r^2\sin^2t-r^2\cos^2t}}\ dr\ d\theta\\
	&= 2\int_0^{2\pi}\int_0^5r\sqrt{\frac{1}{25-r^2}}\ dr\ d\theta \\
	&= 100\pi\approx 314.16
\end{align*}}

\exercise{Find the surface area of the ellipse formed by restricting the plane $f(x,y) = cx+dy+h$ to the region $R$, the disk $x^2+y^2\le1$, where $c$, $d$ and $h$ are some constants. Your answer should be given in terms of $c$ and $d$; why does the value of $h$ not matter?}{Integrating in polar is easiest considering $R$:
\begin{align*}
	S &= \int_{0}^{2\pi}\int_{0}^{1} r\sqrt{1+ c^2+d^2}\ dr\ d\theta\\
	&= \int_0^{2\pi}\frac12\Big(\sqrt{1+c^2+d^2}\Big)\ d\theta \\
	&= \pi\sqrt{1+c^2+d^2}.
\end{align*}
The value of $h$ does not matter as it only shifts the plane vertically (i.e., parallel to the $z$-axis). Different values of $h$ do not create different ellipses in the plane.}

\end{exerciseset}

