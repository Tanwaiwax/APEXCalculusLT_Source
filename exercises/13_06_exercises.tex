\printconcepts

\exercise{The strategy for establishing bounds for triple integrals is ``\underline{\hskip .5in} to \underline{\hskip .5in}, \underline{\hskip .5in} to \underline{\hskip .5in} and \underline{\hskip .5in} to \underline{\hskip .5in}.''}{surface to surface, curve to curve and point to point}

\exercise{Give an informal interpretation of what ``$\ds \iiint_D\ dV$'' means.}{One possible answer is ``sum up lots of little volumes over $D$.''}

\exercise{Give two uses of triple integration.}{Answers can vary. From this section we used triple integration to find the volume of a solid region, the mass of a solid, and the center of mass of a solid.}

\exercise{If an object has a constant density $\delta$ and a volume $V$, what is its mass?}{$\delta V$.}

\printproblems

\begin{exerciseset}{In Exercises}{,  two surfaces $f_1(x,y)$ and $f_2(x,y)$ and a region $R$ in the $x$, $y$ plane are given. Set up and evaluate the double integral that finds the volume between these surfaces over $R$.}

\exercise{$f_1(x,y) = 8-x^2-y^2$, $f_2(x,y) = 2x+y$;\\
$R$ is the square with corners $(-1,-1)$ and $(1,1)$.}{$V = \int_{-1}^1\int_{-1}^1 \big(8-x^2-y^2-(2x+y)\big)\dd x\dd y = 88/3$}

\exercise{$f_1(x,y) = x^2+y^2$, $f_2(x,y) = -x^2-y^2$;\\
$R$ is the square with corners $(0,0)$ and $(2,3)$.}{$V = \int_{0}^2\int_{0}^3 \big(x^2+y^2-(-x^2-y^2)\big)\dd y\dd x = 52$}

\exercise{$f_1(x,y) = \sin x\cos y$, $f_2(x,y) = \cos x\sin y+2$;\\
$R$ is the triangle with corners $(0,0)$, $(\pi,0)$ and $(\pi,\pi)$.}{$V = \int_{0}^{\pi}\int_{0}^x \big(\cos x\sin y+2-\sin x\cos y\big)\dd y\dd x = \pi^2-\pi\approx 6.728$}

\exercise{$f_1(x,y) = 2x^2+2y^2+3$, $f_2(x,y) = 6-x^2-y^2$;\\
$R$ is the disk $x^2+y^2\le1$.}{$V = \int_{-1}^{1}\int_{-\sqrt{1-x^2}}^{\sqrt{1-x^2}} \big(6-x^2-y^2-(2x^2+2y^2+3))\dd y\dd x$. Integrating in polar is easier, giving $V = \int_0^{2\pi}\int_0^1 \big(3-3r^2\big)r\dd r\dd \theta = 3\pi/2$.}

\end{exerciseset}


\begin{exerciseset}{In Exercises}{,  a domain $D$ is described by its bounding surfaces, along with a graph. Set up the triple integrals that give the volume of $D$ in all 6 orders of integration, and find the volume of $D$ by evaluating the indicated triple integral.}

\exercise{$D$\label{13_06_ex_07} is bounded by the coordinate planes and\\
 $z=2-2x/3-2y$.\\
Evaluate the triple integral with order $dz\ dy\ dx$.\\
\myincludeasythree{width=\marginparwidth,
3Droll=0,
3Dortho=0.0049995239824056625,
3Dc2c=0.6666666865348816 0.6666666865348816 0.3333333134651184,
3Dcoo=-9.639361381530762 -8.587801933288574 36.453922271728516,
3Droo=149.9999987284343}{width=\marginparwidth}{figures/fig13_06_ex_07_3D}}{%
	$dz\ dy\ dx$: $\ds\int_0^3\int_0^{1-x/3}\int_0^{2-2x/3-2y}\ dz\ dy\ dx$\\
	$dz\ dx\ dy$: $\ds\int_0^1\int_0^{3-3y}\int_0^{2-2x/3-2y}\ dz\ dx\ dy$\\
	$dy\ dz\ dx$: $\ds\int_0^3\int_0^{2-2x/3}\int_0^{1-x/3-z/2}\ dy\ dz\ dx$\\
	$dy\ dx\ dz$: $\ds\int_0^2\int_0^{3-3z/2}\int_0^{1-x/3-z/2}\ dy\ dx\ dz$\\
	$dx\ dz\ dy$: $\ds\int_0^1\int_0^{2-2y}\int_0^{3-3y-3z/2}\ dx\ dz\ dy$\\
	$dx\ dy\ dz$: $\ds\int_0^2\int_0^{1-z/2}\int_0^{3-3y-3z/2}\ dx\ dy\ dz$\\
	$\ds V = \int_0^3\int_0^{1-x/3}\int_0^{2-2x/3-2y}\ dz\ dy\ dx =1.$}

\exercise{$D$\label{13_06_ex_08} is bounded by the planes $y=0$, $y=2$,  $x=1$, $z=0$ and\\
 $z=(3-x)/2$.\\
Evaluate the triple integral with order $dx\ dy\ dz$.\\
\myincludeasythree{width=\marginparwidth,
3Droll=0,
3Dortho=0.0049995239824056625,
3Dc2c=0.6666666865348816 0.6666666865348816 0.3333333134651184,
3Dcoo=-9.639361381530762 -8.587801933288574 36.453922271728516,
3Droo=149.9999987284343}{width=\marginparwidth}{figures/fig13_06_ex_08_3D}}{%
	$dz\ dy\ dx$: $\ds\int_1^3\int_0^{2}\int_0^{(3-x)/2}\ dz\ dy\ dx$\\
	$dz\ dx\ dy$: $\ds\int_0^2\int_1^{3}\int_0^{(3-x)/2}\ dz\ dx\ dy$\\
	$dy\ dz\ dx$: $\ds\int_1^3\int_0^{(3-x)/2}\int_0^{2}\ dy\ dz\ dx$\\
	$dy\ dx\ dz$: $\ds\int_0^1\int_1^{3-2z}\int_0^{2}\ dy\ dx\ dz$\\
	$dx\ dz\ dy$: $\ds\int_0^2\int_0^{1}\int_1^{3-2z}\ dx\ dz\ dy$\\
	$dx\ dy\ dz$: $\ds\int_0^1\int_0^{2}\int_1^{3-2z}\ dx\ dy\ dz$\\
	$\ds V = \int_0^1\int_0^{2}\int_1^{3-2z}\ dx\ dy\ dz =2.$}

\exercise{$D$ is bounded by the planes $x=0$, $x=2$,  $z=-y$ and by\\
 $z=y^2/2$.\\
Evaluate the triple integral with the order $dy\ dz\ dx$.\\
\myincludeasythree{width=\marginparwidth,
3Droll=0,
3Dortho=0.004999519791454077,
3Dc2c=0.6666666865348816 0.6666666865348816 0.3333333134651184,
3Dcoo=42.38267135620117 -68.19635772705078 51.62697982788086,
3Droo=149.9999987284343}{width=\marginparwidth}{figures/fig13_06_ex_09_3D}}{%
	$dz\ dy\ dx$: $\ds\int_0^2\int_{-2}^{0}\int_{y^2/2}^{-y}\ dz\ dy\ dx$\\
	$dz\ dx\ dy$: $\ds\int_{-2}^0\int_0^{2}\int_{y^2/2}^{-y}\ dz\ dx\ dy$\\
	$dy\ dz\ dx$: $\ds\int_0^2\int_0^{2}\int_{-\sqrt{2z}}^{-z}\ dy\ dz\ dx$\\
	$dy\ dx\ dz$: $\ds\int_0^2\int_0^{2}\int_{-\sqrt{2z}}^{-z}\ dy\ dx\ dz$\\
	$dx\ dz\ dy$: $\ds\int_{-2}^0\int_{y^2/2}^{-y}\int_0^{2}\ dx\ dz\ dy$\\
	$dx\ dy\ dz$: $\ds\int_0^2\int_{-\sqrt{2z}}^{-z}\int_0^{2}\ dx\ dy\ dz$\\
	$\ds V = \int_0^2\int_0^{2}\int_{-\sqrt{2z}}^{-z}\ dy\ dz\ dx =4/3.$}

\exercise{$D$ is bounded by the planes $z=0$, $y=9$,  $x=0$ and by\\[3pt]
 $z=\sqrt{y^2-9x^2}$.\\
Do not evaluate any triple integral.\\
\myincludeasythree{width=\marginparwidth,
3Droll=0,
3Dortho=0.004025847185403109,
3Dc2c=0.8124339580535889 0.30516722798347473 0.4968138635158539,
3Dcoo=-14.771090507507324 42.51134490966797 24.72347640991211,
3Droo=150.00000218030218}{width=\marginparwidth}{figures/fig13_06_ex_10_3D}}{%
$dz\ dy\ dx$: $\ds\int_0^3\int_{3x}^{9}\int_{0}^{\sqrt{y^2-9x^2}}\ dz\ dy\ dx$\\
$dz\ dx\ dy$: $\ds\int_{0}^9\int_0^{y/3}\int_{0}^{\sqrt{y^2-9x^2}}\ dz\ dx\ dy$\\
$dy\ dz\ dx$: $\ds\int_0^3\int_0^{\sqrt{81-9x^2}}\int_{\sqrt{z^2+9x^2}}^{9}\ dy\ dz\ dx$\\
$dy\ dx\ dz$: $\ds\int_0^9\int_0^{\sqrt{9-z^2/9}}\int_{\sqrt{z^2+9x^2}}^{9}\ dy\ dx\ dz$\\
$dx\ dz\ dy$: $\ds\int_{0}^9\int_{0}^{y}\int_0^{\frac13\sqrt{y^2-z^2}}\ dx\ dz\ dy$\\
$dx\ dy\ dz$: $\ds\int_0^9\int_{z}^{9}\int_0^{\frac13\sqrt{y^2-z^2}}\ dx\ dy\ dz$}

\exercise{$D$\label{13_06_ex_11} is bounded by the planes $x=2$, $y=1$,  $z=0$ and \\
 $z=2x+4y-4$.\\
Evaluate the triple integral with the order $dx\ dy\ dz$.\\
\myincludeasythree{width=\marginparwidth,
3Droll=0,
3Dortho=0.004999519791454077,
3Dc2c=0.7375360131263733 -0.6143473982810974 0.2803889811038971,
3Dcoo=32.48771286010742 72.419189453125 49.87675094604492,
3Droo=200.00000360071368}{width=\marginparwidth}{figures/fig13_06_ex_11_3D}}{%
	$dz\ dy\ dx$: $\ds\int_0^2\int_{1-x/2}^{1}\int_{0}^{2x+4y-4}\ dz\ dy\ dx$\\
	$dz\ dx\ dy$: $\ds\int_{0}^1\int_{2-2y}^{2}\int_{0}^{2x+4y-4}\ dz\ dx\ dy$\\
	$dy\ dz\ dx$: $\ds\int_0^2\int_0^{2x}\int_{z/4-x/2+1}^{1}\ dy\ dz\ dx$\\
	$dy\ dx\ dz$: $\ds\int_0^4\int_{z/2}^{2}\int_{z/4-x/2+1}^{1}\ dy\ dx\ dz$\\
	$dx\ dz\ dy$: $\ds\int_{0}^1\int_{0}^{4y}\int_{z/2-2y+2}^2\ dx\ dz\ dy$\\
	$dx\ dy\ dz$: $\ds\int_0^4\int_{z/4}^{1}\int_{z/2-2y+2}^2\ dx\ dy\ dz$\\	
	$\ds V = \int_0^4\int_{z/4}^{1}\int_{z/2-2y+2}^2\ dx\ dy\ dz = 4/3.$}

\exercise{$D$\label{13_06_ex_12} is bounded by the plane $z=2y$ and by $y=4-x^2$.\\
Evaluate the triple integral with the order $dz\ dy\ dx$.\\
\myincludeasythree{width=\marginparwidth,
3Droll=0,
3Dortho=0.004999519791454077,
3Dc2c=0.8651782274246216 -0.4135543704032898 0.28361836075782776,
3Dcoo=1.8630493879318237 57.10300827026367 66.36750793457031,
3Droo=149.99999560908606}{width=\marginparwidth}{figures/fig13_06_ex_12_3D}}{%
$dz\ dy\ dx$: $\ds\int_{-2}^2\int_{0}^{4-x^2}\int_{0}^{2y}\ dz\ dy\ dx$\\
$dz\ dx\ dy$: $\ds\int_{0}^4\int_{-\sqrt{4-y}}^{\sqrt{4-y}}\int_{0}^{2x+4y-4}\ dz\ dx\ dy$\\
$dy\ dz\ dx$: $\ds\int_{-2}^2\int_0^{8-2x^2}\int_{z/2}^{4-x^2}\ dy\ dz\ dx$\\
$dy\ dx\ dz$: $\ds\int_0^8\int_{-\sqrt{4-z/2}}^{\sqrt{4-z/2}}\int_{z/2}^{4-x^2}\ dy\ dx\ dz$\\
$dx\ dz\ dy$: $\ds\int_{0}^4\int_{0}^{2y}\int_{-\sqrt{4-y}}^{\sqrt{4-y}}\ dx\ dz\ dy$\\
$dx\ dy\ dz$: $\ds\int_0^8\int_{z/2}^{4}\int_{-\sqrt{4-y}}^{\sqrt{4-y}}\ dx\ dy\ dz$\\
$\ds V = \int_{-2}^2\int_{0}^{4-x^2}\int_{0}^{2y}\ dz\ dy\ dx = 512/15.$}

\exercise{$D$ is bounded by the coordinate planes and by \\
$y=1-x^2$ and $y=1-z^2$.\\
Do not evaluate  any triple integral. Which order is easier to evaluate: $dz\ dy\ dx$ or $dy\ dz\ dx$? Explain why.\\
\myincludeasythree{width=\marginparwidth,
3Droll=0,
3Dortho=0.004999519791454077,
3Dc2c=0.6666666865348816 0.6666666865348816 0.3333333134651184,
3Dcoo=-11.596293449401855 -12.679614067077637 48.551517486572266,
3Droo=149.9999987284343}{width=\marginparwidth}{figures/fig13_06_ex_13_3D}}{%
$dz\ dy\ dx$: $\ds\int_{0}^1\int_{0}^{1-x^2}\int_{0}^{\sqrt{1-y}}\ dz\ dy\ dx$\\
$dz\ dx\ dy$: $\ds\int_{0}^1\int_{0}^{\sqrt{1-y}}\int_{0}^{\sqrt{1-y}}\ dz\ dx\ dy$\\
$dy\ dz\ dx$: $\ds\int_{0}^1\int_0^{x}\int_{0}^{1-x^2}\ dy\ dz\ dx + \int_{0}^1\int_x^{1}\int_{0}^{1-z^2}\ dy\ dz\ dx$\\
$dy\ dx\ dz$: $\ds\int_0^1\int_{0}^{z}\int_{0}^{1-z^2}\ dy\ dx\ dz + \int_0^1\int_{z}^{1}\int_{0}^{1-x^2}\ dy\ dx\ dz$\\
$dx\ dz\ dy$: $\ds\int_{0}^1\int_{0}^{\sqrt{1-y}}\int_{0}^{\sqrt{1-y}}\ dx\ dz\ dy$\\
$dx\ dy\ dz$: $\ds\int_0^1\int_{0}^{1-z^2}\int_{0}^{\sqrt{1-y}}\ dx\ dy\ dz$\\
Answers will vary. Neither order is particularly ``hard.'' The order $dz\ dy\ dx$ requires integrating a square root, so powers can be messy; the order $dy\ dz\ dx$ requires two triple integrals, but each uses only polynomials.}

\exercise{$D$ is bounded by the coordinate planes and by \\
$z=1-y/3$ and $z=1-x$.\\
Evaluate the triple integral with order $dx\ dy\ dz$.\\
\myincludeasythree{width=\marginparwidth,
3Droll=0,
3Dortho=0.0049995239824056625,
3Dc2c=0.6666666865348816 0.6666666865348816 0.3333333134651184,
3Dcoo=-9.639361381530762 -8.587801933288574 36.453922271728516,
3Droo=149.9999987284343}{width=\marginparwidth}{figures/fig13_06_ex_14_3D}}{%
	$dz\ dy\ dx$: $\ds\int_{0}^1\int_{0}^{3x}\int_{0}^{1-x}\ dz\ dy\ dx +
	\int_{0}^1\int_{3x}^{3}\int_{0}^{1-y/3}\ dz\ dy\ dx$
	$dz\ dx\ dy$: $\ds\int_{0}^3\int_{0}^{y/3}\int_{0}^{1-y/3}\ dz\ dy\ dx+
	\int_{0}^3\int_{y/3}^{1}\int_{0}^{1-x}\ dz\ dx\ dy$
	$dy\ dz\ dx$: $\ds\int_{0}^1\int_0^{1-x}\int_{0}^{3-3z}\ dy\ dz\ dx $
	$dy\ dx\ dz$: $\ds\int_0^1\int_{0}^{1-z}\int_{0}^{3-3z}\ dy\ dx\ dz$
	$dx\ dz\ dy$: $\ds\int_{0}^3\int_{0}^{1-y/3}\int_{0}^{1-z}\ dx\ dz\ dy$
	$dx\ dy\ dz$: $\ds\int_0^1\int_{0}^{3-3z}\int_{0}^{1-z}\ dx\ dy\ dz$
	$\ds V = \int_0^1\int_{0}^{3-3z}\int_{0}^{1-z}\ dx\ dy\ dz = 1$.}

\end{exerciseset}


\begin{exerciseset}{In Exercises}{,  evaluate the triple integral.}

\exercise{$\ds \int_{-\pi/2}^{\pi/2}\int_0^\pi\int_0^\pi\big(\cos x\sin y\sin z\big)\ dz\ dy\ dx$}{8}

\exercise{$\ds \int_{0}^{1}\int_0^x\int_0^{x+y}\big(x+y+z\big)\ dz\ dy\ dx$}{$7/8$}

\exercise{$\ds \int_{0}^{\pi}\int_{0}^{1}\int_{0}^{z}\big(\sin(yz)\big)\ dx\ dy\ dz$}{$\pi$}

\exercise{$\ds \int_{\pi}^{\pi^2}\int_{x}^{x^3}\int_{-y^2}^{y^2}\left(z\frac{x^2y+y^2x}{e^{x^2+y^2}}\right)\ dz\ dy\ dx$}{$0$}

\end{exerciseset}


\exerciseset{In Exercises} 
{,  find the center of mass of the solid represented by the indicated space region $D$ with density function $\delta(x,y,z)$.
}{

\exercise{$D$ is bounded by the coordinate planes and\\
 $z=2-2x/3-2y$; \quad $\delta(x,y,z) = 10$gm/cm$^3$.\\
(Note: this is the same region as used in Exercise \ref{13_06_ex_07}.)
}{$M = 10$, $M_{yz} = 15/2$, $M_{xz}=5/2$, $M_{xy}=5$;\\
$(\overline{x},\overline{y},\overline{z}) = (3/4, 1/4, 1/2)$
}

\exercise{$D$ is bounded by the planes $y=0$, $y=2$,  $x=1$, $z=0$ and\\
 $z=(3-x)/2$; \quad $\delta(x,y,z) = 2$gm/cm$^3$.\\
(Note: this is the same region as used in Exercise \ref{13_06_ex_08}.)
}{$M = 4$, $M_{yz} = 20/3$, $M_{xz}=4$, $M_{xy}=4/3$;\\
$(\overline{x},\overline{y},\overline{z}) = (5/3, 1, 1/3)$
}

\exercise{$D$ is bounded by the planes $x=2$, $y=1$,  $z=0$ and \\
 $z=2x+4y-4$;\quad $\delta(x,y,z) = x^2$lb/in$^3$.\\
(Note: this is the same region as used in Exercise \ref{13_06_ex_11}.)
}{$M = 16/5$, $M_{yz} = 16/3$, $M_{xz}=104/45$, $M_{xy}=32/9$;\\
$(\overline{x},\overline{y},\overline{z}) = (5/3,13/18,10/9) \approx (1.67,0.72,1.11)$
}

\exercise{$D$ is bounded by the plane $z=2y$ and by $y=4-x^2$.\\
$\delta(x,y,z) = y^2$lb/in$^3$.\\
(Note: this is the same region as used in Exercise \ref{13_06_ex_12}.)
}{$M = \frac{65,536}{15}\approx 208.05$, $M_{yz} = 0$, $M_{xz}=\frac{2,097,152}{3465}\approx 605.24$, $M_{xy}=\frac{2,097,152}{3465}\approx 605.24$;\\
$(\overline{x},\overline{y},\overline{z}) = (0,32/11,32/11) \approx (0,2.91,2.91)$
}
}
