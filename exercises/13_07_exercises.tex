\printconcepts

\exercise{Explain the difference between the roles $r$, in cylindrical coordinates, and $\rho$, in spherical coordinates, play in determining the location of a point.}
{In cylindrical, $r$ determines how far from the origin one goes in the $x$-$y$ plane before considering the $z$-component. Equivalently, if on projects a point in cylindrical coordinates onto the $x$-$y$ plane, $r$ will be the distance of this projection from the origin.

In spherical, $\rho$ is the distance from the origin to the point.}

\exercise{Why are points on the $z$-axis not determined uniquely when using cylindrical and spherical coordinates?}
{If $r=0$ or $\rho=0$, then the point in each coordinate system lies on the $z$-axis regardless of the value of $\theta$.}

\exercise{What surfaces are naturally defined using cylindrical coordinates?}
{Cylinders (tubes) centered at the origin, parallel to the $z$-axis; planes parallel to the $z$-axis that intersect the $z$-axis; planes parallel to the $x$-$y$ plane.}

\exercise{What surfaces are naturally defined using spherical coordinates?}
{Spheres centered at the origin; planes parallel to the $z$-axis that intersect the $z$-axis; cones centered on the $z$-axis with point at the origin.}

\printproblems

\begin{exerciseset}{In Exercises}{,  points are given in either the rectangular, cylindrical or spherical coordinate systems. Find the coordinates of the points in the other systems.}

\exercise{\mbox{}\\[-2\baselineskip]\parbox[t]{\linewidth}{\begin{enumerate}
	\item Points in rectangular coordinates:\\
	$(2,2,1)$ and $(-\sqrt{3},1,0)$
	\item Points in cylindrical coordinates:\\
	$(2,\pi/4,2)$ and $(3,3\pi/2,-4)$
	\item Points in spherical coordinates:\\
	$(2,\pi/4,\pi/4)$ and $(1,0,0)$
\end{enumerate}}}{\mbox{}\\[-2\baselineskip]\parbox[t]{\linewidth}{\begin{enumerate}
	\item Cylindrical: $(2\sqrt 2,\pi/4,1)$ and $(2,5\pi/6,0)$\\
		Spherical: $(3,\pi/4,\cos^{-1}(1/3))$ and $(2,5\pi/6,\pi/2)$
	\item Rectangular: $(\sqrt 2,\sqrt 2,2)$ and $(0,-3,-4)$\\
		Spherical: $(2\sqrt 2,\pi/4,\pi/4)$ and\\$(5,3\pi/2,\pi-\tan^{-1}(3/4))$
	\item Rectangular: $(1,1,\sqrt{2})$ and $(0,0,1)$\\
		Cylindrical:	$(\sqrt{2},\pi/4,\sqrt{2})$ and $(0,0,1)$
\end{enumerate}}}

\exercise{\mbox{}\\[-2\baselineskip]\parbox[t]{\linewidth}{\begin{enumerate}
	\item Points in rectangular coordinates:\\
	$(0,1,1)$ and $(-1,0,1)$
	\item Points in cylindrical coordinates:\\
	$(0,\pi,1)$ and $(2,4\pi/3,0)$	
	\item Points in spherical coordinates:\\
	$(2,\pi/6,\pi/2)$ and $(3,\pi,\pi)$
\end{enumerate}}}{\mbox{}\\[-2\baselineskip]\parbox[t]{\linewidth}{\begin{enumerate}
	\item Cylindrical: $(1,\pi/2,1)$ and $(1,\pi,1)$\\
		Spherical: $(\sqrt 2,\pi/2,\pi/4)$ and $(\sqrt{2}, \pi, \pi/4)$
	\item Rectangular: $(0,0,1)$ and $(-1,-\sqrt 3,0)$\\
		Spherical: $(1,\pi,0)$ and $(2,4\pi/3,\pi/2)$
	\item Rectangular: $(\sqrt 3,1,0)$ and $(0,0,-3)$\\
		Cylindrical:	$(2,\pi/6,0)$ and $(0,\pi,-3)$
\end{enumerate}}}

% these are mecmath adaptations

\exercise{\mbox{}\\[-2\baselineskip]\parbox[t]{\linewidth}{\begin{enumerate}
	\item Points in rectangular coordinates:\\
	$(2,2\sqrt3,-1)$ and $(-5,5,6)$
	\item Points in cylindrical coordinates:\\
	$(1,\frac\pi3,-2)$ and $(2,\frac{5\pi}6,3)$
	\item Points in spherical coordinates:\\
	$(4,\frac\pi3,\frac\pi4)$ and $(1,0,\frac\pi2)$
\end{enumerate}}}{\mbox{}\\[-2\baselineskip]\parbox[t]{\linewidth}{\begin{enumerate}
	\item Cylindrical: $(4,\frac\pi3,-1)$ and $(5\sqrt2,\frac{3\pi}4,6)$\\
		Spherical: $(\sqrt{17},\frac\pi3,\cos^{-1}\frac{-1}{\sqrt{17}})$ and\\
		$(\sqrt{86},\frac{3\pi}4,\cos^{-1}\frac6{\sqrt{86}})$
	\item Rectangular: $(\frac12,\frac{\sqrt3}2,-2)$ and $(-\sqrt3,-1,3)$\\
		Spherical: $(\sqrt5,\frac\pi3,\cos^{-1}(-\frac2{\sqrt5}))$ and\\
		$(\sqrt{13},\frac{5\pi}6,\tan^{-1}\frac23)$
	\item Rectangular: $(\sqrt2,\sqrt6,2\sqrt2)$ and $(1,0,0)$\\
		Cylindrical: $(2\sqrt2,\frac\pi3,2\sqrt2)$ and $(1,0,0)$
\end{enumerate}}}

\exercise{\mbox{}\\[-2\baselineskip]\parbox[t]{\linewidth}{\begin{enumerate}
	\item Points in rectangular coordinates:\\
	$(\sqrt{21},-\sqrt{7},0)$ and $(0,\sqrt{2},2)$
	\item Points in cylindrical coordinates:\\
	$(\frac12,\frac\pi2,1)$ and $(6,\frac{5\pi}3,\sqrt7)$
	\item Points in spherical coordinates:\\
	$(3,\frac{3\pi}2,\frac{5\pi}6)$ and $(2,\frac{7\pi}6,\frac{3\pi}4)$
\end{enumerate}}}{\mbox{}\\[-2\baselineskip]\parbox[t]{\linewidth}{\begin{enumerate}
	\item Cylindrical: $(2\sqrt{7},\frac{11\pi}{6},0)$ and $(\sqrt2,\frac\pi2,2)$\\
		Spherical: $(2\sqrt{7},\frac{11\pi}{6},\frac{\pi}{2})$ and $(\sqrt6,\frac\pi2,\cos^{-1}\frac{\sqrt6}3)$
	\item Rectangular: $(0,\frac12,1)$ and $(3,-3\sqrt3,\sqrt7)$\\
		Spherical: $(\frac{\sqrt5}2,\frac\pi2,\cos\frac2{\sqrt5})$ and\\$(\sqrt{43},\frac{5\pi}3,\cos^{-1}\sqrt{\frac7{43}})$
	\item Rectangular: $(0,-\frac32,-\frac{3\sqrt3}2)$ and $(\frac{\sqrt6}2,\frac{\sqrt2}2,\sqrt2)$\\
		Cylindrical: $(\frac32,\frac{3\pi}2,-\frac{3\sqrt3}2)$ and $(\sqrt2,\frac{7\pi}6,\sqrt2)$
\end{enumerate}}}

\end{exerciseset}

\input{exercises/13_07_exset_02}
\exerciseset{In Exercises}{,  standard regions in space, as defined by cylindrical and spherical coordinates, are shown. Set up the triple integral that integrates the given function over the graphed region.}{

\exercise{Cylindrical coordinates, integrating $h(r,\theta,z)$:

\myincludeasythree{width=150pt,
3Droll=0,
3Dortho=0.004999519791454077,
3Dc2c=0.6666666865348816 0.6666666865348816 0.3333333134651184,
3Dcoo=-11.596293449401855 -12.679614067077637 48.551517486572266,
3Droo=149.9999987284343,
3Dlights=Headlamp}{width=150pt}{figures/fig13_07_ex_09_3D}}{$\ds\int_{\theta_1}^{\theta_2}\int_{r_1}^{r_2}\int_{z_1}^{z_2}h(r,\theta,z)r\ dz\ dr\ d\theta$}

\exercise{Cylindrical coordinates, integrating $h(\rho,\theta,\varphi)$:

\myincludeasythree{width=150pt,
3Droll=0,
3Dortho=0.004999519791454077,
3Dc2c=0.6666666865348816 0.6666666865348816 0.3333333134651184,
3Dcoo=-11.596293449401855 -12.679614067077637 48.551517486572266,
3Droo=149.9999987284343,
3Dlights=Headlamp}{width=150pt}{figures/fig13_07_ex_10_3D}}{$\ds\int_{\varphi_1}^{\varphi_2}\int_{\theta_1}^{\theta_2}\int_{\rho_1}^{\rho_2}
h(\rho,\theta,\varphi)\rho^2\sin(\varphi)\ d\rho\ d\theta\ d\varphi$}

}

\exerciseset{In Exercises}{,  a triple integral in cylindrical coordinates is given. Describe the region in space defined by the bounds of the integral.}{

\exercise{$\ds \int_0^{\pi/2}\int_0^2\int_0^2 r\ dz\ dr\ d\theta$}{The region in space is bounded between the planes $z=0$ and $z=2$, inside of the cylinder $x^2+y^2=4$, and the planes $\theta = 0$ and $\theta = \pi/2$: describes a ``wedge'' of a cylinder of height 2 and radius 2; the angle of the wedge is $\pi/2$, or $90^\circ$.}

\exercise{$\ds \int_0^{2\pi}\int_3^4\int_0^5 r\ dz\ dr\ d\theta$}{Bounded between the planes $z=0$ and $z=5$, between the cylinders $x^2+y^2=9$ and $x^2+y^2=16$: describes a ``pipe'' or ``tube'' of length 5, an inner radius of 3 and outer radius of 4.}

\exercise{$\ds \int_0^{2\pi}\int_0^1\int_0^{1-r} r\ dz\ dr\ d\theta$}{Bounded between the plane $z=0$ and the cone $z=1-\sqrt{x^2+y^2}$: describes an inverted cone, with height of 1, point at $(0,0,1)$ and base radius of 1.}

\exercise{$\ds \int_0^{\pi}\int_0^1\int_0^{2-r} r\ dz\ dr\ d\theta$}{Bounded between $y\geq 0$, inside the cylinder $x^2+y^2=1$, above the plane $z=0$ and below the cone $z = 2-\sqrt{x^2+y^2}$: describes cylindrical solid of height 1 and radius 2, topped with an inverted cone of height 1 and base radius 1 with point at $(0,0,2)$.}

\exercise{$\ds \int_0^{\pi}\int_0^3\int_0^{\sqrt{9-r^2}} r\ dz\ dr\ d\theta$}{Describes a quarter of a ball of radius 3, centered at the origin; the quarter resides above the $x$-$y$ plane and above the $x$-$z$ plane. }

\exercise{$\ds \int_0^{2\pi}\int_0^a\int_0^{\sqrt{a^2-r^2}+b} r\ dz\ dr\ d\theta$}{Bounded between the plane $z=0$, inside the cylinder $x^2+y^2 = a^2$, and below the upper hemisphere $z= \sqrt{a^2-x^2-y^2}+b$, with radius $a$ and centered at $(0,0,b)$: describes a cylindrical solid of radius $a$ and height $b$, topped with the upper hemisphere of radius $a$.}

}

\exerciseset{In Exercises}{, a triple integral in spherical coordinates is given. Describe the region in space defined by the bounds of the integral.}{

\exercise{$\ds \int_0^{\pi/2}\int_0^{\pi}\int_0^{1} \rho^2\sin(\varphi)\ d\rho\ d\varphi\ d\theta$}{Describes the portion of the unit ball that resides in the first octant.}

\exercise{$\ds \int_0^{\pi}\int_0^{\pi}\int_1^{1.1} \rho^2\sin(\varphi)\ d\rho\ d\varphi\ d\theta$}{Describes half of a spherical shell (i.e., $y\geq 0$) with inner radius of $1$ and outer radius of $1.1$ centered at the origin.}

\exercise{$\ds \int_0^{2\pi}\int_0^{\pi/4}\int_0^{2} \rho^2\sin(\varphi)\ d\rho\ d\varphi\ d\theta$}{Bounded above the cone $z=\sqrt{x^2+y^2}$ and below the sphere $x^2+y^2+z^2=4$: describes a shape that is somewhat ``diamond''-like; some think of it as looking like an ice cream cone (see \autoref{fig:spherical3}). It describes a cone, where the side makes an angle of $\pi/4$ with the positive $z$-axis, topped by the portion of the ball of radius 2, centered at the origin.}

\exercise{$\ds \int_0^{2\pi}\int_{\pi/6}^{\pi/4}\int_0^{2} \rho^2\sin(\varphi)\ d\rho\ d\varphi\ d\theta$}{It is the region is space bounded below by $z=\sqrt{x^2+y^2}$ and above by the sphere $x^2+y^2+z^2=4$, with the portion above the cone $z=\sqrt3\sqrt{x^2+y^2}$ removed: it describes a cone, where the side makes an angle of $\pi/4$ with the positive $z$-axis, topped by the portion of the ball of radius 2, centered at the origin, with the inner cone with angle $\pi/6$ removed, along with corresponding portion of the ball of radius 2.}

\exercise{$\ds \int_0^{2\pi}\int_{0}^{\pi/6}\int_0^{\sec \varphi} \rho^2\sin(\varphi)\ d\rho\ d\varphi\ d\theta$}{The region in space is bounded below by the cone $z=\sqrt{3}\sqrt{x^2+y^2}$ and above by the plane $z=1$: it describes a cone, with point at the origin, centered along the positive $z$-axis, with height of 1 and base radius  of $\tan(\pi/6) = 1/\sqrt{3}$.}

\exercise{$\ds \int_0^{2\pi}\int_{0}^{\pi/6}\int_0^{a\sec \varphi} \rho^2\sin(\varphi)\ d\rho\ d\varphi\ d\theta$}{The region in space is bounded below by the cone $z=\sqrt{3}\sqrt{x^2+y^2}$ and above by the plane $z=a$: it describes a cone, with point at the origin, centered along the positive $z$-axis, with height of $a$ and base radius  of $a\tan(\pi/6)$.}

}

\exerciseset{In Exercises}{, a solid is described along with its density function. Find the mass of the solid using cylindrical coordinates.}{

\exercise{Bounded by the cylinder $x^2+y^2=4$ and the planes $z=0$ and $z=4$ with density function $\delta(x,y,z) =\sqrt{x^2+y^2}+1$. \label{ex:13_07_ex_23}}{In cylindrical coordinates, the density is $\delta(r,\theta,z) = r+1$. Thus mass is
\[\int_0^{2\pi}\int_0^2\int_0^4 (r+1)r\ dz\ dr\ d\theta = 112\pi/3.\]}

\exercise{Bounded by the cylinders $x^2+y^2=4$ and $x^2+y^2=9$, between the planes $z=0$ and $z=10$ with density function $\delta(x,y,z) =z$. }{In cylindrical coordinates, the density is $\delta(r,\theta,z) = z$. Thus mass is
\[\int_0^{2\pi}\int_2^3\int_0^{10} zr\ dz\ dr\ d\theta = 250\pi.\]}

\exercise{Bounded by $y\geq 0$, the cylinder $x^2+y^2=1$, and between the planes $z=0$ and $z=4-y$ with density function $\delta(x,y,z) =1$. }{In cylindrical coordinates, the density is $\delta(r,\theta,z) = 1$. Thus mass is
\[\int_0^\pi\int_0^1\int_0^{4-r\sin\theta}r\ dz\ dr\ d\theta=2\pi-2/3\approx 5.617.\]}

\exercise{The upper half of the unit ball, bounded between $z= 0$ and $z=\sqrt{1-x^2-y^2}$, with density function $\delta(x,y,z) =1$.\label{ex:13_07_ex_26}}{In cylindrical coordinates, the density is $\delta(r,\theta,z) = 1$. Thus mass is
\[\int_0^{2\pi}\int_0^1\int_{-\sqrt{1-r^2}}^{\sqrt{1-r^2}}r\ dz\ dr\ d\theta=4\pi/3.\]}

}

\input{exercises/13_07_exset_07}
\input{exercises/13_07_exset_08}
\input{exercises/13_07_exset_09}
\exerciseset{In Exercises}{,  a region is space is described. Set up the triple integrals that find the volume of this region using rectangular, cylindrical and spherical coordinates, then comment on which of the three appears easiest to evaluate.}{

\exercise{The region enclosed by the unit sphere, $x^2+y^2+z^2=1$.}{Rectangular: $\int_{-1}^{1}\int_{-\sqrt{1-x^2}}^{\sqrt{1-x^2}}\int_{-\sqrt{1-x^2-y^2}}^{\sqrt{1-x^2-y^2}}\ dz\ dy\ dx$

Cylindrical: $\int_0^{2\pi}\int_0^1\int_{-\sqrt{1-r^2}}^{\sqrt{1-r^2}}r\ dz\ dr\ d\theta$

Spherical: $\int_0^\pi\int_0^{2\pi}\int_0^1 \rho^2\sin(\varphi)\ d\rho\ d\theta\ d\varphi$

Spherical appears simplest, avoiding the integration of square-roots and using techniques such as Substitution; all bounds are constants.}

\exercise{The region enclosed by the cylinder $x^2+y^2=1$ and planes $z=0$ and $z=1$.}{Rectangular: $\int_{-1}^{1}\int_{-\sqrt{1-x^2}}^{\sqrt{1-x^2}}\int_{0}^{1}\ dz\ dy\ dx$


Cylindrical: $\int_0^{2\pi}\int_0^1\int_{0}^{1}r\ dz\ dr\ d\theta$

Spherical: $\int_0^{\pi/4}\int_0^{2\pi}\int_0^{\sec\varphi} \rho^2\sin(\varphi)\ d\rho\ d\theta\ d\varphi + \int_{\pi/4}^{\pi/2}\int_0^{2\pi}\int_0^{\csc\varphi} \rho^2\sin(\varphi)\ d\rho\ d\theta\ d\varphi$

Cylindrical appears simplest, avoiding the integration of square-roots and two triple integrals; all bounds are constants.}

\exercise{The region enclosed by the cone $z=\sqrt{x^2+y^2}$ and plane $z=1$.}{Rectangular: $\int_{-1}^{1}\int_{-\sqrt{1-x^2}}^{\sqrt{1-x^2}}\int_{\sqrt{x^2+y^2}}^{1}\ dz\ dy\ dx$

Cylindrical: $\int_0^{2\pi}\int_0^1\int_{r}^{1}r\ dz\ dr\ d\theta$

Spherical: $\int_0^{\pi/4}\int_0^{2\pi}\int_0^{\sec\varphi} \rho^2\sin(\varphi)\ d\rho\ d\theta\ d\varphi$

Cylindrical appears simplest, avoiding the integration of square-roots that rectangular uses. Spherical is not difficult, though it requires Substitution, an extra step.}

\exercise{The cube enclosed by the planes $x=0$, $x=1$, $y=0$, $y=1$, $z=0$ and $z=1$. (Hint: in spherical, use order of integration $d\rho\ d\varphi\ d\theta$.)}{Rectangular: $\int_{0}^{1}\int_{0}^{1}\int_{0}^{1}\ dz\ dy\ dx$

Cylindrical: $\int_0^{\pi/4}\int_0^{\sec\theta}\int_{0}^{1}r\ dz\ dr\ d\theta + \int_{\pi/4}^{\pi/2}\int_0^{\csc\theta}\int_{0}^{1}r\ dz\ dr\ d\theta$

Spherical: $\int _0^{\pi/4}\int _0^{\tan ^{-1}(\sec \theta)}\int _0^{\sec \varphi}\rho ^2 \sin
   (\varphi)\ d\rho\ d\varphi\ d \theta +$
	
	$\int _0^{\pi/4}\int _{\tan ^{-1}(\sec \theta)}^{\pi/2}\int _0^{\sec\theta\csc\varphi}\rho ^2 \sin
   (\varphi)\ d\rho\ d\varphi\ d \theta +$
	
	$\int _{\pi/4}^{\pi/2}\int _0^{\tan ^{-1}(\csc \theta)}\int _0^{\sec\varphi}\rho ^2 \sin
   (\varphi)\ d\rho\ d\varphi\ d \theta +$
	
	$\int _{\pi/4}^{\pi/2}\int _{\tan ^{-1}(\csc \theta)}^{\pi/2}\int _0^{\csc\theta\csc\varphi}\rho ^2 \sin
   (\varphi)\ d\rho\ d\varphi\ d \theta.$

Rectangular is clearly the simplest.}

}


\exercise{Find the center and radius of the sphere given by the spherical equation
\[
\rho = 4 \sin \phi \cos \theta + 6 \sin \phi \sin \theta - 2 \cos \phi.
\]}{center: $(2,3,-1)$, radius: $\sqrt{14}$}
% \rho^2 = 4\rho\sin\phi\cos\theta + 6\rho\sin\phi\sin\theta - 2\rho\cos\phi
% x^2+y^2+z^2 = 4x + 6y - 2z
% x^2-4x+4-4 + y^2-6y+9-9 z^2+2z+1-1 = 0
% (x-2)^2 + (y-3)^2 + (z+1)^2 = 4+9+1 = 14

% todo the solution to 11.7#29, 31-33
\exercise{Show that for $a \ne 0$, the equation $\rho = 2a \sin \phi \, \cos \theta$ in spherical coordinates describes a sphere centered at $(a,0,0)$ with radius $\abs{a}$.}{}

\exercise{Let $P = (a,\theta,\phi)$ be a point in spherical coordinates, with $a > 0$ and $0 < \phi < \pi$. Then $P$ lies on the sphere $\rho = a$. Since $0 < \phi < \pi$, the line segment from the origin to $P$ can be extended to intersect the cylinder given by $r = a$ (in cylindrical coordinates). Find the cylindrical coordinates of that point of intersection.}{$(a,\theta,a \cot \phi )$}

\exercise{Let $P_1$ and $P_2$ be points whose spherical coordinates are $( \rho_1,\theta_1,\phi_1 )$ and $( \rho_2,\theta_2,\phi_2 )$, respectively. Let $\vecv_1$ be the vector from the origin to $P_1$, and let $\vecv_2$ be the vector from the origin to $P_2$. For the angle $\gamma$ between $\vecv_1$ and $\vecv_2$, show that
  \[
   \cos\gamma=\cos\phi_1\cos\phi_2+\sin\phi_1\sin\phi_2\cos(\theta_2-\theta_1).
  \]
  This formula is used in electrodynamics to prove the addition theorem for spherical harmonics, which provides a general expression for the electrostatic potential at a point due to a unit charge.% See pp.\ 100-102 in \cite{jac}.
}{}

\exercise{Show that the distance $d$ between the points $P_1$ and $P_2$ with cylindrical coordinates $( r_1,\theta_1,z_1 )$ and $( r_2,\theta_2,z_2 )$, respectively, is
  \[
   d = \sqrt{r_1^2 + r_2^2 - 2 r_1\,r_2 \cos (\, \theta_2 - \theta_1 \,)
   + ( z_2 - z_1 )^2} \, .
  \]}{Hint: Use the distance formula for Cartesian coordinates.}

\exercise{Show that the distance $d$ between the points $P_1$ and $P_2$ with spherical coordinates $( \rho_1,\theta_1,\phi_1 )$ and $( \rho_2,\theta_2,\phi_2 )$, respectively, is
  \[
   d = \sqrt{\rho_1^2 + \rho_2^2 - 2 \rho_1\,\rho_2 [ \sin \phi_1 \,
    \sin \phi_2 \,\cos ( \, \theta_2 - \theta_1 \, ) +
    \cos \phi_1 \, \cos \phi_2 ]} \, .
  \]}{}

%\ifthenelse{\boolean{printquestions}}{\columnbreak}{}

% Mecmath follows

\exercise{Prove\label{pr_cyl_jac} \autoref{thm:triple_int_cylindrical} by finding the Jacobian of the cylindrical coordinate transformation.}{}

\exercise{Prove\label{pr_sph_jac} \autoref{thm:triple_int_spherical} by finding the Jacobian of the spherical coordinate transformation.}{}

\exercise{Evaluate $\ds\iint_{R} \sin \left( \frac{x+y}2 \right)\,\cos \left( \frac{x-y}2 \right)\,dA$, where $R$  is the triangle with vertices $(0,0)$, $(2,0)$ and $(1,1)$. (\emph{Hint: Use the change of variables $u=(x+y)/2$, $v=(x-y)/2$.})}{$1-\sin2/2$}
