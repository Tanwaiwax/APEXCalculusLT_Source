\exerciseset{In Exercises}{,  a triple integral in cylindrical coordinates is given. Describe the region in space defined by the bounds of the integral.}{

\exercise{$\ds \int_0^{\pi/2}\int_0^2\int_0^2 r\ dz\ dr\ d\theta$}{The region in space is bounded between the planes $z=0$ and $z=2$, inside of the cylinder $x^2+y^2=4$, and the planes $\theta = 0$ and $\theta = \pi/2$: describes a ``wedge'' of a cylinder of height 2 and radius 2; the angle of the wedge is $\pi/2$, or $90^\circ$.}

\exercise{$\ds \int_0^{2\pi}\int_3^4\int_0^5 r\ dz\ dr\ d\theta$}{Bounded between the planes $z=0$ and $z=5$, between the cylinders $x^2+y^2=9$ and $x^2+y^2=16$: describes a ``pipe'' or ``tube'' of length 5, an inner radius of 3 and outer radius of 4.}

\exercise{$\ds \int_0^{2\pi}\int_0^1\int_0^{1-r} r\ dz\ dr\ d\theta$}{Bounded between the plane $z=0$ and the cone $z=1-\sqrt{x^2+y^2}$: describes an inverted cone, with height of 1, point at $(0,0,1)$ and base radius of 1.}

\exercise{$\ds \int_0^{\pi}\int_0^1\int_0^{2-r} r\ dz\ dr\ d\theta$}{Bounded between $y\geq 0$, inside the cylinder $x^2+y^2=1$, above the plane $z=0$ and below the cone $z = 2-\sqrt{x^2+y^2}$: describes cylindrical solid of height 1 and radius 2, topped with an inverted cone of height 1 and base radius 1 with point at $(0,0,2)$.}

\exercise{$\ds \int_0^{\pi}\int_0^3\int_0^{\sqrt{9-r^2}} r\ dz\ dr\ d\theta$}{Describes a quarter of a ball of radius 3, centered at the origin; the quarter resides above the $x$-$y$ plane and above the $x$-$z$ plane. }

\exercise{$\ds \int_0^{2\pi}\int_0^a\int_0^{\sqrt{a^2-r^2}+b} r\ dz\ dr\ d\theta$}{Bounded between the plane $z=0$, inside the cylinder $x^2+y^2 = a^2$, and below the upper hemisphere $z= \sqrt{a^2-x^2-y^2}+b$, with radius $a$ and centered at $(0,0,b)$: describes a cylindrical solid of radius $a$ and height $b$, topped with the upper hemisphere of radius $a$.}

}
