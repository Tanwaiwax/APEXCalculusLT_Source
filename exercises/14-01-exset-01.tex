\begin{exerciseset}{In Exercises}{, a planar curve $C$ is given along with a surface $f$ that is defined over $C$. Evaluate the line integral $\ds \int_Cf(s)\dd s.$}

\exercise{$C$ is the line segment joining the points $(-2,-1)$ and $(1,2)$; the surface is $f(x,y)=x^2+y^2+2$.}{$12\sqrt{2}$}

\exercise{$C$ is the segment of $y=3x+2$ on $[1,2]$; the surface is $f(x,y)=5x+2y$.}{$41\sqrt{10}/2$}

\exercise{$C$ is the circle with radius 2 centered at the point $(4,2)$; the surface is $f(x,y)=3x-y$.}{$40\pi$}

\exercise{$C$ is the curve given by $\vec r(t)=\bracket{\cos t+t\sin t, \sin t-t\cos t}$ on $[0,2\pi]$; the surface is $f(x,y)=5$.}{$10\pi^2$}

\exercise{$C$ is the piecewise curve composed of the line segments that connect $(0,1)$ to $(1,1)$, then connect $(1,1)$ to $(1,0)$; the surface is $f(x,y)=x+y^2$.}{Over the first subcurve of $C$, the line integral has a value of $3/2$; over the second subcurve, the line integral has a value of $4/3$. The total value of the line integral is thus $17/6$.}

\exercise{$C$ is the piecewise curve composed of the line segment joining the points $(0,0)$ and $(1,1)$, along with the quarter-circle parameterized by $\bracket{\cos t,-\sin t+1}$ on $[0,\pi/2]$(which starts at the point $(1,1)$ and ends at $(0,0)$; the surface is $f(x,y)=x^2+y^2$.}{Over the first subcurve of $C$, the line integral has a value of $2\sqrt{2}/3$; over the second subcurve, the line integral has a value of $\pi-2$. The total value of the line integral is thus $\pi+2\sqrt{2}/3-2$.}

% Mecmath questions follow
\exercise{$C$ is the arc of the unit circle traced from $(1,0)$ to $(0,1)$; the surface is $f(x,y)=xy$.}{$1/2$}

\exercise{$C$ is the line segment from $(0,0)$ to $(1,0)$; the surface is $f(x,y)=\dfrac{x}{x^2 + 1}$}{$(\ln 2)/2$}

\exercise{$C$ is polygonal path from $(0,0)$ to $(3,0)$ to $(3,2)$; the surface is $f(x,y)=2x+y$}{$23$}

\exercise{$C$ is the path from $(2,0)$ counterclockwise along the circle $x^2 + y^2 = 4$ to the point $(-2,0)$ and then back to $(2,0)$ along the $x$-axis; the surface is $f(x,y)=x+y^2$}{$4\pi$}
% C1: <2cos t,2sin t> 0:pi -> int 4cos t+8sin^2 t = 4pi
% C2: <t,0> -2:2 int t = 0

\exercise{$C$ is the curve $y=x^3$ for $0\le x\le1$; the surface is $f(x,y)=\sqrt{1+9xy}$.}{$14/5$}

\exercise{$C$ is the helix $x=\cos t$, $y=\sin t$, $z=t$ for $0\le t\le2\pi$; the surface is $f(x,y,z)=z$}{$2\sqrt2\pi^2$}

\exercise{$C$ is the curve $x=t^2$, $y=t$, $z=1$ for $1\le t\le 2$; the surface is $f(x,y,z)=\dfrac{x}{y} + y + 2yz$}{$(17\sqrt{17}-5\sqrt5)/3$}

\exercise{$C$ is the curve $x=t\sin t$, $y=t\cos t$, $z=\frac{2\sqrt{2}}{3}t^{3/2}$ for $0\le t\le 1$; the surface is $f(x,y,z)=z^2$}{$2/5$}

\end{exerciseset}
