\begin{exerciseset}{In Exercises}{, a vector field $\vec F$ and a curve $C$ are given. Evaluate $\ds\int_C\vec F\cdot\dd\vec r$.}

\exercise{$\vec F =\bracket{y, y^2}$; $C$ is the line segment from $(0,0)$ to $(3,1)$.}{$11/6$. (One parameterization for $C$ is $\vec r(t) =\bracket{3t,t}$ on $0\leq t\leq 1$.)}

\exercise{$\vec F =\bracket{x,x+y}$; $C$ is the portion of the parabola $y=x^2$ from $(0,0)$ to $(1,1)$.}{$5/3$. (One parameterization for $C$ is $\vec r(t) =\bracket{t,t^2}$ on $0\leq t\leq 1$.)}

\exercise{$\vec F =\bracket{y,x}$; $C$ is the top half of the unit circle, beginning at $(1,0)$ and ending at $(-1,0)$.}{$0$. (One parameterization for $C$ is $\vec r(t) =\bracket{\cos t,\sin t}$ on $0\leq t\leq \pi$.)}

\exercise{$\vec F =\bracket{xy,x}$; $C$ is the portion of the curve $y=x^3$ on $-1\leq x\leq 1$.}{$2/5$. (One parameterization for $C$ is $\vec r(t) =\bracket{t,t^3}$ on $-1\leq t\leq 1$.)}

\exercise{$\vec F =\bracket{z,x^2,y}$; $C$ is the line segment from $(1,2,3)$ to $(4,3,2)$.}{$12$. (One parameterization for $C$ is $\vec r(t) =\bracket{1,2,3}+t\bracket{3,1,-1}$ on $0\leq t\leq 1$.)}

\exercise{$\vec F =\bracket{y+z,x+z,x+y}$; $C$ is the helix\\
$\vec r(t) =\bracket{\cos t,\sin t,t/(2\pi)}$ on $0\leq t\leq 2\pi$.}{$1$. %(One parameterization for $C$ is $\vec r(t) =\bracket{1,2,3}+t\bracket{3,1,-1}$ on $0\leq t\leq 1$.)
}

% Mecmath problems follow

\exercise{$\vec F=\veci-\vecj$; $C$ is the line segment from the origin to $(3,2)$}{$1$}

\exercise{$\vec F=y\,\veci-x\,\vecj$; $C$ is the unit circle traced once counterclockwise.}{$-2\pi$}

\exercise{$\vec F=x\,\veci+y\,\vecj$; $C$ is the unit circle traced once counterclockwise.}{$0$}
% C: <cos t,sin t>:  <cos t,sin t>.<-sin t,cos t> = 0

\exercise{$\vec F=(x^2-y)\,\veci+(x-y^2)\,\vecj$; $C$ is the unit circle traced once counterclockwise.}{$2\pi$}

\exercise{$\vec F=xy^2\,\veci+xy^3\,\vecj$; $C$ is the polygonal path from $(0,0)$ to $(1,0)$ to $(0,1)$ to $(0,0)$}{$-1/30$}
% C1: <t,0>: <0,0>.
% C2: <1-t,t>: <(1-t)t^2,(1-t)t^3>.<-1,1> = (1-t)(t^3-t^2) -> -1/30
% C3: <0,1-t>: <0,0>

\exercise{$\vec F=(x^2+y^2)\,\veci$; $C$ is the unit circle centered at $(2,0)$ traced once counterclockwise.}{$0$}

\exercise{$\vec F=\bracket{x^2,xy}$; $C$ is the triangle with vertices $(1,0)$, $(0,1)$, $(-1,0)$ traversed counterclockwise.}{$1/3$}

\exercise{$\vec F=\bracket{y^2,x^2}$; $C$ is the curve $y=1/x$ for $1\le x\le2$, traversed (a) left to right; (b) right to left.}{$-1/2$; $1/2$}

\exercise{$\vec F=\veci-\vecj+\veck$; $C$ is the line segment connecting the origin to $(3,2,1)$}{2}

\exercise{$\vec F=y\,\veci-x\,\vecj+z\,\veck$; $C$ is the helix $x=\cos t$, $y=\sin t$, $z=t$ for $0\le t\le 2\pi$}{$2\pi(\pi-1)$}

\exercise{$\vec F=x\,\veci+y\,\vecj+z\,\veck$; $C$ is the unit circle parallel to the $xy$ plane centered at $(0,0,2)$.}{0}

\exercise{$\vec F=(y-2z)\,\veci+xy\,\vecj+(2xz+y)\,\veck$; $C$ is the curve $x=t$, $y=2t$, $z=t^2-1$ for $0\le t\le1$}{$67/15$}

\exercise{$\vec F=yz\,\veci+xz\,\vecj+xy\,\veck$; $C$ is the polygonal path from $(0,0,0)$ to $(1,0,0)$ to $(1,2,0)$}{0}

\exercise{$\vec F=xy\,\veci+(z-x)\,\vecj+2yz\,\veck$; $C$ is the polygonal path from $(0,0,0)$ to $(1,0,0)$ to $(1,2,0)$ to $(1,2,-2)$}{$6$}

\end{exerciseset}
