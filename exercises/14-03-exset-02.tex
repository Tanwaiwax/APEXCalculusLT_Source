\begin{exerciseset}{In Exercises}{, find the work performed by the force field $\vec F$ moving a particle along the path $C$.}

\exercise{$\vec F =\bracket{y,x^2}$ N; $C$ is the segment of the line $y=x$ from $(0,0)$ to $(1,1)$, where distances are measured in meters.}{$5/6$ joules. (One parametrization for $C$ is $\vec r(t) =\bracket{t,t}$ on $0\leq t\leq 1$.)}

\exercise{$\vec F =\bracket{y,x^2}$ N; $C$ is the portion of $y=\sqrt x$ from $(0,0)$ to $(1,1)$, where distances are measured in meters.}{$13/15$ joules. (One parametrization for $C$ is $\vec r(t) =\bracket{t,\sqrt t}$ on $0\leq t\leq 1$.)}

\exercise{$\vec F =\bracket{2xy,x^2,1}$ lbs; $C$ is the path from $(0,0,0)$ to $(2,4,8)$ via $\vec r(t) =\bracket{t,t^2,t^3}$ on $0\leq t\leq 2$, where distance are measured in feet.}{$24$ ft-lbs. %(One parametrization for $C$ is $\vec r(t) =\bracket{t,\sqrt t}$ on $0\leq t\leq 1$.)
}

\exercise{$\vec F =\bracket{2xy,x^2,1}$ lbs; $C$ is the path from $(0,0,0)$ to $(2,4,8)$ via $\vec r(t) =\bracket{t,2t, 4t}$ on $0\leq t\leq 2$, where distance are measured in feet.}{$24$ ft-lbs. %(One parametrization for $C$ is $\vec r(t) =\bracket{t,\sqrt t}$ on $0\leq t\leq 1$.)
}

% Mecmath problems

\exercise{$\vec F=\bracket{x,xy}$ N; $C$ is the top half of the circle $x^2+y^2=1$ traversed from $(1,0)$ to $(-1,0)$.}{$2/3$ joules}

\exercise{$\vec F=\bracket{x,xy}$ N; $C$ is the the parabola $y=1-x^2$ traversed from $(1,0)$ to $(-1,0)$.}{$8/15$ joules}

\end{exerciseset}
