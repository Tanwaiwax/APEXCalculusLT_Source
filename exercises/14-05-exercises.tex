\printconcepts

\exercise{In your own words, describe what an orientable surface is.}{Answers will vary, though generally should meaningfully include terms like ``two sided''.}

\exercise{Give an example of a non-orientable surface.}{Many possible answers exist; the one given by the book is the Möbius band.}

% 14.4#3
%\exercise{Green's Theorem states, informally, that the circulation around a closed curve that bounds a region $R$ is equal to the sum of \underline{\hskip1in} across $R$.}{the curl of $\vec F$, or $\curl \vec F$}
%
% 14.4#4
%\exercise{The Divergence Theorem states, informally, that the outward flux across a closed curve that bounds a region $R$ is equal to the sum of \underline{\hskip1in} across $R$.}{the divergence of $\vec F$, or $\divv \vec F$}
%
% 14.4#5
%\exercise{Let $\vec F$ be a vector field and let $C_1$ and $C_2$ be any nonintersecting paths except that each starts at point $A$ and ends at point $B$. If \underline{\hskip.5in}$=0$, then $\int_{C_1} \vec F\cdot \vec T\ ds = \int_{C_2} \vec F\cdot \vec T\ ds$.}{$\curl \vec F$}
%
% 14.4#6
%\exercise{Let $\vec F$ be a vector field and let $C_1$ and $C_2$ be any nonintersecting paths except that each starts at point $A$ and ends at point $B$. If \underline{\hskip.5in}$=0$, then $\int_{C_1} \vec F\cdot \vec n\ ds = \int_{C_2} \vec F\cdot \vec n\ ds$.}{$\divv \vec F$}

\printproblems

\exercisesetinstructions{, parameterize the surface defined by the function $z=f(x,y)$ over each of the given regions $R$ of the $x$-$y$ plane.}

\exercise{$z = 3x^2y$;
\begin{enumext}
	\item $R$ is the rectangle bounded by $-1\leq x\leq 1$ and $0\leq y\leq 2$.
	\item	$R$ is the circle of radius 3, centered at $(1,2)$.
	\item	$R$ is the triangle with vertices $(0,0)$, $(1,0)$ and $(0,2)$.
	\item	$R$ is the region bounded by the $x$-axis and the graph of $y = 1-x^2$.
\end{enumext}}{\mbox{}\\[-2\baselineskip]\parbox[t]{\linewidth}{\begin{enumext}
	\item $\vec r(u,v) =\bracket{u,v,3u^2v}$ on $-1\leq u\leq 1$, $0\leq v\leq 2$.
	\item	$\vec r(u,v) =\\\langle3v\cos u+1, 3v\sin u+2,$\\$3(3v\cos u+1)^2(3v\sin u+2)\rangle$, on $0\leq u\leq 2\pi$, $0\leq v\leq 1$.
	\item	$\vec r(u,v) =\bracket{u,v(2-2u),3u^2v(2-2u)}$ on $0\leq u, v\leq 1$.
	\item	$\vec r(u,v) =\bracket{u,v(1-u^2),3u^2v(1-u^2)}$ on $-1\leq u\leq 1$, $0\leq v\leq 1$.
\end{enumext}}}

\exercise{$z = 4x+2y^2$;
\begin{enumext}
	\item $R$ is the rectangle bounded by $1\leq x\leq 4$ and $5\leq y\leq 7$.
	\item	$R$ is the ellipse with major axis of length 8 parallel to the $x$-axis, and minor axis of length 6 parallel to the $y$-axis, centered at the origin.
	\item	$R$ is the triangle with vertices $(0,0)$, $(2,2)$ and $(0,4)$.
	\item	$R$ is the annulus bounded between the circles, centered at the origin, with radius 2 and radius 5.
\end{enumext}}{\mbox{}\\[-2\baselineskip]\parbox[t]{\linewidth}{\begin{enumext}
	\item $\vec r(u,v) =\bracket{u,v,4u+2u^2}$ on $1\leq u\leq 4$, $5\leq v\leq 7$.
	\item	$\vec r(u,v) =\\\bracket{4v\cos u,3v\sin u,16v\cos u+2(3v\sin u)^2}$, on $0\leq u\leq 2\pi$, $0\leq v\leq 1$.
	\item	$\vec r(u,v) =\\\bracket{u,u+v(4-2u),4u+2\big(u+v(4-2u)\big)^2}$ on\\
		$0\leq u\leq 2$, $0\leq v\leq 1$.
	\item	$\vec r(u,v) =\bracket{v\cos u,v\sin u,4v\cos u+2(v\sin u)^2}$ on $0\leq u\leq 2\pi$, $2\leq v\leq 5$.
\end{enumext}}}

\exercisesetend


\begin{exerciseset}{In Exercises}{, a surface $\surfaceS$ in space is described that cannot be defined in terms of a function $z=f(x,y)$. Give a parameterization of \surfaceS.}

\exercise{\surfaceS\ is the rectangle in space with corners at $(0,0,0)$, $(0,2,0)$, $(0,2,1)$ and $(0,0,1)$.}{$\vec r(u,v) =\bracket{0, u, v}$ with $0\leq u\leq 2$, $0\leq v\leq 1$.}

\exercise{\surfaceS\ is the triangle in space with corners at $(1,0,0)$, $(1,0,1)$ and $(0,0,1)$.}{$\vec r(u,v) =\bracket{u, 0, 1-u+vu}$ with $0\leq u\leq 1$, $0\leq v\leq 1$.}

\exercise{\surfaceS\ is the ellipsoid $\ds\frac{x^2}{9} + \frac{y^2}{4}+\frac{z^2}{16} = 1$.}{$\vec r(u,v) =\bracket{3\sin u\cos v, 2\sin u\sin v, 4\cos u}$ with $0\leq u\leq \pi$, $0\leq v\leq 2\pi$.}

\exercise{\surfaceS\ is the elliptic cone $\ds y^2= x^2+\frac{z^2}{16}$, for $-1 \leq y \leq 5$.}{Answers may vary; one solution is\\
$\vec r(u,v) =\bracket{v\cos u, v, 4v\sin u}$ with $0\leq u\leq 2\pi$, $-1\leq v\leq 5$.}

\end{exerciseset}


\input{exercises/14-05-exset-03}

\input{exercises/14-05-exset-04}

\begin{exerciseset}{In Exercises}{, set up the double integral that finds the surface area $S$ of the given surface \surfaceS, then use technology to approximate its value.}

\exercise{\surfaceS\ is the paraboloid $z=x^2+y^2$ over the circular disk of radius 3 centered at the origin.}{$S =\int_0^3\int_0^{2\pi}\sqrt{v^2+4v^4}\dd u\dd v= (37\sqrt{37}-1)\pi/6 \approx 117.319$.}

\exercise{\surfaceS\ is the paraboloid $z=x^2+y^2$ over the triangle with vertices at $(0,0)$, $(0,1)$ and $(1,1)$.}{$S = \int_0^1\int_0^1\sqrt{v^2+4u^2v^2+4v^4}\dd u\dd v \approx 0.931$.}

\exercise{\surfaceS\ is the plane $z=5x-y$ over the region enclosed by the parabola $y=1-x^2$ and the $x$-axis.}{$S =\int_{-1}^1\int_0^{1-v^2}\sqrt{27}\dd u\dd v=4\sqrt3 \approx 6.928$.}

\exercise{\surfaceS\ is the hyperbolic paraboloid $z=x^2-y^2$ over the circular disk of radius 1 centered at the origin.}{$S =\int_0^1\int_{0}^{2\pi}\sqrt{v^2+4v^4}\dd u\dd v = (5\sqrt{5}-1)\pi/6 \approx 5.330$.}

\end{exerciseset}


% 14.3#21
%\exercise{Prove part of \autoref{thm:conservative_field_curl}: let $\vec F =\bracket{M,N,P}$ be a conservative vector field. Show that $\curl \vec F = 0$.}{Since $\vec F$ is conservative, it is the gradient of some potential function. That is, $\nabla f =\bracket{f_x,f_y,f_z} = \vec F =\bracket{M,N,P}$. In particular, $M = f_x$, $N = f_y$ and $P=f_z$.
%
%Note that $\curl \vec F =\bracket{P_y - N_z, M_z-P_x, N_x-M_y } =\bracket{f_{zy} - f_{yz}, f_{xz} - f_{zx}, f_{yx} - f_{xy}}$, which, by \autoref{thm:mixed_partial}, is $\bracket{0,0,0}$.}
