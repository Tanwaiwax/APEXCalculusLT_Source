\printconcepts

\exercise{What are the differences between the Divergence Theorems of \autoref{sec:greensthm} and this section?}{Answers will vary; in \autoref{sec:greensthm}, the Divergence Theorem connects outward flux over a closed curve in the plane to the divergence of the vector field, whereas in this section the Divergence Theorem connects outward flux over a closed surface in space to the divergence of the vector field.}

\exercise{What property of a vector field does the Divergence Theorem relate to flux?}{Divergence.}

\exercise{What property of a vector field does Stokes' Theorem relate to circulation?}{Curl.}

\exercise{Stokes' Theorem is the spatial version of what other theorem?}{Green's Theorem.}

% 14.4#5
%\exercise{Let $\vec F$ be a vector field and let $C_1$ and $C_2$ be any nonintersecting paths except that each starts at point $A$ and ends at point $B$. If \underline{\hskip.5in}$=0$, then $\int_{C_1} \vec F\cdot \vec T\ ds = \int_{C_2} \vec F\cdot \vec T\ ds$.}{$\curl \vec F$}
%
% 14.4#6
%\exercise{Let $\vec F$ be a vector field and let $C_1$ and $C_2$ be any nonintersecting paths except that each starts at point $A$ and ends at point $B$. If \underline{\hskip.5in}$=0$, then $\int_{C_1} \vec F\cdot \vec n\ ds = \int_{C_2} \vec F\cdot \vec n\ ds$.}{$\divv \vec F$}

\printproblems

\input{exercises/14-07-exset-01}

\exercisesetinstructions{, a closed curve $C$ that is the boundary of a surface \surfaceS\ is given along with a vector field $\vec F$. Verify Stokes' Theorem on $C$; that is, show $\oint_C \vec F\cdot\dd\vec r = \iint_{\surfaceS}\bigl(\curl \vec F\,\bigr)\cdot\vec n\dd S$.}

% Mecmath
\exercise{$C$ is the unit circle in the $xy$ plane and $\surfaceS$ upper unit hemisphere; $\vec F=2y\,\veci-x\,\vecj+z\,\veck$}{With an upward normal, both integrals are $-3\pi$.}

% todo solve 15.7#10
\exercise{$C$ is the curve parameterized by $\vec r(t) =\bracket{\cos t, \sin t, 1}$ and $\surfaceS$ is the portion of $z=x^2+y^2$ enclosed by $C$; $\vec F=xy\,\veci+xz\,\vecj+yz\,\veck$}{}

% apex
\exercise{$C$ is the curve parameterized by $\vec r(t) =\bracket{\cos t, \sin t, 1}$ and $\surfaceS$ is the portion of $z=x^2+y^2$ enclosed by $C$; $\vec F =\bracket{z,-x,y}$. 

{\hfill\myincludeasythree{width=90pt,
3Droll=0,
3Dortho=0.004999519791454077,
3Dc2c=0.7499517202377319 0.618543803691864 0.23446987569332123,
3Dcoo=-6.589626312255859 0.43738889694213867 60.739051818847656,
3Droo=200}{width=90pt}{figures/fig14_07_ex_09_3D}\hfill\null}}{Circulation on $C$: $\oint_C \vec F\cdot\dd\vec r = \pi$

$\iint_{\surfaceS}\bigl(\curl \vec F\bigr)\cdot\vec n\dd S = \pi$.}

\exercise{$C$ is the curve parameterized by $\vec r(t) =\bracket{\cos t, \sin t, e^{-1}}$ and $\surfaceS$ is the portion of $z=e^{-x^2-y^2}$ enclosed by $C$; $\vec F =\bracket{-y,x,1}$.

{\hfill\myincludeasythree{width=120pt,
3Droll=0.3839076710105066,
3Dortho=0.004999519791454077,
3Dc2c=0.7258517742156982 0.6457906365394592 0.236841082572937,
3Dcoo=-3.3363380432128906 1.7384066581726074 51.15752410888672,
3Droo=200}{width=120pt}{figures/fig14_07_ex_10_3D}\hfill\null}}{Circulation on $C$: $\oint_C \vec F\cdot\dd\vec r = \pi$

$\iint_{\surfaceS}\bigl(\curl \vec F\bigr)\cdot\vec n\dd S = \pi$.}

\exercise{$C$ is the curve that follows the triangle with vertices at $(0,0,2)$, $(4,0,0)$ and $(0,3,0)$, traversing the the vertices in that order and returning to $(0,0,2)$, and $\surfaceS$ is the portion of the plane $z=2-x/2-2y/3$ enclosed by $C$; $\vec F =\bracket{y,-z,y}$. 

{\hfill\myincludeasythree{width=90pt,
3Droll=-0.8614078233921112,
3Dortho=0.004999519791454077,
3Dc2c=0.7229859232902527 0.6350046396255493 0.2721407115459442,
3Dcoo=52.09402084350586 48.800357818603516 52.90095520019531,
3Droo=200}{width=90pt}{figures/fig14_07_ex_11_3D}\hfill\null}}{Circulation on $C$: The flow along the line from $(0,0,2)$ to $(4,0,0)$ is 0; from $(4,0,0)$ to $(0,3,0)$ it is $-6$, and from $(0,3,0)$ to $(0,0,2)$ it is 6. The total circulation is $0+(-6)+6=0$.

$\iint_{\surfaceS}\bigl(\curl \vec F\bigr)\cdot\vec n\dd S = \iint_{\surfaceS} 0 \dd S = 0$.}

\exercise{$C$ is the curve whose $x$ and $y$ coordinates follow the parabola $y=1-x^2$ from $x=1$ to $x=-1$, then follow the line from $(-1,0)$ back to $(1,0)$, where the $z$ coordinates of $C$ are determined by $f(x,y) = 2x^2+y^2$, and $\surfaceS$ is the portion of $z=2x^2+y^2$ enclosed by $C$; $\vec F =\bracket{y^2+z,x,x^2-y}$.

{\hfill\myincludeasythree{width=90pt,
3Droll=-0.8758858500809813,
3Dortho=0.004999519791454077,
3Dc2c=0.7286513447761536 0.5758740901947021 0.3707241415977478,
3Dcoo=-4.221027374267578 12.607401847839355 52.789634704589844,
3Droo=200}{width=90pt}{figures/fig14_07_ex_12_3D}\hfill\null}}{Circulation on $C$: The flow along the parabola is $-32/15$; the flow along the line is $4/3$. The total circulation is $4/3-32/15 = -4/5$.

$\iint_{\surfaceS}\bigl(\curl \vec F\bigr)\cdot\vec n\dd S = -4/5$.}

\exercisesetend


\input{exercises/14-07-exset-03}

\input{exercises/14-07-exset-04}

\exercisesetinstructions[Exercises]{\ are designed to challenge your understanding and require no computation.}

\exercise{Let \surfaceS\ be any closed surface enclosing a domain $D$. Consider $\vec F_1 =\bracket{x,0,0}$ and $\vec F_2=\bracket{y,y^2,z-2yz}$.

These fields are clearly very different. Why is it that the total outward flux of each field across \surfaceS\ is the same?}{Each field has a divergence of 1; by the Divergence Theorem, the total outward flux across \surfaceS\ is $\iint_D 1\dd S$ for each field.}

\exercise{\mbox{}\\[-2\baselineskip]\parbox[t]{\linewidth}{\begin{enumext}
	\item Green's Theorem can be used to find the area of a region enclosed by a curve by evaluating a line integral with the appropriate choice of vector field $\vec F$. What condition on $\vec F$ makes this possible?
	
	\item	Likewise, Stokes' Theorem can be used to find the surface area of a region enclosed by a curve in space by evaluating a line integral with the appropriate choice of vector field $\vec F$. What condition on $\vec F$ makes this possible?
\end{enumext}}}{\mbox{}\\[-2\baselineskip]\parbox[t]{\linewidth}{\begin{enumext}
	\item $\curl\vec F = 1$. 
	\item	$\curl\vec F\cdot \vec n = 1$, where $\vec n$ is a unit vector normal to $\surfaceS$.
\end{enumext}}}

\exercise{The Divergence Theorem establishes equality between a particular double integral and a particular triple integral. What types of circumstances would lead one to choose to evaluate the triple integral over the double integral? }{Answers will vary. Often the closed surface \surfaceS\ is composed of several smooth surfaces. To measure total outward flux, this may require evaluating multiple double integrals. Each double integral requires the parameterization of a surface and the computation of the cross product of partial derivatives. One triple integral may require less work, especially as the divergence of a vector field is generally easy to compute.}

\exercise{Stokes' Theorem establishes equality between a particular line integral and a particular double integral. What types of circumstances would lead one to choose to evaluate the double integral over the line integral? }{Answers will vary. Often the closed curve $C$ is composed of several smooth curves. To measure the total circulation, one may have to evaluate line integrals along each curve. Each line integral requires the parameterization of its curve. It may be less work to evaluate one single double (i.e., surface) integral.}

% Mecmath, commented out for parity
%\exercise{Show that the flux of any constant vector field through any closed surface is zero.}{}

\exercisesetend


% 14.3#21
%\exercise{Prove part of \autoref{thm:conservative_field_curl}: let $\vec F =\bracket{M,N,P}$ be a conservative vector field. Show that $\curl \vec F = 0$.}{Since $\vec F$ is conservative, it is the gradient of some potential function. That is, $\nabla f = \bracket{f_x,f_y,f_z}= \vec F =\bracket{M, N, P}$. In particular, $M = f_x$, $N = f_y$ and $P=f_z$.
%
%Note that $\curl \vec F =\bracket{P_y - N_z, M_z-P_x, N_x-M_y}=\bracket{f_{zy} - f_{yz}, f_{xz} - f_{zx}, f_{yx} - f_{xy}}$, which, by \autoref{thm:mixed_partial}, is $\bracket{0,0,0}$.}

\exercise{Construct a M\"{o}bius strip from a piece of paper, then draw a line down its center (like the dotted line in \autoref{fig:mobius}(b)). Cut the M\"{o}bius strip along that center line completely around the strip. How many surfaces does this result in? How would you describe them? Are they orientable?}{}

\exercise{Use a computer algebra system to plot the M\"{o}bius strip parametrized as $\vecr(u,v) =$
  \[
   \bracket{\cos u \,(1+v\cos \tfrac{u}{2}),\sin u \,(1+v\cos \tfrac{u}{2}),v\sin \tfrac{u}{2}},
  \]
  where $0\le u\le2\pi$, $-\tfrac12\le v\le\tfrac12$ }{}

% todo solution to 15.7#33,34
\exercise{Let $\surfaceS$ be a closed surface and $\vec F$ a smooth vector field. Show that $\ds\iint_{\surfaceS} (\curl{\vec F}\,)\cdot\vecn\dd S= 0$. (\emph{Hint: Split $\surfaceS$ in half.})}{}

\exercise{Show that Green's Theorem is a special case of Stokes' Theorem. (Hint: Think of how a vector field $f(x,y)=P(x,y)i+Q(x,y)j$ in $\mathbb{R}^{2}$ can be extended in a natural way to be a vector field in $\mathbb{R}^{3}$.)}{}
