\exercisesetinstructions{, a closed curve $C$ that is the boundary of a surface \surfaceS\ is given along with a vector field $\vec F$. Verify Stokes' Theorem on $C$; that is, show $\oint_C \vec F\cdot\dd\vec r = \iint_{\surfaceS}\bigl(\curl \vec F\,\bigr)\cdot\vec n\dd S$.}

% Mecmath
\exercise{$C$ is the unit circle in the $xy$ plane and $\surfaceS$ upper unit hemisphere; $\vec F=2y\,\veci-x\,\vecj+z\,\veck$}{With an upward normal, both integrals are $-3\pi$.}

% todo solve 15.7#10
\exercise{$C$ is the curve parameterized by $\vec r(t) =\bracket{\cos t, \sin t, 1}$ and $\surfaceS$ is the portion of $z=x^2+y^2$ enclosed by $C$; $\vec F=xy\,\veci+xz\,\vecj+yz\,\veck$}{}

% apex
\exercise{$C$ is the curve parameterized by $\vec r(t) =\bracket{\cos t, \sin t, 1}$ and $\surfaceS$ is the portion of $z=x^2+y^2$ enclosed by $C$; $\vec F =\bracket{z,-x,y}$. 

{\hfill\myincludeasythree{width=90pt,
3Droll=0,
3Dortho=0.004999519791454077,
3Dc2c=0.7499517202377319 0.618543803691864 0.23446987569332123,
3Dcoo=-6.589626312255859 0.43738889694213867 60.739051818847656,
3Droo=200}{width=90pt}{figures/fig14_07_ex_09_3D}\hfill\null}}{Circulation on $C$: $\oint_C \vec F\cdot\dd\vec r = \pi$

$\iint_{\surfaceS}\bigl(\curl \vec F\bigr)\cdot\vec n\dd S = \pi$.}

\exercise{$C$ is the curve parameterized by $\vec r(t) =\bracket{\cos t, \sin t, e^{-1}}$ and $\surfaceS$ is the portion of $z=e^{-x^2-y^2}$ enclosed by $C$; $\vec F =\bracket{-y,x,1}$.

{\hfill\myincludeasythree{width=120pt,
3Droll=0.3839076710105066,
3Dortho=0.004999519791454077,
3Dc2c=0.7258517742156982 0.6457906365394592 0.236841082572937,
3Dcoo=-3.3363380432128906 1.7384066581726074 51.15752410888672,
3Droo=200}{width=120pt}{figures/fig14_07_ex_10_3D}\hfill\null}}{Circulation on $C$: $\oint_C \vec F\cdot\dd\vec r = \pi$

$\iint_{\surfaceS}\bigl(\curl \vec F\bigr)\cdot\vec n\dd S = \pi$.}

\exercise{$C$ is the curve that follows the triangle with vertices at $(0,0,2)$, $(4,0,0)$ and $(0,3,0)$, traversing the the vertices in that order and returning to $(0,0,2)$, and $\surfaceS$ is the portion of the plane $z=2-x/2-2y/3$ enclosed by $C$; $\vec F =\bracket{y,-z,y}$. 

{\hfill\myincludeasythree{width=90pt,
3Droll=-0.8614078233921112,
3Dortho=0.004999519791454077,
3Dc2c=0.7229859232902527 0.6350046396255493 0.2721407115459442,
3Dcoo=52.09402084350586 48.800357818603516 52.90095520019531,
3Droo=200}{width=90pt}{figures/fig14_07_ex_11_3D}\hfill\null}}{Circulation on $C$: The flow along the line from $(0,0,2)$ to $(4,0,0)$ is 0; from $(4,0,0)$ to $(0,3,0)$ it is $-6$, and from $(0,3,0)$ to $(0,0,2)$ it is 6. The total circulation is $0+(-6)+6=0$.

$\iint_{\surfaceS}\bigl(\curl \vec F\bigr)\cdot\vec n\dd S = \iint_{\surfaceS} 0 \dd S = 0$.}

\exercise{$C$ is the curve whose $x$ and $y$ coordinates follow the parabola $y=1-x^2$ from $x=1$ to $x=-1$, then follow the line from $(-1,0)$ back to $(1,0)$, where the $z$ coordinates of $C$ are determined by $f(x,y) = 2x^2+y^2$, and $\surfaceS$ is the portion of $z=2x^2+y^2$ enclosed by $C$; $\vec F =\bracket{y^2+z,x,x^2-y}$.

{\hfill\myincludeasythree{width=90pt,
3Droll=-0.8758858500809813,
3Dortho=0.004999519791454077,
3Dc2c=0.7286513447761536 0.5758740901947021 0.3707241415977478,
3Dcoo=-4.221027374267578 12.607401847839355 52.789634704589844,
3Droo=200}{width=90pt}{figures/fig14_07_ex_12_3D}\hfill\null}}{Circulation on $C$: The flow along the parabola is $-32/15$; the flow along the line is $4/3$. The total circulation is $4/3-32/15 = -4/5$.

$\iint_{\surfaceS}\bigl(\curl \vec F\bigr)\cdot\vec n\dd S = -4/5$.}

\exercisesetend
