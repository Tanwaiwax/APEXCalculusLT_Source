\exercisesetinstructions[Exercises]{\ are designed to challenge your understanding and require no computation.}

\exercise{Let \surfaceS\ be any closed surface enclosing a domain $D$. Consider $\vec F_1 =\bracket{x,0,0}$ and $\vec F_2=\bracket{y,y^2,z-2yz}$.

These fields are clearly very different. Why is it that the total outward flux of each field across \surfaceS\ is the same?}{Each field has a divergence of 1; by the Divergence Theorem, the total outward flux across \surfaceS\ is $\iint_D 1\dd S$ for each field.}

\exercise{\mbox{}\\[-2\baselineskip]\parbox[t]{\linewidth}{\begin{enumext}
	\item Green's Theorem can be used to find the area of a region enclosed by a curve by evaluating a line integral with the appropriate choice of vector field $\vec F$. What condition on $\vec F$ makes this possible?
	
	\item	Likewise, Stokes' Theorem can be used to find the surface area of a region enclosed by a curve in space by evaluating a line integral with the appropriate choice of vector field $\vec F$. What condition on $\vec F$ makes this possible?
\end{enumext}}}{\mbox{}\\[-2\baselineskip]\parbox[t]{\linewidth}{\begin{enumext}
	\item $\curl\vec F = 1$. 
	\item	$\curl\vec F\cdot \vec n = 1$, where $\vec n$ is a unit vector normal to $\surfaceS$.
\end{enumext}}}

\exercise{The Divergence Theorem establishes equality between a particular double integral and a particular triple integral. What types of circumstances would lead one to choose to evaluate the triple integral over the double integral? }{Answers will vary. Often the closed surface \surfaceS\ is composed of several smooth surfaces. To measure total outward flux, this may require evaluating multiple double integrals. Each double integral requires the parameterization of a surface and the computation of the cross product of partial derivatives. One triple integral may require less work, especially as the divergence of a vector field is generally easy to compute.}

\exercise{Stokes' Theorem establishes equality between a particular line integral and a particular double integral. What types of circumstances would lead one to choose to evaluate the double integral over the line integral? }{Answers will vary. Often the closed curve $C$ is composed of several smooth curves. To measure the total circulation, one may have to evaluate line integrals along each curve. Each line integral requires the parameterization of its curve. It may be less work to evaluate one single double (i.e., surface) integral.}

% Mecmath, commented out for parity
%\exercise{Show that the flux of any constant vector field through any closed surface is zero.}{}

\exercisesetend
