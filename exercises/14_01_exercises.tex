\printconcepts

\exercise{Explain how a line integral can be used to find the area under a curve.}{When $C$ is a curve in the plane and $f$ is a surface defined over $C$, then $\int_C f(s)\ ds$ describes the area under the spatial curve that lies on $f$, over $C$.}

\exercise{How does the evaluation of a line integral given as $\int_C f(s)\ ds$ differ from a line integral given as $\oint_C f(s)\ ds$?}{The evaluation is the same. The $\oint$ notation signifies that the curve $C$ is a closed curve, though the evaluation is the same.}

\exercise{Why are most line integrals evaluated using\\
\autoref{thm:line1} instead of ``directly'' as $\int_C f(s)\ ds$?}{The variable $s$ denotes the arc-length parameter, which is generally difficult to use. The Key Idea allows one to parametrize a curve using another, ideally easier-to-use, parameter.}

\exercise{Sketch a closed, piecewise smooth curve composed of three subcurves.}{Answers will vary.}

\printproblems

\input{exercises/14_01_exset_01}

\input{exercises/14_01_exset_02}

\input{exercises/14_01_exset_03}

\input{exercises/14_01_exset_04}

% Mecmath problems follow

\exercise{Use a line integral to find the lateral surface area of the part of the cylinder $x^2 + y^2 = 4$ below the plane $x+2y+z=6$ and above the $xy$-plane.}{$24\pi$}

% for a specific example, reverse the direction of the parameterization
%\exercise{Verify\label{pr_double_circ} that the value of the line integral in \autoref{exmp_lineintcyl} is unchanged when using the parametrization of the circle $C$ given in \autoeqref{eqn:lineintcylcwise}.}{}

% immediately following, generalize.  eqn:reversec is x'=x(a+b-t), y'=y(a+b-t)
%\exercise{Prove\label{pr_rev_int} that $\int_C f(x,y)\,ds = \int_{-C} f(x,y)\,ds$. (\emph{Hint: Use \autoeqref{eqn:reversec}.})}{}

% todo solution to 15.1#30,31
\exercise{Prove that the Riemann integral $\int_a^b f(x)\,dx$ is a special case of a line integral.}{}

\exercise{Let $C$ be a curve whose arc length is $L$. Show that $\int_C 1\,ds = L$.}{}
