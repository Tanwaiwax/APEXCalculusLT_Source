\printconcepts

\exercise{Give two quantities that can be represented by a vector field in the plane or in space.}{Answers will vary. Appropriate answers include velocities of moving particles (air, water, etc.); gravitational or electromagnetic forces.}

\exercise{In your own words, describe what it means for a vector field to have a negative divergence at a point.}{Specific answers will vary, though should relate to the idea that ``more of the vector field is moving into that point than out of that point.''}

\exercise{In your own words, describe what it means for a vector field to have a negative curl at a point.}{Specific answers will vary, though should relate to the idea that the vector field is spinning clockwise at that point.}

\exercise{The divergence of a vector field $\vec F$ at a particular point is 0. Does this mean that $\vec F$ is incompressible? Why/why not?}{No; to be incompressible, the divergence needs to be 0 everywhere, not just at one point.}

\printproblems

\exerciseset{In Exercises}{, sketch the given vector field over the rectangle with opposite corners $(-2,-2)$ and $(2,2)$, sketching one vector for every point with integer coordinates (i.e., at $(0,0)$, $(1,2)$, etc.).}{

\exercise{$\vec F =\bracket{x,0}$}{Correct answers should look similar to

\begin{tikzpicture}
\begin{axis}[width=1.16\marginparwidth,height=1.16\marginparwidth,
tick label style={font=\scriptsize},
axis y line=middle,axis x line=middle,
name=myplot,axis on top,axis equal,
xtick={-2,2},ytick={-2,2},
ymin=-2.5,ymax=2.5,xmin=-2.5,xmax=2.5]
\foreach \x in {-2,-1,1,2} {
    \foreach \y in {-2,-1,...,2.11} {
		\edef\vx{((\x)/4)}
		\edef\vy{(0)}
        \edef\temp{\noexpand\draw[->,{\colorone}]
        (axis cs:{(\x)-(\vx)},{(\y)-(\vy)}) -- (axis cs:{(\x)+(\vx)},{(\y)+(\vy)});}
        \temp
    }
}
\end{axis}
\node [right] at (myplot.right of origin) {\scriptsize $x$};
\node [above] at (myplot.above origin) {\scriptsize $y$};
\end{tikzpicture}}

\exercise{$\vec F =\bracket{0,x}$}{Correct answers should look similar to

\begin{tikzpicture}
\begin{axis}[width=1.16\marginparwidth,height=1.16\marginparwidth,
tick label style={font=\scriptsize},
axis y line=middle,axis x line=middle,
name=myplot,axis on top,axis equal,
xtick={-2,2},ytick={-2,2},
ymin=-2.5,ymax=2.5,xmin=-2.5,xmax=2.5]
\foreach \x in {-2,-1,1,2} {
    \foreach \y in {-2,-1,...,2.11} {
		\edef\vx{(0)}
		\edef\vy{((\x)/5)}
        \edef\temp{\noexpand\draw[->,{\colorone}]
        (axis cs:{(\x)-(\vx)},{(\y)-(\vy)}) -- (axis cs:{(\x)+(\vx)},{(\y)+(\vy)});}
        \temp
    }
}
\end{axis}
\node [right] at (myplot.right of origin) {\scriptsize $x$};
\node [above] at (myplot.above origin) {\scriptsize $y$};
\end{tikzpicture}}

\exercise{$\vec F =\bracket{1,-1}$}{Correct answers should look similar to

\begin{tikzpicture}
\begin{axis}[width=1.16\marginparwidth,height=1.16\marginparwidth,
tick label style={font=\scriptsize},
axis y line=middle,axis x line=middle,
name=myplot,axis on top,axis equal,
xtick={-2,2},ytick={-2,2},
ymin=-2.5,ymax=2.5,xmin=-2.5,xmax=2.5]
\foreach \x in {-2,-1,...,2.11} {
    \foreach \y in {-2,-1,...,2.11} {
		\edef\vx{(.2)}
		\edef\vy{(-.2)}
        \edef\temp{\noexpand\draw[->,{\colorone}]
        (axis cs:{(\x)-(\vx)},{(\y)-(\vy)}) -- (axis cs:{(\x)+(\vx)},{(\y)+(\vy)});}
        \temp
    }
}
\end{axis}
\node [right] at (myplot.right of origin) {\scriptsize $x$};
\node [above] at (myplot.above origin) {\scriptsize $y$};
\end{tikzpicture}}

\exercise{$\vec F =\bracket{y^2,1}$}{Correct answers should look similar to

\begin{tikzpicture}
\begin{axis}[width=1.16\marginparwidth,height=1.16\marginparwidth,
tick label style={font=\scriptsize},
axis y line=middle,axis x line=middle,
name=myplot,axis on top,axis equal,
xtick={-2,2},ytick={-2,2},
ymin=-2.5,ymax=2.5,xmin=-2.5,xmax=2.5]
\foreach \x in {-2,-1,...,2.11} {
    \foreach \y in {-2,-1,...,2.11} {
		\edef\vx{((\y)*(\y)/8)}
		\edef\vy{(1/8)}
        \edef\temp{\noexpand\draw[->,{\colorone}]
        (axis cs:{(\x)-(\vx)},{(\y)-(\vy)}) -- (axis cs:{(\x)+(\vx)},{(\y)+(\vy)});}
        \temp
    }
}
\end{axis}
\node [right] at (myplot.right of origin) {\scriptsize $x$};
\node [above] at (myplot.above origin) {\scriptsize $y$};
\end{tikzpicture}}

}


\exerciseset{In Exercises}{, find the divergence and curl of the given vector field.}{

\exercise{$\vec F =\bracket{x,y^2}$}{$\divv \vec F = 1+2y$; $\curl \vec F = 0$}

\exercise{$\vec F =\bracket{-y^2,x}$}{$\divv \vec F = 0$; $\curl \vec F = 1+2y$}

\exercise{$\vec F =\bracket{\cos (xy), \sin (xy)}$}{$\divv \vec F = x\cos(xy)-y\sin(xy)$; $\curl \vec F = y\cos(xy)+x\sin(xy)$}

\exercise{$\ds\vec F =\bracket{\frac{-2x}{(x^2+y^2)^2},\frac{-2y}{(x^2+y^2)^2}}$}{$\divv \vec F = \frac{4}{(x^2+y^2)^2}$; $\curl \vec F = 0$}

\exercise{$\ds\vec F =\bracket{x+y,y+z,x+z}$}{$\divv \vec F = 3$; $\curl \vec F =\bracket{-1,-1,-1}$}

\exercise{$\ds\vec F =\bracket{x^2+z^2,x^2+y^2,y^2+z^2}$}{$\divv \vec F = 2x+2y+2z$; $\curl \vec F =\bracket{2y,2z,2x}$}

\exercise{$\vec F = \nabla f$, where $f(x,y) = \frac12x^2+\frac13y^3$.}{$\divv \vec F = 1+2y$; $\curl\vec F = 0$}

\exercise{$\vec F = \nabla f$, where $f(x,y) = x^2y$.}{$\divv \vec F = 2y$; $\curl\vec F = 0$}

\exercise{$\vec F = \nabla f$, where $f(x,y,z) = x^2y+\sin z$.}{$\divv \vec F = 2y-\sin z$; $\curl\vec F = \vec 0$}

\exercise{$\vec F = \nabla f$, where $\ds f(x,y,z) = \frac1{x^2+y^2+z^2}$.}{$\divv \vec F = \frac{2}{(x^2+y^2+z^2)^2}$; $\curl\vec F = \vec 0$}

}


\exerciseset{In Exercises}{, let $\vecr(x,y,z) = x\,\veci + y\,\vecj +z\,\veck$ and $r = \norm{\vecr}$.  Prove the given formula.}{

% todo solutions to 15.2#19-26
\exercise{$\nabla\,(1/r) = -\vecr/r^3$}{}

\exercise{$\nabla\cdot(\vecr/r^3)=0$}{}

\exercise{$\nabla\,(\ln r) = \vecr/r^2$}{}

% cut for parity
%\exercise{$\divv\,(\vecF + \vecG) = \divv\vecF + \divv\vecG$}{}

\exercise{$\curl\,(\vecF + \vecG) = \curl\vecF + \curl\vecG$}{}

\exercise{$\divv\,(f\,\vecF) = f\,\divv\vecF +\vecF\cdot\nabla f$}{}

\exercise{$\divv\,(\vecF\times\vecG) = \vecG\cdot\curl\vecF-\vecF\cdot\curl\vecG$}{}

\exercise{$\divv\,(\nabla f\times\nabla g) = 0$}{}

\exercise{$\curl\,(f\,\vecF) = f\,\curl\vecF + (\nabla f\,)\times\vecF$}{}

}
