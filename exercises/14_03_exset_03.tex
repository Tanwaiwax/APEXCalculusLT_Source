\exerciseset{In Exercises}{, a conservative vector field $\vec F$ and a curve $C$ are given.
\begin{enumerate}[label={(\alph*)}]
\item	Find a potential function $f$ for $\vec F$. 
\item	Compute $\curl \vec F$.
\item	Evaluate $\ds\int_C \vec F\cdot d\vec r$ directly, i.e., using \autoref{idea:line2}.
\item	Evaluate $\ds\int_C \vec F\cdot d\vec r$  using the Fundamental Theorem of Line Integrals.
\end{enumerate}}{

\exercise{$\vec F =\bracket{y+1,x}$, $C$ is the line segment from $(0,1)$ to $(1,0)$.}{\mbox{}\\[-2\baselineskip]\begin{enumerate}
\item		$f(x,y) = xy+x$
\item	$\curl \vec F = 0$.
\item		$1$. (One parametrization for $C$ is $\vec r(t) =\bracket{t,t-1}$ on $0\leq t\leq 1$.)
\item	$1$ ($f(0,1)=0$ and $f(1,0)=1$)
\end{enumerate}}

\exercise{$\vec F =\bracket{2x+y,2y+x}$, $C$ is curve parametrized by $\vec r(t) =\bracket{t^2-t,t^3-t}$ on $0\leq t\leq 1$. }{\mbox{}\\[-2\baselineskip]\begin{enumerate}
\item		$f(x,y) = x^2+xy+y^2$
\item	$\curl \vec F = 0$.
\item		$0$.% (One parametrization for $C$ is $\vec r(t) =\bracket{t,-1 t}$ on $0\leq t\leq 1$.)
\item	$0$ ($f(0,0)=0$)
\end{enumerate}}

\exercise{$\vec F =\bracket{2xyz,x^2z,x^2y}$, $C$ is curve parametrized by $\vec r(t) =\bracket{2t+1,3t-1,t}$ on $0\leq t\leq 2$.}{\mbox{}\\[-2\baselineskip]\begin{enumerate}
\item		$f(x,y) = x^2yz$
\item	$\curl \vec F = \vec 0$.
\item		$250$.% (One parametrization for $C$ is $\vec r(t) =\bracket{t,-1 t}$ on $0\leq t\leq 1$.)
\item	$250$ ($f(1,-1,0)=0$ $f(5,5,2)=250$)
\end{enumerate}}

\exercise{$\vec F =\bracket{2x,2y,2z}$, $C$ is curve parametrized by $\vec r(t) =\bracket{\cos t,\sin t,\sin(2t)}$ on $0\leq t\leq 2\pi$.}{\mbox{}\\[-2\baselineskip]\begin{enumerate}
\item		$f(x,y) = x^2+y^2+z^2$
\item	$\curl \vec F = \vec 0$.
\item		$0$.% (One parametrization for $C$ is $\vec r(t) =\bracket{t,-1 t}$ on $0\leq t\leq 1$.)
\item	$0$ ($f(1,0,0)=1$)
\end{enumerate}}

% Mecmath problems

\exercise{$\vec F=\bracket{x^2+y^2,2xy}$, $C$ is the unit circle traced once counterclockwise.}{\mbox{}\\[-2\baselineskip]\begin{enumerate}
\item $f(x,y)=xy^2+x^3/3$
\item $\curl\vec F=\vec0$
\item $0$.
\item $0$ ($f(1,0)=1/3$)
\end{enumerate}}

\exercise{$\vec F=\bracket{x^2 + y^2,2xy}$, $C$ is the unit circle traced from $(1,0)$ to $(-1,0)$.}{\mbox{}\\[-2\baselineskip]\begin{enumerate}
\item $f(x,y)=xy^2+x^3/3$
\item $\curl\vec F=\vec0$
\item $2/3$.
\item $2/3$ ($f(1,0)=1/3$, $f(-1,0)=-1/3$)
\end{enumerate}}

% Bevelaqua problems
\exercise{$\vec F=\bracket{xy^2,x^2y}$, $C$ is $x=t\cos t$, $y=t\sin t$ for $0\le t\le5\pi/4$}{\mbox{}\\[-2\baselineskip]\begin{enumerate}
\item $f(x,y)=x^2y^2/2$
\item $\curl\vec F=\vec0$
\item $625\pi^4/2048$
\item $625\pi^4/2048$ ($f(0,0)=0$,\\
$f(-5\pi/4\sqrt2,-5\pi/4\sqrt2)=625\pi^4/2048$
\end{enumerate}}

\exercise{$\vec F=\bracket{2xe^y,2y+x^2e^y}$, $C$ is the polygonal path from $(1,0)$ to $(2,1)$ to $(0,0)$.}{\mbox{}\\[-2\baselineskip]\begin{enumerate}
\item $f(x,y)=y^2+x^2e^y$
\item $\curl\vec F=\vec0$
\item $-1$
\item $-1$ ($f(1,0)=1$, $f(0,0)=0$)
\end{enumerate}}

}
