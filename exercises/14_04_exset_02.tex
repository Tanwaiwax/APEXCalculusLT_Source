\begin{exerciseset}{In Exercises}{, a vector field $\vec F$ and a closed curve $C$, enclosing a region $R$, are given. Verify Green's Theorem by evaluating both $\oint_C\vec F\cdot\dd\vec r$ and $\iint_R \curl \vec F\dd A$, showing they are equal.}

\exercise{$\vec F =\bracket{x-y,x+y}$; $C$ is the closed curve composed of the parabola $y=x^2$ on $0\leq x\leq 2$ followed by the line segment from $(2,4)$ to $(0,0)$.}{The line integral $\oint_C\vec F\cdot\dd\vec r$, over the parabola, is $38/3$; over the line, it is $-10$. The total line integral is thus $38/3-10 = 8/3$. The double integral of $\curl \vec F = 2$ over $R$ also has value $8/3$.}

\exercise{$\vec F =\bracket{-y,x}$; $C$ is the unit circle.}{Both the line integral and double integral have value of $2\pi$.}

\exercise{$\vec F =\bracket{0,x^2}$; $C$ the triangle with corners at $(0,0)$, $(2,0)$ and $(1,1)$.}{Three line integrals need to be computed to compute $\oint_C \vec F\cdot\dd\vec r$. It does not matter which corner one starts from first, but be sure to proceed around the triangle in a counterclockwise fashion.

From $(0,0)$ to $(2,0)$, the line integral has a value of 0. From $(2,0)$ to $(1,1)$ the integral has a value of $7/3$. From $(1,1)$ to $(0,0)$ the line integral has a value of $-1/3$. Total value is 2.

The double integral of $\curl\vec F$ over $R$ also has value 2.}

\exercise{$\vec F =\bracket{x+y,2x}$; $C$ the curve that starts at $(0,1)$, follows the parabola $y=(x-1)^2$ to $(3,4)$, then follows a line back to $(0,1)$.}{Two line integrals need to be computed to compute $\oint_C \vec F\cdot\dd\vec r$. 
Along the parabola, the line integral has value $25.5$. Along the line, the line integral has value $-21$. Together, the total value is $4.5$

The double integral of $\curl\vec F$ over $R$ also has value $4.5$.}

% Mecmath problems

\exercise{$\vec F=\bracket{x^2-y^2,2xy}$; $C$ is the boundary of $R = \{\,(x,y): 0 \le x \le 1,2x^2 \le y \le 2x \,\}$}{$16/15$}

% todo find the solution to 15.4#18
\exercise{$\vec F=\bracket{x^2 y,2xy}$; $C$ is the boundary of $R = \{\,(x,y): 0 \le x \le 1,x^2 \le y \le x \,\}$}{}

\exercise{$\vec F=\bracket{2y,-3x}$; $C$ is the circle $x^2 + y^2 = 1$}{$-5\pi$}

% todo find the solution to 15.4#20
\exercise{$\vec F=\bracket{(e^{x^2}+y^2),(e^{y^2}+x^2)}$; $C$ is the boundary of the triangle with vertices $(0,0)$, $(4,0)$ and $(0,4)$}{}

% Bevelaqua problems

\exercise{$\vec F=\bracket{xy,y}$; $C$ is the square with vertices $(0,0)$, $(1,0)$, $(1,1)$, $(0,1)$.}{$-1/2$}

\exercise{$\vec F=\bracket{y^2,x^2}$; $C$ is the triangle with vertices $(0,0)$, $(1,0)$, $(1,1)$.}{$1/3$}

\exercise{$\vec F=\bracket{y\sin x,xy}$; $C$ is the boundary of the region bordered by $\cos x$ and $\sin x$ for $\frac\pi4\le x\le\frac{5\pi}4$.}{$-\pi/2$}

\exercise{$\vec F=\bracket{y^3,\sin y-x^3}$; where $C$ is the boundary of the annular region between $x^2+y^2=1/4$ and $x^2+y^2=1$, traversed so that the region is always on the left.}{$-45\pi/32$}

% Mecmath problems
\exercise{$\vec F=\bracket{e^x \sin y,y^3 + e^x \cos y}$; $C$ is the boundary of the rectangle with vertices $(1,-1)$, $(1,1)$, $(-1,1)$ and $(-1,-1)$, traversed counterclockwise.}{0}

\exercise{$\vec F=\bracket{\frac{-y}{x^2+y^2},\frac{x}{x^2+y^2}}$; $C$ is the boundary of the annulus $R =\{\,(x,y): 1/4 \le x^2 + y^2 \le 1\,\}$ traversed so that $R$ is always on the left.}{$0$}

\end{exerciseset}

% not in the set, but related

\exercise{For $\vec F=\bracket{y^2,x^2}$, evaluate $\ds\oint_C\vecf\cdot\dd\vecr$ where $C$ is the boundary of the half disk of radius $1$ centered at the origin, in the first and second quadrants, traversed \emph{clockwise}.}{$4/3$}

\exercise{Repeat the previous problem, but use the unit half-disk (centered at the origin) in the fourth and first quadrants.}{$-4/3$}

% without these ending lines, I get an underfull badness 10000 warning.  IDK.
