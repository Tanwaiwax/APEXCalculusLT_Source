\exerciseset{In Exercises}{, a closed curve $C$ enclosing a region $R$ is given. Find the area of $R$ by computing $\oint_C \vec F\cdot d\vec r$ for an appropriate choice of vector field $\vec F$.}{

\exercise{$C$ is the ellipse parametrized by $\vec r(t) =\bracket{4\cos t,3\sin t}$ on $0\leq t\leq 2\pi$. }{Any choice of $\vec F$ is appropriate as long as $\curl \vec F = 1$. When $\vec F =\bracket{-y/2,x/2}$, the integrand of the line integral is simply 6. The area of $R$ is $12\pi$.}

\exercise{$C$ is the curve parametrized by $\vec r(t) =\bracket{\cos t,\sin (2t)}$ on $-\pi/2\leq t\leq \pi/2$. }{Any choice of $\vec F$ is appropriate as long as $\curl \vec F = 1$. The choices of  $\vec F =\bracket{-y,0}$ and $\bracket{0,x}$ each lead to reasonable integrands. The area of $R$ is $4/3$.}

\exercise{$C$ is the curve parametrized by %$\vec r(t) =\bracket{\cos t,\sin (2t)}$
$\vec r(t)=\bracket{3t^2-2t-t^3,2(t-1)^2}$
on $0\leq t\leq 2$. }{Any choice of $\vec F$ is appropriate as long as $\curl \vec F = 1$. The choices of  $\vec F =\bracket{-y,0}$, $\bracket{0,x}$ and $\bracket{-y/2,x/2}$ each lead to reasonable integrands. The area of $R$ is $16/15$.}

\exercise{$C$ is the curve parametrized by $\vec r(t) =\bracket{2\cos t+\frac1{10}\cos(10t),2\sin t+\frac1{10}\sin (10t)}$ on $0\leq t\leq 2\pi$. }{Any choice of $\vec F$ is appropriate as long as $\curl \vec F = 1$. The choice of  $\vec F =\bracket{-y/2,x/2}$ leads to a reasonable integrand after simplification. The area of $R$ is $41\pi/10$.}

}
