\begin{exerciseset}{In Exercises}{, a vector field $\vec F$ and a closed curve $C$, enclosing a region $R$, are given. Verify the Divergence Theorem by evaluating both $\oint_C\vec F\cdot \vec n\dd s$ and $\iint_R \divv \vec F\dd A$, showing they are equal.}

\exercise{$\vec F =\bracket{x-y,x+y}$; $C$ is the closed curve composed of the parabola $y=x^2$ on $0\leq x\leq 2$ followed by the line segment from $(2,4)$ to $(0,0)$.}{The line integral $\oint_C\vec F\cdot \vec n\dd s$, over the parabola, is $-22/3$; over the line, it is $10$. The total line integral is thus $-22/3+10 = 8/3$. The double integral of $\divv \vec F = 2$ over $R$ also has value $8/3$.}

\exercise{$\vec F =\bracket{-y,x}$; $C$ is the unit circle.}{Both the line integral and double integral have value of $0$.}

\exercise{$\vec F =\bracket{0,y^2}$; $C$ the triangle with corners at $(0,0)$, $(2,0)$ and $(1,1)$.}{Three line integrals need to be computed to compute $\oint_C \vec F\cdot \vec n\dd s$. It does not matter which corner one starts from first, but be sure to proceed around the triangle in a counterclockwise fashion.

From $(0,0)$ to $(2,0)$, the line integral has a value of 0. From $(2,0)$ to $(1,1)$ the integral has a value of $1/3$. From $(1,1)$ to $(0,0)$ the line integral has a value of $1/3$. Total value is $2/3$.

The double integral of $\divv\vec F$ over $R$ also has value $2/3$.}

\exercise{$\vec F =\bracket{x^2/2,y^2/2}$; $C$ the curve that starts at $(0,1)$, follows the parabola $y=(x-1)^2$ to $(3,4)$, then follows a line back to $(0,1)$.}{Two line integrals need to be computed to compute $\oint_C \vec F\cdot \vec n\dd s$.
Along the parabola, the line integral has value $159/20$. Along the line, the line integral has value $6$. Together, the total value is $279/20$.

The double integral of $\divv\vec F$ over $R$ also has value $279/20$.}

\end{exerciseset}
