\exerciseset{In Exercises}{, find the surface area $S$ of the given surface \surfaceS. (The associated integrals are computable without the assistance of technology.)}{

\exercise{\surfaceS\ is the plane $z=2x+3y$ over the rectangle $-1\leq x\leq 1$, $2\leq y \leq 3$.}{$S = 2\sqrt{14}$.}

\exercise{\surfaceS\ is the plane $z=x+2y$ over the triangle with vertices at $(0,0)$, $(1,0)$ and $(0,1)$.}{$S = \sqrt{6}/2$.}

\exercise{\surfaceS\ is the plane $z=x+y$ over the circular disk, centered at the origin, with radius 2.}{$S = 4\sqrt{3}\pi$.}

\exercise{\surfaceS\ is the plane $z=x+y$ over the annulus bounded by the circles, centered at the origin, with radius 1 and radius 2.}{$S = 3\sqrt{3}\pi$.}

% Mecmath
\exercise{\surfaceS\ is a sphere of radius $r$. (\emph{Hint: Use spherical coordinates to parametrize the sphere.})}{$4\pi r^2$}

\exercise{\surfaceS\ is a right circular cone of radius $R$ and height $h$. (\emph{Hint: Use the parametrization $x=r\cos\theta$, $y=r\sin\theta$, $z=\frac{h}{R}r$, for $0 \le r \le R$ and $0 \le \theta \le 2\pi$.})}{$\pi R \sqrt{h^2 + R^2}$}

}

% not part of the set, but related

% todo Tim figure out how to display 15.5#23
\exercise{The ellipsoid\index{ellipsoid} $\frac{x^2}{a^2}+\frac{y^2}{b^2}+\frac{z^2}{c^2}=1$ can be parametrized using \emph{ellipsoidal coordinates}\index{coordinates!ellipsoidal}
 \[
  x=a\sin\phi\,\cos\theta , y=b\sin\phi\,\sin\theta , z=c\cos\phi,
 \]
 for $0 \le \theta \le 2\pi$ and $0 \le \phi \le \pi$.
 Show that the surface area of the ellipsoid is given by
 \[%begin{multline*}
  \int_0^{\pi} \int_0^{2\pi} \sin\phi%\times\\
  \sqrt{a^2 b^2 \cos^2 \phi + c^2 (a^2 \sin^2 \theta + b^2 \cos^2 \theta )
  \sin^2 \phi} \,\,d\theta\,d\phi.
 \]%end{multline*}
 (Note: The above double integral can not be evaluated by elementary means. For specific values of $a$, $b$ and $c$ it can be evaluated using numerical methods. An alternative is to express the surface area in terms of \emph{elliptic integrals}.)%, (see \cite[\S\,III.7]{bow}.)
}{}

% todo solution to 15.5#23,24
\exercise{Use \autoref{thm:par_surface_area} to prove that the surface area $S$ over a region $R$ in $\mathbb{R}^{2}$ of a surface $z=f(x,y)$ is given by the formula
  \[
   S = \iint_R \sqrt{1 + \left( \tfrac{\partial f}{\partial x\vphantom{y}} \right)^2 +
   \left( \tfrac{\partial f}{\partial y} \right)^2} \,\,dA .
  \]
  (\emph{Hint: Think of the parametrization of the surface.})}{}
