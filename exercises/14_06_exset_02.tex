\begin{exerciseset}{In Exercises}{, a surface \surfaceS\ and a vector field $\vec F$ are given. Compute the flux of $\vec F$ across $\surfaceS$. (If \surfaceS\ is not a closed surface, choose $\vec n$ so that it has a positive $z$-component, unless otherwise indicated.)}

\exercise{\surfaceS\ is the plane $f(x,y) = 3x+y$ on $0\leq x\leq 1$, $1\leq y\leq 4$; $\vec F =\bracket{x^2,-z,2y}$.}{$24$}

\exercise{\surfaceS\ is the plane $f(x,y) = 8-x-y$ over the triangle with vertices at $(0,0)$, $(1,0)$ and $(1,5)$; $\vec F =\bracket{3,1,2}$.}{$15$}

\exercise{\surfaceS\ is the paraboloid $f(x,y) = x^2+y^2$ over the unit disk; $\vec F =\bracket{1,0,0}$.}{$0$}

\exercise{\surfaceS\ is the unit sphere; $\vec F =\bracket{y-z,z-x,x-y}$.}{$0$}

\exercise{\surfaceS\ is the square in space with corners at $(0,0,0)$, $(1,0,0)$, $(1,0,1)$ and $(0,0,1)$ (choose $\vec n$ such that it has a positive $y$-component); $\vec F =\bracket{0,-z,y}$.}{$-1/2$}

\exercise{\surfaceS\ is the disk in the $y$-$z$ plane with radius 1, centered at $(0,1,1)$ (choose $\vec n$ such that it has a positive $x$-component); $\vec F =\bracket{y,z,x}$.}{$\pi$}

\exercise{\surfaceS\ is the closed surface composed of $\surfaceS_1$, whose boundary is the ellipse in the $x$-$y$ plane described by $\frac{x^2}{25}+\frac{y^2}9 = 1$ and $\surfaceS_2$, part of the elliptical paraboloid $f(x,y) = 1-\frac{x^2}{25}-\frac{y^2}9$ (see graph); $\vec F =\bracket{5,2,3}$.

{\hfill\myincludeasythree{width=120pt,
3Droll=0,
3Dortho=0.004999519791454077,
3Dc2c=0.6666666865348816 0.6666666865348816 0.3333333134651184,
3Dcoo=-14.339106559753418 -14.168757438659668 57.01565933227539,
3Droo=149.9999987284343,
3Dlights=Headlamp}{width=120pt}{figures/fig14_06_ex_13_3D}\hfill}
}{$0$; the flux over $\surfaceS_1$ is $-45\pi$ and the flux over $\surfaceS_2$ is $45\pi$.}

\exercise{\surfaceS\ is the closed surface composed of $\surfaceS_1$, part of the unit sphere and $\surfaceS_2$, part of the plane $z=1/2$ (see graph); $\vec F =\bracket{x,-y,z}$.

{\hfill\myincludeasythree{width=120pt,
3Droll=0,
3Dortho=0.0049995193257927895,
3Dc2c=0.6666666865348816 0.6666666865348816 0.3333333134651184,
3Dcoo=0.05395209789276123 -1.5381652116775513 2.9678030014038086,
3Droo=149.9999987284343,
3Dlights=Headlamp}{width=120pt}{figures/fig14_06_ex_14_3D}\hfill}
}{$9\pi/8$; the flux over $\surfaceS_1$ is $3\pi/4$ (use $\vec r(u,v) =\bracket{\sin u\cos v,\sin u\sin v,\cos u}$ on $\pi/3\leq u\leq \pi$, $0\leq v\leq 2\pi$) and the flux over $\surfaceS_2$ is $3\pi/8$ (use $\vec r(u,v) =\bracket{v\sqrt{3}\cos (u)/2, v\sqrt{3}\sin(u)/2,1/2}$ for $0\leq u\leq 2\pi$, $0\leq v\leq 1$.}

% Mecmath

\exercise{\surfaceS\ is boundary of the solid cube $S = \{\, (x,y,z): 0 \le x,y,z \le 1 \,\}$; $\vec F=\bracket{x,y,z}$. Note that there will be a different outward unit normal vector to each of the six faces of the cube.}{$3$}

\exercise{\surfaceS\ is the part of the plane $6x+3y+2z=6$ with $x \ge 0$, $y \ge 0$, and $z \ge 0$, with the outward unit normal $\vecn$ pointing in the positive $z$ direction; $\vec F=x^2\veci+xy\vecj+z\veck$.}{$7/4$}

\end{exerciseset}
