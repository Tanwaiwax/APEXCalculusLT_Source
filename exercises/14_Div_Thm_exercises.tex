\printproblems

\exerciseset{In Exercises}{, use the Divergence Theorem to evaluate the surface integral $\iint_{\Sigma} \Dotprod{\vecf}{d\sigma}$ of the given vector field $\vecf(x,y,z)$ over the surface $\Sigma$.}{

\exercise{$\vecf(x,y,z) = x\veci + 2y\vecj + 3z\veck$, $\Sigma: x^2 + y^2 + z^2 = 9$}{$216\pi$}

\exercise{$\vecf(x,y,z) = x\veci + y\vecj + z\veck$, $\Sigma:$ boundary of the solid cube $S = \lbrace\, (x,y,z): 0 \le x,y,z \le 1 \,\rbrace$}{3}

\exercise{$\vecf(x,y,z) = x^3\veci + y^3\vecj + z^3\veck$, $\Sigma: x^2 + y^2 + z^2 = 1$}{$12\pi/5$}

\exercise{$\vecf(x,y,z) = 2\veci + 3\vecj + 5\veck$, $\Sigma: x^2 + y^2 + z^2 = 1$}{}

}

\exercise{Show that the flux of any constant vector field through any closed surface is zero.}{}

\exercise{Evaluate the surface integral from Exercise 2 \emph{without} using the Divergence Theorem, i.e. using only \autoref{defn:surfintreal}, as in \autoref{exmp_surfintex}. Note that there will be a different outward unit normal vector to each of the six faces of the cube.}{}

\exercise{Evaluate the surface integral $\iint_{\Sigma} \Dotprod{\vecf}{d\sigma}$, where $\vecf(x,y,z) = x^2 \veci + xy\vecj + z\veck$ and $\Sigma$ is the part of the plane $6x+3y+2z=6$ with $x \ge 0$, $y \ge 0$, and $z \ge 0$, with the outward unit normal $\vecn$ pointing in the positive $z$ direction.}{$15/4$}

\exercise{Use a surface integral to show that the surface area of a sphere of radius $r$ is $4\pi r^2$. (\emph{Hint: Use spherical coordinates to parametrize the sphere.})}{}

\exercise{Use a surface integral to show that the surface area of a right circular cone of radius $R$ and height $h$ is $\pi R \sqrt{h^2 + R^2}$. (\emph{Hint: Use the parametrization $x=r\cos\theta$, $y=r\sin\theta$, $z=\frac{h}{R}r$, for $0 \le r \le R$ and $0 \le \theta \le 2\pi$.})}{}

\exercise{The ellipsoid\index{ellipsoid} $\frac{x^2}{a^2}+\frac{y^2}{b^2}+\frac{z^2}{c^2}=1$ can be parametrized using \emph{ellipsoidal coordinates}\index{coordinates!ellipsoidal}
 \[
  x=a\sin\phi\,\cos\theta , y=b\sin\phi\,\sin\theta , z=c\cos\phi,
 \]
 for $0 \le \theta \le 2\pi$ and $0 \le \phi \le \pi$.
 Show that the surface area $S$ of the ellipsoid is
 \[
  S = \int_0^{\pi} \int_0^{2\pi} \sin\phi \sqrt{a^2 b^2 \cos^2 \phi + c^2 (a^2 \sin^2 \theta + b^2 \cos^2 \theta )
  \sin^2 \phi} \,\,d\theta\,d\phi.
 \]
 (Note: The above double integral can not be evaluated by elementary means. For specific values of $a$, $b$ and $c$ it can be evaluated using numerical methods. An alternative is to express the surface area in terms of \emph{elliptic integrals}.)%, (see \cite[\S\,III.7]{bow}.)
}{}

\exercise{Use \autoref{defn:surfintreal} to prove that the surface area $S$ over a region $R$ in $\mathbb{R}^{2}$ of a surface $z=f(x,y)$ is given by the formula
  \[
   S = \iint_R \sqrt{1 + \left( \tfrac{\partial f}{\partial x\vphantom{y}} \right)^2 +
   \left( \tfrac{\partial f}{\partial y} \right)^2} \,\,dA .
  \]
  (\emph{Hint: Think of the parametrization of the surface.})}{}
