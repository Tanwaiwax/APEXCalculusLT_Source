\printproblems

\exercise{Evaluate $\ds\oint_C (x^2 + y^2 )\,dx + 2xy\,dy$ for $C:$ $x=\cos t$, $y=\sin t$, $0\le t\le 2\pi$.}{0}

\exercise{Evaluate $\ds\int_C (x^2 + y^2 )\,dx + 2xy\,dy$ for $C:$ $x=\cos t$, $y=\sin t$, $0\le t\le \pi$.}{}

\exercise{Is there a potential $F(x,y)$ for $\vecf(x,y) = y\,\veci - x\,\vecj$? If so, find one.}{No.}

\exercise{Is there a potential $F(x,y)$ for $\vecf(x,y) = x\,\veci - y\,\vecj$? If so, find one.}{Yes. $F(x,y)=x^2/2-y^2/2$.}

\exercise{Is there a potential $F(x,y)$ for $\vecf(x,y) = xy^2\,\veci + x^3 y\,\vecj$? If so, find one.}{No.}

\exercise{Let $\vecf(x,y)$ and $\vecg(x,y)$ be vector fields, let $a$ and $b$ be constants, and let $C$ be a curve in $\mathbb{R}^2$. Show that
  \[
   \int_{C}(a\,\vecf \pm b\,\vecg)\cdot d\vecr = a \int_{C}\vecf\cdot d\vecr\pm
   b \int_{C}\vecg\cdot d\vecr .
  \]}{}

\exercise{Let $C$ be a curve whose arc length is $L$. Show that $\int_C 1\,ds = L$.}{}

\exercise{Let $f(x,y)$ and $g(x,y)$ be continuously differentiable real-valued functions in a region $R$. Show that
  \[
   \oint_{C}(f\,\nabla g)\cdot d\vecr = -\oint_{C}(g\,\nabla f)\cdot d\vecr
  \]
  for any closed curve $C$ in $R$. %(\emph{Hint: Use Exercise 21 in Section 2.4.})
  % Show that $\nabla(fg)=f\nabla g+g\nabla f
  }{}

\exercise{\label{contra_green}Let $\vecf(x,y)=\frac{-y}{x^2 + y^2}\,\veci + \frac{x}{x^2 + y^2}\,\vecj$ for all $(x,y)\ne(0,0)$, and $C:$ $x=\cos t$, $y=\sin t$, $0\le t\le 2\pi$.
  \begin{enumerate}
   \item Show that $\vecf = \nabla F$, for $F(x,y) = \tan^{-1}(y/x)$.
   \item Show that $\ds\oint_{C}\vecf\cdot d\vecr = 2\pi$. Does this contradict \autoref{cor:lineintsuffpath}? Explain.
  \end{enumerate}}{(b) No. \emph{Hint:} Think of how $F$ is defined.}

\exercise{Let $g(x)$ and $h(y)$ be differentiable functions, and let $\vecf(x,y)=h(y)\,\veci + g(x)\,\vecj$. Is it possible for \vecf\ to have a potential $F(x,y)$? If so, find an example. You may assume that $F$ would be smooth. (\emph{Hint: Consider the mixed partial derivatives of $F$.})}{Yes. $F(x,y)=axy+bx+cy+d$.}
