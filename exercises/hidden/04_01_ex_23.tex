{Show graphically what happens when Newton's Method is used at different $x_0$ for the function shown.
\iflatexml\begin{enumerate}\else\begin{enumerate*}\fi
\item $x_0=0$
\item $x_0=1$
\item $x_0=3$
\item $x_0=4$
\item $x_0=5$
\iflatexml\end{enumerate}\else\end{enumerate*}\fi

\begin{tikzpicture}
\begin{axis}[width=\marginparwidth, tick label style={font=\scriptsize},
			minor x tick num=1, axis y line=middle, axis x line=middle, ymin=-2,
			ymax=2, xmin=-2, xmax=7, name=myplot]
\addplot [{\colorone},thick,smooth] coordinates {(-2,.2)(-1,.5)(0,1)(1,1.5)(2,0)(4,-1)(6,0)(7,2)};
%\draw [{\colorone},thick] (axis cs:0,1) parabola [bend at end] (axis cs:1,1.5);
%\draw [{\colorone},thick] (axis cs:1,1.5) parabola (axis cs:2,0);
%\draw [{\colorone},thick] (axis cs:2,0) parabola [bend at end] (axis cs:4,-3);
%\draw [{\colorone},thick] (axis cs:4,-3) parabola (axis cs:7,3.75);
%\filldraw [black] (axis cs:2,86) circle (1pt);
%\filldraw [black] (axis cs:3,6) circle (1pt);
\end{axis}

\node [right] at (myplot.right of origin) {\scriptsize $x$};
\node [above] at (myplot.above origin) {\scriptsize $y$};
\end{tikzpicture}
}
{\begin{enumerate}
\item $x_n\to-\infty$
\item $x_1$ is undefined
\item $x_n\to2$
\item $x_1$ is undefined
\item $x_n\to6$
\end{enumerate}
}
