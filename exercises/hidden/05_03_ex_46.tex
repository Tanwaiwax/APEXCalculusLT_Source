{A car accelerates from 0 to 40 mph in 30 seconds. The speedometer reading at each 5 second interval during this time is given in the table below. Estimate how far the car travels during this 30 second period using the velocities at:
\begin{enumerate}
\item The beginning of each time interval.
\item The end of each time interval.
\end{enumerate}

\begin{center}
\begin{tabular} {|c|c|c|c|c|c|c|c|}
\hline
$\mathbf{t}$ (sec)&0&5&10&15&20&25&30\\ \hline
$\mathbf{v}$ (mph)&0&6&14&23&30&36&40\\ \hline
\end{tabular}
\end{center}}
{\begin{enumerate}
\item	$(5\text{ s})((0+6+14+23+30+36)\text{ mph})=545\frac{\text{mi}~\text{s}}{\text{hr}}\times\frac{1\text{ hr}}{3600\text{ s}}\times{5280\text{ ft}}{1\text{ mi}}=799\text{ ft}$
\item	$(5\text{ s})((6+14+23+30+36+40)\text{ mph})=585\frac{\text{mi}~\text{s}}{\text{hr}}\times\frac{1\text{ hr}}{3600\text{ s}}\times{5280\text{ ft}}{1\text{ mi}}=858\text{ ft}$
\end{enumerate}}
