%%% This was originally a style with \usepackage, but inputing is generally
%%% equivalent.  The only real difference is how latexml handles style files.
%%% So we'll input this document as a header instead,
%%% and save \usepackage{customstyle}
%%% for things latexml is having trouble with.
%%% This does mean that the distinction between APEX_format and Header_Calculus
%%% is no longer important, and mostly historical.
%%% The same could be said for Page_Size_Calculus, but it makes some sense to have
%%% that specifically dealing with the page size geometries and nothing else

% do we want to print the keys for the labels? if so, uncomment
%\usepackage[notref,notcite]{showkeys}

%%%%
%% Begins the usepackage section
%%%%

\RequirePackage{amsmath}
\RequirePackage{amssymb}
\RequirePackage{amsthm}

\RequirePackage{graphicx}
\RequirePackage{multicol}
\RequirePackage{tikz}
\usetikzlibrary{calc}
\RequirePackage{makeidx}

\RequirePackage[normalem]{ulem}

%linkbordercolor=white
%\RequirePackage{fancyhdr}
\RequirePackage{calc}

\RequirePackage{latexml}
\lxDocumentID{apex}

%\newcounter{examplecounter}[section]
%\renewcommand{\theexamplecounter}{\arabic{examplecounter}}
%\setcounter{examplecounter}{0}

\newtheoremstyle{apexExample}% name
  {0pt}% Space above, empty = `usual value'
  {0pt}% Space below
  {}% Body font
  {}% Indent amount (empty = no indent, \parindent = para indent)
  {\bfseries}% Thm head font
  {}% Punctuation after thm head
  {\newline}% Space after thm head: " " = normal interword space; \newline = linebreak
%  {\makebox[80pt][l]{\textbf{#1 \arabic{examplecounter}}}\thmnote{\textbf{#3}}}% Thm head spec
  {\makebox[80pt][l]{\textbf{#1 #2}}\thmnote{\textbf{#3}}}% Thm head spec

\newtheoremstyle{apex}% name
  {0pt}% Space above, empty = `usual value'
  {0pt}% Space below
  {}% Body font
  {}% Indent amount (empty = no indent, \parindent = para indent)
  {\bfseries}% Thm head font
  {}% Punctuation after thm head
  {\newline}% Space after thm head: " " = normal interword space; \newline = linebreak
  {\thmname{\textbf{#1}}\thmnumber{ \textbf{#2}}\thmnote{\hskip 20pt \textbf{#3}}}% Thm head spec

\theoremstyle{apexExample}
\newtheorem{exampleEnv}{Example}[section]
%\renewcommand{\theexampleEnv}{\arabic{exampleEnv}}
\newcommand{\exampleEnvautorefname}{Example}
\theoremstyle{apex}

\newcommand{\sectionautorefname}{Section} % the default is lowercase


\makeatletter
\renewenvironment{proof}[1][\proofname]{\par
  \pushQED{\qed}%
  \normalfont \topsep6\p@\@plus6\p@\relax
  \trivlist
  \item[\hskip\labelsep
        \bfseries
    #1]~\\* % something is needed to be able to get a newline
}{%
  \popQED\endtrivlist\@endpefalse
}
\makeatother
\renewcommand{\qedsymbol}{\ensuremath{\square}}


\makeindex

\newcommand{\apex}{A\kern -1pt \lower -2pt\hbox{P}\kern -4pt \lower .7ex\hbox{E}\kern -1pt X}

\renewcommand{\partname}{\protect\thispagestyle{empty}}
\renewcommand{\thepart}{}


%%%%
%%  Create boolean for whether or not to print 3D graphics. 
%%  Also creates command to switch back and forth; "looks better."
%%%%
\newboolean{in_threeD}
\setboolean{in_threeD}{true}
\newcommand{\usethreeDgraphics}{\setboolean{in_threeD}{true}}
\newcommand{\usetwoDgraphics}{\setboolean{in_threeD}{false}}


%%%%
%% Commands to determine whether we print in color or black and white
%%%%

\usepackage{pgfplots}
\pgfplotsset{compat=1.8}

\newboolean{inColor}
\setboolean{inColor}{true}

\message{message at 114}

\pgfplotsset{colormap={coloronemap}{rgb=(.4,.4,1); rgb=(.8,.8,1)}}
\pgfplotsset{colormap={colortwomap}{rgb=(1,.4,.4); rgb=(1,.8,.8)}}
%\usepgfplotslibrary{external}
% only needed for external tikz pictures (and not liked by latexml)
% see http://tex.stackexchange.com/a/1475/107497
\usetikzlibrary{calc}
\usetikzlibrary{shadings}

\message{message at 122}

% these will be renewcommanded
\newcommand{\colorone}{blue}
\newcommand{\colortwo}{red}
\newcommand{\colorthree}{green}
\newcommand{\coloronefill}{blue!15!white}
\newcommand{\colortwofill}{red!15!white}
\newcommand{\colormapone}{rgb=(.4,.4,1); rgb=(.8,.8,1)}
\newcommand{\colormaptwo}{rgb=(1,.4,.4); rgb=(1,.8,.8)}
\newcommand{\colormapplaneone}{rgb=(.7,.7,1); rgb=(.9,.9,1)}
\definecolor{colormaponebottom}{rgb}{.4,.4,1}
\definecolor{colormaponetop}{rgb}{.8,.8,1}
\definecolor{colormaptwobottom}{rgb}{1,.4,.4}
\definecolor{colormaptwotop}{rgb}{1,.8,.8}

% determines the line colors for color and black and white lines.
\newcommand{\colorlinecolor}{blue!95!black!30}
\newcommand{\bwlinecolor}{black!30}

% sets the line color to be in color, as a default
\newcommand{\thelinecolor}{\colorlinecolor}

\newcommand{\setcolorlinecolor}[1]{\renewcommand{\colorlinecolor}{#1}\renewcommand{\thelinecolor}{\colorlinecolor}}
\newcommand{\setbwlinecolor}[1]{\renewcommand{\bwlinecolor}{#1}\renewcommand{\thelinecolor}{\bwlinecolor}}

% this allows the above default to be overriden by using
% the \printincolor and \printinblackandwhite commands
% anywhere in the file. This allows you to switch back
% and forth between bw and color. (Who would want to?)
\newcommand{\colornamesuffix}{}

\newcommand{\printincolor}{\setboolean{inColor}{true}%
\renewcommand{\thelinecolor}{\colorlinecolor}
\renewcommand{\colornamesuffix}{}
% aforementioned renewcommanding
\renewcommand{\colorone}{blue}
\renewcommand{\colortwo}{red}
\renewcommand{\colorthree}{green}
\renewcommand{\coloronefill}{blue!15!white}
\renewcommand{\colortwofill}{red!15!white}
\renewcommand{\colormapone}{rgb=(.4,.4,1); rgb=(.8,.8,1)}
\renewcommand{\colormaptwo}{rgb=(1,.4,.4); rgb=(1,.8,.8)}
\renewcommand{\colormapplaneone}{rgb=(.7,.7,1); rgb=(.9,.9,1)}
\definecolor{colormaponebottom}{rgb}{.4,.4,1}
\definecolor{colormaponetop}{rgb}{.8,.8,1}
\definecolor{colormaptwobottom}{rgb}{1,.4,.4}
\definecolor{colormaptwotop}{rgb}{1,.8,.8}
}

\newcommand{\printinblackandwhite}{\setboolean{inColor}{false}%
\renewcommand{\thelinecolor}{\bwlinecolor}
\renewcommand{\colornamesuffix}{BW}
% the counter to the above renewcommanding
\renewcommand{\colorone}{black}
\renewcommand{\colortwo}{black!50!white}
\renewcommand{\colorthree}{black!25!white}
\renewcommand{\coloronefill}{black!15!white}
\renewcommand{\colortwofill}{black!05!white}
\renewcommand{\colormapone}{rgb=(.4,.4,.4); rgb=(.7,.7,.7)}
\renewcommand{\colormaptwo}{rgb=(.6,.6,.6); rgb=(.9,.9,.9)}
\renewcommand{\colormapplaneone}{rgb=(.8,.8,.8); rgb=(.95,.95,.95)}
\definecolor{colormaponebottom}{rgb}{.4,.4,.4}
\definecolor{colormaponetop}{rgb}{.7,.7,.7}
\definecolor{colormaptwobottom}{rgb}{.6,.6,.6}
\definecolor{colormaptwotop}{rgb}{.9,.9,.9}
}

\newcommand{\threedlines}[4][]{\draw [dashed,#1] (axis cs: #2,#3,#4) -- (axis cs: #2,#3,0) -- (axis cs: #2,0,0)  (axis cs: #2,#3,0)--(axis cs:0,#3,0);}

\newcommand{\mydraw}{\draw (axis cs:0,0,0) -- (axis cs:1,1,0);}

\newcommand{\myincludegraphics}[2][]{\includegraphics[#1]{#2\colornamesuffix}}

% the default is printing in color
\printincolor


%%%
%% Creates a lot of measurements - lengths - to use
%% later on. Explained when a value is set.
%%%


\newlength{\topmarginlength} 
\newlength{\bottommarginlength}
\newlength{\oddpagemarginlength}
\newlength{\evenpagemarginlength}
\newlength{\marginlinelength}
\newlength{\innerpagemarginlength}


% measures how far from the text the example line is to be drawn
\setlength{\marginlinelength}{2pt}

% the height of the top margin
% used in calculating the lines for examples
\setlength{\topmarginlength}{-1in-\voffset}

% the length of the bottom margin (ish)
% actually starts at the top of the page, moves
% through the top margin length then the text height.
\setlength{\bottommarginlength}{-1in-\textheight-2\baselineskip-\voffset-\headheight-\headsep-\topmargin}

% the length of the left hand margin of an odd page
\setlength{\oddpagemarginlength}{1in+\hoffset+\oddsidemargin-2\marginlinelength}

% the length of the left hand margin of an even page
\setlength{\evenpagemarginlength}{1in+\hoffset+\evensidemargin-2\marginlinelength}

%%%%%%%%%%%%%%%%%%%%%%%%%%%%%%%%%%%%%%%%%%%%%%%%%%%%%%%%%%%%%%%%%%%%%%%%%%%%%%
%% Things related to the vertical line to the left of examples
%%%%%%%%%%%%%%%%%%%%%%%%%%%%%%%%%%%%%%%%%%%%%%%%%%%%%%%%%%%%%%%%%%%%%%%%%%%%%%

% the example environment has a regular and starred version.
% the regular version takes 4 arguments: label, title, problem, solution.
% The starred version only has 3: label, title, problem/solution.
\makeatletter
\newcommand{\example}{\@ifstar \examplestarred \examplenostar}
\newcommand{\reverseexampledefault}{\renewcommand{\example}{\@ifstar \examplenostar \examplestarred}}
\newcommand{\restoreexampledefault}{\renewcommand{\example}{\@ifstar \examplestarred \examplenostar}}
\makeatother

% This is the no-star (regular) version of the example command.
\newcommand{\examplenostar}[4]{\examplestarred{#1}{#2}{#3\bigskip

\makebox[65pt][l]{\textsc{\small\textbf{Solution\lxAddClass{solutionTag}}}}% 
#4}}


% black: hsl(x,x,0)
% white: hsl(x,x,100)
% blue: hsl(240,100,50)
% line color: blue!95!black!30 = Hsb(240,.29,.98) = hsl(240,87.7,83.8)




%%%%%%%%%%%%%%%%%%%%%%%%%%%%%%%%%%%%%%%%%%%%%%%%%%%%%%%%%%%%%%%%%%%%%%
%% Define style for Definitions, Theorems and Key Ideas
%%%%%%%%%%%%%%%%%%%%%%%%%%%%%%%%%%%%%%%%%%%%%%%%%%%%%%%%%%%%%%%%%%%%%%


\newlength{\specialboxlength}
\newlength{\specialboxinnerseplength}
\newlength{\specialboxinnerseplengthx}
\newlength{\specialboxinnerseplengthy}

\setlength{\specialboxinnerseplength}{15pt}
\setlength{\specialboxinnerseplengthx}{15pt}
\setlength{\specialboxinnerseplengthy}{10pt}

\newcommand{\setboxwidth}[1]{%
\setlength{\specialboxlength}{\textwidth+#1-2.3\specialboxinnerseplength}}

\newcommand{\restoreboxwidth}{\setboxwidth{0pt}}
\restoreboxwidth

\newcommand{\newspecialbox}[4][]{%
 \makeStyles{#2}{#4}%
 \ifthenelse{\equal{#1}{}}{%
  \definecolor{top#2}{Hsb}{#4,.05,1}% = hsl(#4,100,97.5)
  \ifthenelse{#4=60}{%
   \definecolor{border#2}{Hsb}{#4,.59,.97}% = hsl(#4,90.5,68.4)
   \definecolor{bottom#2}{Hsb}{#4,.28,.97}% = hsl(#4,81.9,83.4)
  }{%
   \definecolor{border#2}{Hsb}{#4,.23,.65}% = hsl(#4,17.6,57.5)
   \definecolor{bottom#2}{Hsb}{#4,.13,.92}% = hsl(#4,42.8,86)
  }%
  \newtheorem{#2Env}{#3}%
  \expandafter\newcommand\csname #2Envautorefname\endcsname{#3}
  \expandafter\newcommand\csname #2\endcsname[4][]{%
   \vspace{\baselineskip}%
   \noindent%
   \flushinner{%
    \noindent%
    \coloredbox{%
     rectangle, text width = \specialboxlength,
     inner xsep=\specialboxinnerseplengthx, inner ysep=\specialboxinnerseplengthy,
     draw=border#2, top color=top#2, bottom color=bottom#2,
     text justified, very thick
    }{%
     draw=black, thick, rectangle, text width=\specialboxlength,
     inner xsep=\specialboxinnerseplengthx, inner ysep=\specialboxinnerseplengthy,
     draw, text justified, very thick
    }{%
     \noindent\begin{#2Env}[{##3}##1]\label{##2}\noindent ##4\end{#2Env}%
    }%
   }%
   \vspace{\baselineskip}%
   \restoreboxwidth%
  }%
 }{% else #1 != '' means we're in exvideo (aka, a qr code to youtube)
  \definecolor{top#2}{Hsb}{#4,.05,1}% %= hsl(#4,100,97.5)
  \definecolor{border#2}{Hsb}{#4,.3,1}% %= hsl(#4,90.5,68.4)
  \definecolor{bottom#2}{Hsb}{#4,.15,1}% %= hsl(#4,81.9,83.4)
  \expandafter\newcommand\csname #2\endcsname[1]{%
   \vspace{\baselineskip}%
   \noindent%
   \flushinner{%
    \noindent%
    \coloredbox{% same options as above
     rectangle, text width = \specialboxlength,
     inner xsep=\specialboxinnerseplengthx, inner ysep=\specialboxinnerseplengthy,
     draw=border#2, top color=top#2, bottom color=bottom#2,
     text justified, very thick
    }{%
     draw=black, thick, rectangle, text width=\specialboxlength,
     inner xsep=\specialboxinnerseplengthx, inner ysep=\specialboxinnerseplengthy,
     draw, text justified, very thick
    }{\noindent ##1}%
   }%
   \vspace{\baselineskip}%
  }%
 }% ends ifthenelse #1=''
}

\newspecialbox{definition}{Definition}{60}
% draw = yellow!95!black!60 = Hsb( 60,.59,.97)
% topc = white!95!yellow    = Hsb( 60,.05,1)
% botc = yellow!90!black!30 = Hsb( 60,.28,.97)

\newspecialbox{theorem}{Theorem}{120}
% draw = green!30!black!50  = Hsb(120,.23,.65)
% topc = white!95!green     = Hsb(120,.05,1)
% botc = green!60!black!20  = Hsb(120,.13,.92)

\newspecialbox{keyidea}{Key Idea}{0}
% draw = red!30!black!50    = Hsb(  0,.23,.65)
% topc = white!95!red       = Hsb(  0,.05,1)
% botc = red!60!black!20    = Hsb(  0,.13,.92)

%\newcommand{\showcolor}[1]{#1 is
%  \extractcolorspecs{#1}{\model}{\mycolor}
%  \convertcolorspec{\model}{\mycolor}{Hsb}{\converted}\converted\ in Hsb}
%
%\begin{document}
%\showcolor{blue!95!black!30}
%\end{document}



%%%%%%%%%%%%%%%%%%%%%%%%%%%%%%%%%%%%%%%%%%%%%%%%%%%%%%%%%%%%%%%%%
%% Begins the exercise section, containing all commands
%% related to creating problem sections.
%%%%%%%%%%%%%%%%%%%%%%%%%%%%%%%%%%%%%%%%%%%%%%%%%%%%%%%%%%%%%%%%%

%\newcommand{\exc}{\addtocounter{excounter}{2}\arabic{excounter}}

\newif\ifmore

\newif\ifexsetmore

% this counter gives an effective, albeit not elegant
% way of using the same command to both print questions
% or the answers, depending on what section you are in.
% showexercises = 1: print questions
% showexercises = 2: print odd answers only
% showexercises = 3: print all answers 
% 
% The subsequent lines sets the value 

\setlength{\columnsep}{20pt}

\newcommand{\printconcepts}{\noindent\textit{\Large Terms and Concepts}\vskip \baselineskip}
\newcommand{\printproblems}{\vskip \baselineskip\noindent\textit{\Large Problems}\vskip \baselineskip}
\newcommand{\printreview}{\vskip \baselineskip\noindent\textit{\Large Review}\vskip \baselineskip}

\newcommand{\inanswersection}{%
	\renewcommand{\printconcepts}{}%
	\renewcommand{\printproblems}{}%
	\renewcommand{\printreview}{}%
}%

%\newcount\showexercises % not used?
\newboolean{printquestions}
\newboolean{printoddanswersonly}

\setboolean{printoddanswersonly}{true}
\newcommand{\printallanswers}{\setboolean{printoddanswersonly}{false}}

\newcount\numberofexercises

\newcounter{numofexer}
\newcounter{negnumofexer}

% used for debugging; not really used anymore
\newcounter{debug}
\setcounter{debug}{0}

\newcounter{exercisecounter}
%\newcounter{IMTcount}
%\newcounter{IMTcount_temp}

% the exercise names can be printed next to the problem
% using these commands. The default is to not print them.
\newboolean{showexercisenames}
\newcommand{\printexercisenames}{\setboolean{showexercisenames}{true}}
\newcommand{\noprintexercisenames}{\setboolean{showexercisenames}{false}}
\noprintexercisenames


% TeX uses a certain system to name input files that it reads from.
% To prevent conflicts, we use the newread command.
\newread\exread %read an example
\newread\exsetread %read an example set
\newread\exansread %read the answer
\newread\printansread% read in the answers file

\newwrite\answrite %write the answers file
% give the answers file the name ``jobname.answers''
\openout\answrite=\jobname.answers

\def\exinput #1 {\ifthenelse{\boolean{printquestions}}{% 
	\openin\exread=#1 
	\read\exread to \tempp%
%	\begin{adjustwidth*}{}{-100pt}% 
	\iflatexml\begin{enumerate}\else\begin{enumerate}[leftmargin=*,topsep=10pt]\fi% 
		\addtocounter{enumi}{\theexercisecounter}%
		\item% 
		\ifthenelse{\boolean{showexercisenames}}{%
		 \noindent\makebox[0pt][l]{\noindent\tiny\hspace{-60pt}\printexercisename #1 }%
	    }{}%
		\tempp 
		\addtocounter{exercisecounter}{1}
	\end{enumerate}
%	\end{adjustwidth*}%
	\closein\exread}% end print questions 
	{% else: print answers
		\ifthenelse{\boolean{printoddanswersonly}}{%
			\openin\exread=#1 
			\read\exread to \tempp % read in the question - we ignore it.
			\addtocounter{exercisecounter}{1}
			\read\exread to \tempp % reads in the answer
			\ifodd \theexercisecounter
				%\else
				\iflatexml\begin{enumerate}\else\begin{enumerate}[leftmargin=*]\fi 
					\addtocounter{enumi}{\theexercisecounter}
					\addtocounter{enumi}{-1}
					\item 
					\ifthenelse{\boolean{showexercisenames}}{%
					 \noindent\makebox[0pt][l]{\noindent\tiny\hspace{-60pt}\printexercisename #1 }%
					}{}
					\tempp 
					%\addtocounter{exercisecounter}{1}
				\end{enumerate} 
			\fi
			\closein\exread 
		}  %ends print odd answers only
		{% print all answers
			\openin\exread=#1 
			\read\exread to \tempp %reads in the question, which is ignored 
			\read\exread to \tempp %reads in the answer
			\iflatexml\begin{enumerate}\else\begin{enumerate}[leftmargin=*]\fi 
				\addtocounter{enumi}{\theexercisecounter}
				\item% 
				\ifthenelse{\boolean{showexercisenames}}{%
				 \noindent\makebox[0pt][l]{\noindent\tiny\hspace{-60pt}\printexercisename #1 }%
				}{}
				\tempp 
				\addtocounter{exercisecounter}{1}
			\end{enumerate}
			\closein\exread
		} % ends print all answers
	}
}

\def\exsetinput #1 {\openin\exsetread=#1%
	\setcounter{numofexer}{0}%
	\setcounter{negnumofexer}{0}% 
	\read\exsetread to \exsettemp%
	\read\exsetread to \exsettemp%
	{\loop%
			\read\exsetread to \exsettemp%
			\ifeof \exsetread \exsetmorefalse \else \exsetmoretrue \fi%
			\ifexsetmore%
					\addtocounter{numofexer}{1}%
					\addtocounter{negnumofexer}{-1}%
		\repeat}%							
	\closein\exsetread%
	\openin\exsetread=#1%
	\ifthenelse{\boolean{printquestions}}{% 
		\read\exsetread to \exsettemp%
		\setcounter{enumi}{\theexercisecounter} %
		\addtocounter{enumi}{1}%
		\ifthenelse{\boolean{showexercisenames}}{%
		 \noindent\makebox[0pt][l]{\noindent\tiny\hspace{-60pt}\printexercisename #1 }%
		}{}% end show exercise names
%		\begin{adjustwidth*}{}{-100pt}%
		\noindent\textbf{\exsettemp\theenumi\addtocounter{enumi}{-1}%
		\addtocounter{enumi}{\thenumofexer}{ -- }\theenumi%
		\addtocounter{enumi}{\thenegnumofexer}%
		\read\exsetread to \exsettemp \exsettemp}%
%		\end{adjustwidth*}%

		{\loop%
				\read\exsetread to \exsettemp%
				\ifeof \exsetread \exsetmorefalse \else \exsetmoretrue \fi%
				\ifexsetmore%
						\exsettemp%
		\repeat}%
	}% ends print exercises; on to print answers
	{	\read\exsetread to \exsettemp%
		\read\exsetread to \exsettemp%
		{\loop%
				\read\exsetread to \exsettemp%
				\ifeof \exsetread \exsetmorefalse \else \exsetmoretrue \fi%
				\ifexsetmore%
						\exsettemp%
		\repeat}%
	}%ends else printing asnwers
	\closein\exsetread%
}%

% \writeToAnsFile in the style file b/c latexml was copying in ``noexpand''

\def\printexercisesgeneric#1#2#3 {%
\writeToAnsFile{#1}{#2}{#3}%
%\strictpagecheck%
\setcounter{exercisecounter}{0}\setboolean{printquestions}{true}%
%\thispagestyle{empty}%
%\begin{adjustwidth*}{}{-100pt}%
\exercisegeometry%
\exerciseheader%
\small%
\noindent\underline{\parbox{\textwidth}{\textbf{\huge#2} \hyperref[sol#3]{(solutions)}\refstepcounter{subsection}\label{exer#3}}}% 
\sffamily%
\vskip\baselineskip%
\begin{multicols}{2}%
	\openin\exansread=#1 
	\ifeof \exansread 
		{No problems written.} 
	\else 
		\loop \read\exansread to \extemp  
			\ifeof \exansread \morefalse \else \moretrue \fi 
			\ifmore \extemp
				\repeat 
	\fi 
	\closein\exansread 
	\end{multicols}
%	\end{adjustwidth*}%
	\restoregeometry
	\regularheader
%	\easypagecheck
	\setlength{\hoffset}{0pt} \rmfamily\normalsize \bigskip%
	} % ends printexercisesgeneric



% The following prints the answers. To print all answers, use command \printallanswers.
% 
\def\printanswers #1 {\setcounter{exercisecounter}{0} \footnotesize \setboolean{printquestions}{false} \openin\printansread=#1 
	\ifeof \printansread 
		{No problems written.} 
	\else 
		\loop \read\printansread to \extemp  
			\ifeof \printansread \morefalse \else \moretrue \fi 
			\ifmore \extemp \fi 
			\ifeof \printansread \morefalse \else \moretrue \fi 
			\ifmore 
				\repeat 
	\fi 
	\closein\printansread
	\small}


\def \printexercisename exercises/#1_#2_#3_#4 {#1 #2 #3 #4}

%%%%%%%%%%%%% Used to automate the answer production at
%%%%%%%%%%%%% end the text.

\ifthenelse{\boolean{longpage}}%
{% if longpage, readsection does nothing, really
\def\readsection #1{#1}%
}%
{% not longpage: usual readsection
\def \readsection #1.#2{#2}
}%

\def \writeexercisestofile #1{%
\write\answrite{%
\noexpand\printanswers{%
exercises/0\thechapter_0\expandafter\readsection #1_exercises.tex}%
\noexpand\bigskip}%
\write\answrite{}%
}

\newcommand{\makeexercisesection}[2][\jobname]{%
\iflatexml\showans\answrite\fi
% showans will write the contents of Calculus.answers to the log
% in latexml, this needs to happen before closing
\immediate\closeout\answrite%
\inanswersection\exercisegeometry\exerciseheader%
\chapter{#2}\thispagestyle{empty}%
\begin{multicols}{2}%
\small\raggedright%
\input{#1.answers}%
\end{multicols}%
\restoregeometry\regularheader%
\setlength{\hoffset}{0pt}\rmfamily}%
