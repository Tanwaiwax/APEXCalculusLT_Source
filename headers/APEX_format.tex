\usepackage[aux]{rerunfilecheck}

\usepackage{etoolbox}

\PassOptionsToPackage{HTML}{xcolor}
%\usepackage[HTML]{xcolor} % must occur before qrcode in apex_style
\usepackage{tikz}
\usetikzlibrary{calc}

% with 10pt font, 1em ~ 10pt ; 1ex ~ 4.3pt

%%Page Size stuff

\usepackage[paperheight=11in,paperwidth=8.5in,%
	inner=1in,textheight=7in,textwidth=320pt,marginparwidth=150pt,%
	marginparsep=32pt,bottom=3in,footskip=1.5in]{geometry}

\newcommand{\exercisegeometry}{%
%	\batchmode%
	\newgeometry{inner=72pt,outer=72pt,textheight=9.25in,tmargin=.75in,
		marginparwidth=150pt,marginparsep=32pt,footskip=29pt}%
%	\errorstopmode%
}
\newcommand{\eendgeometry}{%
%	\batchmode%
	\newgeometry{inner=72pt,outer=36pt,textheight=10in,
		marginparwidth=150pt,marginparsep=32pt}%
%	\errorstopmode%
}
\newcommand{\prefacegeometry}{%
%	\batchmode%
	\newgeometry{inner=1in,textheight=9in,textwidth=320pt,marginparwidth=150pt,%
		marginparsep=32pt,bottom=1in,footskip=1.5in}%
%	\errorstopmode%
}

\newlength{\widest}

%%% This was originally a style with \usepackage, but inputing is generally
%%% equivalent.  The only real difference is how latexml handles style files.
%%% So we'll input this document as a header instead,
%%% and save \usepackage{customstyle}
%%% for things latexml is having trouble with.
%%% This does mean that the distinction between APEX_format and Header_Calculus
%%% is no longer important, and mostly historical.

% do we want to print the keys for the labels? if so, uncomment
%\usepackage[notref,notcite]{showkeys}

\usepackage{amsthm}
\usepackage{amsmath}
%\usepackage{amssymb} % todo ? https://tex.stackexchange.com/a/3000/107497 recommends dropping amssymb in favor of unicode-math
% but then the font loading gets messed up with mathspec
% see also https://tex.stackexchange.com/q/218112/107497 that if the font doesn't have math, then it's a losing battle
%\RequirePackage{unicode-math}

\usepackage{graphicx}
\usepackage{multicol}
\usepackage{makeidx}

\usepackage[normalem]{ulem}

\usepackage{calc}
\usepackage{ragged2e}

%\usepackage[inline]{enumitem}
\usepackage{enumext}

\usepackage[nocomments]{latexml}
\lxDocumentID{apex}

\numberwithin{figure}{section}
\numberwithin{equation}{section}

%%%%%%%%%%%%%%%%%%%%%

\makeindex

\newcommand{\apex}{\texorpdfstring{A\kern -.1em \lower -.5ex\hbox{P}\kern -.25em\lower .5ex\hbox{E}\kern -.1em X}{APEX}}


% Create boolean for whether or not to print 3D graphics. 
% Also creates command to switch back and forth; "looks better."
\newtoggle{in_threeD}
\newcommand{\usethreeDgraphics}{\toggletrue{in_threeD}}
\newcommand{\usetwoDgraphics}{\togglefalse{in_threeD}}
\usethreeDgraphics


\usepackage{pgfplots}
\pgfplotsset{compat=1.8}

\newtoggle{inColor}
\toggletrue{inColor}

\pgfplotsset{colormap={coloronemap}{rgb=(.4,.4,1); rgb=(.8,.8,1)}}
\pgfplotsset{colormap={colortwomap}{rgb=(1,.4,.4); rgb=(1,.8,.8)}}
%\usepgfplotslibrary{external}
% only needed for external tikz pictures (and not liked by latexml)
% see http://tex.stackexchange.com/a/1475/107497
\usetikzlibrary{calc}
\usetikzlibrary{shadings}

% these will be renewcommanded
\newcommand{\colorone}{blue}
\newcommand{\colortwo}{red}
\newcommand{\colorthree}{green}
\newcommand{\coloronefill}{blue!15!white}
\newcommand{\colortwofill}{red!15!white}
\newcommand{\colormapone}{rgb=(.4,.4,1); rgb=(.8,.8,1)}
\newcommand{\colormaptwo}{rgb=(1,.4,.4); rgb=(1,.8,.8)}
\newcommand{\colormapplaneone}{rgb=(.7,.7,1); rgb=(.9,.9,1)}
%\definecolor{colormaponebottom}{rgb}{.4,.4,1}
%\definecolor{colormaponetop}{rgb}{.8,.8,1}
%\definecolor{colormaptwobottom}{rgb}{1,.4,.4}
%\definecolor{colormaptwotop}{rgb}{1,.8,.8}

% determines the line colors for color and black and white lines.
\newcommand{\colorlinecolor}{blue!95!black!30}
\newcommand{\bwlinecolor}{black!30}

% sets the line color to be in color, as a default
\newcommand{\thelinecolor}{\colorlinecolor}

% this allows the above default to be overriden by using
% the \printincolor and \printinblackandwhite commands
% anywhere in the file. This allows you to switch back
% and forth between bw and color. (Who would want to?)
\newcommand{\colornamesuffix}{}

\newcommand{\printincolor}{
 \toggletrue{inColor}%
 % aforementioned renewcommanding
 \renewcommand{\thelinecolor}{\colorlinecolor}
 \renewcommand{\colornamesuffix}{}
 \renewcommand{\colorone}{blue}
 \renewcommand{\colortwo}{red}
 \renewcommand{\colorthree}{green}
 \renewcommand{\coloronefill}{blue!15!white}
 \renewcommand{\colortwofill}{red!15!white}
 \renewcommand{\colormapone}{rgb=(.4,.4,1); rgb=(.8,.8,1)}
 \renewcommand{\colormaptwo}{rgb=(1,.4,.4); rgb=(1,.8,.8)}
 \renewcommand{\colormapplaneone}{rgb=(.7,.7,1); rgb=(.9,.9,1)}
 \definecolor{colormaponebottom}{rgb}{.4,.4,1}
 \definecolor{colormaponetop}{rgb}{.8,.8,1}
 \definecolor{colormaptwobottom}{rgb}{1,.4,.4}
 \definecolor{colormaptwotop}{rgb}{1,.8,.8}
 \setexvideocolor
 \colorizespecialboxes
}

\newcommand{\printinblackandwhite}{
 \togglefalse{inColor}%
 % undoing the above renewcommanding
 \renewcommand{\thelinecolor}{\bwlinecolor}
 \renewcommand{\colornamesuffix}{BW}
 \renewcommand{\colorone}{black}
 \renewcommand{\colortwo}{black!50!white}
 \renewcommand{\colorthree}{black!25!white}
 \renewcommand{\coloronefill}{black!15!white}
 \renewcommand{\colortwofill}{black!05!white}
 \renewcommand{\colormapone}{rgb=(.4,.4,.4); rgb=(.7,.7,.7)}
 \renewcommand{\colormaptwo}{rgb=(.6,.6,.6); rgb=(.9,.9,.9)}
 \renewcommand{\colormapplaneone}{rgb=(.8,.8,.8); rgb=(.95,.95,.95)}
 \definecolor{colormaponebottom}{rgb}{.4,.4,.4}
 \definecolor{colormaponetop}{rgb}{.7,.7,.7}
 \definecolor{colormaptwobottom}{rgb}{.6,.6,.6}
 \definecolor{colormaptwotop}{rgb}{.9,.9,.9}
 \setexvideobw
 \bwizespecialboxes
}


\newcommand{\myincludegraphics}[2][]{%
 \IfFileExists{./#2\colornamesuffix.png}{%
  \includegraphics[#1]{#2\colornamesuffix}%
 }{%
  \IfFileExists{./#2\colornamesuffix.pdf}{%
   \includegraphics[#1]{#2\colornamesuffix}%
  }{%
   \IfFileExists{./#2.png}{%
    \includegraphics[#1]{#2}%
   }{%
    \IfFileExists{./#2.pdf}{%
     \includegraphics[#1]{#2}%
    }{%
     \includegraphics[#1]{#2\colornamesuffix}%
    }%
   }%
  }%
 }%
}



%%%%%%%%%%%%%%%%%%%%%%%%%%%%%%%%%%%%%%%%%%%%%%%%%%%%%%%%%%%%%%%%%%%%%%%%%%%%%%
%% Examples
%%%%%%%%%%%%%%%%%%%%%%%%%%%%%%%%%%%%%%%%%%%%%%%%%%%%%%%%%%%%%%%%%%%%%%%%%%%%%%

\newlength{\boxskipamount}
\setlength{\boxskipamount}{4ex plus 4ex minus 2ex}

%\newlength{\topmarginlength} 
%\newlength{\bottommarginlength}
%\newlength{\oddpagemarginlength}
%\newlength{\evenpagemarginlength}
\newlength{\marginlinelength}
%\newlength{\innerpagemarginlength}

% how far from the text the example line is to be drawn
\setlength{\marginlinelength}{.2em}

% the height of the top margin (used in calculating the lines for examples)
%\setlength{\topmarginlength}{-1in-\voffset}

% the length of the bottom margin (ish)
% actually starts at the top of the page, moves
% through the top margin length then the text height.
%\setlength{\bottommarginlength}{-1in-\textheight-2\baselineskip-\voffset-\headheight-\headsep-\topmargin}

% the length of the left hand margin of an odd page
%\setlength{\oddpagemarginlength}{1in+\hoffset+\oddsidemargin-2\marginlinelength}

% the length of the left hand margin of an even page
%\setlength{\evenpagemarginlength}{1in+\hoffset+\evensidemargin-2\marginlinelength}

\newcommand{\solution}{\bigbreak\par
 \makebox[6.5em][l]{\textsc{\small\textbf{Solution\lxAddClass{solutionTag}}}}%
 \nopagebreak%
}

% black: hsl(x,x,0)
% white: hsl(x,x,100)
% blue: hsl(240,100,50)
% line color: blue!95!black!30 = Hsb(240,.29,.98) = hsl(240,87.7,83.8)

\newlength{\saveparindent}
\setlength{\saveparindent}{\parindent}




%%%%%%%%%%%%%%%%%%%%%%%%%%%%%%%%%%%%%%%%%%%%%%%%%%%%%%%%%%%%%%%%%%%%%%
%% Definitions, Theorems and Key Ideas
%%%%%%%%%%%%%%%%%%%%%%%%%%%%%%%%%%%%%%%%%%%%%%%%%%%%%%%%%%%%%%%%%%%%%%

\newcommand{\newspecialbox}[3]{%
 \AtBeginDocument{\makeStyles{#1}{#3}}%
 \expandafter\newcommand\csname colorize#1\endcsname{
  \definecolor{top#1}{Hsb}{#3,.05,1}% = hsl(#4,100,97.5)
  \ifnumequal{#3}{60}{%
   \definecolor{border#1}{Hsb}{#3,.59,.97}% = hsl(#4,90.5,68.4)
   \definecolor{bottom#1}{Hsb}{#3,.28,.97}% = hsl(#4,81.9,83.4)
  }{%
   \definecolor{border#1}{Hsb}{#3,.23,.65}% = hsl(#4,17.6,57.5)
   \definecolor{bottom#1}{Hsb}{#3,.13,.92}% = hsl(#4,42.8,86)
  }%
 }
 \expandafter\newcommand\csname bwize#1\endcsname{
  \colorlet{top#1}{white}
  \colorlet{bottom#1}{white}
  \colorlet{border#1}{black}
 }
 \newtheorem{#1}{#2}[section]%
 \expandafter\providecommand\csname #1autorefname\endcsname{#2}
 \ifbool{latexml}{%
 }{%
  \tcolorboxenvironment{#1}{
    sharp corners=all,
    enhanced,
    colframe=border#1,
    beforeafter skip=\boxskipamount,
    interior style={top color=top#1, bottom color=bottom#1},
    breakable=true,
    overlay first={\continue{bottom}{#1}},
    overlay middle={\continue{bottom}{#1}\continue{top}{#1}},
    overlay last={\continue{top}{#1}},
    enlargepage flexible=3\baselineskip,
    toggle enlargement=evenpage,
    lines before break=8,
  }%
 }
}

\newcommand{\continue}[1]{%
 \csname continue#1\endcsname{#1}%
}
\newcommand{\continuetext}[2]{%
 \ifstrequal{#1}{top}{%
  \csname #2autorefname\endcsname\ \csname the#2\endcsname\ continued%
 }{%
  (continued)
 }%
}
% can't get this to work
%\newcommand{\northsouth}[2][]{\if thenelse{\equal{#2}{top}}{#1north}{#1south}}

% adapted from https://tex.stackexchange.com/a/545324/107497 by Schrödinger's cat
\newcommand{\continuebottom}[2]{
   \path[font=\small\itshape] (frame.south) node (cont) {\continuetext{#1}{#2}};
   \begin{scope}[decoration={zigzag,amplitude=0.5mm}]
    \path[fill=#1#2]
     decorate {([xshift= 1.2pt]frame.south west) -- (cont.west)} --++ (0,0.5ex)
      -| cycle
     decorate {([xshift=-1.2pt]frame.south east) -- (cont.east)} --++ (0,0.5ex)
      -| cycle;
    \path[fill=white]
     decorate {([xshift= 1.2pt]frame.south west) -- (cont.west)} --++ (0,-0.5ex)
      -| cycle
     decorate {([xshift=-1.2pt]frame.south east) -- (cont.east)} --++ (0,-0.5ex)
      -| cycle;
   \end{scope} 
}
\newcommand{\continuetop}[2]{
   \path[font=\small\itshape] (frame.north) node (thm) {\continuetext{#1}{#2}};
   \begin{scope}[decoration={zigzag,amplitude=0.5mm}]
    \path[fill=#1#2]
     decorate {([xshift= 1.2pt]frame.north west) -- (thm.west)} --++ (0,-0.5ex)
      -| cycle
     decorate {([xshift=-1.2pt]frame.north east) -- (thm.east)} --++ (0,-0.5ex)
      -| cycle;
    \path[fill=white]
     decorate {([xshift= 1.2pt]frame.north west) -- (thm.west)} --++ (0,0.5ex)
      -| cycle
     decorate {([xshift=-1.2pt]frame.north east) -- (thm.east)} --++ (0,0.5ex)
      -| cycle;
   \end{scope} 
}

\newcommand{\colorizespecialboxes}{
 \colorizedefinition
 \colorizetheorem
 \colorizekeyidea
}
\newcommand{\bwizespecialboxes}{
 \bwizedefinition
 \bwizetheorem
 \bwizekeyidea
}




%%%%%%%%%%%%%%%%%%%%%%%%%%%%%%%%%%%%%%%%%%%%%%%%%%%%%%%%%%%%%%%%%
%% Exercises
%% We would like to make better use of enumitem to put implementation
%% details here instead of repeating them, but the pdftagging
%% doesn't deal well with that.  So we'll need to repeat everything every time.
%%%%%%%%%%%%%%%%%%%%%%%%%%%%%%%%%%%%%%%%%%%%%%%%%%%%%%%%%%%%%%%%%

\setlength{\columnsep}{20pt}

\newtoggle{inexercises}

%\makeatletter
%\newcommand{\exercisesubsubsection}{%
% \closeenumerate%
% \@startsection{subsubsection}{3}{-1em}{\bigskipamount}{\bigskipamount}{\Large\textit}*}
%\makeatother

% I'd like to move the \closeenumerate into the \exercisesubsubsection, but I can't figure it out
\newcommand{\printconcepts}{\noindent\closeenumerate\exercisesubsubsection*{\noindent Terms and Concepts}}
\newcommand{\printproblems}{\noindent\closeenumerate\exercisesubsubsection*{\noindent Problems}}
\newcommand{\printreview}{\noindent\closeenumerate\exercisesubsubsection*{\noindent Review}}

%\newlist{sectionexercises}{enumerate}{1}
%\newcounter{saveexercisenum}[section]
%\counterwithin*{sectionexercisesi}{section} % in case we have a exercise set before any exercises that would reset the save exercise enum
%\setlist[sectionexercises]{
%	label=\arabic*.,
%	leftmargin=1.5em,
%    before=\setcounter{sectionexercisesi}{\value{saveexercisenum}},
%    after=\setcounter{saveexercisenum}{\value{sectionexercisesi}},
%}
%\ifbool{latexml}{}{
% \setlist*[sectionexercises]{ref=\arabic*}
%}

%\setenumext[enumext,2]{start=1}
\setenumext[enumext,1]{resume}
\resetenumext[1]{subsection} % reset the resumed counter for exercises

\makeatletter
\newcommand{\printexercises}[1]{%
 \writeToAnsFile{#1}% writeToAnsFile in sty (actually, down below)
 \exercisegeometry% includes a clearpage
 \pagestyle{exercise}%
 \bookmarksetupnext{level=\toclevel@subsection}% otherwise, the level is "section" and everything is messed up
 \exercisesubsection{Exercises \thesection}%
 \stepcounter{subsection}% subsections aren't numbered, but this triggers resetenumext
 \label{exer\thesection}%
 \small%
 \bigskip%
 \begin{multicols}{2}%
  \toggletrue{inexercises}%
  \renewcommand{\Itemautorefname}{Ex\-er\-cise}% local b/c multicols = good
  \input{#1}%
  \closeenumerate%
 \end{multicols}%
 \restoregeometry%
 \pagestyle{prose}%
 %	\easypagecheck
 \setlength{\hoffset}{0pt} \rmfamily\normalsize \bigbreak%
}
\makeatother


\newwrite\answrite %write the answers file
% give the answers file the name ``jobname.answers''
\openout\answrite=\jobname.answers

\newcommand{\writeToAnsFile}[1]{%
 \immediate\write\answrite{%
  \string\answersForSection{\arabic{chapter}}{\arabic{section}}{#1}%
%  \noexpand\answersForSection{\arabic{chapter}}{\arabic{section}}{#1}%
 }%
}
% \noexpand\answersForSection becomes \relax in LaTeXML
% \string works with both

\newcounter{exercisesetcounter}[section]
\renewcommand{\theexercisesetcounter}{\thesection.\arabic{exercisesetcounter}}

\newcounter{saveenumi}

%\counterwithin*{enumXi}{subsection}
% does not work because \exercisesubsection is a fake \subsection

% #1 is "Exercises \thesection"
\newcommand{\exercisesubsectiontitle}[1]{%
 \huge\textbf{\texorpdfstring{\hyperref[sol#1]{#1}}{Exercises}}\hrule
% \setcounter{enumXi}{0}%
}

%\makeatletter
%% the usual \subsection definition has stretchable space in arguments 3-5
%\newcommand{\exercisesubsection}[1]{%
%\setcounter{enumi}{0}%
%\@startsection{subsection}{2}{-.7em}{0pt}{.5ex}{\huge\textbf}{\texorpdfstring{\hyperref[sol#1]{Exercises #1}}{Exercises}}%
%\hrule\vspace{-1.5ex}%
%}
%\makeatother

\newcommand{\exautoref}[1]{%
 \hyperref[#1]{Ex\-er\-cise~\ref*{#1}}%
% {%
%  \renewcommand{\Itemautorefname}{Exercise}% localize the upcoming reference
%  \autoref{#1}%
% }%
% doesn't work?
}

%\newcommand*{\exerenv}{sectionexercises}
\newcommand*{\exerenv}{enumext}

\makeatletter
\newcommand{\openenumerate}{%
 \ifx\@currenvir\exerenv\else%
  \begin{enumext}[widest=22,ref=\arabic*]
%  \begin{sectionexercises}
%  \ifbool{latexml}{%
%   \setcounter{sectionexercisesi}{\value{saveexercisenum}}%
%  }{}%
 \fi%
}
\newcommand{\closeenumerate}{
 \ifx\@currenvir\exerenv%
%  \ifbool{latexml}{%
%   \setcounter{saveexercisenum}{\value{sectionexercisesi}}%
%  }{}%
  \end{enumext}
%  \end{sectionexercises}
 \fi%
}
\makeatother
\newcommand{\closeenumerateinquestions}{\closeenumerate}

 % if the instructions have an enumerate, we want to use the second level
 % we can't have another enumext, because that closed before the instructions
 % so this would happen at level 1.  we could monkey around to make enumext
 % thinks it's at level 2, but this seems easier
\makeatletter
\newcommand{\stepenumeratedepth}{\advance\@enumdepth\@ne}
\makeatother

\newcommand{\exercisesetinstructions}[2][In Exercises]{%
 \setcounter{saveenumi}{\value{enumXi}}%
 \closeenumerateinquestions
 \pagebreak[2]%
 \stepcounter{saveenumi}%
 \stepcounter{exercisesetcounter}%
 \ifnumodd{\value{saveenumi}}{}{%
  \PackageInfo{apex}{%
   Exercise set \theexercisesetcounter\space begins with \arabic{saveenumi}%
  }%
 }%
 \bgroup
 \stepenumeratedepth
% \setenumext[enumext,1]{label=\alph*,wrap-label={(#1)}}% pretend it is level 2
 \noindent#1 \arabic{saveenumi}--\ref*{enumiatendof\theexercisesetcounter}%
% \renewcommand{\theenumi}{(\alph{enumi})}%
 #2%
 \egroup%
 % can't \addtocounter{enumXi}{-1} because #2 may have an enumerate
% \setcounter{enumXi}{\value{saveenumi}}
 \ignorespaces%
 \nopagebreak%
}
\newcommand{\exercisesetend}{%
 \label{enumiatendof\theexercisesetcounter}%
 \closeenumerate%
 \ifnumodd{\value{enumXi}}{%
  \PackageInfo{apex}{%
   Exercise set \theexercisesetcounter\space ends with \arabic{enumXi}%
  }%
 }{}%
}

%\newenvironment{exerciseset}[2]{%
% \stepcounter{sectionexercisesi}
% \stepcounter{exercisesetcounter}%
% \ifnumodd{\value{sectionexercisesi}}{}{%
%  \PackageInfo{apex}{%
%   Exercise set \theexercisesetcounter\space begins with \arabic{sectionexercisesi}%
%  }%
% }%
% {%
%  \setlist[enumerate,1]{label=(\alph*)}% for exercise set instructions
%  \noindent#1 \arabic{sectionexercisesi}--\ref*{enumiatendof\theexercisesetcounter}#2%
% }%
% \addtocounter{sectionexercisesi}{-1}\ignorespaces%
% \nopagebreak%
%}{%
% \label{enumiatendof\theexercisesetcounter}%
% \closeenumerate%
% \ifnumodd{\value{sectionexercisesi}}{%
%  \PackageInfo{apex}{%
%   Exercise set \theexercisesetcounter\space ends with \arabic{sectionexercisesi}%
%  }%
% }{}%
%}
%\BeforeBeginEnvironment{exerciseset}{\closeenumerateinquestions}

\newcommand{\exercise}[2]{%
 \openenumerate%
% \setlist[enumerate,1]{label=(\alph*)}% for exercise instructions
 \item \parbox[t]{\linewidth}{#1}%
}
\newcommand{\showexerciseanswers}{%
 \renewcommand{\exercise}[2]{%
  \ifboolexpr{ togl{printoddanswersonly} and test{\ifnumodd{\value{enumXi}}} }{%
   \stepcounter{enumXi}%
  }{%
   \item \parbox[t]{\linewidth}{\raggedright ##2}%
  }%
 }%
}

\newcommand{\questioncolumnbreak}{\columnbreak}


%%%%%%%%%%%%%%%%%%%%%%%%%%%%%%%%%%%%%%%%%%%%%%%%%%%%%%%%%%%%%%%%%
%% Answers
%%%%%%%%%%%%%%%%%%%%%%%%%%%%%%%%%%%%%%%%%%%%%%%%%%%%%%%%%%%%%%%%%

\newcommand{\printsolutions}[2][\jobname]{%
 \immediate\closeout\answrite%
 \inanswersection\exercisegeometry%
 \pagestyle{exercise}%
 %\thispagestyle{empty}%
 \ifstrequal{#1}{\jobname}{%
  \chapter*{#2}%
 }{%
  {% localize the next line
   \renewcommand{\thefootnote}{}
   \chapter*{#2\footnote{Revised \today}}%
  }%
 }%
 \phantomsection
 \addcontentsline{toc}{chapter}{#2}%
 \begin{multicols}{2}%
  \small\raggedright%
  \input{#1.answers}%
 \end{multicols}%
 \restoregeometry\pagestyle{prose}%
 \setlength{\hoffset}{0pt}\rmfamily%
 \pagestyle{empty}%
 \eendgeometry%
}%

\newcommand{\inanswersection}{%
	\renewcommand{\printconcepts}{}%
	\renewcommand{\printproblems}{}%
	\renewcommand{\printreview}{}%
%	\renewenvironment{exerciseset}[2]{}{}%
	\renewcommand{\exercisesetinstructions}[2][]{}%
	\renewcommand{\exercisesetend}{}%
	\renewcommand{\openenumerate}{}%
	\renewcommand{\closeenumerateinquestions}{}%
	\renewcommand{\questioncolumnbreak}{}%
	\packageinanswersection%
	\showexerciseanswers%
%	\setlist[enumerate,1]{label=\arabic*.}% for solutions
	% LaTeX already does this, but LaTeXML doesn't
}%

\newtoggle{printoddanswersonly}

\toggletrue{printoddanswersonly}
\newcommand{\printallanswers}{\togglefalse{printoddanswersonly}}

%\newcounter{answerchapter}
%\newcounter{answersection}[answerchapter]
%\renewcommand{\theanswersection}{\theanswerchapter.\arabic{answersection}}
\newcommand{\lastanswerchapter}{-1}

\newcommand*{\answersForSection}[3]{%
 \ifnumequal{#1}{\lastanswerchapter}{}{%
  \renewcommand{\lastanswerchapter}{#1}% apparently global. who knew?
  \ifbool{latexml}{}{%
   \belowpdfbookmark{Chapter #1}{solsol#1}%
  }
  % commandeer chapter and section numbering
  \setcounter{chapter}{#1}
  \section*{Chapter~\thechapter\hfill\null}
 }%
 \setcounter{section}{#2}
 \subsection*{\hyperref[exer\thesection]{Exercises~\thesection\hfill\null}}%
 \label{solExercises \thesection}
 \ifnumequal{#2}{0}{%
  \loadAllAnswers{#3}
 }{%
  \loadAnswers{#3}
 }%
}

% only called by the prerequisite sections
\newcommand*{\loadAllAnswers}[1]{%
%	\setcounter{answersection}{-1}%
	\iftoggle{printoddanswersonly}{%
		\togglefalse{printoddanswersonly}%
		\loadAnswers{#1}%
		\toggletrue{printoddanswersonly}%
	}{%
		\loadAnswers{#1}%
	}%
}
\newcommand*{\loadAnswers}[1]{%
%	\stepcounter{answersection}%
 \begin{enumext}[start=1,widest=22]
  \input{#1}
 \end{enumext}
 \bigbreak%
}



% The following creates a ``List of Theorems'', ``Definitions'', and ``Key Ideas''.
% See http://tex.stackexchange.com/q/74857/107497
%\usepackage{thmtools} % continuing theorems and ``List of Theorems''
%\patchcmd\thmtlo@chaptervspacehack
%  {\addtocontents{loe}{\protect\addvspace{10\p@}}}
%  {\addtocontents{loe}{\protect\thmlopatch@endchapter\protect\thmlopatch@chapter{\thechapter}}}
%  {}{failed thmtlo@chaptervspacehack}
%\AtEndDocument{\addtocontents{loe}{\protect\thmlopatch@endchapter}}
%\long\def\thmlopatch@chapter#1#2\thmlopatch@endchapter{%
%  \setbox\z@=\vbox{#2}%
%  \ifdim\ht\z@>\z@
%    \hbox{\bfseries\chaptername\ #1}\nobreak
%    #2
%    \addvspace{10\p@}
%  \fi
%}
%\def\thmlopatch@endchapter{}
%\patchcmd\thmt@mklistcmd
%  {\protect\numberline{\csname the\thmt@envname\endcsname}%
%      \thmt@thmname}{}{}{failed thmt@mklistcmd}
%%\makeatother
%\renewcommand\thmtformatoptarg[1]{#1}

