\usepackage{ifthen}

\PassOptionsToPackage{HTML}{xcolor}
%\usepackage[HTML]{xcolor} % must occur before qrcode in apex_style
\usepackage{tikz}
\usetikzlibrary{calc}


%%Page Size stuff

\usepackage[paperheight=11in,paperwidth=8.5in,%
	inner=1in,textheight=7in,textwidth=320pt,marginparwidth=150pt,%
	marginparsep=32pt,bottom=3in,footskip=1.5in]{geometry}

\newcommand{\exercisegeometry}{%
	\newgeometry{inner=72pt,outer=72pt,textheight=9.25in,tmargin=.75in,
		marginparwidth=150pt,marginparsep=32pt,footskip=29pt}%
}

\newcommand{\eendgeometry}{%
	\newgeometry{inner=72pt,outer=36pt,textheight=10in,
		marginparwidth=150pt,marginparsep=32pt}%
}
\newcommand{\prefacegeometry}{%
	\newgeometry{inner=1in,textheight=9in,textwidth=320pt,marginparwidth=150pt,%
		marginparsep=32pt,bottom=1in,footskip=1.5in}%
}

%%% This was originally a style with \usepackage, but inputing is generally
%%% equivalent.  The only real difference is how latexml handles style files.
%%% So we'll input this document as a header instead,
%%% and save \usepackage{customstyle}
%%% for things latexml is having trouble with.
%%% This does mean that the distinction between APEX_format and Header_Calculus
%%% is no longer important, and mostly historical.

% do we want to print the keys for the labels? if so, uncomment
%\usepackage[notref,notcite]{showkeys}


\RequirePackage{amsmath}
\RequirePackage{amssymb} % todo ? https://tex.stackexchange.com/a/3000/107497 recommends dropping amssymb in favor of unicode-math
\RequirePackage{amsthm}

\RequirePackage{graphicx}
\RequirePackage{multicol}
\RequirePackage{makeidx}

\RequirePackage[normalem]{ulem}

%linkbordercolor=white
%\RequirePackage{fancyhdr}
\RequirePackage{calc}

\RequirePackage[nocomments]{latexml}
\lxDocumentID{apex}

\numberwithin{figure}{section}
\newcounter{savefigure}

\newtheoremstyle{apexExample}% name
  {0pt}% Space above, empty = `usual value'
  {0pt}% Space below
  {}% Body font
  {}% Indent amount (empty = no indent, \parindent = para indent)
  {\bfseries}% Thm head font
  {}% Punctuation after thm head
  {\newline}% Space after thm head: " " = normal interword space; \newline = linebreak
%  {\makebox[80pt][l]{\textbf{#1 \arabic{examplecounter}}}\thmnote{\textbf{#3}}}% Thm head spec
  {\parbox[t]{.25\textwidth}{\textbf{#1 #2}}\parbox[t]{.75\textwidth}{\thmnote{\textbf{#3}}}}
%  {\makebox[80pt][l]{\textbf{#1 #2}}\thmnote{\textbf{#3}}}% Thm head spec

\newtheoremstyle{apex}% name
  {0pt}% Space above, empty = `usual value'
  {0pt}% Space below
  {}% Body font
  {}% Indent amount (empty = no indent, \parindent = para indent)
  {\bfseries}% Thm head font
  {}% Punctuation after thm head
  {\newline}% Space after thm head: " " = normal interword space; \newline = linebreak
  {\thmname{\textbf{#1}}\thmnumber{ \textbf{#2}}\thmnote{\hspace{20pt}\textbf{#3}}}% Thm head spec

\theoremstyle{apexExample}
\newtheorem{exampleEnv}{Example}[section]

\theoremstyle{apex}

\makeatletter
\renewenvironment{proof}[1][\proofname]{\par
  \pushQED{\qed}%
  \normalfont \topsep6\p@\@plus6\p@\relax
  \trivlist
  \item[\hskip\labelsep
        \bfseries
    #1]\mbox{}\\* % something is needed to be able to get a newline
}{%
  \popQED\endtrivlist\@endpefalse
}
\makeatother
\renewcommand{\qedsymbol}{\ensuremath{\square}}


\makeindex

\newcommand{\apex}{\texorpdfstring{A\kern -.1em \lower -.5ex\hbox{P}\kern -.25em\lower .5ex\hbox{E}\kern -.1em X}{APEX}}
%\newcommand{\apex}{A\kern -1pt \lower -2pt\hbox{P}\kern -4pt \lower .7ex\hbox{E}\kern -1pt X}

%\renewcommand{\partname}{\protect\thispagestyle{empty}}
%\renewcommand{\thepart}{}


% Create boolean for whether or not to print 3D graphics. 
% Also creates command to switch back and forth; "looks better."
\newboolean{in_threeD}
\newcommand{\usethreeDgraphics}{\setboolean{in_threeD}{true}}
\newcommand{\usetwoDgraphics}{\setboolean{in_threeD}{false}}
\usethreeDgraphics


\usepackage{pgfplots}
\pgfplotsset{compat=1.8}

\newboolean{inColor}
\setboolean{inColor}{true}

\pgfplotsset{colormap={coloronemap}{rgb=(.4,.4,1); rgb=(.8,.8,1)}}
\pgfplotsset{colormap={colortwomap}{rgb=(1,.4,.4); rgb=(1,.8,.8)}}
%\usepgfplotslibrary{external}
% only needed for external tikz pictures (and not liked by latexml)
% see http://tex.stackexchange.com/a/1475/107497
\usetikzlibrary{calc}
\usetikzlibrary{shadings}

% these will be renewcommanded
\newcommand{\colorone}{blue}
\newcommand{\colortwo}{red}
\newcommand{\colorthree}{green}
\newcommand{\coloronefill}{blue!15!white}
\newcommand{\colortwofill}{red!15!white}
\newcommand{\colormapone}{rgb=(.4,.4,1); rgb=(.8,.8,1)}
\newcommand{\colormaptwo}{rgb=(1,.4,.4); rgb=(1,.8,.8)}
\newcommand{\colormapplaneone}{rgb=(.7,.7,1); rgb=(.9,.9,1)}
%\definecolor{colormaponebottom}{rgb}{.4,.4,1}
%\definecolor{colormaponetop}{rgb}{.8,.8,1}
%\definecolor{colormaptwobottom}{rgb}{1,.4,.4}
%\definecolor{colormaptwotop}{rgb}{1,.8,.8}

% determines the line colors for color and black and white lines.
\newcommand{\colorlinecolor}{blue!95!black!30}
\newcommand{\bwlinecolor}{black!30}

% sets the line color to be in color, as a default
\newcommand{\thelinecolor}{\colorlinecolor}

%\newcommand{\setcolorlinecolor}[1]{%
% \renewcommand{\colorlinecolor}{#1}%
% \renewcommand{\thelinecolor}{\colorlinecolor}}
%\newcommand{\setbwlinecolor}[1]{%
% \renewcommand{\bwlinecolor}{#1}%
% \renewcommand{\thelinecolor}{\bwlinecolor}}

% this allows the above default to be overriden by using
% the \printincolor and \printinblackandwhite commands
% anywhere in the file. This allows you to switch back
% and forth between bw and color. (Who would want to?)
\newcommand{\colornamesuffix}{}

\newcommand{\printincolor}{
 \setboolean{inColor}{true}%
 % aforementioned renewcommanding
 \renewcommand{\thelinecolor}{\colorlinecolor}
 \renewcommand{\colornamesuffix}{}
 \renewcommand{\colorone}{blue}
 \renewcommand{\colortwo}{red}
 \renewcommand{\colorthree}{green}
 \renewcommand{\coloronefill}{blue!15!white}
 \renewcommand{\colortwofill}{red!15!white}
 \renewcommand{\colormapone}{rgb=(.4,.4,1); rgb=(.8,.8,1)}
 \renewcommand{\colormaptwo}{rgb=(1,.4,.4); rgb=(1,.8,.8)}
 \renewcommand{\colormapplaneone}{rgb=(.7,.7,1); rgb=(.9,.9,1)}
 \definecolor{colormaponebottom}{rgb}{.4,.4,1}
 \definecolor{colormaponetop}{rgb}{.8,.8,1}
 \definecolor{colormaptwobottom}{rgb}{1,.4,.4}
 \definecolor{colormaptwotop}{rgb}{1,.8,.8}
 \setexvideocolor
}

\newcommand{\printinblackandwhite}{
 \setboolean{inColor}{false}%
 % undoing the above renewcommanding
 \renewcommand{\thelinecolor}{\bwlinecolor}
 \renewcommand{\colornamesuffix}{BW}
 \renewcommand{\colorone}{black}
 \renewcommand{\colortwo}{black!50!white}
 \renewcommand{\colorthree}{black!25!white}
 \renewcommand{\coloronefill}{black!15!white}
 \renewcommand{\colortwofill}{black!05!white}
 \renewcommand{\colormapone}{rgb=(.4,.4,.4); rgb=(.7,.7,.7)}
 \renewcommand{\colormaptwo}{rgb=(.6,.6,.6); rgb=(.9,.9,.9)}
 \renewcommand{\colormapplaneone}{rgb=(.8,.8,.8); rgb=(.95,.95,.95)}
 \definecolor{colormaponebottom}{rgb}{.4,.4,.4}
 \definecolor{colormaponetop}{rgb}{.7,.7,.7}
 \definecolor{colormaptwobottom}{rgb}{.6,.6,.6}
 \definecolor{colormaptwotop}{rgb}{.9,.9,.9}
 \setexvideobw
}


\newcommand{\myincludegraphics}[2][]{%
 \IfFileExists{./#2\colornamesuffix.png}{%
  \includegraphics[#1]{#2\colornamesuffix.png}%
 }{%
  \IfFileExists{./#2.png}{%
   \includegraphics[#1]{#2.png}%
  }{%
   \IfFileExists{./#2\colornamesuffix.pdf}{%
    \includegraphics[#1]{#2\colornamesuffix.pdf}%
   }{%
    \IfFileExists{./#2.pdf}{%
     \includegraphics[#1]{#2.pdf}%
    }{%
     \includegraphics[#1]{#2\colornamesuffix}%
    }%
   }%
  }%
 }%
}




\newlength{\topmarginlength} 
\newlength{\bottommarginlength}
\newlength{\oddpagemarginlength}
\newlength{\evenpagemarginlength}
\newlength{\marginlinelength}
\newlength{\innerpagemarginlength}


% how far from the text the example line is to be drawn
\setlength{\marginlinelength}{2pt}

% the height of the top margin (used in calculating the lines for examples)
\setlength{\topmarginlength}{-1in-\voffset}

% the length of the bottom margin (ish)
% actually starts at the top of the page, moves
% through the top margin length then the text height.
\setlength{\bottommarginlength}{-1in-\textheight-2\baselineskip-\voffset-\headheight-\headsep-\topmargin}

% the length of the left hand margin of an odd page
\setlength{\oddpagemarginlength}{1in+\hoffset+\oddsidemargin-2\marginlinelength}

% the length of the left hand margin of an even page
\setlength{\evenpagemarginlength}{1in+\hoffset+\evensidemargin-2\marginlinelength}

%%%%%%%%%%%%%%%%%%%%%%%%%%%%%%%%%%%%%%%%%%%%%%%%%%%%%%%%%%%%%%%%%%%%%%%%%%%%%%
%% Things related to the vertical line to the left of examples
%%%%%%%%%%%%%%%%%%%%%%%%%%%%%%%%%%%%%%%%%%%%%%%%%%%%%%%%%%%%%%%%%%%%%%%%%%%%%%

% the example environment has a regular and starred version.
% the regular version takes 4 arguments: label, title, problem, solution.
% The starred version only has 3: label, title, problem/solution.
\makeatletter
\newcommand{\example}{\@ifstar \examplestarred \examplenostar}
%\newcommand{\reverseexampledefault}{%
% \renewcommand{\example}{\@ifstar \examplenostar \examplestarred}%
%}
%\newcommand{\restoreexampledefault}{%
% \renewcommand{\example}{\@ifstar \examplestarred \examplenostar}%
%}
\makeatother


\newcommand{\solution}{\bigbreak\par
\makebox[65pt][l]{\textsc{\small\textbf{Solution\lxAddClass{solutionTag}}}}%
}

% This is the no-star (regular) version of the example command.
\newcommand{\examplenostar}[4]{\examplestarred{#1}{#2}{#3\solution #4}}


% black: hsl(x,x,0)
% white: hsl(x,x,100)
% blue: hsl(240,100,50)
% line color: blue!95!black!30 = Hsb(240,.29,.98) = hsl(240,87.7,83.8)




%%%%%%%%%%%%%%%%%%%%%%%%%%%%%%%%%%%%%%%%%%%%%%%%%%%%%%%%%%%%%%%%%%%%%%
%% Define style for Definitions, Theorems and Key Ideas
%%%%%%%%%%%%%%%%%%%%%%%%%%%%%%%%%%%%%%%%%%%%%%%%%%%%%%%%%%%%%%%%%%%%%%


\newlength{\specialboxlength}
\newlength{\specialboxinnerseplength}
\newlength{\specialboxinnerseplengthx}
\newlength{\specialboxinnerseplengthy}

\setlength{\specialboxinnerseplength}{15pt}
\setlength{\specialboxinnerseplengthx}{15pt}
\setlength{\specialboxinnerseplengthy}{10pt}

\newcommand{\setboxwidth}[1]{%
\setlength{\specialboxlength}{\textwidth+#1-2.3\specialboxinnerseplength}}

\newcommand{\restoreboxwidth}{\setboxwidth{0pt}}
\restoreboxwidth

% can't get these to work
%\newcommand*{\specialboxcoloroptions}[1]{%
%     rectangle, text width = \specialboxlength,
%     inner xsep=\specialboxinnerseplengthx, inner ysep=\specialboxinnerseplengthy,
%     draw=border#1, top color=top#1, bottom color=bottom#1,
%     text justified, very thick
%}
%\newcommand{\specialboxbwoptions}{%
%     draw=black, thick, rectangle, text width=\specialboxlength,
%     inner xsep=\specialboxinnerseplengthx, inner ysep=\specialboxinnerseplengthy,
%     draw, text justified, very thick
%}

\newcommand{\newspecialbox}[4][]{%
 \AtBeginDocument{\makeStyles{#2}{#4}}%
 \ifthenelse{\equal{#1}{}}{%
  \definecolor{top#2}{Hsb}{#4,.05,1}% = hsl(#4,100,97.5)
  \ifthenelse{#4=60}{%
   \definecolor{border#2}{Hsb}{#4,.59,.97}% = hsl(#4,90.5,68.4)
   \definecolor{bottom#2}{Hsb}{#4,.28,.97}% = hsl(#4,81.9,83.4)
  }{%
   \definecolor{border#2}{Hsb}{#4,.23,.65}% = hsl(#4,17.6,57.5)
   \definecolor{bottom#2}{Hsb}{#4,.13,.92}% = hsl(#4,42.8,86)
  }%
  \newtheorem{#2Env}{#3}[section]%
  \expandafter\newcommand\csname #2Envautorefname\endcsname{#3}
  \expandafter\newcommand\csname #2\endcsname[4][]{%
   \bigbreak%
   \noindent%
   \flushinner{%
    \noindent%
    \coloredbox{%
     rectangle, text width = \specialboxlength,
     inner xsep=\specialboxinnerseplengthx, inner ysep=\specialboxinnerseplengthy,
     draw=border#2, top color=top#2, bottom color=bottom#2,
     text justified, very thick
    }{%
     draw=black, thick, rectangle, text width=\specialboxlength,
     inner xsep=\specialboxinnerseplengthx, inner ysep=\specialboxinnerseplengthy,
     draw, text justified, very thick
    }{%
     \noindent\begin{#2Env}[{##3}##1]\label{##2}\noindent ##4\end{#2Env}%
    }%
   }%
   \bigbreak%
   \restoreboxwidth%
  }%
 }{% else #1 != '' means we're in exvideo (aka, a qr code to youtube)
%%%%% this is no longer used
%  \definecolor{top#2}{Hsb}{#4,.05,1}% %= hsl(#4,100,97.5)
%  \definecolor{border#2}{Hsb}{#4,.3,1}% %= hsl(#4,90.5,68.4)
%  \definecolor{bottom#2}{Hsb}{#4,.15,1}% %= hsl(#4,81.9,83.4)
%  \expandafter\newcommand\csname #2\endcsname[1]{%
%   \bigbreak%
%   \noindent%
%%   \flushinner{%
%%    \noindent%
%    \coloredbox{% same options as above
%     rectangle, text width = \specialboxlength,
%     inner xsep=\specialboxinnerseplengthx, inner ysep=\specialboxinnerseplengthy,
%     draw=border#2, top color=top#2, bottom color=bottom#2,
%     text justified, very thick
%    }{%
%     draw=black, thick, rectangle, text width=\specialboxlength,
%     inner xsep=\specialboxinnerseplengthx, inner ysep=\specialboxinnerseplengthy,
%     draw, text justified, very thick
%    }{\noindent ##1}%
%%   }%
%   \bigbreak%
%  }%
 }% ends ifthenelse #1=''
}

\newspecialbox{definition}{Def\-i\-ni\-tion}{60}
% draw = yellow!95!black!60 = Hsb( 60,.59,.97)
% topc = white!95!yellow    = Hsb( 60,.05,1)
% botc = yellow!90!black!30 = Hsb( 60,.28,.97)

\newspecialbox{theorem}{Theo\-rem}{120}
% draw = green!30!black!50  = Hsb(120,.23,.65)
% topc = white!95!green     = Hsb(120,.05,1)
% botc = green!60!black!20  = Hsb(120,.13,.92)

\newspecialbox{keyidea}{Key I\-dea}{0}
% draw = red!30!black!50    = Hsb(  0,.23,.65)
% topc = white!95!red       = Hsb(  0,.05,1)
% botc = red!60!black!20    = Hsb(  0,.13,.92)






%%%%%%%%%%%%%%%%%%%%%%%%%%%%%%%%%%%%%%%%%%%%%%%%%%%%%%%%%%%%%%%%%
%% Begins the exercise section, containing all commands
%% related to creating problem sections.
%%%%%%%%%%%%%%%%%%%%%%%%%%%%%%%%%%%%%%%%%%%%%%%%%%%%%%%%%%%%%%%%%

\setlength{\columnsep}{20pt}

\makeatletter
\newcommand{\exercisesubsubsection}{%
\@startsection{subsubsection}{3}{-1em}{\bigskipamount}{\bigskipamount}{\Large\textit}*}
\makeatother

\newcommand{\printconcepts}{\closeenumerate\exercisesubsubsection{Terms and Concepts}}
\newcommand{\printproblems}{\closeenumerate\exercisesubsubsection{Problems}}
\newcommand{\printreview}{\closeenumerate\exercisesubsubsection{Review}}

\newcommand{\inanswersection}{%
	\setboolean{printquestions}{false}%
	\renewcommand{\printconcepts}{}%
	\renewcommand{\printproblems}{}%
	\renewcommand{\printreview}{}%
}%

\newboolean{printquestions}
\newboolean{printoddanswersonly}

\setboolean{printoddanswersonly}{true}
\newcommand{\printallanswers}{\setboolean{printoddanswersonly}{false}}

\newcount\numberofexercises


\newwrite\answrite %write the answers file
% give the answers file the name ``jobname.answers''
\openout\answrite=\jobname.answers



% \writeToAnsFile (was?) in the style file b/c latexml was copying in ``noexpand''
\newcommand{\writeToAnsFile}[1]{%
 \immediate\write\answrite{%
  \noexpand\answersForSection{\arabic{chapter}}{\arabic{section}}{#1}%
 }%
}



%%%%%%%%%%%%% Used to automate the answer production at
%%%%%%%%%%%%% end the text.


\newcommand{\makeexercisesection}[2][\jobname]{%
%\showans\answrite
% showans will write the contents of Calculus.answers to the log
% in latexml, this needs to happen before closing
\immediate\closeout\answrite%
\inanswersection\exercisegeometry\exerciseheader%
\chapter*{#2}%
\addcontentsline{toc}{chapter}{#2}%
%\thispagestyle{empty}%
\ifthenelse{\equal{#1}{\jobname}}{}{%
 {% to localize the next line
  \let\thefootnote\relax%
  \footnotetext{Revised \today}%
 }%
}%
\begin{multicols}{2}%
\small\raggedright%
\input{#1.answers}%
\end{multicols}%
\restoregeometry\regularheader%
\setlength{\hoffset}{0pt}\rmfamily}%


\setboolean{printquestions}{true}
\newcounter{saveexerciseenum}[section]
\newcounter{exercisesetcounter}
\newboolean{enumopen}
\setboolean{enumopen}{false}

\makeatletter
% the usual \subsection definition has stretchable space in arguments 3-5
\newcommand{\exercisesubsection}[1]{%
\@startsection{subsection}{2}{-.7em}{0pt}{2pt}{\huge\textbf}{\texorpdfstring{\hyperref[sol#1]{Exercises #1}}{Exercises}}%
\hrule\vspace{-1.5ex}%
}
\makeatother

\newcommand{\exautoref}[1]{%
 \hyperref[#1]{Exercise~\ref*{#1}}%
% {%
%  \renewcommand{\Itemautorefname}{Exercise}% localize the upcoming reference
%  \autoref{#1}%
% }%
% doesn't work?
}

%\newcommand{\inputexercises}[1]{}%
\newcommand{\printexercises}[1]{%
 \writeToAnsFile{#1}% writeToAnsFile in sty
 \exercisegeometry%
 \exerciseheader%
 \exercisesubsection{\thesection}%
 \label{exer\thesection}%
 \small%
 \bigskip%
 \begin{multicols}{2}%
  \renewcommand{\Itemautorefname}{Exercise}% local b/c multicols = good
  \input{#1}%
  \closeenumerate%
 \end{multicols}%
 \restoregeometry%
 \regularheader%
 %	\easypagecheck
 \setlength{\hoffset}{0pt} \rmfamily\normalsize \bigbreak%
}

\newcounter{answerchapter}
\newcounter{answersection}[answerchapter]
\renewcommand{\theanswersection}{\theanswerchapter.\arabic{answersection}}
\newcommand{\lastanswerchapter}{-1}

% only called by the prerequisite sections
%\newcommand*{\allAnswersForSection}[3]{
% \subsection*{\hyperref[exer#1.#2]{Exercises #1.#2}}\label{sol#1.#2}
% \loadAllAnswers{#3}
%}
\newcommand*{\answersForSection}[3]{%
 \ifthenelse{\equal{#1}{\lastanswerchapter}}{}{%
  \renewcommand{\lastanswerchapter}{#1}% apparently global. who knew?
  \section*{Chapter #1}
  \ifthenelse{\boolean{latexml}}{%
  }{%
   \belowpdfbookmark{Chapter #1}{solsol#1}%
  }
  % commandeer chapter and section numbering
  \setcounter{chapter}{#1}
%  \setcounter{answerchapter}{#1}
%  \setcounter{answersection}{0}
 }%
 \setcounter{section}{#2}
 \subsection*{\hyperref[exer#1.#2]{Exercises #1.#2}}\label{sol#1.#2}
 \ifthenelse{\equal{#2}{0}}{%
  \loadAllAnswers{#3}
 }{%
  \loadAnswers{#3}
 }%
}

% only called by the prerequisite sections
\newcommand*{\loadAllAnswers}[1]{%
	\setcounter{answersection}{-1}%
	\ifthenelse{\boolean{printoddanswersonly}}{%
		\setboolean{printoddanswersonly}{false}%
		\loadAnswers{#1}%
		\setboolean{printoddanswersonly}{true}%
	}{%
		\loadAnswers{#1}%
	}%
}
\newcommand*{\loadAnswers}[1]{%
	\stepcounter{answersection}%
	\lxenumerate{leftmargin=1.5em}\input{#1}\end{enumerate}%
	\bigbreak%
}

\newcommand{\openenumerate}{\ifthenelse{\boolean{printquestions}%
										\AND\NOT\boolean{enumopen}}{%
 \setboolean{enumopen}{true}%
 \lxenumerate{leftmargin=1.5em}%
 \setcounter{enumi}{\value{saveexerciseenum}}%
}{}}

\newcommand{\closeenumerate}{\ifthenelse{\boolean{printquestions}%
										\AND\boolean{enumopen}}{%
 \setcounter{saveexerciseenum}{\value{enumi}}%
 \end{enumerate}%
 \setboolean{enumopen}{false}%
}{}}

\newcommand{\exerciseset}[3]{\ifthenelse{\boolean{printquestions}}{%
 \closeenumerate%
 \stepcounter{saveexerciseenum}%
 \stepcounter{exercisesetcounter}%
 \ifthenelse{\isodd{\value{saveexerciseenum}}}{}{%
  \PackageInfo{apex}{Section \thesection\space has an exercise set beginning with \arabic{saveexerciseenum}}%
 }%
 \noindent#1 \arabic{saveexerciseenum}--\ref*{enumiatendof\theexercisesetcounter}#2%
 \addtocounter{saveexerciseenum}{-1}\ignorespaces%
 #3
 \label{enumiatendof\theexercisesetcounter}%
 \ifthenelse{\isodd{\value{enumi}}}{%
  \PackageInfo{apex}{Section \thesection\space has an exercise set ending with \arabic{enumi}}%
 }{}%
}{
 #3
}}

\newcommand{\exercise}[2]{\ifthenelse{\boolean{printquestions}}{%
 \openenumerate%
 \item \parbox[t]{\linewidth}{#1}%
}{%
 \ifthenelse{\boolean{printoddanswersonly}\AND\isodd{\value{enumi}}}{%
  \stepcounter{enumi}%
 }{%
  \item \parbox[t]{\linewidth}{#2}%
 }%
}}

\newcommand{\questioncolumnbreak}{%
 \ifthenelse{\boolean{printquestions}}{\columnbreak}{}%
}
