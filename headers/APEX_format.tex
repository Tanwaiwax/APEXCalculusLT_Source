%%% This was originally a style with \usepackage, but inputing is generally
%%% equivalent.  The only real difference is how latexml handles style files.
%%% So we'll input this document as a header instead,
%%% and save \usepackage{customstyle}
%%% for things latexml is having trouble with.

\newboolean{printlabelname}
\setboolean{printlabelname}{false}
\ifthenelse{\boolean{printlabelname}}{\usepackage[notref,notcite]{showkeys}}{}

%%%%
%% Begins the usepackage section
%%%%

\RequirePackage{amsmath}
\RequirePackage{amssymb}
\RequirePackage{amsthm}

\RequirePackage{ifthen}

\RequirePackage{graphicx}
\RequirePackage{multicol}
\RequirePackage{tikz}
\usetikzlibrary{calc}
\RequirePackage{makeidx}

\RequirePackage[normalem]{ulem}

%linkbordercolor=white
%\RequirePackage{fancyhdr}
\RequirePackage{calc}

\RequirePackage{thmtools}

\RequirePackage{latexml}

\newtheoremstyle{apex}% name
  {0pt}% Space above, empty = `usual value'
  {}% Space below
  {}% Body font
  {}% Indent amount (empty = no indent, \parindent = para indent)
  {\bfseries}% Thm head font
  {}% Punctuation after thm head
  {\newline}% Space after thm head: " " = normal interword space; \newline = linebreak
  {\thmname{\textbf{#1}}\thmnumber{ \textbf{#2}}\thmnote{\hskip 20pt \textbf{#3}}}% Thm head spec

\theoremstyle{apex}
\newtheorem{exampleEnv}{Example}[section]


% http://tex.stackexchange.com/q/74857/107497
\usepackage{etoolbox}
\makeatletter
\patchcmd\thmtlo@chaptervspacehack
  {\addtocontents{loe}{\protect\addvspace{10\p@}}}
  {\addtocontents{loe}{\protect\thmlopatch@endchapter\protect\thmlopatch@chapter{\thechapter}}}
  {}{failed thmtlo@chaptervspacehack}
\AtEndDocument{\addtocontents{loe}{\protect\thmlopatch@endchapter}}
\long\def\thmlopatch@chapter#1#2\thmlopatch@endchapter{%
  \setbox\z@=\vbox{#2}%
  \ifdim\ht\z@>\z@
    \hbox{\bfseries\chaptername\ #1}\nobreak
    #2
    \addvspace{10\p@}
  \fi
}
\def\thmlopatch@endchapter{}
\patchcmd\thmt@mklistcmd
  {\protect\numberline{\csname the\thmt@envname\endcsname}%
      \thmt@thmname}{}{}{failed thmt@mklistcmd}
%\makeatother
\renewcommand\thmtformatoptarg[1]{#1}

%\makeatletter
\renewenvironment{proof}[1][\proofname]{\par
  \pushQED{\qed}%
  \normalfont \topsep6\p@\@plus6\p@\relax
  \trivlist
  \item[\hskip\labelsep
        \bfseries
    #1]\indent\par % I don't understand why \indent is needed to get a newline
}{%
  \popQED\endtrivlist\@endpefalse
}
\makeatother
\renewcommand{\qedsymbol}{\ensuremath{\square}}


\makeindex

\newcommand{\apex}{A\kern -1pt \lower -2pt\hbox{P}\kern -4pt \lower .7ex\hbox{E}\kern -1pt X}

\renewcommand{\partname}{\protect\thispagestyle{empty}}
\renewcommand{\thepart}{}


%%%%
%%  Create boolean for whether or not to print 3D graphics. 
%%  Also creates command to switch back and forth; "looks better."
%%%%
\newboolean{in_threeD}
\setboolean{in_threeD}{true}
\newcommand{\usethreeDgraphics}{\setboolean{in_threeD}{true}}
\newcommand{\usetwoDgraphics}{\setboolean{in_threeD}{false}}


%%%%
%% Commands to determine whether we print in color or 
%% black and white
%%%%

\usepackage{pgfplots}
\pgfplotsset{compat=1.8}

\newboolean{inColor}
\setboolean{inColor}{true}

\pgfplotsset{colormap={coloronemap}{rgb=(.4,.4,1); rgb=(.8,.8,1)}}
\pgfplotsset{colormap={colortwomap}{rgb=(1,.4,.4); rgb=(1,.8,.8)}}
\usepgfplotslibrary{external}
\usetikzlibrary{calc}
\usetikzlibrary{shadings}

% these will be renewcommanded
\newcommand{\colorone}{blue}
\newcommand{\colortwo}{red}
\newcommand{\colorthree}{green}
\newcommand{\coloronefill}{blue!15!white}
\newcommand{\colortwofill}{red!15!white}
\newcommand{\colormapone}{rgb=(.4,.4,1); rgb=(.8,.8,1)}
\newcommand{\colormaptwo}{rgb=(1,.4,.4); rgb=(1,.8,.8)}
\newcommand{\colormapplaneone}{rgb=(.7,.7,1); rgb=(.9,.9,1)}
\definecolor{colormaponebottom}{rgb}{.4,.4,1}
\definecolor{colormaponetop}{rgb}{.8,.8,1}
\definecolor{colormaptwobottom}{rgb}{1,.4,.4}
\definecolor{colormaptwotop}{rgb}{1,.8,.8}

% determines the line colors for color and black and white lines.
\newcommand{\colorlinecolor}{blue!95!black!30}
\newcommand{\bwlinecolor}{black!30}

% sets the line color to be in color, as a default
\newcommand{\thelinecolor}{\colorlinecolor}

\newcommand{\setcolorlinecolor}[1]{\renewcommand{\colorlinecolor}{#1}\renewcommand{\thelinecolor}{\colorlinecolor}}
\newcommand{\setbwlinecolor}[1]{\renewcommand{\bwlinecolor}{#1}\renewcommand{\thelinecolor}{\bwlinecolor}}

% this allows the above default to be overriden by using
% the \printincolor and \printinblackandwhite commands
% anywhere in the file. This allows you to switch back
% and forth between bw and color. (Who would want to?)
\newcommand{\colornamesuffix}{}

\newcommand{\printincolor}{\setboolean{inColor}{true}%
\renewcommand{\thelinecolor}{\colorlinecolor}
\renewcommand{\colornamesuffix}{}
% aforementioned renewcommanding
\renewcommand{\colorone}{blue}
\renewcommand{\colortwo}{red}
\renewcommand{\colorthree}{green}
\renewcommand{\coloronefill}{blue!15!white}
\renewcommand{\colortwofill}{red!15!white}
\renewcommand{\colormapone}{rgb=(.4,.4,1); rgb=(.8,.8,1)}
\renewcommand{\colormaptwo}{rgb=(1,.4,.4); rgb=(1,.8,.8)}
\renewcommand{\colormapplaneone}{rgb=(.7,.7,1); rgb=(.9,.9,1)}
\definecolor{colormaponebottom}{rgb}{.4,.4,1}
\definecolor{colormaponetop}{rgb}{.8,.8,1}
\definecolor{colormaptwobottom}{rgb}{1,.4,.4}
\definecolor{colormaptwotop}{rgb}{1,.8,.8}
}

\newcommand{\printinblackandwhite}{\setboolean{inColor}{false}%
\renewcommand{\thelinecolor}{\bwlinecolor}
\renewcommand{\colornamesuffix}{BW}
% the counter to the above renewcommanding
\renewcommand{\colorone}{black}
\renewcommand{\colortwo}{black!50!white}
\renewcommand{\colorthree}{black!25!white}
\renewcommand{\coloronefill}{black!15!white}
\renewcommand{\colortwofill}{black!05!white}
\renewcommand{\colormapone}{rgb=(.4,.4,.4); rgb=(.7,.7,.7)}
\renewcommand{\colormaptwo}{rgb=(.6,.6,.6); rgb=(.9,.9,.9)}
\renewcommand{\colormapplaneone}{rgb=(.8,.8,.8); rgb=(.95,.95,.95)}
\definecolor{colormaponebottom}{rgb}{.4,.4,.4}
\definecolor{colormaponetop}{rgb}{.7,.7,.7}
\definecolor{colormaptwobottom}{rgb}{.6,.6,.6}
\definecolor{colormaptwotop}{rgb}{.9,.9,.9}
}

\newcommand{\threedlines}[4][]{\draw [dashed,#1] (axis cs: #2,#3,#4) -- (axis cs: #2,#3,0) -- (axis cs: #2,0,0)  (axis cs: #2,#3,0)--(axis cs:0,#3,0);}

\newcommand{\mydraw}{\draw (axis cs:0,0,0) -- (axis cs:1,1,0);}

\newcommand{\myincludegraphics}[2][]{\includegraphics[#1]{#2\colornamesuffix}}

% the default is printing in color
\printincolor


%%%
%% Creates a lot of measurements - lengths - to use
%% later on. Explained when a value is set.
%%%


\newlength{\topmarginlength} 
\newlength{\bottommarginlength}
\newlength{\oddpagemarginlength}
\newlength{\evenpagemarginlength}
\newlength{\marginlinelength}


% measures how far from the text the example line is to be drawn
\setlength{\marginlinelength}{2pt}

% the height of the top margin
% used in calculating the lines for examples
\setlength{\topmarginlength}{-1in-\voffset}

% the length of the bottom margin (ish)
% actually starts at the top of the page, moves
% through the top margin length then the text height.
\setlength{\bottommarginlength}{-1in-\textheight-2\baselineskip-\voffset-\headheight-\headsep-\topmargin}

% the length of the left hand margin of an odd page
\setlength{\oddpagemarginlength}{1in+\hoffset+\oddsidemargin-2\marginlinelength}

% the length of the left hand margin of an even page
\setlength{\evenpagemarginlength}{1in+\hoffset+\evensidemargin-2\marginlinelength}


%%%%%%%%%%%%%%%%%%%%%%%%%%%%%%%%%%%%%%%%%%%%%%%%%%%%%%%%%%%%%%%%%%%%%%%%%%%%%%%%%%%%%%%%%%%%%%%%%%%
%% Things related to the vertical line to the left of examples
%%%%%%%%%%%%%%%%%%%%%%%%%%%%%%%%%%%%%%%%%%%%%%%%%%%%%%%%%%%%%%%%%%%%%%%%%%%%%%%%%%%%%%%%%%%%%%%%%%%

% creates a generic style for the lines. You can add lots
% of things here, all separated by commas.
\newcommand{\linestyle}{[thick, \thelinecolor]}

\newcommand{\setlinestyle}[1]{\renewcommand{\linestyle}{[#1, \thelinecolor]}}

% Do you want to draw the lines for examples? If so,
% use the first command. Otherwise, use the second. 
% The first is the default, but you can override it
% by using the second in your main file.
\newboolean{showexamplelines}
\newcommand{\drawexamplelines}{\setboolean{showexamplelines}{true}}
\newcommand{\nodrawexamplelines}{\setboolean{showexamplelines}{false}}

% by default the lines around the examples are drawn
\drawexamplelines

% This is more a debugging tool than a stylistic one.
% This draws a small circle in the margin at the the
% beginning of an example and another immediately following
% the end of the example, in the text. Sometimes it is useful
% to figure out why there seems to be a lot of space between
% the end of the example and the end of the drawn line.
% Too much space can occur, for instance, if the example ends with
% with a ``$$ maths $$'' type environment. You can correct for this
% with a \vskip -\baselineskip command.
\newboolean{showexamplemarks}
\newcommand{\drawexamplemarks}{\setboolean{showexamplemarks}{true}}
\newcommand{\nodrawexamplemarks}{\setboolean{showexamplemarks}{false}}
%\drawexamplemarks
\nodrawexamplemarks

\newboolean{exampleisintext}
\newcommand{\exampleinmargin}{\setboolean{exampleisintext}{false}}
\newcommand{\exampleintext}{\setboolean{exampleisintext}{true}}
\exampleintext

\newcounter{examplecounter}[section]
\setcounter{examplecounter}{0}

% the example environment has a regular and starred version.
% the regular version takes 4 arguments: label, title, problem, solution.
% The starred version only has 3: label, title, problem/solution.
\makeatletter
\newcommand{\example}{\@ifstar \examplestarred \examplenostar}
\newcommand{\reverseexampledefault}{\renewcommand{\example}{\@ifstar \examplenostar \examplestarred}}
\newcommand{\restoreexampledefault}{\renewcommand{\example}{\@ifstar \examplestarred \examplenostar}}
\makeatother

%% This is the no-star (regular) version of
%% the example command.
\newcommand{\examplenostar}[4]{%
%\ifthenelse{\boolean{inColor}}{renewcommand{\thelinecolor}{\colorlinecolor}}{\renewcommand{\thelinecolor}{\bwlinecolor}}%
\noindent%
\ifthenelse{\boolean{exampleisintext}}
{% default: the Example word is in text
\hskip -2\marginlinelength%
\parbox{\marginlinelength}{%
\begin{tikzpicture}[remember picture,overlay]%
\ifthenelse{\boolean{showexamplelines}}{\ifthenelse{\boolean{showexamplemarks}}
{\draw (0,0) circle (1pt) node (#1) {};}%
{\draw (0,0) node (#1) {};}
}
{}% ends the ``Draw example lines and marks?'' booleans
\end{tikzpicture}%
}% ends parbox where line begins
\hskip \marginlinelength% puts us back in line
\parbox{80pt}{\textbf{Example~\refstepcounter{examplecounter}\theexamplecounter\label{#1}}}% ends parbox
%\ifthenelse{\boolean{printlabelname}}%
%{\hskip -35pt%
%\parbox{35pt}{\tiny\texttt{\detokenize{#1}}}%
%}%
%{}%
}% ends the if example in text; next do if not in text
{%
%\hskip -\marginlinelength%
\hskip -70pt%
\parbox{\marginlinelength}{%
\begin{tikzpicture}[remember picture,overlay]%
\ifthenelse{\boolean{showexamplelines}}{\ifthenelse{\boolean{showexamplemarks}}
{\draw (0,0) circle (1pt) node (#1) {};}%
{\draw (0,0) node (#1) {};}
}%
{}% ends the ``Draw example lines and marks?'' booleans
\end{tikzpicture}%
}%
\hskip -\marginlinelength
\parbox{70pt}{\textbf{Example~\refstepcounter{examplecounter}\theexamplecounter\label{#1}}}% ends parbox
}% ends if example in text
\ifthenelse{\isOnePageLong{#1}}%
{%
\textbf{#2}\\
#3 \vskip \baselineskip%
\ifthenelse{\boolean{exampleisintext}}
{%solution is in text
\parbox{65pt}{\textsc{\small\bfseries Solution}}% 
}% solution is in margin
{\hskip -80pt%
\parbox{65pt}{\textsc{\small\bfseries Solution}}% 
\hskip 15pt%
}% end solution in margin
#4\nopagebreak\label{e#1}% writes the full example then draws the lines
\ifthenelse{\boolean{showexamplelines}}{%
\begin{tikzpicture}[remember picture,overlay]%
\ifthenelse{\boolean{showexamplemarks}}
{\draw (0,0) circle (1pt) node (e#1) {};}%
{\draw (0,0) node (e#1) {};}%
\ifthenelse{\boolean{exampleisintext}}
{\draw \linestyle (#1) -- (#1.south |- e#1.south) -- ++(10pt,0);}
{\draw \linestyle ([yshift=-3pt]#1.south) -- ++(70pt-2\marginlinelength,0) node (f#1) {} -- (f#1 |- e#1.south) -- ++(10pt,0);} 
\end{tikzpicture}
}{}% ends if/then/else show lines
}% ends if beginning and end are on same page.
% next is if they are on different pages
{% first draw line from start to bottom of page
\ifthenelse{\boolean{showexamplelines}}{%
\begin{tikzpicture}[remember picture,overlay]% 
\ifthenelse{\isOddPage{#1}}% draws lines based on whether on an even or odd page %
{\node [xshift=\oddpagemarginlength,yshift=\bottommarginlength](bottomleft) at (current page.north west)  {};}%
{\node [xshift=\evenpagemarginlength,yshift=\bottommarginlength](bottomleft) at (current page.north west)  {};}%
\ifthenelse{\boolean{exampleisintext}}% is example in text?
{\draw \linestyle (#1) -- (bottomleft);}%
{\draw \linestyle ([yshift=-3pt]#1.south) -- ++(70pt-2\marginlinelength,0) --(bottomleft);}% 
\end{tikzpicture}%
}{}% ends if/then/else show lines
% end drawing of line
\textbf{#2}\\
#3 \vskip \baselineskip%
\ifthenelse{\boolean{exampleisintext}}%
{%solution is in text
\parbox{65pt}{\textsc{\small\bfseries Solution}}% 
}% solution is in margin
{\hskip -80pt%
\parbox{65pt}{\textsc{\small\bfseries Solution}}% 
\hskip 15pt%
}% end solution in margin
#4\label{e#1}% now write out full example
% now draw line from end to top of page
\ifthenelse{\boolean{showexamplelines}}{%
\begin{tikzpicture}[remember picture,overlay] 
\ifthenelse{\boolean{showexamplemarks}}
{\draw (0,0) circle (1pt) node (e#1) {};}%
{\draw (0,0) node (e#1) {};}%
\ifthenelse{\isOddPage{e#1}}% draws lines based on whether on an even or odd page %
{\node [xshift=\oddpagemarginlength,yshift=\topmarginlength](topleft) at (current page.north west)  {};}
{\node [xshift=\evenpagemarginlength,yshift=\topmarginlength](topleft) at (current page.north west)  {};}
\draw \linestyle (topleft)--(e#1.south -| topleft) -- ++(10pt,0);
\end{tikzpicture}%
}{}% ends if/then/else show lines
}% ends the check for same page or not
%
}%ends the definition of example


\newcommand{\examplestarred}[3]{%
\noindent%
\ifthenelse{\boolean{exampleisintext}}
{% default: the Example word is in text
\hskip -2\marginlinelength%
\parbox{\marginlinelength}{%
\begin{tikzpicture}[remember picture,overlay]%
\ifthenelse{\boolean{showexamplelines}}{\ifthenelse{\boolean{showexamplemarks}}
{\draw (0,0) circle (1pt) node (#1) {};}%
{\draw (0,0) node (#1) {};}
}
{}% ends the ``Draw example lines and marks?'' booleans
\end{tikzpicture}%
}% ends parbox where line begins
\hskip \marginlinelength% puts us back in line
\parbox{80pt}{\textbf{Example~\refstepcounter{examplecounter}\theexamplecounter\label{#1}}}% ends parbox
}% ends the if example in text; next do if not in text
{%
\hskip -70pt%
\parbox{\marginlinelength}{%
\begin{tikzpicture}[remember picture,overlay]%
\ifthenelse{\boolean{showexamplelines}}{\ifthenelse{\boolean{showexamplemarks}}
{\draw (0,0) circle (1pt) node (#1) {};}%
{\draw (0,0) node (#1) {};}
}%
{}% ends the ``Draw example lines and marks?'' booleans
\end{tikzpicture}%
}%
\hskip -\marginlinelength
\parbox{70pt}{\textbf{Example~\refstepcounter{examplecounter}\theexamplecounter\label{#1}}}% ends parbox
}% ends if example in text
\ifthenelse{\isOnePageLong{#1}}%
{%
\textbf{#2}\\
#3 \label{e#1}% writes the full example then draws the lines
\ifthenelse{\boolean{showexamplelines}}{%
\begin{tikzpicture}[remember picture,overlay]%
\ifthenelse{\boolean{showexamplemarks}}
{\draw (0,0) circle (1pt) node (e#1) {};}%
{\draw (0,0) node (e#1) {};}%
\ifthenelse{\boolean{exampleisintext}}
{\draw \linestyle (#1) -- (#1.south |- e#1.south) -- ++(10pt,0);}
{\draw \linestyle ([yshift=-3pt]#1.south) -- ++(70pt-2\marginlinelength,0) node (f#1) {} -- (f#1 |- e#1.south) -- ++(10pt,0);}%
\end{tikzpicture}%
}{}% ends if/then/else show lines
}% ends if beginning and end are on same page.
% next is if they are on different pages
{% first draw line from start to bottom of page
\ifthenelse{\boolean{showexamplelines}}{%
\begin{tikzpicture}[remember picture,overlay]% 
\ifthenelse{\isOddPage{#1}}% draws lines based on whether on an even or odd page %
{\node [xshift=\oddpagemarginlength,yshift=\bottommarginlength](bottomleft) at (current page.north west)  {};}%
{\node [xshift=\evenpagemarginlength,yshift=\bottommarginlength](bottomleft) at (current page.north west)  {};}%
\ifthenelse{\boolean{exampleisintext}}% is example in text?
{\draw \linestyle (#1) -- (bottomleft);}%
{\draw \linestyle ([yshift=-3pt]#1.south) -- ++(70pt-2\marginlinelength,0) --(bottomleft);}%
\end{tikzpicture}%
}{}% ends if/then/else show lines
% end drawing of line
\textbf{#2}\\
#3 \label{e#1}% now write out full example
% now draw line from end to top of page
\ifthenelse{\boolean{showexamplelines}}{%
\begin{tikzpicture}[remember picture,overlay] 
\ifthenelse{\boolean{showexamplemarks}}
{\draw (0,0) circle (1pt) node (e#1) {};}%
{\draw (0,0) node (e#1) {};}%
\ifthenelse{\isOddPage{e#1}}% draws lines based on whether on an even or odd page %
{\node [xshift=\oddpagemarginlength,yshift=\topmarginlength](topleft) at (current page.north west)  {};}
{\node [xshift=\evenpagemarginlength,yshift=\topmarginlength](topleft) at (current page.north west)  {};}
\draw \linestyle (topleft)--(e#1.south -| topleft) -- ++(10pt,0);
\end{tikzpicture}%
}{}% ends if/then/else show lines
}% ends the check for same page or not
%
}%ends the definition of examplestarred

% Draws a line on a page that doesn't contain either
% the beginning or end of an example.
% Takes no arguments; figures out if you 
% are on an even or odd numbered page for
% correct margin calculation
\newcommand{\drawexampleline}{%
\begin{tikzpicture}[remember picture,overlay]
				\ifthenelse{\isodd{\thepage}}{%
				\node [xshift=\oddpagemarginlength,yshift=\topmarginlength](tleft) at (current page.north west)  {};}
				{\node [xshift=\evenpagemarginlength,yshift=\topmarginlength](tleft) at (current page.north west)  {};}
        \draw \linestyle (tleft) -- ++(0,\bottommarginlength+1in);
        \end{tikzpicture}%
}


\renewcommand{\examplenostar}[4]{\examplestarred{#1}{#2}{#3\bigskip

\ifthenelse{\boolean{exampleisintext}}
{%solution is in text
\parbox{65pt}{\textsc{\small\textbf{Solution\lxAddClass{solutionTag}}}}% 
}% solution is in margin
{\hskip -80pt%
\parbox{65pt}{\textsc{\small\textbf{Solution\lxAddClass{solutionTag}}}}% 
\hskip 15pt%
}% end solution in margin
\nopagebreak%
#4\vspace{-\baselineskip}\nopagebreak}\bigskip}


\ifthenelse{\boolean{latexml}}{

% the regular version takes 4 arguments: label, title, problem, solution.
% The starred version only has 3: label, title, problem/solution.
%\renewcommand{\examplenostar}[4]{%
%\begin{exampleEnv}[#2]\label{#1}
%\lxAddClass{inExample}
%#3\vskip \baselineskip%
%\textsc{\small\bfseries Solution\lxAddClass{solutionTag}}
%#4
%\end{exampleEnv}~}

\renewcommand{\examplestarred}[3]{%
\begin{exampleEnv}[#2]\label{#1}
\lxAddClass{inExample}
#3
\end{exampleEnv}~}

\renewcommand{\drawexampleline}{}

}{} % end latexml

% black: hsl(x,x,0)
% white: hsl(x,x,100)
% blue: hsl(240,100,50)
% line color: blue!95!black!30 = Hsb(240,.29,.98) = hsl(240,87.7,83.8)




%%%%%%%%%%%%%%%%%%%%%%%%%%%%%%%%%%%%%%%%%%%%%%%%%%%%%%%%%%%%%%%%%%%%%%%%%%%%%%%%%%%%%%%%%%%%%%%%%%%
%% Define style for Definitions, Theorems and Key Ideas
%%%%%%%%%%%%%%%%%%%%%%%%%%%%%%%%%%%%%%%%%%%%%%%%%%%%%%%%%%%%%%%%%%%%%%%%%%%%%%%%%%%%%%%%%%%%%%%%%%%


\newlength{\specialboxlength}
\newlength{\specialboxinnerseplength}
\newlength{\specialboxinnerseplengthx}
\newlength{\specialboxinnerseplengthy}

\setlength{\specialboxinnerseplength}{15pt}
\setlength{\specialboxinnerseplengthx}{15pt}
\setlength{\specialboxinnerseplengthy}{10pt}

\newcommand{\setboxwidth}[1]{%
\setlength{\specialboxlength}{\textwidth+#1-2.2\specialboxinnerseplength}}

\newcommand{\restoreboxwidth}{%
\setlength{\specialboxlength}{\textwidth-2.2\specialboxinnerseplength}}

\restoreboxwidth

\newcommand{\newspecialbox}[4][]{%
 \makeStyles{#2}{#4}%
 \definecolor{top#2}{Hsb}{#4,.05,1}% = hsl(#4,100,97.5)
 \ifthenelse{#4=60}{%
  \definecolor{border#2}{Hsb}{#4,.59,.97}% = hsl(#4,90.5,68.4)
  \definecolor{bottom#2}{Hsb}{#4,.28,.97}% = hsl(#4,81.9,83.4)
 }{%
  \definecolor{border#2}{Hsb}{#4,.23,.65}% = hsl(#4,17.6,57.5)
  \definecolor{bottom#2}{Hsb}{#4,.13,.92}% = hsl(#4,42.8,86)
 }%
 \ifthenelse{\equal{#1}{}}{%
  \newtheorem{#2Env}{#3}%
  \expandafter\newcommand\csname #2\endcsname[4][]{%
   \noindent\coloredbox{
    rectangle, text width = \specialboxlength,
    inner xsep=\specialboxinnerseplengthx, inner ysep=\specialboxinnerseplengthy,
    draw=border#2, top color=top#2, bottom color=bottom#2,
    text justified, very thick
   }{
    draw=black, thick, rectangle, text width=\specialboxlength,
    inner xsep=\specialboxinnerseplengthx, inner ysep=\specialboxinnerseplengthy,
    draw, text justified, very thick
   }{%
    \begin{#2Env}[{##3}##1]\label{##2}##4\end{#2Env}%
   }%
  }%
 }{% else #1 != ''
  \expandafter\newcommand\csname #2\endcsname[1]{%
   \noindent\coloredbox{ % same options as above
    rectangle, text width = \specialboxlength,
    inner xsep=\specialboxinnerseplengthx, inner ysep=\specialboxinnerseplengthy,
    draw=border#2, top color=top#2, bottom color=bottom#2,
    text justified, very thick
   }{
    draw=black, thick, rectangle, text width=\specialboxlength,
    inner xsep=\specialboxinnerseplengthx, inner ysep=\specialboxinnerseplengthy,
    draw, text justified, very thick
   }{##1}%
  }%
 }% ends ifthenelse #1=''
}

\newspecialbox{definition}{Definition}{60}
% draw = yellow!95!black!60 = Hsb( 60,.59,.97)
% topc = white!95!yellow    = Hsb( 60,.05,1)
% botc = yellow!90!black!30 = Hsb( 60,.28,.97)

\newspecialbox{theorem}{Theorem}{120}
% draw = green!30!black!50  = Hsb(120,.23,.65)
% topc = white!95!green     = Hsb(120,.05,1)
% botc = green!60!black!20  = Hsb(120,.13,.92)

\newspecialbox{keyidea}{Key Idea}{0}
% draw = red!30!black!50    = Hsb(  0,.23,.65)
% topc = white!95!red       = Hsb(  0,.05,1)
% botc = red!60!black!20    = Hsb(  0,.13,.92)

%\newcommand{\showcolor}[1]{#1 is
%  \extractcolorspecs{#1}{\model}{\mycolor}
%  \convertcolorspec{\model}{\mycolor}{Hsb}{\converted}\converted\ in Hsb}




%%%%%%%%%%%%%%%%%%%%%%%%%%%%%%%%%%%%%%%%%%%%%%%%%%%%%%%%%%%%%%%%%
%% Begins the exercise section, containing all commands
%% related to creating problem sections.
%%%%%%%%%%%%%%%%%%%%%%%%%%%%%%%%%%%%%%%%%%%%%%%%%%%%%%%%%%%%%%%%%

\newcommand{\exc}{\addtocounter{excounter}{2}\arabic{excounter}}

\newif\ifmore

\newif\ifexsetmore

% this counter gives an effective, albeit not elegant
% way of using the same command to both print questions
% or the answers, depending on what section you are in.
% showexercises = 1: print questions
% showexercises = 2: print odd answers only
% showexercises = 3: print all answers 
% 
% The subsequent lines sets the value 

\setlength{\columnsep}{20pt}

\newcommand{\printconcepts}{\noindent\textit{\Large Terms and Concepts}\vskip \baselineskip}
\newcommand{\printproblems}{\vskip \baselineskip\noindent\textit{\Large Problems}\vskip \baselineskip}
\newcommand{\printreview}{\vskip \baselineskip\noindent\textit{\Large Review}\vskip \baselineskip}

\newcommand{\inanswersection}{%
	\renewcommand{\printconcepts}{}%
	\renewcommand{\printproblems}{}%
	\renewcommand{\printreview}{}%
}%

%\newcount\showexercises % not used?
\newboolean{printquestions}
\newboolean{printoddanswersonly}

\setboolean{printoddanswersonly}{true}
\newcommand{\printallanswers}{\setboolean{printoddanswersonly}{false}}


\newcount\numberofexercises

\newcounter{numofexer}
\newcounter{negnumofexer}

% used for debugging; not really used anymore
\newcounter{debug}
\setcounter{debug}{0}

\newcounter{exercisecounter}
\newcounter{IMTcount}
\newcounter{IMTcount_temp}

% the exercise names can be printed next to the problem
% using these commands. The default is to not print them.
\newboolean{showexercisenames}
\newcommand{\printexercisenames}{\setboolean{showexercisenames}{true}}
\newcommand{\noprintexercisenames}{\setboolean{showexercisenames}{false}}
\noprintexercisenames


% TeX uses a certain system to name input files that it reads from.
% To prevent conflicts, we use the newread command.
\newread\exread %read an example
\newread\exsetread %read an example set
\newread\exansread %read the answer
\newread\printansread% read in the answers file

\newwrite\answrite %write the answers file
% give the answers file the name ``jobname.answers''
\openout\answrite=\jobname.answers

\def\exinput #1 {\ifthenelse{\boolean{printquestions}}{% 
												\openin\exread=#1 
												\read\exread to \tempp%
%												\begin{adjustwidth*}{}{-100pt}% 
												\iflatexml\begin{enumerate}\else\begin{enumerate}[leftmargin=*,topsep=10pt]\fi% 
													\addtocounter{enumi}{\theexercisecounter}%
													\item% 
													\ifthenelse{\boolean{showexercisenames}}%
													{\tiny{\hskip -60pt}% This line too
												  \makebox[60pt][l]{\printexercisename #1 }%  
												  \small%
												  }{}%
													\tempp 
													\addtocounter{exercisecounter}{1}
												\end{enumerate}
%												\end{adjustwidth*}%
												\closein\exread}% end print questions 
									{% else: print answers
									\ifthenelse{\boolean{printoddanswersonly}}{%i.e, we are printing odd answers, not questions 
												\openin\exread=#1 
												\read\exread to \tempp % read in the question - we ignore it.
												\addtocounter{exercisecounter}{1}
												\read\exread to \tempp % reads in the answer
												\ifodd \theexercisecounter
												%\else
													\iflatexml\begin{enumerate}\else\begin{enumerate}[leftmargin=*]\fi 
													\addtocounter{enumi}{\theexercisecounter}
													\addtocounter{enumi}{-1}
													\item 
													\ifthenelse{\boolean{showexercisenames}}
													{\tiny{\hskip -60pt}% This line too
												  \makebox[60pt][l]{\printexercisename #1 }%
												  \small%
												  }{}
													\tempp 
													%\addtocounter{exercisecounter}{1}
													\end{enumerate} 
												\fi
												\closein\exread 
												}  %ends the \ifnum \showexercises = 2 if statement
												{% print all answers
												\openin\exread=#1 
												\read\exread to \tempp %reads in the question, which is ignored 
												\read\exread to \tempp %reads in the answer
												\iflatexml\begin{enumerate}\else\begin{enumerate}[leftmargin=*]\fi 
													\addtocounter{enumi}{\theexercisecounter}
													\item% 
													\ifthenelse{\boolean{showexercisenames}}
													{\tiny{\hskip -60pt}% This line too
												  \makebox[60pt][l]{\printexercisename #1 }%
												  \small%
												  }{}
													\tempp 
													\addtocounter{exercisecounter}{1}
												\end{enumerate}
												\closein\exread
												} % ends the \ifnum \showexercises=3 if statement
									}
								}

\def\exsetinput #1 {\openin\exsetread=#1%
										\setcounter{numofexer}{0}%
										\setcounter{negnumofexer}{0}% 
										\read\exsetread to \exsettemp%
										\read\exsetread to \exsettemp%
										{\loop%
												\read\exsetread to \exsettemp%
												\ifeof \exsetread \exsetmorefalse \else \exsetmoretrue \fi%
												\ifexsetmore%
														\addtocounter{numofexer}{1}%
														\addtocounter{negnumofexer}{-1}%
											\repeat}%							
										\closein\exsetread%
										\openin\exsetread=#1%
										\ifthenelse{\boolean{printquestions}}{% 
											\read\exsetread to \exsettemp%
											\setcounter{enumi}{\theexercisecounter} %
											\addtocounter{enumi}{1}%
											\ifthenelse{\boolean{showexercisenames}}%
													{\tiny{\hskip -60pt}% This line too
												  \makebox[60pt][l]{\printexercisename #1 }%
												  \small%
												  }{}% end show exercise names
%											\begin{adjustwidth*}{}{-100pt}%
											\noindent\textbf{\exsettemp\theenumi\addtocounter{enumi}{-1}%
											\addtocounter{enumi}{\thenumofexer}{ -- }\theenumi%
											\addtocounter{enumi}{\thenegnumofexer}%
											\read\exsetread to \exsettemp \exsettemp}%
%											\end{adjustwidth*}%
											
											{\loop%
													\read\exsetread to \exsettemp%
													\ifeof \exsetread \exsetmorefalse \else \exsetmoretrue \fi%
													\ifexsetmore%
															\exsettemp%
											\repeat}%
										}% ends print exercises; on to print answers
										{	\read\exsetread to \exsettemp%
											\read\exsetread to \exsettemp%
											{\loop%
													\read\exsetread to \exsettemp%
													\ifeof \exsetread \exsetmorefalse \else \exsetmoretrue \fi%
													\ifexsetmore%
															\exsettemp%
											\repeat}%
										}%ends else printing asnwers
										\closein\exsetread%
								}%

% \writeToAnsFile in the style file b/c latexml was copying in ``noexpand''

\def\printexercisesgeneric#1#2#3 {%
\writeToAnsFile{#1}{#2}{#3}%
%\strictpagecheck%
\setcounter{exercisecounter}{0}\setboolean{printquestions}{true}%
%\thispagestyle{empty}%
%\begin{adjustwidth*}{}{-100pt}%
\exercisegeometry%
\exerciseheader%
\small%
\noindent\underline{\parbox{\textwidth}{\textbf{\huge#2} \hyperref[sol#3]{(solutions)}\refstepcounter{subsection}\label{exer#3}}}% 
\sffamily%
\vskip\baselineskip%
\begin{multicols}{2}%
				\openin\exansread=#1 
				\ifeof \exansread 
					{No problems written.} 
				\else 
					\loop \read\exansread to \extemp  
							\ifeof \exansread \morefalse \else \moretrue \fi 
							\ifmore 
									\extemp
							\repeat 
				\fi 
				\closein\exansread 
				\end{multicols}
%				\end{adjustwidth*}%
				\restoregeometry
				\regularheader
%				\easypagecheck
				\setlength{\hoffset}{0pt} \rmfamily\normalsize \vskip \baselineskip%
				} % ends printexercisesgeneric
				


% The following prints the answers. To print all answers, use command \printallanswers.
% 
\def\printanswers #1 {\setcounter{exercisecounter}{0} \footnotesize \setboolean{printquestions}{false} \openin\printansread=#1 
				\ifeof \printansread 
					{No problems written.} 
				\else 
					\loop \read\printansread to \extemp  
							\ifeof \printansread \morefalse \else \moretrue \fi 
							\ifmore 
									\extemp
							\fi 
							\ifeof \printansread \morefalse \else \moretrue \fi 
							\ifmore 
					\repeat 
				\fi 
				\closein\printansread
				\small}

								
\def \printexercisename exercises/#1_#2_#3_#4 {#1 #2 #3 #4}

%%%%%%%%%%%%% Used to automate the answer production at
%%%%%%%%%%%%% end the text.

\ifthenelse{\boolean{longpage}}%
{% if longpage, readsection does nothing, really
\def\readsection #1{#1}%
}%
{% not longpage: usual readsection
\def \readsection #1.#2{#2}
}%

\def \writeexercisestofile #1{%
\write\answrite{\noexpand\printanswers{exercises/0\thechapter_0\expandafter\readsection #1_exercises.tex} \noexpand \vskip \baselineskip }%
\write\answrite{}%
}

\def \makeexercisesection #1{%
\ifthenelse{\boolean{latexml}}{%
\immediate\write\answrite{\end{multicols}\normalsize}\showans\answrite%
}{%
\immediate\write\answrite{\noexpand\end{multicols}\noexpand\normalsize}%
}
\immediate\closeout\answrite\inanswersection\exercisegeometry\exerciseheader%
\chapter{#1}\thispagestyle{empty}%
\input{\jobname.answers}\restoregeometry\regularheader%
\setlength{\hoffset}{0pt}\rmfamily\normalsize}%

%
% The following is a line of code, not a definition. 
% It writes the first line of the ``.answers'' file
% to set up the proper formatting of that file.
%

\ifthenelse{\boolean{latexml}}{
\write\answrite{\small\raggedright\begin{multicols}{2}}
\write\answrite{\thispagestyle{empty}}
}{
\write\answrite{\noexpand\small\noexpand\raggedright\noexpand\begin{multicols}{2}}
\write\answrite{\noexpand\thispagestyle{empty}}
}
