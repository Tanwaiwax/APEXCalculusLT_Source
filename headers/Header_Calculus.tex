%%%%
% Additional packages to support the book, not part of the APEX style.
%%%%
\usepackage{fancyhdr}
%\usepackage{pgfplots}
%\pgfplotsset{width=\marginparwidth+1pt,compat=1.3}
\usepackage[font=small]{caption}
%,justification=centering

\usepackage{wrapfig}

\usepackage{pgfplots}
\pgfplotsset{compat=1.8}

%%%%
%% These are low level LaTeX commands that determine the 
%% look of the Chapter and Section headings.
%% Note the use in the chapter part of an external file
%% that contains graphics for each chapter start.
%%%%

%%%%
%% Commands for the header, utilizing the fancyhdr
%% (fancy header) package
%%%%

\pagestyle{fancy}
\fancyhead{}
\fancyfoot{}
\renewcommand{\chaptermark}[1]{\markboth{\chaptername\ \thechapter\ \ \ \ {#1}}{}}
\renewcommand{\sectionmark}[1]{\markright{\thesection\ \ \ \  #1}}
\fancyhf{}         %Clears all header and footer fields, in preparation.
\renewcommand{\headrulewidth}{0pt}
\renewcommand{\footrulewidth}{0pt}

\ifthenelse{\boolean{longpage}}%
{}% end header of longpage
{% begin header/footer of not longpage
\fancyhfoffset[LE,RO]{\marginparwidth+\marginparsep}
%\fancyfoot[LE,RO]{\textbf{\thepage}} %Displays the page number in bold in the header,

\fancyfoot[LE]{\begin{minipage}{\textwidth}%
\noindent\hskip\marginparwidth\hskip\marginparsep\hskip-4pt\rule{\textwidth}{.4pt}
\vskip.2\baselineskip
\noindent\hskip\marginparwidth\hskip\marginparsep\hskip-4pt%
Notes:
\vskip 1.5in\textbf{\thepage}
\end{minipage}} 

\fancyfoot[RO]{\begin{minipage}{\textwidth+\marginparwidth+\marginparsep}%
\rule{\textwidth-\marginparwidth-\marginparsep}{.4pt}
\vskip.2\baselineskip
Notes:
\vskip 1.5in
\hfill\textbf{\thepage}
\end{minipage}}       


\fancyhead[LE]{\nouppercase{\leftmark}}
      %Displays the upper-level (chapter) information---
      % as determined above---in non-upper case in the header, 
      %to the right on even pages.
\fancyhead[RO]{\rightmark}
			%Displays the lower-level (section) information---as
      % determined above---in the header, to the left on odd pages.
      
}% End the ifthenelse{longpage}


\ifthenelse{\boolean{longpage}}% for changing the header
{% longpage
\newcommand{\exerciseheader}{}
\newcommand{\regularheader}{}
\newcommand{\endmatheader}{}
}% i.e, the above does nothing
{% now for a real change
\newcommand{\exerciseheader}{%
				\fancyhfoffset[LE,RO]{32pt}%
				\fancyfoot[LE]{\textbf{\thepage}}% 
				\fancyfoot[RO]{\textbf{\thepage}}
				\fancyhead{}% 
}

\newcommand{\regularheader}{%
\fancyhead[LE]{\nouppercase{\leftmark}}%
\fancyhead[RO]{\rightmark}%
\fancyfoot[LE]{\begin{minipage}{\textwidth}%
\noindent\hskip\marginparwidth\hskip\marginparsep\hskip-4pt\rule{\textwidth}{.4pt}
\vskip.2\baselineskip
\noindent\hskip\marginparwidth\hskip\marginparsep\hskip-4pt%
Notes:
\vskip 1.5in\textbf{\thepage}
\end{minipage}} 

\fancyfoot[RO]{\begin{minipage}{\textwidth+\marginparwidth+\marginparsep}%
\rule{\textwidth-\marginparwidth-\marginparsep}{.4pt}
\vskip.2\baselineskip
Notes:
\vskip 1.5in
\hfill\textbf{\thepage}
\end{minipage}}
\fancyhfoffset[LE,RO]{\marginparsep+\marginparwidth}
}

\newcommand{\qendheader}{%
		\fancyhead{}%
		\fancyfoot{}%
		}
}% ends if/then/else exercise/regular headers


%
%  Defining what the chapter titles look like
%

\newdimen\titleheight

\makeatletter
\def\@makechapterhead#1{%
  {\parindent \z@ \raggedright \reset@font
    {\Huge \thechapter: \scshape \textsc #1}
    \par\vskip 10\p@
    \hrule height 1pt
    \vskip 10\p@
  }}
%%  
%%%\makeatletter
%%\def\@makesectionhead#1{%
%%	 {\reset@font\LARGE\itshape\bfseries\strut #1 \thechapter.\thesection \ #1
%%	 }}    

\newcommand{\bmx}[1]{\left[\hskip -3pt\begin{array}{#1} }
\newcommand{\emx}{\end{array}\hskip -3pt\right]}

\newcommand{\btz}{\begin{center}\begin{tikzpicture}}
\newcommand{\etz}{\end{tikzpicture}\end{center}}

\newcommand{\ds}{\displaystyle}

\newcommand{\fp}{\ensuremath{f\,'}}
\newcommand{\fpp}{\ensuremath{f\,''}}

\newcommand{\Fp}{\ensuremath{F\primeskip'}}
\newcommand{\Fpp}{\ensuremath{F\primeskip''}}

\newcommand{\yp}{\ensuremath{y\primeskip'}}
\newcommand{\gp}{\ensuremath{g\primeskip'}}

\newcommand{\dx}{\ensuremath{\Delta x}}
\newcommand{\dy}{\ensuremath{\Delta y}}
%\newcommand{\dz}{\ensuremath{\Delta z}}
\newcommand{\ddz}{\ensuremath{\Delta z}}

\newcommand{\thet}{\ensuremath{\theta}}
\newcommand{\abs}[1]{\ensuremath{%
\left\iflatexml\vert\else\lvert\fi\ #1\ \right\iflatexml\vert\else\rvert\fi}}
\newcommand{\norm}[1]{\ensuremath{%
\left\iflatexml\Vert\else\lVert\fi\ #1\ \right\iflatexml\Vert\else\rVert\fi}}
\newcommand{\vnorm}[1]{\ensuremath{\norm{\vec #1}}}
\newcommand{\snorm}[1]{\norm{#1}}%{\ensuremath{\left\lVert\ #1\ \right\rVert}}
\newcommand{\la}{\left\langle}
\newcommand{\ra}{\right\rangle}
\newcommand{\dotp}[2]{\ensuremath{\vec #1 \cdot \vec #2}}
\newcommand{\proj}[2]{\ensuremath{\text{proj}_{\,\vec #2}{\,\vec #1}}}
\newcommand{\crossp}[2]{\ensuremath{\vec #1 \times \vec #2}}
\newcommand{\veci}{\ensuremath{\vec\imath}}
\newcommand{\vecj}{\ensuremath{\vec\jmath}}
\newcommand{\veck}{\ensuremath{\vec k}}
\newcommand{\vecu}{\ensuremath{\vec u}}
\newcommand{\vecv}{\ensuremath{\vec v}}
\newcommand{\vecw}{\ensuremath{\vec w}}
\newcommand{\vecx}{\ensuremath{\vec x}}
\newcommand{\vecy}{\ensuremath{\vec y}}
\newcommand{\vrp}{\ensuremath{\vec r\, '}}
\newcommand{\vsp}{\ensuremath{\vec s\, '}}
\newcommand{\vrt}{\ensuremath{\vec r(t)}}
\newcommand{\vst}{\ensuremath{\vec s(t)}}
\newcommand{\vvt}{\ensuremath{\vec v(t)}}
\newcommand{\vat}{\ensuremath{\vec a(t)}}
\newcommand{\px}{\ensuremath{\partial x}}
\newcommand{\py}{\ensuremath{\partial y}}
\newcommand{\pz}{\ensuremath{\partial z}}
\newcommand{\pf}{\ensuremath{\partial f}}
\newcommand{\underlinespace}{\underline{\phantom{xxxxxx}}}

\newcommand{\mathN}{\ensuremath{\mathbb{N}}}

\newcommand{\zerooverzero}{\ensuremath{\ds \raisebox{8pt}{\text{``\ }}\frac{0}{0}\raisebox{8pt}{\text{\ ''}}}}


\newcommand{\myrule}{\iflatexml\else\rule[-4pt]{0pt}{13pt}\fi}
\newcommand{\mmrule}{\iflatexml\else\rule[-10pt]{0pt}{15pt}\fi}
\newcommand{\myds}{\ds\mmrule}
\newcommand{\deriv}[2]{\ensuremath{\myds\frac{d}{dx}\left(#1\right)=#2}}
\newcommand{\myint}[2]{\ensuremath{\myds\int #1\ dx=} \ensuremath{\ds #2}}

\DeclareMathOperator{\sech}{sech}
\DeclareMathOperator{\csch}{csch}

\newcommand{\sword}[1]{\textbf{#1}}

\newcommand{\primeskip}{\hskip.75pt}

%%%% Begin Header TikZ

%  Some TiKZ  shortcuts to help make drawing 3D vectors faster.
%

\newcommand{\plotlinecolor}{blue}

%
% Draw x and y tick marks
%
\newcommand{\drawxticks}[1]
{\foreach \x in {#1}
		{\draw  (\x,-.1)--(\x,.1);
			};
}
\newcommand{\drawyticks}[1]
{\foreach \x in {#1}
		{\draw  (-.1,\x)--(.1,\x);
			};
}

\newcommand{\drawxlines}[3]
{\draw[<->] (#1,0) -- (#2,0) node [right] {$x$};
\foreach \x in {#3}
		{\draw  (\x,-.1)--(\x,.1);
			};
}

\newcommand{\drawylines}[3]
{\draw[<->] (0,#1) -- (0,#2) node [above] {$y$};
\foreach \x in {#3}
		{\draw  (-.1,\x)--(.1,\x);
			};
}

\newcommand{\drawxlabels}[1]
{\foreach \x in {#1}
		{\draw  (\x,-.1) node [below] {\scriptsize $\x$};
		};
}

\newcommand{\drawylabels}[1]
{\foreach \x in {#1}
		{\draw  (-.1,\x) node [left] {\scriptsize $\x$};
		};
}

%% draw a box of margin width size to see if figure is properly contained within
\newcommand{\marginsizebox}{\draw (0,0)--(\marginparwidth,0)--(\marginparwidth,3)--(0,3)--cycle;}

%%%%
%%%%

\newcommand{\asyouread}[1]{\begin{tikzpicture}
\ifthenelse{\boolean{inColor}}{\node [preaction={fill=black,opacity=.5,transform canvas={xshift=1mm,yshift=-1mm}}, right color=blue!80!black!30, left color=blue!80] at (0,0) {\textcolor{white}{\textsf{\textit{AS YOU READ $\ldots$}}}};}
{\node [preaction={fill=black,opacity=.5,transform canvas={xshift=1mm,yshift=-1mm}}, right color=black!30, left color=black!10] at (0,0) {\textcolor{white}{\textsf{\textit{AS YOU READ $\ldots$}}}};}
\end{tikzpicture}
\begin{enumerate}
#1
\end{enumerate}
\vskip 20pt}

%%%%
%%  A new figure environment, trying to fix the float problem.
%%
%%%%

\newcounter{myfigurecounter}[chapter]
\renewcommand\themyfigurecounter{\thechapter.\arabic{myfigurecounter}}
\newenvironment{myfigure}{\refstepcounter{myfigurecounter}}{}
\newcommand{\mycaption}[1]{%
\begin{center}%
\vskip -1.5\baselineskip
\begin{tikzpicture}%
\draw (0,0) node [text width=\textwidth,align=center] {Figure \themyfigurecounter: #1};%
\end{tikzpicture}%
\end{center}%
}

\newcommand{\video}[3]{\noindent Watch the video:\\%
 \href{#2}{#3 at \nolinkurl{#2}}%
 \embedVideo{https://www.youtube.com/v/#1?rel=0}%
}

\newcommand{\youtubeVideo}[2]{\video{#1}{https://youtu.be/#1}{#2}}

\newcommand{\jmtVideo}[3]{\noindent Watch the video:\\%
 \href{http://patrickjmt.com/#2/}{#3 at \nolinkurl{http://patrickjmt.com/#2/}}%
 \embedVideo{https://www.youtube.com/v/#1?rel=0}%
}

\newcommand{\khanVideo}[4]{\noindent Watch the video:\\%
 \href{https://www.khanacademy.org/video/#2?utm_campaign=embed}{#3 at \nolinkurl{https://www.khanacademy.org/video/#2}}%
 \embedVideo{https://www.youtube.com/v/#1?rel=0}%
}

\ifthenelse{\boolean{xetex}}%
	{%
	\sffamily
	%%\usepackage{fontspec}
	\usepackage{mathspec}
	\setallmainfonts[Mapping=tex-text]{Calibri}
	\setmainfont[Mapping=tex-text]{Calibri}
	\setsansfont[Mapping=tex-text]{Calibri}
	\setmathsfont(Greek){[cmmi10]}}
	{}
	
\ifthenelse{\boolean{luatex}}%
	{%
	\sffamily
	\usepackage{fontspec}
	\usepackage{unicode-math}
	%\usepackage{mathspec}
	%\setallmainfonts[Mapping=tex-text]{Calibri}
	\setmainfont{Calibri}
	%\setsansfont[Mapping=tex-text]{Calibri}
	\setmathfont[range=\mathup]{Calibri}
	\setmathfont[range=\mathit]{Calibri Italic}
	}
	{}

\RequirePackage[bookmarksnumbered,hidelinks,pdfstartview=FitH]{hyperref}
