\newboolean{abridgeConics}
\setboolean{abridgeConics}{true}

\usepackage{fancyhdr}
%\pgfplotsset{width=\marginparwidth+1pt,compat=1.3}
\usepackage[font=small]{caption}
%,justification=centering

%\usepackage{wrapfig}

\usepackage{booktabs}

%\usepackage{esint}
% for oiint in 14_Div_Thm, oint[ctr]clockwise in 14_Line_Integrals_Props

\setcounter{secnumdepth}{1}
\setcounter{tocdepth}{1}

\newlength{\saveparindent}
\setlength{\saveparindent}{\parindent}

%%%%
%% These are low level LaTeX commands that determine the 
%% look of the Chapter and Section headings.
%% Note the use in the chapter part of an external file
%% that contains graphics for each chapter start.
%% But not really?
%%%%

%%%%
%% Commands for the header, utilizing the fancyhdr
%% (fancy header) package
%%%%

\pagestyle{fancy}
\fancyhead{}
\fancyfoot{}
\renewcommand{\chaptermark}[1]{\markboth{\chaptername\ \thechapter\ \ \ \ {#1}}{}}
\renewcommand{\sectionmark}[1]{\markright{\thesection\ \ \ \  #1}}
\fancyhf{}         %Clears all header and footer fields, in preparation.
\renewcommand{\headrulewidth}{0pt}
\renewcommand{\footrulewidth}{0pt}


%\ifthenelse{\boolean{longpage}}% for changing the header
%{% longpage
% \newcommand{\exerciseheader}{}
% \newcommand{\regularheader}{}
%}% i.e, the above does nothing
%{% now for a real change
\newcommand{\exerciseheader}{%
	\fancyhfoffset[LE,RO]{32pt}%
	\fancyfoot[LE]{\textbf{\thepage}}% 
	\fancyfoot[RO]{\textbf{\thepage}}
	\fancyhead{}% 
}
\newcommand{\regularheader}{%
 \fancyhead[LE]{\nouppercase{\leftmark}}%
    %Displays the upper-level (chapter) information---
    % as determined above---in non-upper case in the header, 
    %to the right on even pages.
 \fancyhead[RO]{\rightmark}%
	%Displays the lower-level (section) information---as
    % determined above---in the header, to the left on odd pages.
 \fancyfoot[LE]{\begin{minipage}{\textwidth}%
 \noindent\hspace{\marginparwidth}\hspace{\marginparsep}\hspace{-4pt}\makebox[0pt][l]{\rule{\textwidth}{.4pt}}
 \vskip.2\baselineskip
 \noindent\hspace{\marginparwidth}\hspace{\marginparsep}\hspace{-4pt}%
 Notes:
 \vskip 1.5in\textbf{\thepage}
 \end{minipage}}

 \fancyfoot[RO]{\begin{minipage}{\textwidth+\marginparwidth+\marginparsep}%
 \rule{\textwidth-\marginparwidth-\marginparsep}{.4pt}
 \vskip.2\baselineskip
 Notes:
 \vskip 1.5in
 \hfill\textbf{\thepage}
 \end{minipage}}
 \fancyhfoffset[LE,RO]{\marginparsep+\marginparwidth}
}
\regularheader % and execute

%}% ends if/then/else exercise/regular headers


%
%  Defining what the chapter titles look like
%

%\newdimen\titleheight

%\usepackage{sectsty} % crashed and burned?
%\chapterfont{\raggedright\scshape\textsc\sectionrule{0pt}{0pt}{-10pt}{1pt}}

\makeatletter
\def\@makeschapterhead#1{%
  {\parindent \z@ \raggedright \reset@font
    {\Huge \scshape \textsc #1}
    \par\vskip 10\p@
    \hrule height 1pt
    \vskip 10\p@
  }}
\def\@makechapterhead#1{%
  {\parindent \z@ \raggedright \reset@font
    {\Huge \thechapter: \scshape \textsc #1}
    \par\vskip 10\p@
    \hrule height 1pt
    \vskip 10\p@
  }}
\makeatother


\newcommand{\apexappendix}{%
 \appendix
 \pagenumbering{arabic}
 \renewcommand{\thepage}{A.\arabic{page}}
 \part*{Appendices\protect\thispagestyle{empty}}
 \iflatexml\else
 \pdfbookmark[part]{Appendices}{appendixbookmark}
 \fi
}


% an enumerate like environment that can be mixed into tabular, array, etc.
\newcounter{anywhereenumi}
\newenvironment{anywhereenum}{%
 \setcounter{anywhereenumi}{0}%
 \renewcommand{\item}[1][]{%
  \ifx.##1.%
  \refstepcounter{anywhereenumi}%
  \makebox[1em][r]{\arabic{anywhereenumi}.}~~%
  \else%
  \makebox[1em][r]{##1.}~~%
  \fi%
 }%
}{}

\newcommand{\ds}{\displaystyle}

\newcommand{\primeskip}{\hskip.75pt}

\newcommand{\fp}{\ensuremath{f\,'}}
\newcommand{\fpp}{\ensuremath{f\,''}}

\newcommand{\Fp}{\ensuremath{F\primeskip'}}
\newcommand{\Fpp}{\ensuremath{F\primeskip''}}

\newcommand{\yp}{\ensuremath{y\primeskip'}}
\newcommand{\gp}{\ensuremath{g\primeskip'}}

%\newcommand{\dx}{\ensuremath{\Delta x}}
%\newcommand{\dy}{\ensuremath{\Delta y}}
%\newcommand{\dz}{\ensuremath{\Delta z}}
%\newcommand{\ddz}{\ensuremath{\Delta z}}

\newcommand*{\abs}[1]{\ensuremath{\left\lvert #1 \right\rvert}}
\newcommand*{\norm}[1]{\ensuremath{\left\lVert #1 \right\rVert}}
\newcommand*{\vnorm}[1]{\ensuremath{\norm{\vec #1}}}
\newcommand{\bracket}[1]{\left\langle #1\right\rangle}
\newcommand*{\dotp}[2]{\ensuremath{\vec #1 \cdot \vec #2}}
\newcommand*{\proj}[2]{\ensuremath{\text{proj}_{\,\vec #2}{\,\vec #1}}}
\newcommand*{\crossp}[2]{\ensuremath{\vec #1 \times \vec #2}}
\newcommand{\vecE}{\ensuremath{\vec E}}
\newcommand{\vecF}{\ensuremath{\vec F}}
\newcommand{\vecG}{\ensuremath{\vec G}}
\newcommand{\vecT}{\ensuremath{\vec T}}
\newcommand{\vece}{\ensuremath{\vec e}}
\newcommand{\vecf}{\ensuremath{\vec f}}
\newcommand{\vecg}{\ensuremath{\vec g}}
\newcommand{\veci}{\ensuremath{\vec\imath}}
\newcommand{\vecj}{\ensuremath{\vec\jmath}}
\newcommand{\veck}{\ensuremath{\vec k}}
\newcommand{\vecl}{\ensuremath{\vec l}}
\newcommand{\vecn}{\ensuremath{\vec n}}
\newcommand{\vecr}{\ensuremath{\vec r}}
\newcommand{\vecu}{\ensuremath{\vec u}}
\newcommand{\vecv}{\ensuremath{\vec v}}
\newcommand{\vecw}{\ensuremath{\vec w}}
\newcommand{\vecx}{\ensuremath{\vec x}}
\newcommand{\vecy}{\ensuremath{\vec y}}
\newcommand{\vrp}{\ensuremath{\vec r\hskip1.25pt '}}
\newcommand{\vsp}{\ensuremath{\vec s\primeskip '}}
\newcommand{\vrt}{\ensuremath{\vec r(t)}}
\newcommand{\vst}{\ensuremath{\vec s(t)}}
\newcommand{\vvt}{\ensuremath{\vec v(t)}}
\newcommand{\vat}{\ensuremath{\vec a(t)}}
\newcommand{\px}{\ensuremath{\partial x}}
\newcommand{\py}{\ensuremath{\partial y}}
\newcommand{\pz}{\ensuremath{\partial z}}
\newcommand{\pf}{\ensuremath{\partial f}}
\newcommand{\underlinespace}{\underline{\phantom{xxxxxx}}}

\newcommand{\zerooverzero}{\dfrac{\makebox[0pt]{\text{`` }0\text{ ''}}}0\ \ }

%\newcommand{\myrule}{\iflatexml\else\rule[-4pt]{0pt}{13pt}\fi}
%\newcommand{\mmrule}{\iflatexml\else\rule[-10pt]{0pt}{15pt}\fi}
%\newcommand{\myds}{\ds\mmrule}
%\newcommand*{\deriv}[2]{\ensuremath{\myds\frac{d}{dx}\left(#1\right)=#2}}
%\newcommand*{\myint}[2]{\ensuremath{\myds\int #1\ dx=} \ensuremath{\ds #2}}


\DeclareMathOperator{\sech}{sech}
\DeclareMathOperator{\csch}{csch}
\DeclareMathOperator{\Div}{div}
\DeclareMathOperator{\grad}{grad}
\DeclareMathOperator{\curl}{curl}
\DeclareMathOperator{\divv}{div}

\newcommand*{\sword}[1]{\textbf{#1}}

\newcommand{\LHequals}{\mathrel{\overset{\text{by LHR}}{=}}}

\newcommand{\surfaceS}{\ensuremath{\mathcal{S}}}

%%%% Begin Header TikZ

%  Some TiKZ  shortcuts to help make drawing 3D vectors faster.
%

%% draw a box of margin width size to see if figure is properly contained within
%\newcommand{\marginsizebox}{\draw (0,0)--(\marginparwidth,0)--(\marginparwidth,3)--(0,3)--cycle;}

%%%%
%%%%

%\newspecialbox[notempty]{exvideo}{ignored}{240}
\AtBeginDocument{\makeStyles{exvideo}{240}}
\definecolor{topexvideo}{Hsb}{240,.05,1}% %= hsl(#4,100,97.5)
\definecolor{borderexvideo}{Hsb}{240,.3,1}% %= hsl(#4,90.5,68.4)
\definecolor{bottomexvideo}{Hsb}{240,.15,1}% %= hsl(#4,81.9,83.4)
\newcommand{\exvideo}[1]{%
 \bigbreak%
 \noindent%
  \coloredbox{% same options newspecialbox
   rectangle, text width = \specialboxlength,
   inner xsep=\specialboxinnerseplengthx, inner ysep=\specialboxinnerseplengthy,
   draw=borderexvideo, top color=topexvideo, bottom color=bottomexvideo,
   text justified, very thick
  }{%
   draw=black, thick, rectangle, text width=\specialboxlength,
   inner xsep=\specialboxinnerseplengthx, inner ysep=\specialboxinnerseplengthy,
   draw, text justified, very thick
  }{\noindent #1}%
 \bigbreak%
}%


% \jmtVideo{youtube code}{jmt url suffix}{actual title}
%\newcommand{\jmtVideo}[3]{\genVideo{#1}{http://patrickjmt.com/#2/}{#3}}

%\newcommand{\khanVideo}[3]{\genVideo[?utm_campaign=embed]{#1}{https://www.khanacademy.org/video/#2}{#3}}


% \mfigure[graphicsoptions]{offset}{caption}{label}{file}
\newcommand{\mfigure}[5][]{%
	\mnote[#2]{%
		\centering\myincludegraphics[#1]{#5}%
		\captionof{figure}{#3}\label{#4}}%
}

% \mtable[offset=0]{caption}{label}{contents}
\newcommand{\mtable}[4][0ex]{%
	\mnote[#1]{\centering\small#4\captionof{figure}{#2}\label{#3}}
}

% \mnote[offset=0]{contents}
\newcommand{\mnote}[2][0pt]{%
	\marginpar{\vspace{#1}%
		\begin{minipage}[c]{\marginparwidth}
			\captionsetup{type=figure}%
			\testmargintop%
			\mbox{}%\makebox[0pt]{\tikz\draw(0,0)circle(1pt);}%
			\\[-\baselineskip]
			#2\ifhmode\unskip\fi
			\testmarginbottom%
			%\makebox[0pt]{\tikz\draw(0,0)circle(2pt);}
		\end{minipage}%
}}
% mnote was in apex_style.sty


%\newenvironment{lxfigure}{%
%	\iflatexml%
%		\begin{figure}[!h]%
%	\else%
%		\noindent\begin{minipage}[t]{\linewidth}\noindent%
%	\fi%
%	\captionsetup{type=figure}%
%}{%
%	\iflatexml\end{figure}\else\end{minipage}\fi%
%}

\newcommand{\tbox}[1]{\begin{tabular}{c}#1\end{tabular}} % a tall box
\newcommand*{\zbox}[1]{\makebox[0pt][c]{#1}} % a zero width box


%\newboolean{chapter_already_has_exercises}




\newboolean{isEarlyTrans}
\setboolean{isEarlyTrans}{false}

\newcommand{\prereqIntro}{The material in this section provides a basic review of and practice problems for pre-calculus skills essential to your success in Calculus. You should take time to review this section and work the suggested problems (checking your answers against those in the back of the book). Since this content is a pre-requisite for Calculus, reviewing and mastering these skills are considered your responsibility. This means that minimal, and in some cases no, class time will be devoted to this section. When you identify areas that you need help with we strongly urge you to seek assistance outside of class from your instructor or other student tutoring service.\bigskip}

\ifthenelse{\boolean{xetex}}%
	{%
	\sffamily
	%%\usepackage{fontspec}
	\usepackage{mathspec}
	\setallmainfonts[Mapping=tex-text]{Calibri}
	\setmainfont[Mapping=tex-text]{Calibri}
	\setsansfont[Mapping=tex-text]{Calibri}
	\setmathsfont(Greek){[cmmi10]}
	}
	{}

%\ifthenelse{\boolean{luatex}}%
%	{%
%	\sffamily
%	\usepackage{fontspec}
%	\usepackage{unicode-math}
%	%\usepackage{mathspec}
%	%\setallmainfonts[Mapping=tex-text]{Calibri}
%	\setmainfont{Calibri}
%	%\setsansfont[Mapping=tex-text]{Calibri}
%	\setmathfont[range=\mathup]{Calibri}
%	\setmathfont[range=\mathit]{Calibri Italic}
%	}
%	{}


\newboolean{calc1included}
\newboolean{calc2included}
\newboolean{calc3included}
\setboolean{calc1included}{false}
\setboolean{calc2included}{false}
\setboolean{calc3included}{false}

\newcommand{\apexindex}{%
 \cleardoublepage
 \phantomsection
 \iflatexml\chapter*{\indexname}\else\addcontentsline{toc}{chapter}{\indexname}\fi
 % set the ``see'' entries that shouldn't refer to a page number
 \ifthenelse{\boolean{calc2included}}{%
  \index{$"!$|see {factorial}} % 08 sequences
 }{}
 \ifthenelse{\boolean{calc3included}}{%
  \index{perpendicular|see{orthogonal}} % 10 dot product
  \index{nabla|see {del operator}} % 12 directional derivatives, 14 vector fields
  \index{$\nabla$|see {del operator}} % 12 directional derivatives, 14 vector fields
  \index{multiple integration|see{iterated integration}} % 13 iterated integrals
  \index{$\dfrac{\partial (x,y,z)}{\partial (u,v,w)}$|see {Jacobian}} % 13 Jacobian
 }{}
 \printindex
}


\usepackage[
	bookmarksnumbered,
	hidelinks,
	pdfstartview=FitH,
	linktoc=all,
	pdfdisplaydoctitle,
	bookmarksdepth=2,
]{hyperref}
\hypersetup{
	pdftitle={APEX Calculus LT},
	pdfauthor={UND Math Dept and Greg Hartman, VMI},
	unicode
}
\usepackage{bookmark}

% these need to come after hyperref
% if they come before and have newcommand, latexml overwrites them
\renewcommand{\sectionautorefname}{Section} % the default is lowercase
\renewcommand{\chapterautorefname}{Chapter} % the default is lowercase
\newcommand{\exampleEnvautorefname}{Example}
\newcommand{\autoeqref}[1]{\hyperref[#1]{\equationautorefname~(\ref*{#1})}}
% autoref doesn't use parentheses

% \apex has to be used after hyperref
% lxNavbar has to come after latexml
\begin{lxNavbar}
\lxRef{coverpagetitle}{\apex\ Calculus LT}\\
\lxContextTOC
\end{lxNavbar}

\lxIncludeJavascriptFile{https://ajax.googleapis.com/ajax/libs/jquery/1.12.2/jquery.min.js}
\lxIncludeJavascriptFile{LaTeXML-maybeMathJax.js}
\lxIncludeJavascriptFile{script.js}
\lxIncludeCssFile{style.css}
\lxIncludeCssFile{LaTeXML-marginpar.css}
\lxIncludeCssFile{LaTeXML-navbar-left.css}
