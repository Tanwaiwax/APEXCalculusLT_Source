\newboolean{abridgeConics}
\setboolean{abridgeConics}{true}

\newcommand{\monthYear}{%
\ifcase \month \or January\or February\or March\or April\or May\or June\or July\or August\or September\or October\or November\or December\fi \space \number \year}
%modified from \today. we could do
%\usepackage[en-US]{datetime2}
%\DTMlangsetup{showdayofmonth=false}
% so that \today is just month and year
%but LaTeXML doesn't have datetime2, so we need this anyway

\usepackage{multirow}
%\pgfplotsset{width=\marginparwidth+1pt,compat=1.3}
\usepackage[font=small]{caption}
%,justification=centering

%\usepackage{wrapfig}

\usepackage{booktabs}

\setcounter{secnumdepth}{1}
\setcounter{tocdepth}{1}

\newlength{\saveparindent}
\setlength{\saveparindent}{\parindent}

\makeatletter
\let\ps@oldplain=\ps@plain % save the plain pagestyle
\makeatother

\usepackage{fancyhdr}

\renewcommand{\chaptermark}[1]{\markboth{\chaptername\ \thechapter\ \ \ \ {#1}}{}}
\renewcommand{\sectionmark}[1]{\markright{\thesection\ \ \ \  #1}}
\renewcommand{\headrulewidth}{0pt}
\renewcommand{\footrulewidth}{0pt}


\fancypagestyle{prose}{%
 \fancyhf{}
 \fancyhead[LE]{\nouppercase{\leftmark}}%
 \fancyhead[RO]{\nouppercase{\rightmark}}%
 \fancyfoot[LE]{\begin{minipage}{\textwidth}%
  \noindent\hspace{\marginparwidth}\hspace{\marginparsep}\hspace{-4pt}%
  \makebox[0pt][l]{\rule{\textwidth}{.4pt}}%
  \vskip.2\baselineskip%
  \noindent\hspace{\marginparwidth}\hspace{\marginparsep}\hspace{-4pt}%
  Notes:%
  \vskip 1.5in\textbf{\thepage}%
 \end{minipage}}

 \fancyfoot[RO]{\begin{minipage}{\textwidth+\marginparwidth+\marginparsep}%
  \rule{\textwidth-\marginparwidth-\marginparsep}{.4pt}
  \vskip.2\baselineskip
  Notes:
  \vskip 1.5in
  \hfill\textbf{\thepage}
 \end{minipage}}
 \fancyhfoffset[LE,RO]{\marginparsep+\marginparwidth}
}
\fancypagestyle{exercise}{%
	\fancyhead{}% 
	\fancyhfoffset[LE,RO]{32pt}%
	\fancyfoot[LE,RO]{\textbf{\thepage}}
}


%
%  Defining what the chapter titles look like
%

%\newdimen\titleheight

\usepackage[newparttoc]{titlesec}
\usepackage{titletoc}

\titleformat{\chapter}[block]{\sloppy\Huge\scshape\textsc}{\thechapter:}{.4em}{}[\titlerule]
\titlespacing*{\chapter}{0pt}{0pt}{2ex}

\makeatletter
\titleformat{\part}[block]
   {\Huge\bfseries\filcenter}
   {}
   {0em}
   {\thispagestyle{empty}}
\titlecontents{part}[0pt]
	{\vspace{3ex}}
	{no numbered parts}
	{\large\bfseries}
	{\titlerule*[1em]{ }}
\makeatother

\let\oldmainmatter\mainmatter
\renewcommand{\mainmatter}{%
 \oldmainmatter
 \fancypagestyle{plain}{% override the default
  \fancyhf{}
  \fancyfoot[RO]{\begin{minipage}{\textwidth+\marginparwidth+\marginparsep}%
   \rule{\textwidth-\marginparwidth-\marginparsep}{.4pt}
   \vskip.2\baselineskip
   Notes:
   \vskip 1.5in
   \hfill\textbf{\thepage}
  \end{minipage}}
  \fancyhfoffset[RO]{\marginparsep+\marginparwidth}
 }
 \pagestyle{prose}
}

\let\oldappendix\appendix
\makeatletter
\renewcommand{\appendix}{%
 \let\ps@plain=\ps@oldplain% restore the pagestyle
 \cleardoublepage
 \oldappendix
 \setcounter{secnumdepth}{-1}
 \pagenumbering{arabic}
 \renewcommand{\thepage}{A.\arabic{page}}
 \renewcommand{\thechapter}{\arabic{chapter}}
 \part*{\appendixname}
% \pagestyle{oldplain}
% \part*{Appendices\protect\thispagestyle{empty}}
% \addcontentsline{toc}{part}{\appendixname}
% \iflatexml\else
% \pdfbookmark[part]{Appendices}{appendixbookmark}
% \fi
}


% an enumerate like environment that can be mixed into tabular, array, etc.
\newcounter{anywhereenumi}
\newenvironment{anywhereenum}{%
 \setcounter{anywhereenumi}{0}%
 \renewcommand{\item}[1][]{%
  \ifx.##1.%
  \refstepcounter{anywhereenumi}%
  \makebox[1em][r]{\arabic{anywhereenumi}.}~~%
  \else%
  \makebox[1em][r]{##1.}~~%
  \fi%
 }%
}{}

\newcommand{\ds}{\displaystyle}

\newcommand{\primeskip}{\hskip.75pt}

\newcommand{\fp}{\ensuremath{f\,'}}
\newcommand{\fpp}{\ensuremath{f\,''}}

\newcommand{\Fp}{\ensuremath{F\primeskip'}}
\newcommand{\Fpp}{\ensuremath{F\primeskip''}}

\newcommand{\yp}{\ensuremath{y\primeskip'}}
\newcommand{\gp}{\ensuremath{g\primeskip'}}

\newcommand*{\abs}[1]{\ensuremath{\left\lvert #1 \right\rvert}}
\newcommand*{\norm}[1]{\ensuremath{\left\lVert #1 \right\rVert}}
\newcommand*{\vnorm}[1]{\ensuremath{\norm{\vec #1}}}
\newcommand{\bracket}[1]{\left\langle #1\right\rangle}
\newcommand*{\dotp}[2]{\ensuremath{\vec #1 \cdot \vec #2}}
\newcommand*{\proj}[2]{\ensuremath{\text{proj}_{\,\vec #2}{\,\vec #1}}}
\newcommand*{\crossp}[2]{\ensuremath{\vec #1 \times \vec #2}}
\newcommand{\vecE}{\ensuremath{\vec E}}
\newcommand{\vecF}{\ensuremath{\vec F}}
\newcommand{\vecG}{\ensuremath{\vec G}}
\newcommand{\vecT}{\ensuremath{\vec T}}
\newcommand{\vece}{\ensuremath{\vec e}}
\newcommand{\vecf}{\ensuremath{\vec f}}
\newcommand{\vecg}{\ensuremath{\vec g}}
\newcommand{\veci}{\ensuremath{\vec\imath}}
\newcommand{\vecj}{\ensuremath{\vec\jmath}}
\newcommand{\veck}{\ensuremath{\vec k}}
\newcommand{\vecl}{\ensuremath{\vec l}}
\newcommand{\vecn}{\ensuremath{\vec n}}
\newcommand{\vecr}{\ensuremath{\vec r}}
\newcommand{\vecu}{\ensuremath{\vec u}}
\newcommand{\vecv}{\ensuremath{\vec v}}
\newcommand{\vecw}{\ensuremath{\vec w}}
\newcommand{\vecx}{\ensuremath{\vec x}}
\newcommand{\vecy}{\ensuremath{\vec y}}
\newcommand{\vrp}{\ensuremath{\vec r\hskip1.25pt '}}
\newcommand{\vsp}{\ensuremath{\vec s\primeskip '}}
\newcommand{\vrt}{\ensuremath{\vec r(t)}}
\newcommand{\vst}{\ensuremath{\vec s(t)}}
\newcommand{\vvt}{\ensuremath{\vec v(t)}}
\newcommand{\vat}{\ensuremath{\vec a(t)}}
\newcommand{\px}{\ensuremath{\partial x}}
\newcommand{\py}{\ensuremath{\partial y}}
\newcommand{\pz}{\ensuremath{\partial z}}
\newcommand{\pf}{\ensuremath{\partial f}}
\newcommand{\underlinespace}{\underline{\phantom{xxxxxx}}}

\newcommand{\zerooverzero}{\dfrac{\makebox[0pt]{\text{`` }0\text{ ''}}}0\ \ }


\DeclareMathOperator{\sech}{sech}
\DeclareMathOperator{\csch}{csch}
\DeclareMathOperator{\Div}{div}
\DeclareMathOperator{\grad}{grad}
\DeclareMathOperator{\curl}{curl}
\DeclareMathOperator{\divv}{div}

%\newcommand*{\sword}[1]{\textbf{#1}}

\newcommand{\LHequals}{\mathrel{\overset{\text{by LHR}}{=}}}

\newcommand{\surfaceS}{\ensuremath{\mathcal{S}}}


%\newspecialbox[notempty]{exvideo}{ignored}{240}
\AtBeginDocument{\makeStyles{exvideo}{240}}
\newcommand{\setexvideocolor}{%
 \definecolor{topexvideo}{Hsb}{240,.05,1}% %= hsl(#4,100,97.5)
 \definecolor{borderexvideo}{Hsb}{240,.3,1}% %= hsl(#4,90.5,68.4)
 \definecolor{bottomexvideo}{Hsb}{240,.15,1}% %= hsl(#4,81.9,83.4)
}
\newcommand{\setexvideobw}{%
 \definecolor{topexvideo}{Hsb}{0,0,1}% white
 \definecolor{bottomexvideo}{Hsb}{0,0,1}% white
 \definecolor{borderexvideo}{Hsb}{0,1,0}% black
}
\newcommand{\exvideo}[1]{%
% \bigbreak%
% \noindent%
%  \coloredbox{% same options newspecialbox
%   rectangle, text width = \specialboxlength,
%   inner xsep=\specialboxinnerseplengthx, inner ysep=\specialboxinnerseplengthy,
%   draw=borderexvideo, top color=topexvideo, bottom color=bottomexvideo,
%   text justified, very thick
%  }{%
%   draw=black, thick, rectangle, text width=\specialboxlength,
%   inner xsep=\specialboxinnerseplengthx, inner ysep=\specialboxinnerseplengthy,
%   draw, text justified, very thick
%  }{\noindent #1}%
% \bigbreak%
%}%
%\renewcommand{\exvideo}[1]{%
 \bigbreak%
 \tcbox[
   colframe=borderexvideo,
   interior style={top color=topexvideo, bottom color=bottomexvideo},
   sharp corners=all,notitle,width=\textwidth,enhanced,tcbox width=forced left
  ]{#1}%
 \bigbreak%
}


% \jmtVideo{youtube code}{jmt url suffix}{actual title}
%\newcommand{\jmtVideo}[3]{\genVideo{#1}{http://patrickjmt.com/#2/}{#3}}

%\newcommand{\khanVideo}[3]{\genVideo[?utm_campaign=embed]{#1}{https://www.khanacademy.org/video/#2}{#3}}


% \mfigure[graphicsoptions]{offset}{caption}{label}{file}
\newcommand{\mfigure}[5][]{%
	\mnote[#2]{%
		\centering\myincludegraphics[#1]{#5}%
		\captionof{figure}{#3}\label{#4}}%
}

% \mtable[offset=0]{caption}{label}{contents}
\newcommand{\mtable}[4][0ex]{%
	\mnote[#1]{\centering\small#4\captionof{figure}{#2}\label{#3}}
}

\usepackage[noadjust]{marginnote}

% \mnote[offset=0]{contents}
\newcommand*{\mnote}[2][0pt]{%
	\marginnote{%
		\begin{minipage}[c]{\marginparwidth}
			\captionsetup{type=figure}%
			\testmargintop%
			\mbox{}%\makebox[0pt]{\tikz\draw(0,0)circle(1pt);}%
			\\[-\baselineskip]
			#2\ifhmode\unskip\fi
			\testmarginbottom%
			%\makebox[0pt]{\tikz\draw(0,0)circle(2pt);}
		\end{minipage}}[#1]%
}
% mnote was in apex_style.sty


%\newenvironment{lxfigure}{%
%	\iflatexml%
%		\begin{figure}[!h]%
%	\else%
%		\noindent\begin{minipage}[t]{\linewidth}\noindent%
%	\fi%
%	\captionsetup{type=figure}%
%}{%
%	\iflatexml\end{figure}\else\end{minipage}\fi%
%}

\newcommand{\tbox}[1]{\begin{tabular}{c}#1\end{tabular}} % a tall box
\newcommand*{\zbox}[1]{\makebox[0pt][c]{#1}} % a zero width box






\newboolean{isEarlyTrans}
\setboolean{isEarlyTrans}{false}

\newcommand{\prereqIntro}{The material in this section provides a basic review of and practice problems for pre-calculus skills essential to your success in Calculus. You should take time to review this section and work the suggested problems (checking your answers against those in the back of the book). Since this content is a pre-requisite for Calculus, reviewing and mastering these skills are considered your responsibility. This means that minimal, and in some cases no, class time will be devoted to this section. When you identify areas that you need help with we strongly urge you to seek assistance outside of class from your instructor or other student tutoring service.\bigskip}

\ifthenelse{\boolean{xetex}}%
	{%
	\sffamily
%	\usepackage{fontspec}
%	\usepackage{unicode-math}
	\usepackage{mathspec}
	\setallmainfonts[Mapping=tex-text]{Calibri}
	\setmainfont[Mapping=tex-text]{Calibri}
	% setallmainfonts claims to setmainfont. but it doesn't?
%	\setmathsfont[Mapping=tex-text]{Calibri}
%	\setmathrm[Mapping=tex-text]{Calibri}
	\setsansfont[Mapping=tex-text]{Calibri}
	\setmathsfont(Greek){[cmmi10]}
	}
	{}

\ifthenelse{\boolean{luatex}}%
	{%
	\sffamily
	\usepackage{fontspec}
	\usepackage{unicode-math}
	%\usepackage{mathspec}
	%\setallmainfonts[Mapping=tex-text]{Calibri}
	\setmainfont{Calibri}
	%\setsansfont[Mapping=tex-text]{Calibri}
	\setmathfont[range=\mathup]{Calibri}
	\setmathfont[range=\mathit]{Calibri Italic}
	}
	{}

%\usepackage[american]{babel}
\usepackage{polyglossia}
\setdefaultlanguage[variant=usmax]{english}
\hyphenation{%
 a-mong
 an-ti-der-iv-a-tive
 an-ti-der-iv-a-tives
 app-roach-es
 bound-ed
 chang-es
 de-creases
 der-iv-a-tive
 e-qual-ly
 ex-am-ples
 in-dis-tin-guish-a-ble
 nu-mer-i-cal-ly
 par-a-bo-la
 proc-ess
 qua-dra-tic
 se-quence
 sketch-ing
 small-er
 smart-er
 Trig-o-no-me-tric
 trig-o-no-me-tric
 wheth-er
}

\usepackage{microtype}


\let\oldprintindex\printindex
\renewcommand{\printindex}{%
 \cleardoublepage
% \chapter{\indexname} % \printindex has its own heading
 \phantomsection
% \iflatexml\chapter*{\indexname}\fi
 \addcontentsline{toc}{chapter}{\indexname}
 \oldprintindex
}


\newboolean{bsc}
\newcommand{\forwhom}{\ifthenelse{\boolean{bsc}}{ for Bismarck State College}{}}


\usepackage[
	bookmarksnumbered,
	hidelinks,
	pdfstartview=FitH,
	linktoc=all,
	pdfdisplaydoctitle,
	bookmarksdepth=2,
]{hyperref}
\hypersetup{
	pdftitle={APEX Calculus LT},
	pdfauthor={UND Math Dept and Greg Hartman, VMI},
	unicode,
    pdflang=EN-US
}
\usepackage{bookmark}


% hyperref changes these
% if they come before and have newcommand, latexml overwrites them
\AtBeginDocument{
 \renewcommand{\chapterautorefname}{Chap\-ter} % the default is lowercase
 \renewcommand{\sectionautorefname}{Sec\-tion} % the default is lowercase
 \renewcommand{\figureautorefname}{Fig\-ure}
 \renewcommand{\appendixname}{Ap\-pen\-di\-ces}
}
\newcommand{\exampleEnvautorefname}{Ex\-am\-ple}
\newcommand{\autoeqref}[1]{\hyperref[#1]{\equationautorefname~(\ref*{#1})}}
% autoref doesn't use parentheses

% \apex has to be used after hyperref
% lxNavbar has to come after latexml
\begin{lxNavbar}
\lxRef{lxApexTOC}{Table of Contents}\\
\lxContextTOC
\end{lxNavbar}

\lxIncludeJavascriptFile{https://ajax.googleapis.com/ajax/libs/jquery/1.12.2/jquery.min.js}
\lxIncludeJavascriptFile{LaTeXML-maybeMathJax.js}
\lxIncludeJavascriptFile{script.js}
\lxIncludeCssFile{style.css}
\lxIncludeCssFile{LaTeXML-marginpar.css}
\lxIncludeCssFile{LaTeXML-navbar-left.css}

% set the defaults, just in case
\printincolor
\usetwoDgraphics
