\documentclass{article}

\usepackage{amsthm}
\usepackage[breakable,skins]{tcolorbox}
\usetikzlibrary{decorations.pathmorphing}

\pgfdeclaredecoration{complete zigzag}{initial}{
 \state{initial}[
  width=+0pt,
  next state=half up,
  persistent precomputation={
   \pgfmathsetmacro\matchinglength{
    \pgfdecoratedinputsegmentlength / int(\pgfdecoratedinputsegmentlength/\pgfdecorationsegmentlength)}
            \setlength{\pgfdecorationsegmentlength}{\matchinglength pt}
   }] {}
 \state{half up}[
  width=+.25\pgfdecorationsegmentlength,
  next state=big down]
  {\pgfpathlineto{\pgfqpoint{.25\pgfdecorationsegmentlength}{\pgfdecorationsegmentamplitude}}
  }
 \state{big down}[switch if less than=+.5\pgfdecorationsegmentlength to center finish,
                  width=+.5\pgfdecorationsegmentlength,
                  next state=big up]
  {
   \pgfpathlineto{\pgfqpoint{.5\pgfdecorationsegmentlength}{-\pgfdecorationsegmentamplitude}}
  }
 \state{big up}[switch if less than=+.5\pgfdecorationsegmentlength to center finish,
                width=+.5\pgfdecorationsegmentlength,
                next state=big down]
  {
   \pgfpathlineto{\pgfqpoint{.5\pgfdecorationsegmentlength}{\pgfdecorationsegmentamplitude}}
  }
 \state{center finish}[width=0pt, next state=final]{}
 \state{final}
  {
   \pgfpathlineto{\pgfpointdecoratedpathlast}
  }
}

\usepackage{lipsum}

\newtheorem{mytheorem}{Theorem}

\newcommand{\continuebottom}[1]{
   \path[font=\small\itshape] (frame.south) node (cont) {(continued)};
   \begin{scope}[decoration={zigzag,amplitude=0.5mm}]
    \path[fill=#1]
     decorate {([xshift=1.2pt]frame.south west) -- (cont.west)} --++ (0,0.5ex) -| cycle
     decorate {([xshift=-1.2pt]frame.south east) -- (cont.east)} --++ (0,0.5ex) -| cycle;
    \path[fill=white]
     decorate {([xshift=1.2pt]frame.south west) -- (cont.west)} --++ (0,-0.5ex) -| cycle
     decorate {([xshift=-1.2pt]frame.south east) -- (cont.east)} --++ (0,-0.5ex) -| cycle;
    \draw[thick,#1,decorate] ([xshift=1.2pt]frame.south west) -- (cont.west);
    \draw[thick,#1,decorate] ([xshift=-1.2pt]frame.south east) -- (cont.east);
   \end{scope} 
}
\newcommand{\continuetop}[1]{
   \path[font=\small\itshape] (frame.north) node (thm) {Theorem \themytheorem\ continued};
   \begin{scope}[decoration={zigzag,amplitude=0.5mm}]
    \path[fill=#1]
     decorate {([xshift=1.2pt]frame.north west) -- (thm.west)} --++ (0,-0.5ex) -| cycle
     decorate {([xshift=-1.2pt]frame.north east) -- (thm.east)} --++ (0,-0.5ex) -| cycle;
    \path[fill=white]
     decorate {([xshift=1.2pt]frame.north west) -- (thm.west)} --++ (0,0.5ex) -| cycle
     decorate {([xshift=-1.2pt]frame.north east) -- (thm.east)} --++ (0,0.5ex) -| cycle;
    \draw[thick,#1,decorate] ([xshift=1.2pt]frame.north west) -- (thm.west);
    \draw[thick,#1,decorate] ([xshift=-1.2pt]frame.north east) -- (thm.east);
   \end{scope} 
}

\tcolorboxenvironment{mytheorem}{
  enhanced,
  colframe=blue,
  interior style={top color=blue!20, bottom color=blue!10},
  breakable=true,
  overlay first={\continuebottom{blue!10}},
  overlay middle={\continuebottom{blue!10}\continuetop{blue!20}},
  overlay last={\continuetop{blue!20}},
%  enlargepage flexible=3\baselineskip,
}
  
\begin{document}

\lipsum[1-3]

\begin{mytheorem}[My Theorem]
 \lipsum[4-5]
\end{mytheorem}

\end{document}

I have a tcolorboxenvironment around an AMS theorem that I would like to be breakable.  The problem is that I would like a more noticeable indicator for the break than the defaults seem to provide.  The best that I've found is to set the option `title after break`, but my other tcolorboxenvironments don't have the title, so this looks strange.

I've tried setting first's borderline south and last's borderline north, but neither has an effect.  I've also tried using first's after upper and last's before lower, also without effect (this outcome would be my preference).  I think it should be possible to achieve the borderline effect via the skin engines, but I'm having trouble understanding that much of the documentation.

How can I better indicate that a tcolorbox is continuing onto the next page?


Thanks for this; I'll have to study complete zigzag for awhile.  I have two questions so far.  (1) It seems that the big challenge for complete zigzags and complete sines is to have it end at just the right time.  But since we can have an empty horizontal break leading into the added text, does that simplify the calculations of "complete zigzag"?  (It does seem that the `\path[fill]` is what makes the fill stop at the decoration, so ``complete zigzag'' isn't needed for that effect.) (2) I newcommanded \continuebottom and \continuetop so that I could have `overlay first={\continuebottom}, overlay middle={\continuebottom\continuetop}, overlay last={\continuetop},` without retyping everything.  Is there any downside to doing so?
