\documentclass[10pt]{article}


%%
% HERE THERE BE DRAGONS
%%

\usepackage{ifthen}

\usepackage{lipsum}
\usepackage{pgfplots}
\pgfplotsset{colormap={coloronemap}{rgb=(.4,.4,1); rgb=(.8,.8,1)}}
\pgfplotsset{colormap={colortwomap}{rgb=(1,.4,.4); rgb=(1,.8,.8)}}
\usepackage{eso-pic,calc}
\usepackage[font=small]{caption}
\usepgfplotslibrary{external}
\usetikzlibrary{calc}
\usetikzlibrary{shadings}

\usepackage[h]{esvect}

\pgfplotsset{compat=1.8}

\usepackage[paperwidth=3in,textheight=10in,marginparwidth=150pt]{geometry}

\usepackage{amsmath}




\ifthenelse{\boolean{xetex}}%
	{\sffamily
	%%\usepackage{fontspec}
	\usepackage{mathspec}
	\setallmainfonts[Mapping=tex-text]{Calibri}
	\setmainfont[Mapping=tex-text]{Calibri}
	\setsansfont[Mapping=tex-text]{Calibri}
	\setmathsfont(Greek){[cmmi10]}}
	{Please compile with XeLaTeX}

\newboolean{colorprint}
\setboolean{colorprint}{true}
%\setboolean{colorprint}{false}

\ifthenelse{\boolean{colorprint}}{%
\newcommand{\colorone}{blue}
\newcommand{\colortwo}{red}
\newcommand{\coloronefill}{blue!15!white}
\newcommand{\colortwofill}{red!15!white}
\newcommand{\colormapone}{rgb=(.4,.4,1); rgb=(.8,.8,1)}
\newcommand{\colormaptwo}{rgb=(1,.4,.4); rgb=(1,.8,.8)}
\newcommand{\colormapplaneone}{rgb=(.7,.7,1); rgb=(.9,.9,1)}
\definecolor{colormaponebottom}{rgb}{.4,.4,1}
\definecolor{colormaponetop}{rgb}{.8,.8,1}
\definecolor{colormaptwobottom}{rgb}{1,.4,.4}
\definecolor{colormaptwotop}{rgb}{1,.8,.8}
}% ends color
{% not color
\newcommand{\colorone}{black}
\newcommand{\colortwo}{black!50!white}
\newcommand{\coloronefill}{black!15!white}
\newcommand{\colortwofill}{black!05!white}
\newcommand{\colormapone}{rgb=(.4,.4,.4); rgb=(.7,.7,.7)}
\newcommand{\colormaptwo}{rgb=(.6,.6,.6); rgb=(.9,.9,.9)}
\newcommand{\colormapplaneone}{rgb=(.8,.8,.8); rgb=(.95,.95,.95)}
\definecolor{colormaponebottom}{rgb}{.4,.4,.4}
\definecolor{colormaponetop}{rgb}{.7,.7,.7}
\definecolor{colormaptwobottom}{rgb}{.6,.6,.6}
\definecolor{colormaptwotop}{rgb}{.9,.9,.9}
}%
\newcommand{\la}{\left\langle}
\newcommand{\ra}{\right\rangle}
\newcommand{\dotp}[2]{\ensuremath{\vec #1 \cdot \vec #2}}
\newcommand{\proj}[2]{\ensuremath{\text{proj}_{\,\vec #2}{\,\vec #1}}}

\newcommand{\fp}{\ensuremath{f\,'}}

\DeclareMathOperator{\sech}{sech}
\DeclareMathOperator{\csch}{csch}

\newcommand{\threedlines}[4][]{\draw [dashed,#1] (axis cs: #2,#3,#4) -- (axis cs: #2,#3,0) -- (axis cs: #2,0,0)  (axis cs: #2,#3,0)--(axis cs:0,#3,0);}

\newcommand{\mydraw}{\draw (axis cs:0,0,0) -- (axis cs:1,1,0);}
\newcommand{\ds}{\displaystyle}


%% no more dragons.  type away %%

% prefer color one for the main graph, and color two for secondary lines

\begin{document}

\begin{tikzpicture}
\begin{axis}[width=\marginparwidth +25pt,tick label style={font=\scriptsize },
			minor x tick num=1,axis y line=middle,axis x line=middle,ytick={.8,.9,1},
			ymin=.75,ymax=1.05,
			%extra y ticks={.1,.3,...,.9},extra y tick labels={},
			xmin=-1.1,xmax=1.2,name=myplot]
\addplot [{\colorone},smooth,thick,domain=.01:1] {sin(deg(x))/x};
%\draw [{\colorone},thick,domain=.01:1,variable=\x] plot ({\x},{sin(\x)/\x});
%\addplot [{\colorone },thick,smooth] coordinates {(-1.,0.841471) (-0.9,0.870363) (-0.8,0.896695) (-0.7,0.920311) (-0.6,0.941071) (-0.5,0.958851) (-0.4,0.973546) (-0.3,0.985067) (-0.2,0.993347) (-0.1,0.998334) (0,1.) (0.1,0.998334) (0.2,0.993347) (0.3,0.985067) (0.4,0.973546) (0.5,0.958851) (0.6,0.941071) (0.7,0.920311) (0.8,0.896695) (0.9,0.870363) (1.,0.841471) };
\draw (axis cs:0,1) circle [radius=1.5pt];
\end{axis}
%\begin{axis}[width=\marginparwidth, tick label style={font=\scriptsize},
%			minor x tick num=1, axis y line=middle, axis x line=middle, ymin=-3,
%			ymax=2, xmin=-2, xmax=7, name=myplot]
%\addplot [{\colortwo},thick,smooth] coordinates {(-2,.2)(-1,.5)(0,1)(1,1.5)(2,0)(4,-1)(6,0)(7,2)};
%\addplot [{\colorone},domain=-2:0,thick] {e^x};
%\draw [{\colorone},thick] (axis cs:0,1) parabola [bend at end] (axis cs:1,1.5);
%\draw [{\colorone},thick] (axis cs:1,1.5) parabola (axis cs:2,0);
%\draw [{\colorone},thick] (axis cs:2,0) parabola [bend at end] (axis cs:4,-3);
%\draw [{\colorone},thick] (axis cs:4,-3) parabola (axis cs:7,3.75);
%%\filldraw [black] (axis cs:2,86) circle (1pt);
%%\filldraw [black] (axis cs:3,6) circle (1pt);
%\end{axis}

\node [right] at (myplot.right of origin) {\scriptsize $x$};
\node [above] at (myplot.above origin) {\scriptsize $y$};
\end{tikzpicture}

\end{document}
