\documentclass[10pt]{article}


%%
% HERE THERE BE DRAGONS
%%

\usepackage{ifthen}

\usepackage{lipsum}
\usepackage{pgfplots}
\pgfplotsset{colormap={coloronemap}{rgb=(.4,.4,1); rgb=(.8,.8,1)}}
\pgfplotsset{colormap={colortwomap}{rgb=(1,.4,.4); rgb=(1,.8,.8)}}
%\pgfplotsset{compat=1.3}
\usepackage{eso-pic,calc}
\usepackage[font=small]{caption}
\usepgfplotslibrary{external}
\usetikzlibrary{calc}
\usetikzlibrary{shadings}

\usepackage[h]{esvect}

\pgfplotsset{compat=1.8}

\usepackage[paperheight=11in,paperwidth=6in,%inner=1in,includeheadfoot,
textheight=10in,%textwidth=345pt,
marginparwidth=150pt]{geometry}
%% end detour
%%

\usepackage{amsmath}




\ifthenelse{\boolean{xetex}}%
	{\sffamily
	%%\usepackage{fontspec}
	\usepackage{mathspec}
	\setallmainfonts[Mapping=tex-text]{Calibri}
	\setmainfont[Mapping=tex-text]{Calibri}
	\setsansfont[Mapping=tex-text]{Calibri}
	\setmathsfont(Greek){[cmmi10]}}
	{Please compile with XeLaTeX}

\newboolean{colorprint}
\setboolean{colorprint}{true}
%\setboolean{colorprint}{false}

\ifthenelse{\boolean{colorprint}}{%
\newcommand{\colorone}{blue}
\newcommand{\colortwo}{red}
\newcommand{\coloronefill}{blue!15!white}
\newcommand{\colortwofill}{red!15!white}
\newcommand{\colormapone}{rgb=(.4,.4,1); rgb=(.8,.8,1)}
\newcommand{\colormaptwo}{rgb=(1,.4,.4); rgb=(1,.8,.8)}
\newcommand{\colormapplaneone}{rgb=(.7,.7,1); rgb=(.9,.9,1)}
\definecolor{colormaponebottom}{rgb}{.4,.4,1}
\definecolor{colormaponetop}{rgb}{.8,.8,1}
\definecolor{colormaptwobottom}{rgb}{1,.4,.4}
\definecolor{colormaptwotop}{rgb}{1,.8,.8}
}% ends color
{% not color
\newcommand{\colorone}{black}
\newcommand{\colortwo}{black!50!white}
\newcommand{\coloronefill}{black!15!white}
\newcommand{\colortwofill}{black!05!white}
\newcommand{\colormapone}{rgb=(.4,.4,.4); rgb=(.7,.7,.7)}
\newcommand{\colormaptwo}{rgb=(.6,.6,.6); rgb=(.9,.9,.9)}
\newcommand{\colormapplaneone}{rgb=(.8,.8,.8); rgb=(.95,.95,.95)}
\definecolor{colormaponebottom}{rgb}{.4,.4,.4}
\definecolor{colormaponetop}{rgb}{.7,.7,.7}
\definecolor{colormaptwobottom}{rgb}{.6,.6,.6}
\definecolor{colormaptwotop}{rgb}{.9,.9,.9}
}%
\newcommand{\la}{\left\langle}
\newcommand{\ra}{\right\rangle}
\newcommand{\dotp}[2]{\ensuremath{\vec #1 \cdot \vec #2}}
\newcommand{\proj}[2]{\ensuremath{\text{proj}_{\,\vec #2}{\,\vec #1}}}

\newcommand{\fp}{\ensuremath{f\,'}}

\DeclareMathOperator{\sech}{sech}
\DeclareMathOperator{\csch}{csch}

\newcommand{\threedlines}[4][]{\draw [dashed,#1] (axis cs: #2,#3,#4) -- (axis cs: #2,#3,0) -- (axis cs: #2,0,0)  (axis cs: #2,#3,0)--(axis cs:0,#3,0);}

\newcommand{\mydraw}{\draw (axis cs:0,0,0) -- (axis cs:1,1,0);}
\newcommand{\ds}{\displaystyle}
\usepackage{multicol}

%% no more dragons.  type away %%

% prefer color one for the main graph, and color two for secondary lines

\begin{document}

This new example will go after 1.4.4 and I also changed the table that will now go directly after 1.4.5

%     \example{ex_absvalue}{Evaluating limits of an absolute value function}
{
Let $\ds f(x) =\frac{|x-1|}{x-1}.$ Evaluate the following. 

		\noindent\begin{minipage}[t]{.5\textwidth}
		\begin{enumerate}
		\item		$\ds \lim_{x\to 1^-} f(x)$
		\item		$\ds \lim_{x\to 1^+} f(x)$
		\end{enumerate}
		\end{minipage}
				\noindent\begin{minipage}[t]{.5\textwidth}
		\begin{enumerate}\addtocounter{enumi}{2}
		\item		$\ds \lim_{x\to 1} f(x)$
		\item		$f(1)$
		\end{enumerate}
		\end{minipage}
		
}
{ We begin by rewriting $|x-1|$ as a piecewise function.$$ |x-1|=\left\{\begin{array}{cc} x-1 & x\geq 1 \\ -(x-1) & x\leq 1\end{array}\right.$$
\begin{enumerate}
\item		$\ds \lim_{x\to 1^-} f(x)=\lim_{x\to 1^-}\frac{-(x-1)}{x-1}=\lim_{x\to 1^-}-1=-1$
\item		$\ds \lim_{x\to 1^+} f(x)=\lim_{x\to 1^+}\frac{x-1}{x-1}=\lim_{x\to 1^+}1=1$
\item 		$\ds \lim_{x\to 1} f(x)$ does not exist because the left and right hand limits are not equal.
\item $f(1)$ is undefined.
\end{enumerate}
}
%\\

In Examples \ref{ex_onesidea} -- \ref{ex_absvalue} we were asked to find both $\ds \lim_{x\to 1}f(x)$ and $f(1)$. Consider the following table:
\begin{center}
\begin{tabular}{ccc} & $\ds \lim_{x\to 1}f(x)$ & $f(1)$ \vspace{2pt}\\ \hline
Example \ref{ex_onesidea} & does not exist & 1 \\
Example \ref{ex_onesideb} & 1 & not defined \\
Example \ref{ex_onesidec} & 0 & 1 \\
Example \ref{ex_onesided} & 1 & 1 \\
Example \ref{ex_absvalue} & does not exist & not defined
\end{tabular}
\end{center}


Add to exercises 12-20:

%%%% I copied the code below from GITHUB. Something isn't working with the enumerate....

%%%%% PLEASE check my solutions!!! %%%

{$\ds \lim_{x\to 4} \frac{|4-x|}{x-4}$

\noindent\begin{minipage}[t]{.5\linewidth}
\begin{enumerate}
\item		$\ds \lim_{x\to 4^-} f(x)$
\item		$\ds \lim_{x\to 4^+} f(x)$
\end{enumerate}
\end{minipage}
\noindent\begin{minipage}[t]{.5\linewidth}
\begin{enumerate}\addtocounter{enumii}{2}
\item		$\ds \lim_{x\to 4} f(x)$
\item		$f(4)$\end{enumerate}
\end{minipage}
}
%%%% SOLUTION
{\begin{enumerate}
\item		$-1$
\item		$1$
\item		Does not exist.
\item		undefined
\end{enumerate}
}

{$\ds \lim_{x\to -2} \frac{x+2}{|x+2|}$

\noindent\begin{minipage}[t]{.5\linewidth}
\begin{enumerate}
\item		$\ds \lim_{x\to -2^-} f(x)$
\item		$\ds \lim_{x\to -2^+} f(x)$
\end{enumerate}
\end{minipage}
\noindent\begin{minipage}[t]{.5\linewidth}
\begin{enumerate}\addtocounter{enumii}{2}
\item		$\ds \lim_{x\to -2} f(x)$
\item		$f(-2)$\end{enumerate}
\end{minipage}
}
%%%% SOLUTION
{\begin{enumerate}
\item		$-1$
\item		$1$
\item		Does not exist.
\item		undefined
\end{enumerate}
}


Eliminate text after Key Idea 2 and replace it with the following example.

Example: Find the horizontal asymptotes of $\ds \frac{x}{\sqrt{x^2+1}}$.

Solution:  We must consider the limits as $x\to \pm \infty$. When $x$ is very large, $x^2+1\approx x^2$ and thus $\sqrt{x^2+1}\approx \sqrt{x^2}= |x|$
\begin{align*}
\lim_{x\to\infty}\frac{x}{\sqrt{x^2+1}}&=\lim_{x\to\infty}\frac{x/x}{\sqrt{x^2/{x^2}+1/{x^2}}}\\
&=\lim_{x\to\infty}\frac{1}{\sqrt{1+1/{x^2}}}\\
&=1
\end{align*}
Therefore, $y=1$ is a horizontal asymptote.
Similarly, 
\begin{align*}
\lim_{x\to-\infty}\frac{x}{\sqrt{x^2+1}}&=\lim_{x\to-\infty}\frac{x/{-x}}{\sqrt{x^2/{x^2}+1/{x^2}}}\\
&=\lim_{x\to-\infty}\frac{-1}{\sqrt{1+1/{x^2}}}\\
&=-1
\end{align*}Therefore, $y=-1$ is also a horizontal asymptote.


Add to the exercises 19-24:

$\ds f(x)=\frac{2x^4+3}{\sqrt{x^8+9}}$		Solution: $y=2$ $y=-2$

$\ds f(x)=\frac{3x^3+4}{\sqrt{x^6+3}}$		Solution: $y=3$ $y=-3$



\end{document}