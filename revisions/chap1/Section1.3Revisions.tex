\documentclass[10pt]{article}


\usepackage{ifthen}

\usepackage{lipsum}
\usepackage{pgfplots}

\usepackage{eso-pic,calc}
\usepackage[font=small]{caption}
\usepgfplotslibrary{external}
\usetikzlibrary{calc}
\usetikzlibrary{shadings}
\usepackage{tikz}
\usetikzlibrary{positioning,chains,fit,shapes,calc,arrows,patterns}
\usepackage{tkz-graph}
\usetikzlibrary{arrows, petri, topaths}
\usepackage{tkz-berge}
\usepackage[all]{xy}
\usepackage{textcomp}
\usepackage[h]{esvect}
\usepackage[normalem]{ulem}

\pgfplotsset{compat=1.8}


\usepackage{amsmath}

\newcommand{\ds}{\displaystyle}


\begin{document}

Theorem 1  p 18 should be changed to only include 1.-8. For number 8, add the comment:  

When $n$ is even, $L$ must be greater than 0. If $n$ is odd, then the statement is true for all $L$.

Text to be inserted below Theorem 1:

We will now prove the Sum Property using the formal definition of a limit from the previous section. We know that $\ds \lim_{x\to c}f(x) = L$ and  $\lim_{x\to c} g(x) = K$. We hope to show that $\ds \lim_{x\to c} (f(x)+g(x)=L+K$.

Proof: We must show that given any $\epsilon>0$, we can find a $\delta>0$ such that $$\text{\ if\ }  0<|x-c|<\delta, \text{\ then\ } |f(x)+g(x)-(L+K)|<\epsilon.$$  We know $\ds \lim_{x\to c}f(x) = L$. So for any $\epsilon_1 >0$, we can find $\delta_1>0$ such that if $0<|x-c|<\delta_1$, then $|f(x)-L|<\epsilon_1$. Similarly we know $\lim_{x\to c} g(x) = K$ so for any $\epsilon_2>0$, we can find $\delta_2>0$ such that if  $0<|x-c|<\delta_2$, then $|g(x)-K|<\epsilon_2$. We will let both $\epsilon_1$ and $\epsilon_2$ be $\frac{\epsilon}{2}$. Now, we have a $\delta_1>0$ and a $\delta_2>0$ such that: \small $$\text{\ if\ }  0<|x-c|<\delta_1, \text{\ then\ } |f(x)-L|<\frac{\epsilon}{2}$$
\begin{center} and \end{center} 
$$\text{\ if\ }  0<|x-c|<\delta_2, \text{\ then\ } |g(x)-K|<\frac{\epsilon}{2}$$
\normalsize
We will choose $\delta=min(\delta_1,\delta_2)>0$. If $0<|x-c|<\delta$, then $|f(x)-L|<\frac{\epsilon}{2}$ and  $|g(x)-K|<\frac{\epsilon}{2}$.  Add the two inequalites together.  $$|f(x)-L|+|g(x)-K|<\frac{\epsilon}{2}+\frac{\epsilon}{2}=\epsilon$$
We will now use the triangle inequality: $|A+B|\leq |A|+|B|$.
$$|f(x)-L+g(x)-K|\leq |f(x)-L|+|g(x)-K|<\epsilon$$
Thus $|(f(x)+g(x))-(L+K)|<\epsilon$, which is what we were trying to show.

The other Basic Limit Properties can be proven in a similar way and are left for the reader.

Make a new Theorem 2 called Limits of Compositions. It follows.

Let $b, c, L$ and $K$ be real numbers, let $n$ be a positive integer, and let $f$ and $g$ be functions with the following limits: 
$$\lim_{x\to c}f(x) = L \text{\ and\ } \lim_{x\to L} g(x) = K$$. Then $\ds \lim_{x\to c} g(f(x))=K.$


Change the title of Theorem 3 on p 20 to Limits of Basic Functions

Insert margin graph of Squeeze theorem on p 22. (I will send this in a different file)

p 25 The solution to example 14 p 25 should be changed to read as follows: 


{We begin by attempting to apply Theorem \ref{thm:lim_continuous} and substituting 1 for $x$ in the quotient. This gives:
		$$\lim_{x\to 1}\frac{x^2-1}{x-1} = \frac{1^2-1}{1-1} = \raisebox{8pt}{\text{``\ }}\frac{0}{0}\raisebox{8pt}{\text{\ ''}},$$ an indeterminate form. We cannot apply the theorem.

%%  Figure is commented out so I could check the compiling %%%%%%%%%%%%%%

%\mfigure{.6}{Graphing $f$ in Example \ref{ex_limit_onept} to understand a limit.}{fig:limitxplus1}{figures/fig_LimitXplus1}
		
		By graphing the function, as in Figure \ref{fig:limitxplus1}, we see that the function seems to be linear, implying that the limit should be easy to evaluate. Recognize that the numerator of our quotient can be factored:
		$$\text{Let \ } f(x)=\frac{x^2-1}{x-1} = \frac{(x-1)(x+1)}{x-1}.$$
		The function is not defined when $x=1$, but for all other $x$, $$\frac{x^2-1}{x-1} = \frac{(x-1)(x+1)}{x-1} = \frac{\hbox{\sout{$(x-1)$}}(x+1)}{\hbox{\sout{$x-1$}}}= x+1.$$
		Clearly $\ds \lim_{x\to 1}x+1 = 2$. Recall that when considering limits, we are not concerned with the value of the function at 1, only the value the function approaches as $x$ approaches 1. Since $(x^2-1)/(x-1)$ and $x+1$ are the same at all points except $x=1$, they both approach the same value as $x$ approaches 1. Therefore we can conclude that $$\lim_{x\to 1}\frac{x^2-1}{x-1}=\lim_{x\to 1}\frac{(x-1)(x+1)}{x-1}=\lim_{x\to 1} x+1=2.$$


Change Theorem 6 conclusion to read:  $$\lim_{x\to c} f(x)=\lim_{x\to c} g(x)=L$$

The text after Theorem 6 up to Example 15 should read:  

The Fundamental Theorem of Algebra tells us that when dealing with a rational function of the form $g(x)/f(x)$ and directly evaluating the limit $\ds \lim_{x\to c} \frac{g(x)}{f(x)}$ returns ``0/0'', 
then $(x-c)$ is a factor of both $g(x)$ and $f(x)$. One can then use algebra to factor this term out, divide, then apply Theorem \ref{thm:limit_allbut1}. Some useful algebraic techniques to rewrite functions that return an indeterminate form when evaluating a limit are given here.  
\begin{enumerate}
\item Factoring and dividing out common factors.
\item Rationalizing the numerator or denominator.
\item Simplifying the expression.
\item Finding a common denominator.
\end{enumerate}
We will demonstrate some of these thechniques in the following examples.\\

In the solution to example 15 the word CANCEL needs to be changed to divide.

Add in the following example right after example 15. 

Evaluate $\ds \lim_{x\to 2} \frac{\sqrt{x^2+4}-2}{x-2}$
Solution:
We begin by applying  \ref{thm:lim_continuous} and substituting 2 for $x$. This returns the familiar indeterminate form of ``0/0''.  We see the radical in the numerator so we will rationalize the numerator. Using Theorem \ref{thm:limit_allbut1} we find that
\begin{align*}
\lim_{x\to 0} \frac{\sqrt{x+4}-2}{x}&=\lim_{x\to 0} \frac{\sqrt{x+4}-2}{x}\cdot \frac{\sqrt{x+4}+2}{\sqrt{x+4}+2}& \\
&=\lim_{x\to 0} \frac{(x+4)-4}{x(\sqrt{x+4}+2)}& \text{ Note that we did not distribute the denominator.}\\
&=\lim_{x\to 0} \frac{x}{x(\sqrt{x+4}+2)} & \text{Simplify the numerator.}\\
&=\lim_{x\to 0} \frac{1}{\sqrt{x+4}+2} & \text{Divide out } x.\\
&=\frac{1}{\sqrt{4}+2}=\frac{1}{4}&.
\end{align*}


Additions to exercises:

Add to group 18-32:

$\ds \lim_{t\to 9} \frac{\sqrt{t}-3}{t-9}$  Solution: $\frac{1}{3}$

$\ds \lim_{x\to 0} \frac{\sqrt{x^2+4}-2}{x^2}$  Solution: $\frac{1}{4}$

$\ds \lim_{t\to 3} \frac{\frac{1}{t}-\frac{1}{3}}{t-3}$ Solution: $-\frac{1}{9}$

$\ds \lim_{x\to 0} \frac{1}{x}-\frac{1}{x^2+x}$  Solution: $1$

$\ds \lim_{t\to 0} \frac{(t-4)^2-16}{t}$ Solution: $-8$

Add to directions of 33-36:  Hint: $-1\leq \sin x\leq 1$ and $-1\leq \cos x\leq 1$.
Add to this group of problems:  $\ds \lim_{x\to 0} x^2\cos \bigg(\frac{1}{x}\bigg)$ Solution: $0$

Change directions of 37-40 to read:  challenge your understanding of limits that can be evaluated using the knowledge gained in this section.
Add to this group of problems: $\ds \lim_{x\to 0} \frac {\tan 4x}{\tan 3x}$ Solution: $\frac{4}{3}$
$\ds \lim_{x\to 0} \frac {\tan 5x}{\sin 7x}$ Solution: $\frac{5}{7}$

Add exercise at end:

Verify $\ds\lim_{x\to 0} \frac{\cos x -1}{x} =0$ \quad Hint: Multiply by $\ds \frac{\cos x +1}{\cos x +1}$
\end{document}