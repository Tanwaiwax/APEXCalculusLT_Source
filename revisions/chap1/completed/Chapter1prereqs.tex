\documentclass[11pt]{report}
\usepackage[letterpaper, total={6.5in, 10in}]{geometry}
%\usepackage{fancyhdr}
%\pagestyle{fancy}
\usepackage{amsmath, amsthm, mathpazo, epic, eepic, color, array}
\usepackage{amssymb}
%\usepackage{graphicx}
\usepackage{cancel}
\usepackage{pgfplots}
\usepackage{multicol}
\pgfplotsset{compat=1.13}
\usepackage{etoolbox}
\makeatletter
\patchcmd{\chapter}{\if@openright\cleardoublepage\else\clearpage\fi}{}{}{}
\makeatother
\usepackage{hyperref}

\usepackage{enumerate}
\usepackage{enumitem}

\usepackage{tikz}
\usetikzlibrary{positioning,chains,fit,shapes,calc,arrows,patterns}
\usepackage{tkz-graph}
\usetikzlibrary{arrows, petri, topaths}
\usepackage{tkz-berge}
\usepackage[all]{xy}
\usepackage{textcomp}

\newboolean{colorprint}
\setboolean{colorprint}{true}
%\setboolean{colorprint}{false}

\ifthenelse{\boolean{colorprint}}{%
\newcommand{\colorone}{blue}
\newcommand{\colortwo}{red}
\newcommand{\coloronefill}{blue!15!white}
\newcommand{\colortwofill}{red!15!white}
\newcommand{\colormapone}{rgb=(.4,.4,1); rgb=(.8,.8,1)}
\newcommand{\colormaptwo}{rgb=(1,.4,.4); rgb=(1,.8,.8)}
\newcommand{\colormapplaneone}{rgb=(.7,.7,1); rgb=(.9,.9,1)}
\definecolor{colormaponebottom}{rgb}{.4,.4,1}
\definecolor{colormaponetop}{rgb}{.8,.8,1}
\definecolor{colormaptwobottom}{rgb}{1,.4,.4}
\definecolor{colormaptwotop}{rgb}{1,.8,.8}
}% ends color
{% not color
\newcommand{\colorone}{black}
\newcommand{\colortwo}{black!50!white}
\newcommand{\coloronefill}{black!15!white}
\newcommand{\colortwofill}{black!05!white}
\newcommand{\colormapone}{rgb=(.4,.4,.4); rgb=(.7,.7,.7)}
\newcommand{\colormaptwo}{rgb=(.6,.6,.6); rgb=(.9,.9,.9)}
\newcommand{\colormapplaneone}{rgb=(.8,.8,.8); rgb=(.95,.95,.95)}
\definecolor{colormaponebottom}{rgb}{.4,.4,.4}
\definecolor{colormaponetop}{rgb}{.7,.7,.7}
\definecolor{colormaptwobottom}{rgb}{.6,.6,.6}
\definecolor{colormaptwotop}{rgb}{.9,.9,.9}
}%

\newlength\tindent
\setlength{\tindent}{\parindent}
\setlength{\parindent}{0pt}
\renewcommand{\indent}{\hspace*{\tindent}}



\pgfplotsset{my style/.append style={axis x line=middle, axis y line=
middle, xlabel={$x$}, ylabel={$y$}, axis equal }}

\pgfplotsset{compat=1.13}
\newcommand{\ds}{\displaystyle}



\begin{document}



\section{Prerequisite Material}

A \textbf{function} $f$ is a rule that assigns each element $x$ from a set (called the domain) to exactly one element, called $f(x)$, in another set. We use the convention that the \textbf{domain} is the set of all real numbers for which the rule makes sense and defines a real number. All possible values of $f(x)$ are called the \textbf{range} of $f$. We use four ways to represent a function.
\begin{enumerate}
\item By a graph
\item By an explicit formula         %%%%%%%%%%%  Tim, can you center this list, please?? %%%%%%%%%%%
\item By a table of values
\item By a verbal description
\end{enumerate}

Throughout the book we will use several representations of any given function to help give us a better understanding of the problem. The graphs below contain most of the base functions we can use to build other functions using transformations.

%\begin{multicols}{3}
\begin{tabular}{c c c}
\begin{tikzpicture}[scale = .65]
\begin{axis}[axis y line=middle, axis x line=middle, xmin=-2.2, xmax=2.2, ymin=-.5, ymax=3.2,name=myplot]
\addplot[{\colorone}, domain=-2:2, thick, smooth]{x*x};
\end{axis}
\node [right] at (myplot.right of origin) {\scriptsize $x$};
\node [above] at (myplot.above origin) {\scriptsize $y$};
%\node[\colorone] at (2.2,0 ){\scriptsize $f(x)=x^2$};
\end{tikzpicture} &

\begin{tikzpicture}[scale = .65]
\begin{axis}[axis y line=middle, axis x line=middle, xmin=-.5, xmax=3.2, ymin=-.5, ymax=2.2,name=myplot, ytick={0, 1,2}, xtick={0,1,2,3}]
\addplot[{\colorone}, domain=-0:3, thick, smooth,samples=100]{sqrt x};
\end{axis}
\node [right] at (myplot.right of origin) {\scriptsize $x$};
\node [above] at (myplot.above origin) {\scriptsize $y$};
\end{tikzpicture}&


\begin{tikzpicture}[scale = .65]
\begin{axis}[axis y line=middle, axis x line=middle, xmin=-2.2, xmax=2.2, ymin=-.5, ymax=2.2,name=myplot, ytick={0, 1,2}]
\addplot[{\colorone}, domain=-2:2, thick, smooth]{abs x};
\end{axis}
\node [right] at (myplot.right of origin) {\scriptsize $x$};
\node [above] at (myplot.above origin) {\scriptsize $y$};
\end{tikzpicture} \\

$y=x^2$ & $y=\sqrt x$& $y=|x|$\\
 & &  \\
 & &  \\

\begin{tikzpicture}[scale = .65]
\begin{axis}[axis y line=middle, axis x line=middle, xmin=-2.2, xmax=2.2, ymin=-2.2, ymax=2.2,name=myplot]
\addplot[{\colorone}, domain=-2:2, thick, smooth]{x*x*x};
\end{axis}
\node [right] at (myplot.right of origin) {\scriptsize $x$};
\node [above] at (myplot.above origin) {\scriptsize $y$};
\end{tikzpicture} &

\begin{tikzpicture}[scale = .65]
\begin{axis}[axis y line=middle, axis x line=middle, xmin=-2.2, xmax=2.2, ymin=-.5, ymax=3.2,name=myplot, ytick={0, 1,2,3}]
\addplot[{\colorone}, domain=-2:2, thick, smooth]{e^x};
\end{axis}
\node [right] at (myplot.right of origin) {\scriptsize $x$};
\node [above] at (myplot.above origin) {\scriptsize $y$};
\end{tikzpicture} &

\begin{tikzpicture}[scale = .65]
\begin{axis}[axis y line=middle, axis x line=middle, xmin=-.5, xmax=3.2, ymin=-2.2, ymax=2.2,name=myplot, ytick={-2,-1,0, 1,2},xtick={0,1,2,3}]
\addplot[{\colorone}, domain=0:3, thick, smooth]{ln x};
\end{axis}
\node [right] at (myplot.right of origin) {\scriptsize $x$};
\node [above] at (myplot.above origin) {\scriptsize $y$};
\end{tikzpicture} \\

$y=x^3$ & $y=e^x$& $y=\ln x$\\
 & &  \\
 & &  \\

\begin{tikzpicture}[scale = .65]
\begin{axis}[axis y line=middle, axis x line=middle, xmin=-.5, xmax=6.5, ymin=-1.2, ymax=1.2,name=myplot, ytick={-1,0,1}, xtick={-6.28318, -4.7123889, -3.14159, -1.5708, 1.5708, 3.14159, 4.7123889, 6.28318}, xticklabels={-$2\pi$, $-\frac{3\pi}{2}$,$-\pi$, $-\frac{\pi}{2}$, $\frac{\pi}{2}$,$\pi$, $\frac{3\pi}{2}$, $2\pi$}]
\addplot[{\colorone}, domain=-.5:6.4, thick, smooth]{sin deg(x)};
\end{axis}
\node [right] at (myplot.right of origin) {\scriptsize $x$};
\node [above] at (myplot.above origin) {\scriptsize $y$};
\end{tikzpicture}&

\begin{tikzpicture}[scale = .65]
\begin{axis}[axis y line=middle, axis x line=middle, xmin=-.5, xmax=6.5, ymin=-1.2, ymax=1.2,name=myplot, ytick={-1,0,1}, xtick={-6.28318, -4.7123889, -3.14159, -1.5708, 1.5708, 3.14159, 4.7123889, 6.28318}, xticklabels={-$2\pi$, $-\frac{3\pi}{2}$,$-\pi$, $-\frac{\pi}{2}$, $\frac{\pi}{2}$,$\pi$, $\frac{3\pi}{2}$, $2\pi$}]
\addplot[{\colorone}, domain=-.5:6.4, thick, smooth]{cos deg(x)};
\end{axis}
\node [right] at (myplot.right of origin) {\scriptsize $x$};
\node [above] at (myplot.above origin) {\scriptsize $y$};
\end{tikzpicture}&

\begin{tikzpicture}[scale = .65]
\begin{axis}[axis y line=middle, axis x line=middle, xmin=-3.2, xmax=3.2, ymin=-3.2, ymax=3.2,name=myplot, ytick={-3,-2,-1,...,3}]
\addplot[{\colorone}, domain=0:3, thick, smooth]{1/x};
\addplot[{\colorone}, domain=-3:0, thick, smooth]{1/x};
\end{axis}
\node [right] at (myplot.right of origin) {\scriptsize $x$};
\node [above] at (myplot.above origin) {\scriptsize $y$};
\end{tikzpicture}\\

$y=\sin x$ & $y=\cos x$ & $\displaystyle y=\frac{1}{x}$\\
 & &  \\
\end{tabular}


need to say $c>0$
\begin{center}
\begin{tabular}{l c}
function & shift of $f(x)$\\
\hline
$y=f(x)+c$ & $$c units upward\\
$y=f(x)-c$ & $c$ units downward\\
$y=f(x+c)$ & $c$ units left\\
$y=f(x-c)$ & $c$ units right\\
\end{tabular}
 \end{center}

need to say $c>1$
\begin{center}
\begin{tabular}{l c}
function &  transformation of $f(x)$\\
\hline
$y=cf(x)$ & stretch vertically by a factor of $c$\\
$y=\frac{1}{c} f(x)$ & shrink vertically by a factor of $c$\\
$y=f(cx)$ & shrink horizontally by a factor of $c$\\
$y=f(\frac{x}{c})$ & stretch horizontally by a factor of $c$\\
$y=-f(x)$ & reflect about the $x$-axis\\
$y=f(-x)$ & reflect about the $y$-axis\\
\end{tabular}
\end{center}

Sketch the graph of the following functions using the base function and the appropriate transformations.

\begin{enumerate}
\begin{multicols}{2}
\item $y=\displaystyle \frac{1}{x+3}$
\item $y=\sqrt{x+3}+1$
\item $y=|x-4|$
\item $y=3\cos x+2$
\item $y=4|x|+1$
\item $y=-\frac{1}{3}(x-2)^2+3$
\item $y=(x-3)^3$
\item$y=|\sin 2x|$
\end{multicols}
\end{enumerate}

We said above that domain is the set of real numbers for which the function (rule) defines a real number and makes sense. Ask yourself, "what values can I put into the function and get a real value out?" There are generally two key expressions that will limit the domain of a function from all real numbers. We may not divide by zero and we may not have a negative number underneath an even root. The following examples illustrate how we restrict the domain when we see these expressions.

Example: Find the domain of the function.

1.    $f(x)=\sqrt{x-4}$ 

Solution:  The square root of a negative number is not defined as a real number so the domain of $f$ will be all real numbers for which 
$x-4\geq 0$ which is $x \geq 4$. In interval notation, this is $[ 4,\infty)$.
 
2. $\ds g(x)= \frac{3}{x^2-9}$

Solution: We cannot divide by zero so we factor the denominator of $g$ and exclude those values where the denominator is zero.  $$g(x)=\frac{3}{x^2-9}=\frac{3}{(x-3)(x+3)}$$ 
We see that $x\not= 3,-3$ for $g$ to be defined, which is written in interval notation as $(-\infty,-3)\cup (-3,3)\cup (3,\infty)$.

3. $\ds h(x)=\frac{1}{\sqrt{x^2-4}}$

Solution: For $h$ to be defined as a real number we must have $x^2-4>0$. This is equivalent to $(x-2)(x+2)>0$. From the sign chart below, we can see that $x^2-4$ will be greater than zero on $(-\infty,-2)\cup(2,\infty)$.

\begin{center}
\setlength{\unitlength}{4.6em}
\begin{picture}(6,2.5)
\scriptsize
\put(3,2.05){\vector(1,0){3}}
\put(3,2.05){\vector(-1,0){3}}
\put(0,2.2){$x^2-4$}
\put(.1,1.7){$x$}
\put(2.1,1.95){\line(0,1){.2}}
\put(4.14,1.95){\line(0,1){.2}}
\put(1.95,1.7){$-2$}
\put(2.05,2.2){$U$}
\put(4.1,1.7){$2$}
\put(4.1,2.2){$U$}
\put(.5,2.2){$+  +  +  +  +  +  +  +  + $}
\put(2.25,2.2){$-  -  -  -  -  -  -  -  -  -  -$}
\put(4.3,2.2){$+  +  +  +  +  +  +  +  + $}
\end{picture}
\end{center}	


Find the domain of the following functions.
\begin{enumerate}
\setcounter{enumi}{8}
\begin{multicols}{2}
\item $g(x)=(x-3)^2+5$
\item $ f(x)=\sqrt{x+7}-3$
\item$f(x)=\sqrt{x^2-6x-7}$
\item $f(x)=3|x-2|+4$
\item $\displaystyle f(x)=\frac{x-3}{x^2-4x+4}$
\item $\displaystyle g(x)=\frac {x-3}{x^2-x+6}$
\item $h(x)=\sin (x+ 3\pi)$
\item $\displaystyle f(x)=\frac{4x+1}{\sqrt{x^2-4}}$
\item $\displaystyle h(x)=\frac{\cos x}{x}$
\item $g(x)=|x^2-x-6|$
\end{multicols}
\end{enumerate}

Sketch the graph of the following piecewise functions.
\begin{enumerate}
\setcounter{enumi}{18}
\begin{multicols}{2}
\item $f(x)=\begin{cases} x^2-3 & x<2\\
x+4 & x\geq 2 \end{cases}$

\item $f(x)=\begin{cases} 3 & x\leq -1\\
2-x^2 & -1<x<4 \\
-3 & x\geq 4 \end{cases}$

\item $f(x)=\begin{cases} 3-x & x<-2\\
x^2+4 & -2\leq x\leq 3\\
e^x & x>3 \end{cases}$

\item $f(x)=\begin{cases} \sin x & x\leq 0\\
2x+1 & x>0 \end{cases}$
\end{multicols}
\end{enumerate}

Given $f(x)$, evaluate the expression.

\begin{enumerate}
\setcounter{enumi}{22}
\item $f(x)=3x^2-2x+6$
\begin{enumerate}
\begin{multicols}{3}
\item $f(2)$
\item $f(-1)$
\item $f(a)$
\item $f(x+h)$
\item $\ds \frac{f(x+h)-f(x)}{h}$
\end{multicols}
\end{enumerate}

\item $f(x)=\sqrt{x-2}$
\begin{enumerate}
\begin{multicols}{3}
\item $f(4)$
\item $f(-3)$
\item $f(t)$
\item $f(x+h)$
\item $\ds \frac{f(x+h)-f(x)}{h}$
\end{multicols}
\end{enumerate}

\item $\ds f(x)=\frac{1}{x}$
\begin{enumerate}
\begin{multicols}{3}
\item $f(-1)$
\item $f(9)$
\item $f(t+3)$
\item $f(x+h)$
\item $\ds \frac{f(x+h)-f(x)}{h}$
\end{multicols}
\end{enumerate}
\end{enumerate}

\end{document}



