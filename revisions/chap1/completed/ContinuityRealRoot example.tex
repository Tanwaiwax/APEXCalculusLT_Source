\documentclass[10pt]{article}


\usepackage{ifthen}

\usepackage{lipsum}
\usepackage{pgfplots}

\usepackage{eso-pic,calc}
\usepackage[font=small]{caption}
\usepgfplotslibrary{external}
\usetikzlibrary{calc}
\usetikzlibrary{shadings}
\usepackage{tikz}
\usetikzlibrary{positioning,chains,fit,shapes,calc,arrows,patterns}
\usepackage{tkz-graph}
\usetikzlibrary{arrows, petri, topaths}
\usepackage{tkz-berge}
\usepackage[all]{xy}
\usepackage{textcomp}
\usepackage[h]{esvect}
\usepackage[normalem]{ulem}

\pgfplotsset{compat=1.8}
\usepackage{amssymb}

\usepackage{amsmath}

\newcommand{\ds}{\displaystyle}


\begin{document}

Insert new example on old page 42. Should come right before example 25 Using the bisection Method. (there is also one paragraph of text it should come before)

Example:  Finding roots

Show that $f(x)=x^3+x+3$ has at least one real root.

Solution:  We must determine an interval on which the function changes from positive to negative values. We start by evaluting $f$ at different values. $f(0)=3>0$ and $f(1)=5>0$. As we choose larger positive values of $x$, we can see that $f(x)$ values will continue to grow. $f(-1)=1>0$ and $f(-2)=-7<0$ so we know $f(x)$ must change sign in $[-2,-1]$.  $f(x)$ is a polynomial so it is continuous on all real numbers so is continuous on $[-2,-1]$. By the Intermediate Value Theorem there is a $c$ in $[-2,-1]$ where $f(x)=0$. Thus $f(x)$ must have at least one real root on $[-2,-1]$.  


Note that in the above example you were not asked to find the root, just to show that the function HAD a root. We could find a smaller interval that would better approximate the root using the \textbf{Bisection Method}. We demonstrate this in the following example.



\end{document}