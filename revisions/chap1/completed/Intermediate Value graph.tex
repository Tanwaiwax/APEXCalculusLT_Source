\documentclass[10pt]{article}


%%
% HERE THERE BE DRAGONS
%%

\usepackage{ifthen}

\usepackage{lipsum}
\usepackage{pgfplots}
\pgfplotsset{colormap={coloronemap}{rgb=(.4,.4,1); rgb=(.8,.8,1)}}
\pgfplotsset{colormap={colortwomap}{rgb=(1,.4,.4); rgb=(1,.8,.8)}}
%\pgfplotsset{compat=1.3}
\usepackage{eso-pic,calc}
\usepackage[font=small]{caption}
\usepgfplotslibrary{external}
\usetikzlibrary{calc}
\usetikzlibrary{shadings}

\usepackage[h]{esvect}

\pgfplotsset{compat=1.8}

\usepackage[paperheight=11in,paperwidth=3in,%inner=1in,includeheadfoot,
textheight=10in,%textwidth=345pt,
marginparwidth=150pt]{geometry}
%% end detour
%%

\usepackage{amsmath}




\ifthenelse{\boolean{xetex}}%
	{\sffamily
	%%\usepackage{fontspec}
	\usepackage{mathspec}
	\setallmainfonts[Mapping=tex-text]{Calibri}
	\setmainfont[Mapping=tex-text]{Calibri}
	\setsansfont[Mapping=tex-text]{Calibri}
	\setmathsfont(Greek){[cmmi10]}}
	{Please compile with XeLaTeX}

\newboolean{colorprint}
\setboolean{colorprint}{true}
%\setboolean{colorprint}{false}

\ifthenelse{\boolean{colorprint}}{%
\newcommand{\colorone}{blue}
\newcommand{\colortwo}{red}
\newcommand{\coloronefill}{blue!15!white}
\newcommand{\colortwofill}{red!15!white}
\newcommand{\colormapone}{rgb=(.4,.4,1); rgb=(.8,.8,1)}
\newcommand{\colormaptwo}{rgb=(1,.4,.4); rgb=(1,.8,.8)}
\newcommand{\colormapplaneone}{rgb=(.7,.7,1); rgb=(.9,.9,1)}
\definecolor{colormaponebottom}{rgb}{.4,.4,1}
\definecolor{colormaponetop}{rgb}{.8,.8,1}
\definecolor{colormaptwobottom}{rgb}{1,.4,.4}
\definecolor{colormaptwotop}{rgb}{1,.8,.8}
}% ends color
{% not color
\newcommand{\colorone}{black}
\newcommand{\colortwo}{black!50!white}
\newcommand{\coloronefill}{black!15!white}
\newcommand{\colortwofill}{black!05!white}
\newcommand{\colormapone}{rgb=(.4,.4,.4); rgb=(.7,.7,.7)}
\newcommand{\colormaptwo}{rgb=(.6,.6,.6); rgb=(.9,.9,.9)}
\newcommand{\colormapplaneone}{rgb=(.8,.8,.8); rgb=(.95,.95,.95)}
\definecolor{colormaponebottom}{rgb}{.4,.4,.4}
\definecolor{colormaponetop}{rgb}{.7,.7,.7}
\definecolor{colormaptwobottom}{rgb}{.6,.6,.6}
\definecolor{colormaptwotop}{rgb}{.9,.9,.9}
}%
\newcommand{\la}{\left\langle}
\newcommand{\ra}{\right\rangle}
\newcommand{\dotp}[2]{\ensuremath{\vec #1 \cdot \vec #2}}
\newcommand{\proj}[2]{\ensuremath{\text{proj}_{\,\vec #2}{\,\vec #1}}}

\newcommand{\fp}{\ensuremath{f\,'}}

\DeclareMathOperator{\sech}{sech}
\DeclareMathOperator{\csch}{csch}

\newcommand{\threedlines}[4][]{\draw [dashed,#1] (axis cs: #2,#3,#4) -- (axis cs: #2,#3,0) -- (axis cs: #2,0,0)  (axis cs: #2,#3,0)--(axis cs:0,#3,0);}

\newcommand{\mydraw}{\draw (axis cs:0,0,0) -- (axis cs:1,1,0);}
\newcommand{\ds}{\displaystyle}



\begin{document}

\begin{tikzpicture}
\begin{axis}[width=\marginparwidth, xtick={.7, 1.5, 4.7}, xticklabels={\scriptsize{$a$},${c}$,\scriptsize{$b$}}, ytick={.54, 2.98,4.16},yticklabels={\scriptsize{$f(b)$},$L$,\scriptsize{$f(a)$}}, axis y line=middle,axis x line=middle, ymin=-.5, ymax=5.5, xmin=-.5, xmax=5.2, name=myplot]
\addplot[{\colorone}, domain=0:5, thick, smooth] {.2*(x-2)^3-2*(x-2)+2};
\draw[{\colortwo},thick] (axis cs:1.5,0) -- (axis cs: 1.5,2.98);
\draw[{\colortwo},thick] (axis cs:1.5,2.98) -- (axis cs: 0,2.98);
\draw[dashed,thin](axis cs: .7,0)--(axis cs:.7,4.16);
\draw[dashed,thin](axis cs: .7,4.16)--(axis cs:0,4.16);
\draw[dashed,thin](axis cs: 4.7,0)--(axis cs:4.7,.54);
\draw[dashed,thin](axis cs: 4.7,.54)--(axis cs:0,.54);
\end{axis}
\node [right] at (myplot.right of origin) {\scriptsize $x$};
\node [above] at (myplot.above origin) {\scriptsize $y$};
%\node at (0,1.8) {\scriptsize $g(x)$};
\node[\colorone] at (3.5,1.1){\scriptsize $y=f(x)$};
%\node[\colortwo] at (.15,.75){\scriptsize $f(x)$};
%\node[\colorone] at (2.2,1.35) {\scriptsize $c$};
\end{tikzpicture}



\begin{tikzpicture}
\begin{axis}[width=\marginparwidth, axis y line=middle, axis x line=middle, name=myplot, ymin=-.5, ymax=5.5, xmin=-.5, xmax=4.2, xtick={1,2,3.5}, xticklabels={\scriptsize{$a$},\scriptsize{?},\scriptsize{$b$}}, ytick={.5,1.5,3.5},yticklabels={\scriptsize{$f(a)$}, $L$, \scriptsize{$f(b)$}}]
\addplot[{\colorone}, domain=0:2, thick, smooth] {x/2};
\addplot[{\colorone}, domain=2:4, thick, smooth] {x};
\fill[white,draw=black,thick] (axis cs:2,2) circle (1.5pt);
\fill[white,draw=black,thick] (axis cs:2,1) circle (1.5pt);
\draw[{\colortwo}, thick](axis cs: 2,0) -- (axis cs: 2,1.5);
\draw[{\colortwo},thick](axis cs: 2,1.5)--(axis cs: 0,1.5);
\draw[dashed,thin](axis cs: 1,0)--(axis cs:1,.5);
\draw[dashed,thin](axis cs: 1,.5)--(axis cs:0,.5);
\draw[dashed,thin](axis cs: 3.5,0)--(axis cs:3.5,3.5);
\draw[dashed,thin](axis cs: 3.5,3.5)--(axis cs:0,3.5);
\end{axis}
\node [right] at (myplot.right of origin) {\scriptsize $x$};
\node [above] at (myplot.above origin) {\scriptsize $y$};
%\node at (0,1.8) {\scriptsize $g(x)$};
\node[\colorone] at (3,2.3){\scriptsize $y=f(x)$};
%\node[\colortwo] at (.15,.75){\scriptsize $f(x)$};
%\node[\colorone] at (2.2,1.35) {\scriptsize $c$};
\end{tikzpicture}
\end{document}



