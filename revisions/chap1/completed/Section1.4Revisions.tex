\documentclass[10pt]{article}


\usepackage{ifthen}

\usepackage{lipsum}
\usepackage{pgfplots}

\usepackage{eso-pic,calc}
\usepackage[font=small]{caption}
\usepgfplotslibrary{external}
\usetikzlibrary{calc}
\usetikzlibrary{shadings}
\usepackage{tikz}
\usetikzlibrary{positioning,chains,fit,shapes,calc,arrows,patterns}
\usepackage{tkz-graph}
\usetikzlibrary{arrows, petri, topaths}
\usepackage{tkz-berge}
\usepackage[all]{xy}
\usepackage{textcomp}
\usepackage[h]{esvect}
\usepackage[normalem]{ulem}

\pgfplotsset{compat=1.8}


\usepackage{amsmath}

\newcommand{\ds}{\displaystyle}


\begin{document}


Delete the first paragraph of the section.  Paragraph after the list of ways a limit fails to exist should end with the statement: We will consider \#2 in more detail in Section 1.5.  %%%%%%%%%% I'm not sure if you add a \ref to the section or not%%%%%%




Example 17 function has been changed to  $\ds f(x) = \begin{cases} 2x & 0\leq x\leq 1 \\ 6-2x & 1<x<2\end{cases},$

Eliminate the graph on p. 34. (Figure 1.24)

Change the Solution of Example 20 to read:

In this example, we will evaluate the limit by only considering the definition of $f$.
		\begin{enumerate}
		\item		As $x$ approaches 1 from the left, $f(x)$ is defined to be $x^2$. Therefore $$\lim_{x\to1^-} f(x)=\lim_{x\to1^-} x^2=1$$.
		\item		As $x$ approaches 1 from the right, $f(x)$ is defined to be $2-x$. Therefore $$ \lim_{x\to 1+} f(x)=\lim_{x\to 1+} 2-x=1$$.
		\item		Since the right and left hand limits are equal at $x=1$, i.e. $\ds \lim_{x\to1^-} f(x)=\ds \lim_{x\to1^+} f(x)=1$, this tells us $\ds \lim_{x\to1} f(x)=1$.
		\item		To find $f(1)$, we use the $x^2$ piece of our function so $f(1)=1$ 		
		\end{enumerate}


Exercises:

Eliminate 7.


Add in a new section of problems after 21. with the following instructions.  The new exercises are listed after the directions.

Sketch the graph of a function $f$ that satisfies all of the given conditions.

$\ds \lim_{x\to 1^-} f(x)=2$, \quad $\ds \lim_{x\to 1^+} f(x)=-3$, \quad $f(1)=0$

$\ds \lim_{x\to -1} f(x)=3$, \quad $\ds \lim_{x\to 3^-} f(x)=1$, \quad $\ds \lim_{x\to 3^+} f(x)=-2$, \quad $f(-1)=1$, \quad $f(3)=-2$

$\ds \lim_{x\to -2^-} f(x)=1$, \quad  $\ds \lim_{-2^+} f(x)=0$, \quad $\ds \lim_{x\to 0^-} f(x)=3$, \quad $\ds \lim_{x\to 0^+} f(x)=-1$, \quad $f(-2)=4$, \quad $f(0)=-3$

$\ds \lim_{x\to 0^-} f(x)=0$, \quad $\ds \lim_{x\to 0^+} f(x)=2$, \quad $\ds \lim_{x\to 4^-} f(x)=-2$, \quad $\ds \lim_{x\to 4^+} f(x)=1$, \quad $f(0)=2$, \quad $f(4)=-2$

The solution to each of the exercises above is        Answers may vary.


In the Review exercises
	Change 23. to $\ds \lim_{h\to 0} \frac{\sqrt{3+h}-\sqrt{3}}{h}$.  Solution: $\frac{1}{2\sqrt{3}}$


	 change 24. to $\ds \lim_{h\to 0} \frac{(2+h)^2-4}{h}$  Solution: 2
\end{document}