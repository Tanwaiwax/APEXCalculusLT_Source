\documentclass[10pt]{article}


\usepackage{ifthen}

\usepackage{lipsum}
\usepackage{pgfplots}

\usepackage{eso-pic,calc}
\usepackage[font=small]{caption}
\usepgfplotslibrary{external}
\usetikzlibrary{calc}
\usetikzlibrary{shadings}
\usepackage{tikz}
\usetikzlibrary{positioning,chains,fit,shapes,calc,arrows,patterns}
\usepackage{tkz-graph}
\usetikzlibrary{arrows, petri, topaths}
\usepackage{tkz-berge}
\usepackage[all]{xy}
\usepackage{textcomp}
\usepackage[h]{esvect}
\usepackage[normalem]{ulem}

\pgfplotsset{compat=1.8}
\usepackage{amssymb}

\usepackage{amsmath}

\newcommand{\ds}{\displaystyle}


\begin{document}
 
This section will be moved so will be Section 1.6.

Add following text right before Example 21:

If $f$ is defined near $c$,  we say that $f$ is \textbf{discontinuous at $\mathbf{c}$} or $f$ has a \textbf{discontinuity at $\mathbf{c}$} if $f$ is not continuous at $c$. We will discuss three types of discontinuities. \begin{enumerate}
\item \textbf{Removable discontinuity}\\ This type of discontinuity is called removable because we could remove the discontinuity by redefining the function at a single point.
\item \textbf{Infinite discontinuity} \\ The function is approaching $\pm \infty$ at some $x$ value.
\item \textbf{Jump discontinuity}\\ The function "jumps" from one value to another.
\end{enumerate}

Insert a Figures with the three graphs of types of discontinuities horizontally across page. Title them according to type of discontinuity


p 39. First 3 sentences of text below Example 23 should be moved before example 23.  

p 40. Theorem 8 Split the theorem so it only contains 1.-6. Include the following text directly after this theorem.

The proofs of each of the parts of Theorem \ref{thm:continuous_functions} follow from the Basic Limit Properties given in Theorem \ref{thm:limit_algebra}. We will prove the product of two continuous functions is continuous now.

Proof:

We know that $f$ and $g$ are continuous at $c$ so by definition we have $$\lim_{x\to c}f(x)=f(c) \quad \text{and} \quad \lim_{x\to c} g(x)=g(c).$$  Therefore,
\begin{align*}
\lim_{x\to c} (f\cdot g)(x)&=\lim_{x\to c} f(x)\cdot g(x)\\
&=\lim_{x\to c}f(x) \cdot \lim_{x\to c} g(x)\\
&=f(c)\cdot g(c)\\
&=(f\cdot g)(c)
\end{align*}

Make a new Theorem titled Continuity of Compositions to read as follows:

Let $f$ be continuous on $I$, where the range of $f$ on $I$ is $J$, and let $g$ be continuous on $J$. Then $$ (g\circ f)(x)=g(f(x))$$ is continuous on $I$.

Eliminate 8. from Theorem 9.

Add the following text right after theorem 9:     Theorem \ref{thm:continuity_algebra} and Theorem \ref{thm:continuous_functions} tell us that the following types of funcitons are continuous on their domains: 
\begin{center}
\begin{tabular}{l l l}
polynomials& rational functions&exponential functions\\
trigonometric functions &  root functions& logarithmic functions
\end{tabular}
\end{center}
In the following example, we will show how we appy the previous theorems.

Insert two margin graphs on p 42. One will illustrate the IVT and the other will show a graph not continuous so IVT doesn't hold.

Add the following Exercises:


33. 
$\ds f(x) = \begin{cases} 
\frac{x+1}{x+4} & x<2\\ x^2-3 &2\leq x\leq 5 \\ 6-2x & x>5\end{cases}$
Solution: $(-\infty,-4)\cup (-4,2)\cup (2,5)\cup (5,\infty)$
34.
$\ds f(x) = \begin{cases} 
\frac{1}{x-1} & x<0\\ 2x^2-3x-1 &0\leq x\leq 2 \\ 5x^2-4x & x>2\end{cases}$
Solution: $(-\infty, 2)\cup(2,\infty)$

Directions for the following 4 problems to be inserted: 

Find the value(s) of $a$ and $b$ so that the function is continuous on $\mathbb{R}$.  %%% this required \usepackage{amssymb} which i added to preamble%%%%%%%%%%%%%%

$\ds g(x)= \begin{cases} 
ax^2+3x & x<2\\
x^3-ax & x\geq 2\end{cases}$			Solution: $a=\frac{1}{3}$

$\ds f(x)= \begin{cases} 
a^2x-ax & x>3\\
4& x\leq 3 \end{cases}$			Solution: $a=-1,\frac{4}{3}$

$\ds f(x)=\begin{cases}
ax-b & x<-1\\
2x^2+3ax+b &-1\leq x<1\\
4&x\geq 1 \end{cases}$				Solution: $a=1, \ b=-1$

$\ds f(x)=\begin{cases}
x^2+2x & x\leq a\\
-1 & x>a 
\end{cases}$					Solution: $a=-1$

Directions for the following 4 problems to be inserted:  Sketch the graph of a function that has the following properties.

$f$ is discontinuous at 3, but continuous from the left at 3, and continuous elsewhere.

$f$ is discontinuous at -1 and 2, but continuous from the right at -1 and continuous from the left at 2, and continuous elsewhere.

$f$ has a jump discontinuity at -2 and an infinite discontinuity at 4 and is continuous elsewhere.

$f$ has a removable discontinuity at 2, is continuous only from the left at 5, and is continuous elsewhere.

Solutions to above questions:  Answers may vary.
Change the directions to 37-40 to read: Show that the functions have at least one real root. 
 and delete the intervals from each of 37-40.
\end{document}