\documentclass[11pt]{report}
\usepackage[letterpaper, total={6.5in, 10in}]{geometry}
%\usepackage{fancyhdr}
%\pagestyle{fancy}
\usepackage{amsmath, amsthm, mathpazo, epic, eepic, color, array}
\usepackage{amssymb}
%\usepackage{graphicx}
\usepackage{cancel}
\usepackage{pgfplots}
\usepackage{multicol}
\pgfplotsset{compat=1.13}
\usepackage{etoolbox}
\makeatletter
\patchcmd{\chapter}{\if@openright\cleardoublepage\else\clearpage\fi}{}{}{}
\makeatother
\usepackage{hyperref}
\usepackage[normalem]{ulem}

\usepackage{enumerate}
\usepackage{enumitem}

\usepackage{tikz}
\usetikzlibrary{positioning,chains,fit,shapes,calc,arrows,patterns}
\usepackage{tkz-graph}
\usetikzlibrary{arrows, petri, topaths}
\usepackage{tkz-berge}
\usepackage[all]{xy}
\usepackage{textcomp}

\newboolean{colorprint}
\setboolean{colorprint}{true}
%\setboolean{colorprint}{false}

\ifthenelse{\boolean{colorprint}}{%
\newcommand{\colorone}{blue}
\newcommand{\colortwo}{red}
\newcommand{\coloronefill}{blue!15!white}
\newcommand{\colortwofill}{red!15!white}
\newcommand{\colormapone}{rgb=(.4,.4,1); rgb=(.8,.8,1)}
\newcommand{\colormaptwo}{rgb=(1,.4,.4); rgb=(1,.8,.8)}
\newcommand{\colormapplaneone}{rgb=(.7,.7,1); rgb=(.9,.9,1)}
\definecolor{colormaponebottom}{rgb}{.4,.4,1}
\definecolor{colormaponetop}{rgb}{.8,.8,1}
\definecolor{colormaptwobottom}{rgb}{1,.4,.4}
\definecolor{colormaptwotop}{rgb}{1,.8,.8}
}% ends color
{% not color
\newcommand{\colorone}{black}
\newcommand{\colortwo}{black!50!white}
\newcommand{\coloronefill}{black!15!white}
\newcommand{\colortwofill}{black!05!white}
\newcommand{\colormapone}{rgb=(.4,.4,.4); rgb=(.7,.7,.7)}
\newcommand{\colormaptwo}{rgb=(.6,.6,.6); rgb=(.9,.9,.9)}
\newcommand{\colormapplaneone}{rgb=(.8,.8,.8); rgb=(.95,.95,.95)}
\definecolor{colormaponebottom}{rgb}{.4,.4,.4}
\definecolor{colormaponetop}{rgb}{.7,.7,.7}
\definecolor{colormaptwobottom}{rgb}{.6,.6,.6}
\definecolor{colormaptwotop}{rgb}{.9,.9,.9}
}%

\newlength\tindent
\setlength{\tindent}{\parindent}
\setlength{\parindent}{0pt}
\renewcommand{\indent}{\hspace*{\tindent}}

\pgfplotsset{my style/.append style={axis x line=middle, axis y line=
middle, xlabel={$x$}, ylabel={$y$}, axis equal }}

\pgfplotsset{compat=1.13}
\newcommand{\ds}{\displaystyle}
\begin{document}

I found this note in my first round of edits for this chapter. We did want Conic Sections in Calculus III:\\
\textbf{Section 9.1 is now 12.0}\\

The following page numbers are from the "Standalone" document you sent.\\

\textbf{p. 1, last paragraph}\\
The geometric definition of the parabola and distance formula can be used to derive the quadratic function whose graph is a parabola with vertex at the origin.
$$y=\frac{1}{4p}x^2$$
Applying transformations of functions we get the following standard form of the parabola.\\ 

\textbf{p. 2, Key Idea 1}\\
For each part move the focus statement into the paragraph (i.e.)\\
1. Vertical Axis... \sout{and} directrix $y=k-p$ and focus at $(h, k+p)$ in standard form is ..."\\ \\
Same for part 2. The hope is that this will make it easier to get the box onto the previous page. \\ \\

\textbf{p. 2}\\
Cut Example 2 \& Figure 10.4\\

\textbf{p. 4, line -6}\\
\sout{The choice of $a$ and $b$ is not without reason} As shown in Figure $10.6$, the value of $a$ the values of $a$ and $b$ have meaning. In general, ...i.e., $2a$."\\

\textbf{p. 6, Definition 45~~~ Hyperbola}\\
"A hyperbola... value of the difference of ..."\\ \\

\textbf{p. 9 Exercises}\\ \\
Keep only the following exercises: \#6-13, 16-19 (delete "Give the location of ... ellipse." from the directions), 24-27, 29-32, 33 - 34 (delete "and foci" from directions), 39-42.


\end{document}

