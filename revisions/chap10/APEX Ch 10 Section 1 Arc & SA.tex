\documentclass[11pt]{report}
\usepackage[letterpaper, total={6.5in, 10in}]{geometry}
%\usepackage{fancyhdr}
%\pagestyle{fancy}
\usepackage{amsmath, amsthm, mathpazo, epic, eepic, color, array}
\usepackage{amssymb}
%\usepackage{graphicx}
\usepackage{cancel}
\usepackage{pgfplots}
\usepackage{multicol}
\pgfplotsset{compat=1.13}
\usepackage{etoolbox}
\makeatletter
\patchcmd{\chapter}{\if@openright\cleardoublepage\else\clearpage\fi}{}{}{}
\makeatother
\usepackage{hyperref}
\usepackage[normalem]{ulem}

\usepackage{enumerate}
\usepackage{enumitem}

\usepackage{tikz}
\usetikzlibrary{positioning,chains,fit,shapes,calc,arrows,patterns}
\usepackage{tkz-graph}
\usetikzlibrary{arrows, petri, topaths}
\usepackage{tkz-berge}
\usepackage[all]{xy}
\usepackage{textcomp}

\newboolean{colorprint}
\setboolean{colorprint}{true}
%\setboolean{colorprint}{false}

\ifthenelse{\boolean{colorprint}}{%
\newcommand{\colorone}{blue}
\newcommand{\colortwo}{red}
\newcommand{\coloronefill}{blue!15!white}
\newcommand{\colortwofill}{red!15!white}
\newcommand{\colormapone}{rgb=(.4,.4,1); rgb=(.8,.8,1)}
\newcommand{\colormaptwo}{rgb=(1,.4,.4); rgb=(1,.8,.8)}
\newcommand{\colormapplaneone}{rgb=(.7,.7,1); rgb=(.9,.9,1)}
\definecolor{colormaponebottom}{rgb}{.4,.4,1}
\definecolor{colormaponetop}{rgb}{.8,.8,1}
\definecolor{colormaptwobottom}{rgb}{1,.4,.4}
\definecolor{colormaptwotop}{rgb}{1,.8,.8}
}% ends color
{% not color
\newcommand{\colorone}{black}
\newcommand{\colortwo}{black!50!white}
\newcommand{\coloronefill}{black!15!white}
\newcommand{\colortwofill}{black!05!white}
\newcommand{\colormapone}{rgb=(.4,.4,.4); rgb=(.7,.7,.7)}
\newcommand{\colormaptwo}{rgb=(.6,.6,.6); rgb=(.9,.9,.9)}
\newcommand{\colormapplaneone}{rgb=(.8,.8,.8); rgb=(.95,.95,.95)}
\definecolor{colormaponebottom}{rgb}{.4,.4,.4}
\definecolor{colormaponetop}{rgb}{.7,.7,.7}
\definecolor{colormaptwobottom}{rgb}{.6,.6,.6}
\definecolor{colormaptwotop}{rgb}{.9,.9,.9}
}%

\newlength\tindent
\setlength{\tindent}{\parindent}
\setlength{\parindent}{0pt}
\renewcommand{\indent}{\hspace*{\tindent}}

\pgfplotsset{my style/.append style={axis x line=middle, axis y line=
middle, xlabel={$x$}, ylabel={$y$}, axis equal }}

\pgfplotsset{compat=1.13}
\newcommand{\ds}{\displaystyle}
\begin{document}

\textbf{Section 10.1}\\

\textbf{p. 583, Chapter Introduction, line 10}\\
Replace "Fittingly" with Ironically.\\ \\


\textbf{p. 583, Figure 10.20}\\
Dave commented that the $x- y-$axis scale for $y=\sin x$ distorts what the function really looks like. He wondered if changing it would be possible or if it would cause problems with seeing the segments in part (b). This is not a big issue - don't worry if we can't change it.\\ \\  

\textbf{p. 584}\\
3rd paragrah, line 3: "segment as the\sout{y} hypotenuse.." \\

4th paragraph, line 1: "As it is \sout{shown} written \sout{here}, this is not a ..."\\

Last line: delete the "=".\\

\textbf{p. 585}\\
Line 1: "...is continuous on $[a,b]$..."\\

\textbf{p. 586, Example 2}\\
Add one more step
$$L= \int_1^2...dx$$
$$=\int_1^2 \sqrt{1+\biggl(\frac{x^2}{16} - \frac{1}{2}+\frac{1}{x^2}\biggr)}~dx$$
$$=\int_1^2 \sqrt{\frac{x^2}{16} + \frac{1}{2}+...}$$
\\

\textbf{p. 587}\\
Figure 10.25: the segment labeled $r$ does not look like the average of the two radii. It looks like one of the radii. Replace current $r$ with $R_1$ and current $R$ with $R_2$ in the figure and add a segment inside the "cylinder" labeled $r$.\\

In the associated text: "where $r = (R_1 + R_2)/2$ is the average of the two radii..."\\


\textbf{Example 588}\\
In the text and equations above Key Idea 34 Dave thought some of the "i"s looked like "j"s. Can you please confirm that all of these are "i"s?\\

\textbf{pp. 589-590, Example 5}\\
Move the first two lines on p. 590, "The solid formed...10.27(a)." so the solution to 1 reads like this: \\
1. The solid formed...10.27(a). Like the integral in Example 4, this integral is easier to setup than to actually integrate. While it is possible to use a trignometric substitution to evaluate this integral it is significantly more difficult than a solution employing the hyperbolic sine:\\


\end{document}

