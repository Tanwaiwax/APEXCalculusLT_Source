\documentclass[11pt]{report}
\usepackage[letterpaper, total={6.5in, 10in}]{geometry}
%\usepackage{fancyhdr}
%\pagestyle{fancy}
\usepackage{amsmath, amsthm, mathpazo, epic, eepic, color, array}
\usepackage{amssymb}
%\usepackage{graphicx}
\usepackage{cancel}
\usepackage{pgfplots}
\usepackage{multicol}
\pgfplotsset{compat=1.13}
\usepackage{etoolbox}
\makeatletter
\patchcmd{\chapter}{\if@openright\cleardoublepage\else\clearpage\fi}{}{}{}
\makeatother
\usepackage{hyperref}
\usepackage[normalem]{ulem}

\usepackage{enumerate}
\usepackage{enumitem}

\usepackage{tikz}
\usetikzlibrary{positioning,chains,fit,shapes,calc,arrows,patterns}
\usepackage{tkz-graph}
\usetikzlibrary{arrows, petri, topaths}
\usepackage{tkz-berge}
\usepackage[all]{xy}
\usepackage{textcomp}

\newboolean{colorprint}
\setboolean{colorprint}{true}
%\setboolean{colorprint}{false}

\ifthenelse{\boolean{colorprint}}{%
\newcommand{\colorone}{blue}
\newcommand{\colortwo}{red}
\newcommand{\coloronefill}{blue!15!white}
\newcommand{\colortwofill}{red!15!white}
\newcommand{\colormapone}{rgb=(.4,.4,1); rgb=(.8,.8,1)}
\newcommand{\colormaptwo}{rgb=(1,.4,.4); rgb=(1,.8,.8)}
\newcommand{\colormapplaneone}{rgb=(.7,.7,1); rgb=(.9,.9,1)}
\definecolor{colormaponebottom}{rgb}{.4,.4,1}
\definecolor{colormaponetop}{rgb}{.8,.8,1}
\definecolor{colormaptwobottom}{rgb}{1,.4,.4}
\definecolor{colormaptwotop}{rgb}{1,.8,.8}
}% ends color
{% not color
\newcommand{\colorone}{black}
\newcommand{\colortwo}{black!50!white}
\newcommand{\coloronefill}{black!15!white}
\newcommand{\colortwofill}{black!05!white}
\newcommand{\colormapone}{rgb=(.4,.4,.4); rgb=(.7,.7,.7)}
\newcommand{\colormaptwo}{rgb=(.6,.6,.6); rgb=(.9,.9,.9)}
\newcommand{\colormapplaneone}{rgb=(.8,.8,.8); rgb=(.95,.95,.95)}
\definecolor{colormaponebottom}{rgb}{.4,.4,.4}
\definecolor{colormaponetop}{rgb}{.7,.7,.7}
\definecolor{colormaptwobottom}{rgb}{.6,.6,.6}
\definecolor{colormaptwotop}{rgb}{.9,.9,.9}
}%

\newlength\tindent
\setlength{\tindent}{\parindent}
\setlength{\parindent}{0pt}
\renewcommand{\indent}{\hspace*{\tindent}}

\pgfplotsset{my style/.append style={axis x line=middle, axis y line=
middle, xlabel={$x$}, ylabel={$y$}, axis equal }}

\pgfplotsset{compat=1.13}
\newcommand{\ds}{\displaystyle}
\begin{document}

\textbf{Section ?.2, Additional Exercises}\\
These all go near \#34-37. Maybe in a section before \#34 so students are more inclined to choose a method instead of use the method suggested for \#34-37.\\

Find a parameterization for the curve.\\

$y=9-4x$\\
Possible Answer: $x=t, y=9-4t$\\

$4x-y^2=5$\\
Possible Answer: $x=\frac{5+t^2}{4}, y=t$\\

$(x+9)^2 + (y-4)^2 =49$\\
Possible Answer: $x=-9+7\cos t, y=4+7\sin t$\\

Find a parametric equation and a parameter interval.

The line segement with endpoints $(-1, -3)$ and $(4,1)$\\
Possible Answer: $x=\frac{5}{4}t+\frac{11}{4}, y=t, [-3,1]$\\

The line segement with endpoints $(-1, 3)$ and $(3,-2)$\\
Possible Answer: $x=-1+4t, y=3-5t, [0,1]$\\

The left half of the parabola $y=x^2 + 2x$\\
Possible Answer: $x=t, y=t^2+2t, (-\infty,-1]$\\

The lower half of the parabola $x=1-y^2$\\
Possible Answer: $x=2t-t^2, y=1-t, [1,\infty)$\\

Find parametric equations and a parameter interval for the motion of a particle that starts at $(1, 0)$ and traces the circle $x^2 + y^2 =1$\\
a. once clockwise\\
Possible Answer: $x=\sin t, y=\cos t, [\pi/2, 5\pi/2]$\\ 

b. once counter-clockwise\\
Possible Answer: $x=\cos t, y=\sin t, [0, 2\pi]$\\ 

c. twice clockwise\\
Possible Answer: $x=\sin t, y=\cos t, [\pi/2, 9\pi/2]$\\ 

d. twice counter-clockwise\\
Possible Answer: $x=\cos t, y=\sin t, [0, 4\pi]$\\ 

Find parametric equations and a parameter interval for the motion of a particle that starts at $(a, 0)$ and traces the circle $\frac{x^2}{a^2} + \frac{y^2}{b^2} =1$\\
a. once clockwise\\
Possible Answer: $x=a\sin t, y=b \cos t, [\pi/2, 5\pi/2]$\\ 

b. once counter-clockwise\\
Possible Answer: $x=a\cos t, y=b \sin t, [0, 2\pi]$\\ 

c. twice clockwise\\
Possible Answer: $x=a\sin t, y=b \cos t, [\pi/2, 9\pi/2]$\\ 

d. twice counter-clockwise\\
Possible Answer: $x=a\cos t, y=b \sin t, [0, 4\pi]$\\ 





\end{document}

