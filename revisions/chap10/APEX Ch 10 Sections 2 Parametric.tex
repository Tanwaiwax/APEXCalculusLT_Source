\documentclass[11pt]{report}
\usepackage[letterpaper, total={6.5in, 10in}]{geometry}
%\usepackage{fancyhdr}
%\pagestyle{fancy}
\usepackage{amsmath, amsthm, mathpazo, epic, eepic, color, array}
\usepackage{amssymb}
%\usepackage{graphicx}
\usepackage{cancel}
\usepackage{pgfplots}
\usepackage{multicol}
\pgfplotsset{compat=1.13}
\usepackage{etoolbox}
\makeatletter
\patchcmd{\chapter}{\if@openright\cleardoublepage\else\clearpage\fi}{}{}{}
\makeatother
\usepackage{hyperref}
\usepackage[normalem]{ulem}

\usepackage{enumerate}
\usepackage{enumitem}

\usepackage{tikz}
\usetikzlibrary{positioning,chains,fit,shapes,calc,arrows,patterns}
\usepackage{tkz-graph}
\usetikzlibrary{arrows, petri, topaths}
\usepackage{tkz-berge}
\usepackage[all]{xy}
\usepackage{textcomp}

\newboolean{colorprint}
\setboolean{colorprint}{true}
%\setboolean{colorprint}{false}

\ifthenelse{\boolean{colorprint}}{%
\newcommand{\colorone}{blue}
\newcommand{\colortwo}{red}
\newcommand{\coloronefill}{blue!15!white}
\newcommand{\colortwofill}{red!15!white}
\newcommand{\colormapone}{rgb=(.4,.4,1); rgb=(.8,.8,1)}
\newcommand{\colormaptwo}{rgb=(1,.4,.4); rgb=(1,.8,.8)}
\newcommand{\colormapplaneone}{rgb=(.7,.7,1); rgb=(.9,.9,1)}
\definecolor{colormaponebottom}{rgb}{.4,.4,1}
\definecolor{colormaponetop}{rgb}{.8,.8,1}
\definecolor{colormaptwobottom}{rgb}{1,.4,.4}
\definecolor{colormaptwotop}{rgb}{1,.8,.8}
}% ends color
{% not color
\newcommand{\colorone}{black}
\newcommand{\colortwo}{black!50!white}
\newcommand{\coloronefill}{black!15!white}
\newcommand{\colortwofill}{black!05!white}
\newcommand{\colormapone}{rgb=(.4,.4,.4); rgb=(.7,.7,.7)}
\newcommand{\colormaptwo}{rgb=(.6,.6,.6); rgb=(.9,.9,.9)}
\newcommand{\colormapplaneone}{rgb=(.8,.8,.8); rgb=(.95,.95,.95)}
\definecolor{colormaponebottom}{rgb}{.4,.4,.4}
\definecolor{colormaponetop}{rgb}{.7,.7,.7}
\definecolor{colormaptwobottom}{rgb}{.6,.6,.6}
\definecolor{colormaptwotop}{rgb}{.9,.9,.9}
}%

\newlength\tindent
\setlength{\tindent}{\parindent}
\setlength{\parindent}{0pt}
\renewcommand{\indent}{\hspace*{\tindent}}

\pgfplotsset{my style/.append style={axis x line=middle, axis y line=
middle, xlabel={$x$}, ylabel={$y$}, axis equal }}

\pgfplotsset{compat=1.13}
\newcommand{\ds}{\displaystyle}
\begin{document}

\textbf{10.2 Parametric Equations}
\vskip .5 truecm

\textbf{p. 593}\\
2nd paragraph: "In the rectangular coordinate system..."\\

\textbf{p. 594, Example 1, 2nd paragraph of the solution}\\
"...These values, along with...the \textbf{orientation} of the graph. This information \sout{helps us determine..."moving."} describes the \textbf{path} of a particle traveling along the curve."\\

\textbf{NOW THE REORGANIZATION BEGINS!}\\

\textbf{pp. 595-596}\\
Starting with line 3: "These examples begin..." through the end of the solution to example 4, move this to "Special Curves" section.\\

Now, the section \textbf{Converting between ... equations} is right after Example 2. 

Change the example in the opening paragraph:\\
It is sometimes useful to rewrite...As an example, given $y=x^2-x-6$, the parametric equations $x=t, y=t^2-t-6$ produce the same parabola. However, other parametrizations...possible alternative.\\

Rewrite Example 5 (now Example 3) as follows:\\
\textbf{Example 3~~~ Converting from rectangular to parametric}\\
Find parametric equations for $f(x)=x^2-x-6$.\\

\textbf{SOLUTION 1}~~~~ For any choice for $x$ we can determine the corresponding $y$ by substitution. If we choose $x=t-1$ then $y=(t-1)^2-(t-1)-6=t^2-3t-4$. Thus $f(x)$ can be represented by the parametric equations
$$x=t-1 ~~~~y=t^2-3t-4$$
On the graph of this parameterization (Figure 10.31) the points have been labeled with the corresponding $t-$values and arrows indicate the path of a particle traveling on this curve. The particle would move from the upper left, down to the vertex at $(.5,-5.75)$ and then up to the right.\\

\textbf{SOLUTION 2}~~~~ If we choose $x=3-t$ then $y=(3-t)^2-(3-t)-6=t^2-5t$. Thus $f(x)$ can also be represented by the parametric equations
$$x=3-t ~~~~y=t^2-5t$$
On the graph of this parameterization (Figure 10.32) the points have been labeled with the corresponding $t-$values and arrows indicate the path of a particle traveling on this curve. The particle would move down from the upper right, to the vertex at $(.5,-5.75)$ and then up to the left.\\

\textbf{SOLUTION 3}~~~~ We can also parameterize any $y=f(x)$ by setting $t=\frac{dy}{dx}$. That is, $t=a$ corresponds to the point on the graph whose tangent line has a slope $a$. Computing $\frac{dy}{dx}$, $f'(x) = 2x-1$ we set $t=2x-1$. Solving for $x$ we find $x=\frac{t+1}{2}$ and by substitution $y=\frac{1}{4}t^2 - \frac{25}{4}$. Thus $f(x)$ can be represented by the parametric equations
$$x=\frac{t+1}{2} ~~~~y=\frac{1}{4}t^2 - \frac{25}{4}$$

The graph of this parameterization is shown in Figure 10.33. To find the point where the tangent line has a slope of $0$, we set $t=0$. This gives us the point (-.5, -5.75)$ which is the vertex of $f(x).\\ \\

\textbf{Example 4~~~ Converting from rectangular to parametric}\\
Find parametric equations for the circle $x^2+y^2=4$.

\textbf{SOLUTION 1:} ~~~~Consider the equivalent equation $\ds \biggl(\frac{x}{2}\biggr)^2+ \biggl(\frac{y}{2}\biggr)^2=1$ and the Pythagorean Identity, $\sin^2t+\cos^2 t=1$. We set $\cos t=\frac{x}{2}$ and $\sin t=\frac{y}{2}$, which gives $x=2\cos t$ and $y=2\sin t$. To trace the circle once, we must have $0\leq t \leq 2\pi$. Note that when $t=0$ a particle tracing the curve would be at the point $(2,0)$ and would move in a counterclockwise direction. \\

\textbf{SOLUTION 2:} ~~~~Another parameterization of the same circle would be $x=2\sin t$ and $y=2\cos t$ for $0\leq t \leq 2\pi$. When $t=0$ a particle would be at the point $(0,2)$ and would move in a clockwise direction. \\

\textbf{SOLUTION 3:} ~~~~We could also let $x=-2\sin t$ and $y=2
\cos t$ for $0\leq t \leq 2\pi$. Also note that we could use $x=2\cos 2t$ and $y=2\sin 2t$ for $0\leq t \leq \pi$. \\ \\

As we have shown in the previous examples, there are many different ways to parameterize any given curve. We sometimes choose the parameter to accurately model physical behavior.\\

\textbf{Example 5~~~ Converting from rectangular to parametric}\\

Find a parameterization that traces the ellipse $\ds \frac{(x-2)^2}{9}+\frac{(y+3)^2}{4}=1$ starting at the point $(-1,-3)$ in a clockwise direction.

\textbf{SOLUTION}~~~~The Pythagorean Identity says $\cos^2t+\sin^2t=1$ so we set $\ds -\cos^2 t =\frac{(x-2)^2}{9}$ and $\ds -\sin^2 t=\frac{(y+3)^2}{4}$. Solving these equations for $x$ and $y$ we have $x=-3\cos t+2$ and $y=-2\sin t-3$  for $0\leq t\leq 2\pi$.\\ \\

\textbf{Example 6~~~ Converting from rectangular to parametric}\\

Find a parameterization for the hyperbola $\ds \frac{(x-2)^2}{9}-\frac{(y-3)^2}{4}=1$.\\

\textbf{SOLUTION}~~~~ We use an alternative form of the Pythagorean Identity, $\sec^2t-\tan^2t=1$. We let  $\ds \sec^2 t=\frac{(x-2)^2}{9}$ and 
$\tan^2 t=\frac{(y-3)^2}{4}$. Solving these equations for $x$ and $y$ we have $x=3\sec t +2$ and $y=2\tan t +3$ for $0\leq t\leq 2\pi$.\\ \\

\textbf{pp. 596 - 597} Old Example 6 is now Example 7\\
\textbf{pp. 597 - 598} Old Example 7 is now Example 8\\
\textbf{pp. 598 - 599} Old Example 8 is now Example 9\\

\textbf{p. 598, paragraph just before Old Example 8 (new 9)}\\
The graphs of these functions \sout{is} are given in ..."\\

\textbf{p. 599}\\
Delete "The Pythagorean Theorem... following Key Idea." and Key Idea 35.\\

\textbf{p. 599}\\
Rename "Special Curves" section to "Graphs of Parametric Equations" Then insert text and examples cut from pp. 595 -596: "These examples begin..." through the end of the solution to old example 4.\\
In the 2nd paragraph of the text from p. 595 the word counterintuitive is hyphenated. Delete the hyphen.\\

Then continue on with "We now present a small gallery of "interesting" and "famous"..." to the end of the section.\\

\textbf{Exercises}\\
The problems I want to add are in a Calculus book in my office. I will send them to you tomorrow (Sunday).

\end{document}

