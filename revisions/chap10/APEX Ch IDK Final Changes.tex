\documentclass[11pt]{report}
\usepackage[letterpaper, total={6.5in, 10in}]{geometry}
%\usepackage{fancyhdr}
%\pagestyle{fancy}
\usepackage{amsmath, amsthm, mathpazo, epic, eepic, color, array}
\usepackage{amssymb}
%\usepackage{graphicx}
\usepackage{cancel}
\usepackage{pgfplots}
\usepackage{multicol}
\pgfplotsset{compat=1.13}
\usepackage{etoolbox}
\makeatletter
\patchcmd{\chapter}{\if@openright\cleardoublepage\else\clearpage\fi}{}{}{}
\makeatother
\usepackage{hyperref}
\usepackage[normalem]{ulem}

\usepackage{enumerate}
\usepackage{enumitem}

\usepackage{tikz}
\usetikzlibrary{positioning,chains,fit,shapes,calc,arrows,patterns}
\usepackage{tkz-graph}
\usetikzlibrary{arrows, petri, topaths}
\usepackage{tkz-berge}
\usepackage[all]{xy}
\usepackage{textcomp}

\newboolean{colorprint}
\setboolean{colorprint}{true}
%\setboolean{colorprint}{false}

\ifthenelse{\boolean{colorprint}}{%
\newcommand{\colorone}{blue}
\newcommand{\colortwo}{red}
\newcommand{\coloronefill}{blue!15!white}
\newcommand{\colortwofill}{red!15!white}
\newcommand{\colormapone}{rgb=(.4,.4,1); rgb=(.8,.8,1)}
\newcommand{\colormaptwo}{rgb=(1,.4,.4); rgb=(1,.8,.8)}
\newcommand{\colormapplaneone}{rgb=(.7,.7,1); rgb=(.9,.9,1)}
\definecolor{colormaponebottom}{rgb}{.4,.4,1}
\definecolor{colormaponetop}{rgb}{.8,.8,1}
\definecolor{colormaptwobottom}{rgb}{1,.4,.4}
\definecolor{colormaptwotop}{rgb}{1,.8,.8}
}% ends color
{% not color
\newcommand{\colorone}{black}
\newcommand{\colortwo}{black!50!white}
\newcommand{\coloronefill}{black!15!white}
\newcommand{\colortwofill}{black!05!white}
\newcommand{\colormapone}{rgb=(.4,.4,.4); rgb=(.7,.7,.7)}
\newcommand{\colormaptwo}{rgb=(.6,.6,.6); rgb=(.9,.9,.9)}
\newcommand{\colormapplaneone}{rgb=(.8,.8,.8); rgb=(.95,.95,.95)}
\definecolor{colormaponebottom}{rgb}{.4,.4,.4}
\definecolor{colormaponetop}{rgb}{.7,.7,.7}
\definecolor{colormaptwobottom}{rgb}{.6,.6,.6}
\definecolor{colormaptwotop}{rgb}{.9,.9,.9}
}%

\newlength\tindent
\setlength{\tindent}{\parindent}
\setlength{\parindent}{0pt}
\renewcommand{\indent}{\hspace*{\tindent}}

\pgfplotsset{my style/.append style={axis x line=middle, axis y line=
middle, xlabel={$x$}, ylabel={$y$}, axis equal }}

\pgfplotsset{compat=1.13}
\newcommand{\ds}{\displaystyle}
\begin{document}

All page numbers refer to the standalone version of Chapter 10 that you sent me this week.\\

\textbf{Section 1}\\ \\

\textbf{p. 5, line -3}. The text about $r_{avg}$ is better. Could the diagram radii be labeled $R_1 ~\&~ R_2$ and the text read "where $r_{avg}$ is the average of $R_1 ~\&~ R_2$" \sout{of the two radii}?\\


\textbf{Section 2}\\ \\

\textbf{p. 12}
Figure 10.10 \& 10.11 tables: Is it possible to make the vertical line column break a double line or thicker to separate $t$ from $x$ and $y$.\\

\textbf{p. 13, line 2}: first word should be "in" instead of "into".\\

\textbf{p. 12, last line}: $whichisthevertexoff(x)$ should be "which is the vertexof $f(x).$\\

\textbf{pp. 13-14, Examples 3 and 4}: Format the "Solution 1", "Solution 2", etc. like it is in Example 1 of Chapter 8.5.\\

\textbf{p. 14, Example 4}:\\
In solution 1 is ther a way to force the equation $\ds \biggl(\frac{x}{2}\biggr)^2+ \biggl(\frac{y}{2}\biggr)^2=1$ to all be on one line. The reformatting of the solutions might fix this, otherwise consder $$\biggl(\frac{x}{2}\biggr)^2+ \biggl(\frac{y}{2}\biggr)^2=1$$

In solution 3: delete also in, "We could \sout{also}..."\\

\textbf{p. 14, Example 5}: There is an error in the solution - this is the new solution\\
Applying the Pythagorean Identity, $\cos^2t+\sin^2t=1$, we set $\ds \cos^2 t =\frac{(x-2)^2}{9}$ and $\ds \sin^2 t=\frac{(y+3)^2}{4}$. Solving these equations for $x$ and $y$ we set $x=-3\cos t+2$ and $y=2\sin t-3$  for $0\leq t\leq 2\pi$.\\ \\

\textbf{p. 14, Example 6}: $\tan^2 t=\frac{(y-3)^2}{4}$ should be in displaystyle $\ds \tan^2 t=\frac{(y-3)^2}{4}$\\

\textbf{p. 21}: I don't think the cycloid section qualifies as an example. Would it be better to put the cyclloid explanation in with Paul's notes on parametric area in section 3?\\

\textbf{Section ?.2, Exercises - reorganized}\\
If we want the order of the exercises to reflect the order the topics are covered in the text then I suggest the following changes:\\
The current \#34-46 and 55-63 should all come before the set from \#20-29 with the current \#41-42 inserted between the current \#57 \& 58.\\


\textbf{Section 3 LOOKS AWESOME!}
\vskip .5 truein

\textbf{Section 4, Cleaning up the P's}\\

\textbf{p. 37, end of paragraph 1}: Should the first line be indented?\\

\textbf{p. 37, end of paragraph 3}: \sout{To avoid confusion...Figure 10.27.}\\

\textbf{pp. 38-40, "Polar to Rectangular" section}:\\
 Delete P in 1st paragraph of this section.\\

In Example 2 part 1 replace P with a point name, $A(2, 2\pi/3)$ and $B(-1, 5\pi/4)$. Use these names in the solution and in Figure 10.30(a).\\ 

In Example 2 part 2 delete the Ps that indicate points - the points are not named with a letter in Figure 10.30(b) so I think just deleting the Ps is ok.\\

\textbf{p. 41, Example 4} has one P to delete. If you think clarification is needed about 
the point $(2,0)$ being polar you could insert "polar" in that sentence.\\

\textbf{p. 47, Example 7}: I missed an = that needs to be an $\approx$ in the last line of the solution, should be $\approx 250.53^\circ$\\

\textbf{p. 48, Cleaning up the P's in the Exercises}: In \#5-6 \& 9-10 use A, B, C, D to label points without the $P=$\\

\textbf{Section ?.4, p. 53}: The first paragraph is (or was it you already cut it) a repeat of the last paragraph on p. 52.\\

\end{document}

