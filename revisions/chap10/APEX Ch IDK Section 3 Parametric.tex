\documentclass[11pt]{report}
\usepackage[letterpaper, total={6.5in, 10in}]{geometry}
%\usepackage{fancyhdr}
%\pagestyle{fancy}
\usepackage{amsmath, amsthm, mathpazo, epic, eepic, color, array}
\usepackage{amssymb}
%\usepackage{graphicx}
\usepackage{cancel}
\usepackage{pgfplots}
\usepackage{multicol}
\pgfplotsset{compat=1.13}
\usepackage{etoolbox}
\makeatletter
\patchcmd{\chapter}{\if@openright\cleardoublepage\else\clearpage\fi}{}{}{}
\makeatother
\usepackage{hyperref}
\usepackage[normalem]{ulem}

\usepackage{enumerate}
\usepackage{enumitem}

\usepackage{tikz}
\usetikzlibrary{positioning,chains,fit,shapes,calc,arrows,patterns}
\usepackage{tkz-graph}
\usetikzlibrary{arrows, petri, topaths}
\usepackage{tkz-berge}
\usepackage[all]{xy}
\usepackage{textcomp}

\newboolean{colorprint}
\setboolean{colorprint}{true}
%\setboolean{colorprint}{false}

\ifthenelse{\boolean{colorprint}}{%
\newcommand{\colorone}{blue}
\newcommand{\colortwo}{red}
\newcommand{\coloronefill}{blue!15!white}
\newcommand{\colortwofill}{red!15!white}
\newcommand{\colormapone}{rgb=(.4,.4,1); rgb=(.8,.8,1)}
\newcommand{\colormaptwo}{rgb=(1,.4,.4); rgb=(1,.8,.8)}
\newcommand{\colormapplaneone}{rgb=(.7,.7,1); rgb=(.9,.9,1)}
\definecolor{colormaponebottom}{rgb}{.4,.4,1}
\definecolor{colormaponetop}{rgb}{.8,.8,1}
\definecolor{colormaptwobottom}{rgb}{1,.4,.4}
\definecolor{colormaptwotop}{rgb}{1,.8,.8}
}% ends color
{% not color
\newcommand{\colorone}{black}
\newcommand{\colortwo}{black!50!white}
\newcommand{\coloronefill}{black!15!white}
\newcommand{\colortwofill}{black!05!white}
\newcommand{\colormapone}{rgb=(.4,.4,.4); rgb=(.7,.7,.7)}
\newcommand{\colormaptwo}{rgb=(.6,.6,.6); rgb=(.9,.9,.9)}
\newcommand{\colormapplaneone}{rgb=(.8,.8,.8); rgb=(.95,.95,.95)}
\definecolor{colormaponebottom}{rgb}{.4,.4,.4}
\definecolor{colormaponetop}{rgb}{.7,.7,.7}
\definecolor{colormaptwobottom}{rgb}{.6,.6,.6}
\definecolor{colormaptwotop}{rgb}{.9,.9,.9}
}%

\newlength\tindent
\setlength{\tindent}{\parindent}
\setlength{\parindent}{0pt}
\renewcommand{\indent}{\hspace*{\tindent}}

\pgfplotsset{my style/.append style={axis x line=middle, axis y line=
middle, xlabel={$x$}, ylabel={$y$}, axis equal }}

\pgfplotsset{compat=1.13}
\newcommand{\ds}{\displaystyle}
\begin{document}

\textbf{Section ?.3 Calculus \& Parametric Equations}\\

This is the section where Dave asked about Cycloids and area under parametricallly defined curves. Did you ask Dave about writing this and/or take this from Paul's notes?\\

\textbf{p. 610, line 1}\\
Add the word possible: "The possible points of inflection..."\\


\textbf{pp. 615-616, Exercises}\\

Here are additional Arc Length and Surface Area exercises.\\

In set of \#33-36 add the following exercises:\\

$\ds x=\cos t,~~~ y=\sin t$ on $\ds [0,2\pi]$\\

Answer: $4\pi$\\

$\ds x=1+3t^2,~~~ y=4+2t^3$ on $\ds [0,1]$\\

Answer: $4\sqrt{2} -2$\\

$\ds x=\frac{t}{1+t},~~~ y=\ln (1+t)$ on $\ds [0,2]$\\

Answer: $\ds -\frac{\sqrt{10}}{3} + \ln(3+\sqrt {10}) + \sqrt 2 - \ln (1+\sqrt 2)$\\

$\ds x=e^t-t,~~~ y=4e^{-t/2}$ on $\ds [-8,3]$\\

Answer: $4e^3 + 11 - e^{-8}$\\ \\


In set of \#41-44 add the following exercises:\\

$\ds x=a\cos^3 \theta,~~~ y=a\sin^3 \theta$ on $\ds [0,\pi/2]$ about the $x-$axis \\

Answer: $\frac{6\pi a^2}{5}$\\

$\ds x=t^3,~~~ y=t^2$ on $\ds [0,1]$ about the $x-$axis\\

Answer: $\frac{2\pi (247\sqrt{13}+64)}{1215}$\\

$\ds x=3t^2,~~~ y=2t^3$ on $\ds [0,5]$ about the $y-$axis\\

Answer: $\frac{24\pi (949\sqrt{26}+1)}{5}$\\



If cycloids and areas of parametric curves are added to this section then exercises for these will also need to be added.



\end{document}

