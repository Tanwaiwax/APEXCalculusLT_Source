\documentclass[10pt]{article}


%%
% HERE THERE BE DRAGONS
%%

\usepackage{ifthen}

\usepackage{lipsum}
\usepackage{pgfplots}
\pgfplotsset{colormap={coloronemap}{rgb=(.4,.4,1); rgb=(.8,.8,1)}}
\pgfplotsset{colormap={colortwomap}{rgb=(1,.4,.4); rgb=(1,.8,.8)}}
\usepackage{eso-pic,calc}
\usepackage[font=small]{caption}
\usepgfplotslibrary{external}
\usetikzlibrary{calc}
\usetikzlibrary{shadings}

\usepackage[h]{esvect}

\pgfplotsset{compat=1.8}

\usepackage[paperwidth=3in,textheight=10in,marginparwidth=150pt]{geometry}

\usepackage{amsmath}

\ifthenelse{\boolean{xetex}}%
	{\sffamily
	%%\usepackage{fontspec}
	\usepackage{mathspec}
	\setallmainfonts[Mapping=tex-text]{Calibri}
	\setmainfont[Mapping=tex-text]{Calibri}
	\setsansfont[Mapping=tex-text]{Calibri}
	\setmathsfont(Greek){[cmmi10]}}
	{Please compile with XeLaTeX}

\newboolean{colorprint}
\setboolean{colorprint}{true}
%\setboolean{colorprint}{false}

\ifthenelse{\boolean{colorprint}}{%
\newcommand{\colorone}{blue}
\newcommand{\colortwo}{red}
\newcommand{\coloronefill}{blue!15!white}
\newcommand{\colortwofill}{red!15!white}
\newcommand{\colormapone}{rgb=(.4,.4,1); rgb=(.8,.8,1)}
\newcommand{\colormaptwo}{rgb=(1,.4,.4); rgb=(1,.8,.8)}
\newcommand{\colormapplaneone}{rgb=(.7,.7,1); rgb=(.9,.9,1)}
\definecolor{colormaponebottom}{rgb}{.4,.4,1}
\definecolor{colormaponetop}{rgb}{.8,.8,1}
\definecolor{colormaptwobottom}{rgb}{1,.4,.4}
\definecolor{colormaptwotop}{rgb}{1,.8,.8}
}% ends color
{% not color
\newcommand{\colorone}{black}
\newcommand{\colortwo}{black!50!white}
\newcommand{\coloronefill}{black!15!white}
\newcommand{\colortwofill}{black!05!white}
\newcommand{\colormapone}{rgb=(.4,.4,.4); rgb=(.7,.7,.7)}
\newcommand{\colormaptwo}{rgb=(.6,.6,.6); rgb=(.9,.9,.9)}
\newcommand{\colormapplaneone}{rgb=(.8,.8,.8); rgb=(.95,.95,.95)}
\definecolor{colormaponebottom}{rgb}{.4,.4,.4}
\definecolor{colormaponetop}{rgb}{.7,.7,.7}
\definecolor{colormaptwobottom}{rgb}{.6,.6,.6}
\definecolor{colormaptwotop}{rgb}{.9,.9,.9}
}%
\newcommand{\la}{\left\langle}
\newcommand{\ra}{\right\rangle}
\newcommand{\dotp}[2]{\ensuremath{\vec #1 \cdot \vec #2}}
\newcommand{\proj}[2]{\ensuremath{\text{proj}_{\,\vec #2}{\,\vec #1}}}

\newcommand{\fp}{\ensuremath{f\,'}}

\DeclareMathOperator{\sech}{sech}
\DeclareMathOperator{\csch}{csch}

\newcommand{\threedlines}[4][]{\draw [dashed,#1] (axis cs: #2,#3,#4) -- (axis cs: #2,#3,0) -- (axis cs: #2,0,0)  (axis cs: #2,#3,0)--(axis cs:0,#3,0);}

\newcommand{\mydraw}{\draw (axis cs:0,0,0) -- (axis cs:1,1,0);}
\newcommand{\ds}{\displaystyle}


%% no more dragons.  type away %%

% prefer color one for the main graph, and color two for secondary lines

\begin{document}

\noindent\textbf{Figure 10.31, $x=t-1$, $y=t^2-3t-4$}\\

\begin{tikzpicture}
\begin{axis}[width=\marginparwidth+25pt,%
tick label style={font=\scriptsize},axis y line=middle,axis x line=middle,name=myplot,%
                        %x=.37\marginparwidth,
                        %y=.37\marginparwidth,
                        %xtick={-10,-5,5,10},
                        %minor x tick num=4,% 
%                       extra x ticks={.33},
%                       extra x tick labels={$1/3$},
                        %ytick={-10,-5,5,10},
                        %minor y tick num=4,%extra y ticks={-5,-3,...,7},%
                        ymin=-7.2,ymax=5.5,%
                        xmin=-3.75,xmax=4.5%
]

%\draw (axis cs:4,-1) node (A) {}
                        %(axis cs:4,3) node (B) {} ;
%\draw [->,>=latex] (A) parabola bend (axis cs:0,1) (B);
\addplot [thick,{\colorone}, smooth,domain=-1.5:4.5,] ({x-1},{x^2-3*x-4});
\draw [->,>=latex] (axis cs:-0.82, -4.51) -- (axis cs:-0.813, -4.53);
\draw [->,>=latex] (axis cs:2.42, -2.57) -- (axis cs:2.55, -2.13);

\filldraw (axis cs:-2,0) circle (1.5pt) node [above left] {\scriptsize $t=-1$};
\filldraw (axis cs:0,-6) circle (1.5pt) node [below left] {\scriptsize $t=1$};
\filldraw (axis cs:2,-4) circle (1.5pt) node [right] {\scriptsize $t=3$};
\filldraw (axis cs:3,0) circle (1.5pt) node [above right] {\scriptsize $t=4$};
%\filldraw (axis cs:1,0) circle (1.5pt) node [above right] {\scriptsize $t=\pi$};
                      
\end{axis}

\node [right] at (myplot.right of origin) {\scriptsize $x$};
\node [above] at (myplot.above origin) {\scriptsize $y$};
\end{tikzpicture}
\vskip .5 truein

\noindent\textbf{Figure 10.32, $x=3-t$, $y=t^2-5t$}\\

\begin{tikzpicture}
\begin{axis}[width=\marginparwidth+25pt,%
tick label style={font=\scriptsize},axis y line=middle,axis x line=middle,name=myplot,%
                        %x=.37\marginparwidth,
                        %y=.37\marginparwidth,
                        %xtick={-10,-5,5,10},
                        %minor x tick num=4,% 
%                       extra x ticks={.33},
%                       extra x tick labels={$1/3$},
                        %ytick={-10,-5,5,10},
                        %minor y tick num=4,%extra y ticks={-5,-3,...,7},%
                        ymin=-7.2,ymax=5.5,%
                        xmin=-3.75,xmax=4.5%
]

%\draw (axis cs:4,-1) node (A) {}
                        %(axis cs:4,3) node (B) {} ;
%\draw [->,>=latex] (A) parabola bend (axis cs:0,1) (B);
\addplot [thick,{\colorone}, smooth,domain=-.5:5.5,] ({3-x},{x^2-5*x});
\draw [->,>=latex] (axis cs:-1.20, -3.43) -- (axis cs:-1.21, -3.4);
\draw [->,>=latex] (axis cs:2.55, -2.13) -- (axis cs:2.42, -2.57);

\filldraw (axis cs:-2,0) circle (1.5pt) node [above left] {\scriptsize $t=5$};
\filldraw (axis cs:0,-6) circle (1.5pt) node [below left] {\scriptsize $t=3$};
\filldraw (axis cs:2,-4) circle (1.5pt) node [right] {\scriptsize $t=1$};
\filldraw (axis cs:3,0) circle (1.5pt) node [above right] {\scriptsize $t=0$};
%\filldraw (axis cs:1,0) circle (1.5pt) node [above right] {\scriptsize $t=\pi$};
                      
\end{axis}

\node [right] at (myplot.right of origin) {\scriptsize $x$};
\node [above] at (myplot.above origin) {\scriptsize $y$};
\end{tikzpicture}
\vskip .5 truein

\noindent\textbf{Figure 10.33, $x=\frac{t+1}{2}$, $y=\frac{1}{4}t^2-\frac{25}{4}$}\\

\begin{tikzpicture}
\begin{axis}[width=\marginparwidth+25pt,%
tick label style={font=\scriptsize},axis y line=middle,axis x line=middle,name=myplot,%
                        %x=.37\marginparwidth,
                        %y=.37\marginparwidth,
                        %xtick={-10,-5,5,10},
                        %minor x tick num=4,% 
%                       extra x ticks={.33},
%                       extra x tick labels={$1/3$},
                        %ytick={-10,-5,5,10},
                        %minor y tick num=4,%extra y ticks={-5,-3,...,7},%
                        ymin=-7.2,ymax=5.5,%
                        xmin=-3.75,xmax=4.5%
]

%\draw (axis cs:4,-1) node (A) {}
                        %(axis cs:4,3) node (B) {} ;
%\draw [->,>=latex] (A) parabola bend (axis cs:0,1) (B);
\addplot [thick,{\colorone}, smooth,domain=-5.75:5.75,] ({(x+1)/2},{.25*x^2-6.25});
\draw [->,>=latex] (axis cs:-0.82, -4.51) -- (axis cs:-0.813, -4.53);
\draw [->,>=latex] (axis cs:2.42, -2.57) -- (axis cs:2.55, -2.13);

\filldraw (axis cs:-2,0) circle (1.5pt) node [above left] {\scriptsize $t=-5$};
\filldraw (axis cs:0,-6) circle (1.5pt) node [below left] {\scriptsize $t=-1$};
\filldraw (axis cs:2,-4) circle (1.5pt) node [right] {\scriptsize $t=3$};
\filldraw (axis cs:3,0) circle (1.5pt) node [above right] {\scriptsize $t=5$};
%\filldraw (axis cs:1,0) circle (1.5pt) node [above right] {\scriptsize $t=\pi$};
                      
\end{axis}

\node [right] at (myplot.right of origin) {\scriptsize $x$};
\node [above] at (myplot.above origin) {\scriptsize $y$};
\end{tikzpicture}


\end{document}
