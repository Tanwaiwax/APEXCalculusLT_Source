\documentclass[11pt]{report}
\usepackage[letterpaper, total={6.5in, 10in}]{geometry}
%\usepackage{fancyhdr}
%\pagestyle{fancy}
\usepackage{amsmath, amsthm, mathpazo, epic, eepic, color, array}
\usepackage{amssymb}
%\usepackage{graphicx}
\usepackage{cancel}
\usepackage{pgfplots}
\usepackage{multicol}
\pgfplotsset{compat=1.13}
\usepackage{etoolbox}
\makeatletter
\patchcmd{\chapter}{\if@openright\cleardoublepage\else\clearpage\fi}{}{}{}
\makeatother
\usepackage{hyperref}
\usepackage[normalem]{ulem}

\usepackage{enumerate}
\usepackage{enumitem}

\usepackage{tikz}
\usetikzlibrary{positioning,chains,fit,shapes,calc,arrows,patterns}
\usepackage{tkz-graph}
\usetikzlibrary{arrows, petri, topaths}
\usepackage{tkz-berge}
\usepackage[all]{xy}
\usepackage{textcomp}

\newboolean{colorprint}
\setboolean{colorprint}{true}
%\setboolean{colorprint}{false}

\ifthenelse{\boolean{colorprint}}{%
\newcommand{\colorone}{blue}
\newcommand{\colortwo}{red}
\newcommand{\coloronefill}{blue!15!white}
\newcommand{\colortwofill}{red!15!white}
\newcommand{\colormapone}{rgb=(.4,.4,1); rgb=(.8,.8,1)}
\newcommand{\colormaptwo}{rgb=(1,.4,.4); rgb=(1,.8,.8)}
\newcommand{\colormapplaneone}{rgb=(.7,.7,1); rgb=(.9,.9,1)}
\definecolor{colormaponebottom}{rgb}{.4,.4,1}
\definecolor{colormaponetop}{rgb}{.8,.8,1}
\definecolor{colormaptwobottom}{rgb}{1,.4,.4}
\definecolor{colormaptwotop}{rgb}{1,.8,.8}
}% ends color
{% not color
\newcommand{\colorone}{black}
\newcommand{\colortwo}{black!50!white}
\newcommand{\coloronefill}{black!15!white}
\newcommand{\colortwofill}{black!05!white}
\newcommand{\colormapone}{rgb=(.4,.4,.4); rgb=(.7,.7,.7)}
\newcommand{\colormaptwo}{rgb=(.6,.6,.6); rgb=(.9,.9,.9)}
\newcommand{\colormapplaneone}{rgb=(.8,.8,.8); rgb=(.95,.95,.95)}
\definecolor{colormaponebottom}{rgb}{.4,.4,.4}
\definecolor{colormaponetop}{rgb}{.7,.7,.7}
\definecolor{colormaptwobottom}{rgb}{.6,.6,.6}
\definecolor{colormaptwotop}{rgb}{.9,.9,.9}
}%

\newlength\tindent
\setlength{\tindent}{\parindent}
\setlength{\parindent}{0pt}
\renewcommand{\indent}{\hspace*{\tindent}}

\pgfplotsset{my style/.append style={axis x line=middle, axis y line=
middle, xlabel={$x$}, ylabel={$y$}, axis equal }}

\pgfplotsset{compat=1.13}
\newcommand{\ds}{\displaystyle}
\begin{document}

\textbf{?.4 Introduction to Polar Coordinates}
\vskip .5 truecm

\textbf{pp. 617 - 618, Example 1 and paragraph right after Example 1}\\
Change format for naming points. Instead of $A=P(r,\theta)$ use $A(r,\theta)$\\


\textbf{p. 619, Example 2}\\
 SOLUTION to 2 part a. insert radians: $\theta = \tan^{-1} 2 \approx 1.11 ~\text{radians} \approx 63.43^\circ$.\\

\textbf{p. 621, Example 4}\\
2nd paragraph of the solution: change the point "Consider the point $P(0,2)$..." to $P(2,0)$ \\

On Figure 10.49 label a few of the the points shown in the table and plotted on the graph\\

\textbf{pp. 624 \& 626}\\
The word cardioid is misspelled as cardiod\\

\textbf{p. 625}\\
For the circles $r=a\cos \theta$ the caption should read "Centered on $x$-axis \& tangent to $y$-axis". Similarly for $r=a\sin \theta$ "Centered on $y$-axis \& tangent to $x$-axis"\\

\textbf{p. 628, line 3}\\
insert the word radians and change 2nd = to $\approx$: "approximately $\theta = 1.9106 ~\text{radians} \approx 109.47^\circ$...\\

\textbf{pp. 629 - 630, Exercises}\\

After \#23 insert the following:\\

\#24 $r=-4\sin(\theta),~~[0,\pi]$\\

\#25 $r=-2\cos(\theta),~~[0,\pi]$\\

\#26 $r=\frac{3}{2}\cos(\theta),~~[0,\pi]$\\

I'm hoping you can use the \#23 solution file to easily generate the solutions to these new problems. If this is more work than I imagine it to be please let me know.\\

After current \#31 insert the following:\\
$r=3\sin(\theta)$\\

Answer: $x^2 +(y-\frac{3}{2})^2 = \frac{9}{4}$\\

$r=-\frac{3}{2}\cos(\theta)$\\

Answer: $(x+\frac{3}{4})^2+y^2 = \frac{9}{16}$\\
\vskip .5 truein


\textbf{?.5 Calculus and Polar Functions}
\vskip .5 truecm

\textbf{p. 631, Key Idea 40}\\

$\ds \frac{dy}{dx} = \frac{\frac{dy}{d\theta}}{\frac{dx}{d\theta}}=$...\\ \\

\textbf{p. 632, Example 1}\\
Add rectangular: "1. Find the rectangular equations of the tangent and normal lines..." \\

Part 2 add note in parentheses: "To find the horizontal lines of tangency...$\frac{dy}{dx}$ is $0$ (when the denominator does not equal 0)."\\


Move Figure 10.54 to next page.\\

\textbf{p. 633, Example 1, part 2 solution}\\
Line 2 on p. 633 add note in parentheses: "by setting...$\frac{dy}{dx}=0$ (when the numerator does not equal 0)."\\

Line 6, the -1 should be -2: $4\sin^2 \theta + \sin \theta - 2 = 0$\\

All $\theta =$ followed by a decimal approximation should be $\theta \approx ...$\\

The decimal approximations associated with $\theta = \sin^{-1} (\frac{-1+\sqrt {33}}{8})$ are wrong. $\theta = 0.6399$ should be $\theta \approx 0.6349$ (this occurs twice) and $3.7815$ radians should be $2.5017$ radians.\\

\textbf{p. 633, last paragraph}\\
Replace the last paragraph with the following:\\
When the graph of the polar function $r=f(\theta)$ intersects the pole, it means that $f(\alpha) = 0$ for some angle $\alpha$. Making this substitution in the formula for $\frac{dy}{dx}$ given in Key Idea 40 we see
$$\frac{dy}{dx} = \frac{f'(\alpha)\sin \alpha + f(\alpha)\cos \alpha}{f'(\alpha)\cos \alpha + f(\alpha)\sin \alpha} = \frac{\sin \alpha}{\cos \alpha} = \tan \alpha$$

\textbf{p. 634, line -2}\\
"...The \sout{length} radian measure of..."\\

\textbf{p. 637, Example 5}\\
The $\frac{1}{2}$ got lost between step 2 and 3. This makes the answer $\pi$ instead of $2\pi$.\\

\textbf{p. 642, Exercises}\\

Keep \#15 the same\\

new \#16 $r=\cos 3\theta, ~~[0,\pi]$\\
Answers: At $(0,\frac{\pi}{6}):~ \theta = \frac{\pi}{6}$ or $y=\sqrt 3 x$\\
\indent At $(0,\frac{\pi}{2}):~ \theta = \frac{\pi}{2}$ or $x=0$\\
\indent At $(0,\frac{5\pi}{6}):~ \theta = -\frac{\pi}{6}$ or $y=-\sqrt 3 x$\\

new \#17 $r=\cos 2\theta,~~[0,2\pi]$\\
Answers: At $(0,\frac{\pi}{4})$ and $(0,\frac{5\pi}{4}):~ \theta = \frac{\pi}{4}$ or $y= x$\\
\indent At $(0,\frac{3\pi}{4})$ and $(0,\frac{7\pi}{4}):~ \theta = -\frac{\pi}{4}$ or $y= -x$\\

new \#18 $r=\sin 2\theta,~~[0,2\pi]$\\
Answers: At $(0,0), (0,\pi)$ and $(0,2\pi):~ \theta = 0$ or $y= 0$\\
\indent At $(0,\frac{\pi}{2})$ and $(0,\frac{3\pi}{2}):~ \theta = \frac{\pi}{2}$ or $x=0$\\

Change the old \#17 Enclosed by the circle: $r=4\sin \theta,~~\frac{\pi}{3} \leq \theta \leq \frac{2\pi}{3}$\\
Answer: $\frac{8\pi}{3}+4\sqrt 3$

\end{document}