\section*{11.2 \ \ More Problems}

\ 


\begin{enumerate}
\item  Let $P = (x_1, y_1, z_1)$ and $Q = (x_2, y_2, z_2)$ be points in space.  Let $M$ be the midpoint of $\overline{PQ}$. Explain why $\overrightarrow{PM} = \overrightarrow{MQ}$ and $\overrightarrow{PM} + \overrightarrow{MQ} = \overrightarrow{{PQ}}$.  Use these facts to find the coordinates of the point $M$ (again). \\

\item  Let $P = (1, 2, 3)$, $Q = (2, -1, 4)$, and $R = (-1, 6, 1)$ be the three corners of a parallelogram in space.  Find the possible locations of the fourth corner. \\

\item  Use vectors to show that the diagonals of a parallelogram bisect each other. \\

\item  A \emph{median} of a triangle is a line segment from a vertex to the midpoint of the opposite side.  Show that the three medians of a triangle intersect in a single point.  This point is called the \emph{centroid} of the triangle.  Show that the distance from a vertex to the centroid is two-thirds the length of the median from this vertex.  \\
\end{enumerate}