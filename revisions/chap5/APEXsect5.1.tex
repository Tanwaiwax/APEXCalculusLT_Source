\documentclass[11pt]{report}
\usepackage{mathtools, amsthm, mathpazo, epic, eepic, color, paralist}
\usepackage{tikz}
\usepackage[margin=1in]{geometry}

\newcommand{\typo}[4]{\item Typo: on line #2 of page #1: \emph{#3} should be \emph{#4}.}
\newcommand{\funcleft}[2]{\item Example #1 on page #2.}


\begin{document}

\chapter*{APEX Section 5.1 Changes}

{\slshape Note: throughout, a positive line number refers to a line that far from the top of the page, a negative line number refers to a line that far from the bottom of the page.}



\section*{Text}

\begin{enumerate} 

\item Move function names to the left in the following places:
\begin{itemize}
\funcleft{111}{195}
\funcleft{112}{195}
\funcleft{113}{200}
\end{itemize}

\item Add the following marginal note on page 191:

Recall from Definition 18 that $dx$ is any nonzero real number and $dy=f'(x)dx$.

\item Suggested reorganization of Theorem 35 on page 193:

First table:

\begin{tabular} {r l  r l}
1. & $\frac{d}{dx}(xf(x)=cf'(x)$ & 1. & $\int cf(x)\,dx = c\int f(x)\,dx$\\
2. & $\frac{d}{dx}(f(x)\pm g(x)=f'(x)\pm g'(x)$& 2. &$\int(f(x)\pm g(x))\,dx =\int f(x)\,dx\pm\int g(x)\,dx$\\
3. & $\frac{d}{dx}C=0$ & 3. & $\int 0\,dx=C$\\
\end{tabular}

Second table:

\begin{tabular} {r l  r l}
1. & $\frac{d}{dx}(x^n)=nx^{n-1}$ & 1. & $\int x^n\,dx =\frac{1}{n+1}x^{n+1}+C \quad (n\neq-1)$\\
2. & $\frac{d}{dx}(\ln x)=\frac{1}{x}$ & 2. & $\int\frac{1}{x}\,dx=\ln|x|+C$\\
3. & $\frac{d}{dx}(e^x)=e^x$ & 3. & $\int e^x\,dx=e^x+C$\\
4. &  $\frac{d}{dx}(\sin x)=\cos x$ & 4. & $\int \cos x\,dx=\sin x+C$\\
5. & $\frac{d}{dx}(\cos x)=-\sin x$ & 5. & $\int \sin x\,dx=-\cos x+C$\\
6. & $\frac{d}{dx}(\tan x)=\sec^2x$ & 6. & $\int \sec^2x\,dx=\tan x+C$\\
7. & $\frac{d}{dx}(\cot x)=-\csc^2x$ & 7. & $\int \csc^2x\,dx=-\cot x+C$\\
8. & $\frac{d}{dx}(\sec x)=\sec x \tan x$ & 8. & $\int \sec x\tan x\,dx=\sec x+C$\\
9. & $\frac{d}{dx}(\csc x)=-\csc x\cot x$ & 9. & $\int \csc x\cot x\,dx=-\csc x+C$\\
\end{tabular}

\typo{194}{-2}{infinite derivatives}{infinitely many derivatives}
\end{enumerate}

\section*{Problems}

\begin{enumerate}
\item Add to current problem section 8-26:
\begin{itemize}
\item $\displaystyle\int \frac{3}{x^4}\,dx$

\item $\displaystyle\int \frac{4x^5-7}{x^3}\,dt$

\item $\displaystyle\int \sqrt x^7\,dx$

\item $\displaystyle\int \frac{x^3-7x}{\sqrt x}\,du$
\end{itemize}

\item Add to current problem section 28-38:

\begin{itemize}
\item $\displaystyle f'(x)=\frac{-2}{x^3}$ and $f(1)=2$

\item $\displaystyle f'(x)=\frac{1}{\sqrt x}$ and $f(4)=0$
\end{itemize}

\item Add the following problems:

\begin{itemize}
\item An object is moving so that its velocity at time $t$ is given by $v(t)=3\sqrt t$. If the object was stationary at time $t=0$, find it's position $s(t)$ at time $t$.

\item A nickel dropped from the top of the North Dakota State Capital Building has acceleration $a(t)=-32$ ft/sec$^2$, initial velocity $v(0)=0$, and initial height $s(0)=241.67$ ft. Ignoring air resistance, how long will it take the nickel to hit the ground?

\item Given the graph of $f$ below, sketch the graph of the antiderivative $F$ of $f$ that passes through the origin. What do the graphs of the other antiderivatives of $f$ look like? %Note: only uncomment pictures when in correct file.
\end{itemize}

\item

%\begin{tikzpicture}
%\begin{axis}[width=\marginparwidth,tick label style={font=\scriptsize},minor x tick num=1, minor y tick num=1, axis y line=middle,axis x l%ine=middle,ymin=-1,ymax=5.9,xmin=-1,xmax=5.9,name=myplot]
%\addplot [{\colorone},domain=0:5,thick] {abs(x-1)+1};
%\filldraw [black] (axis cs:2,86) circle (1pt);
%\filldraw [black] (axis cs:3,6) circle (1pt);
%\end{axis}

%\node [right] at (myplot.right of origin) {\scriptsize $x$};
%\node [above] at (myplot.above origin) {\scriptsize $y$};
%\end{tikzpicture}

\item 
%\begin{tikzpicture}
%\begin{axis}[width=\marginparwidth,tick label style={font=\scriptsize},minor x tick num=1, minor y tick num=1, axis y line=middle,axis x %line=middle,ymin=-3,ymax=4.9,xmin=-1,xmax=5.9,name=myplot]
%\addplot [{\colorone},domain=1:2,thick] {2-x*x};
%\addplot [{\colorone},domain=2:5,thick] {((x*x)/4)-3};
%\filldraw [black] (axis cs:2,86) circle (1pt);
%\filldraw [black] (axis cs:3,6) circle (1pt);
%\end{axis}

%\node [right] at (myplot.right of origin) {\scriptsize $x$};
%\node [above] at (myplot.above origin) {\scriptsize $y$};
%\end{tikzpicture}

\end{enumerate}
\end{document}