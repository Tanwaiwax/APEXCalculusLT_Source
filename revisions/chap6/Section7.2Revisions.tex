\documentclass[10pt]{article}


\usepackage{ifthen}

\usepackage{lipsum}
\usepackage{pgfplots}

\usepackage{eso-pic,calc}
\usepackage[font=small]{caption}
\usepgfplotslibrary{external}
\usetikzlibrary{calc}
\usetikzlibrary{shadings}
\usepackage{tikz}
\usetikzlibrary{positioning,chains,fit,shapes,calc,arrows,patterns}
\usepackage{tkz-graph}
\usetikzlibrary{arrows, petri, topaths}
\usepackage{tkz-berge}
\usepackage[all]{xy}
\usepackage{textcomp}
\usepackage[h]{esvect}
\usepackage[normalem]{ulem}

\pgfplotsset{compat=1.8}
\usepackage{amssymb}

\usepackage{amsmath}

\newcommand{\ds}{\displaystyle}


\begin{document}
The following will replace the paragraph directly above Theorem 54 p 353.


Recognize that this is a Riemann Sum. By taking a limit (as the thickness of the slices goes to 0) we can find the volume exactly. 
 $$\text{Volume}=\lim_{n\to \infty} \sum_{i=1}^n A(x_i)\Delta x$$ with 
$\Delta x=\frac{b-a}{n}$ and $x_i=a+i\Delta x$. We recognize this as a definite integral. 

Top of p 355 insert perpendicular to the axis of rotation.

.... are disks (thin circles), perpendicular to the axis of rotation.  


Example 205 solution first two paragraphs:

A sketch can help us understand this problem. In Figure \ref{fig:disk1}(a) the curve $y=1/x$ is sketched along with the sample slice -- a disk -- at $x$ with radius $R(x)=1/x$. In Figure \ref{fig:disk1} (b) the whole solid is pictured, along with the sample slice. 

The volume of the sample slice shown in part (a) of the figure is approximately $\pi R(x_i)^2\Delta x$, where $R(x_i)$ is the radius of the disk shown and $\Delta x$ is the thickness of that slice. The radius $R(x_i)$ is the distance from the $x$-axis to the curve, hence $R(x_i) = 1/x_i$.

Solution to example 206 should start:

Since the axis of rotation is vertical, our perpendicular cross sections have thickness $\Delta y$ and radius $x=R(y)$. We need need to convert the function into a function of ......

Add text directly after example 206:

The previous two examples demonstrate how taking the same region and rotating it about two different axes will result in different solids and thus volumes.

In Figure 7.17 the caption should read:   Sketching the sample slice and solid in Example 207.


Add the following example before example 208:

Title: Finding volume with the Washer Method

Directions: Find the volume of the solid formed by rotating the region bounded by $y=x^2$ and $x=y^2$ about the $y$-axis.

Solution: A sketch of the region is given in Figure ??????. Rotating about the $y$-axis will produce cross sections in the shape of washers, as shown in Figure ?????; the complete solid is shown in  part (c). Since the axis of rotation is vertical, each radius is a function of $y$. The outside radius of this washer is $R(y)=\sqrt y$ and the inside radius is $r(y)=y^2$. As the region is bounded from $y=0$ to $y=1$, we integrate as follows to compute the volume. 
\begin{align*}
V&=\pi \int_0^1 \left((\sqrt y)^2-(y^2)^2\right) ~dy\\
&=\pi \int_0^1 y-y^4 ~dy\\
&=\pi\left[\frac{1}{2} y^2-\frac{1}{5} y^5\right] \Big|_0^1\\
&=\frac{3\pi}{10} \text{units}^3
\end{align*}


The first sentence in Example 208 solution should read:   The triangular region is sketched in Figure \ref{fig:wash2}(a); the sample slice is sketched in (b) and the full solid is drawn in (c). 


Add the following Text and example after example 208.

In the previous examples, the axis of rotation has either been the $x$ or $y$ axis. We will now consider a problem where the axis of rotation is some other horizontal line.

Title: Finding volume with the Washer Method

Directions:  Find the volume of the solid formed by rotating the region bounded by $y=\sqrt x$ and $y=x$ about $y=2$.

Solution:  Figure ????? shows the region we are rotating (a), a sample slice (b) and the full solid (c). The axis of rotation is horizontal so the radii must be functions of $x$. The radii is the distance from the axis of rotation to the curve so the outside radius of this washer is $R(x)=2-x$ and the inside radius is $r(x)=2-\sqrt x$. The region is bounded from  $x=0$ to $x=1$, thus the volume is
\begin{align*}
V&=\pi \int_0^1 \left( (2-x)^2-(2-\sqrt x)^2\right) ~dx\\
&=\pi \int_0^1 (4-4x+x^2)-(4-4\sqrt x+x)~dx\\
&=\pi \int_0^1 x^2 -5x+4\sqrt x ~dx\\
&=\pi \left[\frac13 x^3-\frac52 x^2+\frac83 x^{3/2}\right]\Big|_0^1\\
&=\frac{\pi}{2} \text{units}^3.
\end{align*}


Exercises:  Eliminate 6 and 10.

\end{document}