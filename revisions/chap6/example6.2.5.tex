\documentclass[10pt]{article}


%%
% HERE THERE BE DRAGONS
%%

\usepackage{ifthen}

\usepackage{lipsum}
\usepackage{pgfplots}
\pgfplotsset{colormap={coloronemap}{rgb=(.4,.4,1); rgb=(.8,.8,1)}}
\pgfplotsset{colormap={colortwomap}{rgb=(1,.4,.4); rgb=(1,.8,.8)}}
%\pgfplotsset{compat=1.3}
\usepackage{eso-pic,calc}
\usepackage[font=small]{caption}
\usepgfplotslibrary{external}
\usetikzlibrary{calc}
\usetikzlibrary{shadings}

\usepackage[h]{esvect}

\pgfplotsset{compat=1.8}

\usepackage[paperheight=11in,paperwidth=3in,%inner=1in,includeheadfoot,
textheight=10in,%textwidth=345pt,
marginparwidth=150pt]{geometry}
%% end detour
%%

\usepackage{amsmath}




\ifthenelse{\boolean{xetex}}%
	{\sffamily
	%%\usepackage{fontspec}
	\usepackage{mathspec}
	\setallmainfonts[Mapping=tex-text]{Calibri}
	\setmainfont[Mapping=tex-text]{Calibri}
	\setsansfont[Mapping=tex-text]{Calibri}
	\setmathsfont(Greek){[cmmi10]}}
	{Please compile with XeLaTeX}

\newboolean{colorprint}
\setboolean{colorprint}{true}
%\setboolean{colorprint}{false}

\ifthenelse{\boolean{colorprint}}{%
\newcommand{\colorone}{blue}
\newcommand{\colortwo}{red}
\newcommand{\coloronefill}{blue!15!white}
\newcommand{\colortwofill}{red!15!white}
\newcommand{\colormapone}{rgb=(.4,.4,1); rgb=(.8,.8,1)}
\newcommand{\colormaptwo}{rgb=(1,.4,.4); rgb=(1,.8,.8)}
\newcommand{\colormapplaneone}{rgb=(.7,.7,1); rgb=(.9,.9,1)}
\definecolor{colormaponebottom}{rgb}{.4,.4,1}
\definecolor{colormaponetop}{rgb}{.8,.8,1}
\definecolor{colormaptwobottom}{rgb}{1,.4,.4}
\definecolor{colormaptwotop}{rgb}{1,.8,.8}
}% ends color
{% not color
\newcommand{\colorone}{black}
\newcommand{\colortwo}{black!50!white}
\newcommand{\coloronefill}{black!15!white}
\newcommand{\colortwofill}{black!05!white}
\newcommand{\colormapone}{rgb=(.4,.4,.4); rgb=(.7,.7,.7)}
\newcommand{\colormaptwo}{rgb=(.6,.6,.6); rgb=(.9,.9,.9)}
\newcommand{\colormapplaneone}{rgb=(.8,.8,.8); rgb=(.95,.95,.95)}
\definecolor{colormaponebottom}{rgb}{.4,.4,.4}
\definecolor{colormaponetop}{rgb}{.7,.7,.7}
\definecolor{colormaptwobottom}{rgb}{.6,.6,.6}
\definecolor{colormaptwotop}{rgb}{.9,.9,.9}
}%
\newcommand{\la}{\left\langle}
\newcommand{\ra}{\right\rangle}
\newcommand{\dotp}[2]{\ensuremath{\vec #1 \cdot \vec #2}}
\newcommand{\proj}[2]{\ensuremath{\text{proj}_{\,\vec #2}{\,\vec #1}}}

\newcommand{\fp}{\ensuremath{f\,'}}

\DeclareMathOperator{\sech}{sech}
\DeclareMathOperator{\csch}{csch}

\newcommand{\threedlines}[4][]{\draw [dashed,#1] (axis cs: #2,#3,#4) -- (axis cs: #2,#3,0) -- (axis cs: #2,0,0)  (axis cs: #2,#3,0)--(axis cs:0,#3,0);}

\newcommand{\mydraw}{\draw (axis cs:0,0,0) -- (axis cs:1,1,0);}
\newcommand{\ds}{\displaystyle}


%% no more dragons.  type away %%

% prefer color one for the main graph, and color two for secondary lines

\begin{document}

  \begin{tikzpicture}
   \begin{axis}[width=\marginparwidth+25pt,
     tick label style={font=\scriptsize},axis y line=middle,
     axis x line=middle,name=myplot,axis on top,
     ymin=-.2,ymax=1.2,xmin=-.2,xmax=1.2]
    \draw[{\colorone},fill={\coloronefill}] plot[smooth,domain=0:1]
     (axis cs:\x,{1/(\x*\x+1)}) |-(axis cs:1,0)
      --(axis cs:1,0)--(axis cs:0,0)
      -- cycle;
    \draw [thick,{\colortwo}] (axis cs: .75,0) -- (axis cs:.75,.64);
    \draw (axis cs:.8,.85) node[above] {\scriptsize$y=\frac1{1+x^2}$};
   \end{axis}
   \node [right] at (myplot.right of origin) {\scriptsize $x$};
   \node [above] at (myplot.above origin) {\scriptsize $y$};
  \end{tikzpicture}

 \begin{tikzpicture}
   \begin{axis}[width=\marginparwidth+25pt,
     tick label style={font=\scriptsize},axis y line=middle,
     axis x line=middle,name=myplot,axis on top,
     ymin=-.2,ymax=1.2,xmin=-.1,xmax=2.2]
    \draw[{\colorone},fill={\coloronefill}] plot[smooth,domain=1:2] (axis cs:\x,1/\x) |- (axis cs:1,0) --(axis cs:2,0)--(axis cs:1,0)-- cycle;
    \draw [thick,{\colortwo}] (axis cs: 1.5,0) --
      node[pos=.5,right,color=black]{\scriptsize$R(x)$} (axis cs: 1.5,.667);
   \end{axis}
   \node [right] at (myplot.right of origin) {\scriptsize $x$};
   \node [above] at (myplot.above origin) {\scriptsize $y$};
  \end{tikzpicture}

\begin{tikzpicture}
\begin{axis}[width=\marginparwidth+25pt,
tick label style={font=\scriptsize},axis y line=middle,axis x line=middle,name=myplot,axis on top,
			xtick=\empty, 
			extra x ticks={1},
			ytick={1},
			ymin=-.2,ymax=1.2,
			xmin=-.1,xmax=1.2]

\addplot [{\coloronefill},fill={\coloronefill},domain=0:1] {sqrt(x)};
\addplot [{\coloronefill},fill={\coloronefill},domain=0:1] {x^2};
\addplot [{\colorone},smooth,thick,domain=0:1] {x^2};
\addplot [{\colorone},smooth,thick,domain=0:1,samples=100] {sqrt x};

\draw [thick,{\colortwo}] (axis cs: 0,.81) -- node[pos=.4,above,color=black]{\scriptsize $R(x)$}(axis cs:.9,.81);
\draw [thick,{\colortwo}] (axis cs: 0,.5) -- node[pos=.4,above,color=black]{\scriptsize $r(x)$} (axis cs:.25,.5);

\end{axis}

\node [right] at (myplot.right of origin) {\scriptsize $x$};
\node [above] at (myplot.above origin) {\scriptsize $y$};
\end{tikzpicture}


%\begin{tikzpicture}
%\begin{axis}%
%[
%width=\marginparwidth+25pt,%
%tick label style={font=\scriptsize},axis on top,
%			axis y line=none,
%%			axis x line=none,
%%%			axis z line=none,%
%			axis lines=center,
%			y dir=reverse,
%%			view={125}{30},
%			name=myplot,
%%			%x=.37\marginparwidth,
%%			%y=.37\marginparwidth,
%			xtick={1,2,3},
%			ztick=\empty,% 
%%			extra x ticks={1,3},
%%			extra x tick labels={$a$,$b$},
%%			ytick={.1,1},
%%			%minor y tick num=1,%extra y ticks={-5,-3,...,7},%
%			ymin=-3.2,ymax=3.2,%
%			xmin=-1.1,xmax=1.1,
%			zmin=-.1,zmax=1.1%
%]
%\addplot3[domain=.35:1,y domain=90:360, samples y=36,surf, shader=flat, colormap={mp2}{\colormaptwo}] ({.71*cos(y)*(x)},{.71*sin(y)*(x)},.5);
%\addplot3[domain=.35:1,y domain=0:180, samples y=36,surf, shader=flat, colormap={mp2}{\colormaptwo}] ({.71*sin(y)*(x)},{.71*cos(y)*(x)},.5);
%
%
%%\addplot3[domain=0:1,y domain=-180:180,surf,samples y=36,shader=flat] ({cos(y)*x},{sin(y)*x},{x^2});
%\addplot3[domain=0:1,samples y=0,thick,black] (x,0,{sqrt x});
%\addplot3[domain=0:1,samples y=0,thick,black] (x,0,{x^2});
%\addplot3[domain=0:1,y domain=0:1,surf,shader=flat,colormap={mp2}{\colormapone},opacity=.2] (x,0,{y*((sqrt x)-x^2)+x^2});
%
%
%\addplot3[domain=0:290,samples y=0,{\colortwo},thick,smooth,] ({.71*cos(x)},{.71*sin(x)},{.5});
%\addplot3[domain=290:360,samples y=0,{\colortwo},thick,smooth,dashed] ({.71*cos(x)},{.71*sin(x)},{.5});
%
%\addplot3[domain=-52:360,samples y=0,{\colortwo},thick,smooth,] ({.25*cos(x)},{.25*sin(x)},{.5});
%
%\end{axis}
%
%\node [right] at (myplot.right of origin)[shift={(-20pt,-15pt)}] {\scriptsize $x$};
%\node [above] at (myplot.above origin) [shift={(0,-17pt)}] {\scriptsize $y$};
%\end{tikzpicture}
%
%
%
%
%\begin{tikzpicture}         %%%%%%%%%%%%%%   Graph THREE %%%%%%%%%%%%
%\begin{axis}%
%[
%width=\marginparwidth+25pt,%
%tick label style={font=\scriptsize},axis on top,
%			axis y line=none,
%%			axis x line=none,
%%%			axis z line=none,%
%			axis lines=center,
%			y dir=reverse,
%%			view={125}{30},
%			name=myplot,
%%			%x=.37\marginparwidth,
%%			%y=.37\marginparwidth,
%			xtick={1},
%			ztick=\empty,% 
%%			extra x ticks={1,3},
%%			extra x tick labels={$a$,$b$},
%%			ytick={.1,1},
%%			%minor y tick num=1,%extra y ticks={-5,-3,...,7},%
%			ymin=-2.2,ymax=2.2,%
%			xmin=-1.1,xmax=1.1,
%			zmin=-.1,zmax=1.1%
%]
%\addplot3[domain=0:1,y domain=0:360, surf,colormap={mp2}{\colormaptwo},opacity=.8,faceted color=black!60,very thin,z buffer=sort,samples=15,samples y=20] ({sin(y)*((sqrt x))},{cos(y)*((sqrt x))},x);
%
%\addplot3[domain=0:1,y domain=0:360, surf,colormap={mp2}{\colormapplaneone},opacity=.5,faceted color=black!40,samples=15,samples y=36,very thin,z buffer=sort] ({sin(y)*(x^2)},{cos(y)*(x^2)},x);
%
%\addplot3[domain=0:1,samples y=0,black,smooth,dashed] (x,0,{sqrt x});
%
%\addplot3[domain=0:1,samples y=0,black,thick,smooth] (x,0,{x^2});
%
%\addplot3[domain=0:360,samples y=0,black,thick,smooth,] ({cos(x)},{sin(x)},1);
%
%\addplot3[domain=-.2:3.3,samples y=0, black, thick, smooth,dashed] (x,0,2);
%
%\end{axis}
%
%\node [right] at (myplot.right of origin)[shift={(-20pt,-15pt)}] {\scriptsize $x$};
%\node [above] at (myplot.above origin) [shift={(0,-17pt)}] {\scriptsize $y$};
%\end{tikzpicture}


\end{document}
