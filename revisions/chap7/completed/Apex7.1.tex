\section{Inverse Functions}

We say that two functions $f$ and $g$ are \emph{inverses} if $g(f(x))=x$ for all $x$ in the domain of $f$ and $f(g(x))=x$ for all $x$ in the domain of $g$. A function can only have an inverse if it is one-to-one, i.e. if we never have $f(x_1)=f(x_2)$ for different elements $x_1$ and $x_2$ of the domain. This is equivalent to saying that the graph of the function passes the horizontal line test. Functions that are not one-to-one may sometimes have partial inverses obtained by restricting the domain of the function.

If $f$ and $g$ are inverses, the domain of $g$ will be the range of $f$ and the range of $g$ will be the domain of $f$. The graphs of $f$ and $g$ will be reflections of each other across the line $y=x$ since $y=f(x)$ if and only if $x=g(y)$ (since the point $(y,x)$ is on the graph of $g$ whenever $(x,y)$ is on the graph of $f$.)  The inverse of $f$ is usually denoted $f^{-1}$, which should not be confused with the function $1/f(x)$.

To determine whether or not two functions are inverses, we check to see whether or not the composition of those functions is the identity, i.e. whether or not $g(f(x))=x$ for all $x$. 

\paragraph{Example.} Determine whether or not the following pairs of functions are inverses:
\begin{itemize}
\item[(a)] $f(x)=3x+1$; $\displaystyle g(x)=\frac{x-1}3$
\item [(b)]$f(x)=x^3+1$; $g(x)=x^{1/3}-1$
\end{itemize}

\paragraph{Solution.}
\begin{itemize}
\item [(a)] To check the composition we plug $f(x)$ in for $x$ in the definition of $g$ as follows: \[ g(f(x))=\frac{f(x)-1}3=\frac{(3x+1)-1}3=\frac{3x}3=x\] Since $g(f(x))=x$ for all $x$, $f$ and $g$ are inverses.
\item [(b)] If we try to proceed as beforewe find that: \[ g(f(x))=(f(x))^{1/3}-1 =(x^3+1)^{1/3}-1\] This doesn't seem to be the same as the identity function $x$. To verify this, we find a number $a$ in the domain of $f$ and show that $g(f(a))\neq a$ for that value. Let's try $x=1$. Since $f(1)=1^3+1=2$, we find that $g(f(1))=g(2)=2^{1/3}-1\approx 0.26$. Since $g(f(1))\neq 1$, these functions are not inverses.
\end{itemize}

To find the inverse $f^{-1}$ of a given function $y=f(x)$, we follow these steps (when possible):
\begin{enumerate}
\item In the equation defining $f$, solve for $x$.
\item Interchange $x$ and $y$.
\item The equation now defines $y=f^{-1}(x)$.
\end{enumerate}

\paragraph{Example.} Find the inverses of the following functions.
\begin{itemize}
\item [(a)] $f(x)=5x+3$
\item [(b)] $g(x)=x^3+1$
\end{itemize}

\paragraph{Solution.} 
\begin{itemize}
\item [(a)] Start with the equation $y=5x+3$ and solve for $x$:
\begin{equation*}
\begin{split}
y&=5x+3\\
y-3&=5x\\
\frac{y}5 &=x\\
\end{split}
\end{equation*}
Next interchange $x$ and $y$ to get $\displaystyle y=\frac{x-3}5$. The inverse of $f$ is: \[ f^{-1}(x)=\frac{x-3}5 \]
\item [(b)] Proceeding as before, we get:
\begin{equation*}
\begin{split}
y&=x^3+1\\
y-1&=x^3\\
(y-1)^{1/3}&=x\\
\end{split}
\end{equation*}
Interchanging $x$ and $y$ yields $y=(x-1)^{1/3}$, so \[g^{-1}(x)=(x-1)^{1/3}\]
\end{itemize}

Let's consider a familar function that is not one-to-one. Let $f(x)=x^2$. This function is not one-to-one because, for example, $f(-2)=f(2)=4$. If we were to try and find an inverse function $g$ for $f$, we encounter a problem. We want $g(f(2))=2$, so we should define $g(4)=2$. On the other hand, we also want $g(f(-2))=-2$, so we should define $g(4)=-2$. Both of these statements cannot be true since we want $g$ to be a function. However, if we restrict the domain of $f$ to the set of nonnegative real numbers, then the function will be one-to-one. The inverse of this function is the familar principle square root function $g(x)=sqrt x$. Since $-2$ is no longer in the domain of $f$, we don't need to worry about $g(f(-2))$ anymore. The function $g(x)=\sqrt x$ is called a partial inverse for $f(x)=x^2$ because it only acts as an inverse on part of the domain of $f$.

\paragraph{The inverse sine function.} We consider the function $f(x)=\sin x$, which is not one-to-one. A piece of the graph of $f$ is given below.

\begin{center}
\begin{tikzpicture}
\draw[<->,thick] (-3.2,0) -- (3.2,0) node[right] {$x$};
\draw[<->,thick] (0,-1.5) -- (0,1.5) node[above] {$y$};
\draw [domain=-3.2:3.2] plot (\x, {sin(\x r)});
\end{tikzpicture}
\end{center}

In order to find an appropriate restriction of the domain of $f$, we look for consecutive critical points where $f$ takes on its minimum and maximum values. In this case, we use the interval $[-\pi/2,\pi/2]$. The graph of $f$ over this interval is sketched below.

\begin{center}
\begin{tikzpicture}
\draw[<->,thick] (-2,0) -- (2,0) node[right] {$x$};
\draw[<->,thick] (0,-1.5) -- (0,1.5) node[above] {$y$};
\draw [domain=-pi/2:pi/2] plot (\x, {sin(\x r)});
\end{tikzpicture}
\end{center}

We define the inverse of $f$ on this restricted range by $y=\arcsin x$ if and only if $\sin y=x$ and $-\pi/2\leq y\leq \pi/2$. The graph is a reflection of the graph of $g$ across the line $y=x$:

\begin{center}
\begin{tikzpicture}
\draw[<->,thick] (-1.5,0) -- (1.5,0) node[right] {$x$};
\draw[<->,thick] (0,-2) -- (0,2) node[above] {$y$};
\draw [domain=-1:1] plot (\x, {rad(asin(\x))});
\end{tikzpicture}
\end{center}

\paragraph{The inverse tangent function.} Next we consider the function $g(x)=\tan x$, which is also not one-to-one. A piece of the graph of $g$ is given below.

\begin{center}
\begin{tikzpicture}
\draw[<->,thick] (-5,0) -- (5,0) node[right] {$x$};
\draw[<->,thick] (0,-3) -- (0,3) node[above] {$y$};
\draw [domain=-1.25:1.25] plot (\x, {tan(\x r)});
\draw [domain=-1.25:1.25] plot (\x-pi, {tan(\x r)});
\draw [domain=-1.25:1.25] plot (\x+pi, {tan(\x r)});
\end{tikzpicture}
\end{center}

In order to find an interval on which the function is one-to-one and on which the function takes on all values in the range, we use an interval between consecutive vertical asymptotes. Traditionally, the interval $(-\pi/2,\pi/2)$ is chosen. Note that we choose the open interval in this case because the function $g$ is not defined at the endpoints. So we define $y=\arctan x$ if and only if $\tan y=x$ and $-\pi/2< y<\pi/2$. The graph of $y=\arctan x$ is shown below. Note that the vertical asymptotes of the original function are reflected to become horizontal asymptotes of the inverse function.

\begin{center}
\begin{tikzpicture}
\draw[<->,thick] (-4,0) -- (4,0) node[right] {$x$};
\draw[<->,thick] (0,-2) -- (0,2) node[above] {$y$};
\draw [domain=-4:4] plot (\x, {rad(atan(\x))});
\end{tikzpicture}
\end{center}

{\bfseries Insert problems here.} Problems 1-3 and 5-8 from the previous section 2.7 fit here. In addition, add the following:
In exercises ??-??, find the inverse of the given function. Indicate how the domain should be restricted if necessary.
\begin{itemize}
\item $f(x)=7x-2$
\item $g(x)=\sqrt{9-x^2}$
\item $r(t)=t^2-6t+9$
\end{itemize}