
\section{Hyperbolic Funtions}

\begin{enumerate}
\item Relabel graphs in Figure 6.13 on page 313 to be clear that the areas are each $A=\frac\theta2$.

\item Insert before Key Idea 16:

{\slshape \paragraph{Example: The derivative of $f(x)=\cosh x$.} We use Definition 23 and the our previous differentiation formulas to find that:
\[\frac d{dx}\cosh x=\frac d{dx}\left( \frac{e^x+e^{-x}}2\right)=\frac{e^x-e^{-x}}2=\sinh x\]
The derivatives of the other hyperbolic functions can be found in a similar manner.
}
\item Insert before Key Idea 17: 

{\slshape Now let's consider the inverses of the hyperbolic functions.We begin with the function $f(x)=\sinh x$. Since $f'(x)=\cosh x>0$ for all real $x$, $f$ is increasing and must be one-to-one. We proceed as in Section 7.1:
\begin{equation*}
\begin{split}
y&=\frac{e^x-e^{-x}}2\\
2y&=e^x-e^{-x} \qquad\text{(now multiply by $e^x$)}\\
2ye^x&=e^{2x}-1 \qquad\text{(a quadratic form )}\\
\left(e^x\right)^2-2ye^x-1&=0 \qquad\text{(use the quadratic formula)}\\
e^x&=\frac{2y\pm\sqrt{4y^2-4}}2\\
e^x&=y\pm\sqrt{y^2+1} \qquad\text{(use the fact that $e^x>0$)}\\
e^x&=y+\sqrt{y^2+1}\\
x&=\ln(y+\sqrt{y^2+1})\\
\end{split}
\end{equation*}
Finally, interchange the variable to find that \[\sinh^{-1} x=\ln(x+\sqrt{x^2+1}).\] In a similar manner we find that the inverses of the other hyperbolic functions are given by:}
\end{enumerate}