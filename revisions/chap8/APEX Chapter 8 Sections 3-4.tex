\documentclass[11pt]{report}
\usepackage[letterpaper, total={6.5in, 10in}]{geometry}
%\usepackage{fancyhdr}
%\pagestyle{fancy}
\usepackage{amsmath, amsthm, mathpazo, epic, eepic, color, array}
\usepackage{amssymb}
%\usepackage{graphicx}
\usepackage{cancel}
\usepackage{pgfplots}
\usepackage{multicol}
\pgfplotsset{compat=1.13}
\usepackage{etoolbox}
\makeatletter
\patchcmd{\chapter}{\if@openright\cleardoublepage\else\clearpage\fi}{}{}{}
\makeatother
\usepackage{hyperref}
\usepackage[normalem]{ulem}

\usepackage{enumerate}
\usepackage{enumitem}

\usepackage{tikz}
\usetikzlibrary{positioning,chains,fit,shapes,calc,arrows,patterns}
\usepackage{tkz-graph}
\usetikzlibrary{arrows, petri, topaths}
\usepackage{tkz-berge}
\usepackage[all]{xy}
\usepackage{textcomp}

\newboolean{colorprint}
\setboolean{colorprint}{true}
%\setboolean{colorprint}{false}

\ifthenelse{\boolean{colorprint}}{%
\newcommand{\colorone}{blue}
\newcommand{\colortwo}{red}
\newcommand{\coloronefill}{blue!15!white}
\newcommand{\colortwofill}{red!15!white}
\newcommand{\colormapone}{rgb=(.4,.4,1); rgb=(.8,.8,1)}
\newcommand{\colormaptwo}{rgb=(1,.4,.4); rgb=(1,.8,.8)}
\newcommand{\colormapplaneone}{rgb=(.7,.7,1); rgb=(.9,.9,1)}
\definecolor{colormaponebottom}{rgb}{.4,.4,1}
\definecolor{colormaponetop}{rgb}{.8,.8,1}
\definecolor{colormaptwobottom}{rgb}{1,.4,.4}
\definecolor{colormaptwotop}{rgb}{1,.8,.8}
}% ends color
{% not color
\newcommand{\colorone}{black}
\newcommand{\colortwo}{black!50!white}
\newcommand{\coloronefill}{black!15!white}
\newcommand{\colortwofill}{black!05!white}
\newcommand{\colormapone}{rgb=(.4,.4,.4); rgb=(.7,.7,.7)}
\newcommand{\colormaptwo}{rgb=(.6,.6,.6); rgb=(.9,.9,.9)}
\newcommand{\colormapplaneone}{rgb=(.8,.8,.8); rgb=(.95,.95,.95)}
\definecolor{colormaponebottom}{rgb}{.4,.4,.4}
\definecolor{colormaponetop}{rgb}{.7,.7,.7}
\definecolor{colormaptwobottom}{rgb}{.6,.6,.6}
\definecolor{colormaptwotop}{rgb}{.9,.9,.9}
}%

\newlength\tindent
\setlength{\tindent}{\parindent}
\setlength{\parindent}{0pt}
\renewcommand{\indent}{\hspace*{\tindent}}

\pgfplotsset{my style/.append style={axis x line=middle, axis y line=
middle, xlabel={$x$}, ylabel={$y$}, axis equal }}

\pgfplotsset{compat=1.13}
\newcommand{\ds}{\displaystyle}
\begin{document}

\textbf{General note about Exercises}\\
I just noticed that in several of the sections of Chapter 8 the directions for the first set of problems starts on the same line as the "Problems" header. This messes with the spacing for that set of directions.
\vskip .5 truecm

\textbf{Section 8.3 Trigonometric Substitution}\\
\vskip .25 truecm

\textbf{p. 411, Example 2}\\
Last line of text: Replace "The reference triangle .... help" with "The lengths of the sides of the reference triangle in Figure 8.11 are determined by the Pythagorean Theorem.\\ \\

\textbf{p. 413, Example 3}\\
Last line of text: add note about Pythagorean Theorem "With $a = 1/2$, and ...we use the Pythagorean Theorem to determine the lengths of the sides of the reference triangle in Figure 8.12." \sout{shows that} \\ \\

\textbf{p. 414, Example 4}\\
Last two lines of text: add note about Pythagorean Theorem "We need to rewrite... of $x$. Using the Pythagorean Theorem we determine the lengths of the sides of the reference triangle \sout{found} in Figure 8.13. We have $\cot \theta...$" \\ \\

\textbf{p. 415, Example 6}\\
I think using $u$ for both substitutions is ok here. What do you think?
\vskip 1 truecm

\textbf{Section 8.4 Partial Fractions}\\
\vskip .25 truecm

\textbf{p. 422, Example 2}\\
Replace "Using the method described... in the equation:" with the following:\\
\indent Before solving for $A$ and $B$ using the method described in Key Idea 25 2(b) we note that the equations
$$\frac{1}{x^2-1} = \frac{A}{x-1} + \frac{B}{x+1} ~\text{ and }~  1 = A(x+1) + B(x-1)$$ are not equivalent. Only the second equation holds for all values of $x$, including $x=-1$ and $x=1$, by continuity of polynomials. Thus, we can choose values for $x$ that eliminate terms in the polynomial to solve for $A$ and $B$.\\

\textbf{p. 423, line 3 (still in Example 2)}: $A=\frac{1}{2}$ not $B$\\

\textbf{p. 423, Example 3}\\
Why do they number the polynomial equation $8.3$? We could number the rational equation as $8.1$ \& the polynomial as $8.2$ and reference them in the margin note as follows:\\

\textbf{Note:} Equations $8.1$ and $8.2$ are not equivalent for $x=1$ and $x=-2$. However, due to the continuity of polynomials we can let $x=1$ to simplify the right hand side to $A(1+2)^2 = 9A$. Since the left hand side is still $1$, we have $1=9A$. Hence, $A=1/9$.\\
Likewise, \sout{the equality holds} when $x=-2$... and solving for B.
 
We could also number the equations in Example 2 and use those references in the explanation I wrote for Example 2. What do you think?

\textbf{pp. 424 -425, Examples 4 \& 5}\\
Do we need to continue to note the continuity of the polynomial? Is there a short way to do this? Maybe something like "As in the previous examples we choose values of $x$ to eliminate terms in the polynomial."\\


\textbf{p. 427, Exercises}

Add the following:\\

\textbf{Insert after the current \#18}\\

$\displaystyle \int \frac{dx}{x^4-x^2}$\\

Solution: $\displaystyle \frac{1}{x} + \frac{1}{2} \ln \biggl|\frac{x-1}{x+1}\biggr| + C$\\

\textbf{Insert after the current \#21}\\ 

$\displaystyle \int \frac{1-x+2x^2-x^3}{x(x^2+1)^2}~dx$\\

Solution: $\displaystyle \ln |x| - \frac{1}{2} \ln (x^2+1) - \tan^{-1} x -\frac{1}{2(x^2 +1)} + C$\\

\textbf{Insert after the current \#23}\\ 

$\displaystyle \int \frac{x^3+x^2+2x+1}{(x^2+1)(x^2+2)}~dx$\\

Solution: $\displaystyle \frac{1}{2} \ln (x^2+1) - \frac{1}{\sqrt 2}\tan^{-1} \biggl(\frac{x}{\sqrt 2}\biggr) + C$\\



$\displaystyle \int \frac{x}{x^4+4x^2+3}~dx$\\

Solution: $\displaystyle -\frac{1}{4} \ln (x^2+3) + \frac{1}{4} \ln (x^2+1) + C =  \frac{1}{4} \ln  \frac{x^2+1}{x^2+3} + C$\\


$\displaystyle \int \frac{x-3}{(x^2+2x+4)^2}~dx$\\

Solution: $\displaystyle \frac{-1}{2(x^2+2x+4)} - \frac{2\sqrt 3}{9} \tan^{-1} 
\biggl(\frac{x+1}{\sqrt 3}\biggr) -\frac{2(x+1)}{3(x^2+2x+4)} + C$\\


\end{document}

