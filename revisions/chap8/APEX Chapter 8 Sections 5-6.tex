\documentclass[11pt]{report}
\usepackage[letterpaper, total={6.5in, 10in}]{geometry}
%\usepackage{fancyhdr}
%\pagestyle{fancy}
\usepackage{amsmath, amsthm, mathpazo, epic, eepic, color, array}
\usepackage{amssymb}
%\usepackage{graphicx}
\usepackage{cancel}
\usepackage{pgfplots}
\usepackage{multicol}
\pgfplotsset{compat=1.13}
\usepackage{etoolbox}
\makeatletter
\patchcmd{\chapter}{\if@openright\cleardoublepage\else\clearpage\fi}{}{}{}
\makeatother
\usepackage{hyperref}
\usepackage[normalem]{ulem}

\usepackage{enumerate}
\usepackage{enumitem}

\usepackage{tikz}
\usetikzlibrary{positioning,chains,fit,shapes,calc,arrows,patterns}
\usepackage{tkz-graph}
\usetikzlibrary{arrows, petri, topaths}
\usepackage{tkz-berge}
\usepackage[all]{xy}
\usepackage{textcomp}

\newboolean{colorprint}
\setboolean{colorprint}{true}
%\setboolean{colorprint}{false}

\ifthenelse{\boolean{colorprint}}{%
\newcommand{\colorone}{blue}
\newcommand{\colortwo}{red}
\newcommand{\coloronefill}{blue!15!white}
\newcommand{\colortwofill}{red!15!white}
\newcommand{\colormapone}{rgb=(.4,.4,1); rgb=(.8,.8,1)}
\newcommand{\colormaptwo}{rgb=(1,.4,.4); rgb=(1,.8,.8)}
\newcommand{\colormapplaneone}{rgb=(.7,.7,1); rgb=(.9,.9,1)}
\definecolor{colormaponebottom}{rgb}{.4,.4,1}
\definecolor{colormaponetop}{rgb}{.8,.8,1}
\definecolor{colormaptwobottom}{rgb}{1,.4,.4}
\definecolor{colormaptwotop}{rgb}{1,.8,.8}
}% ends color
{% not color
\newcommand{\colorone}{black}
\newcommand{\colortwo}{black!50!white}
\newcommand{\coloronefill}{black!15!white}
\newcommand{\colortwofill}{black!05!white}
\newcommand{\colormapone}{rgb=(.4,.4,.4); rgb=(.7,.7,.7)}
\newcommand{\colormaptwo}{rgb=(.6,.6,.6); rgb=(.9,.9,.9)}
\newcommand{\colormapplaneone}{rgb=(.8,.8,.8); rgb=(.95,.95,.95)}
\definecolor{colormaponebottom}{rgb}{.4,.4,.4}
\definecolor{colormaponetop}{rgb}{.7,.7,.7}
\definecolor{colormaptwobottom}{rgb}{.6,.6,.6}
\definecolor{colormaptwotop}{rgb}{.9,.9,.9}
}%

\newlength\tindent
\setlength{\tindent}{\parindent}
\setlength{\parindent}{0pt}
\renewcommand{\indent}{\hspace*{\tindent}}

\pgfplotsset{my style/.append style={axis x line=middle, axis y line=
middle, xlabel={$x$}, ylabel={$y$}, axis equal }}

\pgfplotsset{compat=1.13}
\newcommand{\ds}{\displaystyle}
\begin{document}


\textbf{Section 8.5 Integration Strategies}\\
\vskip .25 truecm

\textbf{p. 428 line -4}\\
Should be easiest instead of easies\\ \\
\textbf{p. 431, paragraph after \#6}\\
First sentence, last word should be rules instead of rule. \\ \\

\textbf{p. 414, Example 1, Solution 2}\\
2nd line of equations delete the - in front of $\int \frac{1}{u^5}~du$. \\ \\

\textbf{p. 435, Exercises}\\
Delete the "(solutions)" after header. The directions need to be bolded. 
\vskip 1 truecm

\textbf{Section 8.6 Improper Integrals}\\
\vskip .25 truecm

\textbf{p. 438}\\
Margin figure captions:\\
In each caption the word Example is hyphenated. A reviewer asked for the word "Example" to be moved to the second line so it doesn't have to be hyphenated. : \} \\

Example 1\\
Replace all $a$ and $b$ with $t$ in solution steps.\\

Example 1 Solution 2 paragraph after $=\infty$.\\
"... Compare the graphs in Figures... how the \sout{graph} values of... \sout{is} are noticeably larger than those of $f(x) = 1/x^2$...."\\


\textbf{p. 439, Example 2}\\
Replace all $b$ with $t$ in solution steps.\\ \\
Replace paragraph after solution steps with: "The $\ln 1 = 0$ and $1/t$ goes to 0, leaving $\displaystyle \lim_{t \to \infty} \frac{\ln t}{t}$ with l'H\^{o}pital's Rule. We have:" Change these $b$'s to $t$ also.\\ \\

\textbf{pp. 440, Example 3}\\
Solution 1 paragraph: "... asymptote at $x=0$. In some sense..."\\
Replace all $a$ with $t$ in solution steps.\\ \\

\textbf{p. 441}\\
Example 3 solution 2 second paragraph: "Clearly the area...negative\sout{!}... continued anyway to apply \sout{with} the Fundamental..."\\

In last line of solution 2 steps $\displaystyle \lim_{t \to 0^-} -\frac{1}{t}+ 1...$ (i.e. the -1 should be +1)\\

last line on the page: "Our first tool is \sout{to understand} knowing... \\ \\

\textbf{p. 442, Example 4}\\
Replace all $b$ with $t$ in solution steps.\\
In parapgraph right after solution steps: "When does this limit....less than $0$. \sout{: when $1-p<0 \implies 1<p$} This is true when $1-p < 0$ or when $1<p$."\\ \\

\textbf{p. 443, line 3}\\
Replace "to compare to" with "in comparisons".\\ \\

\textbf{p. 444, Example 6}\\
SOLUTION~~~~~"As $x$" gets large the square root of the quadratic \sout{inside the square root} function will..."\\ \\

\textbf{p. 445, line 3}\\
"The trouble is...function. \sout{To get rid of it...looks very much like 1/x.}"\\ \\
Replace the deleted text with:\\
We determine the limit by using a technique we learned in Calulus I:
$$\lim_{x \to \infty} \frac{x}{\sqrt{x^2+2x+5}}=\lim_{x \to \infty} \frac{\frac{x}{x}}{{\sqrt{\frac{x^2+2x+5}{x^2}}}}=\lim_{x \to \infty} \frac{1}{\sqrt{1+\frac{2}{x}+ \frac{5}{x^2}}}=1$$
Then continue with "Since we know that ..."\\

\textbf{p. 445, 4th paragraph from the bottom}\\
"This chapter has explored... We learned \sout{Substitution, which... of differentiation as well as} Integration by Parts...Product Rule" \textit{insert:} of differentiation. "We \textit{insert:} also learned..." \\

\textbf{p. 445, line -2}\\
Replace "a number of" with "several"\\ \\

\textbf{Exercises}\\
Add the following:\\

\textbf{Insert after the current \#17}\\

$\displaystyle \int_0^3 \frac{1}{x}~dx$\\

Solution: Diverges\\

$\displaystyle \int_2^5 \frac{dx}{\sqrt{x-2}}$\\

Solution: $2\sqrt 3$\\

$\displaystyle \int_1^9 \frac{dx}{\root 3 \of{9-x}}$\\

Solution: $6$\\

$\displaystyle \int_0^{\frac{\pi}{2}} \sec x~dx$\\

Solution: Diverges\\


\end{document}

