\documentclass[11pt]{report}
\usepackage[letterpaper, total={6.5in, 10in}]{geometry}
%\usepackage{fancyhdr}
%\pagestyle{fancy}
\usepackage{amsmath, amsthm, mathpazo, epic, eepic, color, array}
\usepackage{amssymb}
%\usepackage{graphicx}
\usepackage{cancel}
\usepackage{pgfplots}
\usepackage{multicol}
\pgfplotsset{compat=1.13}
\usepackage{etoolbox}
\makeatletter
\patchcmd{\chapter}{\if@openright\cleardoublepage\else\clearpage\fi}{}{}{}
\makeatother
\usepackage{hyperref}
\usepackage[normalem]{ulem}

\usepackage{enumerate}
\usepackage{enumitem}

\usepackage{tikz}
\usetikzlibrary{positioning,chains,fit,shapes,calc,arrows,patterns}
\usepackage{tkz-graph}
\usetikzlibrary{arrows, petri, topaths}
\usepackage{tkz-berge}
\usepackage[all]{xy}
\usepackage{textcomp}

\newboolean{colorprint}
\setboolean{colorprint}{true}
%\setboolean{colorprint}{false}

\ifthenelse{\boolean{colorprint}}{%
\newcommand{\colorone}{blue}
\newcommand{\colortwo}{red}
\newcommand{\coloronefill}{blue!15!white}
\newcommand{\colortwofill}{red!15!white}
\newcommand{\colormapone}{rgb=(.4,.4,1); rgb=(.8,.8,1)}
\newcommand{\colormaptwo}{rgb=(1,.4,.4); rgb=(1,.8,.8)}
\newcommand{\colormapplaneone}{rgb=(.7,.7,1); rgb=(.9,.9,1)}
\definecolor{colormaponebottom}{rgb}{.4,.4,1}
\definecolor{colormaponetop}{rgb}{.8,.8,1}
\definecolor{colormaptwobottom}{rgb}{1,.4,.4}
\definecolor{colormaptwotop}{rgb}{1,.8,.8}
}% ends color
{% not color
\newcommand{\colorone}{black}
\newcommand{\colortwo}{black!50!white}
\newcommand{\coloronefill}{black!15!white}
\newcommand{\colortwofill}{black!05!white}
\newcommand{\colormapone}{rgb=(.4,.4,.4); rgb=(.7,.7,.7)}
\newcommand{\colormaptwo}{rgb=(.6,.6,.6); rgb=(.9,.9,.9)}
\newcommand{\colormapplaneone}{rgb=(.8,.8,.8); rgb=(.95,.95,.95)}
\definecolor{colormaponebottom}{rgb}{.4,.4,.4}
\definecolor{colormaponetop}{rgb}{.7,.7,.7}
\definecolor{colormaptwobottom}{rgb}{.6,.6,.6}
\definecolor{colormaptwotop}{rgb}{.9,.9,.9}
}%

\newlength\tindent
\setlength{\tindent}{\parindent}
\setlength{\parindent}{0pt}
\renewcommand{\indent}{\hspace*{\tindent}}

\pgfplotsset{my style/.append style={axis x line=middle, axis y line=
middle, xlabel={$x$}, ylabel={$y$}, axis equal }}

\pgfplotsset{compat=1.13}
\newcommand{\ds}{\displaystyle}
\begin{document}


\textbf{OLD 6.2 Integration by Parts}\\
\vskip .5 truecm

All page numbers are from the original APEX text. If that's not helpful let me know how to better reference the location of the changes to be made.\\ \\


\textbf{p. 275, mid page}\

Using differential notation, we can write
\begin{center}
$$u' = \frac{du}{dx} \implies du = u' dx$$
$$v' = \frac{dv}{dx} \implies dv = v' dx$$.
\end{center}

Thus, the equation above can be written as follows:...\\


\textbf{p. 275, Theorem 48 box}\

...and applying FTC part 2 we have...\\

\textbf{p. 276, Example 159}.\\
...$v$ is an antiderivative of $\cos x$, \sout{. We choose} so $v= \sin x$. \\

\textbf{At the end of Example 159}\\
Replace "...Note how the..." with:\\ 
We have two important notes here: 1) notice how the antiderivative contains the product, $x\sin x$. This product is what makes integration by parts necessary. And 2) antidifferentiating $dv$ does result in $v+C$. The intermediate $+C$s are all added together and represented by one $+C$ in the final answer. 

\textbf{In examples 159 - 163}, change the order of all
\begin{align} 
v &= ? \cr
v &= <\text{appropriate function}>
\end{align} 
\indent to \
\begin{align}
dv &= <\text{appropriate function}> \cr 
v &= ?
\end{align}

After the paragraph, "In the example above, we chose $u = x$...which we knew how to do." insert:\\ \\
\textbf{If we had chosen} $u=\cos x$ and $dv= x$ then $du = -\sin x$ and $v = \frac{1}{2} x^2$ then \begin{center}

$$\int x\cos x~dx = \frac{1}{2} x^2 \cos x - \bigl(-\frac{1}{2}\bigr) \int  x^2 \sin x~dx$$
\end{center}
We then need to integragte $x^2 \sin x$, which is more complicated than our original integral, making this an unproductive choice. \\ \\
\textbf{Still on p. 276}\\
Cut mnemonic and all future references to it. I will try to point them all out.\\ \\

\textbf{p. 277, Example 160}.\\
Replace the mnemnic references to: Notice that $x$ becomes simpler when differentiatied and $e^x$ is unchanged by differentiation or integration. This suggests that we should let $u = x$ and $dv = e^x$:\\ \\
Cut "We see $du$ is simpler....This is good."\\ \\

\textbf{Example 161}\\
Replace "The mnemonic suggests" with "Let..." \\ \\

\textbf{p. 278, Example 162}\\
Cut "Our mnemonic suggests letting $u$ be the ... of the exponential"  and "to demonstrate... LIATE" (Tim, you may need to replace some "," with "." to make it read correctly.) \\ \\

\textbf{p. 279, Example 162}\\
After "The integral... to the new integral." Insert "So what should we use for $u$ and $dv$ this time? We may feel like letting the trigonometric function be $dv$ and the exponential be $dv$ was a bad choice last time since we still can't integrate the new integral. However, if we let $u = \sin x$ and $dv = e^x$ this time we will reverse what we just did, taking us back to the beginning. So, we let $u=e^x$..."\\ \\

\textbf{p. 280, Example 163}\\
clever application of...This is a good \sout{sneaky trick} strategy to learn... \\

Putting this all together...very nicely:
\begin{align}
\int \ln x ~dx &= x \ln x - \int x~ \frac{1}{x}~dx \cr
&= x\ln x - \int 1~dx \cr
&= x\ln x + C. \cr
\end{align}

\sout{The new integral...our answer is...+C}\\ \\

\textbf{p. 280, Example 164}\\
SOLUTION  The same strategy \sout{sneaky trick} we used...\\ \\

After the answer insert: Since $1 + x^2 > 0$ we do not need to include the absolute value in $\ln 1+x^2$ term.\\ \\  

\textbf{p. 281, Example 165}\\
Since $u = \ln x$ we can use inverse functions \sout{and conclude that} to solve for...\\ \\

\textbf{p. 282, Example 166}\\
SOLUTION \sout{Our mnemonic suggests letting} To simplify the integral we let $u = \ln x$ and $dv = x^2$. This may seem counterintuitive since the power on the algebraic factor has increased ($v=x^3/3$), but as we see below this is a wise choice:\\ \\

Delete the approximation of 1.07.\\ \\

\textbf{still on p. 282, last paragraph:} Replace "derivation" with "differentiation". \\ \\

\textbf{p. 283,first paragraph:} While the first is calculated easily...the third integral cannot be calculated with techniques in this chapter. We will learn how to evaluate this integral in Chapter <POWER SERIES>. \sout{has not answer ...exact answer}\\ \\

\textbf{p. 284,Exercises:}\\
Delete \#3\\

Add problem after current \#38 (in section on u-sub before integration by parts):
$$\int x^3 e^{x^2}~dx$$

\end{document}

