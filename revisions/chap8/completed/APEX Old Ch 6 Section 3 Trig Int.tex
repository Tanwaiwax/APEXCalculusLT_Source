\documentclass[11pt]{report}
\usepackage[letterpaper, total={6.5in, 10in}]{geometry}
%\usepackage{fancyhdr}
%\pagestyle{fancy}
\usepackage{amsmath, amsthm, mathpazo, epic, eepic, color, array}
\usepackage{amssymb}
%\usepackage{graphicx}
\usepackage{cancel}
\usepackage{pgfplots}
\usepackage{multicol}
\pgfplotsset{compat=1.13}
\usepackage{etoolbox}
\makeatletter
\patchcmd{\chapter}{\if@openright\cleardoublepage\else\clearpage\fi}{}{}{}
\makeatother
\usepackage{hyperref}
\usepackage[normalem]{ulem}

\usepackage{enumerate}
\usepackage{enumitem}

\usepackage{tikz}
\usetikzlibrary{positioning,chains,fit,shapes,calc,arrows,patterns}
\usepackage{tkz-graph}
\usetikzlibrary{arrows, petri, topaths}
\usepackage{tkz-berge}
\usepackage[all]{xy}
\usepackage{textcomp}

\newboolean{colorprint}
\setboolean{colorprint}{true}
%\setboolean{colorprint}{false}

\ifthenelse{\boolean{colorprint}}{%
\newcommand{\colorone}{blue}
\newcommand{\colortwo}{red}
\newcommand{\coloronefill}{blue!15!white}
\newcommand{\colortwofill}{red!15!white}
\newcommand{\colormapone}{rgb=(.4,.4,1); rgb=(.8,.8,1)}
\newcommand{\colormaptwo}{rgb=(1,.4,.4); rgb=(1,.8,.8)}
\newcommand{\colormapplaneone}{rgb=(.7,.7,1); rgb=(.9,.9,1)}
\definecolor{colormaponebottom}{rgb}{.4,.4,1}
\definecolor{colormaponetop}{rgb}{.8,.8,1}
\definecolor{colormaptwobottom}{rgb}{1,.4,.4}
\definecolor{colormaptwotop}{rgb}{1,.8,.8}
}% ends color
{% not color
\newcommand{\colorone}{black}
\newcommand{\colortwo}{black!50!white}
\newcommand{\coloronefill}{black!15!white}
\newcommand{\colortwofill}{black!05!white}
\newcommand{\colormapone}{rgb=(.4,.4,.4); rgb=(.7,.7,.7)}
\newcommand{\colormaptwo}{rgb=(.6,.6,.6); rgb=(.9,.9,.9)}
\newcommand{\colormapplaneone}{rgb=(.8,.8,.8); rgb=(.95,.95,.95)}
\definecolor{colormaponebottom}{rgb}{.4,.4,.4}
\definecolor{colormaponetop}{rgb}{.7,.7,.7}
\definecolor{colormaptwobottom}{rgb}{.6,.6,.6}
\definecolor{colormaptwotop}{rgb}{.9,.9,.9}
}%

\newlength\tindent
\setlength{\tindent}{\parindent}
\setlength{\parindent}{0pt}
\renewcommand{\indent}{\hspace*{\tindent}}

\pgfplotsset{my style/.append style={axis x line=middle, axis y line=
middle, xlabel={$x$}, ylabel={$y$}, axis equal }}

\pgfplotsset{compat=1.13}
\newcommand{\ds}{\displaystyle}
\begin{document}


\textbf{OLD 6.3 Trigonometric Integrals}\\
\vskip .25 truecm

All page numbers are from the original APEX text. If that's not helpful let me know how to better reference the location of the changes to be made.\\ \\

\textbf{p. 286}\\
Introduction paragraph\\
Replace first sentence with the following: Trigonometric functions are usedful for describing periodic behavior. \\ \\

Key Idea 11: In \#3 box replace "power-reducing" with "half-angle"\\ \\

\textbf{p. 287}\\

Examples 167 \& 168: replace the "term" (as in sine or cosine "term") with "factor".\\ \\

Example 167: \sout{This final integral... giving} and show solution vertically instead of horizontally. \\ \\

\textbf{p. 288}\\

Example 168 - The only way to get this last step on one line is to put in a line separating the step from the others. I even thought about using $\cos^7 x$ but I don't think that won't cut it down enough. So, let's try this: between the ingration with $u$ and replacing $u$ with $\sin x$ write "Replacing $u$ with $\sin x$ we have
$$\frac{1}{6} \sin^6 x - .... +C"$$
\\

\textbf{Insert an example between current \#168 \& \#169:}\\
\textbf{New \#169} Integrating powers of sine and cosine
Evaluate $\int \sin^2 x$~dx \\

SOLUTION~~~ The power of sine is even so we employ a half-angle identity, algebra and a $u$-substitution as follows:
\begin{align}
\int \sin^2 x~dx &= \int \frac{1-\cos (2x)}{2}~dx \cr
&= \frac{1}{2} \int 1 - \cos (2x) ~dx \cr
&= \frac{1}{2} \bigl(x - \frac{1}{2}\sin (2x)\bigr) +C \cr
&= \frac{1}{2}x - \frac{1}{4}\sin (2x) +C \cr
\end{align}

\textbf{p. 289}\\
Move "Integrals of the form $\int \sin (mx) \cos (nx)~dx$" to just after the section on "Integrals of the form $\int \tan^m x \sec^n x~dx$." \textbf{BUT before you do that,} in the first line on p. 290 insert "Trigonometry" as follows: "are best approached by....Formulas of Trigonometry found..."\\ \\

\textbf{p. 292, between Examples 171 \& 172 or in the margin next to Example 172}\\ 
These integrals for tangent and secant were previously derived and regularly appear when evaluating integrals of "this form" {OR "the form $\int \tan^m x \sec^n x~dx$" - whichever seems best}. Then put these formulas in a box:
$$\int \tan x~dx = \ln |\sec x|+C$$
$$\int \sec x~dx = \ln |\sec x+\tan x|+C$$
\\

Example \#172 change the order of $v$ and $dv$ like you did in Integration by Parts Section. \\ \\

\textbf{p. 293, Example 173}\\ 
Replace the text and steps on p. 293 that occur after $\ds \int \tan^4 x \sec^2 x~dx - \int \tan^4 x~dx$ with the following:\\
We integrate the first integral with substitution, $u=\tan x$ and $du = \sec^2~dx$; and the second by employing rule \#4 again.
\begin{align}
&= \int u^4 ~du - \int \tan^2 x \tan^2 x ~dx \cr
&= \frac{1}{5} u^5 - \int \tan^2 x (\sec^2 x - 1)~dx\cr
&= \frac{1}{5} \tan^5 x - \int \tan^2 x \sec^2 x~dx + \int \tan^2 x~dx\cr
\end{align}
Then continue with the steps on p. 294.\\ \\

Right after Example 173, and before the relocated $\int \sin (mx) \cos (nx)~dx$ section, insert the following section:\\
\textbf{Integrals of the form: $\int \cot^m x \csc^n x~dx$}\\ \\
Not surprisingly, evaluating integrals of the form $\int \cot^m x \csc^n x~dx$ is similar to evaluating $\int \tan^m x \sec^n x~dx$. The guidelines from Key Idea 12 and the following three facts will be useful:\\
$$\frac{d}{dx}(\cot x) = - \csc^2 x$$
$$\frac{d}{dx}(\csc x) = - \csc x \tan x, \text{and}$$
$$1+\cot^2 x = \csc^2 x$$ 

NEW \textbf{Example 174~~~Integrating powers of cotangent and cosecant}\\
Evaluate $\int \cot^2 x \csc^4 x~dx.$  \\
\textbf{SOLUTION}~~~Since the power of cosecant is even we will let $u=\cot x$ and save a $\csc^2 x$ for the resulting $du=-csc^2 x$.
\begin{align}
\int \cot^2 x \csc^4 x~dx &= \int \cot^2 x \cdot \csc^2 x \cdot \csc^2 x~dx \cr
&=\int \cot^2 x (1 + \cot^2 x) \csc^2 x~dx \cr
&=- \int u^2 (1 + u^2) ~du \cr  
\end{align}
The integration and substitution required to finish this example are similar to that of previous examples in this section. The result is\\
$$-\frac{1}{3} \cot^3 x - \frac{1}{5} \cot^5 x + C$$
\\ \\

\textbf{After relocated section for $\int \sin (mx) \cos (nx)~dx$ and before closing paragraphs on p. 294}\\
Insert the following section:\\ \\
\textbf{Integrating other combinations of trigonometric functions}\\
Combinations of trigonometric functions that we have not discussed in this chapter are evaluated by applying algebra, trigonometric identities and other integration strategies to create an equivalent integrand that we can evaluate. To evaluate "crazy" combinations, those not readily manipulated into a familiar form, one should use integral tables. A table of "common crazy" combinations can be found at the end of this text. \\ \\   


\textbf{p. 295,Exercises:}\\
Replace \#4 with $\int \cos^2 (x)~dx$\\

Answer: $\frac{1}{2}x + \frac{1}{4}\sin 2x + C$\\ \\

Replace \#5 with $\int \cos^4 (x)~dx$\\

Answer: $\frac{3}{4}x + \frac{1}{2}\sin 2x + C$\\ \\

Replace \#9 with $\int \cos^2 x \tan^3 x~dx$\\

Answer: $\frac{1}{2}\cos^2 x - \ln |\cos x| + C$\\ \\

Replace \#11 with $\int \sin^3 x \sqrt {\cos x} ~dx$\\

Answer: $\bigl(\frac{2}{7} \cos^3 x - \frac{2}{3} \cos x\bigr) \sqrt {\cos x} + C$\\ \\

Replace \#17 with $\int \tan^2 x~dx$\\

Answer: $\tan x - x +  C$\\ \\

Insert the following two problems after \#26\\ \\

$\int \csc x~dx$\\

Answer: $\ln |\csc x - \cot x| +  C$\\ \\

$\int \cot^3 x \csc^3 x~dx$\\

Answer: $\frac{1}{3} \csc^3 x - \frac{1}{5} \csc^5 x +  C$\\ \\

Delete \#15 \& 18\\ \\

Add one more exercise after \#33 $\displaystyle \int_{\frac{\pi}{6}}^{\frac{\pi}{2}} \cot^2 x~dx$\\

Answer: $\sqrt 3 - \frac{\pi}{3}$\\ \\



\end{document}

