\documentclass[11pt]{report}
\usepackage[letterpaper, total={6.5in, 10in}]{geometry}
%\usepackage{fancyhdr}
%\pagestyle{fancy}
\usepackage{amsmath, amsthm, mathpazo, epic, eepic, color, array}
\usepackage{amssymb}
%\usepackage{graphicx}
\usepackage{cancel}
\usepackage{pgfplots}
\usepackage{multicol}
\pgfplotsset{compat=1.13}
\usepackage{etoolbox}
\makeatletter
\patchcmd{\chapter}{\if@openright\cleardoublepage\else\clearpage\fi}{}{}{}
\makeatother
\usepackage{hyperref}
\usepackage[normalem]{ulem}

\usepackage{enumerate}
\usepackage{enumitem}

\usepackage{tikz}
\usetikzlibrary{positioning,chains,fit,shapes,calc,arrows,patterns}
\usepackage{tkz-graph}
\usetikzlibrary{arrows, petri, topaths}
\usepackage{tkz-berge}
\usepackage[all]{xy}
\usepackage{textcomp}

\newboolean{colorprint}
\setboolean{colorprint}{true}
%\setboolean{colorprint}{false}

\ifthenelse{\boolean{colorprint}}{%
\newcommand{\colorone}{blue}
\newcommand{\colortwo}{red}
\newcommand{\coloronefill}{blue!15!white}
\newcommand{\colortwofill}{red!15!white}
\newcommand{\colormapone}{rgb=(.4,.4,1); rgb=(.8,.8,1)}
\newcommand{\colormaptwo}{rgb=(1,.4,.4); rgb=(1,.8,.8)}
\newcommand{\colormapplaneone}{rgb=(.7,.7,1); rgb=(.9,.9,1)}
\definecolor{colormaponebottom}{rgb}{.4,.4,1}
\definecolor{colormaponetop}{rgb}{.8,.8,1}
\definecolor{colormaptwobottom}{rgb}{1,.4,.4}
\definecolor{colormaptwotop}{rgb}{1,.8,.8}
}% ends color
{% not color
\newcommand{\colorone}{black}
\newcommand{\colortwo}{black!50!white}
\newcommand{\coloronefill}{black!15!white}
\newcommand{\colortwofill}{black!05!white}
\newcommand{\colormapone}{rgb=(.4,.4,.4); rgb=(.7,.7,.7)}
\newcommand{\colormaptwo}{rgb=(.6,.6,.6); rgb=(.9,.9,.9)}
\newcommand{\colormapplaneone}{rgb=(.8,.8,.8); rgb=(.95,.95,.95)}
\definecolor{colormaponebottom}{rgb}{.4,.4,.4}
\definecolor{colormaponetop}{rgb}{.7,.7,.7}
\definecolor{colormaptwobottom}{rgb}{.6,.6,.6}
\definecolor{colormaptwotop}{rgb}{.9,.9,.9}
}%

\newlength\tindent
\setlength{\tindent}{\parindent}
\setlength{\parindent}{0pt}
\renewcommand{\indent}{\hspace*{\tindent}}

\pgfplotsset{my style/.append style={axis x line=middle, axis y line=
middle, xlabel={$x$}, ylabel={$y$}, axis equal }}

\pgfplotsset{compat=1.13}
\newcommand{\ds}{\displaystyle}
\begin{document}


\textbf{OLD 6.4 Trigonometric Integrals}\\
\vskip .25 truecm

All page numbers are from the original APEX text. If that's not helpful let me know how to better reference the location of the changes to be made.\\ \\

\textbf{p. 296}\\
Example 174, 2nd paragraph of solution: "... We also \sout{wish to} change our bound of integration..."\\

Last line: replace "power-reducing" with "half-angle"\\ \\

\textbf{p. 297}\\
Key Idea 13 box - my notes say that this needs to be reorganized to minimize memorization. Here are my thoughts on how to do that:\\
(a) delete "$dx = a\cos \theta~d\theta$. Thus $\theta = \sin^{-1} (x/a)$," \& move the triangle to the margin by example 177, and replace $a$ with $2$, as it is in the example.\\ \\
(b)  delete "$dx = a\sec^2 \theta~d\theta$. Thus $\theta = \tan^{-1} (x/a)$," \& move the triangle to the margin by example 175, and replace $a$ with $\sqrt 5$, as it is in the example.\\ \\
 delete "$dx = a\sec \theta \tan \theta ~d\theta$. Thus $\theta = \sec^{-1} (x/a)$," \& move the triangle to the margin by example 176, and replace $a$ with $1/2$, as it is in the example.\\ \\

\textbf{p. 298}\\
Example 175 replace the 2nd reference to "Key Idea 13(b)" with a reference to the figure number of the triangle moved to the margin.\\ \\

\textbf{p. 300}\\
Example 176 replace the reference to "Key Idea 13(c)" with a reference to the figure number of the triangle moved to the margin.\\ \\

Example 176, first equation on page 300 rewrite equation: $\displaystyle \tan \theta = \frac{\sqrt{x^2 - \frac{1}{4}}}{\frac{1}{2}} = 2\sqrt{x^2 - \frac{1}{4}}$...\\ What do you think about changing all $1/4$ to $\frac{1}{4}$? Do what looks best.\\ \\

Still in Example 176 add another form of final answer: $$=\frac{1}{4}\biggl(2x \sqrt{4x^2 -1} - \ln \bigl|2x + 2x \sqrt{4x^2 -1}\bigr|\biggr)+C$$

\textbf{p. 300}\\
Example 177 replace the 2nd reference to "Key Idea 13(a)" with a reference to the figure number of the triangle moved to the margin.\\ \\

\textbf{p. 302}\\
First line: replace "power reducing" with "half-angle".\\ \\
Insert a margin note next to the paragraph, "We need to return to the variable x..."\\
\textbf{Note:} Remember the sine and cosine double angle identities. \\ 
\begin{flalign}
\sin (2\theta) &= 2\sin \theta \cos \theta \cr
\text{and} \cr
\cos (2\theta) &= \cos^2 \theta - \sin^2 \theta \cr 
&= 2\cos^2 \theta - 1 \cr
&= 1 - 2\sin^2 \theta.
\end{flalign} 
They are often needed for writing your final answer in terms of $x$.\\ \\

\textbf{p. 303}
Delete Key Idea 14 and it's introductory sentence about "The following equalities..."\\ \\
Example 180 in first paragraph of solution "Using Key Idea 13(b)...As we substitute, we \sout{can also}..." \\ \\



\textbf{p. 304,Exercises:}\\

Delete \#s 6, 7, 8, 10, 18 \& 20\\ \\

Delete the "Note" from Exercises 27-32


\end{document}

