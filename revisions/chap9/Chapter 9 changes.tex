\documentclass[10pt]{article}


%%
% HERE THERE BE DRAGONS
%%

\usepackage{ifthen}

\usepackage{lipsum}
\usepackage{pgfplots}
\pgfplotsset{colormap={coloronemap}{rgb=(.4,.4,1); rgb=(.8,.8,1)}}
\pgfplotsset{colormap={colortwomap}{rgb=(1,.4,.4); rgb=(1,.8,.8)}}
%\pgfplotsset{compat=1.3}
\usepackage{eso-pic,calc}
\usepackage[font=small]{caption}
\usepgfplotslibrary{external}
\usetikzlibrary{calc}
\usetikzlibrary{shadings}

\usepackage[h]{esvect}

\pgfplotsset{compat=1.8}

\usepackage[paperheight=11in,paperwidth=6in,%inner=1in,includeheadfoot,
textheight=10in,%textwidth=345pt,
marginparwidth=150pt]{geometry}
%% end detour
%%

\usepackage{amsmath}




\ifthenelse{\boolean{xetex}}%
	{\sffamily
	%%\usepackage{fontspec}
	\usepackage{mathspec}
	\setallmainfonts[Mapping=tex-text]{Calibri}
	\setmainfont[Mapping=tex-text]{Calibri}
	\setsansfont[Mapping=tex-text]{Calibri}
	\setmathsfont(Greek){[cmmi10]}}
	{Please compile with XeLaTeX}

\newboolean{colorprint}
\setboolean{colorprint}{true}
%\setboolean{colorprint}{false}

\ifthenelse{\boolean{colorprint}}{%
\newcommand{\colorone}{blue}
\newcommand{\colortwo}{red}
\newcommand{\coloronefill}{blue!15!white}
\newcommand{\colortwofill}{red!15!white}
\newcommand{\colormapone}{rgb=(.4,.4,1); rgb=(.8,.8,1)}
\newcommand{\colormaptwo}{rgb=(1,.4,.4); rgb=(1,.8,.8)}
\newcommand{\colormapplaneone}{rgb=(.7,.7,1); rgb=(.9,.9,1)}
\definecolor{colormaponebottom}{rgb}{.4,.4,1}
\definecolor{colormaponetop}{rgb}{.8,.8,1}
\definecolor{colormaptwobottom}{rgb}{1,.4,.4}
\definecolor{colormaptwotop}{rgb}{1,.8,.8}
}% ends color
{% not color
\newcommand{\colorone}{black}
\newcommand{\colortwo}{black!50!white}
\newcommand{\coloronefill}{black!15!white}
\newcommand{\colortwofill}{black!05!white}
\newcommand{\colormapone}{rgb=(.4,.4,.4); rgb=(.7,.7,.7)}
\newcommand{\colormaptwo}{rgb=(.6,.6,.6); rgb=(.9,.9,.9)}
\newcommand{\colormapplaneone}{rgb=(.8,.8,.8); rgb=(.95,.95,.95)}
\definecolor{colormaponebottom}{rgb}{.4,.4,.4}
\definecolor{colormaponetop}{rgb}{.7,.7,.7}
\definecolor{colormaptwobottom}{rgb}{.6,.6,.6}
\definecolor{colormaptwotop}{rgb}{.9,.9,.9}
}%
\newcommand{\la}{\left\langle}
\newcommand{\ra}{\right\rangle}
\newcommand{\dotp}[2]{\ensuremath{\vec #1 \cdot \vec #2}}
\newcommand{\proj}[2]{\ensuremath{\text{proj}_{\,\vec #2}{\,\vec #1}}}

\newcommand{\fp}{\ensuremath{f\,'}}

\DeclareMathOperator{\sech}{sech}
\DeclareMathOperator{\csch}{csch}

\newcommand{\threedlines}[4][]{\draw [dashed,#1] (axis cs: #2,#3,#4) -- (axis cs: #2,#3,0) -- (axis cs: #2,0,0)  (axis cs: #2,#3,0)--(axis cs:0,#3,0);}

\newcommand{\mydraw}{\draw (axis cs:0,0,0) -- (axis cs:1,1,0);}
\newcommand{\ds}{\displaystyle}
\usepackage{multicol}

%% no more dragons.  type away %%

% prefer color one for the main graph, and color two for secondary lines

\begin{document}

%%%%%%   Section 9.1  
Example 9.1.1 Part 2 \\
Move text down to next line after sequence terms.\\

Figure 9.2 a) an $a_n$ is given and uses the variable x. x's should be changed to n's.

Change the first paragraph after Theorem 60 to read:\\

In Section 7.5 Example 3 part 1, we found that $\ds \lim_{x\to \infty} (1+1/x)^x=e$. If we consider the sequence $\ds \{b_n\}=\{(1+1/n)^n\}$, we see that $\ds \lim_{n\to \infty}=e$. Even though it may be difficult to intuitively grasp the behavior of this sequence, we know immediately that it is bounded.  \\ \\


Figure 9.5 is overlaping figure 9.6\\ \\

Add the following exercises to 9.1:\\
	To problems 17-36 add:\\

		$\ds \{a_n\}=\biggl\{\frac{\cos^2 n}{2^n}\biggr\}$		Solution: 0

		$\ds \{a_n\}=\biggl\{\frac{e^n+e^{-n}}{e^{2n}-1}\biggr\}$		Solution: 0

		$\ds \{a_n\}=\biggl\{\frac{\ln n}{\ln 2n}\biggr\}$		Solution: 1

	In the exercises of 9.1  \# 20, 22, 26 need the \{ \} around the sequence term\\

	To the problems 41-44 add:

		$\ds\{a_n\}=\biggl\{\cos \biggl(\frac{n\pi}{2}\biggr)\biggr\}$ 	Solution: not monotonic\\
		
		$\{a_n\}=\{ne^{-n}\}$		Solution: Decreasing



%%% Section 9.2
Figure 9.11 needs to move down.

In Proof of Theorem 64 on p 488, there needs to be a comma after $n\to \infty$. 

Add to Section 9.2 exercises: \\

To 14-26 add:\\

$\ds \sum_{i=1}^{\infty} \frac{\pi^n}{3^{n+1}}$ 		Solution: Diverges \\

$\ds \sum_{i=1}^{\infty} \frac{3^n+2^n}{6^{n}}$		Solution: $3/2$\\

$\ds \sum_{i=1}^{\infty} \frac{3}{n(n+1)}+\frac{5}{4^n}$	Solution: $7/2$

%%% Section 9.3

Add to the exercises a new group of problems with the following instructions:

In Exercises 12-14, find the value(s) of $p$ for which the series is convergent.

$\ds \sum_{i=2}^{\infty} \frac{1}{n(\ln n)^p}$		Solution: $p>1$

$\ds \sum_{i=1}^{\infty} n(1+n^2)^p$		Solution: $p<-1$

$\ds \sum_{i=1}^{\infty} \frac{\ln n}{n^p}$		Solution: $p>1$

$\ds \sum_{i=3}^{\infty} \frac{1}{n\ln n[\ln(\ln n)]}^p$	Solution: $p>1$


%%%  Section 9.4

In the exercises: move exercises 12-19 to the end of the problems.


%%%%%% Section 9.5   

Can you force the video box on page 508 to page 507 so is not so much blank space on page 507??

%%%%% Section 9.6

P 515 at the end of the proof of the Alternating Series Test put a box to end the proof.

p 519 at the end of the third paragraph the sentence should read:

The Riemann Rearrangement Theorem ...the sum is any desired value or infinity.


%%%%  Section 9.7 

We now consider several examples.

Example: Determine wheter the given series converges absolutely, converges conditionally, or diverges.

\begin{enumerate}
\item $\ds \sum_{n=2}^{\infty} \frac{(-1)^nn}{n^2+3}$

\item $\ds \sum_{n=1}^{\infty} \frac{n^2-3n}{4n^2-2n+1}$

\item $\ds \sum_{n=2}^{\infty} \frac{e^n}{(n+3)!}$

\end{enumerate}

Solutions:
\begin{enumerate}
\item We see that this series is alternating so we use the alternating series test. The underlying sequence is $\{a_n\}=\{\frac{n}{n^2+3}\}$ which is positive and decreasing since $a'(n)=\frac{3-n^2}{(n^2+3)^2}<0$ for $n\geq 2$. We also see $\ds \lim_{n\to \infty}\frac{n}{n^2+3}=0$ so by the Alternating Series Test $\ds \sum_{n=2}^{\infty} \frac{(-1)^nn}{n^2+3}$ converges. We now determine if it converges absolutely. Consider the sequence $\ds \sum_{n=2}^{\infty} \biggl|\frac{(-1)^nn}{n^2+3}\biggr|=\ds \sum_{n=2}^{\infty} \frac{n}{n^2+3}$. We compare this series to $\ds \sum_{n=2}^{\infty} \frac{n}{n^2}=\sum_{n=2}^{\infty} \frac{1}{n}$ which is a divergent $p$-series. We also have $\ds \frac{n}{n^2+3}>\frac{n}{n^2}=\frac{1}{n}$ so by the Comparison test, $\ds \sum_{n=2}^{\infty} \frac{n}{n^2+3}$ diverges. Therefore,  $\ds \sum_{n=2}^{\infty} \frac{(-1)^nn}{n^2+3}$ converges conditionally.

\item 
$\ds \lim_{n\to\infty} \frac{n^2-3n}{4n^2-2n+1}=\frac{1}{4}$ so by the Test for Divergence $\ds \sum_{n=1}^{\infty} \frac{n^2-3n}{4n^2-2n+1}$ diverges.

\item 
We see the factorial and use the Ratio Test. All terms of the series are positive so we consider 
\begin{align*}
\lim_{n\to\infty} \frac{a_{n+1}}{a_n}&=\lim_{n\to\infty} \frac{\frac{e^{n+1}}{(n+4)!}}{\frac{e^{n}}{(n+3)!}}\\
&=\lim_{n\to\infty}\frac{e^{n+1}(n+3)!}{e^n(n+4)!}\\
&=\lim_{n\to\infty}\frac{e\cdot e^n(n+3)!}{e^n(n+4)(n+3)!}\\
&=\lim_{n\to\infty}\frac{e}{n+4}=0<1
\end{align*}

So by the Ratio Test, $\ds \sum_{n=2}^{\infty} \frac{e^n}{(n+3)!}$ converges. Because all of the series terms are positive it converges absolutely.


\end{enumerate}

\end{document}

