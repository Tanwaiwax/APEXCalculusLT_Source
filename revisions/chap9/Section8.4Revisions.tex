\documentclass[10pt]{article}


\usepackage{ifthen}

\usepackage{lipsum}
\usepackage{pgfplots}

\usepackage{eso-pic,calc}
\usepackage[font=small]{caption}
\usepgfplotslibrary{external}
\usetikzlibrary{calc}
\usetikzlibrary{shadings}
\usepackage{tikz}
\usetikzlibrary{positioning,chains,fit,shapes,calc,arrows,patterns}
\usepackage{tkz-graph}
\usetikzlibrary{arrows, petri, topaths}
\usepackage{tkz-berge}
\usepackage[all]{xy}
\usepackage{textcomp}
\usepackage[h]{esvect}
\usepackage[normalem]{ulem}

\pgfplotsset{compat=1.8}
\usepackage{amssymb}

\usepackage{amsmath}

\newcommand{\ds}{\displaystyle}


\begin{document}

The first sentence will be changed to :\\
	Theorem \ref{ %Whatever THeorem 63 in Section 8.2 got named
} states that if a series $\ds\sum_{n=1}^\infty a_n$ converges, then $\ds \lim_{n\to \infty} a_n =0$.  



After the Ratio test on p 435 include a link to a proof of it : url for pauls online math notes
http://tutorial.math.lamar.edu/Classes/CalcII/RatioTest.aspx



Add to exercises 5-14:

$\ds\sum_{n=1}^\infty e^{-n}n!$		Solution: Diverges

$\ds \sum_{n=1}^\infty \frac{e^{1/n}}{n^3}$		Solution: Converges





\end{document}