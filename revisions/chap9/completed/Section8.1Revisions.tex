\documentclass[10pt]{article}


\usepackage{ifthen}

\usepackage{lipsum}
\usepackage{pgfplots}

\usepackage{eso-pic,calc}
\usepackage[font=small]{caption}
\usepgfplotslibrary{external}
\usetikzlibrary{calc}
\usetikzlibrary{shadings}
\usepackage{tikz}
\usetikzlibrary{positioning,chains,fit,shapes,calc,arrows,patterns}
\usepackage{tkz-graph}
\usetikzlibrary{arrows, petri, topaths}
\usepackage{tkz-berge}
\usepackage[all]{xy}
\usepackage{textcomp}
\usepackage[h]{esvect}
\usepackage[normalem]{ulem}

\pgfplotsset{compat=1.8}
\usepackage{amssymb}

\usepackage{amsmath}

\newcommand{\ds}{\displaystyle}


\begin{document}



p. 399 \\   
	%%%%%% Tim-you asked for another sequence representation here. I did find OEIS and found other representations, however the closed forms are very complicated and I think they will confuse students. I think that the intro to the solution does a sufficient job of explaining that there is more than one representation. Stewart makes no mention of different representations.
	
	Solution part 2. the last second to last line should read:   That is   $2=1^2+1$   \\ \\
 %%%%The exponent was incorrect%%%

p. 401 Solution 2. Last line should read: \\
	Based on the graph, we suspect that $\ds \lim_{n\to\infty} a_n$ does not exist, but we have not decisively proven it yet.   \\ \\

p 401 add proof after Theorem 56.\\
	Proof:    \\
We know $-|a_n|\leq a_n\leq |a_n|$ and $\ds \lim_{n\to \infty} (-|a_n|)=-\lim_{n\to\infty} |a_n|=0$. Thus by the Squeeze Theorem $\ds \lim_{n\to\infty} a_n =0$. \\ \\



p 403: Example 234 \\
	Delete the series $a_n$ , $b_n$, and $c_n$ and only give the limits.   \\ \\

p 404  The last sentence on the page should start:\\
This logic suggests ...  \\ \\

p407  Delete exclamation point that comes at the end of the second line of the paragraph before Theorem 59. \\ \\

	In Theorem 59 Deleet part 2 and 3 and add the following text directly after: \\

	Convergence of a sequence does not depend on the first $N$ terms of a sequence. \\ \\ %%%TIM-Do you have an example to insert here?%%%% 

Add to the 8.1 Exercises: \\ \\

%%%%% I have two questions to add to the block 17-28. They can be anywhere, just NOT back to back as they are both factorial problems. 

$\ds {a_n}=\frac{(n-3)!}{(n+1)!}$  	    Solution:  Converges to 0 \\ \\


$\ds {a_n}=\frac{(2n+1)!}{(2n-1)!}$       Solution: Diverges


%%%%%%%%%%%  NEW MATERIAL to ADD %%%%%%%%%%%%%%%%%%%%%%%

On p 403 right after the solution to Example 234 add the following theorem and proof.

Theorem:  The sequence $\{ r^n\}$ is convergent if $-1<r\leq 1$ and divergent for all other values of $r$ and $$\lim_{n\to \infty} r^n=\begin{cases} 
0&  -1<r<1\\
1& r=1
\end{cases}$$

Proof:
We can see from Theorem(Something in Chapter ? about exponential functions) and by letting $a=r$ that $$\lim_{n\to \infty} r^n 
\begin{cases}
\infty &  r>1\\
0 & 0<r<1
\end{cases}$$. We also know that $\ds  \lim_{x\to \infty} 1^n=1$, $\ds  \lim_{x\to \infty} 0^n=0$,  and $\ds  \lim_{x\to \infty} (-1)^n$ does not exist. If $-1<r<0$, we know $0<|r|<1$ so $\ds \lim_{x\to \infty} |r^n|=\lim_{x\to \infty} |r|^n=0$ and thus by Theorem \ref{thm:abs_val_seq},$\ds \lim_{x\to \infty} r^n=0$. Therefore, the sequence $\{ r^n\}$ is convergent if $-1<r\leq 1$ and divergent for all other values of $r$.

%%%%%%%%%%%%%%%%%% Additional changes to exercises %%%%%%%%%%%%%%%%%%%

Delete \#30  $a_n=tan n$ and \#33 $a_n=n cos n$ \\ \\

Add to the 12-28 block of questions: \\

$\ds {a_n}=2+\frac{9^n}{8^n}$		Solution: Diverges

$\ds {a_n}=\frac{6^{n+3}}{8^n}$  		Solution: Converges to 0

\end{document}

