\documentclass[12 pt]{amsart}
\include{amsfonts}
\usepackage[text={6.5in,9in},centering]{geometry}
\usepackage{tikz}
\usepackage{tikz-3dplot}


\setlength{\parskip}{.5\baselineskip}
\setlength{\parindent}{0cm} % Default is 15pt.




\begin{document}

\begin{enumerate}
    \item  \emph{p.942, line 9.}  ``the one is well...'' should be ``the\textbf{n} one is well...''  \\
    
    \item  \emph{p.944, paragraphs 4, 5, and 6.}  The argument given in these paragraphs is either wrong or opaque.  Instead we might argue as follows.
    
    \ 
    
    Suppose our curve $C$ has a smooth parameterization $\vec{r}(t) = \langle g(t), h(t) \rangle$ where $a \leq t \leq b$.  Then the arc-length parameter $s$ is given by
    \[
    s = \alpha(t) = \int_a^t \| r'(u) \|\,du.
    \]
    Since $\dfrac{ds}{dt} = \| r'(t) \| > 0$, $s = \alpha(t)$ is a strictly increasing function, and so it has an inverse $t = \beta(s)$.  Now $f$ can be regarded as a function of $s$ by $f(s) = f(g(\beta(s), h(\beta(s))$ where $0 \leq s \leq L = \alpha(b)$.  By definition we have
    \begin{eqnarray*}
    \int_C f(s)\, ds & = & \int_0^L f(s)\, ds  \\
     & = & \int_0^L f(\,g(\beta(s)), h(\beta(s))\,)\, ds.
    \end{eqnarray*}
    If we now make the reverse substitution $s = \alpha(t)$ we have 
    \[
    \int_0^L f(\,g(\beta(s)), h(\beta(s))\,)\, ds 
     = \int_a^b f(g(t), h(t))\,\|r'(t)\|\,dt
    \]
    because $\alpha(a) = 0$, $\alpha(b) = L$, $\beta(\alpha(t)) = t$, and $ds = \| r'(t) \| \, dt$.
    
    \ \\
    
    \item  \emph{p.A.18, answer to Exercise 15.1 \#17.}  Should be $(17\sqrt{17} - 5\sqrt{5})/3$.  \\
    
    \item  \emph{p.961, exercise \#25.}  This should be $\text{div} (\nabla f \times \nabla g)$.  \\
    
    \item  \emph{p.963, after Definition 15.3.1.} We should probably include a paragraph like the following. 
    
    \ 
    
    There is an older (and still frequently used) notation for the line integral of a vector field.  If $\vec{F} = \langle M(x, y), N(x, y) \rangle$ we often write
    \[
    \int_C M(x,y)\,dx + N(x,y)\,dy
    \]
    for the line integral of $\vec{F}$ along $C$.  Similarly if $\vec{F} = \langle M(x,y,z), N(x,y,z), P(x,y,z) \rangle$ we might write
    \[
    \int_C M(x,y,z)\,dx + N(x,y,z)\,dy + P(x,y,z)\,dz
    \]
    for the line integral.  For example
    \[
    \int_C x^2y\,dx + (x - y)\,dy
    \]
    represents the line integral of $\vec{F} = \langle x^2y, x-y \rangle$ along the curve $C$.  \\
    
    \item  \emph{p.968, first full paragraph.}  We should probably replace ``A region'' in the first sentence by ``A connected region''.  \\
    
    
%    \item  Next thing.
%   Sent the above to Tim on 7 May 2019
\end{enumerate}




\end{document}