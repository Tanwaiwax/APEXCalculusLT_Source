\documentclass[11pt]{report}
\usepackage[letterpaper, total={6.5in, 10in}]{geometry}
%\usepackage{fancyhdr}
%\pagestyle{fancy}
\usepackage{amsmath, amsthm, mathpazo, epic, eepic, color, array}
\usepackage{amssymb}
%\usepackage{graphicx}
\usepackage{cancel}
\usepackage{pgfplots}
\usepackage{multicol}
\pgfplotsset{compat=1.13}
\usepackage{etoolbox}
\makeatletter
\patchcmd{\chapter}{\if@openright\cleardoublepage\else\clearpage\fi}{}{}{}
\makeatother
\usepackage{hyperref}
\usepackage[normalem]{ulem}

\usepackage{enumerate}
\usepackage{enumitem}

\usepackage{tikz}
\usetikzlibrary{positioning,chains,fit,shapes,calc,arrows,patterns}
\usepackage{tkz-graph}
\usetikzlibrary{arrows, petri, topaths}
\usepackage{tkz-berge}
\usepackage[all]{xy}
\usepackage{textcomp}

\newboolean{colorprint}
\setboolean{colorprint}{true}
%\setboolean{colorprint}{false}

\ifthenelse{\boolean{colorprint}}{%
\newcommand{\colorone}{blue}
\newcommand{\colortwo}{red}
\newcommand{\coloronefill}{blue!15!white}
\newcommand{\colortwofill}{red!15!white}
\newcommand{\colormapone}{rgb=(.4,.4,1); rgb=(.8,.8,1)}
\newcommand{\colormaptwo}{rgb=(1,.4,.4); rgb=(1,.8,.8)}
\newcommand{\colormapplaneone}{rgb=(.7,.7,1); rgb=(.9,.9,1)}
\definecolor{colormaponebottom}{rgb}{.4,.4,1}
\definecolor{colormaponetop}{rgb}{.8,.8,1}
\definecolor{colormaptwobottom}{rgb}{1,.4,.4}
\definecolor{colormaptwotop}{rgb}{1,.8,.8}
}% ends color
{% not color
\newcommand{\colorone}{black}
\newcommand{\colortwo}{black!50!white}
\newcommand{\coloronefill}{black!15!white}
\newcommand{\colortwofill}{black!05!white}
\newcommand{\colormapone}{rgb=(.4,.4,.4); rgb=(.7,.7,.7)}
\newcommand{\colormaptwo}{rgb=(.6,.6,.6); rgb=(.9,.9,.9)}
\newcommand{\colormapplaneone}{rgb=(.8,.8,.8); rgb=(.95,.95,.95)}
\definecolor{colormaponebottom}{rgb}{.4,.4,.4}
\definecolor{colormaponetop}{rgb}{.7,.7,.7}
\definecolor{colormaptwobottom}{rgb}{.6,.6,.6}
\definecolor{colormaptwotop}{rgb}{.9,.9,.9}
}%

\newlength\tindent
\setlength{\tindent}{\parindent}
\setlength{\parindent}{0pt}
\renewcommand{\indent}{\hspace*{\tindent}}

\pgfplotsset{my style/.append style={axis x line=middle, axis y line=
middle, xlabel={$x$}, ylabel={$y$}, axis equal }}

\pgfplotsset{compat=1.13}
\newcommand{\ds}{\displaystyle}
\begin{document}

All page numbers are from the original APEX text. If that's not helpful let me know how to better reference the location of the changes to be made.\\ \\


\textbf{OLD 9.4 Introduction to Polar Coordinates}
\vskip .5 truecm

\textbf{pp. 526 - 527}\\
\textbf{Example 299} SOLUTION to 2 part b. Delete the 45 degrees from end of answer (i.e. end with $\pi/4$).\\

\textbf{p. 528}\\
Figure 9.40 - make the concentric circles a little lighter.\\

\textbf{p. 529}\\
\textbf{Example 301} SOLUTION ~~~~ First paragraph change choose to chose in the sentence "A common question... With rectangular equations, we often \sout{choose} chose..."\\

Delete 3rd pargraph: \sout "The graph shown uses more points... graph looks like."\\

Delete the Technology Note.

Delete Figure 9.41. \\
On Figure 9.42 label the points shown in the table and on the deleted Figure 9.41. \\
Change the caption on Figure 9.42 to "Graph of the polar function in Example 301."\\

\textbf{p. 529 - 530}\\
On Figures 9.42 and 9.44 make the concentric circles a little lighter.\\
Delete Figure 9.44 (a). If possible label the points from Figure 9.43 (table in the text) on Figure 9.55 (b).\\
Delete the Figure number \& table title from Figure 9.43 - Or do whatever is consistent with what you did in Calculus I text for these "in text" tables.\\

\textbf{p. 532}\\
Third paragraph insert the word "to": "Some curves have very simple... (a shape important \textbf {to} the sensitivity..."\\

\textbf{OLD 9.5 Calculus and Polar Coordinates}
\vskip .5 truecm

\textbf{p. 538}\\
Third paragraph insert the word "to": "We are interested in the lines tangent \textbf{to} a given graph..."\\

\textbf{p. 539}\\
Example 305 SOLUTION part 2:\\
Add boldfaced text to "To find the vertical ..., we \textbf{determine where $\frac{dy}{dx}$ is undefined by setting} \sout{set} the denominator \sout{ $\frac{dy}{dx}=0$} equal to zero."\\

\textbf{p. 539}\\
In the Area section - 4th paragraph - Change indexing on $\theta$:\\ 
$\theta_1$ becomes $\theta_0$\\
$\theta_{n+1}$ becomes $\theta_n$\\
$\theta_i$ becomes $\theta_{i-1}$\\
$\theta_{i+1}$ becomes $\theta_i$\\
There are 2 $f(c_i)^2$ that need to become $[f(c_i)]^2$ in this paragraph and the Area approximation that follows it.\\

\textbf{p. 540}\\
Theorem 83 has an $f(\theta)^2$ that needs to become $[f(\theta)]^2$\\
In margin note next to Theorem 83 change "power reducing" to "half angle".\\

\textbf{p. 543}\\
Example 308 SOLUTION - delete the approximation at end of solution.

Key Idea 42: Add the $f_1, f_2$ version of the integral: $$A = \frac{1}{2} \int_\alpha^\beta [f_1(\theta)]^2 - [f_2(\theta)]^2~d\theta = \frac{1}{2} \int_\alpha^\beta r_1^2 - r_2^2~d\theta$$

\textbf{p. 543 Arc Length}\\
All total there are 10 $f'(t)^2, g'(t)^2, f(\theta)^2, x'(\theta)^2, y'(\theta)^2$ etc. that need brackets.\\

\textbf{p. 544}\\
Add Example in Arc Length section before 311\\

\textbf{Example \#~~~~~ Arc Length of Polar Curves}\\
Find the arc length of the cardiord $r=1+\cos \theta$.\\
\textbf{SOLUTION} \indent With $r=1+\cos \theta$, we have $r' = -\sin \theta$. The cardiod is traced out once on $[0,2\pi]$, giving us our bounds of integration. Applying Key Idea 43?? we have\\

\begin{align}
L & = \int_0^{2\pi} \sqrt{(-\sin \theta)^2  + (1 + \cos \theta)^2}~d\theta \\
& = \int_0^{2\pi} \sqrt{\sin^2 \theta + (1 + 2\cos \theta + \cos \theta)}~d\theta\\
& = \int_0^{2\pi} \sqrt{2 + 2\cos \theta}~d\theta\\
& = \int_0^{2\pi} \sqrt{2 + 2\cos \theta}~\frac{\sqrt {2-2\cos \theta}}{\sqrt {2-2\cos \theta}}~d\theta\\
& = \int_0^{2\pi} \frac{\sqrt{4 - 4\cos^2 \theta}}{\sqrt {2-2\cos \theta}}~d\theta\\
& = 2\int_0^{2\pi} \frac{\sqrt{1 - \cos^2 \theta}}{\sqrt {2-2\cos \theta}}~d\theta\\
& = 2\int_0^{2\pi} \frac{|\sin \theta|}{\sqrt {2-2\cos \theta}}~d\theta\\
\end{align}
Since the $\sin \theta > 0$ on $[0, \pi]$ and $\sin \theta < 0$ on $[\pi, 2\pi]$ we separate the integral into two parts\\
$$2\int_0^{\pi} \frac{\sin \theta}{\sqrt {2-2\cos \theta}}~d\theta - 2\int_{\pi}^{2\pi} \frac{\sin \theta}{\sqrt {2-2\cos \theta}}~d\theta$$
Using the symmetry of the cardiod and $u$-substitution ($u = 2 - 2\cos \theta$) we simplify the integration to\\
\begin{align}
& = 4\int_0^{\pi} \frac{\sin \theta}{\sqrt {2-2\cos \theta}}~d\theta\\
& = 2\int_0^4 \frac{1}{\sqrt u}~du\\
& = 4  u^{1/2}\biggr|_0^4 = 8\\
\end{align}









\end{document}