\documentclass[10pt]{article}


%%
% HERE THERE BE DRAGONS
%%

\usepackage{ifthen}

\usepackage{lipsum}
\usepackage{pgfplots}
\pgfplotsset{colormap={coloronemap}{rgb=(.4,.4,1); rgb=(.8,.8,1)}}
\pgfplotsset{colormap={colortwomap}{rgb=(1,.4,.4); rgb=(1,.8,.8)}}
%\pgfplotsset{compat=1.3}
\usepackage{eso-pic,calc}
\usepackage[font=small]{caption}
\usepgfplotslibrary{external}
\usetikzlibrary{calc}
\usetikzlibrary{shadings}

\usepackage[h]{esvect}

\pgfplotsset{compat=1.8}

\usepackage[paperheight=11in,paperwidth=3in,%inner=1in,includeheadfoot,
textheight=10in,%textwidth=345pt,
marginparwidth=150pt]{geometry}
%% end detour
%%

\usepackage{amsmath}




\ifthenelse{\boolean{xetex}}%
	{\sffamily
	%%\usepackage{fontspec}
	\usepackage{mathspec}
	\setallmainfonts[Mapping=tex-text]{Calibri}
	\setmainfont[Mapping=tex-text]{Calibri}
	\setsansfont[Mapping=tex-text]{Calibri}
	\setmathsfont(Greek){[cmmi10]}}
	{Please compile with XeLaTeX}

\newboolean{colorprint}
\setboolean{colorprint}{true}
%\setboolean{colorprint}{false}

\ifthenelse{\boolean{colorprint}}{%
\newcommand{\colorone}{blue}
\newcommand{\colortwo}{red}
\newcommand{\coloronefill}{blue!15!white}
\newcommand{\colortwofill}{red!15!white}
\newcommand{\colormapone}{rgb=(.4,.4,1); rgb=(.8,.8,1)}
\newcommand{\colormaptwo}{rgb=(1,.4,.4); rgb=(1,.8,.8)}
\newcommand{\colormapplaneone}{rgb=(.7,.7,1); rgb=(.9,.9,1)}
\definecolor{colormaponebottom}{rgb}{.4,.4,1}
\definecolor{colormaponetop}{rgb}{.8,.8,1}
\definecolor{colormaptwobottom}{rgb}{1,.4,.4}
\definecolor{colormaptwotop}{rgb}{1,.8,.8}
}% ends color
{% not color
\newcommand{\colorone}{black}
\newcommand{\colortwo}{black!50!white}
\newcommand{\coloronefill}{black!15!white}
\newcommand{\colortwofill}{black!05!white}
\newcommand{\colormapone}{rgb=(.4,.4,.4); rgb=(.7,.7,.7)}
\newcommand{\colormaptwo}{rgb=(.6,.6,.6); rgb=(.9,.9,.9)}
\newcommand{\colormapplaneone}{rgb=(.8,.8,.8); rgb=(.95,.95,.95)}
\definecolor{colormaponebottom}{rgb}{.4,.4,.4}
\definecolor{colormaponetop}{rgb}{.7,.7,.7}
\definecolor{colormaptwobottom}{rgb}{.6,.6,.6}
\definecolor{colormaptwotop}{rgb}{.9,.9,.9}
}%
\newcommand{\la}{\left\langle}
\newcommand{\ra}{\right\rangle}
\newcommand{\dotp}[2]{\ensuremath{\vec #1 \cdot \vec #2}}
\newcommand{\proj}[2]{\ensuremath{\text{proj}_{\,\vec #2}{\,\vec #1}}}

\newcommand{\fp}{\ensuremath{f\,'}}

\DeclareMathOperator{\sech}{sech}
\DeclareMathOperator{\csch}{csch}

\newcommand{\threedlines}[4][]{\draw [dashed,#1] (axis cs: #2,#3,#4) -- (axis cs: #2,#3,0) -- (axis cs: #2,0,0)  (axis cs: #2,#3,0)--(axis cs:0,#3,0);}

\newcommand{\mydraw}{\draw (axis cs:0,0,0) -- (axis cs:1,1,0);}
\newcommand{\ds}{\displaystyle}


%% no more dragons.  type away %%

% prefer color one for the main graph, and color two for secondary lines

\begin{document}
\begin{tikzpicture}
\begin{axis}%
[
width=\marginparwidth+25pt,%
tick label style={font=\scriptsize},axis on top,
			axis y line=none,
%			axis x line=none,
%%			axis z line=none,%
			axis lines=center,
			y dir=reverse,
%			view={125}{30},
			name=myplot,
%			%x=.37\marginparwidth,
%			%y=.37\marginparwidth,
			xtick={1},
			ztick=\empty,% 
%			extra x ticks={1,3},
%			extra x tick labels={$a$,$b$},
			ytick={1},
%			%minor y tick num=1,%extra y ticks={-5,-3,...,7},%
			ymin=-.2,ymax=1.3,%
			xmin=-.2,xmax=1.3,
			zmin=-.2,zmax=1.3%
]
\addplot3[domain=0:1,y domain=0:1,surf,shader=flat,colormap={mp2}{\colormapone}] (x,0,{y*((sqrt x)-(x^2))+x^2});

\addplot3[domain=0:1,samples y=0,black,thick,smooth,] (x,0,{x^2});

\addplot3[domain=0:1,samples y=0,black,thick,smooth,] (x,0,{sqrt x});


\draw (axis cs: 0,0,.4) -- (axis cs:.63,0,.4);

\draw (axis cs: .8,0,.2) node {\scriptsize $R(y)=\sqrt{y}$};

\filldraw (axis cs:0,0,.4) circle (1pt) 
					(axis cs:.63,0,.4) circle (1pt)
					(axis cs:0,0,.77) circle (1pt)
					(axis cs:.6,0,.77) circle (1pt);

\draw (axis cs: 0,0,.77) -- (axis cs:.6,0,.77);

\draw (axis cs: .4,0,1) node  {\scriptsize $r(y)=y^2$};


\end{axis}

\node [right] at (myplot.right of origin)[shift={(-20pt,-15pt)}] {\scriptsize $x$};
\node [above] at (myplot.above origin) [shift={(0,-10pt)}] {\scriptsize $y$};
\end{tikzpicture}

\end{document}