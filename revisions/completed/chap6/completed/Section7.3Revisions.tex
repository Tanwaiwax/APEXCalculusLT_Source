\documentclass[10pt]{article}


\usepackage{ifthen}

\usepackage{lipsum}
\usepackage{pgfplots}

\usepackage{eso-pic,calc}
\usepackage[font=small]{caption}
\usepgfplotslibrary{external}
\usetikzlibrary{calc}
\usetikzlibrary{shadings}
\usepackage{tikz}
\usetikzlibrary{positioning,chains,fit,shapes,calc,arrows,patterns}
\usepackage{tkz-graph}
\usetikzlibrary{arrows, petri, topaths}
\usepackage{tkz-berge}
\usepackage[all]{xy}
\usepackage{textcomp}
\usepackage[h]{esvect}
\usepackage[normalem]{ulem}

\pgfplotsset{compat=1.8}
\usepackage{amssymb}

\usepackage{amsmath}

\newcommand{\ds}{\displaystyle}


\begin{document}

Bottom of page 361 add the following: 

\begin{align*}
V&=\lim_{n\to \infty} \sum_{i=1}^n 2\pi r_i h_i \Delta x_i\\
&=2\pi \int_a^b r(x)h(x)~dx
\end{align*}

Figure 7.18 p 362 needs color

Example 209:  top of page 363 remove the last part of the sentence (This is the differential element) \\

REMOVE approximations from final answers in all examples in this section.\\


Example 210:  %%%%%%%%%%%%%%%%%%%%
We talked about changing the points here to match example 208....however this is NOT rotated about an axis, it is rotated about x=3 so it is also an example of a region not rotated about an axis. For these reasons I didn't change it. I did change example 212 to be the same function as one we did using the disk/washer method to point out we get the same answer using both methods.\\

In the solution to example 210: the first paragraph and begining of second should read:

The region is sketched in Figure \ref{fig:shell2}(a) along with a line within the region parallel to the axis of rotation. In part (b) of the figure, we see a sample shell, and in part (c) the whole solid is shown.

The height of the sample shell is the distance from $y=1$ to $y=2x+1$, the line that connects the points $(0,1)$ and $(1,3)$. Thus $h(x) = 2x+1-1 = 2x$. The radius of the sample shell is the distance from $x$ to $x=3$; that is, it is $r(x)=3-x$. The $x$-bounds of the region are $x=0$ to $x=1$, giving



Example 211 solution should start:  

The region is sketched in Figure \ref{fig:shell3}(a). In part (b) of the figure the sample shell is drawn, and the solid is sketched in (c). (Note that the triangular region looks ``short and wide'' here, whereas in the previous example the same region looked ``tall and narrow.'' This is because the bounds on the graphs are different.)

The height of the sample shell is an $x$-distance, between $x=\frac12y-\frac12$ and $x=1$.


Change Example 212 as follows \\

Directions:  Find the volume of the solid formed by rotating the region bounded by $y=3x-x^2$ and $y=x$ about the $y$-axis.

Solution:   The region, a sample shell, and the resulting solid are shown in Figure ?????. The radius of a sample shell is $r(x)=x$ ; the height of a sample shell is $h(x)=(3x-x^2)-x=2x-x^2$. The $x$ bounds on the region are $x=0$ to $x=2$ leading to the integral
\begin{align*}
V&=2\pi \int_0^2 x(2x-x^2) ~dx\\
&=2\pi \int_0^2 2x^2-x^3 ~dx\\
&=2\pi \left[ \frac23 x^3-\frac14 x^4 \right] \Big|_0^2\\
&=\frac{4\pi}{3}
\end{align*}

Note that in order to use the Washer Method, we would need to solve $y=3x-x^2$ for $x$, requiring us to complete the square. We must evaluate two integrals as we have two different sample slices. The volume can be computed as 
\begin{align*}
V&=2\pi \int_0^{2} \left(y-\left(\frac32-\sqrt{\frac94 -y}\right)\right)^2~dy \ + \ 2\pi \int_2^{9/4} \left(\left(\frac32+\sqrt{\frac94 -y}\right)-\left(\frac32-\sqrt{\frac94 -y}\right)\right)^2~dy\\
&=2\pi \int_0^2 \left(y-\frac32 +\sqrt{\frac94 -y}\right)^2 ~dy \ + \ 2\pi\int_2^{9/4} \left(2\sqrt{\frac94 -y}\right)^2~dy
\end{align*}
While this integral is not impossible to solve, using the Shell Method gave us a significantly easier way to compute the volume.

Add the following example with the given text before it:

The following example shows how there are times when it does not matter which method you choose to evaluate the volume of a solid. In Example ????????????? we found the volume of the solid formed by rotating the region bounded by $y=\sqrt x$ and $y=x$ about $y=2$.
We will now demonstrate how to find the volume with the shell method. Note that your answer should be the same whichever method you choose.

directions:  Find the volume of the solid formed by rotating the region bounded by $y=\sqrt x$ and $y=x$ about $y=2$ us the Shell Method.

Solution: Since our shells are parallel to the axis of rotation, we must consider the radius and height functions in terms of $y$. The radius of a sample shell will be $r(y)=2-y$ and the height of a sample shell will be $h(y)=y=y^2$. The $y$ bounds for the region will be $y=0$ to $y=1$ resulting in the integral
\begin{align*}
V&=2\pi \int_0^1 (2-y)(y-y^2)~dy\\
&=2\pi\int_0^1 y^3-3y^2+2y~dy\\
&=\frac{\pi}{2} \text{units}^3.
\end{align*}

Delete KEY idea 26.



\end{document}