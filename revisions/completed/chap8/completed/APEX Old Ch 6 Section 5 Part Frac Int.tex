\documentclass[11pt]{report}
\usepackage[letterpaper, total={6.5in, 10in}]{geometry}
%\usepackage{fancyhdr}
%\pagestyle{fancy}
\usepackage{amsmath, amsthm, mathpazo, epic, eepic, color, array}
\usepackage{amssymb}
%\usepackage{graphicx}
\usepackage{cancel}
\usepackage{pgfplots}
\usepackage{multicol}
\pgfplotsset{compat=1.13}
\usepackage{etoolbox}
\makeatletter
\patchcmd{\chapter}{\if@openright\cleardoublepage\else\clearpage\fi}{}{}{}
\makeatother
\usepackage{hyperref}
\usepackage[normalem]{ulem}

\usepackage{enumerate}
\usepackage{enumitem}

\usepackage{tikz}
\usetikzlibrary{positioning,chains,fit,shapes,calc,arrows,patterns}
\usepackage{tkz-graph}
\usetikzlibrary{arrows, petri, topaths}
\usepackage{tkz-berge}
\usepackage[all]{xy}
\usepackage{textcomp}

\newboolean{colorprint}
\setboolean{colorprint}{true}
%\setboolean{colorprint}{false}

\ifthenelse{\boolean{colorprint}}{%
\newcommand{\colorone}{blue}
\newcommand{\colortwo}{red}
\newcommand{\coloronefill}{blue!15!white}
\newcommand{\colortwofill}{red!15!white}
\newcommand{\colormapone}{rgb=(.4,.4,1); rgb=(.8,.8,1)}
\newcommand{\colormaptwo}{rgb=(1,.4,.4); rgb=(1,.8,.8)}
\newcommand{\colormapplaneone}{rgb=(.7,.7,1); rgb=(.9,.9,1)}
\definecolor{colormaponebottom}{rgb}{.4,.4,1}
\definecolor{colormaponetop}{rgb}{.8,.8,1}
\definecolor{colormaptwobottom}{rgb}{1,.4,.4}
\definecolor{colormaptwotop}{rgb}{1,.8,.8}
}% ends color
{% not color
\newcommand{\colorone}{black}
\newcommand{\colortwo}{black!50!white}
\newcommand{\coloronefill}{black!15!white}
\newcommand{\colortwofill}{black!05!white}
\newcommand{\colormapone}{rgb=(.4,.4,.4); rgb=(.7,.7,.7)}
\newcommand{\colormaptwo}{rgb=(.6,.6,.6); rgb=(.9,.9,.9)}
\newcommand{\colormapplaneone}{rgb=(.8,.8,.8); rgb=(.95,.95,.95)}
\definecolor{colormaponebottom}{rgb}{.4,.4,.4}
\definecolor{colormaponetop}{rgb}{.7,.7,.7}
\definecolor{colormaptwobottom}{rgb}{.6,.6,.6}
\definecolor{colormaptwotop}{rgb}{.9,.9,.9}
}%

\newlength\tindent
\setlength{\tindent}{\parindent}
\setlength{\parindent}{0pt}
\renewcommand{\indent}{\hspace*{\tindent}}

\pgfplotsset{my style/.append style={axis x line=middle, axis y line=
middle, xlabel={$x$}, ylabel={$y$}, axis equal }}

\pgfplotsset{compat=1.13}
\newcommand{\ds}{\displaystyle}
\begin{document}


\textbf{OLD 6.5 Partial Fraction Decomposition}\\
\vskip .25 truecm

All page numbers are from the original APEX text. If that's not helpful let me know how to better reference the location of the changes to be made.\\ \\

\textbf{p. 305}\\
Replace the last paragraph with: \\
We start with a rational function $\ds f(x) = \frac{p(x)}{q(x)}$, where $p$ and $q$ do not have any common factors. We first consider the degree of $p$ and $q$. 
\begin{itemize}
\item If the deg$(p) \geq$ deg$(q)$ then we use polynomial long division to divide $q$ into $p$ to determine remainder $r(x)$ where deg$(r) < $ deg$(q)$. We then write $\ds f(x) = s(x) + \frac{r(x)}{q(x)}$ and apply partial fraction decomposition to $\ds\frac{r(x)}{q(x)}$ .
\item If the deg$(p) <$ deg$(q)$ we can apply partial fraction decomposition to $\ds\frac{p(x)}{q(x)}$ without additional work.
\end{itemize}

Partial fraction decomposition is based on an algebraic theorem that guarantees that any polynomial, and hence $q$, can be factored into the product of linear and irreducible quadratics factors. The following Key Idea states how to decompose a rational function into a sum of rational fucntions whose denominators are all of lower degree than $q$. \\ \\

\textbf{p. 306}\\
\textbf{Key Idea \# \indent Partial Fraction Decomposition}\\
Let $\ds \frac {p(x)}{q(x)}$ be a rational function, where deg$(p) <$ deg$(q)$.
\begin{enumerate}
\item \textbf {Factor} $\mathbf{q(x):}$ Write $q(x)$ as the product of its linear and irreducible quadratic factors of the form $(ax+b)^m$ and $(ax^2+bx+c)^n$ where $m$ and $n$ are the highest powers of each factor that divide $q$.
\begin{itemize}
\item \textbf{Linear Terms:} For each linear factor of $q(x)$ the decomposition of $\ds \frac {p(x)}{q(x)}$ will contain the following terms:
$$\ds \frac {A_1}{(ax+b)} + \frac {A_2}{(ax+b)^2}+ \cdots \frac {A_m}{(ax+b)^m}$$
\item \textbf{Irreducible Quadratic Terms:}  For each irreducible quadratic factor of $q(x)$ the decomposition of $\ds \frac {p(x)}{q(x)}$ will contain the following terms:
$$\ds \frac {B_1x+C_1}{(ax^2+bx+c)} + \frac {B_2x+C_2}{(ax^2+bx+c)^2}+ \cdots \frac {B_nx+C_n}{(ax^2+bx+c)^n}$$
\end{itemize}
\item \textbf{Finding the Coefficients }$\mathbf{A_i, B_i, \textbf{and} C_i:}$
\begin{itemize}
\item Set $\ds \frac {p(x)}{q(x)}$ equal to the sum of its linear and irreducible quadratic terms.
$$\ds \frac{p(x)}{q(x)} =\frac {A_1}{(ax+b)} + \cdots \frac {A_m}{(ax+b)^m} + \frac {B_1x+C_1}{(ax^2+bx+c)} + \cdots \frac {B_nx+C_n}{(ax^2+bx+c)^n}$$
\item Multiply this equation by the factored form of $q(x)$ and simplify to clear the denominators. 
\item Solve for the coefficients $A_i, B_i,$ and $C_i$ by
%\begin{itemize}
\begin{enumerate}[label=\alph*)]
\item multiplying out the remaining terms and collecting like powers of $x$, equating the resulting coefficients and solving the resulting system of linear equations, \textbf{or}
\item Substituting in values for $x$ that eliminate terms so the simplified equation can be solved for a coefficient.
\end{enumerate}
\end{itemize}
\end{enumerate}
\vskip .5 truecm
Replace paragraph just below Key Idea box with the following: \\
"The following examples will demonstrate how to put this Key Idea into practice. In Example 181 we focus on the setting up the decomposition of a rational function."\\

\textbf{In Example 181}\\
SOLUTION \indent "The denominator...\sout{cannot be factored} are irreducible quadratics...is a linear \sout{term} factor that divides.."\\ \\

\textbf{p. 308}\\

\textbf{Example 181 continued} - Add a sentence after "Solving ... not hard." In the next example we demonstrate solving for the coefficients using both methods given in Key Idea \#." \\

\textbf{Exercise 182}\\
SOLUTION \indent "The denominator \sout{factors into two linear terms} can be written as the product of two linear factors: ..."\\

\sout{To solve for A and B...} and replace with: "Using the method described in Key Idea \# 2a) to solve for A and B, first multiply..."\\

After solution with method 2a) add:\\
"Using the method described in Key Idea \# 2b) to solve for A and B, we choose values for $x$ that eliminate terms in the equation:
$$1=A(x+1) + B(x-1)$$

If we choose $x=-1$
$$1=A(0) + B(-2)$$
$$B= -\frac{1}{2}.$$ 

Next choose  $x=1$
$$1=A(2) + B(0)$$
$$B= \frac{1}{2}.$$ 

Resulting in the same decomposition as above.\\

\textbf{Between Examples 182 \& 183 add:}\\
In Example 183 we solve for the decomposition coefficients using the system of linear equations (method 2a). The margin note explains how to solve using substitution (method 2b).\\ 

\textbf{p. 309}\\
Example 184 - Delete system of equation process for solving for A \& B and use the following after 
$$19x+30=A(x+3) + B(x-5)$$

If we choose $x=-3$
$$19(-3)+30=A(0) + B(-8)$$
$$B= \frac{27}{8}.$$ 

Next choose  $x=5$
$$19(5)+30=A(8) + B(0)$$
$$A= \frac{125}{8}.$$ 
\\

\textbf{p. 310}\\
\textbf{Between Example 184 \& 185} insert the reminder about $\tan^{-1} x$:\\
"Before the next example we remind the reader of a rational integrand evaluated by trigonometric substitution.
$$\int \frac{1}{x^2+a^2}~dx = \frac{1}{a} \tan^{-1} \bigl(\frac{x}{a}\bigr) + C"$$
\\

Example 185 - Delete system of equation process for solving for A, B \& C and use the following after $$7x^2+31x+54 =A(x^2+6x+11) + (Bx+C)(x+1)$$.

If we choose $x=-1$
$$30 =6A + (-B+C)(0)$$
$$A= 5.$$ 

Eventhough none of the other terms can be zeroed out, we continue by letting $A=5$ and substituting helpful values of $x$. 
Choosing  $x=0$ we notice
$$54 = 55 +C$$
$$C= -1.$$ 
Finally, choose $x=1$ (any value other than -1 and 0 can be used, 1 is easy to work with)
$$92 =90 + (B-1)(2)$$
$$B= 2.$$ 

\sout{Solving the system... }"Thus,...\\ \\


\textbf{p. 311}\\

We could add the following sentence to the end of the 2nd to last paragraph: "The next section will require the reader to determine an appropriate method for evaluating a variety of integrals."\\
Delete the last paragraph.\\ \\


\textbf{p. 312, Exercises:}\\

NO CHANGES!!

\end{document}

