\documentclass{amsart}

\newcommand{\BBR}{\mathbb{R}}
\newcommand{\abs}[1]{\left\lvert#1\right\rvert}
\newcommand{\Ovec}{\vec0}
\newcommand{\xvec}{\vec x}
\newcommand{\norm}[1]{\left\lVert#1\right\rVert}

\begin{document}

\subsection*{Taylor Polynomials for Functions of Two Variables}

We'll take a slight detour at this point.
%You won't be tested on it; consequently, you are free to stop reading this once you get tired of trying to figure it out.  But if you're curious, feel free to read on.

Recall from \ref{thm:taylorthm} that for a function $y=f(x)$, we have the $n\textsuperscript{th}$-order Taylor polynomial about a point $a$:
\[
 p_n(x)
% =f(a)+f'(a)(x-a)+\frac{f''(a)}{2!} (x-a)^2%+\frac{f^{(3)}(a)}{3!}(x-a)^3
% +\dotsm+\frac{f^{(k)}(a)}{k!}(x-a)^k,
 =\sum_{k=0}^n\frac{f^{(k)}(a)}{k!}(x-a)^k
 =p_{n-1}(x)+\frac{(x-a)^n}{n!}f^{(n)}(a),
\]
with an associated remainder $R_n(x)=f(x)-p_n(x)$.  If we know that $\abs{f^{(n+1)}(u)}\le K$ for all $u$ in an open interval containing $a$ and $x$, then we can bound the remainder as well:
\[\abs{R_n(x)}\le K\frac{\abs{x-a}^{n+1}}{(n+1)!}.\]

This section has shown that a function $z=f(x,y)$ can be locally approximated near $(a,b)$ by
\[ L(x,y)=f(a,b)+f_x(a,b)(x-a)+f_y(a,b)(y-b). \]

We can combine these ideas and talk about the Taylor polynomial for $f$ (for simplicity, we'll take $(a,b)=(0,0)$). Let
\begin{align*}\index{Taylor Series!multivariable}
 p_0(x,y)&=f(0,0) \\
 p_1(x,y)&=p_0(x,y)+xf_x(0,0)+yf_y(0,0) \\
 p_2(x,y)&=p_1(x,y)+\frac{x^2}2f_{xx}(0,0)+xyf_{xy}(0,0)+\frac{y^2}2(0,0) \\
 p_3(x,y)&=p_2(x,y)+\frac{x^3}{3!}f_{xxx}(0,0)+\frac{x^2y}2f_{xxy}(0,0)
 +\frac{xy^2}2f_{xyy}(0,0)+\frac{y^3}{3!}f_{yyy}(0,0) \\
 \intertext{and in general,}
 p_n(x,y)&=p_{n-1}(x,y)+\sum_{k=0}^n\frac{x^k y^{n-k}}{k!(n-k)!}
 \frac{\partial^n f}{\partial x^k\partial y^{n-k}}(0,0),
\end{align*}
and define the remainder by $R_n(x,y)=f(x,y)-p_n(x,y)$. Notice that each new term in $p_n$ consists of all the ways to get a polynomial in $x$ and $y$ of order $n$ along with the unique derivative that would make that polynomial a constant, divided by what that constant would be.

It is also possible to prove a bound on $R_n$.  Suppose that
\[\abs{\frac{\partial^n f}{\partial x^k\partial y^{n-k}}(x,y)}\le K\]
for all $k$ and all $(x,y)$ within distance $r$ of the origin. Then for all $(x,y)$ within distance $r$,
\[
 \abs{R_{n-1}(x,y)}
 \le K\frac{(x^2+y^2)^{n/2}}{n!}
 \le K\frac{r^n}{n!}.
\]

% f:R^3\to R. only include if chemistry really wants it

We can continue this idea to a function $f$ of three variables, again centered at the origin for simplicity.  Let
\begin{align*}
 p_0(x,y,z) &= f(0,0,0) \\
 p_n(x,y,z) &= p_{n-1}(x,y,z) + \sum_{k=0}^n\sum_{m=0}^{n-k}
 \frac{x^k y^m z^{n-k-m}}{k!m!(n-m-k)!}
 \frac{\partial^n f}{\partial x^k\partial y^m\partial z^{n-m-k}}(0,0,0),\\
 \intertext{and}
 R_n(x,y,z) &= f(x,y,z)-p_n(x,y,z)
\end{align*}
Suppose
\[\abs{\frac{\partial^n f}{\partial x^k\partial y^m\partial z^{n-k-m}}(x,y,z)}\le K\]
for all $k$ and $m$ and all $(x,y,z)$ within distance $r$ of the origin. Then for all $(x,y,z)$ within distance $r$,
\[
 \abs{R_{n-1}(x,y,z)}
 \le K\frac{(x^2+y^2+z^2)^{n/2}}{n!}
 \le K\frac{r^n}{n!}.
\]

%%%%%%%%%%%%%%%%%%%%%%%%%%%%%%%%%%%%%%%%%%%%%
% APEX omits f:R^d\to R. skip the rest of this

Mathematicians being who they are, they've figured out Taylor polynomials for functions of several variables $f:\BBR^d\to\BBR$ (the most complicated part ends up being the notation). As before, we'll keep this simple by approximating near $\Ovec$. Let $\xvec=\langle x_1,\dotsc,x_n\rangle$, and define
\begin{align*}
 p_0(\xvec) & = f(\Ovec) \\
 p_1(\xvec) & = p_0(\xvec)+\sum_{i=1}^d f_{x_i}(\Ovec)x_i \\
 p_2(\xvec) & = p_1(\xvec)+\sum_{i=1}^d \frac{x_i^2}2f_{x_ix_i}(\Ovec)
 +\sum_{i=1}^d\sum_{j=i+1}^d x_ix_j f_{x_ix_j}(\Ovec) \\
 \intertext{and in general,}
 p_n(\xvec) & = p_{n-1}(\xvec) +
 \sum_{\abs\alpha=n}\frac{\xvec^\alpha}{\alpha!}
 \frac{\partial^n f}{\partial\xvec^\alpha}
 (\Ovec).
\end{align*}

At the end we used multi-index notation: $\alpha=\langle\alpha_1,\dotsc,\alpha_d\rangle$ is a vector of integers with each $\alpha_i\ge 0$, and size $\abs\alpha=\sum_{i=1}^d\alpha_i$. Then $\sum_{\abs\alpha=n}$ means to sum over all possible $\alpha$ with size $n$, and
\begin{align*}
 \alpha! & = \prod_{i=1}^d \alpha_i!,
 & \xvec^\alpha & = \prod_{i=1}^d x_i^{\alpha_i}, \qquad\text{and} &
 \frac{\partial^n}{\partial\xvec^\alpha}f
 & = \frac{\partial^{\alpha_1}}{\partial x_1^{\alpha_1}}
 \dotso\frac{\partial^{\alpha_d}}{\partial x_d^{\alpha_d}}f.
\end{align*}
As before, each new term in $p_n$ consists of all the ways to get a polynomial in $\xvec$ of order $n$ along with the unique derivative that would make that polynomial a constant, divided by what that constant would be.

We then define the remainder by $R_n(\xvec)=f(\xvec)-p_n(\xvec)$, and suppose that
\[\abs{\dfrac{\partial^n f}{\partial\xvec^\alpha}(\xvec)}\le K\]
for all $\abs\alpha=n$ and $\norm\xvec\le r$.  Then for all $\xvec$ such that $\norm\xvec\le r$,
\[\abs{R_{n-1}(\xvec)}\le K\frac{\norm\xvec^n}{n!}\le K\frac{r^n}{n!}.\]

%\subsection*{Math Doctor Bob Videos}
%\begin{itemize}
%\item \href{http://www.youtube.com/watch?v=8MM_7YtjDUs}{Taylor Polynomial of $f(x,y) = y\cos(x+y)$}
%\end{itemize}

\end{document}
