% !TEX program = XeLaTeX
\input{headers/pre_package}

\usepackage{headers/apex_style}
\usepackage[aux]{rerunfilecheck}

\usepackage{etoolbox}

\PassOptionsToPackage{HTML}{xcolor}
%\usepackage[HTML]{xcolor} % must occur before qrcode in apex_style
\usepackage{tikz}
\usetikzlibrary{calc}

% with 10pt font, 1em ~ 10pt ; 1ex ~ 4.3pt

%%Page Size stuff

\usepackage[paperheight=11in,paperwidth=8.5in,%
	inner=1in,textheight=7in,textwidth=320pt,marginparwidth=150pt,%
	marginparsep=32pt,bottom=3in,footskip=1.5in]{geometry}

\newcommand{\exercisegeometry}{%
%	\batchmode%
	\newgeometry{inner=72pt,outer=72pt,textheight=9.25in,tmargin=.75in,
		marginparwidth=150pt,marginparsep=32pt,footskip=29pt}%
%	\errorstopmode%
}
\newcommand{\eendgeometry}{%
%	\batchmode%
	\newgeometry{inner=72pt,outer=36pt,textheight=10in,
		marginparwidth=150pt,marginparsep=32pt}%
%	\errorstopmode%
}
\newcommand{\prefacegeometry}{%
%	\batchmode%
	\newgeometry{inner=1in,textheight=9in,textwidth=320pt,marginparwidth=150pt,%
		marginparsep=32pt,bottom=1in,footskip=1.5in}%
%	\errorstopmode%
}

\newlength{\widest}

%%% This was originally a style with \usepackage, but inputing is generally
%%% equivalent.  The only real difference is how latexml handles style files.
%%% So we'll input this document as a header instead,
%%% and save \usepackage{customstyle}
%%% for things latexml is having trouble with.
%%% This does mean that the distinction between APEX_format and Header_Calculus
%%% is no longer important, and mostly historical.

% do we want to print the keys for the labels? if so, uncomment
%\usepackage[notref,notcite]{showkeys}

\usepackage{amsthm}
\usepackage{amsmath}
%\usepackage{amssymb} % todo ? https://tex.stackexchange.com/a/3000/107497 recommends dropping amssymb in favor of unicode-math
% but then the font loading gets messed up with mathspec
% see also https://tex.stackexchange.com/q/218112/107497 that if the font doesn't have math, then it's a losing battle
%\RequirePackage{unicode-math}

\usepackage{graphicx}
\usepackage{multicol}
\usepackage{makeidx}

\usepackage[normalem]{ulem}

\usepackage{calc}
\usepackage{ragged2e}

%\usepackage[inline]{enumitem}
\usepackage{enumext}

\usepackage[nocomments]{latexml}
\lxDocumentID{apex}

\numberwithin{figure}{section}
\numberwithin{equation}{section}

%%%%%%%%%%%%%%%%%%%%%

\makeindex

\newcommand{\apex}{\texorpdfstring{A\kern -.1em \lower -.5ex\hbox{P}\kern -.25em\lower .5ex\hbox{E}\kern -.1em X}{APEX}}


% Create boolean for whether or not to print 3D graphics. 
% Also creates command to switch back and forth; "looks better."
\newtoggle{in_threeD}
\newcommand{\usethreeDgraphics}{\toggletrue{in_threeD}}
\newcommand{\usetwoDgraphics}{\togglefalse{in_threeD}}
\usethreeDgraphics


\usepackage{pgfplots}
\pgfplotsset{compat=1.8}

\newtoggle{inColor}
\toggletrue{inColor}

\pgfplotsset{colormap={coloronemap}{rgb=(.4,.4,1); rgb=(.8,.8,1)}}
\pgfplotsset{colormap={colortwomap}{rgb=(1,.4,.4); rgb=(1,.8,.8)}}
%\usepgfplotslibrary{external}
% only needed for external tikz pictures (and not liked by latexml)
% see http://tex.stackexchange.com/a/1475/107497
\usetikzlibrary{calc}
\usetikzlibrary{shadings}

% these will be renewcommanded
\newcommand{\colorone}{blue}
\newcommand{\colortwo}{red}
\newcommand{\colorthree}{green}
\newcommand{\coloronefill}{blue!15!white}
\newcommand{\colortwofill}{red!15!white}
\newcommand{\colormapone}{rgb=(.4,.4,1); rgb=(.8,.8,1)}
\newcommand{\colormaptwo}{rgb=(1,.4,.4); rgb=(1,.8,.8)}
\newcommand{\colormapplaneone}{rgb=(.7,.7,1); rgb=(.9,.9,1)}
%\definecolor{colormaponebottom}{rgb}{.4,.4,1}
%\definecolor{colormaponetop}{rgb}{.8,.8,1}
%\definecolor{colormaptwobottom}{rgb}{1,.4,.4}
%\definecolor{colormaptwotop}{rgb}{1,.8,.8}

% determines the line colors for color and black and white lines.
\newcommand{\colorlinecolor}{blue!95!black!30}
\newcommand{\bwlinecolor}{black!30}

% sets the line color to be in color, as a default
\newcommand{\thelinecolor}{\colorlinecolor}

% this allows the above default to be overriden by using
% the \printincolor and \printinblackandwhite commands
% anywhere in the file. This allows you to switch back
% and forth between bw and color. (Who would want to?)
\newcommand{\colornamesuffix}{}

\newcommand{\printincolor}{
 \toggletrue{inColor}%
 % aforementioned renewcommanding
 \renewcommand{\thelinecolor}{\colorlinecolor}
 \renewcommand{\colornamesuffix}{}
 \renewcommand{\colorone}{blue}
 \renewcommand{\colortwo}{red}
 \renewcommand{\colorthree}{green}
 \renewcommand{\coloronefill}{blue!15!white}
 \renewcommand{\colortwofill}{red!15!white}
 \renewcommand{\colormapone}{rgb=(.4,.4,1); rgb=(.8,.8,1)}
 \renewcommand{\colormaptwo}{rgb=(1,.4,.4); rgb=(1,.8,.8)}
 \renewcommand{\colormapplaneone}{rgb=(.7,.7,1); rgb=(.9,.9,1)}
 \definecolor{colormaponebottom}{rgb}{.4,.4,1}
 \definecolor{colormaponetop}{rgb}{.8,.8,1}
 \definecolor{colormaptwobottom}{rgb}{1,.4,.4}
 \definecolor{colormaptwotop}{rgb}{1,.8,.8}
 \setexvideocolor
 \colorizespecialboxes
}

\newcommand{\printinblackandwhite}{
 \togglefalse{inColor}%
 % undoing the above renewcommanding
 \renewcommand{\thelinecolor}{\bwlinecolor}
 \renewcommand{\colornamesuffix}{BW}
 \renewcommand{\colorone}{black}
 \renewcommand{\colortwo}{black!50!white}
 \renewcommand{\colorthree}{black!25!white}
 \renewcommand{\coloronefill}{black!15!white}
 \renewcommand{\colortwofill}{black!05!white}
 \renewcommand{\colormapone}{rgb=(.4,.4,.4); rgb=(.7,.7,.7)}
 \renewcommand{\colormaptwo}{rgb=(.6,.6,.6); rgb=(.9,.9,.9)}
 \renewcommand{\colormapplaneone}{rgb=(.8,.8,.8); rgb=(.95,.95,.95)}
 \definecolor{colormaponebottom}{rgb}{.4,.4,.4}
 \definecolor{colormaponetop}{rgb}{.7,.7,.7}
 \definecolor{colormaptwobottom}{rgb}{.6,.6,.6}
 \definecolor{colormaptwotop}{rgb}{.9,.9,.9}
 \setexvideobw
 \bwizespecialboxes
}


\newcommand{\myincludegraphics}[2][]{%
 \IfFileExists{./#2\colornamesuffix.png}{%
  \includegraphics[#1]{#2\colornamesuffix}%
 }{%
  \IfFileExists{./#2\colornamesuffix.pdf}{%
   \includegraphics[#1]{#2\colornamesuffix}%
  }{%
   \IfFileExists{./#2.png}{%
    \includegraphics[#1]{#2}%
   }{%
    \IfFileExists{./#2.pdf}{%
     \includegraphics[#1]{#2}%
    }{%
     \includegraphics[#1]{#2\colornamesuffix}%
    }%
   }%
  }%
 }%
}



%%%%%%%%%%%%%%%%%%%%%%%%%%%%%%%%%%%%%%%%%%%%%%%%%%%%%%%%%%%%%%%%%%%%%%%%%%%%%%
%% Examples
%%%%%%%%%%%%%%%%%%%%%%%%%%%%%%%%%%%%%%%%%%%%%%%%%%%%%%%%%%%%%%%%%%%%%%%%%%%%%%

\newlength{\boxskipamount}
\setlength{\boxskipamount}{4ex plus 4ex minus 2ex}

%\newlength{\topmarginlength} 
%\newlength{\bottommarginlength}
%\newlength{\oddpagemarginlength}
%\newlength{\evenpagemarginlength}
\newlength{\marginlinelength}
%\newlength{\innerpagemarginlength}

% how far from the text the example line is to be drawn
\setlength{\marginlinelength}{.2em}

% the height of the top margin (used in calculating the lines for examples)
%\setlength{\topmarginlength}{-1in-\voffset}

% the length of the bottom margin (ish)
% actually starts at the top of the page, moves
% through the top margin length then the text height.
%\setlength{\bottommarginlength}{-1in-\textheight-2\baselineskip-\voffset-\headheight-\headsep-\topmargin}

% the length of the left hand margin of an odd page
%\setlength{\oddpagemarginlength}{1in+\hoffset+\oddsidemargin-2\marginlinelength}

% the length of the left hand margin of an even page
%\setlength{\evenpagemarginlength}{1in+\hoffset+\evensidemargin-2\marginlinelength}

\newcommand{\solution}{\bigbreak\par
 \makebox[6.5em][l]{\textsc{\small\textbf{Solution\lxAddClass{solutionTag}}}}%
 \nopagebreak%
}

% black: hsl(x,x,0)
% white: hsl(x,x,100)
% blue: hsl(240,100,50)
% line color: blue!95!black!30 = Hsb(240,.29,.98) = hsl(240,87.7,83.8)

\newlength{\saveparindent}
\setlength{\saveparindent}{\parindent}




%%%%%%%%%%%%%%%%%%%%%%%%%%%%%%%%%%%%%%%%%%%%%%%%%%%%%%%%%%%%%%%%%%%%%%
%% Definitions, Theorems and Key Ideas
%%%%%%%%%%%%%%%%%%%%%%%%%%%%%%%%%%%%%%%%%%%%%%%%%%%%%%%%%%%%%%%%%%%%%%

\newcommand{\newspecialbox}[3]{%
 \AtBeginDocument{\makeStyles{#1}{#3}}%
 \expandafter\newcommand\csname colorize#1\endcsname{
  \definecolor{top#1}{Hsb}{#3,.05,1}% = hsl(#4,100,97.5)
  \ifnumequal{#3}{60}{%
   \definecolor{border#1}{Hsb}{#3,.59,.97}% = hsl(#4,90.5,68.4)
   \definecolor{bottom#1}{Hsb}{#3,.28,.97}% = hsl(#4,81.9,83.4)
  }{%
   \definecolor{border#1}{Hsb}{#3,.23,.65}% = hsl(#4,17.6,57.5)
   \definecolor{bottom#1}{Hsb}{#3,.13,.92}% = hsl(#4,42.8,86)
  }%
 }
 \expandafter\newcommand\csname bwize#1\endcsname{
  \colorlet{top#1}{white}
  \colorlet{bottom#1}{white}
  \colorlet{border#1}{black}
 }
 \newtheorem{#1}{#2}[section]%
 \expandafter\providecommand\csname #1autorefname\endcsname{#2}
 \ifbool{latexml}{%
 }{%
  \tcolorboxenvironment{#1}{
    sharp corners=all,
    enhanced,
    colframe=border#1,
    beforeafter skip=\boxskipamount,
    interior style={top color=top#1, bottom color=bottom#1},
    breakable=true,
    overlay first={\continue{bottom}{#1}},
    overlay middle={\continue{bottom}{#1}\continue{top}{#1}},
    overlay last={\continue{top}{#1}},
    enlargepage flexible=3\baselineskip,
    toggle enlargement=evenpage,
    lines before break=8,
  }%
 }
}

\newcommand{\continue}[1]{%
 \csname continue#1\endcsname{#1}%
}
\newcommand{\continuetext}[2]{%
 \ifstrequal{#1}{top}{%
  \csname #2autorefname\endcsname\ \csname the#2\endcsname\ continued%
 }{%
  (continued)
 }%
}
% can't get this to work
%\newcommand{\northsouth}[2][]{\if thenelse{\equal{#2}{top}}{#1north}{#1south}}

% adapted from https://tex.stackexchange.com/a/545324/107497 by Schrödinger's cat
\newcommand{\continuebottom}[2]{
   \path[font=\small\itshape] (frame.south) node (cont) {\continuetext{#1}{#2}};
   \begin{scope}[decoration={zigzag,amplitude=0.5mm}]
    \path[fill=#1#2]
     decorate {([xshift= 1.2pt]frame.south west) -- (cont.west)} --++ (0,0.5ex)
      -| cycle
     decorate {([xshift=-1.2pt]frame.south east) -- (cont.east)} --++ (0,0.5ex)
      -| cycle;
    \path[fill=white]
     decorate {([xshift= 1.2pt]frame.south west) -- (cont.west)} --++ (0,-0.5ex)
      -| cycle
     decorate {([xshift=-1.2pt]frame.south east) -- (cont.east)} --++ (0,-0.5ex)
      -| cycle;
   \end{scope} 
}
\newcommand{\continuetop}[2]{
   \path[font=\small\itshape] (frame.north) node (thm) {\continuetext{#1}{#2}};
   \begin{scope}[decoration={zigzag,amplitude=0.5mm}]
    \path[fill=#1#2]
     decorate {([xshift= 1.2pt]frame.north west) -- (thm.west)} --++ (0,-0.5ex)
      -| cycle
     decorate {([xshift=-1.2pt]frame.north east) -- (thm.east)} --++ (0,-0.5ex)
      -| cycle;
    \path[fill=white]
     decorate {([xshift= 1.2pt]frame.north west) -- (thm.west)} --++ (0,0.5ex)
      -| cycle
     decorate {([xshift=-1.2pt]frame.north east) -- (thm.east)} --++ (0,0.5ex)
      -| cycle;
   \end{scope} 
}

\newcommand{\colorizespecialboxes}{
 \colorizedefinition
 \colorizetheorem
 \colorizekeyidea
}
\newcommand{\bwizespecialboxes}{
 \bwizedefinition
 \bwizetheorem
 \bwizekeyidea
}




%%%%%%%%%%%%%%%%%%%%%%%%%%%%%%%%%%%%%%%%%%%%%%%%%%%%%%%%%%%%%%%%%
%% Exercises
%% We would like to make better use of enumitem to put implementation
%% details here instead of repeating them, but the pdftagging
%% doesn't deal well with that.  So we'll need to repeat everything every time.
%%%%%%%%%%%%%%%%%%%%%%%%%%%%%%%%%%%%%%%%%%%%%%%%%%%%%%%%%%%%%%%%%

\setlength{\columnsep}{20pt}

\newtoggle{inexercises}

%\makeatletter
%\newcommand{\exercisesubsubsection}{%
% \closeenumerate%
% \@startsection{subsubsection}{3}{-1em}{\bigskipamount}{\bigskipamount}{\Large\textit}*}
%\makeatother

% I'd like to move the \closeenumerate into the \exercisesubsubsection, but I can't figure it out
\newcommand{\printconcepts}{\noindent\closeenumerate\exercisesubsubsection*{\noindent Terms and Concepts}}
\newcommand{\printproblems}{\noindent\closeenumerate\exercisesubsubsection*{\noindent Problems}}
\newcommand{\printreview}{\noindent\closeenumerate\exercisesubsubsection*{\noindent Review}}

%\newlist{sectionexercises}{enumerate}{1}
%\newcounter{saveexercisenum}[section]
%\counterwithin*{sectionexercisesi}{section} % in case we have a exercise set before any exercises that would reset the save exercise enum
%\setlist[sectionexercises]{
%	label=\arabic*.,
%	leftmargin=1.5em,
%    before=\setcounter{sectionexercisesi}{\value{saveexercisenum}},
%    after=\setcounter{saveexercisenum}{\value{sectionexercisesi}},
%}
%\ifbool{latexml}{}{
% \setlist*[sectionexercises]{ref=\arabic*}
%}

%\setenumext[enumext,2]{start=1}
\setenumext[enumext,1]{resume}
\resetenumext[1]{subsection} % reset the resumed counter for exercises

\makeatletter
\newcommand{\printexercises}[1]{%
 \writeToAnsFile{#1}% writeToAnsFile in sty (actually, down below)
 \exercisegeometry% includes a clearpage
 \pagestyle{exercise}%
 \bookmarksetupnext{level=\toclevel@subsection}% otherwise, the level is "section" and everything is messed up
 \exercisesubsection{Exercises \thesection}%
 \stepcounter{subsection}% subsections aren't numbered, but this triggers resetenumext
 \label{exer\thesection}%
 \small%
 \bigskip%
 \begin{multicols}{2}%
  \toggletrue{inexercises}%
  \renewcommand{\Itemautorefname}{Ex\-er\-cise}% local b/c multicols = good
  \input{#1}%
  \closeenumerate%
 \end{multicols}%
 \restoregeometry%
 \pagestyle{prose}%
 %	\easypagecheck
 \setlength{\hoffset}{0pt} \rmfamily\normalsize \bigbreak%
}
\makeatother


\newwrite\answrite %write the answers file
% give the answers file the name ``jobname.answers''
\openout\answrite=\jobname.answers

\newcommand{\writeToAnsFile}[1]{%
 \immediate\write\answrite{%
  \string\answersForSection{\arabic{chapter}}{\arabic{section}}{#1}%
%  \noexpand\answersForSection{\arabic{chapter}}{\arabic{section}}{#1}%
 }%
}
% \noexpand\answersForSection becomes \relax in LaTeXML
% \string works with both

\newcounter{exercisesetcounter}[section]
\renewcommand{\theexercisesetcounter}{\thesection.\arabic{exercisesetcounter}}

\newcounter{saveenumi}

%\counterwithin*{enumXi}{subsection}
% does not work because \exercisesubsection is a fake \subsection

% #1 is "Exercises \thesection"
\newcommand{\exercisesubsectiontitle}[1]{%
 \huge\textbf{\texorpdfstring{\hyperref[sol#1]{#1}}{Exercises}}\hrule
% \setcounter{enumXi}{0}%
}

%\makeatletter
%% the usual \subsection definition has stretchable space in arguments 3-5
%\newcommand{\exercisesubsection}[1]{%
%\setcounter{enumi}{0}%
%\@startsection{subsection}{2}{-.7em}{0pt}{.5ex}{\huge\textbf}{\texorpdfstring{\hyperref[sol#1]{Exercises #1}}{Exercises}}%
%\hrule\vspace{-1.5ex}%
%}
%\makeatother

\newcommand{\exautoref}[1]{%
 \hyperref[#1]{Ex\-er\-cise~\ref*{#1}}%
% {%
%  \renewcommand{\Itemautorefname}{Exercise}% localize the upcoming reference
%  \autoref{#1}%
% }%
% doesn't work?
}

%\newcommand*{\exerenv}{sectionexercises}
\newcommand*{\exerenv}{enumext}

\makeatletter
\newcommand{\openenumerate}{%
 \ifx\@currenvir\exerenv\else%
  \begin{enumext}[widest=22,ref=\arabic*]
%  \begin{sectionexercises}
%  \ifbool{latexml}{%
%   \setcounter{sectionexercisesi}{\value{saveexercisenum}}%
%  }{}%
 \fi%
}
\newcommand{\closeenumerate}{
 \ifx\@currenvir\exerenv%
%  \ifbool{latexml}{%
%   \setcounter{saveexercisenum}{\value{sectionexercisesi}}%
%  }{}%
  \end{enumext}
%  \end{sectionexercises}
 \fi%
}
\makeatother
\newcommand{\closeenumerateinquestions}{\closeenumerate}

 % if the instructions have an enumerate, we want to use the second level
 % we can't have another enumext, because that closed before the instructions
 % so this would happen at level 1.  we could monkey around to make enumext
 % thinks it's at level 2, but this seems easier
\makeatletter
\newcommand{\stepenumeratedepth}{\advance\@enumdepth\@ne}
\makeatother

\newcommand{\exercisesetinstructions}[2][In Exercises]{%
 \setcounter{saveenumi}{\value{enumXi}}%
 \closeenumerateinquestions
 \pagebreak[2]%
 \stepcounter{saveenumi}%
 \stepcounter{exercisesetcounter}%
 \ifnumodd{\value{saveenumi}}{}{%
  \PackageInfo{apex}{%
   Exercise set \theexercisesetcounter\space begins with \arabic{saveenumi}%
  }%
 }%
 \bgroup
 \stepenumeratedepth
% \setenumext[enumext,1]{label=\alph*,wrap-label={(#1)}}% pretend it is level 2
 \noindent#1 \arabic{saveenumi}--\ref*{enumiatendof\theexercisesetcounter}%
% \renewcommand{\theenumi}{(\alph{enumi})}%
 #2%
 \egroup%
 % can't \addtocounter{enumXi}{-1} because #2 may have an enumerate
% \setcounter{enumXi}{\value{saveenumi}}
 \ignorespaces%
 \nopagebreak%
}
\newcommand{\exercisesetend}{%
 \label{enumiatendof\theexercisesetcounter}%
 \closeenumerate%
 \ifnumodd{\value{enumXi}}{%
  \PackageInfo{apex}{%
   Exercise set \theexercisesetcounter\space ends with \arabic{enumXi}%
  }%
 }{}%
}

%\newenvironment{exerciseset}[2]{%
% \stepcounter{sectionexercisesi}
% \stepcounter{exercisesetcounter}%
% \ifnumodd{\value{sectionexercisesi}}{}{%
%  \PackageInfo{apex}{%
%   Exercise set \theexercisesetcounter\space begins with \arabic{sectionexercisesi}%
%  }%
% }%
% {%
%  \setlist[enumerate,1]{label=(\alph*)}% for exercise set instructions
%  \noindent#1 \arabic{sectionexercisesi}--\ref*{enumiatendof\theexercisesetcounter}#2%
% }%
% \addtocounter{sectionexercisesi}{-1}\ignorespaces%
% \nopagebreak%
%}{%
% \label{enumiatendof\theexercisesetcounter}%
% \closeenumerate%
% \ifnumodd{\value{sectionexercisesi}}{%
%  \PackageInfo{apex}{%
%   Exercise set \theexercisesetcounter\space ends with \arabic{sectionexercisesi}%
%  }%
% }{}%
%}
%\BeforeBeginEnvironment{exerciseset}{\closeenumerateinquestions}

\newcommand{\exercise}[2]{%
 \openenumerate%
% \setlist[enumerate,1]{label=(\alph*)}% for exercise instructions
 \item \parbox[t]{\linewidth}{#1}%
}
\newcommand{\showexerciseanswers}{%
 \renewcommand{\exercise}[2]{%
  \ifboolexpr{ togl{printoddanswersonly} and test{\ifnumodd{\value{enumXi}}} }{%
   \stepcounter{enumXi}%
  }{%
   \item \parbox[t]{\linewidth}{\raggedright ##2}%
  }%
 }%
}

\newcommand{\questioncolumnbreak}{\columnbreak}


%%%%%%%%%%%%%%%%%%%%%%%%%%%%%%%%%%%%%%%%%%%%%%%%%%%%%%%%%%%%%%%%%
%% Answers
%%%%%%%%%%%%%%%%%%%%%%%%%%%%%%%%%%%%%%%%%%%%%%%%%%%%%%%%%%%%%%%%%

\newcommand{\printsolutions}[2][\jobname]{%
 \immediate\closeout\answrite%
 \inanswersection\exercisegeometry%
 \pagestyle{exercise}%
 %\thispagestyle{empty}%
 \ifstrequal{#1}{\jobname}{%
  \chapter*{#2}%
 }{%
  {% localize the next line
   \renewcommand{\thefootnote}{}
   \chapter*{#2\footnote{Revised \today}}%
  }%
 }%
 \phantomsection
 \addcontentsline{toc}{chapter}{#2}%
 \begin{multicols}{2}%
  \small\raggedright%
  \input{#1.answers}%
 \end{multicols}%
 \restoregeometry\pagestyle{prose}%
 \setlength{\hoffset}{0pt}\rmfamily%
 \pagestyle{empty}%
 \eendgeometry%
}%

\newcommand{\inanswersection}{%
	\renewcommand{\printconcepts}{}%
	\renewcommand{\printproblems}{}%
	\renewcommand{\printreview}{}%
%	\renewenvironment{exerciseset}[2]{}{}%
	\renewcommand{\exercisesetinstructions}[2][]{}%
	\renewcommand{\exercisesetend}{}%
	\renewcommand{\openenumerate}{}%
	\renewcommand{\closeenumerateinquestions}{}%
	\renewcommand{\questioncolumnbreak}{}%
	\packageinanswersection%
	\showexerciseanswers%
%	\setlist[enumerate,1]{label=\arabic*.}% for solutions
	% LaTeX already does this, but LaTeXML doesn't
}%

\newtoggle{printoddanswersonly}

\toggletrue{printoddanswersonly}
\newcommand{\printallanswers}{\togglefalse{printoddanswersonly}}

%\newcounter{answerchapter}
%\newcounter{answersection}[answerchapter]
%\renewcommand{\theanswersection}{\theanswerchapter.\arabic{answersection}}
\newcommand{\lastanswerchapter}{-1}

\newcommand*{\answersForSection}[3]{%
 \ifnumequal{#1}{\lastanswerchapter}{}{%
  \renewcommand{\lastanswerchapter}{#1}% apparently global. who knew?
  \ifbool{latexml}{}{%
   \belowpdfbookmark{Chapter #1}{solsol#1}%
  }
  % commandeer chapter and section numbering
  \setcounter{chapter}{#1}
  \section*{Chapter~\thechapter\hfill\null}
 }%
 \setcounter{section}{#2}
 \subsection*{\hyperref[exer\thesection]{Exercises~\thesection\hfill\null}}%
 \label{solExercises \thesection}
 \ifnumequal{#2}{0}{%
  \loadAllAnswers{#3}
 }{%
  \loadAnswers{#3}
 }%
}

% only called by the prerequisite sections
\newcommand*{\loadAllAnswers}[1]{%
%	\setcounter{answersection}{-1}%
	\iftoggle{printoddanswersonly}{%
		\togglefalse{printoddanswersonly}%
		\loadAnswers{#1}%
		\toggletrue{printoddanswersonly}%
	}{%
		\loadAnswers{#1}%
	}%
}
\newcommand*{\loadAnswers}[1]{%
%	\stepcounter{answersection}%
 \begin{enumext}[start=1,widest=22]
  \input{#1}
 \end{enumext}
 \bigbreak%
}



% The following creates a ``List of Theorems'', ``Definitions'', and ``Key Ideas''.
% See http://tex.stackexchange.com/q/74857/107497
%\usepackage{thmtools} % continuing theorems and ``List of Theorems''
%\patchcmd\thmtlo@chaptervspacehack
%  {\addtocontents{loe}{\protect\addvspace{10\p@}}}
%  {\addtocontents{loe}{\protect\thmlopatch@endchapter\protect\thmlopatch@chapter{\thechapter}}}
%  {}{failed thmtlo@chaptervspacehack}
%\AtEndDocument{\addtocontents{loe}{\protect\thmlopatch@endchapter}}
%\long\def\thmlopatch@chapter#1#2\thmlopatch@endchapter{%
%  \setbox\z@=\vbox{#2}%
%  \ifdim\ht\z@>\z@
%    \hbox{\bfseries\chaptername\ #1}\nobreak
%    #2
%    \addvspace{10\p@}
%  \fi
%}
%\def\thmlopatch@endchapter{}
%\patchcmd\thmt@mklistcmd
%  {\protect\numberline{\csname the\thmt@envname\endcsname}%
%      \thmt@thmname}{}{}{failed thmt@mklistcmd}
%%\makeatother
%\renewcommand\thmtformatoptarg[1]{#1}


\usepackage{makecell}

\usepackage{amsthm}

\newtheoremstyle{apexExample}% name
  {0pt}% Space above, empty = `usual value'
  {0pt}% Space below
  {}% Body font
  {}% Indent amount (empty = no indent, \parindent = para indent)
  {\bfseries}% Thm head font
  {}% Punctuation after thm head
  {\newline}% Space after thm head
  {\parbox[t]{\ifbool{latexml}{10em}{8em}}{\bfseries\thmname{#1}~\thmnumber{#2}}%
   \thmnote{\parbox[t]{.75\textwidth}{\bfseries\raggedright#3}}%
  }

\newtheoremstyle{apex}% name
  {0pt}% Space above, empty = `usual value'
  {0pt}% Space below
  {}% Body font
  {}% Indent amount (empty = no indent, \parindent = para indent)
  {\bfseries}% Thm head font
  {}% Punctuation after thm head
  {\newline}% Space after thm head
  {\parbox[t]{\ifbool{latexml}{10em}{8em}}{\bfseries\thmname{#1}~\thmnumber{#2}}%
   \thmnote{\parbox[t]{\ifbool{latexml}{.6\textwidth}{.7\textwidth}}{\bfseries\raggedright#3}}%
  }% Thm head spec
  % the padding for the box takes just a bit of room away from these that examples get to keep

\theoremstyle{apexExample}
\newtheorem{example}{Example}[section]
\newcommand{\exampleautorefname}{Ex\-am\-ple}
\theoremstyle{apex}

\makeatletter
\renewenvironment{proof}[1][\proofname]{\pagebreak[2]\par
  \pushQED{\qed}%
  \normalfont \topsep6\p@\@plus6\p@\relax
  \trivlist
  \item[\hskip\labelsep
        \bfseries
    #1]\mbox{}\\* % something is needed to be able to get a newline
}{%
  \popQED\endtrivlist\@endpefalse
}
\makeatother
\renewcommand{\qedsymbol}{\ensuremath{\square}}

\newspecialbox{definition}{Def\-i\-ni\-tion}{60}
% draw = yellow!95!black!60 = Hsb( 60,.59,.97)
% topc = white!95!yellow    = Hsb( 60,.05,1)
% botc = yellow!90!black!30 = Hsb( 60,.28,.97)

\newspecialbox{theorem}{The\-o\-rem}{120}
% draw = green!30!black!50  = Hsb(120,.23,.65)
% topc = white!95!green     = Hsb(120,.05,1)
% botc = green!60!black!20  = Hsb(120,.13,.92)

\newspecialbox{keyidea}{Key I\-dea}{0}
% draw = red!30!black!50    = Hsb(  0,.23,.65)
% topc = white!95!red       = Hsb(  0,.05,1)
% botc = red!60!black!20    = Hsb(  0,.13,.92)



\newtoggle{abridgeConics}
\toggletrue{abridgeConics}

\newcommand{\monthYear}{%
\ifcase \month \or January\or February\or March\or April\or May\or June\or July\or August\or September\or October\or November\or December\fi \space \number \year}
%modified from \today. we could do
%\usepackage[en-US]{datetime2}
%\DTMlangsetup{showdayofmonth=false}
% so that \today is just month and year
%but LaTeXML doesn't have datetime2, so we need this anyway

\usepackage{multirow}
%\pgfplotsset{width=\marginparwidth+1pt,compat=1.3}
\usepackage[font=small,justification=RaggedRight]{caption}

%\usepackage{wrapfig}

\usepackage{booktabs}

\setcounter{secnumdepth}{1}
\setcounter{tocdepth}{1}

\makeatletter
\let\ps@oldplain=\ps@plain % save the plain pagestyle
\makeatother

\usepackage{fancyhdr}

\renewcommand{\chaptermark}[1]{\markboth{\chaptername\ \thechapter\ \ \ \ {#1}}{}}
\renewcommand{\sectionmark}[1]{\markright{\thesection\ \ \ \  #1}}
\renewcommand{\headrulewidth}{0pt}
\renewcommand{\footrulewidth}{0pt}


\fancypagestyle{prose}{%
 \fancyhf{}
 \fancyhead[LE]{\nouppercase{\leftmark}}%
 \fancyhead[RO]{\nouppercase{\rightmark}}%
 \fancyfoot[LE]{\begin{minipage}{\textwidth}%
  \noindent\hspace{\marginparwidth}\hspace{\marginparsep}\hspace{-.4em}%
  \makebox[0pt][l]{\rule{\textwidth}{.4pt}}%
  \vskip.2\baselineskip%
  \noindent\hspace{\marginparwidth}\hspace{\marginparsep}\hspace{-.4em}%
  Notes:%
  \vskip 1.5in\textbf{\thepage}%
 \end{minipage}}

 \fancyfoot[RO]{\begin{minipage}{\textwidth+\marginparwidth+\marginparsep}%
  \rule{\textwidth-\marginparwidth-\marginparsep}{.4pt}
  \vskip.2\baselineskip
  Notes:
  \vskip 1.5in
  \hfill\textbf{\thepage}
 \end{minipage}}
 \fancyhfoffset[LE,RO]{\marginparsep+\marginparwidth}
}
\fancypagestyle{exercise}{%
	\fancyhf{}% 
	\fancyhfoffset[LE,RO]{32pt}%
	\fancyfoot[LE,RO]{\textbf{\thepage}}
}



\let\oldmainmatter\mainmatter
\renewcommand{\mainmatter}{%
 \oldmainmatter
 \fancypagestyle{plain}{% override the default for opening chapters
  \fancyhf{}
  \fancyfoot[RO]{\begin{minipage}{\textwidth+\marginparwidth+\marginparsep}%
   \rule{\textwidth-\marginparwidth-\marginparsep}{.4pt}
   \vskip.2\baselineskip
   Notes:
   \vskip 1.5in
   \hfill\textbf{\thepage}
  \end{minipage}}
  \fancyhfoffset[RO]{\marginparsep+\marginparwidth}
 }
 \pagestyle{prose}
}

\newtoggle{inappendix}
% todo Tim
% \appto\appendix{stuff}
\let\oldappendix\appendix
\makeatletter
\renewcommand{\appendix}{%
 \let\ps@plain=\ps@oldplain% restore the pagestyle
 \cleardoublepage
 \oldappendix
 \toggletrue{inappendix}
 \setcounter{secnumdepth}{-1}
 \pagenumbering{arabic}
 \renewcommand{\thepage}{A.\arabic{page}}
 \renewcommand{\thechapter}{\arabic{chapter}}
 \part*{\appendixname}
% \pagestyle{oldplain}
% \part*{Appendices\protect\thispagestyle{empty}}
% \addcontentsline{toc}{part}{\appendixname}
% \iflatexml\else
% \pdfbookmark[part]{Appendices}{appendixbookmark}
% \fi
}
\makeatother


% an enumerate like environment that can be mixed into tabular, array, etc.
\newcounter{anywhereenumi}
\newenvironment{anywhereenum}{%
 \setcounter{anywhereenumi}{0}%
 \renewcommand{\item}[1][]{%
  \ifx.##1.%
  \refstepcounter{anywhereenumi}%
  \makebox[1em][r]{\arabic{anywhereenumi}.}~~%
  \else%
  \makebox[1em][r]{##1.}~~%
  \fi%
 }%
}{}

\newcommand{\ds}{\displaystyle}

\newcommand{\primeskip}{\ifbool{mmode}{\mkern1.35mu}{\kern.075em}\relax}
%\newcommand{\primeskip}{\hskip.75pt}

\newcommand{\fp}{\ensuremath{f\,'}}
\newcommand{\fpp}{\ensuremath{f\,''}}

\newcommand{\Fp}{\ensuremath{F\primeskip'}}
\newcommand{\Fpp}{\ensuremath{F\primeskip''}}

\newcommand{\yp}{\ensuremath{y\primeskip'}}
\newcommand{\gp}{\ensuremath{g\primeskip'}}

\newcommand{\dd}{\operatorname{d}\!}

\newcommand*{\abs}[1]{\ensuremath{\left\lvert #1 \right\rvert}}
\newcommand*{\norm}[1]{\ensuremath{\left\lVert #1 \right\rVert}}
\newcommand*{\vnorm}[1]{\ensuremath{\norm{\vec #1}}}
\newcommand{\bracket}[1]{\left\langle #1\right\rangle}
\newcommand*{\proj}[2]{\ensuremath{\text{proj}_{\,\vec #2}{\,\vec #1}}}

\newcommand{\vecE}{\ensuremath{\vec E}}
\newcommand{\vecF}{\ensuremath{\vec F}}
\newcommand{\vecG}{\ensuremath{\vec G}}
\newcommand{\vecT}{\ensuremath{\vec T}}
\newcommand{\vece}{\ensuremath{\vec e}}
\newcommand{\vecf}{\ensuremath{\vec f}}
\newcommand{\vecg}{\ensuremath{\vec g}}
\newcommand{\veci}{\ensuremath{\vec\imath}}
\newcommand{\vecj}{\ensuremath{\vec\jmath}}
\newcommand{\veck}{\ensuremath{\vec k}}
\newcommand{\vecl}{\ensuremath{\vec l}}
\newcommand{\vecn}{\ensuremath{\vec n}}
\newcommand{\vecr}{\ensuremath{\vec r}}
\newcommand{\vecu}{\ensuremath{\vec u}}
\newcommand{\vecv}{\ensuremath{\vec v}}
\newcommand{\vecw}{\ensuremath{\vec w}}
\newcommand{\vecx}{\ensuremath{\vec x}}
\newcommand{\vecy}{\ensuremath{\vec y}}
\newcommand{\vrp}{\ensuremath{\vec r\hskip1.25pt '}}
\newcommand{\vsp}{\ensuremath{\vec s\primeskip '}}
\newcommand{\vrt}{\ensuremath{\vec r(t)}}
\newcommand{\vst}{\ensuremath{\vec s(t)}}
\newcommand{\vvt}{\ensuremath{\vec v(t)}}
\newcommand{\vat}{\ensuremath{\vec a(t)}}

\newcommand{\underlinespace}{\underline{\phantom{xxxxxx}}}

\newcommand{\zerooverzero}{\dfrac{\makebox[0pt]{\text{`` }0\text{ ''}}}0\ \ }


\DeclareMathOperator{\sech}{sech}
\DeclareMathOperator{\csch}{csch}
\DeclareMathOperator{\Div}{div}
\DeclareMathOperator{\grad}{grad}
\DeclareMathOperator{\curl}{curl}
\DeclareMathOperator{\divv}{div}

%\newcommand*{\sword}[1]{\textbf{#1}}

\newcommand{\LHequals}{\mathrel{\overset{\text{by LHR}}{=}}}

\newcommand{\surfaceS}{\ensuremath{\mathcal{S}}}


%\newspecialbox[notempty]{exvideo}{ignored}{240}
\AtBeginDocument{\makeStyles{exvideo}{240}}
\newcommand{\setexvideocolor}{%
 \definecolor{topexvideo}{Hsb}{240,.05,1}% %= hsl(#4,100,97.5)
 \definecolor{borderexvideo}{Hsb}{240,.3,1}% %= hsl(#4,90.5,68.4)
 \definecolor{bottomexvideo}{Hsb}{240,.15,1}% %= hsl(#4,81.9,83.4)
}
\newcommand{\setexvideobw}{%
 \definecolor{topexvideo}{Hsb}{0,0,1}% white
 \definecolor{bottomexvideo}{Hsb}{0,0,1}% white
 \definecolor{borderexvideo}{Hsb}{0,1,0}% black
}
\newcommand{\exvideo}[1]{%
 \tcbox[
   colframe=borderexvideo,
   beforeafter skip=\boxskipamount,
   interior style={top color=topexvideo, bottom color=bottomexvideo},
   sharp corners=all,
   notitle,
   width=\textwidth,
   enhanced,
   tcbox width=forced left
  ]{#1}%
}


% \jmtVideo{youtube code}{jmt url suffix}{actual title}
%\newcommand{\jmtVideo}[3]{\genVideo{#1}{http://patrickjmt.com/#2/}{#3}}

%\newcommand{\khanVideo}[3]{\genVideo[?utm_campaign=embed]{#1}{https://www.khanacademy.org/video/#2}{#3}}


% \mfigure[graphicsoptions]{offset}{caption}{label}{file}
\newcommand{\mfigure}[5][]{%
	\mnote[#2]{%
		\centering\myincludegraphics[#1]{#5}%
		\captionsetup{type=figure}\caption{#3}\label{#4}}%
}

% \mtable[offset=0]{caption}{label}{contents}
\newcommand{\mtable}[4][0ex]{%
	\mnote[#1]{\centering\small#4\captionsetup{type=figure}%
		\caption{#2}\label{#3}}%
}

%\ifbool{latexml}{
% \newcommand{\ignoreoptional}[1][]{}
% \newcommand{\marginnote}[1]{\marginpar{#1}\ignoreoptional}
%}{
% \usepackage[noadjust]{marginnote}
%}

% mnote is in apex_style.sty


%\newenvironment{lxfigure}{%
%	\iflatexml%
%		\begin{figure}[!h]%
%	\else%
%		\noindent\begin{minipage}[t]{\linewidth}\noindent%
%	\fi%
%	\captionsetup{type=figure}%
%}{%
%	\iflatexml\end{figure}\else\end{minipage}\fi%
%}

\newcommand{\tbox}[1]{\begin{tabular}{c}#1\end{tabular}} % a tall box
\newcommand*{\zbox}[1]{\makebox[0pt][c]{#1}} % a zero width box






\newtoggle{isEarlyTrans}
\togglefalse{isEarlyTrans}

\newcommand{\prereqIntro}{The material in this section provides a basic review of and practice problems for pre-calculus skills essential to your success in Calculus. You should take time to review this section and work the suggested problems (checking your answers against those in the back of the book). Since this content is a pre-requisite for Calculus, reviewing and mastering these skills are considered your responsibility. This means that minimal, and in some cases no, class time will be devoted to this section. When you identify areas that you need help with we strongly urge you to seek assistance outside of class from your instructor or other student tutoring service.\bigskip}

\ifbool{xetex}%
	{%
	\sffamily
%	\usepackage{fontspec}
%	\usepackage{unicode-math}
	\usepackage{mathspec}
	\setallmainfonts[Mapping=tex-text]{Calibri}
	\setmainfont[Mapping=tex-text]{Calibri}
	% setallmainfonts claims to setmainfont. but it doesn't?
%	\setmathsfont[Mapping=tex-text]{Calibri}
%	\setmathrm[Mapping=tex-text]{Calibri}
	\setsansfont[Mapping=tex-text]{Calibri}
	\setmathsfont(Greek){[cmmi10]}
	}
	{}

\ifbool{luatex}%
	{%
	\sffamily
	\usepackage{fontspec}
	\usepackage{unicode-math}
	%\usepackage{mathspec}
	%\setallmainfonts[Mapping=tex-text]{Calibri}
	\setmainfont{Calibri}
	%\setsansfont[Mapping=tex-text]{Calibri}
	\setmathfont[range=\mathup]{Calibri}
	\setmathfont[range=\mathit]{Calibri Italic}
	}
	{}

\ifbool{latexml}{
 \usepackage[american]{babel}
}{
 \usepackage{polyglossia}
 \setdefaultlanguage[variant=usmax]{english}
 \renewcommand*{\englishhyphenmins}{22}
 \AfterEndPreamble{
 \hyphenation{%
  an-ti-der-iv-a-tive
  an-ti-der-iv-a-tives
  app-rox-i-mate
  cen-tered
  chang-es
  con-struc-tions
  de-creas-es
  Der-iv-a-tive
  der-iv-a-tive
  dis-place-ment
  dis-tance
  e-qual-ly
  ex-am-ples
  Func-tions
 % Hô-pi-tal % doesn't hyphenate L'Hôpital's
  im-pli-cit
  in-dis-tin-guish-a-ble
  in-fall-i-ble
 % %L'Hô-pi-tal's % ' causes: ! Not a letter.
 % % see https://tex.stackexchange.com/a/165091/107497 for fix and pitfalls
  meth-od
  of-ten
  proc-ess
  re-fer-ring
  qua-dra-tic
  sa-li-ent
  se-quence
  sketch-ing
  smart-er
  sub-sti-tute
  The-o-rem
  Trig-o-no-me-tric
  trig-o-no-me-tric
  wheth-er
 }}
}

% lets try to reduce bad boxes
\usepackage{microtype}
\hfuzz=2pt
\vfuzz=1.5\baselineskip
% ignore overfull < this amount
%\newdimen\hfuzz % lock it in?
%\newdimen\vfuzz % lock it in?
%\hbadness=10000
\vbadness=9999
% ignore underfull > this amount
\parskip=0pt plus \baselineskip
\baselineskip=1\baselineskip plus .3\baselineskip


\usepackage[nottoc]{tocbibind}
%\let\oldprintindex\printindex
%\renewcommand{\printindex}{%
% \cleardoublepage
%% \chapter{\indexname} % \printindex has its own heading
% \phantomsection
%% \iflatexml\chapter*{\indexname}\fi
% \addcontentsline{toc}{chapter}{\indexname}
% \oldprintindex
%}


\newtoggle{bsc} % default false
\newcommand{\forwhom}{\iftoggle{bsc}{ for Bismarck State College}{}}


\usepackage[
	bookmarksnumbered,
	hidelinks,
	pdfstartview=FitH,
	linktoc=all,
	pdfdisplaydoctitle,
	bookmarksdepth=2,
]{hyperref}
\hypersetup{
	pdftitle={APEX Calculus LT},
	pdfauthor={UND Math Dept and Greg Hartman, VMI},
	unicode,
    pdflang=EN-US
}
\ifbool{latexml}{}{
 \usepackage{bookmark}
}


% hyperref changes these
% if they come before and have newcommand, latexml overwrites them
\AtBeginDocument{
 \renewcommand{\chapterautorefname}{Chap\-ter} % the default is lowercase
 \renewcommand{\sectionautorefname}{Sec\-tion} % the default is lowercase
 \renewcommand{\figureautorefname}{Fig\-ure}
 \renewcommand{\appendixname}{Ap\-pen\-di\-ces}
}
\newcommand{\exampleEnvautorefname}{Ex\-am\-ple}
\newcommand{\autoeqref}[1]{\hyperref[#1]{\equationautorefname~(\ref*{#1})}}
% autoref doesn't use parentheses

% \apex has to be *used* after hyperref
% lxNavbar has to come after latexml
\begin{lxNavbar}
\lxRef{lxApexTOC}{Table of Contents}\\
\lxContextTOC
\end{lxNavbar}

\lxIncludeCssFile{style.css}
\lxIncludeCssFile{LaTeXML-marginpar.css}
\lxIncludeCssFile{LaTeXML-navbar-left.css}
\lxIncludeJavascriptFile{%
https://ajax.googleapis.com/ajax/libs/jquery/1.12.2/jquery.min.js}
\lxIncludeJavascriptFile{script.js}
\lxIncludeJavascriptFile{LaTeXML-maybeMathJax.js}

% set the defaults, just in case
\printincolor
\usetwoDgraphics

\input{options}
\newcommand{\forwhom}{}

\printallanswers
\printincolor
%\usetwoDgraphics
\usethreeDgraphics

\begin{document}

%\frontmatter

%\iflatexml\tableofcontents\chapter*{\apex~Calculus}\fi

\newgeometry{right=1.2in}

\vspace*{\stretch{1.5}}

\begin{flushright}

\ifthenelse{\boolean{latexml}}{%
 \chapter*[APEX \thetitle]{\textsc{\large \apex \Huge\ \thetitle}}
 \addcontentsline{toc}{chapter}{APEX \thetitle}
}{
 \textsc{\large \apex \Huge\ \thetitle}
}
\label{coverpagetitle}
\\

\ifthenelse{\boolean{isEarlyTrans}}{Early}{Late} Transcendentals \\

\vspace{\stretch{1}}%

{\Large University of North Dakota}\bigskip

\normalsize

Adapted from \apex\ Calculus by

Gregory Hartman, Ph.D.

\emph{\small Department of Applied Mathematics}

\emph{\small Virginia Military Institute}

\vspace{\stretch{4}}%

{\small Revised \monthYear\forwhom}

\end{flushright}
\normalsize

\restoregeometry


\thispagestyle{empty}
\clearpage

%\iflatexml\chapter*{Licensing}\fi
\newgeometry{right=2in,top=2in}% this is an even numbered page, so left <-> right

\begingroup

\settowidth{\cellwidth}{\small\emph{Department of Applied Mathematics}\quad}

\ifbool{latexml}{
 \chapter*{Contributing Authors}
 \addcontentsline{toc}{chapter}{Contributing Authors}
 \noindent
}{%
 \noindent\emph{Contributing Authors}\\[\baselineskip]%
}%
\parbox[t]{\cellwidth}{Troy Siemers, Ph.D.\\\small
\emph{Department of Applied Mathematics}\\
\emph{Virginia Military Institute}}
\parbox[t]{\cellwidth}{Michael Corral\\\small
\emph{Mathematics}\\
\emph{Schoolcraft College}}
\\[\baselineskip]
\parbox[t]{\cellwidth}{Brian Heinold, Ph.D.\\\small
\emph{Department of Mathematics}\\
\emph{and Computer Science}\\
\emph{Mount Saint Mary's University}}
\parbox[t]{\cellwidth}{Paul Dawkins, Ph.D.\\\small
\emph{Department of Mathematics}\\
\emph{Lamar University}}
\\[\baselineskip]
\parbox[t]{\cellwidth}{Dimplekumar Chalishajar, Ph.D.\\\small
\emph{Department of Applied Mathematics}\\
\emph{Virginia Military Institute}}
\\[2\baselineskip]
\parbox[t]{\cellwidth}{\emph{Editor}\\
Jennifer Bowen, Ph.D.\\\small
\emph{Department of Mathematics}\\
\emph{and Computer Science}\\
\emph{The College of Wooster}}
%\end{tabular}

\vspace{1in}

\noindent
\begin{minipage}[t]{\cellwidth}\mbox{}\\
\href{http://creativecommons.org/licenses/by-nc/4.0/}{\includegraphics[alt={Creative Commons by-nc license}]{figures/raw/by-nc}}
\end{minipage}%
\begin{minipage}[t]{.4\linewidth}\raggedright\mbox{}\\
\noindent Copyright\\
\copyright~2015 Gregory Hartman\\
\copyright~2025 Department of Mathematics,\\
University of North Dakota\medskip

\noindent
This work is licensed under a\iflatexml\ \else\\\fi
\href{http://creativecommons.org/licenses/by-nc/4.0/}{Creative~Commons
Attribution-NonCommercial
4.0~International~License}.\\
Resale and reproduction restricted.
\end{minipage}

\endgroup

\restoregeometry

\thispagestyle{empty}
\clearpage

\addtocontents{toc}{\protect\thispagestyle{empty}}

%\ifthenelse{\boolean{latexml}}{%
%\iflatexml\chapter*{Table of Contents}\fi%}{%
\iflatexml\else
\addcontentsline{toc}{chapter}{Table of Contents}%
%}
\tableofcontents
\fi
\clearpage{\pagestyle{empty}\cleardoublepage}

\prefacegeometry
\chapter*{Preface}
\addcontentsline{toc}{chapter}{Preface}
\pagestyle{plain} % doesn't take?
\thispagestyle{empty}

\subsection{A Note on Using this Text}

Thank you for reading this short preface. Allow us to share a few key points about the text so that you may better understand what you will find beyond this page.

This text comprises a three-volume series on Calculus.
The first part covers material taught in many ``Calculus 1'' courses: limits, derivatives, and the basics of integration, found in Chapters~\ref{chapter:limits} through
\iftoggle{bsc}{\ref{chapter:integration}}{\ref{chapter:app_of_int}}.
The second text covers material often taught in ``Calculus 2'': integration and its applications, along with an introduction to sequences, series and Taylor Polynomials, found in
Chapters~\iftoggle{bsc}{\ref{chapter:app_of_int}}{\ref{chapter:diff_conc}}
through \ref{chapter:planar_curves}. The third text covers topics common in ``Calculus 3'' or ``Multivariable Calculus'': parametric equations, polar coordinates, vector-valued functions, and functions of more than one variable, found in Chapters \ref{chapter:vectors} through \ref{chapter:vector_calc}. All three are available separately for free.

Printing the entire text as one volume makes for a large, heavy, cumbersome book. One can certainly only print the pages they currently need, but some prefer to have a nice, bound copy of the text. Therefore this text has been split into these three manageable parts, each of which can be purchased separately.

A result of this splitting is that sometimes material is referenced that is not contained in the present text. The context should make it clear whether the ``missing'' material comes before or after the current portion. Downloading the appropriate pdf, or the entire \emph{\apex\ Calculus LT} pdf, will give access to these topics.

\subsection{For Students: How to Read this Text}

Mathematics textbooks have a reputation for being hard to read. High-level mathematical writing often seeks to say much with few words, and this style often seeps into texts of lower-level topics. This book was written with the goal of being easier to read than many other calculus textbooks, without becoming too verbose. 

Each chapter and section starts with an introduction of the coming material, hopefully setting the stage for ``why you should care,'' and ends with a look ahead to see how the just-learned material helps address future problems. Additionally, each chapter includes a section zero, which provides a basic review and practice problems of pre-calculus skills. Since this content is a pre-requisite for calculus, reviewing and mastering these skills are considered your responsibility. This means that it is your responsibility to seek assistance outside of class from your instructor, a math resource center or other math tutoring available on-campus.  A solid understanding of these skills is essential to your success in solving calculus problems.

\emph{Please read the text;} it is written to explain the concepts of Calculus. There are numerous examples to demonstrate the meaning of definitions, the truth of theorems, and the application of mathematical techniques. When you encounter a sentence you don't understand, read it again. If it still doesn't make sense, read on anyway, as sometimes confusing sentences are explained by later sentences.

\emph{You don't have to read every equation.} The examples generally show ``all'' the steps needed to solve a problem. Sometimes reading through each step is helpful; sometimes it is confusing. When the steps are illustrating a new technique, one probably should follow each step closely to learn the new technique. When the steps are showing the mathematics needed to find a number to be used later, one can usually skip ahead and see how that number is being used, instead of getting bogged down in reading how the number was found.

\emph{Some proofs have been delayed until later (or omitted completely).} In mathematics, \emph{proving} something is always true is extremely important, and entails much more than testing to see if it works twice. However, students often are confused by the details of a proof, or become concerned that they should have been able to construct this proof on their own. To alleviate this potential problem, we do not include the more difficult proofs in the text. The interested reader is highly encouraged to find other proofs online or from their instructor. In most cases, one is very capable of understanding what a theorem \emph{means} and \emph{how to apply it} without knowing fully \emph{why} it is true.

\emph{Work through the examples.}  The best way to learn mathematics is to do it.  Reading about it (or watching someone else do it) is a poor substitute.  For this reason, every page has a place for \emph{you} to put \emph{your} notes so that \emph{you} can work out the examples.  That being said, sometimes it is useful to watch someone work through an example.  For this reason, this text also provides links to online videos where someone is working through a similar problem.  If you want even more videos, these are generally chosen from
\iflatexml\begin{itemize}\else\begin{itemize}[nosep]\fi
\item Khan Academy: \url{https://www.khanacademy.org/}
\item Math Doctor Bob: \url{http://www.mathdoctorbob.org/}
\item Just Math Tutorials: \url{http://patrickjmt.com/} (unfortunately, they're not well organized)
\end{itemize}
Some other sites you may want to consider are
\iflatexml\begin{itemize}\else\begin{itemize}[nosep]\fi
\item Larry Green's Calculus Videos: \url{http://www.ltcconline.net/greenl/courses/105/videos/VideoIndex.htm}
\item Mathispower4u: \url{http://www.mathispower4u.com/}
\item Yay Math: \url{http://www.yaymath.org/} (for prerequisite material)
\end{itemize}
All of these sites are completely free (although some will ask you to donate).  Here's a sample one:

\youtubeVideo{ILNfpJTZLxk}{Practical Advice for Those Taking College Calculus}


\subsection{Thanks from Greg Hartman}

There are many people who deserve recognition for the important role they have played in the development of this text. First, I thank Michelle for her support and encouragement, even as this ``project from work'' occupied my time and attention at home. Many thanks to Troy Siemers, whose most important contributions extend far beyond the sections he wrote or the 227 figures he coded in Asymptote for 3D interaction.  He provided incredible support, advice and encouragement for which I am very grateful. My thanks to Brian Heinold and Dimplekumar Chalishajar for their contributions and to Jennifer Bowen for reading through so much material and providing great feedback early on. Thanks to Troy, Lee Dewald, Dan Joseph, Meagan Herald, Bill Lowe, John David, Vonda Walsh, Geoff Cox, Jessica Libertini and other faculty of VMI who have given me numerous suggestions and corrections based on their experience with teaching from the text. (Special thanks to Troy, Lee \& Dan for their patience in teaching Calc III while I was still writing the Calc III material.) Thanks to Randy Cone for encouraging his tutors of VMI's Open Math Lab to read through the text and check the solutions, and thanks to the tutors for spending their time doing so. A very special thanks to Kristi Brown and Paul Janiczek who took this opportunity far above \& beyond what I expected, meticulously checking every solution and carefully reading every example. Their comments have been extraordinarily helpful. I am also thankful for the support provided by Wane Schneiter, who as my Dean provided me with extra time to work on this project. I am blessed to have so many people give of their time to make this book better.

\subsection{\apex\ --- Affordable Print and Electronic teXts}

\apex\ is a consortium of authors  who collaborate to produce high-quality, low-cost textbooks. The current textbook-writing paradigm is facing a potential revolution as desktop publishing and electronic formats increase in popularity. However, writing a good textbook is no easy task, as the time requirements alone are substantial. It takes countless hours of work to produce text, write examples and exercises, edit and publish. Through collaboration, however, the cost to any individual can be lessened, allowing us to create texts that we freely distribute electronically and sell in printed form for an incredibly low cost. Having said that, nothing is entirely free; someone always bears some cost. This text ``cost'' the authors of this book their time, and that was not enough. \emph{\apex\ Calculus} would not exist had not the Virginia Military Institute, through a generous Jackson-Hope grant, given the lead author significant time away from teaching so he could focus on this text.

Each text is available as a free .pdf, protected by a Creative Commons Attribution --- Noncommercial 4.0 copyright. That means you can give the .pdf to anyone you like, print it in any form you like, and even edit the original content and redistribute it. If you do the latter, you must clearly reference this work and you cannot sell your edited work for money.

We encourage others to adapt this work to fit their own needs. One might add sections that are ``missing'' or remove sections that your students won't need. The source files can be found at \url{https://github.com/APEXCalculus}.

You can learn more at \texttt{\href{http://www.vmi.edu/APEX}{www.vmi.edu/APEX}}.

~\hfill Greg Hartman

\subsection{Creating \apex~LT}

Starting with the source at \url{https://github.com/APEXCalculus},
faculty at the University of North Dakota made several substantial changes to create \apex\ Late Transcendentals.  The most obvious change was to rearrange the text to delay proving the derivative of transcendental functions until Calculus 2.  UND added Sections \ref{sec:inv_funcs} and \ref{sec:exp_log}, adapted several sections from other resources, created the prerequisite sections, included links to videos and Geogebra, and added several examples and exercises.  In the end, every section had some changes (some more substantial than others), resulting in a document that is about 10\% longer. The source files can now be found at\iflatexml\ \else\\\fi
\url{https://github.com/teepeemm/APEXCalculusLT_Source}.
%Jerry Metzger provided many of the links to the videos.

Extra thanks are due
to Michael Corral for allowing us to use portions of his Vector Calculus, available at \texttt{\href{http://www.mecmath.net/}{www.mecmath.net/}}
(specifically, %\ref{sec:other_systems},
\autoref{sec:lagrange} and the Jacobian in \autoref{sec:cylindrical_spherical})
and
to Paul Dawkins for allowing us to use portions of his online math notes from \texttt{\href{http://tutorial.math.lamar.edu/}{tutorial.math.lamar.edu/}} (specifically, Sections \ref{sec:int_techniques} and \ref{sec:series_techniques}, as well as ``Area with Parametric Equations'' in \autoref{sec:par_calc}).
The work on Calculus III was partially supported by the NDUS OER Initiative.


\subsection{Electronic Resources}

A distinctive feature of \apex\ is interactive, 3D graphics in the .pdf version. Nearly all graphs of objects in space can be rotated, shifted, and zoomed in/out so the reader can better understand the object illustrated. 

Currently, the only pdf viewers that support these 3D graphics for computers are Adobe Reader \& Acrobat. To activate the interactive mode, click on the image. Once activated, one can click/drag to rotate the object and use the scroll wheel on a mouse to zoom in/out. (A great way to investigate an image is to first zoom in on the page of the pdf viewer so the graphic itself takes up much of the screen, then zoom inside the graphic itself.) A CTRL-click/drag pans the object left/right or up/down. By right-clicking on the graph one can access a menu of other options, such as changing the lighting scheme or perspective. One can also revert the graph back to its default view. If you wish to deactivate the interactivity, one can right-click and choose the ``Disable Content'' option.

\newcommand{\threedurl}{https://sites.und.edu/timothy.prescott/apex/prc/}

\iflatexml\else
\noindent
\begin{minipage}[t]{.74\linewidth}%
\setlength{\parindent}{\saveparindent}
\indent
\fi
The situation is more interesting for tablets and smartphones.  The 3D graphics files have been arrayed at \url{\threedurl}.  At the bottom of the page are links to Android and iOS apps that can display the interactive files.
\iflatexml\else
The QR code to the right will take you to that page.
\end{minipage}
\quad
\begin{minipage}[t]{2cm}%
\vspace{-.5\baselineskip}\qrcode{\threedurl}%
\end{minipage}
\fi

\iflatexml\else
Additionally, a web version of the book is available at \url{https://sites.und.edu/timothy.prescott/apex/web/}.  While we have striven to make the pdf accessible for non-print formats, html is far better in this regard.
\fi


\restoregeometry

\iffalse

The rest of this file is commented out.

Michael Corral's permission for Vector Calc:

Delivered-To: teepeemm+und@gmail.com
Received: by 10.12.137.196 with SMTP id 4csp206633qvs;
        Tue, 11 Oct 2016 21:50:21 -0700 (PDT)
X-Received: by 10.28.208.204 with SMTP id h195mr902815wmg.25.1476247821239;
        Tue, 11 Oct 2016 21:50:21 -0700 (PDT)
Return-Path: <timothy.prescott.und+caf_=teepeemm+und=gmail.com@gmail.com>
Received: from mail-wm0-f46.google.com (mail-wm0-f46.google.com. [74.125.82.46])
        by mx.google.com with ESMTPS id iu2si7937925wjb.79.2016.10.11.21.50.20
        for <teepeemm+und@gmail.com>
        (version=TLS1_2 cipher=ECDHE-RSA-AES128-GCM-SHA256 bits=128/128);
        Tue, 11 Oct 2016 21:50:21 -0700 (PDT)
Received-SPF: pass (google.com: domain of timothy.prescott.und+caf_=teepeemm+und=gmail.com@gmail.com designates 74.125.82.46 as permitted sender) client-ip=74.125.82.46;
Authentication-Results: mx.google.com;
       spf=pass (google.com: domain of timothy.prescott.und+caf_=teepeemm+und=gmail.com@gmail.com designates 74.125.82.46 as permitted sender) smtp.mailfrom=timothy.prescott.und+caf_=teepeemm+und=gmail.com@gmail.com
Received: by mail-wm0-f46.google.com with SMTP id c78so5170585wme.1
        for <teepeemm+und@gmail.com>; Tue, 11 Oct 2016 21:50:20 -0700 (PDT)
X-Google-DKIM-Signature: v=1; a=rsa-sha256; c=relaxed/relaxed;
        d=1e100.net; s=20130820;
        h=x-original-authentication-results:x-gm-message-state:delivered-to
         :date:from:to:subject:in-reply-to:message-id:references:user-agent
         :mime-version;
        bh=4QZTTpi4S0EQNJbS8tptvA7JD1EZlV+X7XZHmQ8AkGM=;
        b=gnrdxJXm28XRjisuz3lkoHQGazVvxlHaU9qMt5eZbcjAVXe3KQXotfrPD/BI7qlcBv
         326C0dy43JJCte8GViUowUtK760ErySLnTcijMjJ5mALRmfugtcqjk+jYXN457iLlWIz
         P/dyAn1gAUp8dU7n3nNc0J1wlMBKNAobVHEg/pDKMPBLC76Rx6A0vkcDrU/VxjtaNoJf
         Pm/rzcGxcSRxVkWdRTFtw7lJYTae7BV3gHd2ptASaSglpGvZCyE/Q0wRCGfZZ9WdRtz2
         cBmqwpId+BSIxAzhqicY+PcVj8b4/Dt1JIg6KRamlb72cKizaAREBTUsTYuMRp58QRn8
         GIwg==
X-Original-Authentication-Results: mx.google.com;
       spf=neutral (google.com: 173.247.247.235 is neither permitted nor denied by best guess record for domain of mcorral@mecmath.net) smtp.mailfrom=mcorral@mecmath.net
X-Gm-Message-State: AA6/9RmAB4vzE4sB8058XycdVu6s2+1v2SYnRZQBET7qioZ9UKk72GLVyunHUnk/NPDHWdnr+q+aEOQG//nqrkZ3aEE3/QM=
X-Received: by 10.194.157.193 with SMTP id wo1mr8038181wjb.22.1476247820810;
        Tue, 11 Oct 2016 21:50:20 -0700 (PDT)
X-Forwarded-To: teepeemm+und@gmail.com
X-Forwarded-For: timothy.prescott.und@gmail.com teepeemm+und@gmail.com
Delivered-To: timothy.prescott.und@gmail.com
Received: by 10.80.183.175 with SMTP id h44csp359244ede;
        Tue, 11 Oct 2016 21:50:19 -0700 (PDT)
X-Received: by 10.99.110.78 with SMTP id j75mr6174565pgc.2.1476247819627;
        Tue, 11 Oct 2016 21:50:19 -0700 (PDT)
Return-Path: <mcorral@mecmath.net>
Received: from biz104.inmotionhosting.com (biz104.inmotionhosting.com. [173.247.247.235])
        by mx.google.com with ESMTPS id t5si3802730pgb.173.2016.10.11.21.50.18
        for <timothy.prescott.und@gmail.com>
        (version=TLS1 cipher=AES128-SHA bits=128/128);
        Tue, 11 Oct 2016 21:50:19 -0700 (PDT)
Received-SPF: neutral (google.com: 173.247.247.235 is neither permitted nor denied by best guess record for domain of mcorral@mecmath.net) client-ip=173.247.247.235;
Received: from c-73-191-129-108.hsd1.mi.comcast.net ([73.191.129.108]:35732 helo=banana.sluggo.net) by biz104.inmotionhosting.com with esmtpa (Exim 4.87) (envelope-from <mcorral@mecmath.net>) id 1buBUe-0006uW-AF for timothy.prescott.und@gmail.com; Tue, 11 Oct 2016 21:50:17 -0700
Date: Wed, 12 Oct 2016 00:56:58 -0400 (EDT)
From: Michael Corral <mcorral@mecmath.net>
X-X-Sender: mcorral@banana.sluggo.net
To: Timothy Prescott <timothy.prescott.und@gmail.com>
Subject: Re: using Vector Calculus with a different open license
In-Reply-To: <CAJv=1OksH0MhTddy9Yu4=BoHo8733_oCjHU8aCUbjv2ZxdY=Rg@mail.gmail.com>
Message-ID: <alpine.LFD.2.20.1610120051340.7844@banana.sluggo.net>
References: <CAJv=1OksH0MhTddy9Yu4=BoHo8733_oCjHU8aCUbjv2ZxdY=Rg@mail.gmail.com>
User-Agent: Alpine 2.20 (LFD 67 2015-01-07)
X-Mailer: Alpine 2.02 <Fedora 15 x86_64>
MIME-Version: 1.0
Content-Type: multipart/mixed; BOUNDARY="-1463747071-1550951339-1476248223=:7844"
X-OutGoing-Spam-Status: No, score=-1.0
X-AntiAbuse: This header was added to track abuse, please include it with any abuse report
X-AntiAbuse: Primary Hostname - biz104.inmotionhosting.com
X-AntiAbuse: Original Domain - gmail.com
X-AntiAbuse: Originator/Caller UID/GID - [47 12] / [47 12]
X-AntiAbuse: Sender Address Domain - mecmath.net
X-Get-Message-Sender-Via: biz104.inmotionhosting.com: authenticated_id: mcorral@mecmath.net
X-Authenticated-Sender: biz104.inmotionhosting.com: mcorral@mecmath.net
X-Source: 
X-Source-Args: 
X-Source-Dir: 

---1463747071-1550951339-1476248223=:7844
Content-Type: text/plain; charset=UTF-8; format=flowed
Content-Transfer-Encoding: 8BIT

Hello Professor Prescott,

Sorry for the delay in replying, my mailbox gets bombarded with
spam and it often takes me a while to sift through it all.

Yes, you have my permission to use those sections from my Vector
Calculus book with the same license as the Apex book.

I'm glad you found that material useful.

Thanks,

Michael Corral
Schoolcraft College

On 09/28/16, Timothy Prescott wrote:
> Professor Corral,
> 
> The University of North Dakota is in the process of adopting Apex Calculus
> for our entire calculus sequence.  Unfortunately for us, it does not
> currently have a chapter on Line and Surface Integrals (it is also missing a
> section on Lagrange multipliers, and the change of variables formula for
> multiple integration).  We were wondering if we would be able to use those
> sections from your Vector Calculus book (along with some of the exercises as
> well, probably).
> 
> I realize that this is currently allowed under the book?s GNU Free
> Documentation License.  But Apex Calculus uses a CC-BY-NC 4.0 license. 
> Would you be willing to allow us to use your material with the same license
> as Apex?
> 
> Thank you for your time,
> 
> Tim Prescott
> Associate Professor of Mathematics
> University of North Dakota
> 
>
---1463747071-1550951339-1476248223=:7844--


Paul Dawkin's permission for Paul's Online Math Notes:

Delivered-To: teepeemm+und@gmail.com
Received: by 10.12.133.134 with SMTP id o6csp1283228qva;
        Tue, 8 Nov 2016 05:18:53 -0800 (PST)
X-Received: by 10.194.2.198 with SMTP id 6mr12218551wjw.51.1478611133415;
        Tue, 08 Nov 2016 05:18:53 -0800 (PST)
Return-Path: <timothy.prescott.und+caf_=teepeemm+und=gmail.com@gmail.com>
Received: from mail-wm0-f41.google.com (mail-wm0-f41.google.com. [74.125.82.41])
        by mx.google.com with ESMTPS id mc8si35239494wjb.127.2016.11.08.05.18.53
        for <teepeemm+und@gmail.com>
        (version=TLS1_2 cipher=ECDHE-RSA-AES128-GCM-SHA256 bits=128/128);
        Tue, 08 Nov 2016 05:18:53 -0800 (PST)
Received-SPF: pass (google.com: domain of timothy.prescott.und+caf_=teepeemm+und=gmail.com@gmail.com designates 74.125.82.41 as permitted sender) client-ip=74.125.82.41;
Authentication-Results: mx.google.com;
       spf=pass (google.com: domain of timothy.prescott.und+caf_=teepeemm+und=gmail.com@gmail.com designates 74.125.82.41 as permitted sender) smtp.mailfrom=timothy.prescott.und+caf_=teepeemm+und=gmail.com@gmail.com
Authentication-Results: spf=none (sender IP is ) smtp.mailfrom=michele.iiams@email.und.edu;
Received: by mail-wm0-f41.google.com with SMTP id p190so242528778wmp.1
        for <teepeemm+und@gmail.com>; Tue, 08 Nov 2016 05:18:53 -0800 (PST)
X-Google-DKIM-Signature: v=1; a=rsa-sha256; c=relaxed/relaxed;
        d=1e100.net; s=20130820;
        h=x-original-authentication-results:x-gm-message-state:delivered-to
         :from:to:subject:thread-topic:thread-index:date:message-id
         :references:in-reply-to:accept-language:content-language
         :spamdiagnosticoutput:mime-version;
        bh=4kSirSYfDGZUsnfr1Vb8STyHxn3qnhDe1dcvINCiK5U=;
        b=lj0ITCnwrXmaSy8uXbLHT2ObUU1tvmsYhcrtXalC2PCKUwkyC07i0mjJSMoOJ0XHvz
         pqvpI18JYRD2Mhkf8/3pdmgvDvWrXHoIOiCmGmWnQ0aK1boZ8ZfCcijzJspIeG/e2Qu+
         l4+71bcS/5PTUEXKVcLUv8ektKTg/uIxB5iZHK7q+lR0wt40S7RnNCzjqXOfbsfHw/qu
         ewAkPNyDBC4r927dTcR6GEnZut2p2wAGZ1fowHIpPc85Wx42wyfkMq4h4SaJyMuf5aOM
         2P+wLTCKq3++o6lXM02wPflYzGnjuzDGG/QZt/L3crUaJMw2oG7I7C36J6OChYb0G1dG
         TgPw==
X-Original-Authentication-Results: mx.google.com;
       spf=pass (google.com: domain of michele.iiams@email.und.edu designates 104.47.38.126 as permitted sender) smtp.mailfrom=michele.iiams@email.und.edu
X-Gm-Message-State: ABUngve/ExvFC7TJh3/a4vpVWfb5VK1kUMFTgWHSzA56pyoE2plfGvOxaR8562yczUM5tOcXMNjq4z9f4MORNhJESDIiLN8=
X-Received: by 10.28.135.207 with SMTP id j198mr14293705wmd.109.1478611132870;
        Tue, 08 Nov 2016 05:18:52 -0800 (PST)
X-Forwarded-To: teepeemm+und@gmail.com
X-Forwarded-For: timothy.prescott.und@gmail.com teepeemm+und@gmail.com
Delivered-To: timothy.prescott.und@gmail.com
Received: by 10.80.183.137 with SMTP id h9csp859157ede;
        Tue, 8 Nov 2016 05:18:52 -0800 (PST)
X-Received: by 10.107.195.206 with SMTP id t197mr12580417iof.221.1478611132036;
        Tue, 08 Nov 2016 05:18:52 -0800 (PST)
Return-Path: <michele.iiams@email.und.edu>
Received: from NAM02-BL2-obe.outbound.protection.outlook.com (mail-bl2nam02on0126.outbound.protection.outlook.com. [104.47.38.126])
        by mx.google.com with ESMTPS id r6si9819679ith.85.2016.11.08.05.18.51
        for <timothy.prescott.und@gmail.com>
        (version=TLS1_2 cipher=ECDHE-RSA-AES128-SHA bits=128/128);
        Tue, 08 Nov 2016 05:18:51 -0800 (PST)
Received-SPF: pass (google.com: domain of michele.iiams@email.und.edu designates 104.47.38.126 as permitted sender) client-ip=104.47.38.126;
Received: from MWHPR08MB2974.namprd08.prod.outlook.com (10.173.240.140) by MWHPR08MB2782.namprd08.prod.outlook.com (10.173.239.12) with Microsoft SMTP Server (version=TLS1_2, cipher=TLS_ECDHE_RSA_WITH_AES_256_CBC_SHA384_P384) id 15.1.707.6; Tue, 8 Nov 2016 13:18:48 +0000
Received: from MWHPR08MB2974.namprd08.prod.outlook.com ([10.173.240.140]) by MWHPR08MB2974.namprd08.prod.outlook.com ([10.173.240.140]) with mapi id 15.01.0707.006; Tue, 8 Nov 2016 13:18:47 +0000
From: "Iiams, Michele" <michele.iiams@email.und.edu>
To: "Prescott, Timothy" <timothy.prescott@email.und.edu>
Subject: Fwd: Permission to use another section of your Calculus II notes
Thread-Topic: Permission to use another section of your Calculus II notes
Thread-Index: AQHSOVab9+h2/DCJmECwdGJyzg58G6DPEU4AgAABS4I=
Date: Tue, 8 Nov 2016 13:18:47 +0000
Message-ID: <ek6uw3t7hsit4k5e9q11wp9b.1478611124428@email.android.com>
References: <MWHPR08MB29747D6C8920390167ED8DF3AFA60@MWHPR08MB2974.namprd08.prod.outlook.com>,<CALDs=E=oujWn2etJAwwtJrWWT1y+Jqb5nzK-aCO1fGvadOiDqA@mail.gmail.com>
In-Reply-To: <CALDs=E=oujWn2etJAwwtJrWWT1y+Jqb5nzK-aCO1fGvadOiDqA@mail.gmail.com>
Accept-Language: en-US
Content-Language: en-US
X-MS-Has-Attach: 
X-MS-Exchange-Inbox-Rules-Loop: timothy.prescott@email.und.edu
X-MS-TNEF-Correlator: 
x-ms-exchange-messagesentrepresentingtype: 1
x-originating-ip: [2600:1014:b047:6b3a:38a4:65f8:a92d:c144]
x-ms-office365-filtering-correlation-id: 4d08d5b9-fa36-4096-40ed-08d407d9c605
x-microsoft-exchange-diagnostics: 1;MWHPR08MB2782;24:ig3Mns3v7+IIIVzP1b8sDE6nD0GME60ic+KXHmQY90CKzkm56Yay4XRxdd7saKmQYgdSN8i02ep0saS12JuJnPWlcvZp1sQ046yX0WdMqY4=
x-microsoft-antispam: UriScan:;BCL:0;PCL:0;RULEID:;SRVR:MWHPR08MB2782;
x-exchange-antispam-report-test: UriScan:(34617014829592);
x-exchange-antispam-report-cfa-test: BCL:0;PCL:0;RULEID:(9101524098)(601004)(2401047)(8121501046)(3002001)(10201501046);SRVR:MWHPR08MB2782;BCL:0;PCL:0;RULEID:;SRVR:MWHPR08MB2782;
x-forefront-antispam-report: SFV:SKI;SFS:;DIR:INB;SFP:;SCL:-1;SRVR:MWHPR08MB2782;H:MWHPR08MB2974.namprd08.prod.outlook.com;FPR:;SPF:None;LANG:en;SFV:NSPM;SFS:(10019020)(7916002)(377454003)(24454002)(2473002)(199003)(189002)(86362001)(122556002)(88552002)(2950100002)(102836003)(101416001)(54356999)(3280700002)(99286002)(42882006)(105586002)(77096005)(68736007)(8936002)(110136003)(75432002)(8676002)(9686002)(586003)(7906003)(6116002)(6636002)(2900100001)(33646002)(63666004)(7846002)(107886002)(2171001)(95246002)(450100001)(81156014)(7736002)(89836001)(81166006)(87936001)(97736004)(5660300001)(3660700001)(50986999)(76176999)(189998001)(106356001)(106116001)(89122001)(92566002)(51650200001);DIR:OUT;SFP:1102;SCL:1;SRVR:MWHPR08MB2782;H:MWHPR08MB2974.namprd08.prod.outlook.com;FPR:;SPF:None;PTR:InfoNoRecords;A:1;MX:1;LANG:en;
spamdiagnosticoutput: 1:0
x-forefront-prvs: 01208B1E18
received-spf: None (protection.outlook.com: email.und.edu does not designate permitted sender hosts)
Content-Type: multipart/alternative; boundary="_000_ek6uw3t7hsit4k5e9q11wp9b1478611124428emailandroidcom_"
MIME-Version: 1.0
X-OriginatorOrg: email.und.edu
X-MS-Exchange-CrossTenant-originalarrivaltime: 08 Nov 2016 13:18:47.7012 (UTC)
X-MS-Exchange-CrossTenant-fromentityheader: Hosted
X-MS-Exchange-CrossTenant-id: ec37a091-b9a6-47e5-98d0-903d4a419203
X-MS-Exchange-Transport-CrossTenantHeadersStamped: MWHPR08MB2782

--_000_ek6uw3t7hsit4k5e9q11wp9b1478611124428emailandroidcom_
Content-Type: text/plain; charset="us-ascii"
Content-Transfer-Encoding: quoted-printable





Sent from my Verizon, Samsung Galaxy smartphone


-------- Original message --------
From: Paul Dawkins <pdawkins@gmail.com>
Date: 11/8/16 7:14 AM (GMT-06:00)
To: "Iiams, Michele" <michele.iiams@email.und.edu>
Subject: Re: Permission to use another section of your Calculus II notes

You are welcome to do that!

Paul.

On Mon, Nov 7, 2016 at 6:25 PM, Iiams, Michele <michele.iiams@email.und.edu=
<mailto:michele.iiams@email.und.edu>> wrote:

Paul,


This is Michele Iiams from the University of North Dakota. We are in the fi=
nal phase of revising  the OER text for Calculus II and have discovered one=
 more missing piece. We need a section on Areas of Parametric Curves and Cy=
cloids. I am writing to ask permission to use your work at  http://tutorial=
.math.lamar.edu/Classes/CalcII/ParaArea.aspx to fill this need.


We are grateful for your previous consent to use integration and series sec=
tions. Once our work is complete we will send you a link to the finished pr=
oduct.


Michele


Michele Iiams
Mathematics Department
University of North Dakota

"Never trust atoms. They make up everything." AmericInn sign in Grand Forks=
, ND


--_000_ek6uw3t7hsit4k5e9q11wp9b1478611124428emailandroidcom_
Content-Type: text/html; charset="us-ascii"
Content-Transfer-Encoding: quoted-printable

<html>
<head>
<meta http-equiv=3D"Content-Type" content=3D"text/html; charset=3Dus-ascii"=
>
<meta content=3D"text/html; charset=3Dutf-8">
</head>
<body>
<div><br>
</div>
<div><br>
</div>
<div><br>
</div>
<div><br>
</div>
<div id=3D"composer_signature">
<div dir=3D"auto" style=3D"font-size:85%; color:#575757">Sent from my Veriz=
on, Samsung Galaxy smartphone</div>
</div>
<div><br>
</div>
<div><br>
</div>
<div>-------- Original message --------</div>
<div>From: Paul Dawkins &lt;pdawkins@gmail.com&gt; </div>
<div>Date: 11/8/16 7:14 AM (GMT-06:00) </div>
<div>To: &quot;Iiams, Michele&quot; &lt;michele.iiams@email.und.edu&gt; </d=
iv>
<div>Subject: Re: Permission to use another section of your Calculus II not=
es </div>
<div><br>
</div>
<div>
<div dir=3D"ltr">You are welcome to do that!
<div><br>
</div>
<div>Paul.</div>
</div>
<div class=3D"gmail_extra"><br>
<div class=3D"gmail_quote">On Mon, Nov 7, 2016 at 6:25 PM, Iiams, Michele <=
span dir=3D"ltr">
&lt;<a href=3D"mailto:michele.iiams@email.und.edu" target=3D"_blank">michel=
e.iiams@email.und.edu</a>&gt;</span> wrote:<br>
<blockquote class=3D"gmail_quote" style=3D"margin:0 0 0 .8ex; border-left:1=
px #ccc solid; padding-left:1ex">
<div dir=3D"ltr">
<div id=3D"m_8033978807228229300divtagdefaultwrapper" dir=3D"ltr" style=3D"=
font-size:12pt; color:#000000; font-family:Calibri,Arial,Helvetica,sans-ser=
if">
<p>Paul,</p>
<p><br>
</p>
<p>This is Michele Iiams from the University of North Dakota.&nbsp;We are i=
n the final phase of revising&nbsp;&nbsp;the OER text for Calculus II and h=
ave discovered one&nbsp;more missing piece. We need a section on Areas of P=
arametric Curves and Cycloids. I am writing to ask permission
 to use your work at&nbsp;&nbsp;<a href=3D"http://tutorial.math.lamar.edu/C=
lasses/CalcII/ParaArea.aspx" class=3D"m_8033978807228229300OWAAutoLink" id=
=3D"m_8033978807228229300LPlnk499613" target=3D"_blank">http://tutorial.mat=
h.<wbr>lamar.edu/Classes/CalcII/<wbr>ParaArea.aspx</a>&nbsp;to
 fill this need.</p>
<br>
<p></p>
<h2 style=3D"margin:12pt 0in 3pt; text-indent:0in; font-size:14pt; font-fam=
ily:Arial,sans-serif; font-style:italic; border:none; padding:0in">
<a name=3D"m_8033978807228229300__Toc170864981"><span style=3D"font-size:12=
pt; font-family:Cambria,serif"></span></a></h2>
We are grateful for your previous&nbsp;consent to use integration and serie=
s sections.&nbsp;Once our work is complete we&nbsp;will send you a link to =
the finished product.
<p></p>
<p><br>
</p>
<p>Michele</p>
<p><br>
</p>
<div id=3D"m_8033978807228229300Signature">
<div id=3D"m_8033978807228229300divtagdefaultwrapper" style=3D"font-size:12=
pt; color:#000000; background-color:#ffffff; font-family:Calibri,Arial,Helv=
etica,sans-serif">
<div><font face=3D"Tahoma" size=3D"2">Michele Iiams</font></div>
<div><font face=3D"tahoma" size=3D"2">Mathematics Department</font></div>
<div><font face=3D"tahoma" size=3D"2">University of North Dakota</font></di=
v>
<div><font face=3D"tahoma" size=3D"2"><br>
</font></div>
<div><font face=3D"tahoma" size=3D"2"><em>&quot;Never trust atoms. They mak=
e up everything.&quot; AmericInn sign in Grand Forks, ND</em></font></div>
</div>
</div>
</div>
</div>
</blockquote>
</div>
<br>
</div>
</div>
</body>
</html>

--_000_ek6uw3t7hsit4k5e9q11wp9b1478611124428emailandroidcom_--

\fi



\mainmatter

\pagestyle{fancy}

To simplify notation, we often express the gradient as $\nabla f =<f_x, f_y>$. It is often useful to think of the gradient $\nabla$ as an operator:\index{del operator}\index{$\nabla$|see {del operator}}
%\index{nabla|see {del operator}}

\cleardoublepage

Another item of notation will become useful: the ``del operator.''\index{del operator} Recall in \autoref{sec:directional_derivative} how we used the symbol $\nabla$\index{$\nabla$|see {del operator}} (pronounced ``del'') to represent the gradient of a function of two variables. That is, if $z = f(x,y)$, then ``del $f$\,'' $= \nabla f =\bracket{f_x, f_y}$.

\cleardoublepage

Another item of notation will become useful: the ``del operator.''\index{del operator} Recall how we used the symbol $\nabla$\index{$\nabla$|see {del operator}}

\cleardoublepage

%\printindex

%\setcounter{chapter}{12}
%\setcounter{section}{5}
%\input{text/12_Directional_Derivatives}
%
\setcounter{chapter}{14}
\setcounter{section}{1}
\section{Vector Fields}\label{sec:vector_fields}

We have studied functions of two and three variables, where the input of such functions is a point (either a point in the plane or in space) and the output is a number.

We could also create functions where the input is a point (again, either in the plane or in space), but the output is a \emph{vector}. For instance, we could create the following function: $\vec F(x,y) =\bracket{x+y, x-y}$, where $\vec F(2,3) =\bracket{5,-1}$. We are to think of $\vec F$ assigning the vector $\bracket{5,-1}$ to the point $(2,3)$; in some sense, the vector $\bracket{5,-1}$ lies at the point $(2,3)$. 

Such functions are extremely useful in any context where magnitude and direction are important. For instance, we could create a function $\vec F$ that represents the electromagnetic force exerted at a point by a electromagnetic field, or the velocity of air as it moves across an airfoil. 

Because these functions are so important, we need to formally define them.

\mtable{Demonstrating methods of graphing vector fields.}{fig:vectorfieldintro}{%
\begin{tikzpicture}
\begin{axis}[width=1.16\marginparwidth,height=1.16\marginparwidth,
tick label style={font=\scriptsize},
axis y line=middle,axis x line=middle,
name=myplot,axis on top,axis equal,
xtick={-3,-2,...,3},ytick={-3,-2,...,3},
ymin=-3.3,ymax=3.3,xmin=-3.3,xmax=3.3]
\draw[thick,draw={\colorone},->](axis cs:-1,-1)--(axis cs:-3,-1);
\draw[thick,draw={\colorone},->](axis cs:-1,0)--(axis cs:-2,-1);
\draw[thick,draw={\colorone},->](axis cs:-1,1)--(axis cs:-1,-1);
\draw[thick,draw={\colorone},->](axis cs:0,-1)--(axis cs:-1,0);
\draw[thick,draw={\colorone},->](axis cs:0,1)--(axis cs:1,0);
\draw[thick,draw={\colorone},->](axis cs:1,-1)--(axis cs:1,1);
\draw[thick,draw={\colorone},->](axis cs:1,0)--(axis cs:2,1);
\draw[thick,draw={\colorone},->](axis cs:1,1)--(axis cs:3,1);
\end{axis}
\node [right] at (myplot.right of origin) {\scriptsize $x$};
\node [above] at (myplot.above origin) {\scriptsize $y$};
\end{tikzpicture}
\\[-5pt](a)\\[10pt]
\begin{tikzpicture}
\begin{axis}[width=1.16\marginparwidth,height=1.16\marginparwidth,
tick label style={font=\scriptsize},
axis y line=middle,axis x line=middle,
name=myplot,axis on top,axis equal,
xtick={-3,-2,...,3},ytick={-3,-2,...,3},
ymin=-3.3,ymax=3.3,xmin=-3.3,xmax=3.3]
\draw[thick,draw={\colorone},->](axis cs:0,-1)--(axis cs:-2,-1);
\draw[thick,draw={\colorone},->](axis cs:-.5,.5)--(axis cs:-1.5,-.5);
\draw[thick,draw={\colorone},->](axis cs:-1,2)--(axis cs:-1,0);
\draw[thick,draw={\colorone},->](axis cs:.5,-1.5)--(axis cs:-.5,-.5);
\draw[thick,draw={\colorone},->](axis cs:-.5,1.5)--(axis cs:.5,.5);
\draw[thick,draw={\colorone},->](axis cs:1,-2)--(axis cs:1,0);
\draw[thick,draw={\colorone},->](axis cs:.5,-.5)--(axis cs:1.5,.5);
\draw[thick,draw={\colorone},->](axis cs:0,1)--(axis cs:2,1);
\end{axis}
\node [right] at (myplot.right of origin) {\scriptsize $x$};
\node [above] at (myplot.above origin) {\scriptsize $y$};
\end{tikzpicture}
\\[-5pt](b)}

\definition{def:vector_field}{Vector Field}
{\begin{enumerate}
	\item A \textbf{vector field in the plane} is a function $\vec F(x,y)$ whose domain is a subset of $\mathbb{R}^2$ and whose output is a two--dimensional vector:\index{vector field}
	\[\vec F(x,y) =\bracket{M(x,y), N(x,y)}.\]
	
	\item A \textbf{vector field in space} is a function $\vec F(x,y,z)$ whose domain is a subset of $\mathbb{R}^3$ and whose output is a three--dimensional vector:
	\[\vec F(x,y,z) =\bracket{M(x,y,z), N(x,y,z), P(x,y,z)}.\]
\end{enumerate}}

This definition may seem odd at first, as a special type of function is called a ``field.'' However, as the function determines a ``field of vectors'', we can say the field is \emph{defined by} the function, and thus the field \emph{is} a function.

Visualizing vector fields helps cement this connection. When graphing a vector field in the plane, the general idea is to draw the vector $\vec F(x,y)$ at the point $(x,y)$. For instance, using $\vec F(x,y) =\bracket{x+y,x-y}$ as before, at $(1,1)$ we would draw $\bracket{2,0}$. 

In \autoref{fig:vectorfieldintro}(a), one can see that the vector $\bracket{2,0}$ is drawn \emph{starting from} the point $(1,1)$. A total of 8 vectors are drawn, with the $x$- and $y$-values of $-1,0,1$. In many ways, the resulting graph is a mess; it is hard to tell what this field ``looks like.''

In \autoref{fig:vectorfieldintro}(b), the same field is redrawn with each vector $\vec F(x,y)$ drawn \emph{centered on} the point $(x,y)$. This makes for a better looking image, though the long vectors can cause confusion: when one vector intersects another, the image looks cluttered.

\mtable{Demonstrating methods of graphing vector fields.}{fig:vectorfieldintrob}{%
\begin{tikzpicture}
\begin{axis}[width=1.16\marginparwidth,height=1.16\marginparwidth,
tick label style={font=\scriptsize},
axis y line=middle,axis x line=middle,
name=myplot,axis on top,axis equal,
xtick={-3,-2,...,3},ytick={-3,-2,...,3},
ymin=-3.3,ymax=3.3,xmin=-3.3,xmax=3.3]
\draw[ultra thick,draw={\colorone},->](axis cs:1,-1.5)--(axis cs:1,-.5);
\draw[ultra thick,draw={\colorone},->](axis cs:-1,1.5)--(axis cs:-1,.5);
\draw[ultra thick,draw={\colorone},->](axis cs:.5,1)--(axis cs:1.5,1);
\draw[ultra thick,draw={\colorone},->](axis cs:-.5,-1)--(axis cs:-1.5,-1);
\draw[thick,draw={\colorone},->](axis cs:.75,-.25)--(axis cs:1.25,.25);
\draw[thick,draw={\colorone},->](axis cs:-.75,.25)--(axis cs:-1.25,-.25);
\draw[thick,draw={\colorone},->](axis cs:-.25,1.25)--(axis cs:.25,.75);
\draw[thick,draw={\colorone},->](axis cs:.25,-1.25)--(axis cs:-.25,-.75);
\end{axis}
\node [right] at (myplot.right of origin) {\scriptsize $x$};
\node [above] at (myplot.above origin) {\scriptsize $y$};
\end{tikzpicture}
\\[-5pt](a)\\[10pt]
\begin{tikzpicture}
\begin{axis}[width=1.16\marginparwidth,height=1.16\marginparwidth,
tick label style={font=\scriptsize},
axis y line=middle,axis x line=middle,
name=myplot,axis on top,axis equal,
xtick={-3,-2,...,3},ytick={-3,-2,...,3},
ymin=-3.3,ymax=3.3,xmin=-3.3,xmax=3.3]
\foreach \x in {-3,-2.25,...,3.11} {
    \foreach \y in {-3,-2.25,...,3.11} {
		\edef\vx{(((\x)+(\y))/20)}
		\edef\vy{(((\x)-(\y))/20)}
        \edef\temp{\noexpand\draw[->,{\colorone}](axis cs:{\x-\vx},{\y-\vy})--(axis cs:{\x+\vx},{\y+\vy});}
        \temp
    }
}
\end{axis}
\node [right] at (myplot.right of origin) {\scriptsize $x$};
\node [above] at (myplot.above origin) {\scriptsize $y$};
\end{tikzpicture}
\\[-5pt](b)}

A common way to address this problem is limit the length of each arrow, and represent long vectors with thick arrows, as done in \autoref{fig:vectorfieldintrob}(a). Usually we do not use a graph of a vector field to determine exactly the magnitude of a particular vector. Rather, we are more concerned with the relative magnitudes of vectors: which are bigger than others? Thus limiting the length of the vectors is not problematic.

Drawing arrows with variable thickness is best done with technology; search the documentation of your favorite graphing program for terms like ``vector fields'' or ``slope fields'' to learn how. Technology obviously allows us to plot many vectors in a vector field nicely; in \autoref{fig:vectorfieldintrob}(b), we see the same vector field drawn with many vectors, and finally get a clear picture of how this vector field behaves. (If this vector field represented the velocity of air moving across a flat surface, we could see that the air tends to move either to the upper--right or lower--left, and moves very slowly near the origin.)

\mtable{Graphing a vector field in space.}{fig:vectorfieldintroc}{%
\myincludeasythree{width=\marginparwidth,
3Droll=0,
3Dortho=0.004676746670156717,
3Dc2c=0.7426115870475769 -0.4871146082878113 0.45961663126945496,
3Dcoo=40.4605598449707 -31.11708641052246 21.908815383911133,
3Droo=141.89772583942957,
3Dlights=Headlamp}{width=\marginparwidth}{figures/figvectorfieldintro_f_3D}}

We can similarly plot vector fields in space, as shown in \autoref{fig:vectorfieldintroc}, though it is not often done. The plots get very busy very quickly, as there are lots of arrows drawn in a small amount of space. In \autoref{fig:vectorfieldintroc} the field $\vec F =\bracket{-y,x,z}$ is graphed. If one could view the graph from above, one could see the arrows point in a cirlce about the $z$-axis. One should also note how the arrows far from the origin are larger than those close to the origin. 

It is good practice to try to visualize certain vector fields in one's head. For instance, consider a point mass at the origin and the vector field that represents the gravitational force exerted by the mass at any point in the room. The field would consist of arrows pointing toward the origin, increasing in size as they near the origin (as the gravitational pull is strongest near the point mass).

\youtubeVideo{XGWhfSHl8Eo}{Vector Fields --- Sketching}

\subsection{Vector Field Notation and Del Operator}

\autoref{def:vector_field} defines a vector field $\vec F$ using the notation \[
\vec F(x,y) =\bracket{M(x,y), N(x,y)}
\quad \text{and}\quad
\vec F(x,y,z) =\bracket{M(x,y,z), N(x,y,z),P(x,y,z)}.
\]
That is, the components of $\vec F$ are each functions of $x$ and $y$ (and also $z$ in space). As done in other contexts, we will drop the ``of $x$, $y$ and $z$'' portions of the notation and refer to vector fields in the plane and in space as 
\[\vec F =\bracket{M, N}\quad \text{and} \quad \vec F  =\bracket{M,N,P},\]
respectively, as this shorthand is quite convenient.

Another item of notation will become useful: the ``del operator.'' \index{del operator}Recall in \autoref{sec:directional_derivative} how we used the symbol $\nabla$ (pronounced ``del'') to represent the gradient of a function of two variables. That is, if $z = f(x,y)$, then ``del $f$\,'' $= \nabla f =\bracket{f_x, f_y}$. 

We now define $\nabla$ to be the ``del operator.'' It is a vector whose components are partial derivative operations. 

In the plane, $\ds\nabla =\bracket{\frac{\partial}{\partial x}, \frac{\partial}{\partial y}}$; in space, $\ds\nabla =\bracket{\frac{\partial}{\partial x}, \frac{\partial}{\partial y},\frac{\partial}{\partial z}}$. 

With this definition of $\nabla$, we can better understand the gradient $\nabla f$. As $f$ returns a scalar, the properties of scalar and vector multiplication gives
\[
\nabla f
=\bracket{\frac{\partial}{\partial x}, \frac{\partial}{\partial y}}f
=\bracket{\frac{\partial}{\partial x}\,f, \frac{\partial}{\partial y}\,f}
=\bracket{f_x, f_y}.
\]

Now apply the del operator $\nabla$ to vector fields. Let $\vec F =\bracket{x+\sin y,y^2+z,x^2}$. We can use vector operations and find the dot product of $\nabla$ and $\vec F$:
\begin{align*}
	\nabla \cdot \vec F
	&=\bracket{\frac{\partial}{\partial x}, \frac{\partial}{\partial y},\frac{\partial}{\partial z}}\cdot \bracket{x+\sin y,y^2+z,x^2}\\
	&= \frac{\partial}{\partial x}(x+\sin y)+ \frac{\partial}{\partial y}(y^2+z) + \frac{\partial}{\partial z}(x^2) \\
	&=1+2y.
\end{align*}

We can also compute their cross products:\small
\begin{align*}
	&\nabla\times \vec F\\
	&= {\bracket{\frac{\partial}{\partial y}\big(x^2\big)-\frac{\partial}{\partial z}\big(y^2+z\big),\frac{\partial}{\partial z}\big(x+\sin y\big)-\frac{\partial}{\partial x}\big(x^2\big),\frac{\partial}{\partial x}\big(y^2+z\big)-\frac{\partial}{\partial y}\big(x+\sin y\big)}}\\
	&=\bracket{-1,-2x,-\cos y}.
\end{align*}\normalsize

We do not yet know why we would want to compute the above. However, as we next learn about properties of vector fields, we will see how these dot and cross products with the del operator are quite useful.

\subsection{Divergence and Curl}

Two properties of vector fields will prove themselves to be very important: divergence and curl. Each is a special ``derivative'' of a vector field; that is, each measures an instantaneous rate of change of a vector field.

If the vector field represents the velocity of a fluid or gas, then the \sword{divergence}\index{divergence}\index{vector field!divergence of} of the field is a measure of the ``compressibility'' of the fluid. If the divergence is negative at a point, it means that the fluid is compressing: more fluid is going into the point than is going out. If the divergence is positive, it means the fluid is expanding: more fluid is going out at that point than going in. A divergence of zero means the same amount of fluid is going in as is going out. If the divergence is zero at all points, we say the field is \sword{incompressible}.\index{incompressible vector field}

It turns out that the proper measure of divergence is simply $\nabla \cdot \vec F$, as stated in the following definition.

\definition{def:divergence}{Divergence of a Vector Field}
{The \sword{divergence} of a vector field $\vec F$ is\index{divergence}\index{vector field!divergence of}
\[\divv \vec F = \nabla \cdot \vec F.\]
\begin{itemize}
	\item In the plane, with $\vec F =\bracket{M,N}$, $\divv \vec F = M_x+N_y$.
	\item	In space, with $\vec F =\bracket{M,N,P}$, $\divv \vec F = M_x+N_y+P_z$.
\end{itemize}}

%Divergence is a measure of the \emph{compressibility} of the ``substance'' the field represents. Imagine drawing a small box anywhere on the graph of the vector field in \autoref{fig:vectorfieldintrob}(b). There would be some vectors pointing into the box, and some arrows pointing out of the box. In short, divergence measures ``the amount going out minus the amount going in.'' If the vector field represented the flow of air being pumped into an enclosed room, then there is likely more air going into a region than going out. Via a limit process, we can shrink the box to a point, and divergence measures the amount going out of the point vs. the amount going into the point. If the divergence of a field is zero everywhere, we say that the field represents an \sword{incompressible fluid} -- that is, the same amount goes into each point as goes out. (It turns out that the divergence of the field in \autoref{fig:vectorfieldintrob} is zero everywhere.)

\sword{Curl}\index{vector field!curl of}\index{curl} is a measure of the spinning action of the field. Let $\vec F$ represent the flow of water over a flat surface. If a small round cork were held in place at a point in the water, would the water cause the cork to spin? No spin corresponds to zero curl; counterclockwise spin corresponds to positive curl and clockwise spin corresponds to negative curl. 

In space, things are a bit more complicated. Again let $\vec F$ represent the flow of water, and imagine suspending a tennis ball in one location in this flow. The water may cause the ball to spin along an axis. If so, the curl of the vector field is a \emph{vector} (not a \emph{scalar}, as before), parallel to the axis of rotation, following a right hand rule: when the thumb of one's right hand points in the direction of the curl, the ball will spin in the direction of the curling fingers of the hand.

In space, it turns out the proper measure of curl is $\nabla \times \vec F$, as stated in the following definition. To find the curl of a planar vector field $\vec F =\bracket{M,N}$, embed it into space as $\vec F =\bracket{M, N, 0}$ and apply the cross product definition. Since $M$ and $N$ are functions of just $x$ and $y$ (and not $z$), all partial derivatives with respect to $z$ become 0 and the result is simply $\bracket{0,0,N_x-M_y}$. The third component is the measure of curl of a planar vector field. 

\definition{def:curl}{Curl of a Vector Field}
{\begin{itemize}
	\item Let $\vec F =\bracket{M,N}$ be a vector field in the plane. The \sword{curl} of $\vec F$ is $\curl \vec F = N_x - M_y$.\index{curl}
	\item Let $\vec F =\bracket{M,N,P}$ be a vector field in space. The \sword{curl} of $\vec F$ is $\curl \vec F = \nabla \times \vec F =\bracket{P_y-N_z,M_z-P_x,N_x - M_y}$.
\end{itemize}}

We adopt the convention of referring to curl as $\nabla \times \vec F$, regardless of whether $\vec F$ is a vector field in two or three dimensions. 

We now practice computing these quantities.

\mtable{The vector fields in parts (a) and (b) in \autoref{ex_vectorfield1}.}{fig:vectorfield1a}{%
\begin{tikzpicture}
\begin{axis}[width=1.16\marginparwidth,height=1.16\marginparwidth,
tick label style={font=\scriptsize},
axis y line=middle,axis x line=middle,
name=myplot,axis on top,axis equal,
xtick={-1,1},ytick={-1,1},
ymin=-1.2,ymax=1.2,xmin=-1.2,xmax=1.2]
\foreach \x in {-1,-.75,...,1.11} {
    \foreach \y in {-1,-.867,...,1.11} {
		\edef\vx{((\y)/20)}
		\edef\vy{(0)}
        \edef\temp{\noexpand\draw[->,{\colorone}](axis cs:{\x-\vx},{\y-\vy})--(axis cs:{\x+\vx},{\y+\vy});}
        \temp
    }
}
\end{axis}
\node [right] at (myplot.right of origin) {\scriptsize $x$};
\node [above] at (myplot.above origin) {\scriptsize $y$};
\end{tikzpicture}
\\[-5pt](a)\\[10pt]
\begin{tikzpicture}
\begin{axis}[width=1.16\marginparwidth,height=1.16\marginparwidth,
tick label style={font=\scriptsize},
axis y line=middle,axis x line=middle,
name=myplot,axis on top,axis equal,
xtick={-1,1},ytick={-1,1},
ymin=-1.2,ymax=1.2,xmin=-1.2,xmax=1.2]
\foreach \x in {-1,-.867,...,1.11} {
    \foreach \y in {-1,-.867,...,1.11} {
		\edef\vx{(-(\y)/30)}
		\edef\vy{((\x)/30)}
        \edef\temp{\noexpand\draw[->,{\colorone}](axis cs:{\x-\vx},{\y-\vy})--(axis cs:{\x+\vx},{\y+\vy});}
        \temp
    }
}
\end{axis}
\node [right] at (myplot.right of origin) {\scriptsize $x$};
\node [above] at (myplot.above origin) {\scriptsize $y$};
\end{tikzpicture}
\\[-5pt](b)}

\example{ex_vectorfield1}{Computing divergence and curl of planar vector fields}
{For each of the planar vector fields given below, view its graph and try to visually determine if its divergence and curl are 0. Then compute the divergence and curl.

%\noindent\begin{minipage}[t]{.48\linewidth}
%\begin{enumerate}
	%\item $\vec F =\bracket{y,0}$ (see \autoref{fig:vectorfield1a}(a))
	%\item $\vec F =\bracket{-y,x}$ (see \autoref{fig:vectorfield1a}(b))
%\end{enumerate}
%\end{minipage}\begin{minipage}[t]{.6\linewidth}
%\begin{enumerate}\addtocounter{enumi}{2}
	%\item $\vec F =\bracket{x,y}$ (see \autoref{fig:vectorfield1b}(a))
	%\item $\vec F =\bracket{\cos y, \sin x}$ 
		%(see \autoref{fig:vectorfield1b}(b))
%\end{enumerate}
%\end{minipage}
\begin{enumerate}
	\item $\vec F =\bracket{y,0}$ (see \autoref{fig:vectorfield1a}(a))
	\item $\vec F =\bracket{-y,x}$ (see \autoref{fig:vectorfield1a}(b))
	\item $\vec F =\bracket{x,y}$ (see \autoref{fig:vectorfield1b}(a))
	\item $\vec F =\bracket{\cos y, \sin x}$ (see \autoref{fig:vectorfield1b}(b))
\end{enumerate}}
{\begin{enumerate}
	\item The arrow sizes are constant along any horizontal line, so if one were to draw a small box anywhere on the graph, it would seem that the same amount of fluid would enter the box as exit. Therefore it seems the divergence is zero; it is, as 
	\[
	\divv\vec F
	= \nabla \cdot \vec F
	= M_x + N_y
	= \frac{\partial}{\partial x}(y) + \frac{\partial}{\partial y}(0) = 0.
	\]
	At any point on the $x$-axis, arrows above it move to the right and arrows below it move to the left, indicating that a cork placed on the axis would spin clockwise. A cork placed anywhere above the $x$-axis would have water above it moving to the right faster than the water below it, also creating a clockwise spin. A clockwise spin also appears to be created at points below the $x$-axis. Thus it seems the curl should be negative (and not zero). Indeed, it is:
	\[
	\curl \vec F = \nabla\times\vec F
	= N_x-M_y = \frac{\partial}{\partial x}(0) - \frac{\partial}{\partial y}(y) = -1.
	\]
	
	\item It appears that all vectors that lie on a circle of radius $r$, centered at the  origin, have the same length (and indeed this is true). That implies that the divergence should be zero: draw any box on the graph, and any fluid coming in will lie along a circle that takes the same amount of fluid out. Indeed, the divergence is zero, as
	\[
	\divv\vec F = \nabla \cdot \vec F
	= M_x + N_y = \frac{\partial}{\partial x}(-y) + \frac{\partial}{\partial y}(x)
	= 0.
	\]
	
		Clearly this field moves objects in a circle, but would it induce a cork to spin? It appears that yes, it would: place a cork anywhere in the flow, and the point of the cork closest to the origin would feel less flow than the point on the cork farthest from the origin, which would induce a counterclockwise flow. Indeed, the curl is positive:
	\[
	\curl \vec F = \nabla\times\vec F = N_x-M_y
	= \frac{\partial}{\partial x}(x) - \frac{\partial}{\partial y}(-y) = 1-(-1) = 2.
	\]
	Since the curl is constant, we conclude the induced spin is the same no matter where one is in this field.
	
	\mtable{The vector fields in parts (c) and (d) in \autoref{ex_vectorfield1}.}{fig:vectorfield1b}{%
\begin{tikzpicture}
\begin{axis}[width=1.16\marginparwidth,height=1.16\marginparwidth,
tick label style={font=\scriptsize},
axis y line=middle,axis x line=middle,
name=myplot,axis on top,axis equal,
xtick={-1,1},ytick={-1,1},
ymin=-1.2,ymax=1.2,xmin=-1.2,xmax=1.2]
\foreach \x in {-1,-.8,...,1.11} {
    \foreach \y in {-1,-.8,...,1.11} {
		\edef\vx{((\x)/15)}
		\edef\vy{((\y)/15)}
        \edef\temp{\noexpand\draw[->,{\colorone}]
        (axis cs:{(\x)-(\vx)},{(\y)-(\vy)}) -- (axis cs:{(\x)+(\vx)},{(\y)+(\vy)});}
        \temp
    }
}
\end{axis}
\node [right] at (myplot.right of origin) {\scriptsize $x$};
\node [above] at (myplot.above origin) {\scriptsize $y$};
\end{tikzpicture}
	\\[-5pt](a)\\[10pt]
\begin{tikzpicture}
\begin{axis}[width=1.16\marginparwidth,height=1.16\marginparwidth,
tick label style={font=\scriptsize},
axis y line=middle,axis x line=middle,
name=myplot,axis on top,axis equal,minor tick num=2,
xtick={-6,-3,...,6},ytick={-6,-3,...,6},
ymin=-7,ymax=7,xmin=-7,xmax=7]
\foreach \x in {-6,-5,...,6.11} {
    \foreach \y in {-6,-5,...,6.11} {
		\edef\vx{((cos(deg(\y)))/2)}
		\edef\vy{((sin(deg(\x)))/2)}
        \edef\temp{\noexpand\draw[->,{\colorone}]
        (axis cs:{(\x)-(\vx)},{(\y)-(\vy)}) -- (axis cs:{(\x)+(\vx)},{(\y)+(\vy)});}
        \temp
    }
}
\end{axis}
\node [right] at (myplot.right of origin) {\scriptsize $x$};
\node [above] at (myplot.above origin) {\scriptsize $y$};
\end{tikzpicture}
	\\[-5pt](b)}

		\item At the origin, there are many arrows pointing out but no arrows pointing in. We conclude that at the origin, the divergence must be positive (and not zero). If one were to draw a box anywhere in the field, the edges farther from the origin would have larger arrows passing through them than the edges close to the origin, indicating that more is going from a point than going in. This indicates a positive (and not zero) divergence. This is correct:
	\[
	\divv\vec F = \nabla \cdot \vec F
	= M_x + N_y = \frac{\partial}{\partial x}(x) + \frac{\partial}{\partial y}(y)
	= 1+1=2.
	\]
	
	One may find this curl to be harder to determine visually than previous examples. One might note that any arrow that induces a clockwise spin on a cork will have an equally sized arrow inducing a counterclockwise spin on the other side, indicating no spin and no curl. This is correct, as
	\[
	\curl \vec F = \nabla\times\vec F
	= N_x-M_y = \frac{\partial}{\partial x}(y) - \frac{\partial}{\partial y}(x) = 0.
	\]
	
	%\drawexampleline
	\item	One might find this divergence hard to determine visually as large arrows appear in close proximity to small arrows, each pointing in different directions. Instead of trying to rationalize a guess, we compute the divergence:
	\[
	\divv\vec F = \nabla \cdot \vec F = M_x + N_y
	= \frac{\partial}{\partial x}(\cos y) + \frac{\partial}{\partial y}(\sin x) = 0.
	\] 
	Perhaps surprisingly, the divergence is 0.
	
	Will all the loops of different directions in the field, one is apt to reason the curl is variable. Indeed, it is:
	\[
	\curl \vec F = \nabla\times\vec F = N_x-M_y
	= \frac{\partial}{\partial x}(\sin x) - \frac{\partial}{\partial y}(\cos y)
	= \cos x + \sin y.
	\]
	Depending on the values of $x$ and $y$, the curl may be positive, negative, or zero.\eoehere
\end{enumerate}}%\clearpage

\example{ex_vectorfield2}{Computing divergence and curl of vector fields in space}
{Compute the divergence and curl of each of the following vector fields.
\begin{enumerate}
	\item $\vec F =\bracket{x^2+y+z, -x-z, x+y}$
	\item	$\vec F =\bracket{e^{xy}, \sin(x+z),x^2+y}$
\end{enumerate}
}
{We compute the divergence and curl of each field following the definitions.
\begin{enumerate}
	\item $\divv \vec F = \nabla \cdot \vec F = M_x+N_y+P_z = 2x+0+0= 2x.$
	
	$\begin{aligned}\curl\vec F = \nabla \times \vec F &=\bracket{P_y-N_z,M_z-P_x,N_x - M_y}\\ &=\bracket{1 - (-1), 1-1,-1-(1)}=\bracket{2,0,-2}.
	\end{aligned}$
	
	For this particular field, no matter the location in space, a spin is induced with axis parallel to $\bracket{2,0,-2}.$
	\item $\divv \vec F = \nabla \cdot \vec F = M_x+N_y+P_z = ye^{xy}+0+0= ye^{xy}.$
		
	$\begin{aligned}\curl\vec F = \nabla \times \vec F &=\bracket{P_y-N_z,M_z-P_x,N_x - M_y}\\ &=\bracket{1-\cos(x+z), -2x, \cos(x+z) - xe^{xy}}.\eoehere \end{aligned}$
\end{enumerate}}

\example{ex_vectorfield3}{Creating a field representing gravitational force}
{The force of gravity between two objects is inversely proportional to the square of the distance between the objects. Locate a point mass at the origin. Create a vector field $\vec F$ that represents the gravitational pull of the point mass at any point $(x,y,z)$. Find the divergence and curl of this field. 
}
{The point mass pulls toward the origin, so at $(x,y,z)$, the force will pull in the direction of $\bracket{-x, -y, -z}$. To get the proper magnitude, it will be useful to find the unit vector in this direction. Dividing by its magnitude, we have
\[
\vec u
= \bracket{\frac{-x}{\sqrt{x^2+y^2+z^2}}, \frac{-y}{\sqrt{x^2+y^2+z^2}},\frac{-z}{\sqrt{x^2+y^2+z^2}}}.
\]
The magnitude of the force is inversely proportional to the square of the distance between the two points. Letting $k$ be the constant of proportionality, we have the magnitude as $\ds\frac{k}{x^2+y^2+z^2}$. Multiplying this magnitude by the unit vector above, we have the desired vector field:
%
\mtable[-6\baselineskip]{A vector field representing a planar gravitational force.}{fig:vectorfield3}{\begin{tikzpicture}
\begin{axis}[width=1.16\marginparwidth,height=1.16\marginparwidth,
tick label style={font=\scriptsize},
axis y line=middle,axis x line=middle,name=myplot,axis on top,axis equal,
xtick={-1,1},ytick={-1,1},
ymin=-1.2,ymax=1.2,xmin=-1.2,xmax=1.2]
\foreach \x in {-1,-.6,...,1.11} {
    \foreach \y in {-1,-.6,...,1.11} {
		\edef\vx{((-\x)/(sqrt((\x)*(\x)+(\y)*(\y)))^3)}
		\edef\vy{((-\y)/(sqrt((\x)*(\x)+(\y)*(\y)))^3)}
        \edef\temp{\noexpand\draw[->,{\colorone}](axis cs:{\x-((\vx)/100)},{\y-((\vy)/100)})--(axis cs:{\x+((\vx)/100)},{\y+((\vy)/100)});}
        \temp
    }
}
\end{axis}
\node [right] at (myplot.right of origin) {\scriptsize $x$};
\node [above] at (myplot.above origin) {\scriptsize $y$};
\end{tikzpicture}}
%
\[
\vec F
=\bracket{\frac{-kx}{(x^2+y^2+z^2)^{3/2}}, \frac{-ky}{(x^2+y^2+z^2)^{3/2}},\frac{-kz}{(x^2+y^2+z^2)^{3/2}}}.
\]
We leave it to the reader to confirm that $\divv \vec F = 0$ and $\curl \vec F = \vec 0$.

The analogous planar vector field is given in \autoref{fig:vectorfield3}. Note how all arrows point to the origin, and the magnitude gets very small when ``far'' from the origin.}

A function $z=f(x,y)$ naturally induces a vector field, $\vec F = \nabla f =\bracket{f_x,f_y}$. Given what we learned of the gradient in \autoref{sec:directional_derivative}, we know that the vectors of $\vec F$ point in the direction of greatest increase of $f$. Because of this, $f$ is said to be the \sword{potential function}\index{potential function}\index{vector field!potential function of} of $\vec F$. Vector fields that are the gradient of potential functions will play an important role in the next section.

\mtable[-3\baselineskip]{The vector field $\vec F = \nabla f$ and a graph of a function $z=f(x,y)$ in \autoref{ex_vectorfield4}.}{fig:vectorfield4}{%
\begin{tikzpicture}
\begin{axis}[width=1.16\marginparwidth,height=1.16\marginparwidth,
tick label style={font=\scriptsize},
axis y line=middle,axis x line=middle,name=myplot,axis on top,axis equal,
xtick={-1,1},ytick={-1,1},
ymin=-1.2,ymax=1.2,xmin=-1.2,xmax=1.2]
\foreach \x in {-1,-.8,...,1.11} {
    \foreach \y in {-1,-.8,...,1.11} {
		\edef\vx{(-2*(\x))}
		\edef\vy{(-4*(\y))}
        \edef\temp{\noexpand\draw[->,{\colorone}](axis cs:{\x-((\vx)/50)},{\y-((\vy)/50)})--(axis cs:{\x+((\vx)/50)},{\y+((\vy)/50)});}
        \temp
    }
}
\end{axis}
\node [right] at (myplot.right of origin) {\scriptsize $x$};
\node [above] at (myplot.above origin) {\scriptsize $y$};
\end{tikzpicture}
\\[-5pt](a)\\[10pt]
\myincludeasythree{width=1.16\marginparwidth,
3Droll=0,
3Dortho=0.004781691357493401,
3Dc2c=0.7686017751693726 -0.47064894437789917 0.4332907199859619,
3Dcoo=0.000002273424570375937 -0.0000013264116205391474 56.08348083496094,
3Droo=117.22547128132403,
3Dlights=Headlamp}{width=1.16\marginparwidth}{figures/figvectorfield4b_3D}
\\(b)}

\example{ex_vectorfield4}{A vector field that is the gradient of a potential function}
{Let $f(x,y) = 3-x^2-2y^2$ and let $\vec F = \nabla f$. Graph $\vec F$, and find the divergence and curl of $\vec F$. 
}
{Given $f$, we find $\vec F = \nabla f =\bracket{-2x,-4y}$. A graph of $\vec F$ is given in \autoref{fig:vectorfield4}(a). In part (b) of the figure, the vector field is given along with a graph of the surface itself; one can see how each vector is pointing in the direction of ``steepest uphill'', which, in this case, is not simply just ``toward the origin.''

We leave it to the reader to confirm that $\divv \vec F = -6$ and $\curl \vec F = 0$.}

There are some important concepts visited in this section that will be revisited in subsequent sections and again at the very end of this chapter. One is: given a vector field $\vec F$, both $\divv\vec F$ and $\curl\vec F$ are measures of rates of change of $\vec F$. The divergence measures how much the field spreads (diverges) at a point, and the curl measures how much the field twists (curls) at a point. Another important concept is this: given $z=f(x,y)$, the gradient $\nabla f$ is also a measure of a rate of change of $f$. We will see how the integrals of these rates of change produce meaningful results.

This section introduces the concept of a vector field. The next section ``applies calculus'' to vector fields. A common application is this: let $\vec F$ be a vector field representing a force (hence it is called a ``force field,'' though this name has a decidedly comic-book feel) and let a particle move along a curve $C$ under the influence of this force. What work is performed by the field on this particle? The solution lies in correctly applying the concepts of line integrals in the context of vector fields.


\printexercises{exercises/14_02_exercises}
%Consider the same field as before, letting it represent the flow of water across a flat surface. If one were to drop a small cork into the water, it would certainly move according to the pattern of the field. 


\clearpage

\apexappendix
%\appendix
%\appendix
%\pagenumbering{arabic}
%\renewcommand{\thepage}{A.\arabic{page}}
%\part*{Appendices\protect\thispagestyle{empty}}
%\iflatexml\else
%\pdfbookmark[part]{Appendices}{appendixbookmark}
%\fi

%\makeexercisesection{Standalone Solutions To All Problems}

\pagestyle{empty}
\eendgeometry

\cleardoublepage
\phantomsection
\addcontentsline{toc}{chapter}{\indexname}
\printindex

%\backmatter
%
%\pagestyle{empty}

\cleardoublepage

\ifthenelse{\boolean{latexml}}{\chapter*{Important Formulas}}{}

\subsection*{Differentiation Rules}
\lxAddClass{diffRules}
\footnotesize
\noindent\begin{minipage}[t]{.20\linewidth}
\begin{enumerate}
\item \deriv{cx}{c}
\item	\deriv{u\pm v}{u'\pm v'}
\item	\deriv{u\cdot v}{uv'+ u'v}
\item	\deriv{\frac uv}{\frac{vu'-uv'}{v^2}}
\item	\deriv{u(v)}{u'(v)v'}
\item	\deriv{c}{0}
\item	\deriv{x}{1}
\item	\deriv{x^n}{nx^{n-1}}
\item	\deriv{e^x}{e^x}
\end{enumerate}
\end{minipage}%
\begin{minipage}[t]{.23\linewidth}
\begin{enumerate}\addtocounter{enumi}{9}
\item	\deriv{a^x}{\ln a\cdot a^x}
\item	\deriv{\ln x}{\frac{1}{x}}
\item	\deriv{\log_a x}{\frac{1}{\ln a}\cdot \frac1x}
\item	\deriv{\sin x}{\cos x}
\item	\deriv{\cos x}{-\sin x}
\item	\deriv{\csc x}{-\csc x\cot x}
\item	\deriv{\sec x}{\sec x\tan x}
\item	\deriv{\tan x}{\sec^2 x}
\item	\deriv{\cot x}{-\csc^2 x}
\end{enumerate}
\end{minipage}%
\begin{minipage}[t]{.25\linewidth}
\begin{enumerate}\addtocounter{enumi}{18}
\item	\deriv{\sin^{-1}x}{\frac{1}{\sqrt{1-x^2}}}
\item	\deriv{\cos^{-1}x}{\frac{-1}{\sqrt{1-x^2}}}
\item	\deriv{\csc^{-1}x}{\frac{-1}{|x|\sqrt{x^2-1}}}
\item	\deriv{\sec^{-1}x}{\frac{1}{|x|\sqrt{x^2-1}}}
\item	\deriv{\tan^{-1}x}{\frac{1}{1+x^2}}
\item	\deriv{\cot^{-1}x}{\frac{-1}{1+x^2}}
\item	\deriv{\cosh x}{\sinh x}
\item \deriv{\sinh x}{\cosh x}
\item \deriv{\tanh x}{\sech^2 x}
\end{enumerate}
\end{minipage}%
\begin{minipage}[t]{.25\linewidth}
\begin{enumerate}\addtocounter{enumi}{27}
\item \deriv{\sech x}{-\sech x\tanh x}
\item	\deriv{\csch x}{-\csch x\coth x}
\item	\deriv{\coth x}{-\csch^2 x}
\item	\deriv{\cosh^{-1}x}{\frac1{\sqrt{x^2-1}}}
\item	\deriv{\sinh^{-1}x}{\frac1{\sqrt{x^2+1}}}
\item	\deriv{\sech^{-1}x}{\frac{-1}{x\sqrt{1-x^2}}}
\item	\deriv{\csch^{-1}x}{\frac{-1}{|x|\sqrt{1+x^2}}}
\item	\deriv{\tanh^{-1}x}{\frac1{1-x^2}}
\item	\deriv{\coth^{-1}x}{\frac1{1-x^2}}
\end{enumerate}
\end{minipage}
\vspace{4\baselineskip}

\subsection*{Integration Rules}
\lxAddClass{intRules}
\noindent\begin{minipage}[t]{.30\linewidth}
\begin{enumerate}
\item	\myint{c\cdot f(x)}{c\int f(x)\ dx}
\item	\myint{f(x)\pm g(x)}{}\\
$\ds \int f(x)\ dx \pm \int g(x)\ dx$
\item	\myint{0}{C}
\item	\myint{1}{x+C}
\item	\myint{x^n}{\frac{1}{n+1}x^{n+1}+C, \ n\neq -1}\\
$\ n\neq -1$
\item	\myint{e^x}{e^x+C}
\item	\myint{a^x}{\frac{1}{\ln a}\cdot a^x+C}
\item	\myint{\frac{1}{x}}{\ln |x| + C}
\item	\myint{\cos x}{\sin x+C}
\item	\myint{\sin x}{-\cos x+C}
\end{enumerate}
\end{minipage}%
\begin{minipage}[t]{.31\linewidth}
\begin{enumerate}\addtocounter{enumi}{10}
\item	\myint{\tan x}{-\ln |\cos x|+C}
\item	\myint{\sec x}{\ln |\sec x+\tan x|+C}
\item	\myint{\csc x}{-\ln |\csc x+\cot x|+C}
\item	\myint{\cot x}{\ln |\sin x|+C}
\item	\myint{\sec^2 x}{\tan x+C}
\item	\myint{\csc^2x}{-\cot x+C}
\item	\myint{\sec x\tan x}{\sec x+C}
\item	\myint{\csc x\cot x}{-\csc x+C}
\item	\myint{\cos^2x}{\frac12x+\frac14\sin\big(2x\big)+C}
\item	\myint{\sin^2x}{\frac12x-\frac14\sin\big(2x\big)+C}
\item	\myint{\frac{1}{x^2+a^2}}{\frac1a\tan^{-1}\left(\frac xa\right)+C}
\end{enumerate}
\end{minipage}%
\begin{minipage}[t]{.38\linewidth}
\begin{enumerate}\addtocounter{enumi}{21}
\item	\myint{\frac{1}{\sqrt{a^2-x^2}}}{\sin^{-1}\left(\frac xa\right)+C}
\item	\myint{\frac{1}{x\sqrt{x^2-a^2}}}{\frac1a\sec^{-1}\left(\frac{|x|}{a}\right)+C}
\item	\myint{\cosh x}{\sinh x+C}
\item	\myint{\sinh x}{\cosh x+C}
\item	\myint{\tanh x}{\ln(\cosh x)+C}
\item	\myint{\coth x}{\ln |\sinh  x|+C}
\item	\myint{\frac{1}{\sqrt{x^2-a^2}}}{\ln\big|x+\sqrt{x^2-a^2}\big|+C}
\item	\myint{\frac{1}{\sqrt{x^2+a^2}}}{\ln\big|x+\sqrt{x^2+a^2}\big|+C}
\item	\myint{\frac{1}{a^2-x^2}}{\frac1{2a}\ln\left|\frac{a+x}{a-x}\right|+C}
\item	\myint{\frac{1}{x\sqrt{a^2-x^2}}}{\frac1a\ln\left(\frac{x}{a+\sqrt{a^2-x^2}}\right)+C}
\item	\myint{\frac{1}{x\sqrt{x^2+a^2}}}{\frac1a\ln\left|\frac{x}{a+\sqrt{x^2+a^2}}\right|+C}
\end{enumerate}
\end{minipage}
\normalsize

\clearpage

\noindent%
\begin{minipage}[t]{.53\linewidth}
\subsection*{The Unit Circle}

\begin{tikzpicture}[scale=3]
\draw [<->,>=latex] (-1.5,0) -- (1.4,0) node [right] {\scriptsize $x$};
\draw [<->,>=latex] (0,-1.3) -- (0,1.3) node [above] {\scriptsize $y$};
\foreach \x / \y / \z / \w / \v in {
	0/0/{1,0}/right/white,
	30/{\pi/6}/{\frac{\sqrt{3}}2,\frac 12}/above right/none,%
	45/{\pi/4}/{\frac{\sqrt{2}}2,\frac{\sqrt{2}}2}/above right/none,
	60/{\pi/3}/{\frac{1}2,\frac{\sqrt{3}}2}/{above right}/none,
	90/ {\pi/2}/{0,1}/above/white,%
	120/{2\pi/3}/{-\frac{1}2,\frac{\sqrt{3}}2}/above left/none, 
	135/{3\pi/4}/{-\frac{\sqrt{2}}2,\frac{\sqrt{2}}2}/above left/none, 
	150/ {5\pi/6}/{-\frac{\sqrt{3}}2,\frac{1}2}/above left/none,%
	180/ {\pi}/{-1,0}/left/white, 
	210/{7\pi/6}/{-\frac{\sqrt{3}}2,-\frac{1}2}/below left/none, 
	225/{5\pi/4}/{-\frac{\sqrt{2}}2,-\frac{\sqrt{2}}2}/below left/none, 
	240/{4\pi/3}/{-\frac{1}2,-\frac{\sqrt{3}}2}/below left/none,
	270/{3\pi/2}/{0,-1}/below/white, 
	300/{5\pi/3}/{\frac{1}2,-\frac{\sqrt{3}}2}/below right/none, 
	315/{7\pi/4}/{\frac{\sqrt{2}}2,-\frac{\sqrt{2}}2}/below right/none, 
	330/{11\pi/6}/{\frac{\sqrt{3}}2,-\frac{1}2}/below right/none%
}
{%
	\draw (\x:.65cm) node [fill=\v] {\scriptsize \x$^\circ$};
	\draw (\x:.85cm) node [fill=\v] {\scriptsize $\y$};
	\draw (\x:1cm) node [\w,fill=\v] {\scriptsize $\left(\z\right)$};
	\draw [fill=black] (\x:1) circle (.5pt);
}
\draw [thick] (0,0) circle (1);
\end{tikzpicture}
\end{minipage}%
%
\begin{minipage}[t]{.45\linewidth}
\subsection*{Definitions of the Trigonometric Functions}

\noindent%
\small
%\begin{minipage}[t]{.48\linewidth}
\subsubsection*{Unit Circle Definition}
%\textbf{\normalsize Unit Circle Definition}

\noindent%
\begin{minipage}{.56\linewidth}
\centering
\vskip 0in\begin{tikzpicture}[>=latex,scale=1.5,thick]
\draw [<->](-1.3,0)--(1.3,0) node [right] {$x$};
\draw [<->] (0,-1.3) -- (0,1.3) node [above] {$y$};
\draw (0,0) circle (1);
\draw [fill= black] (-.6,.8) circle (1pt);
\draw (0,0) -- (-.6,.8) node [above left] {$(x,y)$};
\draw [->] (.5,0) arc (0:127:.5);
\draw [dashed,thin] (-.6,.8) -- (-.6,0) node [pos=.5,left] {$y$};
\draw (-.3,0) node [below] {$x$};
\draw (.45,.45) node {$\theta$};
\end{tikzpicture}
\end{minipage}%
\begin{minipage}{.4\linewidth}
\small
\begin{align*}
\sin\theta &= y & \cos\theta &= x \\
\csc\theta &= \dfrac1y & \sec\theta &= \dfrac1x \\
\tan\theta &= \frac yx & \cot\theta &= \frac xy
\end{align*}
\end{minipage}
%
\subsubsection*{Right Triangle Definition}

\noindent%
\begin{minipage}{.56\linewidth}
 \centering
 \begin{tikzpicture}[thick]
  \draw (0,0) -- (2.5,0) node [below,pos=.5] {Adjacent} -- (2.5,2) node [pos=.5,rotate=-90,shift={(0pt,7pt)}] {Opposite} -- (0,0) node [pos=.5,above,rotate=38.7] {Hypotenuse} node [shift={(20pt,8pt)}] {$\theta$};
  \draw[->,>=latex] (1,0) arc (0:38.7:1);
  \draw (2.2,0) -- (2.2,.3) -- (2.5,.3);
 \end{tikzpicture}
\end{minipage}%
\begin{minipage}{.4\linewidth}
 \small
 \begin{align*}
  \sin\theta &= \frac{\text{O}}{\text{H}} & \csc\theta &= \frac{\text{H}}{\text{O}} \\
  \cos\theta &= \frac{\text{A}}{\text{H}} & \sec\theta &= \frac{\text{H}}{\text{A}} \\
  \tan\theta &= \frac{\text{O}}{\text{A}} & \cot\theta &= \frac{\text{A}}{\text{O}}
 \end{align*}
\end{minipage}
\end{minipage}

\subsection*{Common Trigonometric Identities}

\noindent%
\begin{minipage}[t]{.25\linewidth}
	\subsubsection*{Pythagorean~Identities}
	\begin{align*}
		\sin ^2x+\cos ^2x= 1 \\
		\tan^2x+ 1 = \sec^2 x \\
		1 + \cot^2x=\csc^2 x
	\end{align*}
\end{minipage}%
\begin{minipage}[t]{.45\linewidth}
	\subsubsection*{Cofunction Identities}
	\begin{align*}
		\sin\left(\frac{\pi}{2}-x\right) &= \cos x &
		\csc\left(\frac{\pi}{2}-x\right) &= \sec x \\
		\cos\left(\frac{\pi}{2}-x\right) &= \sin x &
		\sec\left(\frac{\pi}{2}-x\right) &= \csc x \\
		\tan\left(\frac{\pi}{2}-x\right) &= \cot x &
		\cot\left(\frac{\pi}{2}-x\right) &= \tan x
	\end{align*}
\end{minipage}%
\begin{minipage}[t]{.25\linewidth}
	\subsubsection*{Double~Angle~Formulas}
	\begin{align*}
		\sin 2x &= 2\sin x\cos x \\
		\cos 2x &= \cos^2x - \sin^2 x \\
		&= 2\cos^2x-1 \\
		&= 1-2\sin^2x \\
		\tan 2x &= \frac{2\tan x}{1-\tan^2 x}
	\end{align*}
\end{minipage}

\bigskip

\noindent%
\begin{minipage}[t]{.44\linewidth}
\subsubsection*{Sum to Product Formulas}
\begin{align*}
\sin x+\sin y &= 2\sin \left(\frac{x+y}2\right)\cos\left(\frac{x-y}2\right) &~\\
\sin x-\sin y &= 2\sin \left(\frac{x-y}2\right)\cos\left(\frac{x+y}2\right) \\
\cos x+\cos y &= 2\cos \left(\frac{x+y}2\right)\cos\left(\frac{x-y}2\right) \\
\cos x-\cos y &= 2\sin \left(\frac{x+y}2\right)\sin\left(\frac{y-x}2\right)
\end{align*}
\end{minipage}%
\begin{minipage}[t]{.3\linewidth}
\subsubsection*{Power--Reducing Formulas}
\begin{align*}
\sin^2 x &= \frac{1-\cos 2x}{2} & \vphantom{\left(\frac11\right)}\\
\cos^2 x &= \frac{1+\cos 2x}{2} & \vphantom{\left(\frac11\right)}\\
\tan^2 x &= \frac{1-\cos 2x}{1+\cos 2x}
\end{align*}
\end{minipage}%
\begin{minipage}[t]{.25\linewidth}
\subsubsection*{Even/Odd Identities}
\begin{align*}
\sin(-x) &= -\sin x &~\\
\cos(-x) &= \phantom{-}\cos x \\
\tan(-x) &= -\tan x \\
\csc(-x) &= -\csc x \\
\sec(-x) &= \phantom{-}\sec x \\
\cot(-x) &= -\cot x
\end{align*}
\end{minipage}

\bigskip

\noindent
\begin{minipage}[t]{.45\linewidth}
\subsubsection*{Product to Sum Formulas}
\begin{align*}
\sin x\sin y &= \frac12\big(\cos(x-y)-\cos(x+y)\big) &~\\
\cos x\cos y &= \frac12\big(\cos(x-y)+\cos(x+y)\big) \\
\sin x\cos y &= \frac12\big(\sin(x+y)+\sin(x-y)\big)
\end{align*}
\end{minipage}%
\begin{minipage}[t]{.45\linewidth}
\subsubsection*{Angle Sum/Difference Formulas}
\begin{align*}
\sin (x\pm y) &= \sin x\cos y \pm \cos x\sin y & \vphantom{\frac11}\\
\cos (x\pm y) &= \cos x\cos y \mp \sin x\sin y & \vphantom{\frac11}\\
\tan (x\pm y) &= \frac{\tan x\pm \tan y}{1\mp \tan x\tan y}
\end{align*}
\end{minipage}

\clearpage

\subsection*{Areas and Volumes}

\begin{tabular}{llll}
	{\begin{minipage}[t]{.22\linewidth}
		\subsubsection*{Triangles}
		\begin{flalign*}
			&h=a\sin\theta &\\
			&\text{Area} = \frac12bh \\
			&\text{Law of Cosines:} \\
			&c^2=a^2+b^2-2ab\cos\theta
		\end{flalign*}
		~ % to force some space between this row and the next
	\end{minipage}}
	&
	\begin{minipage}[t]{.22\linewidth}
		~\vspace{0pt}\\
		\begin{tikzpicture}[x=30pt,y=30pt,thick]
			\draw (0,0) -- node [below]  { $b$} (3,0) node [shift={(-15pt,8pt)}] {$\theta$} -- node [above right] { $a$} (2,1.5) -- node [above left] { $c$} (0,0);
			\draw (2.7,0) arc (180:125:.3);
			\draw [dashed] (2,1.5) -- (2,0) node [pos=.5,left] {$h$};
			\draw (2,.2) -- (1.8,.2) -- (1.8,0);
		\end{tikzpicture}
	\end{minipage}
	&
	{\begin{minipage}[t]{.22\linewidth}
		\subsubsection*{Right Circular Cone}
		\begin{flalign*}
			&\text{Volume} = \frac 13 \pi r^2h &\\
			&\text{Surface Area} = \\
			&\pi r\sqrt{r^2+h^2} +\pi r^2
		\end{flalign*}
	\end{minipage}}
	&
	\begin{minipage}[t]{.22\linewidth}
		~\vspace{0pt}\\
		\begin{tikzpicture}[x=13pt,y=15pt,thick]
			\begin{scope}[xscale=2]
				\draw (-1,0) arc (-180:0:1);
				\draw [dashed] (1,0) arc (0:180:1);
			\end{scope}
			\draw (-2,.1) -- (0,3) -- (2,.15);
			\draw [dashed] (0,3) -- node [left] {\small $h$} (0,0);
			\draw [dashed] (0,0) -- node [above] {\small $r$} (2,0);
			\draw [fill=black] (0,0) circle (1pt);
		\end{tikzpicture}
	\end{minipage}
	\\
	\begin{minipage}[t]{.23\linewidth}
		\subsubsection*{Parallelograms}
		Area = $bh$
	\end{minipage}
	&
	\begin{minipage}[t]{.22\linewidth}
		~\vspace{0pt}\\
		\begin{tikzpicture}[x=30pt,y=25pt,thick]
			\draw (0,0) -- node [below]  { $b$} (2,0) -- (3,1.5) -- (1,1.5) -- (0,0);
			\draw [dashed] (1,1.5) -- node [right] {$h$} (1,0);
			\draw (.8,0) -- (.8,.2) -- (1,.2);
		\end{tikzpicture}
	\end{minipage}
	&
	{\begin{minipage}[t]{.22\linewidth}
		\subsubsection*{Right Circular Cylinder}
		\begin{flalign*}
			&\text{Volume} = \pi r^2h &\\
			&\text{Surface Area} = \\
			&2\pi rh  +2\pi r^2
		\end{flalign*}
	\end{minipage}}
	&
	\begin{minipage}[t]{.22\linewidth}
		~\vspace{0pt}\\
		\begin{tikzpicture}[x=13pt,y=14pt,thick]
			\begin{scope}[xscale=2]
				\draw (-1,0) arc (-180:0:1);
				\draw [dashed] (1,0) arc (0:180:1);
			\end{scope}
			\draw (0,2.5) ellipse [x radius=2,y radius=1];
			\draw (-2,0) -- (-2,2.5) (2,0) -- node [right] {$h$} (2,2.5);
			\draw (0,2.5) -- node [above] {$r$} (2,2.5);
			\draw [fill=black] (0,2.5) circle (1pt);
		\end{tikzpicture}\bigskip\\~
	\end{minipage}
	\\\addlinespace[4\baselineskip]
	\begin{minipage}[t]{.23\linewidth}
		\subsubsection*{Trapezoids}
		Area = $\frac12(a+b)h$
	\end{minipage}
	&
	\begin{minipage}[t]{.22\linewidth}
		~\vspace{0pt}\\
		\begin{tikzpicture}[x=30pt,y=25pt,thick]
			\draw (0,0) -- node [below,pos=.7]  { $b$} (3,0) -- (2.5,1.5) -- node [above] {$a$} (1.5,1.5) -- (0,0);
			\draw [dashed] (1.5,1.5) -- node [right] {$h$} (1.5,0);
			\draw (1.3,0) -- (1.3,.2) -- (1.5,.2);
		\end{tikzpicture}\bigskip\\~
	\end{minipage}
	&
	{\begin{minipage}[t]{.22\linewidth}
		\subsubsection*{Sphere}
		\begin{flalign*}
			&\text{Volume} = \frac43\pi r^3 &\\
			&\text{Surface Area} = 4\pi r^2
		\end{flalign*}
	\end{minipage}}
	&
	\begin{minipage}[t]{.22\linewidth}
		~\vspace{0pt}\\
		\begin{tikzpicture}[x=13pt,y=13pt,thick]
			\begin{scope}[xscale=2]
				\draw (-1,0) arc (-180:0:1);
				\draw [dashed] (1,0) arc (0:180:1);
			\end{scope}
			\draw (0,0) circle (2);
			\draw [dashed] (0,0) -- node [above] {$r$} (2,0);
			\draw [fill=black] (0,0) circle (1pt);
		\end{tikzpicture}
	\end{minipage}
	\\\addlinespace[4\baselineskip]
	{\begin{minipage}[t]{.22\linewidth}
		\subsubsection*{Circles}
		\begin{flalign*}
			&\text{Area} = \pi r^2 &\\
			&\text{Circumference} = 2\pi r
		\end{flalign*}
	\end{minipage}}
	&
	\begin{minipage}[t]{.22\linewidth}
		~\vspace{0pt}\\
		\begin{tikzpicture}[x=30pt,y=30pt,thick]
			\draw (0,0) circle (1);
			\draw [dashed] (0,0) -- node [above] {$r$} (1,0);
			\draw [fill=black] (0,0) circle (1pt);
		\end{tikzpicture}
	\end{minipage}
	&
	{\begin{minipage}[t]{.22\linewidth}
		\subsubsection*{General Cone}
		\begin{flalign*}
			&\text{Area of Base} = A &\\
			&\text{Volume} = \frac13Ah
		\end{flalign*}
	\end{minipage}}
	&
	\begin{minipage}[t]{.22\linewidth}
		~\vspace{0pt}\\
		\begin{tikzpicture}[x=13pt,y=10pt,thick]
			\begin{scope}
				\clip (0,0) rectangle (4,-2.5);
				\draw [smooth] plot coordinates {(0,0) (1,1.5) (2,1.5) (4,0) (3,-1) (2,-1.5) (1,-2) (0,0)};
			\end{scope}
			\begin{scope}
				\clip (0,0) rectangle (4,2.5);
				\draw [smooth,dashed] plot coordinates {(0,0) (1,1.5) (2,1.5) (4,0) (3,-1) (2,-1.5) (1,-2) (0,0)};
			\end{scope}
			\draw (0,0) -- (2,4) -- (4,0);
			\draw [dashed] (2,0) -- node [right] {$h$}(2,4);
			\draw [fill=black] (2,0) circle (1pt);
			\draw (1.5,-.75) node {$A$};
		\end{tikzpicture}\bigskip\\~
	\end{minipage}
	\\\addlinespace[4\baselineskip]
	{\begin{minipage}[t]{.22\linewidth}
		\subsubsection*{Sectors of Circles}
		\begin{flalign*}
			&\theta \text{ in radians} &\\
			&\text{Area} = \frac12\theta r^2 \\
			&s=r\theta
		\end{flalign*}
	\end{minipage}}
	&
	\begin{minipage}[t]{.22\linewidth}
		~\vspace{0pt}\\
		\begin{tikzpicture}[x=30pt,y=30pt,thick]
			\draw (2,0) arc (0:50:2) -- (0,0);
			\draw [] (0,0) -- node [below] {$r$} (2,0);
			\draw [fill=black] (0,0) circle (1pt);
			\draw (1.95,1.0) node {$s$};
			\draw (0,0) node [shift={(15pt,8pt)}] {$\theta$};
		\end{tikzpicture}
	\end{minipage}
	&
	{\begin{minipage}[t]{.22\linewidth}
		\subsubsection*{General Right Cylinder}
		\begin{flalign*}
			&\text{Area of Base} = A &\\
			&\text{Volume} = Ah
		\end{flalign*}
	\end{minipage}}
	&
	\begin{minipage}[t]{.22\linewidth}
		~\vspace{0pt}\\
		\begin{tikzpicture}[x=13pt,y=10pt,thick]
			\begin{scope}
				\clip (0,0) rectangle (4,-2.5);
				\draw [smooth] plot coordinates {(0,0) (1,1.5) (2,1.5) (4,0) (3,-1) (2,-1.5) (1,-2) (0,0)};
			\end{scope}
			\begin{scope}
				\clip (0,0) rectangle (4,2.5);
				\draw [smooth,dashed] plot coordinates {(0,0) (1,1.5) (2,1.5) (4,0) (3,-1) (2,-1.5) (1,-2) (0,0)};
			\end{scope}
			\begin{scope}[shift={(0,4)}]
				\draw [smooth] plot coordinates {(0,0) (1,1.5) (2,1.5) (4,0) (3,-1) (2,-1.5) (1,-2) (0,0)};
			\end{scope}
			\draw (0,0) -- (0,4) (4,0) -- (4,4) node [pos=.5,right] {$h$};
			\draw (2,0) node {$A$};
		\end{tikzpicture}
	\end{minipage}
\end{tabular}

\clearpage

\section*{Algebra}

\subsection*{Factors and Zeros of Polynomials}
Let $p(x) = a_n x^n + a_{n-1} x^{n-1} + \cdots + a_1 x + a_0$ be a polynomial.  If $p(a)=0$, then $a$ is a $zero$ of the polynomial and a solution
of the equation $p(x)=0$.  Furthermore, $(x-a)$ is a $factor$ of the polynomial.

\subsection*{Fundamental Theorem of Algebra}
An $n$th degree polynomial has $n$ (not necessarily distinct) zeros.  Although all of these zeros may be imaginary, a real polynomial of odd degree
must have at least one real zero.

\subsection*{Quadratic Formula}
If $p(x) = ax^2 + bx + c$, %and $0 \le b^2 - 4ac$,
then the zeros of $p$ are $x=\dfrac{-b\pm \sqrt{b^2-4ac}}{2a}$

\subsection*{Special Factoring}
\begin{flalign*}
x^2 - a^2 &= (x-a)(x+a)
&
x^3 \pm a^3 &= (x\pm a)(x^2\mp ax+a^2)
&
x^4 - a^4 &= (x^2-a^2)(x^2+a^2)
\end{flalign*}

\subsection*{Binomial Theorem}
\begin{align*}
(x+y)^2 &= x^2 + 2xy + y^2 &
(x+y)^3 &= x^3 + 3x^2y + 3xy^2 + y^3 \\
(x+y)^4 &= x^4 + 4x^3y + 6x^2y^2 + 4xy^3 + y^4 &
(x+y)^n &=\sum_{i=0}^n \binom{n}{k}x\primeskip^{n-k}y\primeskip^k
\end{align*}

\subsection*{Rational Zero Theorem}
If $p(x) = a_n x^n + a_{n-1} x^{n-1} + \dotsb + a_1 x + a_0$ has integer coefficients, then every $rational$ $zero$ of $p$ is of the form
$x=r/s$, where $r$ is a factor of $a_0$ and $s$ is a factor of $a_n$.

\subsection*{Factoring by Grouping}
$ac x^3 + adx^2 + bcx + bd = ax^2(cs+d)+b(cx+d)=(ax^2+b)(cx+d)$

\subsection*{Arithmetic Operations}
\begin{align*}
&ab+ac=a(b+c) && \frac{a}{b}+\frac{c}{d} = \frac{ad+bc}{bd} && \frac{a+b}{c} = \frac{a}{c} + \frac{b}{c} \\[.3\baselineskip]
&\frac{\left(\dfrac{a}{b}\right)}{\left(\dfrac{c}{d}\right)}=\left(\frac{a}{b}\right)\left(\frac{d}{c}\right)=\frac{ad}{bc} 
&& \frac{\left(\dfrac{a}{b}\right)}{c} = \frac{a}{bc}
&& \frac{a}{\left(\dfrac{b}{c}\right)} = \frac{ac}{b} \\[.3\baselineskip]
&a\left(\frac{b}{c}\right)= \frac{ab}{c} && \frac{a-b}{c-d}=\frac{b-a}{d-c} && \frac{ab+ac}{a}=b+c
\end{align*}

\subsection*{Exponents and Radicals}
\begin{flalign*}
&a^0=1, \; \; a \ne 0 & (ab)^x&=a^xb^x & a^xa^y &= a^{x+y} & \sqrt{a}&=a^{1/2} & \frac{a^x}{a^y}&=a^{x-y} & \sqrt[n]{a}&=a^{1/n} \\
&\left(\frac{a}{b}\right)^x=\frac{a^x}{b^x} & \sqrt[n]{a^m}&=a^{m/n} & a^{-x}&=\frac{1}{a^x} & \sqrt[n]{ab}&=\sqrt[n]{a}\sqrt[n]{b} &
(a^x)^y&=a^{xy} & \sqrt[n]{\frac{a}{b}}&=\frac{\sqrt[n]{a}}{\sqrt[n]{b}}
\end{flalign*}

\clearpage

\section*{Additional Formulas}

\subsection*{Summation Formulas}

\begin{align*}
\sum^n_{i=1}{c} &= cn
&
\sum^n_{i=1}{i} &= \frac{n(n+1)}{2}
&
\sum^n_{i=1}{i\hskip1pt^2} &= \frac{n(n+1)(2n+1)}{6}
&
\sum^n_{i=1}{i\hskip1pt^3} &= \left(\frac{n(n+1)}{2}\right)^2
\end{align*}

\subsection*{Trapezoidal Rule}

\noindent$\ds\int_a^b{f(x)}\ dx \approx \frac{\Delta x}{2}\left[f(x_1)+2f(x_2) + 2f(x_3) + \dotsb + 2f(x_{n}) + f(x_{n+1})\right]$\smallskip\\
with  $\text{Error} \leq \dfrac{(b-a)^3}{12n^2}\left[ \max \abs{\fpp(x)}\right]$

\subsection*{Simpson's Rule}

\noindent$\ds\int_a^b{f(x)}\ dx \approx \frac{\Delta x}{3}\left[f(x_1)+4f(x_2) + 2f(x_3) + 4f(x_4) + \dotsb + 2f(x_{n-1}) + 4f(x_{n}) + f(x_{n+1})\right] 
$\smallskip\\
with $\text{Error} \leq \dfrac{(b-a)^5}{180n^4}\left[ \max \abs{f\primeskip^{(4)}(x)}\right]$\bigskip\bigskip

\noindent
\begin{tabular}{ll}
 \begin{minipage}[t]{.4\linewidth}
  \subsection*{Arc Length}
  $\ds L = \int_a^b{\sqrt{1+ f\,'(x)^2}}\ dx$\bigskip\\~
 \end{minipage}
 &
 % also add volume of revolution, or nothing at all
% \begin{minipage}[t]{.4\linewidth}
%  \subsection*{Surface of Revolution}
%  $\ds S = 2\pi \int_a^b{f(x) \sqrt{1+ f\,'(x)^2}}\ dx  $\smallskip\\
%  {\small (where $f(x)\geq 0$)}\medskip\\
%  $\ds S = 2\pi \int_a^b{x \sqrt{1+ f\,'(x)^2}}\ dx 
%  $\smallskip\\
%  {\small (where $a,b \geq 0$)}\bigskip\\~
% \end{minipage}
 \\
 \begin{minipage}[t]{.4\linewidth}
  \subsection*{Work Done by a Variable Force}
  $\ds W = \int_a^b{F(x)}\ dx$
 \end{minipage}
 &
 \begin{minipage}[t]{.4\linewidth}
  \subsection*{Force Exerted by a Fluid}
  $\ds F = \int_a^b{w\,d(y)\,\ell(y)}\ dy$
 \end{minipage}
\end{tabular}

\bigskip

\subsection*{Taylor Series Expansion for $f(x)$}
\noindent$\ds p_n(x) = f(c) + \fp(c)(x-c) + \frac{\fpp(c)}{2!}(x-c)^2 + \frac{f\,'''(c)}{3!}(x-c)^3 + \dotsb + \frac{f\,^{(n)}(c)}{n!}(x-c)^n$
\bigskip

%\subsection*{Maclaurin Series Expansion for $f(x)$} %{, where $c=0$}
%\noindent$\ds p_n(x) = f(0) + \fp(0)x + \frac{\fpp(0)}{2!}x^2 + \frac{f\,'''(0)}{3!}x^3 + \dotsb + \frac{f\,^{(n)}(0)}{n!}x^n$

\clearpage

\subsection*{Summary of Tests for Series}

\begin{center}
\addtolength{\tabcolsep}{6pt}
\begin{tabular}{ccccc}

\toprule
Test & Series & \parbox{1in}{\centering Condition(s) of Convergence} & \parbox{1in}{\centering Condition(s) of Divergence} & Comment \\\midrule

$n^{\text{th}}$-Term & $\ds\sum_{n=1}^\infty a_n$ &  & $\displaystyle{\lim_{n \to \infty} a_n \neq 0}$ & \parbox{1in}{\centering cannot show convergence.}\\[3\defaultaddspace]

\parbox{.7in}{\centering Geometric\\Series} & $\ds\sum_{n=0}^\infty ar\primeskip^n$ & $ \abs{r}< 1$ & $\abs{r}\geq 1$ & Sum $=\dfrac a{1-r}$ \\[6\defaultaddspace]

\parbox[t]{.7in}{\centering Telescoping\\Series} & $\ds\sum_{n=1}^\infty b_n-b_{n+m}$ & $\ds{\lim_{n \to \infty} b_n = L}$ & & \parbox[t]{1in}{\centering Sum $=$\\$\ds\left(\sum_{n=1}^m b_n\right)-L$} \\\addlinespace

$p$-Series & $\ds\sum_{n=1}^\infty(an+b)^{-p}$ & $p>1$ & $p\leq 1$ & \\[3\defaultaddspace]

\parbox[t]{.7in}{\centering Integral\\Test} & $\ds\sum_{n=1}^\infty a_n$ & \parbox[t]{1in}{\centering$\ds\int_1^\infty a(n)\ dn$\smallskip\\ converges} & \parbox[t]{1in}{\centering$\ds\int_1^\infty a(n)\ dn$\smallskip\\ diverges} & \parbox[t]{1in}{\centering $a_n = a(n)$ must be continuous and decreasing} \\[6\defaultaddspace]

\parbox[t]{.7in}{\centering Direct\\Comparison} & $\ds\sum_{n=1}^\infty a_n$ & \parbox[t]{1in}{\centering$\ds\sum_{n=0}^\infty b_n $\smallskip\\converges and\smallskip\\$0\leq a_n\leq b_n$}
& \parbox[t]{1in}{\centering$\ds\sum_{n=0}^\infty b_n $\smallskip\\diverges and\smallskip\\$0\leq b_n\leq a_n$} & \\[12\defaultaddspace]

\parbox[t]{.7in}{\centering Limit\\Comparison} & $\ds\sum_{n=1}^\infty a_n$ & \parbox[t]{1.3in}{\centering$\ds\sum_{n=0}^\infty b_n $\smallskip\\converges and\smallskip\\$\displaystyle \lim_{n\to\infty} a_n/b_n \geq 0$}
& \parbox[t]{1in}{\centering$\ds\sum_{n=0}^\infty b_n $\smallskip\\diverges and\begin{align*}\lim_{n\to\infty} a_n/b_n &> 0\\\text{or }&=\infty\end{align*}} \\

Ratio Test & $\ds\sum_{n=1}^\infty a_n$ & \parbox{1in}{\centering$\ds\lim_{n\to\infty} \frac{a_{n+1}}{a_n}  < 1$}
& \parbox{1in}{\begin{align*}\lim_{n\to\infty} \frac{a_{n+1}}{a_n} &> 1\\\text{or } &=\infty\end{align*}} & 
\parbox{1in}{\centering $\{a_n\}$ must be positive}\\

Root Test & $\ds\sum_{n=1}^\infty a_n$ & \parbox{1in}{\centering$\ds\lim_{n\to\infty} \big(a_n\big)^{1/n} < 1$}
& \parbox{1.2in}{\begin{align*}\lim_{n\to\infty} \big(a_n\big)^{1/n} &> 1\\\text{or } &=\infty\end{align*}} & 
\parbox{1in}{\centering $\{a_n\}$ must be positive}\\\bottomrule

\end{tabular}

\end{center}


\end{document}
