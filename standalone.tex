% !TEX program = XeLaTeX
\input{headers/pre_package}

\usepackage{headers/apex_style}

\usepackage[aux]{rerunfilecheck}

\usepackage{etoolbox}

\PassOptionsToPackage{HTML}{xcolor}
%\usepackage[HTML]{xcolor} % must occur before qrcode in apex_style
\usepackage{tikz}
\usetikzlibrary{calc}

% with 10pt font, 1em ~ 10pt ; 1ex ~ 4.3pt

%%Page Size stuff

\usepackage[paperheight=11in,paperwidth=8.5in,%
	inner=1in,textheight=7in,textwidth=320pt,marginparwidth=150pt,%
	marginparsep=32pt,bottom=3in,footskip=1.5in]{geometry}

\newcommand{\exercisegeometry}{%
%	\batchmode%
	\newgeometry{inner=72pt,outer=72pt,textheight=9.25in,tmargin=.75in,
		marginparwidth=150pt,marginparsep=32pt,footskip=29pt}%
%	\errorstopmode%
}
\newcommand{\eendgeometry}{%
%	\batchmode%
	\newgeometry{inner=72pt,outer=36pt,textheight=10in,
		marginparwidth=150pt,marginparsep=32pt}%
%	\errorstopmode%
}
\newcommand{\prefacegeometry}{%
%	\batchmode%
	\newgeometry{inner=1in,textheight=9in,textwidth=320pt,marginparwidth=150pt,%
		marginparsep=32pt,bottom=1in,footskip=1.5in}%
%	\errorstopmode%
}

\newlength{\widest}

%%% This was originally a style with \usepackage, but inputing is generally
%%% equivalent.  The only real difference is how latexml handles style files.
%%% So we'll input this document as a header instead,
%%% and save \usepackage{customstyle}
%%% for things latexml is having trouble with.
%%% This does mean that the distinction between APEX_format and Header_Calculus
%%% is no longer important, and mostly historical.

% do we want to print the keys for the labels? if so, uncomment
%\usepackage[notref,notcite]{showkeys}

\usepackage{amsthm}
\usepackage{amsmath}
%\usepackage{amssymb} % todo ? https://tex.stackexchange.com/a/3000/107497 recommends dropping amssymb in favor of unicode-math
% but then the font loading gets messed up with mathspec
% see also https://tex.stackexchange.com/q/218112/107497 that if the font doesn't have math, then it's a losing battle
%\RequirePackage{unicode-math}

\usepackage{graphicx}
\usepackage{multicol}
\usepackage{makeidx}

\usepackage[normalem]{ulem}

\usepackage{calc}
\usepackage{ragged2e}

%\usepackage[inline]{enumitem}
\usepackage{enumext}

\usepackage[nocomments]{latexml}
\lxDocumentID{apex}

\numberwithin{figure}{section}
\numberwithin{equation}{section}

%%%%%%%%%%%%%%%%%%%%%

\makeindex

\newcommand{\apex}{\texorpdfstring{A\kern -.1em \lower -.5ex\hbox{P}\kern -.25em\lower .5ex\hbox{E}\kern -.1em X}{APEX}}


% Create boolean for whether or not to print 3D graphics. 
% Also creates command to switch back and forth; "looks better."
\newtoggle{in_threeD}
\newcommand{\usethreeDgraphics}{\toggletrue{in_threeD}}
\newcommand{\usetwoDgraphics}{\togglefalse{in_threeD}}
\usethreeDgraphics


\usepackage{pgfplots}
\pgfplotsset{compat=1.8}

\newtoggle{inColor}
\toggletrue{inColor}

\pgfplotsset{colormap={coloronemap}{rgb=(.4,.4,1); rgb=(.8,.8,1)}}
\pgfplotsset{colormap={colortwomap}{rgb=(1,.4,.4); rgb=(1,.8,.8)}}
%\usepgfplotslibrary{external}
% only needed for external tikz pictures (and not liked by latexml)
% see http://tex.stackexchange.com/a/1475/107497
\usetikzlibrary{calc}
\usetikzlibrary{shadings}

% these will be renewcommanded
\newcommand{\colorone}{blue}
\newcommand{\colortwo}{red}
\newcommand{\colorthree}{green}
\newcommand{\coloronefill}{blue!15!white}
\newcommand{\colortwofill}{red!15!white}
\newcommand{\colormapone}{rgb=(.4,.4,1); rgb=(.8,.8,1)}
\newcommand{\colormaptwo}{rgb=(1,.4,.4); rgb=(1,.8,.8)}
\newcommand{\colormapplaneone}{rgb=(.7,.7,1); rgb=(.9,.9,1)}
%\definecolor{colormaponebottom}{rgb}{.4,.4,1}
%\definecolor{colormaponetop}{rgb}{.8,.8,1}
%\definecolor{colormaptwobottom}{rgb}{1,.4,.4}
%\definecolor{colormaptwotop}{rgb}{1,.8,.8}

% determines the line colors for color and black and white lines.
\newcommand{\colorlinecolor}{blue!95!black!30}
\newcommand{\bwlinecolor}{black!30}

% sets the line color to be in color, as a default
\newcommand{\thelinecolor}{\colorlinecolor}

% this allows the above default to be overriden by using
% the \printincolor and \printinblackandwhite commands
% anywhere in the file. This allows you to switch back
% and forth between bw and color. (Who would want to?)
\newcommand{\colornamesuffix}{}

\newcommand{\printincolor}{
 \toggletrue{inColor}%
 % aforementioned renewcommanding
 \renewcommand{\thelinecolor}{\colorlinecolor}
 \renewcommand{\colornamesuffix}{}
 \renewcommand{\colorone}{blue}
 \renewcommand{\colortwo}{red}
 \renewcommand{\colorthree}{green}
 \renewcommand{\coloronefill}{blue!15!white}
 \renewcommand{\colortwofill}{red!15!white}
 \renewcommand{\colormapone}{rgb=(.4,.4,1); rgb=(.8,.8,1)}
 \renewcommand{\colormaptwo}{rgb=(1,.4,.4); rgb=(1,.8,.8)}
 \renewcommand{\colormapplaneone}{rgb=(.7,.7,1); rgb=(.9,.9,1)}
 \definecolor{colormaponebottom}{rgb}{.4,.4,1}
 \definecolor{colormaponetop}{rgb}{.8,.8,1}
 \definecolor{colormaptwobottom}{rgb}{1,.4,.4}
 \definecolor{colormaptwotop}{rgb}{1,.8,.8}
 \setexvideocolor
 \colorizespecialboxes
}

\newcommand{\printinblackandwhite}{
 \togglefalse{inColor}%
 % undoing the above renewcommanding
 \renewcommand{\thelinecolor}{\bwlinecolor}
 \renewcommand{\colornamesuffix}{BW}
 \renewcommand{\colorone}{black}
 \renewcommand{\colortwo}{black!50!white}
 \renewcommand{\colorthree}{black!25!white}
 \renewcommand{\coloronefill}{black!15!white}
 \renewcommand{\colortwofill}{black!05!white}
 \renewcommand{\colormapone}{rgb=(.4,.4,.4); rgb=(.7,.7,.7)}
 \renewcommand{\colormaptwo}{rgb=(.6,.6,.6); rgb=(.9,.9,.9)}
 \renewcommand{\colormapplaneone}{rgb=(.8,.8,.8); rgb=(.95,.95,.95)}
 \definecolor{colormaponebottom}{rgb}{.4,.4,.4}
 \definecolor{colormaponetop}{rgb}{.7,.7,.7}
 \definecolor{colormaptwobottom}{rgb}{.6,.6,.6}
 \definecolor{colormaptwotop}{rgb}{.9,.9,.9}
 \setexvideobw
 \bwizespecialboxes
}


\newcommand{\myincludegraphics}[2][]{%
 \IfFileExists{./#2\colornamesuffix.png}{%
  \includegraphics[#1]{#2\colornamesuffix}%
 }{%
  \IfFileExists{./#2\colornamesuffix.pdf}{%
   \includegraphics[#1]{#2\colornamesuffix}%
  }{%
   \IfFileExists{./#2.png}{%
    \includegraphics[#1]{#2}%
   }{%
    \IfFileExists{./#2.pdf}{%
     \includegraphics[#1]{#2}%
    }{%
     \includegraphics[#1]{#2\colornamesuffix}%
    }%
   }%
  }%
 }%
}



%%%%%%%%%%%%%%%%%%%%%%%%%%%%%%%%%%%%%%%%%%%%%%%%%%%%%%%%%%%%%%%%%%%%%%%%%%%%%%
%% Examples
%%%%%%%%%%%%%%%%%%%%%%%%%%%%%%%%%%%%%%%%%%%%%%%%%%%%%%%%%%%%%%%%%%%%%%%%%%%%%%

\newlength{\boxskipamount}
\setlength{\boxskipamount}{4ex plus 4ex minus 2ex}

%\newlength{\topmarginlength} 
%\newlength{\bottommarginlength}
%\newlength{\oddpagemarginlength}
%\newlength{\evenpagemarginlength}
\newlength{\marginlinelength}
%\newlength{\innerpagemarginlength}

% how far from the text the example line is to be drawn
\setlength{\marginlinelength}{.2em}

% the height of the top margin (used in calculating the lines for examples)
%\setlength{\topmarginlength}{-1in-\voffset}

% the length of the bottom margin (ish)
% actually starts at the top of the page, moves
% through the top margin length then the text height.
%\setlength{\bottommarginlength}{-1in-\textheight-2\baselineskip-\voffset-\headheight-\headsep-\topmargin}

% the length of the left hand margin of an odd page
%\setlength{\oddpagemarginlength}{1in+\hoffset+\oddsidemargin-2\marginlinelength}

% the length of the left hand margin of an even page
%\setlength{\evenpagemarginlength}{1in+\hoffset+\evensidemargin-2\marginlinelength}

\newcommand{\solution}{\bigbreak\par
 \makebox[6.5em][l]{\textsc{\small\textbf{Solution\lxAddClass{solutionTag}}}}%
 \nopagebreak%
}

% black: hsl(x,x,0)
% white: hsl(x,x,100)
% blue: hsl(240,100,50)
% line color: blue!95!black!30 = Hsb(240,.29,.98) = hsl(240,87.7,83.8)

\newlength{\saveparindent}
\setlength{\saveparindent}{\parindent}




%%%%%%%%%%%%%%%%%%%%%%%%%%%%%%%%%%%%%%%%%%%%%%%%%%%%%%%%%%%%%%%%%%%%%%
%% Definitions, Theorems and Key Ideas
%%%%%%%%%%%%%%%%%%%%%%%%%%%%%%%%%%%%%%%%%%%%%%%%%%%%%%%%%%%%%%%%%%%%%%

\newcommand{\newspecialbox}[3]{%
 \AtBeginDocument{\makeStyles{#1}{#3}}%
 \expandafter\newcommand\csname colorize#1\endcsname{
  \definecolor{top#1}{Hsb}{#3,.05,1}% = hsl(#4,100,97.5)
  \ifnumequal{#3}{60}{%
   \definecolor{border#1}{Hsb}{#3,.59,.97}% = hsl(#4,90.5,68.4)
   \definecolor{bottom#1}{Hsb}{#3,.28,.97}% = hsl(#4,81.9,83.4)
  }{%
   \definecolor{border#1}{Hsb}{#3,.23,.65}% = hsl(#4,17.6,57.5)
   \definecolor{bottom#1}{Hsb}{#3,.13,.92}% = hsl(#4,42.8,86)
  }%
 }
 \expandafter\newcommand\csname bwize#1\endcsname{
  \colorlet{top#1}{white}
  \colorlet{bottom#1}{white}
  \colorlet{border#1}{black}
 }
 \newtheorem{#1}{#2}[section]%
 \expandafter\providecommand\csname #1autorefname\endcsname{#2}
 \ifbool{latexml}{%
 }{%
  \tcolorboxenvironment{#1}{
    sharp corners=all,
    enhanced,
    colframe=border#1,
    beforeafter skip=\boxskipamount,
    interior style={top color=top#1, bottom color=bottom#1},
    breakable=true,
    overlay first={\continue{bottom}{#1}},
    overlay middle={\continue{bottom}{#1}\continue{top}{#1}},
    overlay last={\continue{top}{#1}},
    enlargepage flexible=3\baselineskip,
    toggle enlargement=evenpage,
    lines before break=8,
  }%
 }
}

\newcommand{\continue}[1]{%
 \csname continue#1\endcsname{#1}%
}
\newcommand{\continuetext}[2]{%
 \ifstrequal{#1}{top}{%
  \csname #2autorefname\endcsname\ \csname the#2\endcsname\ continued%
 }{%
  (continued)
 }%
}
% can't get this to work
%\newcommand{\northsouth}[2][]{\if thenelse{\equal{#2}{top}}{#1north}{#1south}}

% adapted from https://tex.stackexchange.com/a/545324/107497 by Schrödinger's cat
\newcommand{\continuebottom}[2]{
   \path[font=\small\itshape] (frame.south) node (cont) {\continuetext{#1}{#2}};
   \begin{scope}[decoration={zigzag,amplitude=0.5mm}]
    \path[fill=#1#2]
     decorate {([xshift= 1.2pt]frame.south west) -- (cont.west)} --++ (0,0.5ex)
      -| cycle
     decorate {([xshift=-1.2pt]frame.south east) -- (cont.east)} --++ (0,0.5ex)
      -| cycle;
    \path[fill=white]
     decorate {([xshift= 1.2pt]frame.south west) -- (cont.west)} --++ (0,-0.5ex)
      -| cycle
     decorate {([xshift=-1.2pt]frame.south east) -- (cont.east)} --++ (0,-0.5ex)
      -| cycle;
   \end{scope} 
}
\newcommand{\continuetop}[2]{
   \path[font=\small\itshape] (frame.north) node (thm) {\continuetext{#1}{#2}};
   \begin{scope}[decoration={zigzag,amplitude=0.5mm}]
    \path[fill=#1#2]
     decorate {([xshift= 1.2pt]frame.north west) -- (thm.west)} --++ (0,-0.5ex)
      -| cycle
     decorate {([xshift=-1.2pt]frame.north east) -- (thm.east)} --++ (0,-0.5ex)
      -| cycle;
    \path[fill=white]
     decorate {([xshift= 1.2pt]frame.north west) -- (thm.west)} --++ (0,0.5ex)
      -| cycle
     decorate {([xshift=-1.2pt]frame.north east) -- (thm.east)} --++ (0,0.5ex)
      -| cycle;
   \end{scope} 
}

\newcommand{\colorizespecialboxes}{
 \colorizedefinition
 \colorizetheorem
 \colorizekeyidea
}
\newcommand{\bwizespecialboxes}{
 \bwizedefinition
 \bwizetheorem
 \bwizekeyidea
}




%%%%%%%%%%%%%%%%%%%%%%%%%%%%%%%%%%%%%%%%%%%%%%%%%%%%%%%%%%%%%%%%%
%% Exercises
%% We would like to make better use of enumitem to put implementation
%% details here instead of repeating them, but the pdftagging
%% doesn't deal well with that.  So we'll need to repeat everything every time.
%%%%%%%%%%%%%%%%%%%%%%%%%%%%%%%%%%%%%%%%%%%%%%%%%%%%%%%%%%%%%%%%%

\setlength{\columnsep}{20pt}

\newtoggle{inexercises}

%\makeatletter
%\newcommand{\exercisesubsubsection}{%
% \closeenumerate%
% \@startsection{subsubsection}{3}{-1em}{\bigskipamount}{\bigskipamount}{\Large\textit}*}
%\makeatother

% I'd like to move the \closeenumerate into the \exercisesubsubsection, but I can't figure it out
\newcommand{\printconcepts}{\noindent\closeenumerate\exercisesubsubsection*{\noindent Terms and Concepts}}
\newcommand{\printproblems}{\noindent\closeenumerate\exercisesubsubsection*{\noindent Problems}}
\newcommand{\printreview}{\noindent\closeenumerate\exercisesubsubsection*{\noindent Review}}

%\newlist{sectionexercises}{enumerate}{1}
%\newcounter{saveexercisenum}[section]
%\counterwithin*{sectionexercisesi}{section} % in case we have a exercise set before any exercises that would reset the save exercise enum
%\setlist[sectionexercises]{
%	label=\arabic*.,
%	leftmargin=1.5em,
%    before=\setcounter{sectionexercisesi}{\value{saveexercisenum}},
%    after=\setcounter{saveexercisenum}{\value{sectionexercisesi}},
%}
%\ifbool{latexml}{}{
% \setlist*[sectionexercises]{ref=\arabic*}
%}

%\setenumext[enumext,2]{start=1}
\setenumext[enumext,1]{resume}
\resetenumext[1]{subsection} % reset the resumed counter for exercises

\makeatletter
\newcommand{\printexercises}[1]{%
 \writeToAnsFile{#1}% writeToAnsFile in sty (actually, down below)
 \exercisegeometry% includes a clearpage
 \pagestyle{exercise}%
 \bookmarksetupnext{level=\toclevel@subsection}% otherwise, the level is "section" and everything is messed up
 \exercisesubsection{Exercises \thesection}%
 \stepcounter{subsection}% subsections aren't numbered, but this triggers resetenumext
 \label{exer\thesection}%
 \small%
 \bigskip%
 \begin{multicols}{2}%
  \toggletrue{inexercises}%
  \renewcommand{\Itemautorefname}{Ex\-er\-cise}% local b/c multicols = good
  \input{#1}%
  \closeenumerate%
 \end{multicols}%
 \restoregeometry%
 \pagestyle{prose}%
 %	\easypagecheck
 \setlength{\hoffset}{0pt} \rmfamily\normalsize \bigbreak%
}
\makeatother


\newwrite\answrite %write the answers file
% give the answers file the name ``jobname.answers''
\openout\answrite=\jobname.answers

\newcommand{\writeToAnsFile}[1]{%
 \immediate\write\answrite{%
  \string\answersForSection{\arabic{chapter}}{\arabic{section}}{#1}%
%  \noexpand\answersForSection{\arabic{chapter}}{\arabic{section}}{#1}%
 }%
}
% \noexpand\answersForSection becomes \relax in LaTeXML
% \string works with both

\newcounter{exercisesetcounter}[section]
\renewcommand{\theexercisesetcounter}{\thesection.\arabic{exercisesetcounter}}

\newcounter{saveenumi}

%\counterwithin*{enumXi}{subsection}
% does not work because \exercisesubsection is a fake \subsection

% #1 is "Exercises \thesection"
\newcommand{\exercisesubsectiontitle}[1]{%
 \huge\textbf{\texorpdfstring{\hyperref[sol#1]{#1}}{Exercises}}\hrule
% \setcounter{enumXi}{0}%
}

%\makeatletter
%% the usual \subsection definition has stretchable space in arguments 3-5
%\newcommand{\exercisesubsection}[1]{%
%\setcounter{enumi}{0}%
%\@startsection{subsection}{2}{-.7em}{0pt}{.5ex}{\huge\textbf}{\texorpdfstring{\hyperref[sol#1]{Exercises #1}}{Exercises}}%
%\hrule\vspace{-1.5ex}%
%}
%\makeatother

\newcommand{\exautoref}[1]{%
 \hyperref[#1]{Ex\-er\-cise~\ref*{#1}}%
% {%
%  \renewcommand{\Itemautorefname}{Exercise}% localize the upcoming reference
%  \autoref{#1}%
% }%
% doesn't work?
}

%\newcommand*{\exerenv}{sectionexercises}
\newcommand*{\exerenv}{enumext}

\makeatletter
\newcommand{\openenumerate}{%
 \ifx\@currenvir\exerenv\else%
  \begin{enumext}[widest=22,ref=\arabic*]
%  \begin{sectionexercises}
%  \ifbool{latexml}{%
%   \setcounter{sectionexercisesi}{\value{saveexercisenum}}%
%  }{}%
 \fi%
}
\newcommand{\closeenumerate}{
 \ifx\@currenvir\exerenv%
%  \ifbool{latexml}{%
%   \setcounter{saveexercisenum}{\value{sectionexercisesi}}%
%  }{}%
  \end{enumext}
%  \end{sectionexercises}
 \fi%
}
\makeatother
\newcommand{\closeenumerateinquestions}{\closeenumerate}

 % if the instructions have an enumerate, we want to use the second level
 % we can't have another enumext, because that closed before the instructions
 % so this would happen at level 1.  we could monkey around to make enumext
 % thinks it's at level 2, but this seems easier
\makeatletter
\newcommand{\stepenumeratedepth}{\advance\@enumdepth\@ne}
\makeatother

\newcommand{\exercisesetinstructions}[2][In Exercises]{%
 \setcounter{saveenumi}{\value{enumXi}}%
 \closeenumerateinquestions
 \pagebreak[2]%
 \stepcounter{saveenumi}%
 \stepcounter{exercisesetcounter}%
 \ifnumodd{\value{saveenumi}}{}{%
  \PackageInfo{apex}{%
   Exercise set \theexercisesetcounter\space begins with \arabic{saveenumi}%
  }%
 }%
 \bgroup
 \stepenumeratedepth
% \setenumext[enumext,1]{label=\alph*,wrap-label={(#1)}}% pretend it is level 2
 \noindent#1 \arabic{saveenumi}--\ref*{enumiatendof\theexercisesetcounter}%
% \renewcommand{\theenumi}{(\alph{enumi})}%
 #2%
 \egroup%
 % can't \addtocounter{enumXi}{-1} because #2 may have an enumerate
% \setcounter{enumXi}{\value{saveenumi}}
 \ignorespaces%
 \nopagebreak%
}
\newcommand{\exercisesetend}{%
 \label{enumiatendof\theexercisesetcounter}%
 \closeenumerate%
 \ifnumodd{\value{enumXi}}{%
  \PackageInfo{apex}{%
   Exercise set \theexercisesetcounter\space ends with \arabic{enumXi}%
  }%
 }{}%
}

%\newenvironment{exerciseset}[2]{%
% \stepcounter{sectionexercisesi}
% \stepcounter{exercisesetcounter}%
% \ifnumodd{\value{sectionexercisesi}}{}{%
%  \PackageInfo{apex}{%
%   Exercise set \theexercisesetcounter\space begins with \arabic{sectionexercisesi}%
%  }%
% }%
% {%
%  \setlist[enumerate,1]{label=(\alph*)}% for exercise set instructions
%  \noindent#1 \arabic{sectionexercisesi}--\ref*{enumiatendof\theexercisesetcounter}#2%
% }%
% \addtocounter{sectionexercisesi}{-1}\ignorespaces%
% \nopagebreak%
%}{%
% \label{enumiatendof\theexercisesetcounter}%
% \closeenumerate%
% \ifnumodd{\value{sectionexercisesi}}{%
%  \PackageInfo{apex}{%
%   Exercise set \theexercisesetcounter\space ends with \arabic{sectionexercisesi}%
%  }%
% }{}%
%}
%\BeforeBeginEnvironment{exerciseset}{\closeenumerateinquestions}

\newcommand{\exercise}[2]{%
 \openenumerate%
% \setlist[enumerate,1]{label=(\alph*)}% for exercise instructions
 \item \parbox[t]{\linewidth}{#1}%
}
\newcommand{\showexerciseanswers}{%
 \renewcommand{\exercise}[2]{%
  \ifboolexpr{ togl{printoddanswersonly} and test{\ifnumodd{\value{enumXi}}} }{%
   \stepcounter{enumXi}%
  }{%
   \item \parbox[t]{\linewidth}{\raggedright ##2}%
  }%
 }%
}

\newcommand{\questioncolumnbreak}{\columnbreak}


%%%%%%%%%%%%%%%%%%%%%%%%%%%%%%%%%%%%%%%%%%%%%%%%%%%%%%%%%%%%%%%%%
%% Answers
%%%%%%%%%%%%%%%%%%%%%%%%%%%%%%%%%%%%%%%%%%%%%%%%%%%%%%%%%%%%%%%%%

\newcommand{\printsolutions}[2][\jobname]{%
 \immediate\closeout\answrite%
 \inanswersection\exercisegeometry%
 \pagestyle{exercise}%
 %\thispagestyle{empty}%
 \ifstrequal{#1}{\jobname}{%
  \chapter*{#2}%
 }{%
  {% localize the next line
   \renewcommand{\thefootnote}{}
   \chapter*{#2\footnote{Revised \today}}%
  }%
 }%
 \phantomsection
 \addcontentsline{toc}{chapter}{#2}%
 \begin{multicols}{2}%
  \small\raggedright%
  \input{#1.answers}%
 \end{multicols}%
 \restoregeometry\pagestyle{prose}%
 \setlength{\hoffset}{0pt}\rmfamily%
 \pagestyle{empty}%
 \eendgeometry%
}%

\newcommand{\inanswersection}{%
	\renewcommand{\printconcepts}{}%
	\renewcommand{\printproblems}{}%
	\renewcommand{\printreview}{}%
%	\renewenvironment{exerciseset}[2]{}{}%
	\renewcommand{\exercisesetinstructions}[2][]{}%
	\renewcommand{\exercisesetend}{}%
	\renewcommand{\openenumerate}{}%
	\renewcommand{\closeenumerateinquestions}{}%
	\renewcommand{\questioncolumnbreak}{}%
	\packageinanswersection%
	\showexerciseanswers%
%	\setlist[enumerate,1]{label=\arabic*.}% for solutions
	% LaTeX already does this, but LaTeXML doesn't
}%

\newtoggle{printoddanswersonly}

\toggletrue{printoddanswersonly}
\newcommand{\printallanswers}{\togglefalse{printoddanswersonly}}

%\newcounter{answerchapter}
%\newcounter{answersection}[answerchapter]
%\renewcommand{\theanswersection}{\theanswerchapter.\arabic{answersection}}
\newcommand{\lastanswerchapter}{-1}

\newcommand*{\answersForSection}[3]{%
 \ifnumequal{#1}{\lastanswerchapter}{}{%
  \renewcommand{\lastanswerchapter}{#1}% apparently global. who knew?
  \ifbool{latexml}{}{%
   \belowpdfbookmark{Chapter #1}{solsol#1}%
  }
  % commandeer chapter and section numbering
  \setcounter{chapter}{#1}
  \section*{Chapter~\thechapter\hfill\null}
 }%
 \setcounter{section}{#2}
 \subsection*{\hyperref[exer\thesection]{Exercises~\thesection\hfill\null}}%
 \label{solExercises \thesection}
 \ifnumequal{#2}{0}{%
  \loadAllAnswers{#3}
 }{%
  \loadAnswers{#3}
 }%
}

% only called by the prerequisite sections
\newcommand*{\loadAllAnswers}[1]{%
%	\setcounter{answersection}{-1}%
	\iftoggle{printoddanswersonly}{%
		\togglefalse{printoddanswersonly}%
		\loadAnswers{#1}%
		\toggletrue{printoddanswersonly}%
	}{%
		\loadAnswers{#1}%
	}%
}
\newcommand*{\loadAnswers}[1]{%
%	\stepcounter{answersection}%
 \begin{enumext}[start=1,widest=22]
  \input{#1}
 \end{enumext}
 \bigbreak%
}



% The following creates a ``List of Theorems'', ``Definitions'', and ``Key Ideas''.
% See http://tex.stackexchange.com/q/74857/107497
%\usepackage{thmtools} % continuing theorems and ``List of Theorems''
%\patchcmd\thmtlo@chaptervspacehack
%  {\addtocontents{loe}{\protect\addvspace{10\p@}}}
%  {\addtocontents{loe}{\protect\thmlopatch@endchapter\protect\thmlopatch@chapter{\thechapter}}}
%  {}{failed thmtlo@chaptervspacehack}
%\AtEndDocument{\addtocontents{loe}{\protect\thmlopatch@endchapter}}
%\long\def\thmlopatch@chapter#1#2\thmlopatch@endchapter{%
%  \setbox\z@=\vbox{#2}%
%  \ifdim\ht\z@>\z@
%    \hbox{\bfseries\chaptername\ #1}\nobreak
%    #2
%    \addvspace{10\p@}
%  \fi
%}
%\def\thmlopatch@endchapter{}
%\patchcmd\thmt@mklistcmd
%  {\protect\numberline{\csname the\thmt@envname\endcsname}%
%      \thmt@thmname}{}{}{failed thmt@mklistcmd}
%%\makeatother
%\renewcommand\thmtformatoptarg[1]{#1}


\usepackage{makecell}

\usepackage{amsthm}

\newtheoremstyle{apexExample}% name
  {0pt}% Space above, empty = `usual value'
  {0pt}% Space below
  {}% Body font
  {}% Indent amount (empty = no indent, \parindent = para indent)
  {\bfseries}% Thm head font
  {}% Punctuation after thm head
  {\newline}% Space after thm head
  {\parbox[t]{\ifbool{latexml}{10em}{8em}}{\bfseries\thmname{#1}~\thmnumber{#2}}%
   \thmnote{\parbox[t]{.75\textwidth}{\bfseries\raggedright#3}}%
  }

\newtheoremstyle{apex}% name
  {0pt}% Space above, empty = `usual value'
  {0pt}% Space below
  {}% Body font
  {}% Indent amount (empty = no indent, \parindent = para indent)
  {\bfseries}% Thm head font
  {}% Punctuation after thm head
  {\newline}% Space after thm head
  {\parbox[t]{\ifbool{latexml}{10em}{8em}}{\bfseries\thmname{#1}~\thmnumber{#2}}%
   \thmnote{\parbox[t]{\ifbool{latexml}{.6\textwidth}{.7\textwidth}}{\bfseries\raggedright#3}}%
  }% Thm head spec
  % the padding for the box takes just a bit of room away from these that examples get to keep

\theoremstyle{apexExample}
\newtheorem{example}{Example}[section]
\newcommand{\exampleautorefname}{Ex\-am\-ple}
\theoremstyle{apex}

\makeatletter
\renewenvironment{proof}[1][\proofname]{\pagebreak[2]\par
  \pushQED{\qed}%
  \normalfont \topsep6\p@\@plus6\p@\relax
  \trivlist
  \item[\hskip\labelsep
        \bfseries
    #1]\mbox{}\\* % something is needed to be able to get a newline
}{%
  \popQED\endtrivlist\@endpefalse
}
\makeatother
\renewcommand{\qedsymbol}{\ensuremath{\square}}

\newspecialbox{definition}{Def\-i\-ni\-tion}{60}
% draw = yellow!95!black!60 = Hsb( 60,.59,.97)
% topc = white!95!yellow    = Hsb( 60,.05,1)
% botc = yellow!90!black!30 = Hsb( 60,.28,.97)

\newspecialbox{theorem}{The\-o\-rem}{120}
% draw = green!30!black!50  = Hsb(120,.23,.65)
% topc = white!95!green     = Hsb(120,.05,1)
% botc = green!60!black!20  = Hsb(120,.13,.92)

\newspecialbox{keyidea}{Key I\-dea}{0}
% draw = red!30!black!50    = Hsb(  0,.23,.65)
% topc = white!95!red       = Hsb(  0,.05,1)
% botc = red!60!black!20    = Hsb(  0,.13,.92)



\newtoggle{abridgeConics}
\toggletrue{abridgeConics}

\newcommand{\monthYear}{%
\ifcase \month \or January\or February\or March\or April\or May\or June\or July\or August\or September\or October\or November\or December\fi \space \number \year}
%modified from \today. we could do
%\usepackage[en-US]{datetime2}
%\DTMlangsetup{showdayofmonth=false}
% so that \today is just month and year
%but LaTeXML doesn't have datetime2, so we need this anyway

\usepackage{multirow}
%\pgfplotsset{width=\marginparwidth+1pt,compat=1.3}
\usepackage[font=small,justification=RaggedRight]{caption}

%\usepackage{wrapfig}

\usepackage{booktabs}

\setcounter{secnumdepth}{1}
\setcounter{tocdepth}{1}

\makeatletter
\let\ps@oldplain=\ps@plain % save the plain pagestyle
\makeatother

\usepackage{fancyhdr}

\renewcommand{\chaptermark}[1]{\markboth{\chaptername\ \thechapter\ \ \ \ {#1}}{}}
\renewcommand{\sectionmark}[1]{\markright{\thesection\ \ \ \  #1}}
\renewcommand{\headrulewidth}{0pt}
\renewcommand{\footrulewidth}{0pt}


\fancypagestyle{prose}{%
 \fancyhf{}
 \fancyhead[LE]{\nouppercase{\leftmark}}%
 \fancyhead[RO]{\nouppercase{\rightmark}}%
 \fancyfoot[LE]{\begin{minipage}{\textwidth}%
  \noindent\hspace{\marginparwidth}\hspace{\marginparsep}\hspace{-.4em}%
  \makebox[0pt][l]{\rule{\textwidth}{.4pt}}%
  \vskip.2\baselineskip%
  \noindent\hspace{\marginparwidth}\hspace{\marginparsep}\hspace{-.4em}%
  Notes:%
  \vskip 1.5in\textbf{\thepage}%
 \end{minipage}}

 \fancyfoot[RO]{\begin{minipage}{\textwidth+\marginparwidth+\marginparsep}%
  \rule{\textwidth-\marginparwidth-\marginparsep}{.4pt}
  \vskip.2\baselineskip
  Notes:
  \vskip 1.5in
  \hfill\textbf{\thepage}
 \end{minipage}}
 \fancyhfoffset[LE,RO]{\marginparsep+\marginparwidth}
}
\fancypagestyle{exercise}{%
	\fancyhf{}% 
	\fancyhfoffset[LE,RO]{32pt}%
	\fancyfoot[LE,RO]{\textbf{\thepage}}
}



\let\oldmainmatter\mainmatter
\renewcommand{\mainmatter}{%
 \oldmainmatter
 \fancypagestyle{plain}{% override the default for opening chapters
  \fancyhf{}
  \fancyfoot[RO]{\begin{minipage}{\textwidth+\marginparwidth+\marginparsep}%
   \rule{\textwidth-\marginparwidth-\marginparsep}{.4pt}
   \vskip.2\baselineskip
   Notes:
   \vskip 1.5in
   \hfill\textbf{\thepage}
  \end{minipage}}
  \fancyhfoffset[RO]{\marginparsep+\marginparwidth}
 }
 \pagestyle{prose}
}

\newtoggle{inappendix}
% todo Tim
% \appto\appendix{stuff}
\let\oldappendix\appendix
\makeatletter
\renewcommand{\appendix}{%
 \let\ps@plain=\ps@oldplain% restore the pagestyle
 \cleardoublepage
 \oldappendix
 \toggletrue{inappendix}
 \setcounter{secnumdepth}{-1}
 \pagenumbering{arabic}
 \renewcommand{\thepage}{A.\arabic{page}}
 \renewcommand{\thechapter}{\arabic{chapter}}
 \part*{\appendixname}
% \pagestyle{oldplain}
% \part*{Appendices\protect\thispagestyle{empty}}
% \addcontentsline{toc}{part}{\appendixname}
% \iflatexml\else
% \pdfbookmark[part]{Appendices}{appendixbookmark}
% \fi
}
\makeatother


% an enumerate like environment that can be mixed into tabular, array, etc.
\newcounter{anywhereenumi}
\newenvironment{anywhereenum}{%
 \setcounter{anywhereenumi}{0}%
 \renewcommand{\item}[1][]{%
  \ifx.##1.%
  \refstepcounter{anywhereenumi}%
  \makebox[1em][r]{\arabic{anywhereenumi}.}~~%
  \else%
  \makebox[1em][r]{##1.}~~%
  \fi%
 }%
}{}

\newcommand{\ds}{\displaystyle}

\newcommand{\primeskip}{\ifbool{mmode}{\mkern1.35mu}{\kern.075em}\relax}
%\newcommand{\primeskip}{\hskip.75pt}

\newcommand{\fp}{\ensuremath{f\,'}}
\newcommand{\fpp}{\ensuremath{f\,''}}

\newcommand{\Fp}{\ensuremath{F\primeskip'}}
\newcommand{\Fpp}{\ensuremath{F\primeskip''}}

\newcommand{\yp}{\ensuremath{y\primeskip'}}
\newcommand{\gp}{\ensuremath{g\primeskip'}}

\newcommand{\dd}{\operatorname{d}\!}

\newcommand*{\abs}[1]{\ensuremath{\left\lvert #1 \right\rvert}}
\newcommand*{\norm}[1]{\ensuremath{\left\lVert #1 \right\rVert}}
\newcommand*{\vnorm}[1]{\ensuremath{\norm{\vec #1}}}
\newcommand{\bracket}[1]{\left\langle #1\right\rangle}
\newcommand*{\proj}[2]{\ensuremath{\text{proj}_{\,\vec #2}{\,\vec #1}}}

\newcommand{\vecE}{\ensuremath{\vec E}}
\newcommand{\vecF}{\ensuremath{\vec F}}
\newcommand{\vecG}{\ensuremath{\vec G}}
\newcommand{\vecT}{\ensuremath{\vec T}}
\newcommand{\vece}{\ensuremath{\vec e}}
\newcommand{\vecf}{\ensuremath{\vec f}}
\newcommand{\vecg}{\ensuremath{\vec g}}
\newcommand{\veci}{\ensuremath{\vec\imath}}
\newcommand{\vecj}{\ensuremath{\vec\jmath}}
\newcommand{\veck}{\ensuremath{\vec k}}
\newcommand{\vecl}{\ensuremath{\vec l}}
\newcommand{\vecn}{\ensuremath{\vec n}}
\newcommand{\vecr}{\ensuremath{\vec r}}
\newcommand{\vecu}{\ensuremath{\vec u}}
\newcommand{\vecv}{\ensuremath{\vec v}}
\newcommand{\vecw}{\ensuremath{\vec w}}
\newcommand{\vecx}{\ensuremath{\vec x}}
\newcommand{\vecy}{\ensuremath{\vec y}}
\newcommand{\vrp}{\ensuremath{\vec r\hskip1.25pt '}}
\newcommand{\vsp}{\ensuremath{\vec s\primeskip '}}
\newcommand{\vrt}{\ensuremath{\vec r(t)}}
\newcommand{\vst}{\ensuremath{\vec s(t)}}
\newcommand{\vvt}{\ensuremath{\vec v(t)}}
\newcommand{\vat}{\ensuremath{\vec a(t)}}

\newcommand{\underlinespace}{\underline{\phantom{xxxxxx}}}

\newcommand{\zerooverzero}{\dfrac{\makebox[0pt]{\text{`` }0\text{ ''}}}0\ \ }


\DeclareMathOperator{\sech}{sech}
\DeclareMathOperator{\csch}{csch}
\DeclareMathOperator{\Div}{div}
\DeclareMathOperator{\grad}{grad}
\DeclareMathOperator{\curl}{curl}
\DeclareMathOperator{\divv}{div}

%\newcommand*{\sword}[1]{\textbf{#1}}

\newcommand{\LHequals}{\mathrel{\overset{\text{by LHR}}{=}}}

\newcommand{\surfaceS}{\ensuremath{\mathcal{S}}}


%\newspecialbox[notempty]{exvideo}{ignored}{240}
\AtBeginDocument{\makeStyles{exvideo}{240}}
\newcommand{\setexvideocolor}{%
 \definecolor{topexvideo}{Hsb}{240,.05,1}% %= hsl(#4,100,97.5)
 \definecolor{borderexvideo}{Hsb}{240,.3,1}% %= hsl(#4,90.5,68.4)
 \definecolor{bottomexvideo}{Hsb}{240,.15,1}% %= hsl(#4,81.9,83.4)
}
\newcommand{\setexvideobw}{%
 \definecolor{topexvideo}{Hsb}{0,0,1}% white
 \definecolor{bottomexvideo}{Hsb}{0,0,1}% white
 \definecolor{borderexvideo}{Hsb}{0,1,0}% black
}
\newcommand{\exvideo}[1]{%
 \tcbox[
   colframe=borderexvideo,
   beforeafter skip=\boxskipamount,
   interior style={top color=topexvideo, bottom color=bottomexvideo},
   sharp corners=all,
   notitle,
   width=\textwidth,
   enhanced,
   tcbox width=forced left
  ]{#1}%
}


% \jmtVideo{youtube code}{jmt url suffix}{actual title}
%\newcommand{\jmtVideo}[3]{\genVideo{#1}{http://patrickjmt.com/#2/}{#3}}

%\newcommand{\khanVideo}[3]{\genVideo[?utm_campaign=embed]{#1}{https://www.khanacademy.org/video/#2}{#3}}


% \mfigure[graphicsoptions]{offset}{caption}{label}{file}
\newcommand{\mfigure}[5][]{%
	\mnote[#2]{%
		\centering\myincludegraphics[#1]{#5}%
		\captionsetup{type=figure}\caption{#3}\label{#4}}%
}

% \mtable[offset=0]{caption}{label}{contents}
\newcommand{\mtable}[4][0ex]{%
	\mnote[#1]{\centering\small#4\captionsetup{type=figure}%
		\caption{#2}\label{#3}}%
}

%\ifbool{latexml}{
% \newcommand{\ignoreoptional}[1][]{}
% \newcommand{\marginnote}[1]{\marginpar{#1}\ignoreoptional}
%}{
% \usepackage[noadjust]{marginnote}
%}

% mnote is in apex_style.sty


%\newenvironment{lxfigure}{%
%	\iflatexml%
%		\begin{figure}[!h]%
%	\else%
%		\noindent\begin{minipage}[t]{\linewidth}\noindent%
%	\fi%
%	\captionsetup{type=figure}%
%}{%
%	\iflatexml\end{figure}\else\end{minipage}\fi%
%}

\newcommand{\tbox}[1]{\begin{tabular}{c}#1\end{tabular}} % a tall box
\newcommand*{\zbox}[1]{\makebox[0pt][c]{#1}} % a zero width box






\newtoggle{isEarlyTrans}
\togglefalse{isEarlyTrans}

\newcommand{\prereqIntro}{The material in this section provides a basic review of and practice problems for pre-calculus skills essential to your success in Calculus. You should take time to review this section and work the suggested problems (checking your answers against those in the back of the book). Since this content is a pre-requisite for Calculus, reviewing and mastering these skills are considered your responsibility. This means that minimal, and in some cases no, class time will be devoted to this section. When you identify areas that you need help with we strongly urge you to seek assistance outside of class from your instructor or other student tutoring service.\bigskip}

\ifbool{xetex}%
	{%
	\sffamily
%	\usepackage{fontspec}
%	\usepackage{unicode-math}
	\usepackage{mathspec}
	\setallmainfonts[Mapping=tex-text]{Calibri}
	\setmainfont[Mapping=tex-text]{Calibri}
	% setallmainfonts claims to setmainfont. but it doesn't?
%	\setmathsfont[Mapping=tex-text]{Calibri}
%	\setmathrm[Mapping=tex-text]{Calibri}
	\setsansfont[Mapping=tex-text]{Calibri}
	\setmathsfont(Greek){[cmmi10]}
	}
	{}

\ifbool{luatex}%
	{%
	\sffamily
	\usepackage{fontspec}
	\usepackage{unicode-math}
	%\usepackage{mathspec}
	%\setallmainfonts[Mapping=tex-text]{Calibri}
	\setmainfont{Calibri}
	%\setsansfont[Mapping=tex-text]{Calibri}
	\setmathfont[range=\mathup]{Calibri}
	\setmathfont[range=\mathit]{Calibri Italic}
	}
	{}

\ifbool{latexml}{
 \usepackage[american]{babel}
}{
 \usepackage{polyglossia}
 \setdefaultlanguage[variant=usmax]{english}
 \renewcommand*{\englishhyphenmins}{22}
 \AfterEndPreamble{
 \hyphenation{%
  an-ti-der-iv-a-tive
  an-ti-der-iv-a-tives
  app-rox-i-mate
  cen-tered
  chang-es
  con-struc-tions
  de-creas-es
  Der-iv-a-tive
  der-iv-a-tive
  dis-place-ment
  dis-tance
  e-qual-ly
  ex-am-ples
  Func-tions
 % Hô-pi-tal % doesn't hyphenate L'Hôpital's
  im-pli-cit
  in-dis-tin-guish-a-ble
  in-fall-i-ble
 % %L'Hô-pi-tal's % ' causes: ! Not a letter.
 % % see https://tex.stackexchange.com/a/165091/107497 for fix and pitfalls
  meth-od
  of-ten
  proc-ess
  re-fer-ring
  qua-dra-tic
  sa-li-ent
  se-quence
  sketch-ing
  smart-er
  sub-sti-tute
  The-o-rem
  Trig-o-no-me-tric
  trig-o-no-me-tric
  wheth-er
 }}
}

% lets try to reduce bad boxes
\usepackage{microtype}
\hfuzz=2pt
\vfuzz=1.5\baselineskip
% ignore overfull < this amount
%\newdimen\hfuzz % lock it in?
%\newdimen\vfuzz % lock it in?
%\hbadness=10000
\vbadness=9999
% ignore underfull > this amount
\parskip=0pt plus \baselineskip
\baselineskip=1\baselineskip plus .3\baselineskip


\usepackage[nottoc]{tocbibind}
%\let\oldprintindex\printindex
%\renewcommand{\printindex}{%
% \cleardoublepage
%% \chapter{\indexname} % \printindex has its own heading
% \phantomsection
%% \iflatexml\chapter*{\indexname}\fi
% \addcontentsline{toc}{chapter}{\indexname}
% \oldprintindex
%}


\newtoggle{bsc} % default false
\newcommand{\forwhom}{\iftoggle{bsc}{ for Bismarck State College}{}}


\usepackage[
	bookmarksnumbered,
	hidelinks,
	pdfstartview=FitH,
	linktoc=all,
	pdfdisplaydoctitle,
	bookmarksdepth=2,
]{hyperref}
\hypersetup{
	pdftitle={APEX Calculus LT},
	pdfauthor={UND Math Dept and Greg Hartman, VMI},
	unicode,
    pdflang=EN-US
}
\ifbool{latexml}{}{
 \usepackage{bookmark}
}


% hyperref changes these
% if they come before and have newcommand, latexml overwrites them
\AtBeginDocument{
 \renewcommand{\chapterautorefname}{Chap\-ter} % the default is lowercase
 \renewcommand{\sectionautorefname}{Sec\-tion} % the default is lowercase
 \renewcommand{\figureautorefname}{Fig\-ure}
 \renewcommand{\appendixname}{Ap\-pen\-di\-ces}
}
\newcommand{\exampleEnvautorefname}{Ex\-am\-ple}
\newcommand{\autoeqref}[1]{\hyperref[#1]{\equationautorefname~(\ref*{#1})}}
% autoref doesn't use parentheses

% \apex has to be *used* after hyperref
% lxNavbar has to come after latexml
\begin{lxNavbar}
\lxRef{lxApexTOC}{Table of Contents}\\
\lxContextTOC
\end{lxNavbar}

\lxIncludeCssFile{style.css}
\lxIncludeCssFile{LaTeXML-marginpar.css}
\lxIncludeCssFile{LaTeXML-navbar-left.css}
\lxIncludeJavascriptFile{%
https://ajax.googleapis.com/ajax/libs/jquery/1.12.2/jquery.min.js}
\lxIncludeJavascriptFile{script.js}
\lxIncludeJavascriptFile{LaTeXML-maybeMathJax.js}

% set the defaults, just in case
\printincolor
\usetwoDgraphics


\printallanswers
\printincolor
\usetwoDgraphics

\begin{document}

\setcounter{chapter}{10}
\apexchapter{Vectors}{chapter:vectors}

This chapter introduces a new mathematical object, the \sword{vector}. Defined in \autoref{sec:vector_intro}, we will see that vectors provide a powerful language for describing quantities that have magnitude and direction. A simple example of such a quantity is force: when applying a force, one is generally interested in how much force is applied (i.e., the magnitude of the force) and the direction in which the force is applied. Vectors will play an important role in many of the subsequent chapters in this text. 

This chapter begins with moving our mathematics out of the plane and into ``space.'' That is, we begin to think mathematically not only in two dimensions, but in three. With this foundation, we can explore vectors both in the plane and in space. 

\section{Introduction to Cartesian Coordinates in Space}\label{sec:space_coord}

Up to this point in this text we have considered mathematics in a 2-dimensional world. We have plotted graphs on the $x$-$y$ plane using rectangular and polar coordinates and found the area of regions in the plane. We have considered properties of \emph{solid} objects, such as volume and surface area, but only by first defining a curve in the plane and then rotating it out of the plane.

While there is wonderful mathematics to explore in ``2D,'' we live in a ``3D'' world and eventually we will want to apply mathematics involving this third dimension. In this section we introduce Cartesian coordinates in space and explore basic surfaces. This will lay a foundation for much of what we do in the remainder of the text.\\

Each point $P$ in space can be represented with an ordered triple, $P=(a,b,c)$, where $a$, $b$ and $c$ represent the relative position of $P$ along the $x$-, $y$- and $z$-axes, respectively. Each axis is perpendicular to the other two.

Visualizing points in space on paper can be problematic, as we are trying to represent a 3-dimensional concept on a 2-dimensional medium. We cannot draw three lines representing the three axes in which each line is perpendicular to the other two. Despite this issue, standard conventions exist for plotting shapes in space that we will discuss that are more than adequate.

\mtable{Illustrating the right hand rule.  Figure courtesy of \href{https://commons.wikimedia.org/wiki/File:Right_hand_rule_simple.png}{user:Schorschi2} / \href{http://commons.wikimedia.org/}{Wikimedia Commons} / Public Domain.}{fig:right_hand_rule}{\centering
\begin{tikzpicture}[baseline=-10ex]
 \draw[->](0,0)--(0,1)node[above]{$z$};
 \draw[->](0,0)--(-.5,-.3)node[below left]{$x$};
 \draw[->](0,0)--(.5,-.3)node[below right]{$y$};
\node at(2,.5){\rotatebox{41}{\includegraphics[width=.5\marginparwidth]{figures/raw/Right_hand_rule_simple.png}}};
\end{tikzpicture}
\vspace{-2\baselineskip}}

One convention is that the axes must conform to the \textbf{right hand rule}. This rule states that when the fingers of the right hand extend in the direction of the positive $x$-axis and curve toward the positive $y$-axis, then the extended thumb will point in the direction of the positive $z$-axis. (It may take some thought to verify this, but this system is inherently different from the one created by using the ``left hand rule.'')
Another way to view the rule is that when the index finger of the right hand extends in the direction of the positive $x$-axis, and the middle finger (bent ``inward'' so it is perpendicular to the palm) points along the positive $y$-axis, then the extended thumb will point in the direction of the positive $z$-axis.%
\index{right hand rule!of Cartesian coordinates}

As long as the coordinate axes are positioned so that they follow this rule, it does not matter how the axes are drawn on paper. There are two popular methods that we briefly discuss.
\mtable{Plotting the point $P=(2,1,3)$ in space.}{fig:cartcoord1}{%
\myincludeasythree{width=\marginparwidth,
3Droll=0,
3Dortho=0.0045,
3Dc2c=11 5 2.8,
3Dcoo=0 0 0,
3Droo=200}{width=\marginparwidth}{figures/figcartcoord1_3D}}

In \autoref{fig:cartcoord1} we see the point $P=(2,1,3)$ plotted on a set of axes. The basic convention here is that the $x$-$y$ plane is drawn in its standard way, with the $z$-axis down to the left. The perspective  is that the paper represents the $x$-$y$ plane and the positive $z$ axis is coming up, off the page. This method is preferred by many engineers. Because it can be hard to tell where a single point lies in relation to all the axes, dashed lines have been added to let one see how far along each axis the point lies.

One can also consider the $x$-$y$ plane as being a horizontal plane in, say, a room, where the positive $z$-axis is pointing up. When one steps back and looks at this room, one might draw the axes as shown in \autoref{fig:cartcoord2}. The same point $P$ is drawn, again with dashed lines. This point of view is preferred by most mathematicians, and is the convention adopted by this text.

Just as the $x$- and $y$-axes divide the plane into four quadrants, the $x$-, $y$-, and $z$-coordinate planes divide space into eight octants.  The octant in which $x$, $y$, and $z$ are positive is called the first octant.  We will not give special names for the other seven octants.

\mtable{Plotting the point $P=(2,1,3)$ in space with a perspective used in this text.}{fig:cartcoord2}{%
\myincludeasythree{width=\marginparwidth,
3Droll=0,
3Dortho=0.004,
3Dc2c=0.6666666865348816 0.6666666865348816 0.3333333730697632,
3Dcoo=16.7497615814209 8.84995174407959 40.36832046508789,
3Droo=129.79413580025474}{width=\marginparwidth}{figures/figcartcoord2_3D}}

\subsection{Measuring Distances}

It is of critical importance to know how to measure distances between points in space. The formula for doing so is based on measuring distance in the plane and the Pythagorean theorem, and is known (in both contexts) as the Euclidean measure of distance.

\begin{definition}[Distance In Space]\label{def:space_distance}
Let $P=(x_1,y_1,z_1)$ and $Q = (x_2,y_2,z_2)$ be points in space. The distance $D$ between $P$ and $Q$ is \index{distance!between points in space}
\[D = \sqrt{(x_2-x_1)^2+(y_2-y_1)^2+(z_2-z_1)^2}.\]
\end{definition}

We refer to the line segment that connects points $P$ and $Q$ in space as $\overline{PQ}$, and refer to the length of this segment as $\norm{\overline{PQ}}$. The above distance formula allows us to compute the length of this segment.\\

\begin{example}[Length of a line segment]\label{ex_space1}
Let $P = (1,4,-1)$ and let $Q = (2,1,1)$. Draw the line segment $\overline{PQ}$ and find its length.
\solution
The points $P$ and $Q$ are plotted in \autoref{fig:space1}; no special consideration needs to be made to draw the line segment connecting these two points; simply connect them with a straight line. One \emph{cannot} actually measure this line on the page and deduce anything meaningful; its true length must be measured analytically. Applying \autoref{def:space_distance}, we have
%
\mtable{Plotting points $P$ and $Q$ in \autoref{ex_space1}.}{fig:space1}{%
\myincludeasythree{width=\marginparwidth,
3Droll=0,
3Dortho=0.0044,
3Dc2c=0.6474442481994629 0.6759132146835327 0.35207563638687134,
3Dcoo=32.306640625 39.99113082885742 5.668694019317627,
3Droo=113.20770262248418}{width=\marginparwidth}{figures/figspace1_3D}}
%
\[
\norm{\overline{PQ}}= \sqrt{(2-1)^2+(1-4)^2+(1-(-1))^2} = \sqrt{14}%\approx 3.74
.
\]
\end{example}

\subsection{Spheres}

\index{sphere}Just as a circle is the set of all points in the \emph{plane} equidistant from a given point (its center), a sphere is the set of all points in \emph{space} that are equidistant from a given point. \autoref{def:space_distance} allows us to write an equation of the sphere.

We start with a point $C = (a,b,c)$ which is to be the center of a sphere with radius $r$. If a point $P=(x,y,z)$ lies on the sphere, then $P$ is $r$ units from $C$; that is, 
\[\norm{\overline{PC}}= \sqrt{(x-a)^2+(y-b)^2+(z-c)^2} = r.\]
Squaring both sides, we get the standard equation of a sphere in space with center at $C=(a,b,c)$ with radius $r$, as given in the following Key Idea.

\begin{keyidea}[Standard Equation of a Sphere in Space]\label{idea:sphere}
The standard equation of the sphere with radius $r$, centered at $C=(a,b,c)$, is
\[(x-a)^2+(y-b)^2+(z-c)^2=r^2.\]
\end{keyidea}

\youtubeVideo{fE_PWxyohXQ}{Example of Equation of a Sphere}

\begin{example}[Equation of a sphere]\label{ex_space2}
Find the center and radius of the sphere defined by $x^2+2x+y^2-4y+z^2-6z=2$.
\solution
To determine the center and radius, we must put the equation in standard form. This requires us to complete the square (three times).
\begin{align*}
x^2+2x+y^2-4y+z^2-6z&=2 \\
(x^2+2x+1) + (y^2-4y+4)+ (z^2-6z+9) - 14 &= 2\\
(x+1)^2 + (y-2)^2 + (z-3)^2 &= 16
\end{align*}
The sphere is centered at $(-1,2,3)$ and has a radius of 4.
\end{example}

The equation of a sphere is an example of an implicit function defining a surface in space. In the case of a sphere, the variables $x$, $y$ and $z$ are all used. We now consider situations where surfaces are defined where one or two of these variables are absent.

\subsection{Introduction to Planes in Space}

The coordinate axes naturally define three planes (shown in \autoref{fig:coordplanes}), the \textbf{coordinate planes}: the $x$-$y$ plane, the $y$-$z$ plane and the $x$-$z$ plane. The $x$-$y$ plane is characterized as the set of all points in space where the $z$-value is 0. %(Likewise, the $x$-$z$ plane is all points where the $y$-value is 0.) 
\index{planes!coordinate plane}\index{planes!introduction}
This, in fact, gives us an equation that describes this plane: $z=0$. Likewise, the $x$-$z$ plane is all points where the $y$-value is 0, characterized by $y=0$.\\

\noindent\begin{minipage}[t]{\linewidth}\noindent%
\captionsetup{type=figure}%
\flushinner{%
%\noindent\begin{minipage}{\textwidth}
\begin{tabular}{ccc}
\myincludeasythree{width=.3\textwidth,
3Droll=0,
3Dortho=0.004,
3Dc2c=4 4 2,
3Dcoo=0 0 0,
3Droo=150}{width=.3\textwidth}{figures/figspacexy_3D}
&
\myincludeasythree{width=.3\textwidth,
3Droll=0,
3Dortho=0.004,
3Dc2c=4 4 2,
3Dcoo=0 0 0,
3Droo=150}{width=.3\textwidth}{figures/figspaceyz_3D}
&
\myincludeasythree{width=.3\textwidth,
3Droll=0,
3Dortho=0.004,
3Dc2c=4 4 2,
3Dcoo=0 0 0,
3Droo=150}{width=.3\textwidth}{figures/figspacexz_3D}
\\
the $x$-$y$ plane & the $y$-$z$ plane & the $x$-$z$ plane
\end{tabular}}
\caption{The coordinate planes.}\label{fig:coordplanes}
\end{minipage}

% todo Tim this would look better if the bl-ue plane would not stop exactly at the z-axis and if the tip of the x-axis would stick out
\mtable{The plane $x=2$.}{fig:space2}{%
\myincludeasythree{width=\marginparwidth,
3Droll=0,
3Dortho=0.0044,
3Dc2c=4 4 2,
3Dcoo=0 0 0,
3Droo=150}{width=\marginparwidth}{figures/figspace2_3D}}

% todo Mention the general drawing trick: if a plane is parallel to a coordinate axis, draw grid lines parallel to that axis onto the plane.  Also draw an example of an oblique plane.
\begin{example}[A plane in three dimensions]\label{ex_space_x_is_two}
The equation $x=2$ describes all points in space where the $x$-value is 2. This is a plane, parallel to the $y$-$z$ coordinate plane, shown in \autoref{fig:space2}.
\end{example}

\begin{example}[Regions defined by planes]\label{ex_space3}
Sketch the region defined by the inequalities $-1\leq y\leq 2$.
\solution
%
\mtable{Sketching the boundaries of a region in \autoref{ex_space3}.}{fig:space3}{%
\myincludeasythree{width=\marginparwidth,
3Droll=0,
3Dortho=0.0044,
3Dc2c=4 2.5 2,
3Dcoo=0 0 0,
3Droo=150}{width=\marginparwidth}{figures/figspace3_3D}}
%
The region is all points between the planes $y=-1$ and $y=2$. These planes are sketched in \autoref{fig:space3}, which are parallel to the $x$-$z$ plane. Thus the region extends infinitely in the $x$ and $z$ directions, and is bounded by planes in the $y$ direction.
\end{example}

\subsection{Cylinders}

The equation $x=1$ obviously lacks the $y$ and $z$ variables, meaning it defines points where the $y$ and $z$ coordinates can take on any value. Now consider the equation $x^2+y^2=1$ \emph{in space.} In \emph{the plane}, this equation describes a circle of radius 1, centered at the origin. In space, the $z$ coordinate is not specified, meaning it can take on any value. In \autoref{fig:spacecylinder1} (a), we show part of the graph of the equation $x^2+y^2=1$ by sketching 3 circles: the bottom one has a constant $z$-value of $-1.5$, the middle one has a $z$-value of 0 and the top circle has a $z$-value of 1. By plotting \emph{all} possible $z$-values, we get the  surface shown in \autoref{fig:spacecylinder1} (b). This surface looks like a ``tube,'' or a ``cylinder'', which leads to our next definition.
%; mathematicians call this surface a \textbf{cylinder} for an entirely different reason.

\mtable{Sketching $x^2+y^2=1$.}{fig:spacecylinder1}{%
\myincludeasythree{width=\marginparwidth,
3Droll=0,
3Dortho=0.004,
3Dc2c=4 4 2,
3Dcoo=0 0 0,
3Droo=150}{width=\marginparwidth}{figures/figspacecylinder1_3D}
\\[5pt](a)\\[5pt]
\myincludeasythree{width=\marginparwidth,
3Droll=0,
3Dortho=0.004,
3Dc2c=4 4 2,
3Dcoo=0 0 0,
3Droo=150}{width=\marginparwidth}{figures/figspacecylinder1b_3D}
\\[5pt](b)}

\begin{definition}[Cylinder]\label{def:cylinder}
Let $C$ be a curve in a plane and let $L$ be a line not parallel to $C$. A \textbf{cylinder} is the set of all lines parallel to $L$ that pass through $C$. The curve $C$ is the \textbf{directrix} of the cylinder, and the lines are the \textbf{rulings}.\index{cylinder}\index{directrix}
\end{definition}

In this text, we consider curves $C$ that lie in planes parallel to one of the coordinate planes, and lines $L$ that are perpendicular to these planes, forming \textbf{right cylinders}. Thus the directrix can be defined using equations involving 2 variables, and the rulings will be parallel to the axis of the 3$^\text{rd}$ variable.

In the example preceding the definition, the curve $x^2+y^2=1$ in the $x$-$y$ plane is the directrix and the rulings are lines parallel to the $z$-axis. (Any circle shown in \autoref{fig:spacecylinder1} can be considered a directrix; we simply choose the one where $z=0$.) Sample rulings can also be viewed in part (b) of the figure. More examples will help us understand this definition.

\begin{example}[Graphing cylinders]\label{ex_space4}
Graph the following cylinders.
\[\text{1.}\quad z=y^2\qquad \text{2.}\quad x=\sin z\]
\solution
\begin{enumerate}
	\item We can view the equation $z=y^2$ as a parabola in the $y$-$z$ plane, as illustrated in \autoref{fig:space4a} (a). As $x$ does not appear in the equation, the rulings are lines through this parabola parallel to the $x$-axis, shown in (b). These rulings give a general idea as to what the surface looks like, drawn in (c).
	
{\centering
\captionsetup{type=figure}%
\begin{tabular}{ccc}
\myincludeasythree{width=.3\linewidth,
3Droll=0,
3Dortho=0.004,
3Dc2c=4 4 2,
3Dcoo=0 0 75,
3Droo=150}{width=.3\linewidth}{figures/figspace4a_3D}
 &
\myincludeasythree{width=.3\linewidth,
3Droll=0,
3Dortho=0.004,
3Dc2c=4 4 2,
3Dcoo=0 0 75,
3Droo=150}{width=.3\linewidth}{figures/figspace4b_3D}
 &
\myincludeasythree{width=.3\linewidth,
3Droll=0,
3Dortho=0.004,
3Dc2c=4 4 2,
3Dcoo=0 0 75,
3Droo=150}{width=.3\linewidth}{figures/figspace4c_3D}
\\(a) & (b) & (c)
\end{tabular}
\caption{Sketching the cylinder defined by $z=y^2$.}\label{fig:space4a}
}% end centering
	
	\item		We can view the equation $x=\sin z$ as a sine curve that exists in the $x$-$z$ plane, as shown in \autoref{fig:space4b} (a). The rules are parallel to the $y$ axis as the variable $y$ does not appear in the equation $x=\sin z$; some of these are shown in part (b). The surface is shown in part (c) of the figure. 

{\centering
\captionsetup{type=figure}%
\begin{tabular}{ccc}
\myincludeasythree{width=.3\linewidth,
3Droll=0,
3Dortho=0.0045,
3Dc2c=4 4 3,
3Dcoo=0 0 0,
3Droo=150}{width=.3\linewidth}{figures/figspace4d_3D}
 &
\myincludeasythree{width=.3\linewidth,
3Droll=0,
3Dortho=0.0045,
3Dc2c=4 4 3,
3Dcoo=0 0 0,
3Droo=150}{width=.3\linewidth}{figures/figspace4e_3D} 
&
\myincludeasythree{width=.3\linewidth,
3Droll=0,
3Dortho=0.0045,
3Dc2c=4 4 3,
3Dcoo=0 0 0,
3Droo=150}{width=.3\linewidth}{figures/figspace4f_3D}
\\(a) & (b) & (c)
\end{tabular}
\caption{Sketching the cylinder defined by $x=\sin z$.}\label{fig:space4b}
}% end centering
\end{enumerate}
\end{example}

\subsection{Surfaces of Revolution}

\mtable{Introducing surfaces of revolution.}{fig:surf_rev_intro}{%
\myincludeasythree{width=\marginparwidth,
3Droll=0,
3Dortho=0.0045,
3Dc2c=.395 .79 .356,
3Dcoo=75 0 0,
3Droo=130}{width=\marginparwidth}{figures/figsurf_rev_intro_3D}
\\(a)\\
\myincludeasythree{width=\marginparwidth,
3Droll=0,
3Dortho=0.0045,
3Dc2c=.395 .79 .356,
3Dcoo=75 0 0,
3Droo=130}{width=\marginparwidth}{figures/figsurf_rev_introb_3D}
\\(b)}

One of the applications of integration we learned previously was to find the volume of solids of revolution --- solids formed by revolving a curve about a horizontal or vertical axis. We now consider how to find the equation of the surface of such a solid.

Consider the surface formed by revolving the curve $y=\sqrt{x}$ in the $x$-$y$ plane about the $x$-axis. Cross-sections of this surface parallel to the $y$-$z$ plane are circles, as shown in \autoref{fig:surf_rev_intro}(a). Each circle has equation of the form $y^2+z^2=r^2$ for some radius $r$. The radius is a function of $x$; in fact, it is $r(x) = \sqrt{x}$. Thus the equation of the surface shown in \autoref{fig:surf_rev_intro}(b) is $y^2+z^2=(\sqrt{x})^2.$

We generalize the above principles to give the equations of surfaces formed by revolving curves about the coordinate axes.

\begin{keyidea}[Surfaces of Revolution, Part 1]\label{idea:surf_of_revol}
Let $r$ be a radius function. \index{surface of revolution}
\begin{enumerate}
	\item The equation of the surface formed by revolving $y=r(x)$ or $z=r(x)$ about the $x$-axis is $y^2+z^2=r(x)^2$.
	\item The equation of the surface formed by revolving $x=r(y)$ or $z=r(y)$ about the $y$-axis is $x^2+z^2=r(y)^2$.
	\item The equation of the surface formed by revolving $x=r(z)$ or $y=r(z)$ about the $z$-axis is $x^2+y^2=r(z)^2$.
\end{enumerate}
\end{keyidea}

\begin{example}[Finding equation of a surface of revolution]\label{ex_surfrev1}
Let $y=\sin z$ on $[0,\pi]$. Find the equation of the surface of revolution formed by revolving $y=\sin z$ about the $z$-axis.
\solution
Using \autoref{idea:surf_of_revol}, we find the surface has equation $x^2+y^2=\sin^2z$. The curve is sketched in \autoref{fig:surfrev1}(a) and the surface is drawn in \autoref{fig:surfrev1}(b).

Note how the surface (and hence the resulting equation) is the same if we began with the curve $x=\sin z$, which is also drawn in \autoref{fig:surfrev1}(a).
\end{example}

\mtable[2\baselineskip]{Revolving $y=\sin z$ about the $z$-axis in \autoref{ex_surfrev1}.}{fig:surfrev1}{%
\myincludeasythree{width=.8\marginparwidth,
3Droll=0,
3Dortho=0.0046,
3Dc2c=.66 .67 .32,
3Dcoo=0 0 75,
3Droo=130}{width=.8\marginparwidth}{figures/figsurfrev1a_3D}
\\(a)\\
\myincludeasythree{width=.8\marginparwidth,
3Droll=0,
3Dortho=0.0046,
3Dc2c=.66 .67 .32,
3Dcoo=0 0 75,
3Droo=130}{width=.8\marginparwidth}{figures/figsurfrev1b_3D}
\\(b)}

This particular method of creating surfaces of revolution is limited. For instance, in \autoref{ex_shell4} of \autoref{sec:shell_method} we found the volume  of the solid formed by revolving $y=\sin x$ about the $y$-axis. Our current method of forming surfaces can only rotate $y=\sin x$ about the $x$-axis. Trying to rewrite $y=\sin x$ as a function of $y$ is not trivial, as simply writing $x=\sin^{-1}y$ only gives part of the region we desire.

What we desire is a way of writing the surface of revolution formed by rotating $y=f(x)$ about the $y$-axis. We start by first recognizing this surface is the same as revolving $z=f(x)$ about the $z$-axis, although it has a different orientation. This will give us a more natural way of viewing the surface. 

A value of $x$ is a measurement of distance from the $z$-axis. At the distance $r$, we plot a $z$-height of $f(r)$. When rotating $f(x)$ about the $z$-axis, we want all points a distance of $r$ from the $z$-axis in the $x$-$y$ plane to have a $z$-height of $f(r)$. All such points satisfy the equation $r^2=x^2+y^2$; hence $r=\sqrt{x^2+y^2}$. Replacing $r$ with $\sqrt{x^2+y^2}$ in $f(r)$ gives $z=f(\sqrt{x^2+y^2})$. This is the equation of the surface.

\begin{keyidea}[Surfaces of Revolution, Part 2]\label{idea:surf_of_revol2}
Let $z=f(x)$, $x\geq 0$, be a curve in the $x$-$z$ plane. The surface formed by revolving this curve about the $z$-axis has equation $z=f\bigl(\sqrt{x^2+y^2}\bigr)$.\index{surface of revolution}
\end{keyidea}

\mtable[\baselineskip]{Revolving $z=\sin x$ about the $z$-axis in \autoref{ex_surfrev2}.}{fig:surfrev2}{%
\myincludeasythree{width=.8\marginparwidth,
3Droll=0,
3Dortho=0.0046,
3Dc2c=.66 .67 .32,
3Dcoo=0 0 10,
3Droo=130}{width=.8\marginparwidth}{figures/figsurfrev2a_3D}
\\(a)\\[5pt]
\myincludeasythree{width=.8\marginparwidth,
3Droll=0,
3Dortho=0.0046,
3Dc2c=.66 .67 .32,
3Dcoo=0 0 25,
3Droo=130}{width=.8\marginparwidth}{figures/figsurfrev2b_3D}
\\(b)}
%
\begin{example}[Finding equation of surface of revolution]\label{ex_surfrev2}
Find the equation of the surface found by revolving $z=\sin x$ about the $z$-axis.
\solution
Using \autoref{idea:surf_of_revol2}, the surface has equation $z=\sin\bigl(\sqrt{x^2+y^2}\bigr)$. The curve and surface are graphed in \autoref{fig:surfrev2}.
\end{example}

\subsection{Quadric Surfaces}

Spheres, planes and cylinders are important surfaces to understand. We now consider one last type of surface, a \textbf{quadric surface}. The definition may look intimidating, but we will show how to analyze these surfaces in an illuminating way.

\begin{definition}[Quadric Surface]\label{def:quadric}
A \textbf{quadric surface} is the graph of the general second-degree equation in three variables:
\index{quadric surface!definition}
\[Ax^2+By^2+Cz^2+Dxy+Exz+Fyz+Gx+Hy+Iz+J=0.\]
\end{definition}

When the coefficients $D$, $E$ or $F$ are not zero, the basic shapes of the quadric surfaces are rotated in space. We will focus on quadric surfaces where these coefficients are 0; we will not consider rotations. There are six basic quadric surfaces: the elliptic paraboloid, elliptic cone, ellipsoid, hyperboloid of one sheet, hyperboloid of two sheets, and the hyperbolic paraboloid.

\mtable[5\baselineskip]{The elliptic paraboloid $z=x^2/4+y^2$.}{fig:quadric1}{%
\myincludeasythree{width=.8\marginparwidth,
3Droll=0,
3Dortho=0.00617163721472025,
3Dc2c=0.6847050189971924 0.6847050189971924 0.24971593916416168,
3Dcoo=-6.000957012176514 2.2318615913391113 85.33463287353516,
3Droo=150}{width=.8\marginparwidth}{figures/figquadric_parb_3D}}

We study each shape by considering \textbf{traces}, \index{trace}\index{quadric surface!trace}%
that is, intersections of each surface with a plane parallel to a coordinate plane. For instance, consider the elliptic paraboloid $z= x^2/4+y^2$, shown in \autoref{fig:quadric1}. If we intersect this shape with the plane $z=d$\ \  (i.e., replace $z$ with $d$), we have the equation:
\begin{align*}
d &= \frac{x^2}4+y^2.
\intertext{Divide both sides by $d$:}
1 &= \frac{x^2}{4d} + \frac{y^2}{d}.
\end{align*}
This describes an ellipse --- so cross sections parallel to the $x$-$y$ coordinate plane are ellipses. This ellipse is drawn in the figure.

Now consider cross sections parallel to the $x$-$z$ plane. For instance, letting $y=0$ gives the equation $z=x^2/4$, clearly a parabola. Intersecting with the plane $x=0$ gives a cross section defined by $z=y^2$, another parabola. These parabolas are also sketched in the figure. 

Thus we see where the elliptic paraboloid gets its name: some cross sections are ellipses, and others are parabolas.

Such an analysis can be made with each of the quadric surfaces. We give a sample equation of each, provide a sketch with representative traces, and describe these traces.

%\clearpage
\exercisegeometry
\pagestyle{exercise}

\index{quadric surface!gallery|(}
\noindent
\begin{minipage}{1.1\linewidth}
 \captionsetup{type=figure}%
 \noindent%
 \begin{minipage}[c]{.35\linewidth}
  \mbox{}\\
  \myincludeasythree{width=\marginparwidth,
3Droll=0,
3Dortho=0.005,
3Dc2c=.85 .85 .31,
3Dcoo=0 0 75,
3Droo=150}{width=\marginparwidth}{figures/figquadric_par_3D}
 \end{minipage}%
 \begin{minipage}[c]{.2\linewidth}
  \mbox{}\\
  \begin{tabular}{cc}
   \textbf{Plane}  & \textbf{Trace} \\\midrule
   $x=d$ & Parabola \\
   $y=d$ & Parabola \\
   $z=d$ & Ellipse
  \end{tabular}
 \end{minipage}%
 \begin{minipage}[c]{.45\linewidth}
  \mbox{}\\
  \myincludeasythree{width=\marginparwidth,
3Droll=0,
3Dortho=0.005,
3Dc2c=.85 .85 .31,
3Dcoo=0 0 75,
3Droo=150}{width=\marginparwidth}{figures/figquadric_parb_3D}
 \end{minipage}
 \caption{\quad\textbf{Elliptic Paraboloid}\qquad$z=\dfrac{x^2}{a^2}+\dfrac{y^2}{b^2}$}
 \label{fig_elliptic_paraboloid}
 \bigskip
 \begin{minipage}{.75\linewidth}
  One variable in the equation of the elliptic paraboloid will be raised to the first power; above, this is the $z$ variable. The paraboloid will ``open'' in the direction of this variable's axis. Thus $x= y^2/a^2+z^2/b^2$ is an elliptic paraboloid that opens along the $x$-axis.\bigskip

  Multiplying the right hand side by $(-1)$ defines an elliptic paraboloid that ``opens'' in the opposite direction.\index{quadric surface!elliptic paraboloid}
 \end{minipage}
\end{minipage}

\vfill
\noindent\hrulefill
\vfill

\noindent
\begin{minipage}{1.1\linewidth}
 \captionsetup{type=figure}%
 \noindent%
 \begin{minipage}[c]{.25\linewidth}
  \mbox{}\\
  \myincludeasythree{width=\marginparwidth,
3Droll=0,
3Dortho=0.0067,
3Dc2c=.68 .68 .25,
3Dcoo=0 0 0,
3Droo=150}{width=\marginparwidth}{figures/figquadric_cone_3D}
 \end{minipage}%
 \begin{minipage}[c]{.2\linewidth}
  \begin{tabular}{cc}
   \textbf{Plane}  & \textbf{Trace} \\\midrule
   $x=0$ & Crossed Lines \\
   $y=0$ & Crossed Lines\\\\
   $x=d$ & Hyperbola\\
   $y=d$ & Hyperbola\\
   $z=d$ & Ellipse
  \end{tabular}
 \end{minipage}%
 \begin{minipage}[c]{.55\linewidth}
  \mbox{}\\
  \myincludeasythree{width=.8\marginparwidth,
3Droll=0,
3Dortho=0.0067,
3Dc2c=.68 .68 .25,
3Dcoo=0 0 0,
3Droo=150}{width=.8\marginparwidth}{figures/figquadric_coneb_3D}
\ %
  \myincludeasythree{width=.8\marginparwidth,
3Droll=0,
3Dortho=0.0067,
3Dc2c=.68 .68 .25,
3Dcoo=0 0 0,
3Droo=150}{width=.8\marginparwidth}{figures/figquadric_conec_3D}
 \end{minipage}
 \caption{\quad\textbf{Elliptic Cone}\qquad$z^2=\dfrac{x^2}{a^2}+\dfrac{y^2}{b^2}$}
 \label{fig_elliptic_cone}
 \bigskip
 \begin{minipage}{.75\linewidth}
  One can rewrite the equation as $z^2-x^2/a^2-y^2/{b^2} = 0$. The one variable with a positive coefficient corresponds to the axis that the cones ``open'' along. \index{quadric surface!elliptic cone}
 \end{minipage}
\end{minipage}

\clearpage

\noindent
\begin{minipage}{1.1\linewidth}
 \captionsetup{type=figure}%
 \noindent%
 \begin{minipage}[c]{.4\linewidth}
  \mbox{}\\
  \myincludeasythree{width=\marginparwidth,
3Droll=0,
3Dortho=0.005,
3Dc2c=.4 .87 .3,
3Dcoo=0 0 0,
3Droo=150}{width=\marginparwidth}{figures/figquadric_hyp_par_3D}
 \end{minipage}%
 \begin{minipage}[c]{.4\linewidth}
  \mbox{}\\
  \begin{tabular}{cc}
   \textbf{Plane}  & \textbf{Trace} \\\midrule
   $x=d$ & Parabola\\
   $y=d$ & Parabola\\
   $z=d$ & Hyperbola
  \end{tabular}
 \end{minipage}
 \bigskip\\
 \begin{minipage}[c]{.4\linewidth}
  \mbox{}\\
  \myincludeasythree{width=\marginparwidth,
3Droll=0,
3Dortho=0.005,
3Dc2c=.4 .87 .3,
3Dcoo=0 0 0,
3Droo=150}{width=\marginparwidth}{figures/figquadric_hyp_parb_3D}
 \end{minipage}%
 \begin{minipage}[c]{.4\linewidth}
  \mbox{}\\
  \myincludeasythree{width=\marginparwidth,
3Droll=0,
3Dortho=0.005,
3Dc2c=.4 .87 .3,
3Dcoo=0 0 0,
3Droo=150}{width=\marginparwidth}{figures/figquadric_hyp_parc_3D}
 \end{minipage}
 \caption{\quad\textbf{Hyperbolic Paraboloid}\qquad$z=\dfrac{x^2}{a^2}-\dfrac{y^2}{b^2}$}
 \label{fig_hyperbolic_paraboloid}
 \bigskip
  The parabolic traces will open along the axis of the one variable that is raised to the first power.\index{quadric surface!hyperbolic paraboloid}
\end{minipage}

\vfill
\noindent\hrulefill
\vfill

\noindent
\begin{minipage}{1.1\linewidth}
 \captionsetup{type=figure}%
 \noindent%
 \begin{minipage}[c]{.3\linewidth}
  \mbox{}\\
  \myincludeasythree{width=\marginparwidth,
3Droll=0,
3Dortho=0.005,
3Dc2c=.85 .85 .31,
3Dcoo=0 0 0,
3Droo=150}{width=\marginparwidth}{figures/figquadric_ellipsoid_3D}
 \end{minipage}%
 \begin{minipage}[c]{.2\linewidth}
  \begin{tabular}{cc}
   \textbf{Plane}  & \textbf{Trace} \\\midrule
   $x=d$ & Ellipse\\
   $y=d$ & Ellipse\\
   $z=d$ & Ellipse
  \end{tabular}
 \end{minipage}%
 \begin{minipage}[c]{.5\linewidth}
  \mbox{}\\
  \myincludeasythree{width=\marginparwidth,
3Droll=0,
3Dortho=0.005,
3Dc2c=.85 .85 .31,
3Dcoo=0 0 0,
3Droo=150}{width=\marginparwidth}{figures/figquadric_ellipsoidb_3D}
 \end{minipage}
 \caption{\quad\textbf{Ellipsoid}\qquad$\dfrac{x^2}{a^2}+\dfrac{y^2}{b^2}+\dfrac{z^2}{c^2}=1$}
 \bigskip
  If $a=b=c\neq0$, the ellipsoid is a sphere with radius $a$; compare to \autoref{idea:sphere}.\index{quadric surface!ellipsoid}\index{quadric surface!sphere}
\end{minipage}

\clearpage

\noindent
\begin{minipage}{1.1\linewidth}
 \captionsetup{type=figure}%
 \noindent%
 \begin{minipage}[c]{.35\linewidth}
  \mbox{}\\
  \myincludeasythree{width=\marginparwidth,
3Droll=0,
3Dortho=0.005,
3Dc2c=.85 .85 .31,
3Dcoo=0 0 0,
3Droo=150}{width=\marginparwidth}{figures/figquadric_hyp_one_sheet_3D}
 \end{minipage}%
 \begin{minipage}[c]{.23\linewidth}
  \mbox{}\\
  \begin{tabular}{cc}
   \textbf{Plane}  & \textbf{Trace} \\\midrule
   $x=d$ & Hyperbola\\
   $y=d$ & Hyperbola\\
   $z=d$ & Ellipse
  \end{tabular}
 \end{minipage}%
 \begin{minipage}[c]{.4\linewidth}
  \mbox{}\\
  \myincludeasythree{width=\marginparwidth,
3Droll=0,
3Dortho=0.005,
3Dc2c=.85 .85 .31,
3Dcoo=0 0 0,
3Droo=150}{width=\marginparwidth}{figures/figquadric_hyp_one_sheetb_3D}
 \end{minipage}
 \caption{\quad\textbf{Hyperboloid of One Sheet}\qquad$\dfrac{x^2}{a^2}+\dfrac{y^2}{b^2}-\dfrac{z^2}{c^2}=1$}
 \label{fig_hyperboloid_one_sheet}
 \bigskip
  The one variable with a negative coefficient corresponds to the axis along which the hyperboloid ``opens''.\index{quadric surface!hyperboloid of one sheet}
\end{minipage}

\vfill
\noindent\hrulefill
\vfill

\noindent
\begin{minipage}{1.1\linewidth}
 \captionsetup{type=figure}%
 \noindent%
 \begin{minipage}[c]{.3\linewidth}
  \mbox{}\\
  \myincludeasythree{width=\marginparwidth,
3Droll=0,
3Dortho=0.005,
3Dc2c=.85 .85 .31,
3Dcoo=0 0 -15,
3Droo=150}{width=\marginparwidth}{figures/figquadric_hyp_two_sheet_3D}
 \end{minipage}%
 \begin{minipage}[c]{.22\linewidth}
  \mbox{}\\
  \begin{tabular}{cc}
   \textbf{Plane} & \textbf{Trace} \\\midrule
   $x=d$ & Hyperbola\\
   $y=d$ & Hyperbola\\
   $z=d$ & Ellipse
  \end{tabular}
 \end{minipage}%
 \begin{minipage}[c]{.45\linewidth}
  \mbox{}\\
  \myincludeasythree{width=\marginparwidth,
3Droll=0,
3Dortho=0.005,
3Dc2c=.85 .85 .31,
3Dcoo=0 0 -15,
3Droo=150}{width=\marginparwidth}{figures/figquadric_hyp_two_sheetb_3D}
 \end{minipage}
 \caption{\quad\textbf{Hyperboloid of Two Sheets}\qquad$\dfrac{z^2}{c^2}-\dfrac{x^2}{a^2}-\dfrac{y^2}{b^2}=1$}
 \label{fig_hyperboloid_two_sheets}
 \bigskip
 \begin{minipage}{.75\linewidth}
  The one variable with a positive coefficient corresponds to the axis along which the hyperboloid ``opens''. In the case illustrated, when $\abs d<\abs c$, there is no trace in the plane $z=d$.\index{quadric surface!hyperboloid of two sheets}
 \end{minipage}
\end{minipage}

\index{quadric surface!gallery|)}

\restoregeometry
%\clearpage
\pagestyle{prose}

\youtubeVideo{x6c2DdOrkQI}{Introduction to Quadric Surfaces}

\mtable[-1in]{Sketching an elliptic paraboloid.}{fig:space5a}{%
	\myincludeasythree{width=.8\marginparwidth,
3Droll=0,
3Dortho=0.0045,
3Dc2c=.7 .66 .18,
3Dcoo=0 30 0,
3Droo=150}{width=.8\marginparwidth}{figures/figspace5ab_3D}
\\(a)\\
	\myincludeasythree{width=.8\marginparwidth,
3Droll=0,
3Dortho=0.0045,
3Dc2c=.7 .66 .18,
3Dcoo=0 30 0,
3Droo=150}{width=.8\marginparwidth}{figures/figspace5a_3D}
\\(b)}

\begin{example}[Sketching quadric surfaces]\label{ex_space5}
Sketch the quadric surface defined by the given equation.
\[
 \text{1.}\quad y=\frac{x^2}{4}+\frac{z^2}{16}\qquad
 \text{2.}\quad x^2+\frac{y^2}{9}+\frac{z^2}{4}=1\qquad
 \text{3.}\quad z=y^2-x^2.
\]
\solution
\begin{enumerate}
	\item $\ds y=\frac{x^2}{4}+\frac{z^2}{16}$:\\
	We first identify the quadric by pattern-matching with the equations given previously. Only two surfaces have equations where one variable is raised to the first power, the elliptic paraboloid and the hyperbolic paraboloid. In the latter case, the other variables have different signs, so we conclude that this describes a hyperbolic paraboloid. As the variable with the first power is $y$, we note the paraboloid opens along the $y$-axis. 
	
	To make a decent sketch by hand, we need only draw a few traces. In this case, the traces $x=0$ and $z=0$ form parabolas that outline the shape.
	
	$x=0$:	The trace is the parabola $y=z^2/16$
	
	$z=0$: 	The trace is the parabola $y=x^2/4$.
	
	Graphing each trace in the respective plane creates a sketch as shown in \autoref{fig:space5a}(a). This is enough to give an idea of what the paraboloid looks like. The surface is filled in in (b).
	
	\item		$\ds x^2+\frac{y^2}{9}+\frac{z^2}{4}=1:$\\
	%
	\mtable[-.7in]{Sketching an ellipsoid.}{fig:space5b}{%
	\myincludeasythree{width=.8\marginparwidth,
3Droll=0,
3Dortho=0.0045,
3Dc2c=.7 .66 .18,
3Dcoo=0 0 0,
3Droo=150}{width=.8\marginparwidth}{figures/figspace5b_3D}
\\(a)\\
	\myincludeasythree{width=.8\marginparwidth,
3Droll=0,
3Dortho=0.0045,
3Dc2c=.7 .66 .18,
3Dcoo=0 0 0,
3Droo=150}{width=.8\marginparwidth}{figures/figspace5bb_3D}
\\(b)}
	%
	This is an ellipsoid. We can get a good idea of its shape by drawing the traces in the coordinate planes.
	
	$x=0$: 	The trace is the ellipse $\ds\frac{y^2}{9}+\frac{z^2}{4}=1$. The major axis is along the $y$-axis with length 6 (as $b=3$, the length of the axis is 6); the minor axis is along the $z$-axis with length 4.
	
	$y=0$:	The trace is the ellipse $\ds x^2+\frac{z^2}{4}=1.$ The major axis is along the $z$-axis, and the minor axis has length 2 along the $x$-axis.
	
	$z=0$:	The trace is the ellipse $\ds x^2+\frac{y^2}{9}=1,$ with major axis along the $y$-axis. 
	
	Graphing each trace in the respective plane creates a sketch as shown in \autoref{fig:space5b}(a). Filling in the surface gives \autoref{fig:space5b}(b).
	
	\item		$\ds z=y^2-x^2$:\\
	This defines a hyperbolic paraboloid, very similar to the one shown in the gallery of quadric sections. Consider the traces in the $y-z$ and $x-z$ planes:

	\mtable[-7\baselineskip]{Sketching a hyperbolic paraboloid.}{fig:space5c}{%
	\myincludeasythree{width=\marginparwidth,
3Droll=0,
3Dortho=0.0045,
3Dc2c=.7 .55 .43,
3Dcoo=0 0 0,
3Droo=150}{width=\marginparwidth}{figures/figspace5c_3D}
\\(a)\\
	\myincludeasythree{width=\marginparwidth,
3Droll=0,
3Dortho=0.0045,
3Dc2c=.7 .55 .43,
3Dcoo=0 0 0,
3Droo=150}{width=\marginparwidth}{figures/figspace5cb_3D}
\\(b)}
	
	$x=0$: 	The trace is $z=y^2$, a parabola opening up in the $y-z$ plane.
	
	$y=0$: 	The trace is $z=-x^2$, a parabola opening down in the $x-z$ plane. 
	
	Sketching these two parabolas gives a sketch like that in \autoref{fig:space5c} (a), and filling in the surface gives a sketch like (b).
\end{enumerate}
\end{example}

\begin{example}[Identifying quadric surfaces]\label{ex_space6}
Consider the quadric surface shown in \autoref{fig:space6}. Which of the following equations best fits this surface?
\begin{align*}
\text{(a)}\quad x^2-y^2-\dfrac{z^2}{9}&=0 & \text{(c)}\quad z^2-x^2-y^2&=1 \\
\text{(b)}\quad x^2-y^2-z^2&=1 & \text{(d)}\quad 4x^2-y^2-\dfrac{z^2}9&=1
\end{align*}
\solution
The image clearly displays a hyperboloid of two sheets. The gallery informs us that the equation will have a form similar to $\frac{z^2}{c^2}-\frac{x^2}{a^2}-\frac{y^2}{b^2}=1$. 

We can immediately eliminate option (a), as the constant in that equation is not 1.

\mtable[5\baselineskip]{A possible equation of this quadric surface is found in \autoref{ex_space6}.}{fig:space6}{%
\myincludeasythree{width=\marginparwidth,
3Droll=0,
3Dortho=0.0045,
3Dc2c=.67 .67 .33,
3Dcoo=0 0 0,
3Droo=150}{width=\marginparwidth}{figures/figspace6_3D}}

The hyperboloid ``opens'' along the $x$-axis, meaning $x$ must be the only variable with a  positive coefficient, eliminating (c).

The hyperboloid is wider in the $z$-direction than in the $y$-direction, so we need an equation where $c>b$. This eliminates (b), leaving us with (d). We should verify that the equation given in (d), $4x^2-y^2-\frac{z^2}9=1$, fits.

We already established that this equation describes a hyperboloid of two sheets that opens in the $x$-direction and is wider in the $z$-direction than in the $y$. Now note the coefficient of the $x$-term. Rewriting $4x^2$ in standard form, we have: $\ds 4x^2 = \frac{x^2}{(1/2)^2}$. Thus when $y=0$ and $z=0$, $x$ must be $1/2$; i.e., each hyperboloid ``starts'' at $x=1/2$. This matches our figure.

We conclude that $\ds 4x^2-y^2-\frac{z^2}9=1$ best fits the graph.
\end{example}

This section has introduced points in space and shown how equations can describe surfaces. The next sections explore \emph{vectors}, an important mathematical object that we'll use to explore curves in space.

\printexercises{exercises/10_01_exercises}


\appendix

\pagenumbering{arabic}\renewcommand{\thepage}{A.\arabic{page}}

\makeexercisesection{Standalone Solutions To All Problems}

\end{document}
