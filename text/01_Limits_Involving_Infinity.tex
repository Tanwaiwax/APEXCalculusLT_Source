\section{Limits Involving Infinity}\label{sec:limits_infty}

In \autoref{def:limit} we stated that in the equation $\ds \lim_{x\to c}f(x) = L$, both $c$ and $L$ were numbers. In this section we relax that definition a bit by considering situations when it makes sense to let $c$ and/or $L$ be ``infinity.''

As a motivating example, consider $f(x) = 1/x^2$, as shown in \autoref{fig:oneoverxsquared}. Note how, as $x$ approaches 0, $f(x)$ grows very, very large. It seems appropriate, and descriptive, to state that\vspace{-.5\baselineskip}
\[\lim_{x\rightarrow 0} \frac1{x^2}=\infty.\]
Also note that as $x$ gets very large, $f(x)$ gets very, very small. We could represent this concept with notation such as\vspace{-.3\baselineskip}
\[\lim_{x\rightarrow \infty} \frac1{x^2}=0.\]

\mtable{Graphing $f(x) = 1/x^2$ for values of $x$ near 0.}{fig:oneoverxsquared}{\pdftooltip{\begin{tikzpicture}
\begin{axis}[width=1.16\marginparwidth,tick label style={font=\scriptsize},
minor x tick num=1,axis y line=middle,axis x line=middle,
ymin=-.1,ymax=110,xmin=-1.1,xmax=1.1,name=myplot]
\addplot [draw={\colorone},smooth,thick,domain=-1:-.1] {1/(x*x)};
\addplot [draw={\colorone},smooth,thick,domain=.1:1] {1/(x*x)};
\end{axis}
\node [right] at (myplot.right of origin) {\scriptsize $x$};
\node [above] at (myplot.above origin) {\scriptsize $y$};
\end{tikzpicture}}{A curve near the x axis when x is far from 0, but getting arbitrarily high as x approaches 0.}}

We explore both types of use of $\infty$ in turn.

\begin{definition}[Limit of Infinity, $\infty$]\label{def:limit_of_infinity}%
We say $\ds \lim_{x\rightarrow c} f(x)=\infty$ if for every $M>0$ there exists $\delta>0$ such that for all $x\neq c$, if  $\abs{x-c}<\delta$, then $f(x)\geq M$. \index{limit!of infinity}
\end{definition}

This is just like the $\epsilon$-$\delta$ definition from \autoref{sec:limit_def}.  In that definition, given any (small) value $\epsilon$, if we let $x$ get close enough to $c$ (within $\delta$ units of $c$) then $f(x)$ is guaranteed to be within $\epsilon$ of $f(c)$.  Here, given any (large) value $M$, if we let $x$ get close enough to $c$ (within $\delta$ units of $c$), then $f(x)$ will be at least as large as $M$.  In other words, if we get close enough to $c$, then we can make $f(x)$ as large as we want.  We can define limits equal to $-\infty$ in a similar way.

It is important to note that by saying $\ds \lim_{x\to c}f(x) = \infty$ we are implicitly stating that \emph{the} limit of $f(x)$, as $x$ approaches $c$, \emph{does not exist.} A limit only exists when $f(x)$ approaches an actual numeric value. We use the concept of limits that approach infinity because it is helpful and descriptive.

\youtubeVideo{-vwcLvb9A0s}{Calculus --- Infinite Limits}

\begin{example}[Evaluating limits involving infinity]\label{ex_inflim1}%
Find $\ds \lim_{x\to1}\frac1{(x-1)^2}$ as shown in \autoref{fig:inflim1}.

\mtable{Observing infinite limit as $x\to 1$ in \autoref{ex_inflim1}.}{fig:inflim1}{\pdftooltip{\begin{tikzpicture}
\begin{axis}[width=\marginparwidth,tick label style={font=\scriptsize},
minor x tick num=1,axis y line=middle,axis x line=middle,
ymin=-1,ymax=110,xmin=-.1,xmax=2.1,name=myplot]
\addplot [draw={\colorone},smooth,thick,domain=0:.9] {1/((1-x)*(1-x))};
\addplot [draw={\colorone},smooth,thick,domain=1.1:2] {1/((1-x)*(1-x))};
\draw [dashed,thick] (axis cs: 1,0) -- (axis cs: 1,100);
\end{axis}
\node [right] at (myplot.right of origin) {\scriptsize $x$};
\node [above] at (myplot.above origin) {\scriptsize $y$};
\end{tikzpicture}}{The same asymptote as the previous picture, but now at x=1.}}
\solution
In \autoref{ex_no_limit2} of \autoref{sec:limit_intro}, by inspecting values of $x$ close to 1 we concluded that this limit does not exist.  That is, it cannot equal any real number.  But the limit could be infinite.  And in fact, we see that the function does appear to be growing larger and larger, as $f(.99)=10^4$, $f(.999)=10^6$, $f(.9999)=10^8$.  A similar thing happens on the other side of 1.  In general, let a ``large'' value $M$ be given. Let $\delta=1/\sqrt{M}$. If $x$ is within $\delta$ of 1, i.e., if $\abs{x-1}<1/\sqrt{M}$, then:\vspace{-.5\baselineskip}
	\begin{align*}
	\abs{x-1} &< \frac{1}{\sqrt{M}} \\
	(x-1)^2 &< \frac{1}{M}\\
	\frac{1}{(x-1)^2} &> M,
	\end{align*}
	which is what we wanted to show.  So we may say $\ds\lim_{x\rightarrow 1}1/{(x-1)^2}=\infty$.
\end{example}

\begin{example}[Evaluating limits involving infinity]\label{ex_inflim2}%
Find $\ds\lim_{x\to0}\frac1x$, as shown in \autoref{fig:oneoverx}.

\mtable{Evaluating $\ds\lim_{x\rightarrow 0}\frac1x$.}{fig:oneoverx}{\pdftooltip{\begin{tikzpicture}
\begin{axis}[width=1.16\marginparwidth,tick label style={font=\scriptsize},
minor x tick num=1,axis y line=middle,axis x line=middle,
ymin=-55,ymax=55,xmin=-1.1,xmax=1.1,name=myplot]
\addplot [draw={\colorone},smooth,thick,domain=-1:-.02] {1/x};
\addplot [draw={\colorone},smooth,thick,domain=.02:1] {1/x};
\end{axis}
\node [right] at (myplot.right of origin) {\scriptsize $x$};
\node [above] at (myplot.above origin) {\scriptsize $y$};
\end{tikzpicture}}{A curve starting near y=0 when x is far from 0.  As x approaches 0 from the right, the y values get arbitrarily big.  As x approaches 0 from the left, the y values get arbitrarily small.}}
\solution
It is easy to see that the function grows without bound near 0, but it does so in different ways on different sides of 0.  Since its behavior is not consistent, we cannot say that $\ds \lim_{x\to 0}\frac{1}{x}=\infty$. However, we can make a statement about one-sided limits. We can state that $\ds \lim_{x\to0^+}\frac1x=\infty$ and $\ds \lim_{x\to0^-}\frac1x=-\infty$.
\end{example}

\subsection{Vertical asymptotes}\index{asymptote!vertical}

\begin{definition}[Vertical Asymptote]\label{def:vert_asymp}%
The function $f(x)$ has a \textbf{vertical asymptote at} $\mathbf{x=c}$ if any one of the following is true:
\[
\lim_{x\to c^-} f(x)=\pm \infty, \quad \lim_{x\to c^+} f(x)=\pm \infty, \quad \text{or} \quad \lim_{x\to c} f(x)=\pm \infty
\]
\end{definition}

%If the limit of $f(x)$ as $x$ approaches $c$ from either the left or right (or both) is $\infty$ or $-\infty$, we say the function has a \textbf{vertical asymptote} at $c$.\\

\begin{example}[Finding vertical asymptotes]\label{ex_vertasy1}%
Find the vertical asymptotes of $f(x)=\dfrac{3x}{x^2-4}$.
\solution
Vertical asymptotes occur where the function grows without bound; this can occur at values of $c$ where the denominator is 0. When $x$ is near $c$, the denominator is small, which in turn can make the function take on large values.  In the case of the given function, the denominator is 0 at $x=\pm 2$.  We will consider the limits as $x$ approaches $\pm 2$ from the left and right to determine the vertical asymptotes. \vspace{-.3\baselineskip}
%
\mtable{Graphing $f(x) = \dfrac{3x}{x^2-4}$.}{fig:multipleasymptotes}{\pdftooltip{\begin{tikzpicture}
\begin{axis}[width=1.16\marginparwidth,tick label style={font=\scriptsize},
minor x tick num=1,axis y line=middle,axis x line=middle,
ymin=-16,ymax=16,xmin=-6.1,xmax=6.1,name=myplot]
\addplot [draw={\colorone},smooth,thick,domain=-6:-2.1] {3*x/(x*x-4)};
\addplot [draw={\colorone},smooth,thick,domain=-1.9:1.9] {3*x/(x*x-4)};
\addplot [draw={\colorone},smooth,thick,domain=2.1:6] {3*x/(x*x-4)};
\draw [dashed,thick] (axis cs:-2,-16) -- (axis cs:-2,16);
\draw [dashed,thick] (axis cs:2,-16) -- (axis cs: 2,16);
\end{axis}
\node [right] at (myplot.right of origin) {\scriptsize $x$};
\node [above] at (myplot.above origin) {\scriptsize $y$};
\end{tikzpicture}}{A curve starting near (-6,0), curving toward -∞ as x approaches -2 from the the left.  Another curve is at ∞ as x approaches -2 from the right, goes through the origin, and then toward -∞ as x approaches 2 from the left.  A final curve is at ∞ as x approaches 2 from the right, and goes toward (6,0).}}%
%
\begin{align*}
\lim_{x\to  2^+}\frac{3x}{(x-2)(x+2)}&= \infty\\
\lim_{x\to  2^-}\frac{3x}{(x-2)(x+2)}&=-\infty\\
\lim_{x\to -2^+}\frac{3x}{(x-2)(x+2)}&= \infty\\
\lim_{x\to -2^-}\frac{3x}{(x-2)(x+2)}&=-\infty
\end{align*}
We can graphically confirm the limits above by looking at \autoref{fig:multipleasymptotes}. Thus the vertical asymptotes are at $x=\pm2$.
\end{example}

When a rational function has a vertical asymptote at $x=c$, we can conclude that the denominator is 0 at $x=c$. However, just because the denominator is 0 at a certain point does not mean there is a vertical asymptote there.  For instance, $f(x)=(x^2-1)/(x-1)$ does not have a vertical asymptote at $x=1$, as shown in \autoref{fig:noasy}. 

\mtable{Graphically showing that\\
$f(x)=\dfrac{x^2-1}{x-1}$ does not have an asymptote at $x=1$.}{fig:noasy}{\pdftooltip{\begin{tikzpicture}
\begin{axis}[width=1.16\marginparwidth,tick label style={font=\scriptsize},
minor x tick num=1,axis y line=middle,axis x line=middle,
ymin=-.2,ymax=3.2,xmin=-1.2,xmax=2.2,name=myplot]
\addplot [draw={\colorone},smooth,thick] coordinates {(-1,0) (2,3)};
\fill[white,draw={\colortwo}] (axis cs:1,2) circle (1pt);
\end{axis}
\node [right] at (myplot.right of origin) {\scriptsize $x$};
\node [above] at (myplot.above origin) {\scriptsize $y$};
\end{tikzpicture}}{A line segment connecting (-1,0) to (2,3) with a hollow dot at (1,2).}}

While the denominator does get small near $x=1$, the numerator gets small too, matching the denominator step for step. In fact, factoring the numerator, we get\vspace{-.5\baselineskip}
\[f(x)=\frac{(x-1)(x+1)}{x-1}.\]
Dividing out common term, we get that $f(x)=x+1$ for $x\not=1$.   So there is clearly no asymptote, rather a hole exists in the graph at $x=1$.\bigskip

The above example may seem a little contrived.  Another example demonstrating this important concept is $f(x)= (\sin x)/x$. We have considered this function several times in the previous sections. We found that $\ds \lim_{x\to0}\frac{\sin x}{x}=1$; i.e., there is no vertical asymptote. No simple algebraic manipulation makes this fact obvious; we used the Squeeze Theorem in \autoref{sec:limit_analytically} to prove this.\bigskip

If the denominator is 0 at a certain point but the numerator is not, then there will usually be a vertical asymptote at that point. On the other hand, if the numerator and denominator are both zero at that point, then there may or may not be a vertical asymptote at that point.  This case where the numerator and denominator are both zero returns us to an important topic.

\subsection{Indeterminate Forms}
\index{limit!indeterminate form}\index{indeterminate form}

%When working with limits and infinity, it is important not to go beyond what the rules of algebra and limits allow.  Consider again the limit below:
We have seen how the limits 
\[\lim_{x\rightarrow 0}\frac{\sin x}{x}\quad \text{and}\quad \lim_{x\to1}\frac{x^2-1}{x-1}\]
each return the indeterminate form ``$0/0$'' when we blindly plug in $x=0$ and $x=1$, respectively. However, $0/0$ is not a valid arithmetical expression. It gives no indication that the respective limits are 1 and 2.% (hence the use of the word ``indeterminate.'')
%Blindly plugging in $x=0$ would give us the expression $0/0$.  This is not a valid arithmetical expression because division by 0 is not allowed.  In fact, as we have seen already, the correct value of the limit is 1.

%The expression $0/0$ is called an \emph{indeterminate form}.  For an idea as to where the name comes from, consider the following limit:
%\[\lim_{x\rightarrow 1}\frac{x^2-1}{x-1}.\]
%Blindly plugging in $x=1$ here gives $0/0$.  However, this time, the correct value of the limit is 2, which can be seen by factoring the numerator and then plugging in $x=1$.  So we have seen that the initial expression $0/0$ can correspond to a limit of 1 or a limit of 2.  In fact, w

With a little cleverness, one can come up with $0/0$ expressions which have a limit of $\infty$, 0, or any other real number.  That is why this expression is called \emph{indeterminate}.

A key concept to understand is that such limits do not really return $0/0$. Rather, keep in mind that we are taking \emph{limits}. What is really happening is that the numerator is shrinking to 0 while the denominator is also shrinking to 0. The respective rates at which they do this are very important and determine the actual value of the limit.

An indeterminate form indicates that one needs to do more work in order to compute the limit. That work may be algebraic (such as factoring and dividing) or it may require a tool such as the Squeeze Theorem. %algebraically manipulate the expression in some way in order to compute the limit.  It may also indicate that you need a completely different approach, like with $(\sin x)/x$. 
 In a later section we will learn a technique called L'Hôpital's Rule that provides another way to handle indeterminate forms.  
 
Some other common indeterminate forms are $\infty-\infty$, $\infty\cdot 0$, $\infty/\infty$, $0^0$, $\infty^0$ and $1^{\infty}$. Again, keep in mind that these are the ``blind'' results of evaluating a limit, and each, in and of itself, has no meaning. The expression $\infty-\infty$ does not really mean ``subtract infinity from infinity.'' Rather, it means ``One quantity is subtracted from the other, but both are growing without bound.'' What is the result? It is possible to get every value between $-\infty$ and $\infty$

Note that $1/0$ and $\infty/0$ are not indeterminate forms, though they are not exactly valid mathematical expressions, either.  In each, the function is growing without bound, indicating that the limit will be $\infty$, $-\infty$, or simply not exist if the left- and right-hand limits do not match.


\subsection{Limits at Infinity and Horizontal Asymptotes}

At the beginning of this section we briefly considered what happens to $f(x) = 1/x^2$ as $x$ grew very large. 
Graphically, it concerns the behavior of the function to the ``far right'' of the graph. We make this notion more explicit in the following definition.

%{\tcbset{grow to right by=6.5em}}%
\begin{definition}[Limits at Infinity]\label{def:limit_at_infinity}%
~\\[-2\baselineskip]\index{limit!at infinity}\begin{enumerate}
\item We say $\ds\lim_{x\to\infty} f(x)=L$ if for every $\epsilon>0$ there exists $M>0$ such that if $x\geq M$, then $\abs{f(x)-L}<\epsilon$.

\item We say $\ds\lim_{x\to-\infty} f(x)=L$ if for every $\epsilon>0$ there exists $M<0$ such that if $x\leq M$, then $\abs{f(x)-L}<\epsilon$.

 %In other words, the limit is $L$ if no matter how close you want to get to $L$, for large enough values of $x$ ($x>M$), you can get that close.
%\item  If $\ds\lim_{x\rightarrow\infty} f(x)=L$ or $\ds\lim_{x\rightarrow-\infty} f(x)=L$, we say that $y=L$ is a \textbf{horizontal asymptote} of $f$.
\end{enumerate}
\end{definition}

\begin{definition}[Horizontal Asymptote]\label{def:horiz_asymp}%
The function $f(x)$ has a \textbf{horizontal asymptote at} $\mathbf{y=L}$ if either\index{asymptote!horizontal}
\[\lim_{x\to \infty} f(x)=L \quad \text{or} \quad \lim_{x\to -\infty} f(x)=L\]
\end{definition}

%We can define $\lim_{x\rightarrow-\infty} f(x)=L$ in an analogous way.  
We can also define limits such as $\ds\lim_{x\to\infty}f(x)=\infty$ by combining this definition with \autoref{def:limit_of_infinity}. %It is a good exercise to try this.

\mtable[-1in]{Using a graph and a table to approximate a horizontal asymptote in \autoref{ex_hzasy1}.}{fig:hzasy1}{%
\pdftooltip{\begin{tikzpicture}
\begin{axis}[width=1.16\marginparwidth,tick label style={font=\scriptsize},
minor x tick num=1,axis y line=middle,axis x line=middle,
ymin=-.2,ymax=1.1,xmin=-21,xmax=21,name=myplot]
\addplot [draw={\colorone},smooth,thick,domain=-20:20] {x*x/(x*x+4)};
\draw [dashed,thick](axis cs:-21,1) -- (axis cs:21,1);
\end{axis}
\node [right] at (myplot.right of origin) {\scriptsize $x$};
\node [above] at (myplot.above origin) {\scriptsize $y$};
\end{tikzpicture}}{A curve starting near (-20,4), curving toward the origin, and turning back toward (20,4).  There is also a dashed line at y=4.}
\\(a)\smallskip\\
\tagpdfsetup{table/header-rows={1}}
 \begin{tabular}{rl}
	$x$ & $f(x)$ \\ \midrule
	10 & 0.9615 \\
	100 & 0.9996 \\
	10000 & 0.999996\\
	$-10$ & 0.9615 \\
	$-100$ & 0.9996 \\
	$-10000$ & 0.999996
 \end{tabular} \\
 (b)}%

\begin{example}[Approximating horizontal asymptotes]\label{ex_hzasy1}%
Approximate the horizontal asymptote(s) of $\ds f(x)=\frac{x^2}{x^2+4}$.
\solution
We will approximate the horizontal asymptotes by approximating the limits\vspace{-.3\baselineskip}
\[
\lim_{x\to-\infty} \frac{x^2}{x^2+4}\quad \text{and}\quad \lim_{x\to\infty} \frac{x^2}{x^2+4}.
\]
\autoref{fig:hzasy1}(a) shows a sketch of $f$, and part (b) gives values of $f(x)$ for large magnitude values of $x$. It seems reasonable to conclude from both of these sources that $f$ has a horizontal asymptote at $y=1$.

Later, we will show how to determine this analytically.
\end{example}

Horizontal asymptotes can take on a variety of forms. \autoref{fig:hzasy}(a) shows that $f(x) = x/(x^2+1)$ has a horizontal asymptote of $y=0$, where 0 is approached from both above and below.

\autoref{fig:hzasy}(b) shows that $f(x) =x/\sqrt{x^2+1}$ has two horizontal asymptotes; one at $y=1$ and the other at $y=-1$.

\autoref{fig:hzasy}(c) shows that $f(x) = (\sin x)/x$ has even more interesting behavior than at just $x=0$; as $x$ approaches $\pm\infty$, $f(x)$ approaches 0, but oscillates as it does this.

\noindent\begin{minipage}[t]{\linewidth}\noindent%
\captionsetup{type=figure}%
\flushinner{%
\tagpdfsetup{table/header-rows={2}}
\begin{tabular}{ c c c }
\pdftooltip{\begin{tikzpicture}
\begin{axis}[width=1.16\marginparwidth,tick label style={font=\scriptsize},
minor x tick num=1,axis y line=middle,axis x line=middle,
ymin=-1.1,ymax=1.1,xmin=-21,xmax=21,name=myplot]
\addplot [draw={\colorone},smooth,thick,domain=-20:20] {x/(x*x+1)};
\end{axis}
\node [right] at (myplot.right of origin) {\scriptsize $x$};
\node [above] at (myplot.above origin) {\scriptsize $y$};
\end{tikzpicture}}{A curve starting near (-20,0), going toward (-1,-1/2), then through the origin, then (1,1/2), and finally toward (20,0).}
&
\pdftooltip{\begin{tikzpicture}
\begin{axis}[width=1.16\marginparwidth,tick label style={font=\scriptsize},
minor x tick num=1,axis y line=middle,axis x line=middle,
ymin=-1.1,ymax=1.1,xmin=-21,xmax=21,name=myplot]
\addplot [draw={\colorone},smooth,thick,domain=-20:20,samples=100] {x/sqrt(x*x+1)};
\end{axis}
\node [right] at (myplot.right of origin) {\scriptsize $x$};
\node [above] at (myplot.above origin) {\scriptsize $y$};
\end{tikzpicture}}{A curve starting near (-20,-1), going through the origin, and then toward (20,1).}
&
\pdftooltip{\begin{tikzpicture}
\begin{axis}[width=1.16\marginparwidth,tick label style={font=\scriptsize},
minor x tick num=1,axis y line=middle,axis x line=middle,
ymin=-.3,ymax=1.1,xmin=-21,xmax=21,name=myplot]
\addplot [draw={\colorone},smooth,thick,domain=-20:-.1] {sin(deg(x))/x};
\addplot [draw={\colorone},smooth,thick,domain=.1:20] {sin(deg(x))/x};
\end{axis}
\node [right] at (myplot.right of origin) {\scriptsize $x$};
\node [above] at (myplot.above origin) {\scriptsize $y$};
\end{tikzpicture}}{A curve wiggling back and forth across the x axis.  The wiggles are smaller when x is far from the origin, and the curve has its peak at (0,1).}
\\
(a) & (b) & (c)
\end{tabular}}
\caption{Considering different types of horizontal asymptotes.}
\label{fig:hzasy}
\end{minipage}

\mnote{\textbf{Note:} With our definitions, we can also now say that Theorems \ref{thm:limit_algebra}, \ref{thm:limit_composition}, and \ref{thm:sqz} also hold when $c=-\infty$ and $c=\infty$.}

We can analytically evaluate limits at infinity for rational functions once we understand $\ds\lim_{x\rightarrow\infty} 1/x$.  As $x$ gets larger and larger, the $1/x$ gets smaller and smaller, approaching 0.  We can, in fact, make $1/x$ as small as we want by choosing a large enough value of $x$.  Given $\epsilon$, we can make $1/x<\epsilon$  by choosing $x>1/\epsilon$.  Thus we have $\lim_{x\rightarrow\infty} 1/x=0$.  
It is now not much of a jump to conclude the following:

{\tcbset{grow to right by=2em}
\begin{theorem}[Limits of $\dfrac1{x^n}$]\label{thm:lim_of_x_to_n}%
For any $n>0$, 
\[
\lim_{x\to\infty}\frac{1}{x^n}=0
\quad\text{and}\quad
\lim_{x\to-\infty}\frac{1}{x^n}=0
\qquad\text{(provided $x^n$ is defined for $x<0$)}
\]
\end{theorem}}

%\[\lim_{x\rightarrow\infty}\frac1{x^n}=0\quad \text{and}\quad \lim_{x\rightarrow-\infty}\frac1{x^n}=0\]

Now suppose we need to compute the following limit:
\[\lim_{x\rightarrow\infty}\frac{x^3+2x+1}{4x^3-2x^2+9}.\]
A good way of approaching this is to divide through the numerator and denominator by $x^3$ (hence dividing by 1), which is the largest power of $x$ to appear in the function.  Doing this, we get
\begin{align*}
\lim_{x\to\infty}\frac{x^3+2x+1}{4x^3-2x^2+9} &=
\lim_{x\to\infty}\frac{1/x^3}{1/x^3}\cdot\frac{x^3+2x+1}{4x^3-2x^2+9}\\ &=\lim_{x\to\infty}\frac{x^3/x^3+2x/x^3+1/x^3}{4x^3/x^3-2x^2/x^3+9/x^3}\\ &= \lim_{x\to\infty}\frac{1+2/x^2+1/x^3}{4-2/x+9/x^3}\\
&=\frac{1+0+0}{4-0+0}=\frac14.
\end{align*}
We used the rules for limits (which also hold for limits at infinity), as well as the fact about limits of $1/x^n$. This procedure works for any rational function and is highlighted in the following Key Idea.

\begin{keyidea}[Finding Limits of Rational Functions at Infinity]\label{key:rat_lim_at_inf}%
Let $f(x)$ be a rational function of the following form:
\[
f(x)
=\frac{a_nx^n + a_{n-1}x^{n-1}+\dots + a_1x + a_0}
{b_mx^m + b_{m-1}x^{m-1} + \dots + b_1x + b_0},
\]
where any of the coefficients may be 0 except for $a_n$ and $b_m$.\\
To determine $\ds\lim_{x\to \infty}f(x)$ or $\ds\lim_{x\to-\infty}f(x)$:
\begin{enumerate}
\item Divide the numerator and denominator by $x^m$.
\item Simplify as much as possible.
\item Use \autoref{thm:lim_of_x_to_n} to find the limit.
\end{enumerate}
\end{keyidea}

%We can see why this is true.
If the highest power of $x$ is the same in both the numerator and denominator (i.e.\ $n=m$), we will be in a situation like the example above, where we will divide by $x^n$ and in the limit all the terms will approach 0 except for $a_nx^n/x^n$ and $b_mx^m/x^n$. Since $n=m$, this will leave us with the limit $a_n/b_m$.  If $n<m$, then after dividing through by $x^m$, all the terms in the numerator will approach 0 in the limit, leaving us with $0/b_m$ or 0.  If $n>m$, and we try dividing through by $x^n$, we end up with all the terms in the denominator tending toward 0, while the $x^n$ term in the numerator does not approach 0.  This is indicative of some sort of infinite limit.

Intuitively, as $x$ gets very large, all the terms in the numerator are small in comparison to $a_nx^n$, and likewise all the terms in the denominator are small compared to $b_nx^m$.  If $n=m$, looking only at these two important terms, we have $(a_nx^n)/(b_nx^m)$.  This reduces to $a_n/b_m$.  If $n<m$, the function behaves like $a_n/(b_mx^{m-n})$, which tends toward 0.  If $n>m$, the function behaves like $a_nx^{n-m}/b_m$, which will tend to either $\infty$ or $-\infty$ depending on the values of $n$, $m$, $a_n$, $b_m$ and whether you are looking for $\lim_{x\rightarrow\infty} f(x)$ or $\lim_{x\rightarrow-\infty} f(x)$.

This procedure works for any rational function.  In fact, it gives us the following key idea.

\begin{keyidea}[Limits of Rational Functions at Infinity]\label{thm:lim_rational_fn_at_infty}%
Let $f(x)$ be a rational function of the following form:
\[f(x)=\frac{a_nx^n + a_{n-1}x^{n-1}+\dots + a_1x + a_0}{b_mx^m + b_{m-1}x^{m-1} + \dots + b_1x + b_0},\]
where any of the coefficients may be 0 except for $a_n$ and $b_m$.
\begin{enumerate}
\item If $n=m$, then $\ds\lim_{x\rightarrow\infty} f(x) = \lim_{x\rightarrow-\infty} f(x) = \frac{a_n}{b_m}$.
\item If $n<m$, then $\ds\lim_{x\rightarrow\infty} f(x) = \lim_{x\rightarrow-\infty} f(x) = 0$.
\item If $n>m$, then $\ds\lim_{x\rightarrow\infty} f(x)$ and $\ds\lim_{x\to-\infty} f(x)$ are both infinite.
\end{enumerate}
\end{keyidea}

%With care, we can quickly evaluate limits at infinity for a large number of functions by considering the largest powers of $x$. For instance, consider again $\ds\lim_{x\to\pm\infty}\frac{x}{\sqrt{x^2+1}},$ graphed in \autoref{fig:hzasy}(b). When $x$ is very large, $x^2+1 \approx x^2$. Thus \[\sqrt{x^2+1}\approx \sqrt{x^2} = \abs{x},\quad \text{and}\quad \frac{x}{\sqrt{x^2+1}} \approx \frac{x}{\abs{x}}.\] This expression is 1 when $x$ is positive and $-1$ when $x$ is negative. Hence we get asymptotes of $y=1$ and $y=-1$, respectively.

\begin{example}[Horizontal Asymptotes Involving Square Roots]\label{ex_sqrt_asy}%
Find the horizontal asymptotes of $\dfrac{x}{\sqrt{x^2+1}}$.
\solution We must consider the limits as $x\to \pm \infty$. When $x$ is very large, $x^2+1\approx x^2$ and thus $\sqrt{x^2+1}\approx \sqrt{x^2}=\abs x$.
\begin{align*}
\lim_{x\to\infty}\frac{x}{\sqrt{x^2+1}}
&=\lim_{x\to\infty}\frac{x/x}{\sqrt{x^2/{x^2}+1/{x^2}}}\\
&=\lim_{x\to\infty}\frac{1}{\sqrt{1+1/{x^2}}}\\
&=1
\end{align*}
Therefore, $y=1$ is a horizontal asymptote.
Similarly, 
\begin{align*}
\lim_{x\to-\infty}\frac{x}{\sqrt{x^2+1}}
&=\lim_{x\to-\infty}\frac{x/(-x)}{\sqrt{x^2/{x^2}+1/{x^2}}}\\
&=\lim_{x\to-\infty}\frac{-1}{\sqrt{1+1/{x^2}}}\\
&=-1
\end{align*}
Therefore, $y=-1$ is also a horizontal asymptote.
\end{example}

\begin{example}[Finding a limit of a rational function]\label{ex_hzasy2}%
Confirm analytically that $y=1$ is the horizontal asymptote of $\ds f(x) = \frac{x^2}{x^2+4}$, as approximated in \autoref{ex_hzasy1}.
\solution
Before using \autoref{thm:lim_rational_fn_at_infty}, let's use the technique of evaluating limits at infinity of rational functions that led to that theorem. The largest power of $x$ in $f$ is 2, so divide the numerator and denominator of $f$ by $x^2$, then take limits.\vspace{-.3\baselineskip}
\begin{align*}
	\lim_{x\to\infty}\frac{x^2}{x^2+4}
	&= \lim_{x\to\infty}\frac{x^2/x^2}{x^2/x^2+4/x^2}\\
	&= \lim_{x\to\infty}\frac{1}{1+4/x^2}\\
	&= \frac1{1+0}\\
	&= 1.
\end{align*}

We can also use \autoref{thm:lim_rational_fn_at_infty} directly; in this case $n=m$ so the limit is the ratio of the leading coefficients of the numerator and denominator, i.e., $1/1=1$.
\end{example}

\begin{example}[Finding limits of rational functions]\label{ex_hzasy3}%
(a) Analytically evaluate the following limits, and (b) Use \autoref{thm:lim_rational_fn_at_infty} to evaluate each limit.
\begin{multicols}{2}
	\begin{enumerate}\lxAddClass{columns2}
		\item	$\ds\lim_{x\to-\infty}\frac{x^2+2x-1}{x^3+1}$
		\item	$\ds\lim_{x\to\infty}\frac{x^2+2x-1}{1-x-3x^2}$
		\item	$\ds\lim_{x\to\infty}\frac{x^2-1}{3-x}$
		\item[]
	\end{enumerate}
\end{multicols}
\clearpage
\solution
%
\mtable{Visualizing the functions in \autoref{ex_hzasy3}.}{fig:hzasy3}{%
\pdftooltip{\begin{tikzpicture}
\begin{axis}[width=\marginparwidth,tick label style={font=\scriptsize},
minor x tick num=1,axis y line=middle,axis x line=middle,
ymin=-.6,ymax=.6,xmin=-41,xmax=2,name=myplot]
\addplot [draw={\colorone},smooth,thick,domain=-40:-2] {(x*x+2*x-1)/(x*x*x+1)};
\end{axis}
\node [right] at (myplot.right of origin) {\scriptsize $x$};
\node [above] at (myplot.above origin) {\scriptsize $y$};
\end{tikzpicture}}{A curve starting near (-40,0), going toward (-5,-0.1), and then upward.}
\\ (a) \\[10pt]
\pdftooltip{\begin{tikzpicture}
\begin{axis}[width=\marginparwidth,tick label style={font=\scriptsize},
minor x tick num=1,minor y tick num = 4,axis y line=middle,axis x line=middle,
ymin=-.55,ymax=.55,xmin=-1,xmax=41,name=myplot]
\addplot [draw={\colorone},smooth,thick,domain=1.86:40] {(x*x+2*x-1)/(1-x-3*x*x)};
\draw [dashed,thick](axis cs:-1,-.33333) -- (axis cs:41,-.3333);
\end{axis}
\node [right] at (myplot.right of origin) {\scriptsize $x$};
\node [above] at (myplot.above origin) {\scriptsize $y$};
\end{tikzpicture}}{A curve starting near (-1,-0.5) and then going toward (40,-1/3).  There is a dashed line at y=-1/3.}
\\ (b) \\[10pt]
\pdftooltip{\begin{tikzpicture}
\begin{axis}[width=\marginparwidth,tick label style={font=\scriptsize},
minor x tick num=1,axis y line=middle,axis x line=middle,
ymin=-50,ymax=5,xmin=-5,xmax=41,name=myplot]
\addplot [draw={\colorone},smooth,thick,domain=4:40] {(x*x-1)/(3-x)};
\end{axis}
\node [right] at (myplot.right of origin) {\scriptsize $x$};
\node [above] at (myplot.above origin) {\scriptsize $y$};
\end{tikzpicture}}{A curve starting near (4,-15), starting upward, and then becoming a line going diagonally down and right.}
\\ (c)}
%
\begin{enumerate}
	\item	\begin{enumerate}
		\item Divide numerator and denominator by $x^3$.
			\begin{align*}
				\lim_{x\to -\infty} \frac{x^2+2x-1}{x^3+1}
				&= \lim_{x\to -\infty} \frac{x^2/x^3+2x/x^3-1/x^3}{x^3/x^3+1/x^3}\\
				&=\lim_{x\to -\infty} \frac{1/x+2/x^2-1/x^3}{1+1/x^3}\\
				&=\frac{0+0+0}{1+0}=0
			\end{align*}
		\item The highest power of $x$ is in the denominator.  Therefore, the limit is 0; see \autoref{fig:hzasy3}(a).
		\end{enumerate}
	\item	\begin{enumerate}
		\item Divide numerator and denominator by $x^2$.
			\begin{align*}
				\lim_{x\to \infty} \frac{x^2+2x-1}{1-x-3x^2}
				&=\lim_{x\to\infty}
				\frac{x^2/x^2+2x/x^2-1/x^2}{1/x^2-x/x^2-3x^2/x^2}\\
				&=\lim_{x\to \infty} \frac{1+2/x-1/x^2}{1/x^2-1/x-3}\\
				&=\frac{1+0-0}{0-0-3}=-\frac{1}{3}
			\end{align*}
		\item The highest power of $x$ is $x^2$, which occurs in both the numerator and denominator.  The limit is therefore the ratio of the coefficients of $x^2$, which is $-1/3$. See \autoref{fig:hzasy3}(b).
		\end{enumerate}
	\item	\begin{enumerate}
		\item Divide numerator and denominator by $x$.
			\begin{align*}
				\lim_{x\to\infty}\frac{x^2-1}{3-x}
				&=\lim_{x\to \infty}\frac{x^2/x-1/x}{3/x-x/x}\\
				&=\lim_{x\to \infty}\frac{x-1/x}{3/x-1}\\
				&=-\infty
			\end{align*}
		\item The highest power of $x$ is in the numerator so the limit will be $\infty$ or $-\infty$.  To see which, consider only the dominant terms from the numerator and denominator, which are $x^2$ and $-x$.  The expression in the limit will behave like $x^2/(-x)=-x$ for large values of $x$.  Therefore, the limit is $-\infty$. See \autoref{fig:hzasy3}(c).
	\end{enumerate}
\end{enumerate}
\end{example}

\printexercises{exercises/01-06-exercises}

% todo add a practical example here, like the cost to remove 100% of something and vertical asymptotes and the the long term limit of a population to demonstrate limits at infinity.  Or maybe just put these into the exercises.

% maybe
%Do both limits to infinity and limits resulting in infinity.
%Define more clearly indeterminate form.
%Thm: Rules about limits involving infinity
