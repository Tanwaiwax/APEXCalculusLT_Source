\section{One Sided Limits}\label{sec:limit_continuity}

In \autoref{sec:limit_intro} we explored the three ways in which limits of functions failed to exist: 
	\begin{enumerate}
	\item	The function approached different values from the left and right,
	\item	The function grows without bound, and 
	\item	The function oscillates.
	\end{enumerate}
	
In this section we explore in depth the concepts behind \#1 by introducing the \textit{one-sided limit}. We begin with formal definitions that are very similar to the definition of the limit given in \autoref{sec:limit_def}, but the notation is slightly different and ``$x\neq c$'' is replaced with either ``$x<c$'' or ``$x>c$.'' We will consider \#2 in more detail in \autoref{sec:limits_infty}.

\definition{def:onesidedlimit}{One Sided Limits}
{\textbf{Left-Hand Limit} \index{limit!one sided}\index{limit!right handed}\index{limit!left handed}

Let $I$ be an open interval with right endpoint $c$, and let $f$ be a function defined on $I$. %, except possibly at $c$. 
The \sword{limit of $f(x)$, as $x$ approaches $c$ from the left, is $L$}, or, \sword{the left--hand limit of $f$ at $c$ is $L$}, denoted by  
\[\displaystyle \lim_{x\rightarrow c^-} f(x) = L,\]
means that given any $\epsilon > 0$, there exists $\delta > 0$ such that for all $x< c$,  
if  $\abs{x-c}<\delta$, then $\abs{f(x)-L}<\epsilon$.\\
%Let $f$ be a function defined on an open interval containing $c$. 
%The notation $$ \lim_{x\rightarrow c^-} f(x) = L, $$ read as ``the limit of $f(x)$ as $x$ approaches $c$ from the left is $L$,'' or ``the \textit{left-hand limit of $f$ at $c$ is L}'' 
%means that given any $\epsilon > 0$, there exists $\delta > 0$ such that 
%$|x - c| < \delta$ implies $|f(x) - L| < \epsilon$, for all $x<c$.\\

\textbf{Right-Hand Limit}

Let $I$ be an open interval with left endpoint $c$, and let $f$ be a function defined on $I$. %, except possibly at $c$. 
The \sword{limit of $f(x)$, as $x$ approaches $c$ from the right, is $L$}, or, \sword{the right--hand limit of $f$ at $c$ is $L$}, denoted by  
\[\displaystyle \lim_{x\rightarrow c^+} f(x) = L,\]
means that given any $\epsilon > 0$, there exists $\delta > 0$ such that for all $x> c$,  
if  $\abs{x-c}<\delta$, then $\abs{f(x)-L}<\epsilon$.
%Let $f$ be a function defined on an open interval containing $c$. The notation $$ \lim_{x\rightarrow c^+} f(x) = L, $$ read as ``the limit of $f(x)$ as $x$ approaches $c$ from the right is $L$,'' or ``the \textit{right-hand limit of $f$ at $c$ is L}'' 
%means that given any $\epsilon > 0$, there exists $\delta > 0$ such that 
%$|x - c| < \delta$ implies $|f(x) - L| < \epsilon$, for all $x>c$.
}

Practically speaking, when evaluating a left-hand limit, we consider only values of $x$ ``to the left of $c$,'' i.e., where $x<c$. The admittedly imperfect notation $x\to c^-$ is used to imply that we look at values of $x$ to the left of $c$. The notation has nothing to do with positive or negative values of either $x$ or $c$. A similar statement holds for evaluating right-hand limits; there we consider only values of $x$ to the right of $c$, i.e., $x>c$. We can use the theorems from previous sections to help us evaluate these limits; we just restrict our view to one side of $c$.

\youtubeVideo{nOnd3SiYZqM}{One-sided limits from graphs}

We practice evaluating left and right-hand limits through a series of ex\-am\-ples.

\example{ex_onesidea}{Evaluating one sided limits}{Let $\ds f(x) = \begin{cases} 2x & 0\leq x\leq 1 \\ 6-2x & 1<x<2\end{cases},$ as shown in \autoref{fig:onesided1}. Find each of the following: 

\mtable{A graph of $f$ in \autoref{ex_onesidea}.}{fig:onesided1}{\begin{tikzpicture}
\begin{axis}[width=1.16\marginparwidth,tick label style={font=\scriptsize},minor x tick num=1,axis y line=middle,axis x line=middle,ymin=-.4,ymax=4.4,xmin=-.4,xmax=2.4,name=myplot]
\addplot [draw={\colorone},smooth,thick] coordinates {(0,0) (1,2)};
\fill[black,draw=black] (axis cs:1,2) circle (1.5pt);
\fill[black,draw=black] (axis cs:0,0) circle (1.5pt);
\addplot [draw={\colorone},smooth,thick] coordinates {(1,4) (2,2)};
\fill[white,draw=black,thick] (axis cs:1,4) circle (1.5pt);
\fill[white,draw=black,thick] (axis cs:2,2) circle (1.5pt);
\end{axis}
\node [right] at (myplot.right of origin) {\scriptsize $x$};
\node [above] at (myplot.above origin) {\scriptsize $y$};
\end{tikzpicture}}% ends the mtable

\noindent
\begin{minipage}[t]{.5\textwidth}
\begin{enumerate}
\item	$\ds \lim_{x\to 1^-} f(x)$
\item	$\ds \lim_{x\to 1^+} f(x)$
\item	$\ds \lim_{x\to 1} f(x)$
\item	$\ds f(1)$
\end{enumerate}
\end{minipage}%
\begin{minipage}[t]{.5\textwidth}
\begin{enumerate}\addtocounter{enumi}{4}
\item	$\ds \lim_{x\to 0^+} f(x)$
\item	$f(0)$
\item	$\ds \lim_{x\to 2^-} f(x)$
\item	$f(2)$
\end{enumerate}
\end{minipage}}
{For these problems, the visual aid of the graph is likely more effective in evaluating the limits than using $f$ itself. Therefore we will refer often to the graph.
\begin{enumerate}
	\item	As $x$ goes to 1 \textit{from the left}, we see that $f(x)$ is approaching the value of 2. Therefore $\ds \lim_{x\to 1^-} f(x) =2.$
	\item	As $x$ goes to 1 \textit{from the right}, we see that $f(x)$ is approaching the value of 4. Recall that it does not matter that there is an ``open circle'' there; we are evaluating a limit, not the value of the function. Therefore $\ds \lim_{x\to 1^+} f(x)=4$.
	\item	\textit{The} limit of $f$ as $x$ approaches 1 does not exist, as discussed in the first section. The function does not approach one particular value, but two different values from the left and the right.
	\item	Using the definition and by looking at the graph we see that $f(1) = 2$.
	\item	As $x$ goes to 0 from the right, we see that $f(x)$ is also approaching 0. Therefore $\ds \lim_{x\to 0^+} f(x)=0$. Note we cannot consider a left-hand limit at 0 as $f$ is not defined for values of $x<0$.
	\item	Using the definition and the graph, $f(0) = 0$.
	\item	As $x$ goes to 2 from the left, we see that $f(x)$ is approaching the value of 2. Therefore $\ds \lim_{x\to 2^-} f(x)=2.$
	\item	The graph and the definition of the function show that $f(2)$ is not defined.\eoehere
\end{enumerate}}

Note how the left and right-hand limits were different at $x=1$. This, of course, causes \textit{the} limit to not exist. The following theorem states what is fairly intuitive: \textit{the} limit exists precisely when the left and right-hand limits are equal.

\theorem{thm:leftrightlimits}{Limits and One Sided Limits}
{Let $f$ be a function defined on an open interval $I$ containing $c$. \index{limit!does not exist} Then \[\lim_{x\to c}f(x) = L\] if, and only if, \[\lim_{x\to c^-}f(x) = L \quad \text{and} \quad \lim_{x\to c^+}f(x) = L.\]}

The phrase ``if, and only if'' means the two statements are \textit{equivalent}: they are either both true or both false. If the limit equals $L$, then the left and right hand limits both equal $L$. If the limit is not equal to $L$, then at least one of the left and right-hand limits is not equal to $L$ (it may not even exist).
			
One thing to consider in Examples \ref{ex_onesidea} -- \ref{ex_onesided} is that the value of the function may or may not be equal to the value(s) of its left- or right-hand limits, even when these limits agree.

\example{ex_onesideb}{Evaluating limits of a piecewise--defined function}{Let $f(x) = \begin{cases} 2-x & 0<x<1 \\ (x-2)^2 & 1<x<2 \end{cases},$ as shown in \autoref{fig:onesidedb}. Evaluate the following.\\
\noindent
\begin{minipage}[t]{.5\textwidth}
	\begin{enumerate}
		\item	$\ds \lim_{x\to 1^-} f(x)$
		\item	$\ds \lim_{x\to 1^+} f(x)$
		\item	$\ds \lim_{x\to 1} f(x)$
		\item	$\ds f(1)$
	\end{enumerate}
\end{minipage}%
\begin{minipage}[t]{.5\textwidth}
	\begin{enumerate}\addtocounter{enumi}{4}
		\item	$\ds \lim_{x\to 0^+} f(x)$
		\item	$f(0)$
		\item	$\ds \lim_{x\to 2^-} f(x)$
		\item	$f(2)$
	\end{enumerate}	
\end{minipage}
\mfigure{1in}{A graph of $f$ from \autoref{ex_onesideb}}{fig:onesidedb}{figures/figOneSidedLimits2}}
{Again we will evaluate each using both the definition of $f$ and its graph.
\begin{enumerate}
	\item	As $x$ approaches 1 from the left, we see that $f(x)$ approaches 1. Therefore $\ds \lim_{x\to 1^-} f(x)=1.$
	\item	As $x$ approaches 1 from the right, we see that again $f(x)$ approaches 1. Therefore $\ds \lim_{x\to 1+} f(x)=1$.
	\item	\textit{The} limit of $f$ as $x$ approaches 1 exists and is 1, as $f$ approaches 1 from both the right and left. Therefore $\ds \lim_{x\to 1} f(x)=1$.
	\item	$f(1)$ is not defined. Note that 1 is not in the domain of $f$ as defined by the problem, which is indicated on the graph by an open circle when $x=1$.
	\item	As $x$ goes to 0 from the right, $f(x)$ approaches 2. So $\ds \lim_{x\to 0^+} f(x)=2$.
	\item	$f(0)$  is not defined as $0$ is not in the domain of $f$.
	\item	As $x$ goes to 2 from the left, $f(x)$ approaches 0. So $\ds \lim_{x\to 2^-} f(x)=0$.
	\item	$f(2)$  is not defined as 2 is not in the domain of $f$.\eoehere
\end{enumerate}}

\example{ex_onesidec}{Evaluating limits of a piecewise--defined function}{Let $f(x) = \begin{cases} (x-1)^2 & 0\leq x\leq 2, x\neq 1\\ 1 & x=1\end{cases},$ as shown in \autoref{fig:onesidedc}. Evaluate the following.\\
%
\mfigure{-1in}{Graphing $f$ in \autoref{ex_onesidec}}{fig:onesidedc}{figures/figOneSidedLimits3}
%
\noindent
\begin{minipage}[t]{.5\textwidth}
	\begin{enumerate}
		\item	$\ds \lim_{x\to 1^-} f(x)$
		\item	$\ds \lim_{x\to 1^+} f(x)$
	\end{enumerate}
\end{minipage}%
\begin{minipage}[t]{.5\textwidth}
	\begin{enumerate}\addtocounter{enumi}{2}
		\item	$\ds \lim_{x\to 1} f(x)$
		\item	$f(1)$
	\end{enumerate}
\end{minipage}}
{It is clear by looking at the graph that both the left and right-hand limits of $f$, as $x$ approaches 1, is 0. Thus it is also clear that \textit{the} limit is 0; i.e., $\ds \lim_{x\to 1} f(x) = 0$. It is also clearly stated that $f(1) = 1$.}

\example{ex_onesided}{Evaluating limits of a piecewise--defined function}{Let $f(x) = \begin{cases} x^2 & 0\leq x\leq 1 \\ 2-x & 1<x\leq 2\end{cases}.$ Evaluate the following.\\
\noindent
\begin{minipage}[t]{.5\textwidth}
	\begin{enumerate}
		\item	$\ds \lim_{x\to 1^-} f(x)$
		\item	$\ds \lim_{x\to 1^+} f(x)$
	\end{enumerate}
\end{minipage}%
\begin{minipage}[t]{.5\textwidth}
	\begin{enumerate}\addtocounter{enumi}{2}
		\item	$\ds \lim_{x\to 1} f(x)$
		\item	$f(1)$
	\end{enumerate}
\end{minipage}}
{In this example, we will evaluate the limit by only considering the definition of $f$.
\begin{enumerate}
	\item	As $x$ approaches 1 from the left, $f(x)$ is defined to be $x^2$. Therefore \[\lim_{x\to1^-} f(x)=\lim_{x\to1^-} x^2=1.\]
	\item	As $x$ approaches 1 from the right, $f(x)$ is defined to be $2-x$. Therefore \[\lim_{x\to 1+} f(x)=\lim_{x\to 1+} 2-x=1.\]
	\item	Since the right and left hand limits are equal at $x=1$, i.e., $\ds \lim_{x\to1^-} f(x)=\ds \lim_{x\to1^+} f(x)=1$, this tells us $\ds \lim_{x\to1} f(x)=1$.
	\item	To find $f(1)$, we use the $x^2$ piece of our function, so $f(1)=1$.\eoehere
\end{enumerate}%
%It is clear from the definition of the function and its graph that all of the following are equal:
%\mfigure{0in}{Graphing $f$ in \autoref{ex_onesided}}{fig:onesidedd}{figures/figOneSidedLimits4}
%$$ \lim_{x\to 1^-} f(x) = \lim_{x\to 1^+} f(x) =\lim_{x\to 1} f(x) =f(1) = 1.$$
}

\example{ex_absvalue}{Evaluating limits of an absolute value function}
{Let $f(x) =\dfrac{\abs{x-1}}{x-1}.$ Evaluate the following.\\
\begin{minipage}[t]{.5\textwidth}
	\begin{enumerate}
		\item	$\ds \lim_{x\to 1^-} f(x)$
		\item	$\ds \lim_{x\to 1^+} f(x)$
	\end{enumerate}
\end{minipage}%
\begin{minipage}[t]{.5\textwidth}
	\begin{enumerate}\addtocounter{enumi}{2}
		\item	$\ds \lim_{x\to 1} f(x)$
		\item	$f(1)$
	\end{enumerate}
\end{minipage}}
{We begin by rewriting $\abs{x-1}$ as a piecewise function.
\[\abs{x-1}=\begin{cases}x-1 & x\geq 1 \\ -(x-1) & x\leq 1\end{cases}\]
\begin{enumerate}
\item	$\ds \lim_{x\to 1^-} f(x)=\lim_{x\to 1^-}\frac{-(x-1)}{x-1}=\lim_{x\to 1^-}-1=-1$
\item	$\ds \lim_{x\to 1^+} f(x)=\lim_{x\to 1^+}\frac{x-1}{x-1}=\lim_{x\to 1^+}1=1$
\item 	$\ds \lim_{x\to 1} f(x)$ does not exist because the left and right hand limits are not equal.
\item $f(1)$ is undefined.\eoehere
\end{enumerate}}

In Examples \ref{ex_onesidea} -- \ref{ex_absvalue} we were asked to find both $\ds \lim_{x\to 1}f(x)$ and $f(1)$. Consider the following table:
\begin{center}
\begin{tabular}{ccc}
 & $\ds \lim_{x\to 1}f(x)$ & $f(1)$ \\ \midrule
\autoref{ex_onesidea} & does not exist & 2 \\
\autoref{ex_onesideb} & 1 & not defined \\
\autoref{ex_onesidec} & 0 & 1 \\
\autoref{ex_onesided} & 1 & 1 \\
\autoref{ex_absvalue} & does not exist & not defined
\end{tabular}
\end{center}

Only in \autoref{ex_onesided} do both the function and the limit exist and agree. This seems ``nice;'' in fact, it seems ``normal.'' This is in fact an important situation which we explore in \autoref{sec:continuity}, entitled ``Continuity.'' In short, a \textit{continuous function} is one in which when a function approaches a value as $x\rightarrow c$ (i.e., when $\ds \lim_{x\to c} f(x) = L$), it actually \textit{attains} that value at $c$. Such functions behave nicely as they are very predictable.

In the next section we examine one more aspect of limits: limits that involve infinity.

\printexercises{exercises/01_04_exercises}
