\section{Implicit Differentiation}\label{sec:imp_deriv}

In the previous sections we learned to find the derivative, $ \frac{dy}{dx}$, or $y\primeskip'$, when $y$ is given \textit{explicitly} as a function of $x$. That is, if we know $y=f(x)$ for some function $f$, we can find $y\primeskip'$. For example, given  $y=3x^2-7$, we can easily find $y\primeskip'=6x$. (Here we explicitly state how $x$ and $y$ are related. Knowing $x$, we can directly find $y$.)

Sometimes the relationship between $y$ and $x$ is not explicit; rather, it is \textit{implicit}. For instance, we might know that $x^2-y=4$. This equality defines a relationship between $x$ and $y$; if we know $x$, we could figure out $y$. Can we still find $y\primeskip'$?  In this case, sure; we  solve for $y$ to get $y=x^2-4$ (hence we now know $y$ explicitly)  and then differentiate to get $y\primeskip'=2x$.

Sometimes the \textit{implicit} relationship between $x$ and $y$ is complicated.  Suppose we are given $\sin(y)+y^3=6-x^3$. A graph of this equation is given in \autoref{fig:implicit1}. In this case there is absolutely no way to solve for $y$ in terms of elementary functions.  The surprising thing is, however, that we can still find $y\primeskip'$ via a process known as \sword{implicit differentiation}.\index{implicit differentiation}\index{derivative!implicit}

\mtable{A graph of the equation $\sin (y)+y^3=6-x^3$.}{fig:implicit1}{\begin{tikzpicture}
\begin{axis}[width=1.16\marginparwidth,tick label style={font=\scriptsize},
minor x tick num=1,minor y tick num=1,axis y line=middle,axis x line=middle,
ymin=-3.2,ymax=3.2,xmin=-3.2,xmax=3.2,axis equal,name=myplot]
\addplot [draw={\colorone},smooth,thick] coordinates {(2.89,-2.61) (2.78,-2.46) (2.65,-2.28) (2.52,-2.09) (2.36,-1.82)(2.22,-1.57) (2.1,-1.33) (2.,-1.07) (1.94,-0.842) (1.89,-0.607)(1.85,-0.295) (1.82,-0.0424) (1.79,0.214) (1.76,0.471) (1.71,0.686)
(1.65,0.899) (1.61,1.) (1.51,1.18) (1.39,1.32) (1.29,1.42) (1.18,1.5)(1.07,1.55) (0.96,1.6) (0.794,1.65) (0.643,1.68) (0.416,1.7)(0.00338,1.71) (-0.214,1.71) (-0.423,1.72) (-0.643,1.74) (-0.947,1.8)(-1.28,1.93) (-1.49,2.04) (-1.67,2.14) (-1.83,2.24) (-2.07,2.43)(-2.31,2.61) (-2.52,2.79) (-2.68,2.92)};
\end{axis}
\node [right] at (myplot.right of origin) {\scriptsize $x$};
\node [above] at (myplot.above origin) {\scriptsize $y$};
\end{tikzpicture}}

Implicit differentiation is a technique based on the Chain Rule that is used to find a derivative when the relationship between the variables is given implicitly rather than explicitly (solved for one variable in terms of the other).\bigskip

We begin by reviewing the Chain Rule. Let $f$ and $g$ be functions of $x$. Then \[\frac{d}{dx}\Big(f(g(x))\Big) = \fp(g(x))\cdot g'(x).\]
Suppose now that $y=g(x)$. We can rewrite the above as
\begin{equation}
\frac{d}{dx}\Big(f(y)\Big) = \fp(y)\cdot y\primeskip', \quad \text{or}\quad
\frac{d}{dx}\Big(f(y)\Big) = \fp(y)\cdot \frac{dy}{dx} .\label{eq:implicit1}
\end{equation}
These equations look strange; the key concept to learn here is that we can find $y\primeskip'$ even if we don't exactly know how $y$ and $x$ relate.\\

\youtubeVideo{2CsQ_l1S2_Y}{Showing explicit and implicit differentiation give same result}

We demonstrate this process in the following example.

\example{ex_implicit1}{Using Implicit Differentiation}{Find $y\primeskip'$ given that $\sin(y) + y^3=6-x^3$.}
{We start by taking the derivative of both sides (thus maintaining the equality). We have:
\[\frac{d}{dx}\Big(\sin(y) + y^3\Big)=\frac{d}{dx}\Big(6-x^3\Big).\]
The right hand side is easy; it returns $-3x^2$. 

The left hand side requires more consideration. We take the derivative term-by-term.  Using the technique derived from \autoeqref{eq:implicit1} above, we can see that
\[\frac{d}{dx}\Big(\sin y\Big) = \cos y \cdot y\primeskip'.\] %The derivative of $\sin(y)$ is $\cos(y)y\primeskip'$.  The reason for this is the chain rule. Since $y$ itself depends on $x$, the term $\sin(y)$ is really a function inside of a function, $y$ being a function of $x$.

We apply the same process to the $y^3$ term. 
\[
\frac{d}{dx}\Big(y^3\Big) = \frac{d}{dx}\Big((y)^3\Big) = 3(y)^2\cdot y\primeskip'.
\]
%Similarly, the derivative of $y^3$ is $3y^2y\primeskip'$.  
Putting this together with the right hand side, we have
\[\cos(y)y\primeskip'+3y^2y\primeskip' = -3x^2.\]
Now solve for $y\primeskip'$.
\begin{align*}
	\cos(y)y\primeskip'+3y^2y\primeskip' 	&= -3x^2.\\
	\big(\cos y+3y^2\big)y\primeskip' &=	-3x^2\\
	y\primeskip'&=	\frac{-3x^2}{\cos y+3y^2}
\end{align*}

This equation for $y\primeskip'$ probably seems unusual for it contains both $x$ and $y$ terms. How is it to be used? We'll address that next.}

Implicit functions are generally harder to deal with than explicit functions. With an explicit function, given an $x$ value, we have an explicit formula for computing the corresponding $y$ value. With an implicit function, one often has to find $x$ and $y$ values \textit{at the same time} that satisfy the equation. It is much easier to demonstrate that a given point satisfies the equation than to actually find such a point.

For instance, we can affirm easily that the point $(\sqrt[3]{6},0)$ lies on the graph of the equation $\sin y + y^3=6-x^3$. Plugging in $0$ for $y$, we see the left hand side is $0$. Setting $x=\sqrt[3]6$, we see the right hand side is also $0$; the equation is satisfied. The following example finds an equation of the tangent line to this equation at this point.

\example{ex_implicit2}{Using Implicit Differentiation to find a tangent line}{%
Find the equation of the line tangent to the implicitly defined curve $\sin y + y^3=6-x^3$ at the point $(\sqrt[3]6,0)$.}%
{In \autoref{ex_implicit1} we found that
\[y\primeskip' = \frac{-3x^2}{\cos y +3y^2}.\]
We find the slope of the tangent line at the point  $(\sqrt[3]6,0)$ by substituting $\sqrt[3]6$ for $x$ and $0$ for $y$. Thus at the point $(\sqrt[3]6,0)$, we have the slope as
\[y\primeskip' = \frac{-3(\sqrt[3]{6})^2}{\cos 0 + 3\cdot0^2} = \frac{-3\sqrt[3]{36}}{1} \approx -9.91.\]

\mtable{The equation $\sin y+y^3 = 6-x^3$ and its tangent line at the point $(\sqrt[3]{6},0)$.}{fig:implicit2}{\begin{tikzpicture}
\begin{axis}[width=1.16\marginparwidth,tick label style={font=\scriptsize},
minor x tick num=1,minor y tick num=1,axis y line=middle,axis x line=middle,
ymin=-3.2,ymax=3.2,xmin=-3.2,xmax=3.2,name=myplot,axis equal]
\addplot [draw={\colorone},smooth,thick] coordinates {(2.89,-2.61) (2.78,-2.46) (2.65,-2.28) (2.52,-2.09) (2.36,-1.82)(2.22,-1.57) (2.1,-1.33) (2.,-1.07) (1.94,-0.842) (1.89,-0.607)(1.85,-0.295) (1.82,-0.0424) (1.79,0.214) (1.76,0.471) (1.71,0.686)
(1.65,0.899) (1.61,1.) (1.51,1.18) (1.39,1.32) (1.29,1.42) (1.18,1.5)(1.07,1.55) (0.96,1.6) (0.794,1.65) (0.643,1.68) (0.416,1.7)(0.00338,1.71) (-0.214,1.71) (-0.423,1.72) (-0.643,1.74) (-0.947,1.8)(-1.28,1.93) (-1.49,2.04) (-1.67,2.14) (-1.83,2.24) (-2.07,2.43)(-2.31,2.61) (-2.52,2.79) (-2.68,2.92)};
\addplot [draw={\colortwo},smooth,thick,domain=1.5:2.15] {-9.91*x+18};
\end{axis}
\node [right] at (myplot.right of origin) {\scriptsize $x$};
\node [above] at (myplot.above origin) {\scriptsize $y$};
\end{tikzpicture}}
Therefore an equation of the tangent line to the implicitly defined curve $\sin y + y^3=6-x^3$ at the point $(\sqrt[3]{6},0)$ is
\[y = -3\sqrt[3]{36}(x-\sqrt[3]{6})+0 \approx -9.91x+18.\]
The curve and this tangent line are shown in \autoref{fig:implicit2}.}


This suggests a general method for implicit differentiation.  For the steps below assume $y$ is a function of $x$.
\begin{enumerate}
\item Take the derivative of each term in the equation.  Treat the $x$ terms like normal.  When taking the derivatives of $y$ terms, the usual rules apply except that, because of the Chain Rule, we need to multiply each term by $y\primeskip'$.
\item Get all the $y\primeskip'$ terms on one side of the equal sign and put the remaining terms on the other side.
\item Factor out $y\primeskip'$;  solve for $y\primeskip'$ by dividing.
\end{enumerate}

\paragraph{Practical Note:}When working by hand, it may be beneficial to use the symbol $\frac{dy}{dx}$ instead of $y\primeskip'$, as the latter can be easily confused for $y$ or $y^1$.

\example{ex_implicit3}{Using Implicit Differentiation}{Given the implicitly defined function $y^3+x^2y^4=1+2x$, find $y\primeskip'$.}
{We will take the implicit derivatives term by term. Using the Chain Rule the derivative of $y^3$ is $3y^2 y\primeskip'$.

The second term, $x^2y^4$ is a little more work. It requires the Product Rule as it is the product of two functions of $x$: $x^2$ and $y^4$. We see that $\dfrac{d}{dx}(x^2y^4)$ is
\begin{gather*}
x^2 \cdot \frac{d}{dx}(y^4) + \frac{d}{dx}(x^2) \cdot y^4 \\
x^2 \cdot (4y^3y') + 2x \cdot y^4 
\end{gather*}
The first part of this expression requires a $y\primeskip'$ because we are taking the derivative of a $y$ term. The second part does not require it because we are taking the derivative of $x^2$.

The derivative of the right hand side of the equation is found to be $2$. In all, we get:
\[3y^2y\primeskip' + 4x^2y^3y\primeskip' + 2xy^4 = 2.\]

Move terms around so that the left side consists only of the $y\primeskip'$ terms and the right side consists of all the other terms:
\[3y^2y\primeskip' + 4x^2y^3y\primeskip' = 2-2xy^4.\]
Factor out $y\primeskip'$ from the left side and solve to get
\[y\primeskip' = \frac{2-2xy^4}{3y^2+4x^2y^3}.\]

To confirm the validity of our work, let's find the equation of a tangent line to this curve at a point. It is easy to confirm that the point $(0,1)$ lies on the graph of this curve. At this point, $y\primeskip' = 2/3$. So the equation of the tangent line is $y = 2/3(x-0)+1$. The equation and its tangent line are graphed in \autoref{fig:implicit4}.

\mtable{A graph of the equation $y^3+x^2y^4=1+2x$ along with its tangent line at the point $(0,1)$.}{fig:implicit4}{\begin{tikzpicture}
\begin{axis}[width=1.16\marginparwidth,tick label style={font=\scriptsize},
minor x tick num=1,minor y tick num=1,axis y line=middle,axis x line=middle,
ymin=-10.5,ymax=2.49,xmin=-1.5,xmax=10.5,name=myplot,axis equal]
\addplot [draw={\colorone},smooth,thick] coordinates {(-0.324,-9.34) (-0.336,-8.66) (-0.351,-7.95) (-0.371,-7.13)
(-0.396,-6.3) (-0.414,-5.68) (-0.451,-4.91) (-0.472,-4.48)(-0.497,-4.02) (-0.525,-3.57) (-0.56,-3.15) (-0.61,-2.63)(-0.664,-2.18) (-0.727,-1.71) (-0.762,-1.29) (-0.668,-0.805)(-0.513,-0.29) (-0.453,0.439) (-0.308,0.71) (0.0816,1.05)(0.452,1.15) (0.893,1.13) (1.33,1.08) (1.79,1.02) (2.24,0.975)(2.82,0.925) (3.4,0.881) (4.11,0.844) (4.91,0.781) (5.77,0.766)(6.76,0.74) (7.59,0.722) (8.26,0.706) (8.93,0.689) (9.69,0.671)
};
\addplot [draw={\colorone},smooth,thick] coordinates {(9.6,-0.677) (9.13,-0.696) (8.66,-0.697) (8.21,-0.707) (7.72,-0.722)(7.3,-0.738) (6.79,-0.754) (5.98,-0.778) (5.1,-0.81) (4.34,-0.858)(3.84,-0.893) (3.4,-0.927) (2.77,-0.99) (2.27,-1.07) (1.82,-1.17)(1.38,-1.34) (0.97,-1.7) (0.779,-2.1) (0.678,-2.5) (0.593,-3.)(0.534,-3.66) (0.483,-4.38) (0.444,-5.04) (0.415,-5.71) (0.393,-6.38)(0.375,-7.05) (0.362,-7.59) (0.343,-8.38) (0.334,-8.9) (0.322,-9.33)(0.317,-9.83) };
\addplot [draw={\colortwo},smooth,thick,domain=-1.4:2.2] {2/3*(x)+1};
\end{axis}
\node [right] at (myplot.right of origin) {\scriptsize $x$};
\node [above] at (myplot.above origin) {\scriptsize $y$};
\end{tikzpicture}}

Notice how our curve looks much different than other functions we have worked with up to this point.
% For one, it fails the vertical line test.
Such curves are important in many areas of mathematics, so developing tools to deal with them is also important.}

\example{ex_implicit5}{Using Implicit Differentiation}{Given the implicitly defined curve $\sin(x^2y^2)+y^3=x+y$, find $y\primeskip'$.}
{Differentiating term by term, we find the most difficulty in the first term.  It requires both the Chain and Product Rules.
\begin{align*}
	\frac{d}{dx}\left(\sin (x^2y^2)\right)
	&= \cos(x^2y^2) \cdot \frac{d}{dx} (x^2y^2)\\
	&= \cos(x^2y^2) \cdot \left(x^2 \frac{d}{dx} (y^2)
	+ \frac{d}{dx} (x^2) \cdot y^2\right)\\
	&= \cos(x^2y^2) \cdot (x^2 \cdot 2yy\primeskip'+2xy^2)\\
	&= 2(x^2yy\primeskip'+xy^2)\cos(x^2y^2). \\
\end{align*}

We leave the derivatives of the other terms to the reader. After taking the derivatives of both sides, we have
\[2(x^2yy\primeskip'+xy^2)\cos(x^2y^2) + 3y^2y\primeskip' = 1 + y\primeskip'.\]

\mtable{A graph of the equation $\sin(x^2y^2)+y^3=x+y$ and certain tangent lines.}{fig:implicit5}{\begin{tikzpicture}
\begin{axis}[width=1.16\marginparwidth,tick label style={font=\scriptsize},
minor x tick num=1,minor y tick num=1,axis y line=middle,axis x line=middle,
ymin=-1.95,ymax=1.95,xmin=-1.95,xmax=1.95,name=myplot,axis equal]
\addplot [draw={\colorone},smooth,thick] coordinates {(-1.91,-1.57) (-1.82,-1.61) (-1.68,-1.61) (-1.7,-1.49) (-1.74,-1.38)(-1.76,-1.28) (-1.64,-1.26) (-1.54,-1.29) (-1.36,-1.35) (-1.2,-1.42)(-1.09,-1.46) (-0.989,-1.49) (-0.86,-1.5) (-0.714,-1.45)(-0.566,-1.35) (-0.327,-1.18) (-0.15,-1.08)(0.107,-0.947) (0.216,-0.895) (0.332,-0.832)
(0.429,-0.755) (0.464,-0.623) (0.429,-0.52) (0.308,-0.335)(0.142,-0.144) (-0.071,0.0714) (-0.239,0.261) (-0.336,0.45)
(-0.336,0.622) (-0.278,0.757) (-0.17,0.884) (-0.046,0.975)(0.0714,1.03) (0.216,1.07) (0.48,1.09) (0.714,1.07) (0.929,1.05)(1.15,1.07) (1.24,1.13) (1.27,1.25) (1.28,1.37) (1.29,1.49)(1.36,1.58) (1.47,1.57) (1.59,1.52) (1.68,1.48) (1.79,1.43)(1.89,1.39) (1.98,1.36) (1.98,1.6)};
\end{axis}
\node [right] at (myplot.right of origin) {\scriptsize $x$};
\node [above] at (myplot.above origin) {\scriptsize $y$};
\end{tikzpicture}
\\ (a) \\
\begin{tikzpicture}
\begin{axis}[width=1.16\marginparwidth,tick label style={font=\scriptsize},minor x tick num=1,minor y tick num=1,axis y line=middle,axis x line=middle,ymin=-1.95,ymax=1.95,xmin=-1.95,xmax=1.95,name=myplot,axis equal]

\addplot [draw={\colorone},smooth,thick] coordinates
	{(-1.91,-1.57) (-1.82,-1.61) (-1.68,-1.61) (-1.7,-1.49) (-1.74,-1.38)
	 (-1.76,-1.28) (-1.64,-1.26) (-1.54,-1.29) (-1.36,-1.35) (-1.2,-1.42)
	 (-1.09,-1.46) (-0.989,-1.49) (-0.86,-1.5) (-0.714,-1.45) (-0.566,-1.35)
	 (-0.327,-1.18) (-0.15,-1.08) (0.107,-0.947) (0.216,-0.895) (0.332,-0.832)
	 (0.429,-0.755) (0.464,-0.623) (0.429,-0.52) (0.308,-0.335) (0.142,-0.144)
	 (-0.071,0.0714) (-0.239,0.261) (-0.336,0.45) (-0.336,0.622) (-0.278,0.757)
	 (-0.17,0.884) (-0.046,0.975)(0.0714,1.03) (0.216,1.07) (0.48,1.09) (0.714,1.07)
	 (0.929,1.05) (1.15,1.07) (1.24,1.13) (1.27,1.25) (1.28,1.37) (1.29,1.49)
	 (1.36,1.58) (1.47,1.57) (1.59,1.52) (1.68,1.48) (1.79,1.43) (1.89,1.39)
	 (1.98,1.36) (1.98,1.6)};

\addplot [draw={\colortwo},smooth,thick,domain=-.75:.75] {-x};
\addplot [draw={\colortwo},smooth,thick,domain=-.75:.75] {0.5*x+1};
\addplot [draw={\colortwo},smooth,thick,domain=-.75:.75] {0.5*x-1};
\end{axis}

\node [right] at (myplot.right of origin) {\scriptsize $x$};
\node [above] at (myplot.above origin) {\scriptsize $y$};
\end{tikzpicture}
\\ (b) }

We now have to be careful to properly solve for $y\primeskip'$, particularly because of the product on the left.  It is best to multiply out the product.  Doing this, we get
\[2x^2y\cos(x^2y^2)y\primeskip' + 2xy^2\cos(x^2y^2) + 3y^2y\primeskip' = 1 + y\primeskip'.\]
From here we can safely move around terms to get the following:
\[2x^2y\cos(x^2y^2)y\primeskip' + 3y^2y\primeskip' - y\primeskip' = 1 - 2xy^2\cos(x^2y^2).\]
Then we can solve for $y\primeskip'$ to get
\[y\primeskip' = \frac{1 - 2xy^2\cos(x^2y^2)}{2x^2y\cos(x^2y^2)+3y^2-1}.\]

A graph of this implicit equation is given in \autoref{fig:implicit5}(a). It is easy to verify that the points $(0,0)$, $(0,1)$ and $(0,-1)$ all lie on the graph. We can find the slopes of the tangent lines at each of these points using our formula for $y\primeskip'$. 

At $(0,0)$, the slope is $-1$.

At $(0,1)$, the slope is $1/2$.

At $(0,-1)$, the slope is also $1/2$.\\
The tangent lines have been added to the graph of the function in \autoref{fig:implicit5}(b).}

Quite a few ``famous'' curves have equations that are given implicitly.  We can use implicit differentiation to find the slope at various points on those curves. We investigate two such curves in the next examples.\bigskip

\example{ex_implicit7}{Finding slopes of tangent lines to a circle}{Find the slope of the tangent line to the circle $x^2+y^2=1$ at the point $(1/2, \sqrt{3}/2)$.}
{Taking derivatives, we get $2x+2yy\primeskip'=0$.  Solving for $y\primeskip'$  gives: \[\ds y\primeskip' = \frac{-x}{y}.\]
This is a clever formula. Recall that the slope of the line through the origin and the point $(x,y)$ on the circle will be $y/x$. We have found that the slope of the tangent line to the circle at that point is the opposite reciprocal of $y/x$, namely, $-x/y$. Hence these two lines are always perpendicular.

At the point $(1/2, \sqrt{3}/2)$, we have the tangent line's slope as
\[y\primeskip' = \frac{-1/2}{\sqrt{3}/2} = \frac{-1}{\sqrt{3}} \approx -0.577.\]

A graph of the circle and its tangent line at $(1/2,\sqrt{3}/2)$ is given in \autoref{fig:implicit7}, along with a thin dashed line from the origin that is perpendicular to the tangent line. (It turns out that all normal lines to a circle pass through the center of the circle.)

\mtable[-1in]{The unit circle with its tangent line at $(1/2,\sqrt{3}/2)$.}{fig:implicit7}{\begin{tikzpicture}
\begin{axis}[width=1.16\marginparwidth,tick label style={font=\scriptsize},
minor x tick num=1,minor y tick num=1,axis y line=middle,axis x line=middle,
ymin=-1.2,ymax=1.2,xmin=-1.2,xmax=1.2,name=myplot,axis equal]
\addplot [draw={\colorone},smooth,thick,domain=0:360] ({cos(x)},{sin(x)});
\addplot [draw={\colortwo},smooth,thick,domain=.1:.9] {-.577*(x-.5)+.866};
\filldraw [black] (axis cs:.5,.866) node [above right] {\scriptsize $(1/2,\sqrt{3}/2)$} circle (1pt);
\draw [thin,draw={\colortwo},dashed] (axis cs:0,0) -- (axis cs:.5,.866);
\end{axis}
\node [right] at (myplot.right of origin) {\scriptsize $x$};
\node [above] at (myplot.above origin) {\scriptsize $y$};
\end{tikzpicture}}}

This section has shown how to find the derivatives of implicitly defined functions, whose graphs include a wide variety of interesting and unusual shapes. Implicit differentiation can also be used to further our understanding of ``regular'' differentiation. 

%One hole in our current understanding of derivatives is this: what is the derivative of the square root function? That is,
%\[\frac{d}{dx}\big(\sqrt{x}\big) = \frac{d}{dx}\big(x^{1/2}\big) = \text{?}\]
%
%We allude to a possible solution, as we can write the square root function as a power function with a rational (or, fractional) power. We are then tempted to apply the Power Rule and obtain
%\[\frac{d}{dx}\big(x^{1/2}\big) = \frac12x^{-1/2} = \frac{1}{2\sqrt{x}}.\]
%
%The trouble with this is that the Power Rule was initially defined only for positive integer powers, $n>0$. While we did not justify this at the time, generally the Power Rule is proved using something called the Binomial Theorem, which deals only with positive integers. The Quotient Rule allowed us to extend the Power Rule to negative integer powers. Implicit Differentiation allows us to extend the Power Rule to rational powers, as shown below.
%
%Let $y = x^{m/n}$, where $m$ and $n$ are integers with no common factors (so $m=2$ and $n=5$ is fine, but $m=2$ and $n=4$ is not). We can rewrite this explicit function implicitly as $y^n = x^m$. Now apply implicit differentiation.
%
%\begin{align*}
%	y &= x^{m/n} \\
%	y^n &= x^m \\
%	\frac{d}{dx}\big(y^n\big) &= \frac{d}{dx}\big(x^m\big) \\
%	n\cdot y^{n-1}\cdot y\primeskip' &= m\cdot x^{m-1} \\
%	y\primeskip' &= \frac{m}{n} \frac{x^{m-1}}{y^{n-1}} && \mbox{\small (now substitute $x^{m/n}$ for $y$)} \\
%	&= \frac{m}{n} \frac{x^{m-1}}{(x^{m/n})^{n-1}} && \mbox{\small (apply lots of algebra)}\\
%%	&= \frac{m}{n} \frac{x^{m-1}}{x^{m(n-1)/n}} \\
%%	&=	\frac{m}n	x^{(m-1)-m(n-1)/n} \\
%%	&=	\frac{m}n x^{((m-1)n-m(n-1))/n} \\
%	&= \frac{m}n x^{(m-n)/n}\\
%	&= \frac{m}n x^{m/n -1}.
%\end{align*}
%
%The above derivation is the key to the proof extending the Power Rule to rational powers. Using limits, we can extend this once more to include \textit{all} powers, including irrational (even transcendental!) powers, giving the following theorem.
%
%\theorem{thm:finalpower}{Power Rule for Differentiation}
%{Let $f(x) = x^n$, where $n\neq 0$ is a real number. Then $f$ is a differentiable function, and $\fp(x) = n\cdot x^{n-1}$.\index{derivative!Power Rule}\index{Power Rule!differentiation}}
%
%This theorem allows us to say the derivative of $x^\pi$ is $\pi x^{\pi -1}$. 
%
%We now apply this final version of the Power Rule in the next example, the second investigation of a ``famous'' curve.\\
%
%\example{ex_implicit8}{Using the Power Rule}{Find the slope of $x^{2/3}+y^{2/3}=8$ at the point $(8,8)$.}
%{This is a particularly interesting curve called an \emph{astroid}.  It is the shape traced out by a point on the edge of a circle that is rolling around inside of a larger circle, as shown in \autoref{fig:implicit8}(a).
%
%\mtable{An astroid, traced out by a point on the smaller circle as it rolls inside the larger circle, along with a tangent line.}{fig:implicit8}{
%\begin{tikzpicture}
%\begin{axis}[width=1.16\marginparwidth,tick label style={font=\scriptsize},
%minor x tick num=1,minor y tick num=1,axis y line=middle,axis x line=middle,
%ymin=-24,ymax=24,xmin=-24,xmax=24,name=myplot,equal axis]
%\addplot [draw={\colorone},smooth,thick,domain=0:360,samples=52] ({22.63*cos(x)^3},{22.63*sin(x)^3});
%\addplot [black,smooth,thick,domain=0:360] ({22.63*cos(x)},{22.63*sin(x)});% {};
%\addplot [draw={\colortwo},smooth,thick,domain=0:360] ({5.66*cos(x)-12},{5.66*sin(x)-12});
%\filldraw [black] (axis cs:-8,-8) circle (1pt);
%\end{axis}
%\node [right] at (myplot.right of origin) {\scriptsize $x$};
%\node [above] at (myplot.above origin) {\scriptsize $y$};
%\end{tikzpicture}
%\\(a) \\
%\begin{tikzpicture}
%\begin{axis}[width=1.16\marginparwidth,tick label style={font=\scriptsize},
%minor x tick num=1,minor y tick num=1,axis y line=middle,axis x line=middle,
%ymin=-24,ymax=24,xmin=-24,xmax=24,name=myplot,equal axis]
%\addplot [draw={\colorone},smooth,thick,domain=0:360,samples=52] ({22.63*cos(x)^3},{22.63*sin(x)^3});
%\addplot [draw={\colortwo},smooth,thick,domain=1:15] {-1*(x-8)+8};
%\filldraw [black] (axis cs:8,8) node [shift={(10pt,8pt)}] {\scriptsize $(8,8)$} circle (1pt);
%\end{axis}
%\node [right] at (myplot.right of origin) {\scriptsize $x$};
%\node [above] at (myplot.above origin) {\scriptsize $y$};
%\end{tikzpicture}
%\\(b)}
%
%
%To find the slope of the astroid at the point $(8,8)$, we take the derivative implicitly.
%\begin{align*}
%	\frac23x^{-1/3}+\frac23y^{-1/3}y\primeskip'&=0\\	
%	\frac23y^{-1/3}y\primeskip' &= -\frac23x^{-1/3}\\
%	y\primeskip'&=	-\frac{x^{-1/3}}{y^{-1/3}}\\
%	y\primeskip'&=	-\frac{y^{1/3}}{x^{1/3}} = -\sqrt[3]{\frac{y}x}.
%\end{align*}
%
%Plugging in $x=8$ and $y=8$, we get a slope of $-1$. The astroid, with its tangent line at $(8,8)$, is shown in \autoref{fig:implicit8}(b).}

\subsection{Implicit Differentiation and the Second Derivative}

We can use implicit differentiation to find higher order derivatives. In theory, this is simple: first find $\frac{dy}{dx}$, then take its derivative with respect to $x$. In practice, it is not hard, but it often requires a bit of algebra. We demonstrate this in an example.

\example{ex_implicit9}{Finding the second derivative}{Given $x^2+y^2=1$, find $\ds\frac{d^2y}{dx^2} = y\primeskip''$. }
{We found that $y\primeskip' = \frac{dy}{dx} = -x/y$ in \autoref{ex_implicit7}. To find $y\primeskip''$, we apply implicit differentiation to $y\primeskip'$.
\begin{align*}
	y\primeskip''
	&= \frac{d}{dx}\big(y\primeskip'\big) \\
	&= \frac{d}{dx}\left(-\frac xy\right) && \text{\small now use the Quotient Rule} \\
	&= -\frac{y(1) - x(y\primeskip')}{y^2} && \text{\small replace $y\primeskip'$ with $-x/y$} \\
	&= -\frac{y-x(-x/y)}{y^2}\\
%	&= -\frac{y+x^2/y}{y^2} \\
%	&= -\frac{y+x^2/y}{y^2}\cdot\frac yy \\
	&= -\frac{y^2+x^2}{y^3}, && \text{\small since we were given $x^2+y^2=1$} \\
	&= -\frac1{y^3}.
\end{align*}
%While this is not a particularly simple expression, it is usable.
We can see that $y\primeskip''>0$ when $y<0$ and $y\primeskip''<0$ when $y>0$. In \autoref{sec:concavity}, we will see how this relates to the shape of the graph.}

% todo write a better section summary for implicit differentiation
Implicit differentiation proves to be useful as it allows us to find the instantaneous rates of change of a variety of functions.
% In particular, it extended the Power Rule to rational exponents, which we then extended to all real numbers.
% In \autoref{sec:deriv_inverse_function}, implicit differentiation will be used to find the derivatives of \textit{inverse} functions, such as $y=\sin^{-1} x$.
We close with a small gallery of ``interesting'' and ``famous'' curves along with the implicit equations that produce them.

\noindent\begin{minipage}[t]{\linewidth}\noindent%
\captionsetup{type=figure}%
\flushinner{%
 \begin{tabular}{ccc}
 % astroid
\begin{tikzpicture}
\begin{axis}[width=1.16\marginparwidth,tick label style={font=\scriptsize},
axis y line=middle,axis x line=middle,name=myplot,xtick={-1,1},ytick={-1,1},
ymin=-1.1,ymax=1.1,xmin=-1.1,xmax=1.1,axis equal]
\addplot [thick,draw={\colorone}, smooth,domain=0:360,samples=360]
 ({(cos(x))^3},{(sin(x))^3});
\end{axis}
\node [right] at (myplot.right of origin) {\scriptsize $x$};
\node [above] at (myplot.above origin) {\scriptsize $y$};
\end{tikzpicture}
  &
  % fattened circle with n=2
\begin{tikzpicture}
\begin{axis}[width=1.16\marginparwidth,tick label style={font=\scriptsize},
axis y line=middle,axis x line=middle,name=myplot,xtick={-1,1},ytick={-1,1},
ymin=-1.1,ymax=1.1,xmin=-1.1,xmax=1.1,axis equal]
\addplot [thick,draw={\colorone}, smooth,domain=0:90,samples=90]
 ({sqrt(cos(x))},{sqrt(sin(x))});
\addplot [thick,draw={\colorone}, smooth,domain=0:90,samples=90]
 ({-sqrt(cos(x))},{sqrt(sin(x))});
\addplot [thick,draw={\colorone}, smooth,domain=0:90,samples=90]
 ({-sqrt(cos(x))},{-sqrt(sin(x))});
\addplot [thick,draw={\colorone}, smooth,domain=0:90,samples=90]
 ({sqrt(cos(x))},{-sqrt(sin(x))});
\end{axis}
\node [right] at (myplot.right of origin) {\scriptsize $x$};
\node [above] at (myplot.above origin) {\scriptsize $y$};
\end{tikzpicture}
   &
   % Elliptic curve y^2=x^3+ax+b with a=-2, b=2
\begin{tikzpicture}
\begin{axis}[width=1.16\marginparwidth,tick label style={font=\scriptsize},
axis y line=middle,axis x line=middle,name=myplot,xtick={-2,2},ytick={-2,2},
ymin=-3.1,ymax=3.1,xmin=-3.1,xmax=3.1,axis equal]
\addplot [thick,draw={\colorone}, smooth,domain=-1.769:3.1,samples=40]
 {sqrt(x*x*x-2*x+2)};
\addplot [thick,draw={\colorone}, smooth,domain=-1.769:3.1,samples=40]
 {-sqrt(x*x*x-2*x+2)};
\end{axis}
\node [right] at (myplot.right of origin) {\scriptsize $x$};
\node [above] at (myplot.above origin) {\scriptsize $y$};
\end{tikzpicture}
   \\
  \parbox{150pt}{\centering Astroid\\$x^{2/3}+y^{2/3}=1$} &
  \parbox{150pt}{\centering Fattened circle\\$x^{2n}+y^{2n}=1$} &
  \parbox{150pt}{\centering Elliptic curve\\$y^2=x^3+ax+b$} \bigskip\\
  % Cassini ovals with a=1, b=1.1
\begin{tikzpicture}
\begin{axis}[width=1.16\marginparwidth,tick label style={font=\scriptsize},
axis y line=middle,axis x line=middle,name=myplot,
ymin=-1,ymax=1,xmin=-1.6,xmax=1.6,axis equal]
\addplot[thick,draw={\colorone},smooth,domain=0:360,samples=90]
 ({cos(x)*sqrt(cos(2*x)+sqrt(1.1^4-(sin(2*x))^2))},
  {sin(x)*sqrt(cos(2*x)+sqrt(1.1^4-(sin(2*x))^2))});
  % polar form courtesy of http://mathworld.wolfram.com/CassiniOvals.html
\end{axis}
\node [right] at (myplot.right of origin) {\scriptsize $x$};
\node [above] at (myplot.above origin) {\scriptsize $y$};
\end{tikzpicture}
  &
  % devil's curve with a=1, b=.8
  \begin{tikzpicture}
   \begin{axis}[width=\marginparwidth,tick label style={font=\scriptsize},
                axis y line=middle,axis x line=middle,name=myplot,
                ymin=-1.5,ymax=1.5,xmin=-1.5,xmax=1.5,axis equal]
    \addplot [thick,draw={\colorone}, smooth,domain=-.8:.8,samples=40]
     {-sqrt(5-sqrt(100*x^4-64*x^2+25))/sqrt(10)};
    \addplot [thick,draw={\colorone}, smooth,domain=-.8:.8,samples=40]
     {sqrt(5-sqrt(100*x^4-64*x^2+25))/sqrt(10)};
    \addplot [thick,draw={\colorone}, smooth,domain=-1.5:1.5,samples=40]
     {-sqrt(5+sqrt(100*x^4-64*x^2+25))/sqrt(10)};
    \addplot [thick,draw={\colorone}, smooth,domain=-1.5:1.5,samples=40]
     {sqrt(5+sqrt(100*x^4-64*x^2+25))/sqrt(10)};
   \end{axis}
   \node [right] at (myplot.right of origin) {\scriptsize $x$};
   \node [above] at (myplot.above origin) {\scriptsize $y$};
  \end{tikzpicture}
  &
  % folium of Descartes
  \begin{tikzpicture}
   \begin{axis}[width=\marginparwidth,tick label style={font=\scriptsize},
                axis y line=middle,axis x line=middle,name=myplot,
                ymin=-2.1,ymax=2.1,xmin=-2.1,xmax=2.1,axis equal]
    \addplot [thick,draw={\colorone}, smooth,domain=-40:130,samples=20]
     ({3*sin(x)*cos(x)^2/((sin(x))^3+(cos(x))^3)},{3*sin(x)^2*cos(x)/((sin(x))^3+(cos(x))^3)});
   \end{axis}
   \node [right] at (myplot.right of origin) {\scriptsize $x$};
   \node [above] at (myplot.above origin) {\scriptsize $y$};
  \end{tikzpicture} \\
  \parbox{150pt}{\centering Cassini ovals\\$((x-a)^2+y^2)((x+a)^2+y^2)=b^4$} &
  \parbox{150pt}{\centering Devil's curve\\$y^2(y^2-a^2)=x^2(x^2-b^2)$} &
  \parbox{150pt}{\centering Folium of Descartes\\$x^3+y^3=3axy$}
 \end{tabular}}
\end{minipage}\bigskip

The astroid and folium of Descartes also appear in \autoref{sec:param_eqs}.  Other important implicit curves are the conic sections, which are discussed in \autoref{sec:conic_sections}.  Also, the lemniscate, cardioid, and lima\c con seen in \autoref{sec:polar} have particularly nice implicit representations.\bigskip

% other possibilities
% trident curve?
% Folium curve??
% Kampyle of Eudoxus is polar
% Tschirnhausen cubic is polar
% conchoid is polar


In this chapter we have defined the derivative, given rules to facilitate its computation, and given the derivatives of a number of standard functions. We restate the most important of these in the following theorem, intended to be a reference for further work.

\theorem{thm:deriv_glossary}{Glossary of Derivatives of Elementary Functions}
{Let $u$ and $v$ be differentiable functions, and let $c$ and $n$ be real numbers, $n\neq 0$. \\
\begin{anywhereenum}
\renewcommand{\arraystretch}{1.6}
\begin{tabular}{ll}
	\item		$\frac{d}{dx}\big(cu\big) = cu'$ &
	\item		$\frac{d}{dx}\big(u\pm v\big) = u'\pm v'$ \\
	\item		$\frac{d}{dx}\big(u\cdot v\big) = uv'+u'v$ &
	\item		$\frac{d}{dx}\big(\frac uv\big) = \frac{u'v-uv'}{v^2}$ \\
	\item		$\frac{d}{dx}\big(u(v)\big) = u'(v)v'$ &
	\item		$\frac{d}{dx}\big(x^n\big) = nx^{n-1}$ \\
	\item		$\frac{d}{dx}\big(c\big) = 0$ &
	\item		$\frac{d}{dx}\big(x\big) = 1$ \\
	\item		$\frac{d}{dx}\big(\ln x\big) = \frac{1}{x}$ &
	\item		$\frac{d}{dx}\big(e^x\big) = e^x$ \\
	\item		$\frac{d}{dx}\big(\sin x\big) = \cos x$ &
	\item		$\frac{d}{dx}\big(\cos x\big) = -\sin x$ \\
	\item		$\frac{d}{dx}\big(\tan x\big) = \sec^2x$ &
	\item		$\frac{d}{dx}\big(\cot x\big) = -\csc^2x$ \\
	\item		$\frac{d}{dx}\big(\sec x\big) = \sec x\tan x$\qquad\null &
	\item		$\frac{d}{dx}\big(\csc x\big) = -\csc x\cot x$
%	\item		$\frac{d}{dx}\big(a^x\big) = \ln a\cdot a^x$ \\
%	\item		$\frac{d}{dx}\big(\log_a x\big) = \frac{1}{\ln a}\cdot\frac{1}{x}$ &
%	\item		$\frac{d}{dx}\big(\sin^{-1}x\big) = \frac{1}{\sqrt{1-x^2}}$ &
%	\item		$\frac{d}{dx}\big(\csc^{-1}x\big) = -\frac{1}{\abs{x}\sqrt{x^2-1}}$ \\
%	\item		$\frac{d}{dx}\big(\tan^{-1}x\big) = \frac{1}{1+x^2}$ &
%	\item		$\frac{d}{dx}\big(\cos^{-1}x\big) = -\frac{1}{\sqrt{1-x^2}}$
%	\item		$\frac{d}{dx}\big(\sec^{-1}x\big) = \frac{1}{\abs{x}\sqrt{x^2-1}}$
%	\item		$\frac{d}{dx}\big(\cot^{-1}x\big) = -\frac{1}{1+x^2}$
\end{tabular}
\end{anywhereenum}}

\printexercises{exercises/02_06_exercises}
