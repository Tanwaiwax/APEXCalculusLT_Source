\section{Extreme Values}\label{sec:extreme_values}

\mnote{\textbf{Note:} The extreme values of a function are ``$y$'' values, values the function attains, not the input values.}

Given any quantity described by a function, we are often interested in the largest and/or smallest values that quantity attains. For instance, if a function describes the speed of an object, it seems reasonable to want to know the fastest/slowest the object traveled. If a function describes the value of a stock, we might want to know the highest/lowest values the stock attained over the past year. We call such values \textit{extreme values.}

\begin{definition}[Extreme Values]\label{def:extreme_values}
Let $f$ be defined on an interval $I$ containing $c$.\index{extreme values}\index{absolute minimum}\index{minimum!absolute}\index{absolute maximum}\index{maximum!absolute}
\begin{enumerate}
	\item	$f(c)$ is the \textbf{minimum} (also, \textbf{absolute minimum}) of $f$ on $I$ if $f(c) \leq f(x)$ for all $x$ in $I$.
	\item	$f(c)$ is the \textbf{maximum} (also, \textbf{absolute maximum}) of $f$ on $I$ if $f(c) \geq f(x)$ for all $x$ in $I$.
\end{enumerate}
The maximum and minimum values are the \textbf{extreme values}, or \textbf{extrema}, of $f$ on $I$.\index{extrema!absolute}
\end{definition}

\mtable[-2.5in]{Graphs of functions with and without extreme values.}{fig:extreme}{\begin{tikzpicture}
\begin{axis}[width=1.16\marginparwidth,tick label style={font=\scriptsize},
minor x tick num=1, axis y line=middle,axis x line=middle,
ymin=-.9,ymax=5.5,xmin=-2.2,xmax=2.2,name=myplot]
\addplot [thick,draw={\colorone},smooth,domain=-2:2] {-x^2+5};
\filldraw [thick,draw={\colorone},fill=white] (axis cs:2,1) circle (1.5pt);
\filldraw [thick,draw={\colorone},fill=white] (axis cs:-2., 1) circle (1.5pt);
\draw [thick] (axis cs:-2,0) node {\textbf{(}};
\draw [thick] (axis cs:2,0) node {\textbf{)}};
\draw [ultra thick] (axis cs:-2,0) -- (axis cs:2,0);
\end{axis}
\node [right] at (myplot.right of origin) {\scriptsize $x$};
\node [above] at (myplot.above origin) {\scriptsize $y$};
\end{tikzpicture}
\\(a)\\
\begin{tikzpicture}
\begin{axis}[width=1.16\marginparwidth,tick label style={font=\scriptsize},
minor x tick num=1, axis y line=middle,axis x line=middle,
ymin=-.9,ymax=5.5,xmin=-2.2,xmax=2.2,name=myplot]
\addplot [thick,draw={\colorone},smooth,domain=-2:2] {-x^2+5};
\filldraw [thick,draw={\colorone}] (axis cs:2,1) circle (1.5pt);
\filldraw [thick,draw={\colorone}] (axis cs:-2., 1) circle (1.5pt);
\filldraw [thick,draw={\colorone}] (axis cs:0,3) circle (1.5pt);
\filldraw [thick,draw={\colorone},fill=white] (axis cs:0,5) circle (1.5pt);
\draw [thick] (axis cs:-2,0) node {\textbf{[}};
\draw [thick] (axis cs:2,0) node {\textbf{]}};
\draw [ultra thick] (axis cs:-2,0) -- (axis cs:2,0);
\end{axis}
\node [right] at (myplot.right of origin) {\scriptsize $x$};
\node [above] at (myplot.above origin) {\scriptsize $y$};
\end{tikzpicture}
\\(b)\\
\begin{tikzpicture}
\begin{axis}[width=1.16\marginparwidth,tick label style={font=\scriptsize},
minor x tick num=1, axis y line=middle,axis x line=middle,
ymin=-.9,ymax=5.5,xmin=-2.2,xmax=2.2,name=myplot]
\addplot [thick,draw={\colorone},smooth,domain=-2:2] {-x^2+5};
\filldraw [thick,draw={\colorone}] (axis cs:2,1) circle (1.5pt);
\filldraw [thick,draw={\colorone}] (axis cs:-2., 1) circle (1.5pt);
\draw [thick] (axis cs:-2,0) node {\textbf{[}};
\draw [thick] (axis cs:2,0) node {\textbf{]}};
\draw [ultra thick] (axis cs:-2,0) -- (axis cs:2,0);
\end{axis}
\node [right] at (myplot.right of origin) {\scriptsize $x$};
\node [above] at (myplot.above origin) {\scriptsize $y$};
\end{tikzpicture}
\\(c)}

Consider \autoref{fig:extreme}. The function displayed in (a) has a maximum, but no minimum, as the interval over which the function is defined is open. In (b), the function has a minimum, but no maximum; there is a discontinuity in the ``natural'' place for the maximum to occur. Finally, the function shown in (c) has both a maximum and a minimum; note that the function is continuous and the interval on which it is defined is closed. 
 
It is possible for discontinuous functions defined on an open interval to have both a maximum and minimum value, but we have just seen examples where they did not. On the other hand, continuous functions on a closed interval \textit{always} have a maximum and minimum value.
 
\begin{theorem}[The Extreme Value Theorem]\label{thm:extremeVal}
Let $f$ be a continuous function defined on a finite closed interval $I$. Then $f$ has both a maximum and minimum value on $I$.%\hfill\hyperref[pf:extremeVal]{(See the proof.)}
\index{Extreme Value Theorem}
\end{theorem}

%As with \autoref{thm:IVT}, this theorem seems obvious but depends on there being enough real numbers to find the extrema.  To see that it does not work with the rational numbers, we can consider the function $f(x)=x^3-6x$ on $[-2,2]$.

This theorem states that $f$ has extreme values, but it does not offer any advice about how/where to find these values. The process can seem to be fairly easy, as the next example illustrates. After the example, we will draw on lessons learned to form a more general and powerful method for finding extreme values.

\youtubeVideo{3-6bdDXzl9M}{Finding Critical Numbers --- Example 2}

\begin{example}[Approximating extreme values]\label{ex_extval1}
Consider $f(x) = 2x^3-9x^2$ on $I=[-1,5]$, as graphed in \autoref{fig:extval1}. Approximate the extreme values of $f$.
%
\mtable{A graph of $f(x) = 2x^3-9x^2$ as in \autoref{ex_extval1}.}{fig:extval1}{\begin{tikzpicture}
\begin{axis}[width=1.16\marginparwidth,tick label style={font=\scriptsize},
minor x tick num=5,xtick={-1,5}, axis y line=middle,axis x line=middle,
ymin=-31,ymax=31,xmin=-1.5,xmax=5.5,name=myplot]
\addplot [thick,draw={\colorone},smooth,domain=-1:5] {2*x^3-9*x^2};
\filldraw [thick,draw={\colorone}] (axis cs:5,25) node [black,above] {\tiny$(5,25)$} circle (1pt);
\filldraw [thick,draw={\colorone}] (axis cs:-1., -11) circle (1pt);
\filldraw [draw={\colorone}] (axis cs: 3,-27) node [black,shift={(0,9pt)}] {\tiny$(3,-27)$} circle (1pt);
\filldraw [draw={\colorone}] (axis cs: -1,-11) node [black,shift={(4pt,-5pt)}] {\tiny$(-1,-11)$} circle (1pt) ;
\filldraw [draw={\colorone}] (axis cs: 0,0) node [black,above right] {\tiny$(0,0)$} circle (1pt); 
\end{axis}
\node [right] at (myplot.right of origin) {\scriptsize $x$};
\node [above] at (myplot.above origin) {\scriptsize $y$};
\end{tikzpicture}}
\solution
The graph is drawn in such a way to draw attention to certain points. It certainly seems that the smallest $y$ value is $-27$, found when $x=3$. It also seems that the largest $y$ value is 25, found at the endpoint of $I$, $x=5$. We use the word \textit{seems}, for by the graph alone we cannot be sure the smallest value is not less than $-27$. Since the problem asks for an approximation, we approximate the extreme values to be $25$ and $-27$.
\end{example}

Notice how the minimum value came at ``the bottom of a valley,'' and the maximum value came at an endpoint. Also note that while $0$ is not an extreme value, it would be if we narrowed our interval to $[-1,4]$. The idea that the point $(0,0)$ is the location of an extreme value for some interval is important, leading us to a definition.

\begin{definition}[Relative Minimum and Relative Maximum]\label{def:rel_ext}
Let $f$ be defined on an interval $I$ containing $c$.\index{minimum!relative/local} \index{maximum!relative/local}\index{extrema!relative}
\begin{enumerate}
	\item	If there is an open interval containing $c$ such that $f(c)$ is the minimum value of $f$ on that interval, then $f(c)$ is a \textbf{relative minimum} of $f$. We also say that $f$ has a relative minimum at $(c,f(c))$.
	\item	If there is an open interval containing $c$ such that $f(c)$ is the maximum value of $f$ on that interval, then $f(c)$ is a \textbf{relative maximum} of $f$. We also say that $f$ has a relative maximum at $(c,f(c))$.
\end{enumerate}

The relative maximum and minimum values comprise the \textbf{relative extrema} of $f$.
\end{definition}

\mnote[-1in]{\textbf{Note:} The terms \textit{local minimum}, \textit{local maximum}, and \textit{local extrema} are often used as synonyms for \textit{relative minimum}, \textit{relative maximum}, and \textit{relative extrema}.}

We briefly practice using these definitions.

\mtable{A graph of $f(x) = (3x^4-4x^3-12x^2+5)/5$ as in \autoref{ex_extval2}.}%
{fig:extval2}{\begin{tikzpicture}
\begin{axis}[width=1.16\marginparwidth,tick label style={font=\scriptsize},
axis y line=middle,axis x line=middle,name=myplot,xtick={-2,-1,1,2,3},
ytick={-6,-4,-2,0,2,4,6},minor y tick num=1,ymin=-6.5,ymax=6.9,xmin=-2.2,xmax=3.2]
\addplot [thick,draw={\colorone},smooth,domain=-2:3] {(3*x^4-4*x^3-12*x^2+5)/5};
\end{axis}
\node [right] at (myplot.right of origin) {\scriptsize $x$};
\node [above] at (myplot.above origin) {\scriptsize $y$};
\end{tikzpicture}}

\begin{example}[Approximating relative extrema]\label{ex_extval2}
Consider $f(x) = (3x^4-4x^3-12x^2+5)/5$, as shown in \autoref{fig:extval2}. Approximate the relative extrema of $f$. At each of these points, evaluate $\fp$.
\solution
We still do not have the tools to exactly find the relative extrema, but the graph does allow us to make reasonable approximations. It seems $f$ has relative minima at $x=-1$ and $x=2$, with values of $f(-1)=0$ and $f(2) = -5.4$. It also seems that $f$ has a relative maximum at the point $(0,1)$. 

We approximate the relative minima to be $0$ and $-5.4$; we approximate the relative maximum to be $1$.

It is straightforward to evaluate $\fp(x) =\frac15(12x^3-12x^2-24x)$ at $x=0, 1$ and $2$. In each case, $\fp(x) = 0$.
\end{example}

\mtable{A graph of $f(x) = (x-1)^{2/3}+2$ as in \autoref{ex_extval3}.}{fig:extval3}{\begin{tikzpicture}
\begin{axis}[width=1.16\marginparwidth,tick label style={font=\scriptsize},
axis y line=middle,axis x line=middle,name=myplot,
ymin=-.5,ymax=3.5,xmin=-.6,xmax=2.6]
\addplot [thick,draw={\colorone},domain=-1.144:1.144] ({x*x*x+1},{x*x+2});
\end{axis}
\node [right] at (myplot.right of origin) {\scriptsize $x$};
\node [above] at (myplot.above origin) {\scriptsize $y$};
\end{tikzpicture}}

\begin{example}[Approximating relative extrema]\label{ex_extval3}
Approximate the relative extrema of $f(x) = (x-1)^{2/3}+2$, shown in \autoref{fig:extval3}. At each of these points, evaluate $\fp$.
\solution
The figure implies that $f$ does not have any relative maxima, but has a relative minimum at $(1,2)$. In fact, the graph suggests that not only is this point a relative minimum, $y=f(1)=2$ is \textit{the} minimum value of the function.

We compute $\fp(x) = \frac23(x-1)^{-1/3}$. When $x=1$, $\fp$ is undefined.
\end{example}

What can we learn from the previous two examples? We were able to visually approximate relative extrema, and at each such point, the derivative was either 0 or it was not defined. This observation holds for all functions, leading to a definition and a theorem.

\begin{definition}[Critical Numbers and Critical Points]\label{def:criticalnum}
Let $f$ be defined at $c$. The value $c$ is a \textbf{critical point} (or \textbf{critical number}) of $f$ if $\fp(c)=0$ or $\fp(c)$ is not defined.\index{critical number}\index{critical point}\index{critical value}\bigskip

If $c$ is a critical number of $f$, then $f(c)$ is a \textbf{critical value} of $f$.
\end{definition}

\begin{theorem}[Fermat's Theorem]\label{thm:criticalpts}
Let a function $f$ have a relative extrema at the point $(c,f(c))$. Then $c$ is a critical point of $f$.
\end{theorem}%\hfill\hyperref[pf:criticalpts]{(See the proof.)}

It isn't too hard to see why this should be true.  If $\fp$ is defined at a relative extreme, then the tangent line must be horizontal.  Otherwise, we'd be able to move along the graph in the direction given by the tangent line to get a more extreme value.

\mtable{A graph of $f(x)=x^3$ which has a critical point of $x=0$, but no relative extrema.}{fig:extreme4}{\begin{tikzpicture}
\begin{axis}[width=1.16\marginparwidth,tick label style={font=\scriptsize},
axis y line=middle,axis x line=middle,name=myplot,xtick={-1,1},ytick={-1,1},
ymin=-1.1,ymax=1.1,xmin=-1.1,xmax=1.1]
\addplot [thick,draw={\colorone},domain=-1:1] {x^3};
\end{axis}
\node [right] at (myplot.right of origin) {\scriptsize $x$};
\node [above] at (myplot.above origin) {\scriptsize $y$};
\end{tikzpicture}}

Be careful to understand that this theorem states  ``All relative extrema occur at critical points.'' It does not say ``All critical points produce relative extrema.'' For instance, consider $f(x) = x^3$. Since $\fp(x) = 3x^2$, it is straightforward to determine that $x=0$ is a critical point of $f$. However, $f$ has no relative extrema, as illustrated in \autoref{fig:extreme4}.\bigskip

\autoref{thm:extremeVal} states that a continuous function on a closed interval will have absolute extrema, that is, both an absolute maximum and an absolute minimum. These extrema occur either at the endpoints or at critical points in the interval. We combine these concepts to offer a strategy for finding extrema.

\begin{keyidea}[Finding Extrema on a Closed Interval]\label{idea:extrema}
Let $f$ be a continuous function defined on a closed interval $[a,b]$. To find the maximum and minimum values of $f$ on $[a,b]$:\index{extrema!finding}
	\begin{enumerate}
	\item		Evaluate $f$ at the endpoints $a$ and $b$ of the interval.
	\item		Find the critical points of $f$ in $(a,b)$.
	\item		Evaluate $f$ at each critical point.
	\item		The absolute maximum of $f$ is the largest of these values, and the absolute minimum of $f$ is the least of these values.
	\end{enumerate}
\end{keyidea}

We practice these ideas in the next examples.

\begin{example}[Finding extreme values]\label{ex_extval4}
Find the extreme values of $f(x) = 2x^3+3x^2-12x$ on $[0,3]$, graphed in \autoref{fig:extval4}(a).
\solution
We follow the steps outlined in \autoref{idea:extrema}. We first evaluate $f$ at the endpoints:
%
\mtable[-.5in]{A graph and table of extreme values of $f(x) = 2x^3+3x^2-12x$ on $[0,3]$ as in \autoref{ex_extval4}.}{fig:extval4}{%
\begin{tikzpicture}
\begin{axis}[width=1.16\marginparwidth,tick label style={font=\scriptsize},
axis y line=middle,axis x line=middle,name=myplot,
ymin=-7.5,ymax=45.5,xmin=-.4,xmax=3.3]
\addplot [thick,draw={\colorone},domain=0:3] {2*x^3+3*x^2-12*x};
\end{axis}
\node [right] at (myplot.right of origin) {\scriptsize $x$};
\node [above] at (myplot.above origin) {\scriptsize $y$};
\end{tikzpicture}
\\(a)\smallskip\\
\begin{tabular}{cr}
 $x$ & $f(x)$ \\ \midrule
 0 & 0 \\
 1 & $-7$\\
 3 & 45
\end{tabular}
\\(b)}
%
\[f(0)=2(0)^3+3(0)^2-12(0) = 0 \quad \text{and}\quad f(3)=2(3)^3+3(3)^2-12(3) =45.\]
Next, we find the critical points of $f$ on $[0,3]$. We see that $\fp(x) = 6x^2+6x-12 = 6(x+2)(x-1)$; therefore the critical points of $f$ are $x=-2$ and $x=1$. Since $x=-2$ does not lie in the interval $[0,3]$, we ignore it. Evaluating $f$ at the only critical point in our interval gives: $f(1)=2(1)^3+3(1)^2-12(1) = -7$. 

The table in \autoref{fig:extval4}(b) gives $f$ evaluated at the ``important'' $x$ values in $[0,3]$. We can easily see the maximum and minimum values of $f$: the maximum value is $45$ and the minimum value is $-7$.
\end{example}

Note that all this was done without the aid of a graph; this work followed an analytic algorithm and did not depend on any visualization. \autoref{fig:extval4}(a) shows $f$ and we can confirm our answer, but it is important to  understand that these answers can be found without graphical assistance.

We practice again.

\begin{example}[Finding extreme values]\label{ex_extval5}
Find the maximum and minimum values of $f$ on $[-4,2]$, where
\[f(x) = \begin{cases} (x-1)^2 & x\leq 0 \\ x+1 & x>0 \end{cases}.\]
\solution
Here $f$ is piecewise-defined, but we can still apply \autoref{idea:extrema} because it is continuous. Evaluating $f$ at the endpoints gives: 
\mtable{A table of extreme values and graph of $f(x)$ on $[-4,2]$ as in \autoref{ex_extval5}.}{fig:extval5}{\begin{tabular}{rr}
 $x$ & $f(x)$ \\ \midrule
 $-4$ & 25 \\
 0 & 1 \\
 2 & 3
\end{tabular}
\\(a)\\
\begin{tikzpicture}
\begin{axis}[width=1.16\marginparwidth,tick label style={font=\scriptsize},
axis y line=middle,axis x line=middle,name=myplot,
ymin=-.9,ymax=25.5,xmin=-4.5,xmax=2.5]
\addplot [thick,draw={\colorone},domain=-4:0] {(x-1)^2};
\addplot [thick,draw={\colorone},domain=0:2] {(x+1)};
\end{axis}
\node [right] at (myplot.right of origin) {\scriptsize $x$};
\node [above] at (myplot.above origin) {\scriptsize $y$};
\end{tikzpicture}
\\(b)}
\[f(-4)=(-4-1)^2=(-5)^2 = 25 \quad \text{and} \quad f(2)=2+1 = 3.\]

We now find the critical points of $f$. We have to define $\fp$ in a piecewise manner; it is
\[\fp(x) =\begin{cases} 2(x-1) & x < 0 \\ 1 & x>0 \end{cases} .\]
Note that while $f$ is defined for all of $[-4,2]$, $\fp$ is not, as the derivative of $f$ does not exist when $x=0$. (From the left, the derivative approaches $-2$; from the right the derivative is 1.) Thus one critical point of $f$ is $x=0$.

We now set $\fp(x) = 0$. When $x >0$, $\fp(x)$ is never 0.  When $x<0$, $\fp(x)$ is also never 0. (We may be tempted to say that $\fp(x) = 0 $ when $x=1$. However, this is nonsensical, for we only consider $\fp(x) = 2(x-1)$ when $x<0$, so we will ignore a solution that says $x=1$.) 

So we have three important $x$ values to consider: $x= -4, 2$ and $0$. We have already evaluated the first two, and $f(0)=(0-1)^2=(-1)^2=1$.  Collecting these values into \autoref{fig:extval5}(a), we see that the absolute minimum of $f$ is 1 and the absolute maximum of $f$ is $25$. Our answer is confirmed by the graph of $f$ in \autoref{fig:extval5}(b).
\end{example}

\mtable[-1in]{A table of extreme values and graph of $f(x)= \cos (x^2)$ on $[-2,2]$ in \autoref{ex_extval6}.}{table:ext6}{%
\begin{tabular}{cl}
 $x$ & $f(x)$ \\ \midrule
 $-2$ & $-0.65$ \\
 $-\sqrt{\pi}$ & $-1$ \\
 0 & $\phantom{-}1$ \\
 $\sqrt{\pi}$ & $-1$ \\
 2 & $-0.65$
\end{tabular}
\\(a)\\
\begin{tikzpicture}
\begin{axis}[width=1.16\marginparwidth,tick label style={font=\scriptsize},
axis y line=middle,axis x line=middle,name=myplot,
ymin=-1.1,ymax=1.1,xmin=-2.2,xmax=2.2]
\addplot [thick,draw={\colorone},domain=-2:2,samples=40,smooth] {cos(deg(x^2))};
\end{axis}
\node [right] at (myplot.right of origin) {\scriptsize $x$};
\node [above] at (myplot.above origin) {\scriptsize $y$};
\end{tikzpicture}
\\(b)}

\begin{example}[Finding extreme values]\label{ex_extval6}
Find the extrema of  $f(x) = \cos (x^2)$ on $[-2,2]$.
\solution
We again use \autoref{idea:extrema}. Evaluating $f$ at the endpoints of the interval gives: $f(-2) = f(2) = \cos (4) \approx -0.6536.$ We now find the critical points of $f$.

Applying the Chain Rule, we find $\fp(x) = -2x\sin (x^2)$. Set $\fp(x) = 0$ and solve for $x$ to find the critical points of $f$. 

We have $\fp(x) = 0$ when $x = 0$ and when $\sin (x^2) = 0$. In general, $\sin t = 0$ when $t = \dotsc -2\pi, -\pi, 0, \pi, \dotsc$ Thus $\sin (x^2) = 0$ when $x^2 = 0, \pi, 2\pi, \dotsc$ ($x^2$ is always positive so we ignore $-\pi$, etc.) So $\sin (x^2)=0$ when $x=0, \pm\sqrt\pi, \pm\sqrt{2\pi}, \dotsc$. The only values to fall in the given interval of $[-2,2]$ are $0$ and $\pm\sqrt\pi$, approximately $\pm 1.77$.

We again construct a table of important values in \autoref{table:ext6}(a). In this example we have 5 values to consider: $x= 0$, $\pm 2$, $\pm\sqrt{\pi}$. 

From the table it is clear that the maximum value of $f$ on $[-2,2]$ is 1; the minimum value is $-1$. The graph in \autoref{table:ext6}(b) confirms our results.
\end{example}

We consider one more example.

\mtable[-.5in]{A graph and table of extrema of $f(x)=\sqrt{1-x^2}$ on $[-1,1]$ as in \autoref{ex_extval7}.}{fig:extval7}{%
\begin{tikzpicture}
\begin{axis}[width=1.16\marginparwidth,tick label style={font=\scriptsize},
axis y line=middle,axis x line=middle,name=myplot,axis equal,
ytick={1},ymin=-.2,ymax=1.1,xmin=-1.2,xmax=1.2]
\addplot [thick,draw={\colorone},smooth,domain=0:180] ({cos(x)},{sin(x)});
\end{axis}
\node [right] at (myplot.right of origin) {\scriptsize $x$};
\node [above] at (myplot.above origin) {\scriptsize $y$};
\end{tikzpicture}
\\(a)\smallskip\\
\begin{tabular}{rc}
 $x$ & $f(x)$ \\ \midrule
 $-1$ & 0 \\
 0 & 1 \\
 1 & 0
\end{tabular}
\\(b)}

\begin{example}[Finding extreme values]\label{ex_extval7}
Find the extreme values of $f(x) = \sqrt{1-x^2}$.
\solution
A closed interval is not given, so we find the extreme values of $f$ on its domain. This $f$ is defined whenever $1-x^2\geq 0$; thus the domain of $f$ is $[-1,1]$. Evaluating $f$ at either endpoint returns 0.

Using the Chain Rule, we find $\ds \fp(x) = \frac{-x}{\sqrt{1-x^2}}$. The critical points of $f$ are found when $\fp(x) = 0$ or when $\fp$ is undefined. It is straightforward to find that $\fp(x) = 0$ when $x=0$, and $\fp$ is undefined when $x=\pm 1$, the endpoints of the interval. The table of important values is given in \autoref{fig:extval7}(b). The maximum value is 1, and the minimum value is 0.
\end{example}

\mnote[.5in]{\textbf{Note:} We implicitly found the derivative of $x^2+y^2=1$, the unit circle, in \autoref{ex_implicit7} as $\frac{dy}{dx} = -x/y$. In \autoref{ex_extval7}, half of the unit circle is given as $y=f(x) = \sqrt{1-x^2}$. We found $\fp(x) = \frac{-x}{\sqrt{1-x^2}}$. Recognize that the denominator of this fraction is $y$; that is, we again found $\fp(x) = \frac{dy}{dx} = -x/y.$}

We have seen that continuous functions on closed intervals always have a maximum and minimum value, and we have also developed a technique to find these values. In the next section, we further our study of the information we can glean from ``nice'' functions with the Mean Value Theorem. On a closed interval, we can find the \textit{average rate of change} of a function (as we did at the beginning of Chapter 2). We will see that differentiable functions always have a point at which their \textit{instantaneous} rate of change is same as the \textit{average} rate of change. This is surprisingly useful, as we'll see.

\printexercises{exercises/03_01_exercises}
