\section{The Mean Value Theorem}\label{sec:mvt}

We motivate this section with the following question: Suppose you leave your house and drive to your friend's house in a city 100 miles away, completing the trip in two hours.  Is there necessarily a moment during the trip when you are going 50 miles per hour?

In answering this question, it is clear that the \emph{average} speed for the entire trip is 50 mph (i.e. 100 miles in 2 hours), but the question is whether or not your \emph{instantaneous} speed is ever exactly 50 mph. More simply, does your speedometer ever read exactly 50 mph?  \autoref{fig:mvt_ex} shows a graphical interpretation of this question.  The answer, under some very reasonable assumptions, is ``yes.''

\mtable[.5in]{Distance traveled as a function of time.  Must there be a tangent line parallel to the average slope?}{fig:mvt_ex}{\pdftooltip{\begin{tikzpicture}
\begin{axis}[width=\marginparwidth, tick label style={font=\scriptsize},
			ymajorgrids=true, xmajorgrids=true, xmin=0, xmax=2, ymin=0, ymax=100,
			xlabel={Time (hours)}, ylabel={Distance (miles)}]
\addplot [draw={\colorone},thick,smooth] coordinates {(0,0)(.5,10)(1,50)(1.5,90)(2,100)};
\addplot [draw={\colortwo},thick,smooth] coordinates {(0,0)(2,100)};
\end{axis}
\end{tikzpicture}}{A straight line from the origin to the point (2,100).  A different curve begins at the origin, and starts out below the straight line.  Then it curves up to cross the line at (1,50) and is above the line before finishing at the same (2,100).}}

Let's now see why this situation is in a calculus text by translating it into mathematical symbols.\bigbreak

First assume that the function $y = f(t)$ gives the distance (in miles) traveled from your home at time $t$ (in hours) where $0\le t\le 2$.  In particular, this gives $f(0)=0$ and $f(2)=100$.  The slope of the secant line (average velocity) connecting the starting and ending points $(0,f(0))$ and $(2,f(2))$ is therefore 
\[
\frac{\Delta f}{\Delta t} = \frac{f(2)-f(0)}{2-0} = \frac{100-0}{2} = 50 \, \text{mph}.
\]

The slope at any point on the graph itself (instantaneous velocity) is given by the derivative $\fp(t)$.  So, since the answer to the question above is ``yes,'' this means that at some time during the trip, the derivative takes on the value of 50 mph.  Symbolically, 
\[\fp(c) = \frac{f(2)-f(0)}{2-0} = 50\]
for some time $0\le c \le 2.$\bigskip

How about more generally?  Given any function $y=f(x)$ and a range $a\le x\le b$ does the value of the derivative at some point between $a$ and $b$ have to match the slope of the secant line connecting the points $(a,f(a))$ and $(b,f(b))$?  Or equivalently, does the equation 
$\fp(c) = \frac{f(b)-f(a)}{b-a}$ have to hold for some $a < c < b$?

Let's look at two functions in an example.

\begin{example}[Comparing average and instantaneous rates of change]\label{ex_mvt1}
Consider functions
%
\mtable[2in]{A graph of $f_1(x) = 1/x^2$ and $f_2(x) = \abs{x}$ in \autoref{ex_mvt1}.}{fig:mvt1}%
{\pdftooltip{\begin{tikzpicture}
 \begin{axis}[width=1.16\marginparwidth,tick label style={font=\scriptsize},
   minor x tick num=1,axis y line=middle,axis x line=middle,
   ymin=-.6,ymax=3.9,xmin=-1.95,xmax=1.95,name=myplot]
  \addplot[draw={\colorone},smooth,thick,domain=-1.95:-.507] {1/(x*x)};
  \addplot[draw={\colorone},smooth,thick,domain=.507:1.95] {1/(x*x)};
  \filldraw [fill={\colorone},draw={\colorone},thick] (axis cs:-1,1) circle (1pt);
  \filldraw [fill={\colorone},draw={\colorone},thick] (axis cs:1,1) circle (1pt);
  \draw [draw={\colorone},dashed,thick] (axis cs:-1,1) -- (axis cs:1,1);
 \end{axis}
 \node [right] at (myplot.right of origin) {\scriptsize $x$};
 \node [above] at (myplot.above origin) {\scriptsize $y$};
\end{tikzpicture}}{A curve beginning near (-2,.25) and curving upward to get arbitrarily large as x nears zero.  Afterward, the curve is a mirror image across the y-axis, decreasing to finish near (2,.25).  There is a dashed line joining the points (-1,1) and (1,1).}
\smallskip\\(a)\\
\pdftooltip{\begin{tikzpicture}
\begin{axis}[width=1.16\marginparwidth,tick label style={font=\scriptsize},
minor x tick num=1,axis y line=middle,axis x line=middle,
ymin=-.1,ymax=1.2,ytick={.5,1},xmin=-1.4,xmax=1.4,name=myplot,axis equal]
\filldraw [fill={\colorone},draw={\colorone},thick] (axis cs:-1,1) circle (1pt);
\filldraw [fill={\colorone},draw={\colorone},thick] (axis cs:1,1) circle (1pt);
\draw [draw={\colorone},dashed,thick] (axis cs:-1,1) -- (axis cs:1,1); 
\draw [draw={\colorone},thick] (axis cs:-1.2,1.2) -- (axis cs:0,0); 
\draw [draw={\colorone},thick] (axis cs:0,0) -- (axis cs:1.2,1.2);
\end{axis}
\node [right] at (myplot.right of origin) {\scriptsize $x$};
\node [above] at (myplot.above origin) {\scriptsize $y$};
\end{tikzpicture}}{A line segment from (-1.2,1.2) to the origin and then up to (1.2,1.2).  There is a dashed line joining the points (-1,1) and (1,1).}
\\(b)}
%
\[f_1(x)=\frac{1}{x^2}\quad \text{and} \quad f_2(x) = \abs{x}\]
with $a=-1$ and $b=1$ as shown in \autoref{fig:mvt1}(a) and (b), respectively. Both functions have a value of 1 at $a$ and $b$.  Therefore the slope of the secant line connecting the end points is $0$ in each case.  But if you look at the plots of each, you can see that there are no points on either graph where the tangent lines have slope zero. Therefore we have found that there is no $c$ in $[-1,1]$ such that \[\fp(c) = \frac{f(1)-f(-1)}{1-(-1)} = 0.\]
\end{example}

So what went ``wrong\primeskip''?  It may not be surprising to find that the discontinuity of $f_1$ and the corner of $f_2$ play a role.  If our functions had been continuous and differentiable, would we have been able to find that special value $c$? This is our motivation for the following theorem.

\mtable{A graph of illustrating the Mean Value Theorem of Differentiation}{fig:mvt_illus}{\pdftooltip{\begin{tikzpicture}
 \begin{axis}[width=1.16\marginparwidth,axis equal,
   axis x line=middle, axis y line=middle, xmin=-.2, xmax=2.2, ymin=-.2, ymax=1.5,
   xtick={.5, 1.1, 2}, ytick={.3,.75,1.02},
   yticklabels={\scriptsize$f(a)$, ,\scriptsize$f(b)$},
   xticklabels={\scriptsize$a$, \scriptsize$c$, \scriptsize$b$},name=myplot]
  \draw (axis cs: .5,.3) .. controls (axis cs: 1,1.3) and
   (axis cs: 1.2,1.1) .. (axis cs: 2,.75);
  \addplot[mark=*, only marks] coordinates {(.5,.3)(2,.75)(1.1,1.02)};
  \addplot[domain=.2:2.3, draw=\colorone, smooth] {.3*x+.15}
	node [color=black,pos=.55,below right] {\scriptsize{secant line}};
  \addplot[domain=.2:2.3, draw=\colorone, dashed, smooth] {(.3*x+.15)+.54}
	node[color=black,pos=.7,above left]{\scriptsize{tangent line}};
  \draw[dashed, thin, draw=\colortwo] (axis cs:1.1,0)-- (axis cs:1.1,1.02);
 \end{axis}
 \node [right] at (myplot.right of origin) {$x$};
 \node [above] at (myplot.above origin) {$y$};
\end{tikzpicture}}{A curve starting at a solid dot at (0.5,0.3), going up to a solid dot at (2,0.75), and finishing at a solid dot near (1.1,1).  A secant line goes through the first and third dots.  A tangent line goes through the middle dot and is tangent to the curve.}}

\begin{theorem}[The Mean Value Theorem of Differentiation]\label{thm:mvt}
Let $y=f(x)$ be a continuous function on the closed interval $[a,b]$ and differentiable on the open interval $(a,b)$. There exists a value $c$, $a < c < b$, such that \index{Mean Value Theorem!of differentiation}\index{derivative!Mean Value Theorem}
\[\fp(c) = \frac{f(b)-f(a)}{b-a}.\]
That is, there is a value $c$ in $(a,b)$ where the instantaneous rate of change of $f$ at $c$ is equal to the average rate of change of $f$ on $[a,b]$.
\end{theorem}

Note that the reasons that the functions in \autoref{ex_mvt1} fail are indeed that $f_1$ has a discontinuity on the interval $[-1,1]$ and $f_2$ is not differentiable at the origin.\bigskip

We will give a proof of the Mean Value Theorem below. To do so, we use Rolle's Theorem, stated here.

\begin{theorem}[Rolle's Theorem]\label{thm:rolles}
Let $f$ be continuous on $[a,b]$ and differentiable on $(a,b)$, where $f(a) = f(b)$. There is some $c$ in $(a,b)$ such that $\fp(c) = 0.$\index{Rolle's Theorem}
\end{theorem}

\mtable{A graph of $f(x) = x^3-5x^2+3x+5$, where $f(a) = f(b)$. Note the existence of $c$, where $a<c<b$, where $\fp(c)=0$.}{fig:mvt3}{\pdftooltip{\begin{tikzpicture}
\begin{axis}[width=1.16\marginparwidth,tick label style={font=\scriptsize},
axis y line=middle,axis x line=middle,name=myplot,xtick={-1,1,2},
extra x ticks={-.39,1.22,.333},extra x tick labels={$a$,$b$,$c$},
ymin=-5.2,ymax=8,xmin=-1.5,xmax=2.5]
\addplot [thick,draw={\colorone},smooth,domain=-1.5:3.5] {x^3-5*x^2+3*x+5};
\addplot [dashed,thin,domain=-1:2] {3};
\filldraw (axis cs:-.39,3) circle (1pt);
\filldraw (axis cs:1.22,3) circle (1pt);
\addplot [thin,dashed,domain=-.75:1.5] {5.48};
\filldraw (axis cs:.3333,5.48) circle (1pt);
\end{axis}
\node [right] at (myplot.right of origin) {\scriptsize $x$};
\node [above] at (myplot.above origin) {\scriptsize $y$};
\end{tikzpicture}}{A curve beginning near (-1.1,-5.7) increasing to cross the x-axis and then the y-axis, peaking near (.3,5.5), and then descending to cross the x axis and finish near (2.5,-3.1).  There is a dashed horizontal line touching the peak, and another dashed horizontal line below that intersecting the curve at two points where x equals a and b.}}

Consider \autoref{fig:mvt3} where the graph of a function $f$ is given, where $f(a) = f(b)$. It should make intuitive sense that if $f$ is differentiable (and hence, continuous) that there would be a value $c$ in $(a,b)$ where $\fp(c)=0$; that is, there would be a relative maximum or minimum of $f$ in $(a,b)$. Rolle's Theorem guarantees at least one; there may be more. 

Rolle's Theorem is really just a special case of the Mean Value Theorem. If $f(a) = f(b)$, then the \emph{average} rate of change on $(a,b)$ is $0$, and the theorem guarantees some $c$ where $\fp(c)=0$. We will prove Rolle's Theorem, then use it to prove the Mean Value Theorem.

\begin{proof}[Proof of Rolle's Theorem]
Let $f$ be differentiable on $(a,b)$ where $f(a)=f(b)$. We consider two cases. 

\begin{description}[leftmargin=0pt]
\item[Case 1:] Consider the case when $f$ is constant on $[a,b]$; that is, $f(x) = f(a) = f(b)$ for all $x$ in $[a,b]$. Then $\fp(x) = 0$ for all $x$ in $[a,b]$, showing there is at least one value $c$ in $(a,b)$ where $\fp(c)=0$.

\item[Case 2:] Now assume that $f$ is not constant on $[a,b]$. The Extreme Value Theorem guarantees that $f$ has a maximal and minimal value on $[a,b]$, found either at the endpoints or at a critical point in $(a,b)$. Since $f(a)=f(b)$ and $f$ is not constant, it is clear that the maximum and minimum cannot \emph{both} be found at the endpoints. Assume, without loss of generality, that the maximum of $f$ is not found at the endpoints. Therefore there is a $c$ in $(a,b)$ such that $f(c)$ is the maximum value of $f$. By \autoref{thm:criticalpts}, $c$ must be a critical point of $f$; since $f$ is differentiable, we have that $\fp(c) = 0$, completing the proof of the theorem.\qedhere
\end{description}
\end{proof}

% todo Tim does the next example need an introduction?
% IVT says some function has a root.  Can there be other roots?  Rolle's can be used to show there aren't additional
%An interesting use of Rolle's Theorem is to prove that a function cannot have too many roots, as in our next example.

\begin{example}[Exactly One Root]\label{ex_rolle1}
Show that $f(x)=8x^7+x^3+3x+2$ has exactly one real root.
\solution
We'll do this in two steps.  The first step is to use the Intermediate Value Theorem to show that there is at least one root.  The second step is to use Rolle's Theorem to show that there is at most one root.  (Because $f$ is a polynomial, it is continuous and differentiable, so both of these theorems apply.)

We can apply the Intermediate Value Theorem on the interval $[-1,0]$.  Since $f(-1)=-10<0<f(0)=2$, the Intermediate Value Theorem tells us that there is at least one place in $[-1,0]$ where $f(x)=0$.  This means that there is at least one root, but there may be more in the interval (and there may be more outside the interval where we haven't even looked).

We will now use Rolle's Theorem to show that $f$ has at most one root. Suppose for this paragraph that $f$ had two (or more) roots. Then by Rolle's Theorem, there is some $c$ in between the roots so that $0=\fp(c)=56x^6+3x^2+3$. But this cannot happen, since $\fp$ is always at least 3.

Therefore, $f$ has at most one root.  Combining this with ``there is at least one root'', we see that $f$ has exactly one root.  (Notice that because both the Intermediate Value Theorem and Rolle's Theorem are existential theorems, we don't know what the root is, only that it must exist.)
\end{example}

We will now use Rolle's Theorem to prove the Mean Value Theorem.

\begin{proof}[Proof of the Mean Value Theorem]
Define the function
\[g(x) = f(x) - \frac{f(b)-f(a)}{b-a}x.\]
We know $g$ is differentiable on $(a,b)$ and  continuous on $[a,b]$ since $f$ is.  We also see that
\begin{align*}
g(b)-g(a)&=f(b)-\frac{f(b)-f(a)}{b-a}b-f(a)+\frac{f(b)-f(a)}{b-a}a\\
&=\bigl(f(b)-f(a)\bigr)-\frac{f(b)-f(a)}{b-a}(b-a)=0
\end{align*}
which shows that $g(a)=g(b)$. We can then apply Rolle's theorem to guarantee the existence of $c \in (a,b)$ such that $g\primeskip'(c) = 0$.  But note that
\[0= g\primeskip'(c) = \fp(c) - \frac{f(b)-f(a)}{b-a} \ ;\]
hence
\[\fp(c) = \frac{f(b)-f(a)}{b-a},\]
which is what we sought to prove.
\end{proof}

Going back to the very beginning of the section, we see that the only assumption we would need about our distance function $f(t)$ is that it be continuous and differentiable for $t$ from 0 to 2 hours (both reasonable assumptions).  By the Mean Value Theorem, we are guaranteed a time during the trip where our instantaneous speed is 50 mph. This fact is used in practice. Some law enforcement agencies monitor traffic speeds while in aircraft. They do not measure speed with radar, but rather by timing individual cars as they pass over lines painted on the highway whose distances apart are known. The officer is able to measure the \emph{average} speed of a car between the painted lines; if that average speed is greater than the posted speed limit, the officer is assured that the driver exceeded the speed limit at some time.

Note that the Mean Value Theorem is an \emph{existence} theorem. It states that a special value $c$ \emph{exists}, but it does not give any indication about how to find it. It turns out that when we need the Mean Value Theorem, existence is all we need.

\youtubeVideo{xYOrYLq3fE0}{The Mean Value Theorem}

\begin{example}[Using the Mean Value Theorem]\label{ex_mvt2}
Consider $f(x) = x^3+5x+5$ on $[-3,3]$. Find $c$ in $[-3,3]$ that satisfies the Mean Value Theorem.
\solution
The average rate of change of $f$ on $[-3,3]$ is:
		\[\frac{f(3)-f(-3)}{3-(-3)} = \frac{84}{6} = 14.\]
		
We want to find $c$ such that $\fp(c) = 14$. We find $\fp(x) = 3x^2+5$. We set this equal to 14 and solve for $x$. 
		\begin{align*}
		\fp(x) &= 14 \\
		3x^2 +5 &= 14\\
		x^2  &= 3\\
		x &= \pm \sqrt{3} \approx \pm 1.732
		\end{align*}
		
\mtable{Demonstrating the Mean Value Theorem in \autoref{ex_mvt2}.}{fig:mvt4}{\pdftooltip{\begin{tikzpicture}
\begin{axis}[width=1.16\marginparwidth,tick label style={font=\scriptsize},
axis y line=middle,axis x line=middle,name=myplot,xtick={-3,-2,-1,1,2,3},
ymin=-45,ymax=51,xmin=-3.2,xmax=3.2]
\addplot [thick,draw={\colorone},smooth,domain=-3:3] {x^3+5*x+5};
\draw [dashed,thick] (axis cs:-3,-37) -- (axis cs:3,47);
\filldraw (axis cs:-3,-37) circle (1pt);
\filldraw (axis cs:3,47) circle (1pt);
\filldraw (axis cs:-1.732,-8.86) circle (.75pt);
\addplot [thick,dashed,domain=-2.8:-.1,draw={\colortwo}] {14*(x+1.732)-13.86+5};
\filldraw (axis cs:1.732,18.86) circle (.75pt);
\addplot [thick,dashed,domain=-.1:3,draw={\colortwo}] {14*(x-1.732)+13.86+5};
\end{axis}
\node [right] at (myplot.right of origin) {\scriptsize $x$};
\node [above] at (myplot.above origin) {\scriptsize $y$};
\end{tikzpicture}}{A curve beginning at (-3,-37), increasing quickly and then more slowly to cross the x axis near x=-1 and then the y-axis at y=5 before increasing quickly again to finish at (3,47).  A dashed line connects the starting and ending points.  Another dashed line is tangent the curve above the first dashed line, while a third dashed line is tangent to the curve below the first dashed line.  All three dashed lines are parallel.}}

We have found 2 values $c$ in $[-3,3]$ where the instantaneous rate of change is equal to the average rate of change; the Mean Value Theorem guaranteed at least one. In \autoref{fig:mvt4} $f$ is graphed with a dashed line representing the average rate of change; the lines tangent to $f$ at $x=\pm \sqrt{3}$ are also given. Note how these lines are parallel (i.e., have the same slope) as the dashed line.
\end{example}

While the Mean Value Theorem has practical use (for instance, the speed monitoring application mentioned before), it is mostly used to advance other theory. We will use it in the next section to relate  the shape of a graph to its derivative.

%
%EXERCISES
%1. Draw a couple of possible graphs of time versus distance traveled for the situation at the beginning of the section (traveling to your friend's house).  Also, mark on your graphs the times at which you are traveling exactly 50 mph.
%2. Estimate the values of $x$ for which the speed matches the overall average speed for the entire trip (with picture).

\printexercises{exercises/03-02-exercises}
