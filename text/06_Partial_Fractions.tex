% this was orphaned from u-substitution

%\begin{example}[Integration by substitution: simplifying first]\label{ex_sub9}
%Evaluate $\ds\int \frac{x^3+4x^2+8x+5}{x^2+2x+1}\dd x$.
%\solution
%One may try to start by setting $u$ equal to either the numerator or denominator; in each instance, the result is not workable. 
%
%When dealing with rational functions (i.e., quotients made up of polynomial functions), it is an almost universal rule that everything works better when the degree of the numerator is less than the degree of the denominator. Hence we use polynomial division.
%
%We skip the specifics of the steps, but note that when $x^2+2x+1$ is divided into $x^3+4x^2+8x+5$, it goes in $x+2$ times with a remainder of $3x+3$. Thus 
%\[\frac{x^3+4x^2+8x+5}{x^2+2x+1} = x+2 + \frac{3x+3}{x^2+2x+1}.\]
%Integrating $x+2$ is simple. The fraction can be integrated by setting $u = x^2+2x+1$, giving $\dd u = (2x+2)\dd x$. This is very similar to the numerator. Note that $\dd u/2 = (x+1)\dd x$ and then consider the following:
%\begin{align*} % this had ``rule'' commands scattered throughout
%\int \frac{x^3+4x^2+8x+5}{x^2+2x+1}\dd x
% & = \int \left(x+2 + \frac{3x+3}{x^2+2x+1}\right)\dd x \\
% &= \int (x+2)\dd x + \int \frac{3(x+1)}{x^2+2x+1}\dd x \\
% & = \frac12x^2+2x+C_1 + \int \frac{3}{u}\frac{\dd u}{2} \\
% &= \frac12x^2+2x+C_1 + \frac32\ln\abs u + C_2 \\
% &= \frac12x^2+2x+\frac32\ln\abs{x^2+2x+1} + C.
%\end{align*}
%In some ways, we ``lucked out'' in that after dividing, substitution was able to be done. In later sections we'll develop techniques for handling rational functions where substitution is not directly feasible.
%\end{example}



\section{Partial Fraction Decomposition}\label{sec:partial_fraction}

In this section we investigate the antiderivatives of rational functions. Recall that rational functions are functions of the form $f(x)= \frac{p(x)}{q(x)}$, where $p(x)$ and $q(x)$ are polynomials and $q(x)\neq 0$. Such functions arise in many contexts, one of which is the solving of certain fundamental differential equations.

We begin with an example that demonstrates the motivation behind this section. Consider the integral $\ds\int \frac{1}{x^2-1}\dd x$. We do not have a simple formula for this (if the denominator were $x^2+1$, we would recognize the antiderivative as being the arctangent function). It can be solved using Trigonometric Substitution, but note how the integral is easy to evaluate once we realize:

%This integral is not difficult to evaluate once one realizes the following fact: 
\[\frac{1}{x^2-1} = \frac{1/2}{x-1} - \frac{1/2}{x+1}.\]
Thus 
\begin{align*}
\int\frac{1}{x^2-1}\dd x &= \int\frac{1/2}{x-1}\dd x - \int\frac{1/2}{x+1}\dd x \\
			&= \frac12\ln\abs{x-1}-\frac12\ln\abs{x+1}+C.
\end{align*}

This section teaches how to \emph{decompose}
\[\frac1{x^2-1}\quad  \text{into}\quad  \frac{1/2}{x-1}-\frac{1/2}{x+1}.\]

We start with a rational function $f(x)=\dfrac{p(x)}{q(x)}$, where $p$ and $q$ do not have any common factors. We first consider the degree of $p$ and $q$. 
\begin{itemize}
	\item If the $\deg(p)\ge\deg(q)$ then we use polynomial long division to divide $q$ into $p$ to determine a remainder $r(x)$ where $\deg(r)<\deg(q)$. We then write $f(x) =s(x)+\dfrac{r(x)}{q(x)}$ and apply partial fraction decomposition to $\dfrac{r(x)}{q(x)}$.
	\item If the $\deg(p)<\deg(q)$ we can apply partial fraction decomposition to $\dfrac{p(x)}{q(x)}$ without additional work.
\end{itemize}

Partial fraction decomposition is based on an algebraic theorem that guarantees that any polynomial, and hence $q$, can use real numbers to factor into the product of linear and irreducible quadratic factors.
\mnote[-.5\baselineskip]{An \emph{irreducible quadratic} is one that cannot factor into linear terms with real coefficients.}\index{irreducible quadratic}
The following Key Idea states how to decompose a rational function into a sum of rational functions whose denominators are all of lower degree than $q$.

{
\tcbset{grow to right by=6em}
\begin{keyidea}[Partial Fraction Decomposition]\label{idea:partial_fraction}
Let $\dfrac{p(x)}{q(x)}$ be a rational function, where $\deg(p)<\deg(q)$.
\begin{enumerate}
	\item \textbf {Factor} $\mathbf{q(x):}$ Write $q(x)$ as the product of its linear and irreducible quadratic factors of the form $(ax+b)^m$ and $(ax^2+bx+c)^n$ where $m$ and $n$ are the highest powers of each factor that divide $q$.
	\begin{itemize}
		\item \textbf{Linear Terms:} For each linear factor of $q(x)$ the decomposition of $\dfrac{p(x)}{q(x)}$ will contain the following terms:
		\[\frac{A_1}{(ax+b)}+\frac {A_2}{(ax+b)^2}+\dotsb\frac{A_m}{(ax+b)^m}\]
		\item \textbf{Irreducible Quadratic Terms:}  For each irreducible quadratic factor of $q(x)$ the decomposition of $\dfrac {p(x)}{q(x)}$ will contain the following terms:
		\[
		 \dfrac{B_1x+C_1}{(ax^2+bx+c)}+\frac{B_2x+C_2}{(ax^2+bx+c)^2}
		 +\dotsb\frac{B_nx+C_n}{(ax^2+bx+c)^n}
		\]
	\end{itemize}
	\item \textbf{Finding the Coefficients $\mathbf{A_i}$, $\mathbf{B_i}$, and $\mathbf{C_i}$:}
	\begin{itemize}
		\item Set $\dfrac{p(x)}{q(x)}$ equal to the sum of its linear and irreducible quadratic terms.
		\[
		 \frac{p(x)}{q(x)}
		 =\frac{A_1}{(ax+b)}+\dotsb\frac{A_m}{(ax+b)^m}
		 +\frac{B_1x+C_1}{(ax^2+bx+c)}+\dotsb\frac{B_nx+C_n}{(ax^2+bx+c)^n}
		\]
		\item Multiply this equation by the factored form of $q(x)$ and simplify to clear the denominators. 
		\item Solve for the coefficients $A_i, B_i,$ and $C_i$ by
		\begin{enumerate}
			\item multiplying out the remaining terms and collecting like powers of $x$, equating the resulting coefficients and solving the resulting system of linear equations, \textbf{or}
			\item substituting in values for $x$ that eliminate terms so the simplified equation can be solved for a coefficient.
		\end{enumerate}
	\end{itemize}
\end{enumerate}
\end{keyidea}
}

\youtubeVideo{6qVgHWxdlZ0}{Integration Using method of Partial Fractions}

The following examples will demonstrate how to put this Key Idea into practice. In \autoref{ex_pf1}, we focus on the setting up the decomposition of a rational function.

\begin{example}[Decomposing into partial fractions]\label{ex_pf1}
Decompose $\ds f(x)=\frac{1}{(x+5)(x-2)^3(x^2+x+2)(x^2+x+7)^2}$ without solving for the resulting coefficients.
\solution
The denominator is already factored, as both $x^2+x+2$ and $x^2+x+7$ are irreducible quadratics. We need to decompose $f(x)$ properly. Since $(x+5)$ is a linear factor that divides the denominator, there will be a
\[\frac{A}{x+5}\]
term in the decomposition.

As $(x-2)^3$ divides the denominator, we will have the following terms in the decomposition:
\[\frac{B}{x-2},\quad \frac{C}{(x-2)^2}\quad \text{and}\quad \frac{D}{(x-2)^3}.\]

The $x^2+x+2$ term in the denominator results in a $\ds\frac{Ex+F}{x^2+x+2}$ term.

Finally, the $(x^2+x+7)^2$ term results in the terms
\[\frac{Gx+H}{x^2+x+7}\quad \text{and}\quad \frac{Ix+J}{(x^2+x+7)^2}.\]
All together, we have
\begin{multline*}
	\frac{1}{(x+5)(x-2)^3(x^2+x+2)(x^2+x+7)^2}= \\
	\frac{A}{x+5} + \frac{B}{x-2}+ \frac{C}{(x-2)^2}+\frac{D}{(x-2)^3}+\\
	\frac{Ex+F}{x^2+x+2}+\frac{Gx+H}{x^2+x+7}+\frac{Ix+J}{(x^2+x+7)^2}
\end{multline*}
Solving for the coefficients $A$, $B$, \dots, $J$ would be a bit tedious but not ``hard.''  In the next example we demonstrate solving for the coefficients using both methods given in \autoref{idea:partial_fraction}.
\end{example}

%(-37 x-39)/(140800 (x^2+x+2))+(67804 x+21113)/(520524225 (x^2+x+7))+(89 x-32)/(296595 (x^2+x+7)^2)+665617/(5015768576 (x-2))-1/(5501034 (x+5))-1119/(6889792 (x-2)^2)+1/(9464 (x-2)^3)

\begin{example}[Decomposing into partial fractions]\label{ex_pf2}
Perform the partial fraction decomposition of $\ds \frac{1}{x^2-1}$.
\solution
The denominator can be written as the product of two linear factors: $x^2-1 = (x-1)(x+1)$. Thus 
\begin{equation}\label{eq:decomp2}
 \frac{1}{x^2-1} = \frac{A}{x-1} + \frac{B}{x+1}.
\end{equation}
Using the method described in \autoref{idea:partial_fraction} 2(a) to solve for $A$ and $B$, first multiply through by $x^2-1 = (x-1)(x+1)$:
\begin{align}\label{eq:pf2}
	1
	&= \frac{A(x-1)(x+1)}{x-1}+\frac{B(x-1)(x+1)}{x+1} \notag\\
	&= A(x+1) + B(x-1)\\
	&= Ax+A + Bx-B \notag\\
	&= (A+B)x + (A-B)\qquad\text{collect like terms.}\notag
\end{align}
The next step is key. %Note the equality we have:
%\[1 = (A+B)x+(A-B).\]
For clarity's sake, rewrite the equality we have as
\[0x+1 = (A+B)x+(A-B).\]
On the left, the coefficient of the $x$ term is 0; on the right, it is $(A+B)$. Since both sides are equal for all values of $x$, we must have that $0=A+B$. Likewise, on the left, we have a constant term of 1; on the right, the constant term is $(A-B)$. Therefore we have $1=A-B$.

We have two linear equations with two unknowns. This one is easy to solve by hand, leading to 
\[
 \begin{aligned}A+B&=0\\A-B&=1\end{aligned}
 \qquad\Rightarrow\qquad
 \begin{aligned}A&=1/2\\B&=-1/2.\end{aligned}
\]
Thus
\[\frac{1}{x^2-1}=\frac{1/2}{x-1}-\frac{1/2}{x+1}.\]

Before solving for $A$ and $B$ using the method described in \autoref{idea:partial_fraction} 2(b), we note that Equations \eqref{eq:decomp2} and \eqref{eq:pf2} are not equivalent. Only the second equation holds for all values of $x$, including $x=-1$ and $x=1$, by continuity of polynomials. Thus, we can choose values for $x$ that eliminate terms in the polynomial to solve for $A$ and $B$.
\[1=A(x+1) + B(x-1).\]
If we choose $x=-1$,
\begin{align*}
 1&=A(0) + B(-2) \\
 B&=-\frac{1}{2}.
\end{align*}
Next choose $x=1$:
\begin{align*}
 1&=A(2) + B(0) \\
 A&=\frac{1}{2}.
\end{align*}
Resulting in the same decomposition as above.
\end{example}

In \autoref{ex_pf3}, we solve for the decomposition coefficients using the system of linear equations (method 2a). The margin note explains how to solve using substitution (method 2b).

\begin{example}[Integrating using partial fractions]\label{ex_pf3}
Use partial fraction decomposition to integrate $\ds\int\frac{1}{(x-1)(x+2)^2}\dd x$.
\solution
We decompose the integrand as follows, as described by \autoref{idea:partial_fraction}:
\begin{equation}\label{eq:decomp3}
 \frac{1}{(x-1)(x+2)^2} = \frac{A}{x-1} + \frac{B}{x+2} + \frac{C}{(x+2)^2}.
\end{equation}
To solve for $A$, $B$ and $C$, we multiply both sides by $(x-1)(x+2)^2$ and collect like terms:
%
\mnote{\textbf{Note:} Equations \eqref{eq:decomp3} and \eqref{eq:pf3} are not equivalent for $x=1$ and $x=-2$. However, due to the continuity of polynomials we can let $x=1$ to simplify the right hand side to $A(1+2)^2=9A$. Since the left hand side is still $1$, we have $1=9A$, so that $A=1/9$.\bigskip\\
Likewise,when $x=-2$; this leads to the equation $1=-3C$. Thus $C = -1/3$.\bigskip\\
Knowing $A$ and $C$, we can find the value of $B$ by choosing yet another value of $x$, such as $x=0$, and solving for $B$.}
%
\begin{align}
	1
	&= A(x+2)^2 + B(x-1)(x+2) + C(x-1)\label{eq:pf3}\\
	&= Ax^2+4Ax+4A + Bx^2 + Bx-2B + Cx-C \notag \\
	&= (A+B)x^2 + (4A+B+C)x + (4A-2B-C)\notag
\end{align}
We have
\[0x^2+0x+ 1 = (A+B)x^2 + (4A+B+C)x + (4A-2B-C)\]
leading to the equations 
\[A+B = 0, \quad 4A+B+C = 0 \quad \text{and} \quad 4A-2B-C = 1.\]
These three equations of three unknowns lead to a unique solution:
\[A = 1/9,\quad B = -1/9 \quad \text{and} \quad C = -1/3.\]
Thus 
\[
\int\frac{1}{(x-1)(x+2)^2}\dd x = \int \frac{1/9}{x-1}\dd x + \int \frac{-1/9}{x+2}\dd x + \int \frac{-1/3}{(x+2)^2}\dd x.
\]

Each can be integrated with a simple substitution with $u=x-1$ or $u=x+2$.
% (or by directly applying \autoref{idea:linearsub} as the denominators are linear functions).
The end result is
\[\int\frac{1}{(x-1)(x+2)^2}\dd x = \frac19\ln\abs{x-1}-\frac19\ln\abs{x+2} +\frac1{3(x+2)}+C.\]
\end{example}

\begin{example}[Integrating using partial fractions]\label{ex_pf4}
Use partial fraction decomposition to integrate $\ds \int \frac{x^3}{(x-5)(x+3)}\dd x$.
\solution
\autoref{idea:partial_fraction} presumes that the degree of the numerator is less than the degree of the denominator. Since this is not the case here, we begin by using polynomial division to reduce the degree of the numerator. We omit the steps, but encourage the reader to verify that
\[\frac{x^3}{(x-5)(x+3)} = x+2+\frac{19x+30}{(x-5)(x+3)}.\]
Using \autoref{idea:partial_fraction}, we can rewrite the new rational function as:
\[\frac{19x+30}{(x-5)(x+3)} = \frac{A}{x-5} + \frac{B}{x+3}\]
for appropriate values of $A$ and $B$. Clearing denominators, we have 
\[19x+30=A(x+3)+B(x-5).\]
As in the previous examples we choose values of $x$ to eliminate terms in the polynomial.  If we choose $x=-3$,
\begin{align*}
 19(-3)+30&=A(0) + B(-8) \\
 B&= \frac{27}{8}.
\end{align*}
Next choose $x=5$:
\begin{align*}
 19(5)+30&=A(8) + B(0)\\
 A&= \frac{125}{8}.
\end{align*}
We can now integrate:
\begin{align*}
	\int \frac{x^3}{(x-5)(x+3)}\dd x
	&= \int\left(x+2+\frac{125/8}{x-5}+\frac{27/8}{x+3}\right)\dd x \\
	&= \frac{x^2}2+2x+\frac{125}{8}\ln\abs{x-5}+\frac{27}8\ln\abs{x+3}+C.
\end{align*}
\end{example}

Before the next example we remind the reader of a rational integrand evaluated by trigonometric substitution:
\[\int\frac1{x^2+a^2}\dd x=\frac1a\tan^{-1}\left(\frac xa\right) + C.\]

\begin{example}[Integrating using partial fractions]\label{ex_pf5}
Use partial fraction decomposition to evaluate $\ds \int\frac{7x^2+31x+54}{(x+1)(x^2+6x+11)}\dd x$.
\solution
The degree of the numerator is less than the degree of the denominator so we begin by applying \autoref{idea:partial_fraction}. We have:
\begin{align*}
\frac{7x^2+31x+54}{(x+1)(x^2+6x+11)} &= \frac{A}{x+1} + \frac{Bx+C}{x^2+6x+11}. \\
\intertext{Now clear the denominators.}
7x^2+31x+54 &= A(x^2+6x+11) + (Bx+C)(x+1).
\end{align*}
Again, we choose values of $x$ to eliminate terms in the polynomial.  If we choose $x=-1$,
\begin{align*}
 30&=6A + (-B+C)(0)\\
 A&= 5.
\end{align*}

Although none of the other terms can be zeroed out, we continue by letting $A=5$ and substituting helpful values of $x$. 
Choosing $x=0$, we notice
\begin{align*}
 54&= 55 +C \\
 C&= -1.
\end{align*}
Finally, choose $x=1$ (any value other than $-1$ and $0$ can be used, $1$ is easy to work with)
\begin{align*}
 92&=90 + (B-1)(2)\\
 B&= 2.
\end{align*}
Thus
\[
 \int\frac{7x^2+31x+54}{(x+1)(x^2+6x+11)}\dd x
 = \int\left(\frac{5}{x+1} + \frac{2x-1}{x^2+6x+11}\right)\dd x.
\]

The first term of this new integrand is easy to evaluate; it leads to a $5\ln\abs{x+1}$ term. The second term is not hard, but takes several steps and uses substitution techniques.

The integrand $\dfrac{2x-1}{x^2+6x+11}$ has a quadratic in the denominator and a linear term in the numerator. This leads us to try substitution. Let $u=x^2+6x+11$, so $du=(2x+6)\dd x$. The numerator is $2x-1$, not $2x+6$, but we can get a $2x+6$ term in the numerator by adding 0 in the form of ``$7-7$.''
\begin{align*}
	\frac{2x-1}{x^2+6x+11} &= \frac{2x-1+7-7}{x^2+6x+11} \\
	&= \frac{2x+6}{x^2+6x+11} - \frac{7}{x^2+6x+11}.
\end{align*}
We can now integrate the first term with substitution, yielding $\ln\abs{x^2+6x+11}$. The final term can be integrated using arctangent. First, complete the square in the denominator:
\[\frac{7}{x^2+6x+11} = \frac{7}{(x+3)^2+2}.\]
An antiderivative of the latter term can be found using \autoref{idea:trigsub} and substitution:
\[
 \int \frac7{x^2+6x+11}\dd x=\frac7{\sqrt2}\tan^{-1}\left(\frac{x+3}{\sqrt2}\right)+C.
\]

Let's start at the beginning and put all of the steps together.
\begin{align*}
	\int&\frac{7x^2+31x+54}{(x+1)(x^2+6x+11)}\dd x \\
	&= \int\left(\frac{5}{x+1} + \frac{2x-1}{x^2+6x+11}\right)\dd x \\
	&= \int\frac{5}{x+1}\dd x  + \int\frac{2x+6}{x^2+6x+11}\dd x -\int\frac{7}{x^2+6x+11}\dd x \\
	&= 5\ln\abs{x+1}+\ln\abs{x^2+6x+11}-\frac7{\sqrt2}\tan^{-1}\left(\frac{x+3}{\sqrt2}\right)+C.
\end{align*}
As with many other problems in calculus, it is important to remember that one is not expected to ``see'' the final answer immediately after seeing the problem. Rather, given the initial problem, we break it down into smaller problems that are easier to solve. The final answer is a combination of the answers of the smaller problems.
\end{example}

Partial Fraction Decomposition is an important tool when dealing with rational functions. Note that at its heart, it is a technique of algebra, not calculus, as we are rewriting a fraction in a new form. Regardless, it is very useful in the realm of calculus as it lets us evaluate a certain set of ``complicated'' integrals.  The next section will require the reader to determine an appropriate method for evaluating a variety of integrals.

\printexercises{exercises/06_04_exercises}
