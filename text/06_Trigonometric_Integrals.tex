
% this section gets 06_01_ex_27 ?

\section{Trigonometric Integrals}\label{sec:trigint}

Trigonometric functions are useful for describing periodic behavior. This section describes several techniques for finding antiderivatives of certain combinations of trigonometric functions.

\subsection{Integrals of the form \texorpdfstring{$\ds \int \sin^m x\cos^n x\dd x$}{∫(sin x)\^{}m (cos x)\^{}n dx}}

In learning the technique of Substitution, we saw the integral $\int \sin x\cos x\dd x$ in \autoref{ex_sub10}. The integration was not difficult, and one could easily evaluate the indefinite integral by letting $u=\sin x$ or by letting $u = \cos x$. This integral is easy since the power of both sine and cosine is 1.

We generalize this and consider integrals of the form $\int \sin^mx\cos^nx\dd x$, where $m,n$ are nonnegative integers. Our strategy for evaluating these integrals is to use the identity $\cos^2x+\sin^2x=1$ to convert high powers of one trigonometric function into the other, leaving a single sine or cosine term in the integrand. We summarize the general technique in the following Key Idea.

\youtubeVideo{zyg9k1je7Fg}{Trigonometric Integrals --- Part 2 of 6}


{%
\tcbset{grow to right by=4em} % 4 sufficient
\begin{keyidea}[Integrals Involving Powers of Sine and Cosine]\label{idea:trig_int_1}%
Consider $\ds \int \sin^mx\cos^nx\dd x$, where $m,n$ are nonnegative integers.\index{integration!of trig.\ powers}
\begin{enumerate}
	\item	If $m$ is odd, then $m=2k+1$ for some integer $k$. Rewrite \small
		\[
		\sin^mx = \sin^{2k+1}x = \sin^{2k}x\sin x = (\sin^2x)^k\sin x = (1-\cos^2x)^k\sin x.
		\]
		\normalsize Then \small
		\[
		\int \sin^mx\cos^nx\dd x = \int (1-\cos^2x)^k\sin x\cos^nx\dd x = -\int (1-u^2)^ku^n\dd u,
		\]
		\normalsize where $u = \cos x$ and $\dd u = -\sin x\dd x$. 
	\item	If $n$ is odd, then using substitutions similar to that outlined above we have \small
		\[\int \sin^mx\cos^nx\dd x = \int u^m(1-u^2)^k\dd u,\]
		\normalsize where $u = \sin x$ and $\dd u = \cos x\dd x$.
	\item	If both $m$ and $n$ are even, use the half-angle identities \small
		\[
		\cos^2x = \frac{1+\cos (2x)}{2} \quad \text{and}\quad \sin^2x = \frac{1-\cos(2x)}2
		\]
		\normalsize to reduce the degree of the integrand. Expand the result and apply the principles of this Key Idea again.
	\end{enumerate}
\end{keyidea}%
}

We practice applying \autoref{idea:trig_int_1} in the next examples.

\begin{example}[Integrating powers of sine and cosine]\label{ex_trigint1}%
Evaluate $\ds\int\sin^5x\cos^8x\dd x$.
\solution
The power of the sine factor is odd, so we rewrite $\sin^5x$ as
\[\sin^5x = \sin^4x\sin x = (\sin^2x)^2\sin x = (1-\cos^2x)^2\sin x.\]

Our integral is now $\ds \int (1-\cos^2x)^2\cos^8x\sin x\dd x$. Let $u = \cos x$, hence $\dd u = -\sin x\dd x$. Making the substitution and expanding the integrand gives
\begin{align*}
 \int (1-\cos^2x)^2\cos^8x\sin x\dd x
 &= -\int (1-u^2)^2u^8\dd u \\
 &= -\int \bigl(1-2u^2+u^4\bigr)u^8\dd u \\
 &= -\int \bigl(u^8-2u^{10}+u^{12}\bigr)\dd u \\
 &= -\frac19u^9 + \frac2{11}u^{11} - \frac1{13}u^{13} + C \\
 &=-\frac19\cos^9 x + \frac2{11}\cos^{11} x - \frac1{13}\cos^{13} x + C.
\end{align*}
\end{example}

\begin{example}[Integrating powers of sine and cosine]\label{ex_trigint2}%
Evaluate $\ds \int\sin^5x\cos^9x\dd x$.
\solution
Because the powers of both the sine and cosine factors are odd, we can apply the techniques of \autoref{idea:trig_int_1} to either power.
We choose to work with the power of the sine factor since that has a smaller exponent.
% We choose to work with the power of the cosine factor since the previous example used the sine factor's power.

We rewrite $\sin^5x$ as
\begin{align*}
 \sin^5x&=\sin^4x\sin x\\
 &=(1-\cos^2x)^2\sin x\\
 &=(1-2\cos^2x+\cos^4x)\sin x.
\end{align*}
This lets us rewrite the integral as
\[
\int\sin^5x\cos^9x\dd x=\int\bigl(1-2\cos^2x+\cos^4x\bigr)\sin x\cos^9x\dd x.
\]

Substituting and integrating with $u=\cos x$ and $\dd u=-\sin x\dd x$, we have
\begin{align*}
\int\bigl(1-2\cos^2x+\cos^4x\bigr)&\sin x\cos^9x\dd x\\
&=-\int\bigl(1-2u^2+u^4\bigr)u^9\dd u\\
&=-\int u^9-2u^{11}+u^{13}\dd u\\
&=-\frac1{10}u^{10}+\frac16u^{12}-\frac1{14}u^{14}+C\\
&=-\frac1{10}\cos^{10}x+\frac16\cos^{12}x-\frac1{14}\cos^{14}x+C.
\end{align*}

Instead, another approach would be to rewrite $\cos^9x$ as
\begin{align*} \cos^9 x &= \cos^8x\cos x \\
				&= (\cos^2x)^4\cos x \\
				&= (1-\sin^2x)^4\cos x \\
				&= (1-4\sin^2x+6\sin^4x-4\sin^6x+\sin^8x)\cos x.
\end{align*}

We rewrite the integral as 
\[\int\sin^5x\cos^9x\dd x = \int\bigl(\sin^5x\bigr)\bigl(1-4\sin^2x+6\sin^4x-4\sin^6x+\sin^8x\bigr)\cos x\dd x.\]

Now substitute and integrate, using $u = \sin x $ and $\dd u = \cos x\dd x$.
\begin{align*}
 \int & \bigl(\sin^5x\bigr)\bigl(1-4\sin^2x+6\sin^4x-4\sin^6x+\sin^8x\bigr)\cos x\dd x \\
 &=\int u^5(1-4u^2+6u^4-4u^6+u^8)\dd u \\
 &= \int\bigl(u^5-4u^7+6u^9-4u^{11}+u^{13}\bigr)\dd u \\
 &= \frac16u^6-\frac12u^8+\frac35u^{10}-\frac13u^{12}+\frac{1}{14}u^{14}+C\\
 &= \frac16\sin^6 x-\frac12\sin^8 x+\frac35\sin^{10} x-\frac13\sin^{12} x+\frac{1}{14}\sin^{14} x+C.
\end{align*}
%
\end{example}

\paragraph{Technology Note:} The work we are doing here can be a bit tedious, but the skills developed (problem solving, algebraic manipulation, etc.) are important. Nowadays problems of this sort are often solved using a computer algebra system. The powerful program \textit{Mathematica}\textsuperscript{\textregistered} integrates $\int \sin^5x\cos^9x\dd x$ as

{\small
\begin{multline*}
 f(x)=\\
 -\frac{45 \cos (2 x)}{16384}-\frac{5 \cos (4 x)}{8192}+\frac{19 \cos (6
   x)}{49152}+\frac{\cos (8 x)}{4096}-\frac{\cos (10 x)}{81920}-\frac{\cos (12
   x)}{24576}-\frac{\cos (14 x)}{114688},
\end{multline*}}
which clearly has a different form than our second answer in \autoref{ex_trigint2}, which is
%
\mtable{A plot of $f(x)$ and $g(x)$ from \autoref{ex_trigint2} and the Technology Note.}{fig:trigint2}{\pdftooltip{\begin{tikzpicture}
\begin{axis}[width=\marginparwidth,tick label style={font=\scriptsize},
axis y line=middle,axis x line=middle,name=myplot,axis on top,
ytick={-.002,.002,.004},yticklabels={$-0.002$,$0.002$,$0.004$},
ymin=-.003,ymax=0.005,xmin=-.1,xmax=3.15,scaled ticks=false]
% should we actually do this instead of coordinates?
\addplot [draw={\colortwo},thick,smooth] coordinates {(0,-0.0027879) (0.15708,-0.0027856) (0.31416,-0.0026798) (0.47124,-0.0020322) (0.62832,-0.00060997) (0.7854,0.00089518)(0.94248,0.0017233) (1.0996,0.0019485) (1.2566,0.0019734) (1.4137,0.0019741) (1.5708,0.0019741) (1.7279,0.0019741) (1.885,0.0019734) (2.042,0.0019485) (2.1991,0.0017233) (2.3562,0.00089518) (2.5133,-0.00060997) (2.6704,-0.0020322) (2.8274,-0.0026798) (2.9845,-0.0027856) (3.1416,-0.0027879)};
\draw (axis cs:2.6,0.004) node {\scriptsize $g(x)$};
\addplot [draw={\colorone},thick,smooth] coordinates {(0,0) (0.15708,0) (0.31416,0.00010807) (0.47124,0.0007557) (0.62832,0.0021779) (0.7854,0.003683) (0.94248,0.0045111)
(1.0996,0.0047364) (1.2566,0.0047612) (1.4137,0.0047619)
(1.5708,0.0047619) (1.7279,0.0047619) (1.885,0.0047612) (2.042,0.0047364) (2.1991,0.0045111) (2.3562,0.003683) (2.5133,0.0021779) (2.6704,0.0007557) (2.8274,0.00010807) (2.9845,0) (3.1416,0)};
\draw (axis cs:2.4,-0.002) node {\scriptsize $f(x)$};
\end{axis}
\node [right] at (myplot.right of origin) {\scriptsize $x$};
\node [above] at (myplot.above origin) {\scriptsize $y$};
\end{tikzpicture}}{ALT-TEXT-TO-BE-DETERMINED}}
\[
 g(x)=\frac16\sin^6 x-\frac12\sin^8 x+\frac35\sin^{10} x-\frac13\sin^{12} x+\frac{1}{14}\sin^{14} x.
\]
\autoref{fig:trigint2} shows a graph of $f$ and $g$; they are clearly not equal, but they differ \emph{only by a constant}: $g(x) = f(x) + C$ for some constant $C$. We have two different antiderivatives of the same function, meaning both answers are correct.\bigskip

%This is a common trigonometric integral.\bigskip

\begin{example}[Integrating powers of sine and cosine]\label{ex_sub8}%
Evaluate $\ds \int \sin^2x\dd x$.
\solution
The power of sine is even so we employ a half-angle identity, algebra and a u- substitution as follows:
\begin{align*}
	\int \sin^2x\dd x
	&= \int \frac{1-\cos(2x)}2\dd x \\
	&= \frac12\int 1-\cos(2x)\dd x \\
	&= \frac12\left(x - \frac12\sin(2x)\right)+C \\
	&= \frac12x - \frac14\sin(2x) + C.
\end{align*}
\end{example}

\begin{example}[Integrating powers of sine and cosine]\label{ex_trigint3}%
Evaluate $\ds\int\cos^4x\sin^2x\dd x$.
\solution
The powers of sine and cosine are both even, so we employ the half-angle formulas and algebra as follows.
\begin{align*}
	\int \cos^4x\sin^2x\dd x
	&= \int\left(\frac{1+\cos(2x)}{2}\right)^2\left(\frac{1-\cos(2x)}2\right)\dd x \\
	&= \int\frac{1+2\cos(2x)+\cos^2(2x)}4\cdot\frac{1-\cos(2x)}2\dd x\\
	&=	\int \frac18\bigl(1+\cos(2x)-\cos^2(2x)-\cos^3(2x)\bigr)\dd x
\end{align*}
The $\cos(2x)$ term is easy to integrate.
%, especially with \autoref{idea:linearsub}.
The $\cos^2(2x)$ term is another trigonometric integral with an even power, requiring the half-angle formula again. The $\cos^3(2x)$ term is a cosine function with an odd power, requiring a substitution as done before. We integrate each in turn below.

\begin{gather*}
\int\cos(2x)\dd x = \frac12\sin(2x)+C.\\
\int\cos^2(2x)\dd x = \int \frac{1+\cos(4x)}2\dd x
= \frac12\bigl(x+\frac14\sin(4x)\bigr)+C.
\end{gather*}

Finally, we rewrite $\cos^3(2x)$ as
\[\cos^3(2x) = \cos^2(2x)\cos(2x) = \bigl(1-\sin^2(2x)\bigr)\cos(2x).\]
Letting $u=\sin(2x)$, we have $\dd u = 2\cos(2x)\dd x$, hence
\begin{align*}
\int \cos^3(2x)\dd x &= \int\bigl(1-\sin^2(2x)\bigr)\cos(2x)\dd x\\
							&= \int \frac12(1-u^2)\dd u\\
							&= \frac12\Bigl(u-\frac13u^3\Bigr)+C\\
							&=	\frac12\Bigl(\sin(2x)-\frac13\sin^3(2x)\Bigr)+C
\end{align*}

Putting all the pieces together, we have
\begin{align*}
	\int &\cos^4x\sin^2x\dd x \\
	&=\int \frac18\bigl(1+\cos(2x)-\cos^2(2x)-\cos^3(2x)\bigr)\dd x \\
	&= \frac18
	\Bigl[x+\frac12\sin(2x)-\frac12\bigl(x+\frac14\sin(4x)\bigr)
	-\frac12\Bigl(\sin(2x)-\frac13\sin^3(2x)\Bigr)\Bigr]
	+C \\
	&=\frac18\Bigl[\frac12x-\frac18\sin(4x)+\frac16\sin^3(2x)\Bigr]+C.
\end{align*}
\end{example}

The process above was a bit long and tedious, but being able to work a problem such as this from start to finish is important.

\subsection{Integrals of the form \texorpdfstring{$\ds\int\tan^mx\sec^nx\dd x$}{∫(tan x)\^{}m (sec x)\^{}n dx}}

When evaluating integrals of the form $\int \sin^mx\cos^nx\dd x$, the Pythagorean Theorem allowed us to convert even powers of sine into even powers of cosine, and vice versa. If, for instance, the power of sine was odd, we pulled out one $\sin x$ and converted the remaining even power of $\sin x$ into a function using powers of $\cos x$, leading to an easy substitution.

The same basic strategy applies to integrals of the form $\int \tan^mx\sec^n x\dd x$, albeit a bit more nuanced. The following three facts will prove useful:
\begin{itemize}
\item $\frac{\dd}{\dd x}(\tan x) = \sec^2x$, 
\item $\frac{\dd}{\dd x}(\sec x) = \sec x\tan x$ , and 
\item	$1+\tan^2x = \sec^2x$ (the Pythagorean Theorem).
\end{itemize}

If the integrand can be manipulated to separate a $\sec^2x$ term with the remaining secant power even, or if a $\sec x\tan x$ term can be separated with the remaining $\tan x$ power even, the Pythagorean Theorem can be employed, leading to a simple substitution. This strategy is outlined in the following Key Idea.

{
\tcbset{grow to right by=13em}
\begin{keyidea}[Integrals Involving Powers of Tangent and Secant]\label{idea:trig_int_2}%
Consider $\ds\int\tan^mx\sec^nx\dd x$, where $m$ and $n$ are nonnegative integers.\index{integration!of trig. powers}
\begin{enumerate}
\item		If $n$ is even, then $n=2k$ for some integer $k$. Rewrite $\sec^nx$ as 
\[\sec^nx = \sec^{2k}x = \sec^{2k-2}x\sec^2x = (1+\tan^2x)^{k-1}\sec^2x.\]
Then
\[
\int\tan^mx\sec^nx\dd x=\int\tan^mx(1+\tan^2x)^{k-1}\sec^2x\dd x = \int u^m(1+u^2)^{k-1}\dd u,
\]
where $u = \tan x$ and $\dd u = \sec^2x\dd x$.

\item		If $m$ is odd and $n>0$, then $m=2k+1$ for some integer $k$. Rewrite $\tan^mx\sec^nx$ as
\[
\tan^mx\sec^nx = \tan^{2k+1}x\sec^nx = \tan^{2k}x\sec^{n-1}x\sec x\tan x = (\sec^2x-1)^k\sec^{n-1}x\sec x\tan x.
\]
Then
\[
\int\tan^mx\sec^nx\dd x=\int(\sec^2x-1)^k\sec^{n-1}x\sec x\tan x\dd x = \int(u^2-1)^ku^{n-1}\dd u,
\]
where $u = \sec x$ and $\dd u = \sec x\tan x\dd x$.

\item If $n$ is odd and $m$ is even, then $m=2k$ for some integer $k$. Convert $\tan^mx $ to $(\sec^2x-1)^k$. Expand the new integrand and use Integration By Parts, with $\dd v = \sec^2x\dd x$.

\item		If $m$ is even and $n=0$, rewrite $\tan^mx$ as
\[
\tan^mx = \tan^{m-2}x\tan^2x = \tan^{m-2}x(\sec^2x-1) = \tan^{m-2}\sec^2x-\tan^{m-2}x.
\]
So
\[
\int\tan^mx\dd x = \underbrace{\int\tan^{m-2}x\sec^2x\dd x}_{\text{\small apply rule \#1}}\quad - \underbrace{\int\tan^{m-2}x\dd x}_{\text{\small apply rule \#4 again}}.
\]

\end{enumerate}
\end{keyidea}
}

The techniques described in items 1 and 2 of \autoref{idea:trig_int_2} are relatively straightforward, but the techniques in items 3 and 4 can be rather tedious. A few examples will help with these methods.

\begin{example}[Integrating powers of tangent and secant]\label{ex_trigint5}%
Evaluate $\ds\int \tan^2x\sec^6x\dd x$.
\solution
Since the power of secant is even, we use rule \#1 from \autoref{idea:trig_int_2} and pull out a $\sec^2x$ in the integrand. We convert the remaining powers of secant into powers of tangent.
\begin{align*}
\int \tan^2x\sec^6x\dd x &= \int\tan^2x\sec^4x\sec^2x\dd x \\
		&= \int \tan^2x\bigl(1+\tan^2x\bigr)^2\sec^2x\dd x \\
\intertext{Now substitute, with $u=\tan x$, with $\dd u = \sec^2x\dd x$.}
		&=\int u^2\bigl(1+u^2\bigr)^2\dd u\\
\intertext{We leave the integration and subsequent substitution to the reader. The final answer is}
		&=\frac13\tan^3x+\frac25\tan^5x+\frac17\tan^7x+C.
\end{align*}
\end{example}

We derived integrals for tangent and secant in \autoref{sec:substitution} and will regularly use them when evaluating integrals of the form $\tan^m x \sec^n x \dd x$.  As a reminder:
\begin{align*}
 \int\tan x\dd x &=\ln\abs{\sec x}+C \\
 \int\sec x\dd x &=\ln\abs{\sec x+\tan x}+C
\end{align*}

\begin{example}[Integrating powers of tangent and secant]\label{ex_trigint6}%
Evaluate $\ds\int \sec^3x\dd x$.
\solution
We apply rule \#3 from \autoref{idea:trig_int_2} as the power of secant is odd and the power of tangent is even (0 is an even number). We use Integration by Parts; the rule suggests letting $\dd v = \sec^2x\dd x$, meaning that $u = \sec x$. \\
\noindent\begin{minipage}[t]{\linewidth}\noindent%
\captionsetup{type=figure}%
\[
\begin{aligned}
u&= \sec x & \dd v&=\sec^2 x\dd x\\
\dd u&= \text{?} & v&=\text{?}
\end{aligned}
\qquad\Rightarrow\qquad
\begin{aligned}
u&= \sec x & \dd v&=\sec^2 x\dd x\\
\dd u&= \sec x\tan x\dd x & v&=\tan x
\end{aligned}
\]
\caption{Setting up Integration by Parts.}\label{fig:trigint1}
\end{minipage}

Employing Integration by Parts, we have
\begin{align*}
\int \sec^3x\dd x
 	&=	\int \underbrace{\sec x}_u\cdot\underbrace{\sec^2 x\dd x}_{\dd v}\\
	&=	\sec x\tan x - \int \sec x\tan^2x\dd x. \\
\intertext{This new integral also requires applying rule \#3 of \autoref{idea:trig_int_2}:}
	&= \sec x\tan x - \int \sec x \bigl(\sec^2 x-1\bigr)\dd x\\
	&=	\sec x\tan x - \int \sec^3x\dd x + \int \sec x\dd x \\
	&= \sec x\tan x -\int \sec^3x\dd x + \ln\abs{\sec x+\tan x}
\end{align*}
%
\mnote{\textbf{Note:} Remember that in \autoref{ex_sub7}, we found that $\int\sec x\dd x=\ln\abs{\sec x+\tan x}+C$}
%
In previous applications of Integration by Parts, we have seen where the original integral has reappeared in our work. We resolve this by adding $\int \sec^3x\dd x$ to both sides, giving:
\begin{align*}
2\int \sec^3x\dd x &= \sec x\tan x + \ln\abs{\sec x+\tan x} \\
\int \sec^3x\dd x &= \frac12\Bigl(\sec x\tan x + \ln\abs{\sec x+\tan x}\Bigr)+C.
\end{align*}
\end{example}

We give one more example.

\begin{example}[Integrating powers of tangent and secant]\label{ex_trigint7}%
Evaluate $\ds\int\tan^6x\dd x$.
\solution
We employ rule \#4 of \autoref{idea:trig_int_2}. 
\begin{align*}
	\int \tan^6x\dd x
	&= \int \tan^4x\tan^2x\dd x \\
	&= \int\tan^4x\bigl(\sec^2x-1\bigr)\dd x\\
	&= \int\tan^4x\sec^2x\dd x - \int\tan^4x\dd x \\
\intertext{We integrate the first integral with substitution, $u=\tan x$ and $\dd u=\sec^2x\dd x$; and the second by employing rule \#4 again.}
	&= \int u^4\dd u-\int\tan^2 x\tan^2 x\dd x \\
	&=	\frac15\tan^5x-\int\tan^2x\bigl(\sec^2x-1\bigr)\dd x \\
	&= \frac15\tan^5x -\int\tan^2x\sec^2x\dd x + \int\tan^2x\dd x\\
\intertext{Again, use substitution for the first integral and rule \#4 for the second.}
	&= \frac15\tan^5x-\frac13\tan^3x+\int\bigl(\sec^2x-1\bigr)\dd x \\
	&=	 \frac15\tan^5x-\frac13\tan^3x+\tan x - x+C.
\end{align*}
\end{example}

\subsection{Integrals of the form \texorpdfstring{$\ds\int\cot^mx\csc^nx\dd x$}{∫(cot x)\^{}m (csc x)\^{}n dx}}

Not surprisingly, evaluating integrals of the form $\int\cot^mx\csc^nx\dd x$ is similar to evaluating $\int\tan^mx\sec^nx\dd x$. The guidelines from \autoref{idea:trig_int_2} and the following three facts will be useful:
\begin{align*}
 \frac{\dd}{\dd x}(\cot x) &= -\csc^2x \\
 \frac{\dd}{\dd x}(\csc x) &= -\csc x\cot x,\qquad\text{and} \\
 \csc^2 x &= \cot^2x+1
\end{align*}

\begin{example}[Integrating powers of cotangent and cosecant]\label{ex_int_cot_csc}%
Evaluate $\ds\int\cot^2x\csc^4x\dd x$
\solution
Since the power of cosecant is even we will let $u=\cot x$ and save a $\csc^2x$ for the resulting $\dd u=-\csc^2x\dd x$.
\begin{align*}
 \int\cot^2x\csc^4x\dd x
 &=\int\cot^2x\csc^2x\csc^2x\dd x \\
 &=\int\cot^2x(1+\cot^2x)\csc^2x\dd x \\
 &=-\int u^2(1+u^2)\dd u.
\end{align*}
The integration and substitution required to finish this example are similar to that of previous examples in this section. The result is
\[-\frac13\cot^3x-\frac15\cot^5x+C.\]
\end{example}

\subsection{Integrals of the form \texorpdfstring{$\ds \int\sin(mx)\sin(nx)\dd x,$ $\ds\int \cos(mx)\cos(nx)\dd x$, and $\ds\int \sin(mx)\cos(nx)\dd x$.}{∫sin(mx)sin(nx)dx, ∫cos(mx)cos(nx)dx, and ∫sin(mx)cos(nx)dx}}

Functions that contain products of sines and cosines of differing periods are important in many applications including the analysis of sound waves. Integrals of the form 
\[
\int\sin(mx)\sin(nx)\dd x,\quad \int \cos(mx)\cos(nx)\dd x \quad \text{and}\quad\int \sin(mx)\cos(nx)\dd x
\]
are best approached by first applying the Product to Sum Formulas of Trigonometry found in the back cover of this text, namely
\begin{align*}
\sin(mx)\sin(nx) &= \frac12\Bigl[\cos\bigl((m-n)x\bigr)-\cos\bigl((m+n)x\bigr)\Bigr] \\
\cos(mx)\cos(nx) &= \frac12\Bigl[\cos\bigl((m-n)x\bigr)+\cos\bigl((m+n)x\bigr)\Bigr] \\
\sin(mx)\cos(nx) &=	\frac12\Bigl[\sin\bigl((m-n)x\bigl)+\sin\bigl((m+n)x\bigr)\Bigr]
\end{align*}

\begin{example}[Integrating products of $\sin(mx)$ and $\cos(nx)$]\label{ex_trigint4}%
Evaluate $\ds\int\sin(5x)\cos(2x)\dd x$.
\solution
The application of the formula and subsequent integration are straightforward:
\begin{align*}
	\int\sin(5x)\cos(2x)\dd x
	&= \int \frac12\Bigl[\sin(3x)+\sin(7x)\Bigr]\dd x \\
	&= -\frac16\cos(3x) - \frac1{14}\cos(7x) + C.
\end{align*}
\end{example}

\subsection{Integrating other combinations of trigonometric functions}

Combinations of trigonometric functions that we have not discussed in this chapter are evaluated by applying algebra, trigonometric identities and other integration strategies to create an equivalent integrand that we can evaluate. To evaluate ``crazy'' combinations, those not readily manipulated into a familiar form, one should use integral tables. A table of ``common crazy'' combinations can be found at the end of this text.

These latter examples were admittedly long, with repeated applications of the same rule. Try to not be overwhelmed by the length of the problem, but rather admire how robust this solution method is. A trigonometric function of a high power can be systematically reduced to trigonometric functions of lower powers until all antiderivatives can be computed. 

The next section introduces an integration technique known as Trigonometric Substitution, a clever combination of Substitution and the Pythagorean Theorem.

\printexercises{exercises/06-03-exercises}
