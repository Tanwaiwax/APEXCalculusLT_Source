\section{Area Between Curves}\label{sec:ABC}

We are often interested in knowing the area of a region. Forget momentarily that we addressed this already in \autoref{sec:FTC} and approach it instead using the technique described in \autoref{idea:app_of_defint}. 

\bgroup % leaving newcommand in the mtable causes errors
\newcommand{\f}[1]{.25*sin(deg(2*#1))+1.25}
\newcommand{\g}[1]{.25*(#1-2)^2+.2}
\mtable[-3in]{Subdividing a region into vertical slices and approximating the areas with rectangles.}{fig:abcintro}{\small%
\begin{tikzpicture}
\begin{axis}[width=1.16\marginparwidth,tick label style={font=\scriptsize},
axis y line=middle,axis x line=middle,name=myplot,axis on top,xtick=\empty,
extra x ticks={.5,3},extra x tick labels={$a$,$b$},ytick=\empty,
ymin=-.2,ymax=2,xmin=-.5,xmax=3.5]
\addplot[smooth,draw={\coloronefill},fill={\coloronefill},area style,domain=.5:3]
 {.25*sin(deg(2*x))+1.25}\closedcycle;
\addplot[smooth,draw=white,fill=white,area style,domain=.5:3]
 {.25*(x-2)^2+.2}\closedcycle;
\addplot [smooth,thick, draw={\colorone},domain=-.25:3.25] {.25*sin(deg(2*x))+1.25} node [shift={(5pt,7pt)} ,black] {\scriptsize $f(x)$};
\addplot [smooth,thick, draw={\colorone},domain=-.25:3.25] {.25*(x-2)^2+.2}node [shift={(5pt,7pt)} ,black] {\scriptsize $g(x)$};
\end{axis}
\node [right] at (myplot.right of origin) {\scriptsize $x$};
\node [above] at (myplot.above origin) {\scriptsize $y$};
\end{tikzpicture}
\\ (a)
\\
\begin{tikzpicture}
\begin{axis}[width=1.16\marginparwidth,tick label style={font=\scriptsize},
axis y line=middle,axis x line=middle,name=myplot,axis on top,xtick=\empty,
extra x ticks={.5,3},extra x tick labels={$a$,$b$},ytick=\empty,
ymin=-.2,ymax=2,xmin=-.5,xmax=3.5]
\addplot[smooth,draw={\coloronefill},fill={\coloronefill},area style,domain=.5:3]
 {.25*sin(deg(2*x))+1.25}\closedcycle;
\addplot[smooth,draw=white,fill=white,area style,domain=.5:3]
 {.25*(x-2)^2+.2}\closedcycle;
\addplot [smooth,thick, draw={\colorone},domain=-.25:3.25] {.25*sin(deg(2*x))+1.25} node [shift={(5pt,7pt)} ,black] {\scriptsize $f(x)$};
\addplot [smooth,thick, draw={\colorone},domain=-.25:3.25] {.25*(x-2)^2+.2}node [shift={(5pt,7pt)} ,black] {\scriptsize $g(x)$};
\foreach\x in{.5,.75,...,3} {
 \edef\temp{\noexpand\draw[thick]
  (axis cs:\x,{.25*(\x-2)^2+.2})--(axis cs:\x,{.25*sin(deg(2*\x))+1.25});}\temp
}
\end{axis}
\node [right] at (myplot.right of origin) {\scriptsize $x$};
\node [above] at (myplot.above origin) {\scriptsize $y$};
\end{tikzpicture}
\\(b)
\\
\begin{tikzpicture}
\begin{axis}[width=1.16\marginparwidth,tick label style={font=\scriptsize},
axis y line=middle,axis x line=middle,name=myplot,axis on top,xtick=\empty,
extra x ticks={.5,3},extra x tick labels={$a$,$b$},ytick=\empty,
ymin=-.2,ymax=2,xmin=-.5,xmax=3.5]
\addplot[smooth,draw={\coloronefill},fill={\coloronefill},area style,domain=.5:3]
 {\f{x}}\closedcycle;
\addplot[smooth,draw=white,fill=white,area style,domain=.5:3]
 {\g{x}}\closedcycle;
% {.25*(x-2)^2+.2}\closedcycle;
\addplot [smooth,thick, draw={\colorone},domain=-.25:3.25] {\f{x}} node [shift={(5pt,7pt)} ,black] {\scriptsize $f(x)$};
\addplot [smooth,thick, draw={\colorone},domain=-.25:3.25] {\g{x}}node [shift={(5pt,7pt)} ,black] {\scriptsize $g(x)$};
\foreach\x in{.5,.75,...,2.75} {
 \edef\c{(\x+.25*(1+sin(deg(3*\x)))/2)}
 \edef\temp{
  \noexpand\draw[thick](axis cs:\x,{\f{\c}})rectangle(axis cs:{\x+.25},{\g{\c}});
  \noexpand\filldraw(axis cs:{\c},{\f{\c}})circle(1.3pt);
  \noexpand\filldraw(axis cs:{\c},{\g{\c}})circle(1.3pt);
 }\temp
}
\end{axis}
\node [right] at (myplot.right of origin) {\scriptsize $x$};
\node [above] at (myplot.above origin) {\scriptsize $y$};
\end{tikzpicture}
\\ (c)
}
\egroup

Let $Q$ be the area of a region bounded by continuous functions $f$ and $g$. If we break the region into many subregions, we have an obvious equation:
\begin{center}
Total Area = sum of the areas of the subregions.
\end{center}
The issue to address next is how to systematically break a region into subregions. A graph will help. Consider \autoref{fig:abcintro} (a) where a region between two curves is shaded. While there are many ways to break this into subregions, one particularly efficient way is to ``slice'' it vertically, as shown in \autoref{fig:abcintro} (b), into $n$ equally spaced slices.

We now approximate the area of a slice. Again, we have many options, but using a rectangle seems simplest. Picking any $x$-value $c_i$ in the $i^\text{ th}$ slice, we set the height of the rectangle to be $f(c_i)-g(c_i)$, the difference of the corresponding $y$-values. The width of the rectangle is a small difference in $x$-values, which we represent with $\Delta x$. \autoref{fig:abcintro} (c) shows sample points $c_i$ chosen in each subinterval and appropriate rectangles drawn.
%(Each of these rectangles represents a differential element.)
Each slice has an area approximately equal to $\bigl(f(c_i)-g(c_i)\bigr)\Delta x$; hence, the total area is approximately the Riemann Sum\vspace{-.5\baselineskip}
\[Q \approx \sum_{i=1}^n \bigl(f(c_i)-g(c_i)\bigr)\Delta x.\]
Taking the limit as $n\to \infty$ gives the exact area as $\int_a^b \bigl(f(x)-g(x)\bigr)\dd x.$

\begin{theorem}[Area Between Curves]\label{thm:areabetweencurves}
Let $f(x)$ and $g(x)$ be continuous functions defined on $[a,b]$ where $f(x)\geq g(x)$ for all $x$ in $[a,b]$. The area of the region bounded by the curves $y=f(x)$, $y=g(x)$ and the lines $x=a$ and $x=b$ is \index{integration!area between curves}
\[\int_a^b \bigl(f(x)-g(x)\bigr)\dd x.\]
\end{theorem}

Often, we do not know which function is greater (or they switch within the domain of integration).  If so, we can say that the area is $\int_a^b\abs{f(x)-g(x)}\dd x$, which may involve dividing the domain of integration into pieces.

\youtubeVideo{DRFyNHdVgUA}{Finding Areas Between Curves}

\begin{example}[Finding area enclosed by curves]\label{ex_abc1}
Find the area of the region bounded by $f(x) = \sin x+2$, $g(x) = \frac12\cos (2x)-1$, $x=0$ and $x=4\pi$, as shown in \autoref{fig:abc1}.
\solution
\mtable{Graphing an enclosed region in \autoref{ex_abc1}.}{fig:abc1}{\begin{tikzpicture}
\begin{axis}[width=\marginparwidth,tick label style={font=\scriptsize},
axis y line=middle,axis x line=middle,name=myplot,axis on top,extra x ticks={12.57},
extra x tick labels={$4\pi$},ymin=-2,ymax=3.5,xmin=-.5,xmax=13.5]
\addplot[smooth,draw={\coloronefill},fill={\coloronefill},area style,domain=0:4*pi]
 {.5*cos(deg(2*x))-1.2}\closedcycle;
\addplot [smooth,draw={\coloronefill},fill={\coloronefill},area style,domain=0:4*pi]
 {sin(deg(x))+2} \closedcycle;
\addplot [smooth,thick, draw={\colorone},domain=0:4*pi,samples=40]
 {sin(deg(x))+2} node [shift={(-15pt,7pt)} ,black] {\scriptsize $f(x)$};
\addplot [smooth,thick, draw={\colorone},domain=0:4*pi,samples=40]
 {.5*cos(deg(2*x))-1.2}node [shift={(-25pt,-16pt)} ,black] {\scriptsize $g(x)$};
\end{axis}
\node [right] at (myplot.right of origin) {\scriptsize $x$};
\node [above] at (myplot.above origin) {\scriptsize $y$};
\end{tikzpicture}}
%
The graph verifies that the upper boundary of the region is given by $f$ and the lower bound is given by $g$. Therefore the area of the region is the value of the integral
\begin{align*} 
	\int_0^{4\pi} \bigl(f(x)- g(x)\bigr)\dd x
	& = \int_0^{4\pi} \Bigl(\sin x+2 - \bigl(\frac12\cos (2x)-1\bigr)\Bigr)\dd x \\
	&= -\cos x -\frac14\sin(2x)+3x\Big|_0^{4\pi}\\
	&=	12\pi %\approx 37.7
	\ \text{units}^2.
\end{align*}
\end{example}

% originally in FTC section
\begin{example}[Finding area between curves]\label{ex_ftc6}
Find the area of the region enclosed by $y=x^2+x-5$ and $y=3x-2$.
\solution
It will help to sketch these two functions, as done in \autoref{fig:ftc6}. 
\mtable{Sketching the region enclosed by $y=x^2+x-5$ and $y=3x-2$ in \autoref{ex_ftc6}.}{fig:ftc6}{\begin{tikzpicture}
\begin{axis}[width=\marginparwidth,tick label style={font=\scriptsize},
axis y line=middle,axis x line=middle,name=myplot,axis on top,xtick={-2,-1,1,2,3,4},
ytick={5,10,15},ymin=-10,ymax=16,xmin=-2.5,xmax=4.5]
\addplot [smooth,thick, draw={\colorone},domain=-2:4] {x^2+x-5}; 
\draw (axis cs: 2.5,15) node [black] {\scriptsize $y=x^2+x-5$};
\addplot [smooth,thick, draw={\colorone},domain=-2:4] {3*x-2}; 
\draw (axis cs: -1.5,-9) node [black] {\scriptsize $y=3x-2$};
\end{axis}
\node [right] at (myplot.right of origin) {\scriptsize $x$};
\node [above] at (myplot.above origin) {\scriptsize $y$};
\end{tikzpicture}}
The region whose area we seek is completely bounded by these two functions; they seem to intersect at $x=-1$ and $x=3$. To check, set $x^2+x-5=3x-2$ and solve for $x$:\vspace{-.5\baselineskip}
\begin{align*}
	x^2+x-5 &= 3x-2 \\
	(x^2+x-5) - (3x-2) &= 0\\
	x^2-2x-3 &= 0\\
	(x-3)(x+1) &= 0\\
	x&=-1,\ 3.
\end{align*}
Following \autoref{thm:areabetweencurves}, the area is 
\begin{align*}
	\int_{-1}^3\bigl(3x-2 -(x^2+x-5)\bigr)\dd x &= \int_{-1}^3 (-x^2+2x+3)\dd x \\
	&=\left.\left(-\frac13x^3+x^2+3x\right)\right|_{-1}^3 \\
	&=-\frac13(27)+9+9-\left(\frac13+1-3\right)\\
	&= 10\frac23 = 10.\overline{6}
\end{align*}
\end{example}

\mtable{Graphing a region enclosed by two functions in \autoref{ex_abc2}.}{fig:abc2}{\begin{tikzpicture}
\begin{axis}[width=1.16\marginparwidth,tick label style={font=\scriptsize},
axis y line=middle,axis x line=middle,name=myplot,axis on top,xtick={1,2,3,4},
ymin=-5,ymax=3.5,xmin=-.5,xmax=4.7]
\addplot[smooth,draw={\coloronefill},fill={\coloronefill},area style,domain=1:2]
 {x^3-7*x^2+12*x-3}\closedcycle;
\addplot[smooth,draw=white,fill=white,area style,domain=1:2]
 {5-2*x}\closedcycle;
\addplot[smooth,draw={\coloronefill},fill={\coloronefill},area style,domain=2:2.5]
 {5-2*x}\closedcycle;
\addplot[smooth,draw=white,fill=white,area style,domain=2:2.2391]
 {x^3-7*x^2+12*x-3}\closedcycle;
\addplot[smooth,draw={\coloronefill},fill={\coloronefill},area style,domain=2.2391:4]
 {x^3-7*x^2+12*x-3}\closedcycle;
\addplot[smooth,draw=white,fill=white,area style,domain=2.5:4]
 {5-2*x}\closedcycle;
\addplot [smooth,thick, draw={\colorone},domain=-.2:4.5,samples=30] {x^3-7*x^2+12*x-3};
\addplot [smooth,thick, draw={\colorone},domain=0:4.5] {5-2*x};
\end{axis}
\node [right] at (myplot.right of origin) {\scriptsize $x$};
\node [above] at (myplot.above origin) {\scriptsize $y$};
\end{tikzpicture}}

\begin{example}[Finding total area enclosed by curves]\label{ex_abc2}
Find the total area of the region enclosed by the functions $f(x) = -2x+5$ and $g(x) = x^3-7x^2+12x-3$ as shown in \autoref{fig:abc2}.
\solution
A quick calculation shows that $f=g$ at $x=1, 2$ and 4. One can proceed thoughtlessly by computing $\ds \int_1^4\bigl(f(x)-g(x)\bigr)\dd x$, but this ignores the fact that on $[1,2]$, $g(x)>f(x)$. (In fact, the thoughtless integration returns $-9/4$, hardly the expected value of an \emph{area}.) Thus we compute the total area by breaking the interval $[1,4]$ into two subintervals, $[1,2]$ and $[2,4]$ and using the proper integrand in each.
\begin{align*}
	\text{Total Area}
	&= \int_1^2 \bigl(g(x)-f(x)\bigr)\dd x + \int_2^4\bigl(f(x)-g(x)\bigr)\dd x\\
	&= \int_1^2 \bigl(x^3-7x^2+14x-8\bigr) \dd x
	+ \int_2^4\bigl(-x^3+7x^2-14x+8\bigr)\dd x\\
	&= \frac5{12} + \frac83
	 %\\&
	= \frac{37}{12} % \approx 3.083
	\ \text{units}^2.
\end{align*}
\end{example}

The previous example makes note that we are expecting area to be \emph{positive}. When first learning about the definite integral, we interpreted it as ``signed area under the curve,'' allowing for ``negative area.'' That doesn't apply here; area is to be positive.

The previous example also demonstrates that we often have to break a given region into subregions before applying \autoref{thm:areabetweencurves}. The following example shows another situation where this is applicable, along with an alternate view of applying the Theorem.

\mtable{Graphing a region for \autoref{ex_abc3}.}{fig:abc3}{\begin{tikzpicture}
\begin{axis}[width=1.16\marginparwidth,tick label style={font=\scriptsize},
axis y line=middle,axis x line=middle,name=myplot,axis on top,xtick={1,2},
ymin=-.1,ymax=3.5,xmin=-.1,xmax=2.5]
\addplot[draw={\coloronefill},fill={\coloronefill},area style,domain=0:1]{sqrt(x)+2}\closedcycle;
\addplot[draw={\coloronefill},fill={\coloronefill},area style,domain=1:2]{3-(x-1)^2}\closedcycle;
\draw[fill=white,draw=white](axis cs:0,0)rectangle(axis cs:2,2);
\addplot[draw={\colorone},domain=0:1]{sqrt(x)+2};
\addplot[draw={\colorone},domain=1:2]{3-(x-1)^2};
\draw[draw={\colorone}](axis cs:0,2)--(axis cs:2,2);
\draw (axis cs:.5,3.1) node {\scriptsize $y=\sqrt{x}+2$} (axis cs:1.8,3.2) node {\scriptsize $y=-(x-1)^2+3$};
\end{axis}
\node [right] at (myplot.right of origin) {\scriptsize $x$};
\node [above] at (myplot.above origin) {\scriptsize $y$};
\end{tikzpicture}}

\begin{example}[Finding area: integrating with respect to $y$]\label{ex_abc3}
Find the area of the region enclosed by the functions $y=\sqrt{x}+2$, $y=-(x-1)^2+3$ and $y=2$, as shown in \autoref{fig:abc3}.
\solution
We give two approaches to this problem. In the first approach, we notice that the region's ``top'' is defined by two different curves. On $[0,1]$, the top function is $y=\sqrt{x}+2$; on $[1,2]$, the top function is $y=-(x-1)^2+3$. Thus we compute the area as the sum of two integrals:

\begin{align*}
	\text{Total Area}
	&= \int_0^1 \Bigl(\bigl(\sqrt{x}+2\bigr)-2\Bigr)\dd x + \int_1^2 \Bigl(\bigl(-(x-1)^2+3\bigr)-2\Bigr)\dd x \\
	&= 2/3 + 2/3
	%\\&
	=4/3.
\end{align*}

The second approach is clever and very useful in certain situations. We are used to viewing curves as functions of $x$; we input an $x$-value and a $y$-value is returned. Some curves can also be described as functions of $y$: input a $y$-value and an $x$-value is returned. We can rewrite the equations describing the boundary by solving for $x$:\vspace{-.3\baselineskip}
\begin{align*}
y=\sqrt{x}+2 & \quad\Rightarrow\quad x=(y-2)^2 \\
y=-(x-1)^2+3 & \quad\Rightarrow\quad x=\sqrt{3-y}+1.
\end{align*}

\mtable{The region used in \autoref{ex_abc3} with boundaries relabeled as functions of $y$.}{fig:abc3b}{\begin{tikzpicture}
\begin{axis}[width=1.16\marginparwidth,tick label style={font=\scriptsize},
axis y line=middle,axis x line=middle,name=myplot,axis on top,xtick={1,2},
ymin=-.1,ymax=3.5,xmin=-.1,xmax=2.5]
\addplot[draw={\coloronefill},fill={\coloronefill},area style,domain=0:1]{sqrt(x)+2}\closedcycle;
\addplot[draw={\coloronefill},fill={\coloronefill},area style,domain=1:2]{3-(x-1)^2}\closedcycle;
\draw[fill=white,draw=white](axis cs:0,0)rectangle(axis cs:2,2);
\addplot[draw={\colorone},domain=0:1]{sqrt(x)+2};
\addplot[draw={\colorone},domain=1:2]{3-(x-1)^2};
\draw[draw={\colorone}](axis cs:0,2)--(axis cs:2,2);
\draw (axis cs:.5,3.1) node {\scriptsize $x=(y-2)^2$} (axis cs:1.8,3.2) node {\scriptsize $x=\sqrt{3-y}+1$};
\filldraw [draw={\colortwo},fill={\colortwofill},thick] (axis cs: 0.0625,2.2) rectangle (axis cs:1.867,2.3);
\end{axis}
\node [right] at (myplot.right of origin) {\scriptsize $x$};
\node [above] at (myplot.above origin) {\scriptsize $y$};
\end{tikzpicture}}	

\autoref{fig:abc3b} shows the region with the boundaries relabeled. A
% differential element, a
horizontal rectangle is also pictured. The width of the rectangle is a small change in $y$: $\Delta y$. The height of the rectangle is a difference in $x$-values. The ``top'' $x$-value is the largest value, i.e., the rightmost. The ``bottom'' $x$-value is the smaller, i.e., the leftmost. Therefore the height of the rectangle is
\[\bigl(\sqrt{3-y}+1\bigr) - (y-2)^2.\]

The area is found by integrating the above function with respect to $y$ with the appropriate bounds. We determine these by considering the $y$-values the region occupies. It is bounded below by $y=2$, and bounded above by $y=3$. That is, both the ``top'' and ``bottom'' functions exist on the $y$ interval $[2,3]$. Thus
\begin{align*}
	\text{Total Area}
	&= \int_2^3 \bigl(\sqrt{3-y}+1 - (y-2)^2\bigr)\dd y \\
	&= \Bigl(-\frac23(3-y)^{3/2}+y-\frac13(y-2)^3\Bigr)\Big|_2^3 \\
	&= 4/3.
\end{align*}
The important thing to notice is that by integrating with respect to $y$ instead of $x$, we only had to do one integral and did not need to find the point at which to switch from one integration to another.
\end{example}

This calculus-based technique of finding area can be useful even with shapes that we normally think of as ``easy.'' \autoref{ex_abc4} computes the area of a triangle. While the formula ``$\frac12\times\text{base}\times\text{height}$'' is well known, in arbitrary triangles it can be nontrivial to compute the height. Calculus makes the problem simple.

\begin{example}[Finding the area of a triangle]\label{ex_abc4}
Compute the area of the regions bounded by the lines
\mtable{Graphing a triangular region in \autoref{ex_abc4}.}{fig:abc4}{\begin{tikzpicture}
\begin{axis}[width=1.16\marginparwidth,axis equal,
	tick label style={font=\scriptsize},axis y line=middle,axis x line=middle,
	name=myplot,axis on top,%xtick={1,2,3},
	ymin=-.1,ymax=5.5,xmin=-.1,xmax=5.5]
\draw [draw={\colorone},thick,fill={\coloronefill}]
  (axis cs:3,0)
  -- node[below left,color=black] {$y=3-x$} (axis cs:1,2)
  -- node[above left,color=black] {$y=x+1$} (axis cs:4,5)
  -- node[pos=.7,below right,color=black] {$y=5x-15$} cycle;
\end{axis}
\node [right] at (myplot.right of origin) {\scriptsize $x$};
\node [above] at (myplot.above origin) {\scriptsize $y$};
\end{tikzpicture}}
$y=3-x$, $y=x+1$ and $y=5x-15$, as shown in \autoref{fig:abc4}.
\solution
Recognize that there are two ``bottom'' functions to this region, causing us to use two definite integrals.
\begin{align*}
	\text{Total Area}
	&= \int_1^3\bigl((x+1)-(3-x)\bigr)\dd x + \int_3^4\bigl((x+1)-(5x-15)\bigr)\dd x \\
%	&= \int_1^3 2x-2\dd x + \int_3^4 16-4x\dd x \\
%	&= \left.x^2-2x\right|_1^3 + \left.16x-2x^2\right|_3^4 \\
%	&= (3- -1) + (32-30) \\
	&= 4+2
	%\\&
	= 6.
\end{align*}
We can also approach this by converting each function into a function of $y$. This also requires 2 integrals, so there isn't really any advantage to doing so. We do it here for demonstration purposes.

The ``top'' function is always $x=\frac y5+3$ while there are two ``bottom'' functions: $x=3-y$ and $x=y-1$. Being mindful of the proper integration bounds, we have
\begin{align*}
	\text{Total Area}
	&= \int_0^2\left(\left(\frac y5+3\right) - (3-y)\right)\dd y
	+ \int_2^5\left(\left(\frac y5+3\right) - (y-1)\right)\dd y \\
%	&= \int_0^2\left(\frac{6y}5\right)\dd y+ \int_2^5\left(-\frac{4y}5+4\right)\dd y \\
%	&= \left.\frac{3y^2}5\right|_0^2 + \left[-\frac{2y^2}5+4y\right]_2^5 \\
%	&= \frac{12}5 + \left[10-\frac{32}5\right] \\
	&= \frac{12}5 + \frac{18}5
	% \\
%	&= \frac{30}5 \\
	%&
	= 6.
\end{align*}
Of course, the final answer is the same (and we see that integrating with respect to $x$ was probably easier, since it avoided fractions).
\end{example}

% move this to numerical integration?
%While we have focused on producing exact answers, we are also able to make approximations using the principle of \autoref{thm:areabetweencurves}. The integrand in the theorem is a distance (``top minus bottom''); integrating this distance function gives an area. By taking discrete measurements of distance, we can approximate an area using numerical integration techniques developed in \autoref{sec:numerical_integration}. The following example demonstrates this.
%
%\begin{example}[Numerically approximating area]\label{ex_abc5}
%To approximate the area of a lake, shown in \autoref{fig:abc5}(a),  the ``length'' of the lake is measured at 200-foot increments as shown in \autoref{fig:abc5}(b), where the lengths are given in hundreds of feet. Approximate the area of the lake.
%\solution
%The measurements of length can be viewed as measuring ``top minus bottom'' of two functions. The exact answer is found by integrating $\ds \int_0^{12} \bigl(f(x)-g(x)\bigr)\dd x$, but of course we don't know the functions $f$ and $g$. Our discrete measurements instead allow us to approximate.
%
%\mtable{(a) A sketch of a lake, and (b) the lake with length measurements.}{fig:abc5}{\begin{tikzpicture}[scale=.4]%[width=1.16\marginparwidth]
%\begin{axis}[width=1.16\marginparwidth,tick label style={font=\scriptsize},
%axis y line=middle,axis x line=middle,name=myplot,axis on top,
%xtick={1,2,3,4,5,6,7,8,9,10,11,12},ymin=-.1,ymax=8.5,xmin=-.1,xmax=12.5]
%\draw [draw={\colorone},thick,fill={\coloronefill},smooth] plot coordinates {(0,3.2)(0.5,3.417)(1.,3.637)(1.5,4.041)(2.,4.505)(2.5,5.)(3.,5.495)(3.5,5.959)(4.,6.363)(4.5,6.683)(5.,6.898)(5.5,6.995)(6.,6.968)(6.5,6.819)(7.,6.556)(7.5,6.197)(8.,5.763)(8.5,5.282)(9.,4.784)(9.5,4.298)(10.,3.857)(10.5,3.486)(11.,3.21)(11.5,2.8)(11.5,2)(11.,1.545)(10.5,1.314)(10.,1.102)(9.5,0.9154)(9.,0.7585)(8.5,0.6361)(8.,0.5514)(7.5,0.5069)(7.,0.5038)(6.5,0.5421)(6.,0.6208)(5.5,0.7378)(5.,0.8897)(4.5,1.072)(4.,1.281)(3.5,1.509)(3.,1.751)(2.5,2.)(2.,2.249)(1.5,2.491)(1.,2.719)(0.5,2.9102)(0,3.15)(0,3.2)};
%\end{tikzpicture}
%\\(a)\\
%\begin{tikzpicture}
%\begin{axis}[width=1.16\marginparwidth,tick label style={font=\scriptsize},
%axis y line=middle,axis x line=middle,name=myplot,axis on top,
%xtick={1,2,3,4,5,6,7,8,9,10,11,12},ytick={1,2,3,4,5,6,7,8},
%ymin=-.1,ymax=8.5,xmin=-.1,xmax=12.5]
%\addplot [draw={\colorone},thick,fill={\coloronefill},smooth] coordinates {(0,3.2)(0.5,3.417)(1.,3.637)(1.5,4.041)(2.,4.505)(2.5,5.)(3.,5.495)(3.5,5.959)(4.,6.363)(4.5,6.683)(5.,6.898)(5.5,6.995)(6.,6.968)(6.5,6.819)(7.,6.556)(7.5,6.197)(8.,5.763)(8.5,5.282)(9.,4.784)(9.5,4.298)(10.,3.857)(10.5,3.486)(11.,3.21)(11.5,2.8)(11.5,2)(11.,1.545)(10.5,1.314)(10.,1.102)(9.5,0.9154)(9.,0.7585)(8.5,0.6361)(8.,0.5514)(7.5,0.5069)(7.,0.5038)(6.5,0.5421)(6.,0.6208)(5.5,0.7378)(5.,0.8897)(4.5,1.072)(4.,1.281)(3.5,1.509)(3.,1.751)(2.5,2.)(2.,2.249)(1.5,2.491)(1.,2.719)(0.5,2.9102)(0,3.15)(0,3.2)};
%\draw (axis cs:2,4.5) -- (axis cs:2,2.249) node [shift={(-3pt,0pt)},rotate=90,pos=.5] {\scriptsize 2.25}
%(axis cs:4,6.36) -- (axis cs:4,1.28) node [shift={(-3pt,0pt)},rotate=90,pos=.5] {\scriptsize 5.08}
%(axis cs:6,6.97) -- (axis cs:6,0.62) node [shift={(-3pt,0pt)},rotate=90,pos=.5] {\scriptsize 6.35}
%(axis cs:8,5.76) -- (axis cs:8,.55) node  [shift={(-3pt,0pt)},rotate=90,pos=.5] {\scriptsize 5.21}
%(axis cs:10,3.86) -- (axis cs:10,1.1) node [shift={(-3pt,0pt)},rotate=90,pos=.5] {\scriptsize 2.76};
%\end{axis}
%\node [right] at (myplot.right of origin) {\scriptsize $x$};
%\node [above] at (myplot.above origin) {\scriptsize $y$};
%\end{tikzpicture}
%\\ (b)}
%We have the following data points:
%\[(0,0),\ (2,2.25),\ (4,5.08),\ (6,6.35),\ (8,5.21),\ (10,2.76),\ (12,0).\]
%We also have that $\Delta x=\frac{b-a}{n} = 2$, so Simpson's Rule gives
%\begin{align*}
%	\text{Area}
%	&\approx \frac{2}{3}
%\Bigl(1\cdot0+4\cdot2.25+2\cdot5.08+4\cdot6.35+2\cdot5.21+4\cdot2.76+1\cdot0\Bigr)\\
%	&= 44.01\overline{3} \ \text{units}^2.
%\end{align*}
%
%Since the measurements are in hundreds of feet, units${}^2 = (100\ \text{ft})^2 = 10,000\ \text{ft}^2$, giving a total area of $440,133\ \text{ft}^2$. (Since we are approximating, we'd likely say the area was about $440,000\ \text{ft}^2$, which is a little more than 10 acres.)
%\end{example}

In the next section we apply \autoref{idea:app_of_defint} to finding the volumes of certain solids.

\printexercises{exercises/07-01-exercises}
