\section{Area Between Curves}\label{sec:ABC}

We are often interested in knowing the area of a region. Forget momentarily that we addressed this already in \autoref{sec:FTC} and approach it instead using the technique described in \autoref{idea:app_of_defint}. 

\mtable[-3in]{Subdividing a region into vertical slices and approximating the areas with rectangles.}{fig:abcintro}{\small%
\begin{tabular}{c}%
\myincludegraphics{figures/figabcintroa} \\ (a)\\
\myincludegraphics{figures/figabcintrob} \\ (b)\\
\myincludegraphics{figures/figabcintroc} \\ (c)
\end{tabular}}

Let $Q$ be the area of a region bounded by continuous functions $f$ and $g$. If we break the region into many subregions, we have an obvious equation:
\begin{center}
Total Area = sum of the areas of the subregions.
\end{center}
The issue to address next is how to systematically break a region into subregions. A graph will help. Consider \autoref{fig:abcintro} (a) where a region between two curves is shaded. While there are many ways to break this into subregions, one particularly efficient way is to ``slice'' it vertically, as shown in \autoref{fig:abcintro} (b), into $n$ equally spaced slices.

We now approximate the area of a slice. Again, we have many options, but using a rectangle seems simplest. Picking any $x$-value $c_i$ in the $i^\text{ th}$ slice, we set the height of the rectangle to be $f(c_i)-g(c_i)$, the difference of the corresponding $y$-values. The width of the rectangle is a small difference in $x$-values, which we represent with $\dx$. \autoref{fig:abcintro} (c) shows sample points $c_i$ chosen in each subinterval and appropriate rectangles drawn.
%(Each of these rectangles represents a differential element.)
Each slice has an area approximately equal to $\big(f(c_i)-g(c_i)\big)\dx$; hence, the total area is approximately the Riemann Sum
$$Q \approx \sum_{i=1}^n \big(f(c_i)-g(c_i)\big)\dx.$$
Taking the limit as $n\to \infty$ gives the exact area as $\int_a^b \big(f(x)-g(x)\big)\ dx.$

\theorem{thm:areabetweencurves}{Area Between Curves}
{Let $f(x)$ and $g(x)$ be continuous functions defined on $[a,b]$ where $f(x)\geq g(x)$ for all $x$ in $[a,b]$. The area of the region bounded by the curves $y=f(x)$, $y=g(x)$ and the lines $x=a$ and $x=b$ is \index{integration!area between curves}
\[\int_a^b \big(f(x)-g(x)\big)\ dx.\]}

Often, we do not know which function is greater (or they switch within the domain of integration).  In that case, we can say that the area is $\int_a^b\abs{f(x)-g(x)}\ dx$, which may involve dividing the domain of integration into pieces.

\youtubeVideo{DRFyNHdVgUA}{Finding Areas Between Curves}

\example{ex_abc1}{Finding area enclosed by curves}{Find the area of the region bounded by $f(x) = \sin x+2$, $g(x) = \frac12\cos (2x)-1$, $x=0$ and $x=4\pi$, as shown in \autoref{fig:abc1}.}
{\mfigure{0in}{Graphing an enclosed region in \autoref{ex_abc1}.}{fig:abc1}{figures/figabc1}
%
The graph verifies that the upper boundary of the region is given by $f$ and the lower bound is given by $g$. Therefore the area of the region is the value of the integral
\begin{align*} 
	\int_0^{4\pi} \big(f(x)- g(x)\big)\ dx
	& = \int_0^{4\pi} \Big(\sin x+2 - \big(\frac12\cos (2x)-1\big)\Big)\ dx \\
	&= -\cos x -\frac14\sin(2x)+3x\Big|_0^{4\pi}\\
	&=	12\pi %\approx 37.7
	\ \text{units}^2.\eoehere
\end{align*}}

% originally in FTC section
\example{ex_ftc6}{Finding area between curves}{Find the area of the region enclosed by $y=x^2+x-5$ and $y=3x-2$.}
{It will help to sketch these two functions, as done in \autoref{fig:ftc6}. 
\mfigure{0in}{Sketching the region enclosed by $y=x^2+x-5$ and $y=3x-2$ in \autoref{ex_ftc6}.}{fig:ftc6}{figures/figftc6}
The region whose area we seek is completely bounded by these two functions; they seem to intersect at $x=-1$ and $x=3$. To check, set $x^2+x-5=3x-2$ and solve for $x$:
\begin{align*}
	x^2+x-5 &= 3x-2 \\
	(x^2+x-5) - (3x-2) &= 0\\
	x^2-2x-3 &= 0\\
	(x-3)(x+1) &= 0\\
	x&=-1,\ 3.
\end{align*}
Following \autoref{thm:areabetweencurves}, the area is 
\begin{align*}
	\int_{-1}^3\big(3x-2 -(x^2+x-5)\big)\ dx &= \int_{-1}^3 (-x^2+2x+3)\ dx \\
	&=\left.\left(-\frac13x^3+x^2+3x\right)\right|_{-1}^3 \\
	&=-\frac13(27)+9+9-\left(\frac13+1-3\right)\\
	&= 10\frac23 = 10.\overline{6}\eoehere
\end{align*}}

\mfigure{0in}{Graphing a region enclosed by two functions in \autoref{ex_abc2}.}{fig:abc2}{figures/figabc2}

\example{ex_abc2}{Finding total area enclosed by curves}{Find the total area of the region enclosed by the functions $f(x) = -2x+5$ and $g(x) = x^3-7x^2+12x-3$ as shown in \autoref{fig:abc2}.}
{A quick calculation shows that $f=g$ at $x=1, 2$ and 4. One can proceed thoughtlessly by computing $\ds \int_1^4\big(f(x)-g(x)\big)\ dx$, but this ignores the fact that on $[1,2]$, $g(x)>f(x)$. (In fact, the thoughtless integration returns $-9/4$, hardly the expected value of an \textit{area}.) Thus we compute the total area by breaking the interval $[1,4]$ into two subintervals, $[1,2]$ and $[2,4]$ and using the proper integrand in each.
\begin{align*}
	\text{Total Area}
	&= \int_1^2 \big(g(x)-f(x)\big)\ dx + \int_2^4\big(f(x)-g(x)\big)\ dx\\
	&= \int_1^2 \big(x^3-7x^2+14x-8\big) \ dx
	+ \int_2^4\big(-x^3+7x^2-14x+8\big)\ dx\\
	&= \frac5{12} + \frac83 \\
	&= \frac{37}{12} % \approx 3.083
	\ \text{units}^2.\eoehere
\end{align*}}

The previous example makes note that we are expecting area to be \textit{positive}. When first learning about the definite integral, we interpreted it as ``signed area under the curve,'' allowing for ``negative area.'' That doesn't apply here; area is to be positive.

The previous example also demonstrates that we often have to break a given region into subregions before applying \autoref{thm:areabetweencurves}. The following example shows another situation where this is applicable, along with an alternate view of applying the Theorem.\\

\example{ex_abc3}{Finding area: integrating with respect to $y$}{Find the area of the region enclosed by the functions $y=\sqrt{x}+2$, $y=-(x-1)^2+3$ and $y=2$, as shown in \autoref{fig:abc3}.}
{We give two approaches to this problem. In the first approach, we notice that the region's ``top'' is defined by two different curves. On $[0,1]$, the top function is $y=\sqrt{x}+2$; on $[1,2]$, the top function is $y=-(x-1)^2+3$. 
\mfigure{0in}{Graphing a region for \autoref{ex_abc3}.}{fig:abc3}{figures/figabc3}
Thus we compute the area as the sum of two integrals:
\begin{align*}
	\text{Total Area}
	&= \int_0^1 \Big(\big(\sqrt{x}+2\big)-2\Big)\ dx + \int_1^2 \Big(\big(-(x-1)^2+3\big)-2\Big)\ dx \\
	&= 2/3 + 2/3\\
	&=4/3.
\end{align*}

The second approach is clever and very useful in certain situations. We are used to viewing curves as functions of $x$; we input an $x$-value and a $y$-value is returned. Some curves can also be described as functions of $y$: input a $y$-value and an $x$-value is returned. We can rewrite the equations describing the boundary by solving for $x$:
\begin{align*}
y=\sqrt{x}+2 & \quad\Rightarrow\quad x=(y-2)^2 \\
y=-(x-1)^2+3 & \quad\Rightarrow\quad x=\sqrt{3-y}+1.
\end{align*}

\autoref{fig:abc3b} shows the region with the boundaries relabeled. A
% differential element, a
horizontal rectangle is also pictured. The width of the rectangle is a small change in $y$: $\Delta y$. The height of the rectangle is a difference in $x$-values. The ``top'' $x$-value is the largest value, i.e., the rightmost. The ``bottom'' $x$-value is the smaller, i.e., the leftmost. Therefore the height of the rectangle is
\[\big(\sqrt{3-y}+1\big) - (y-2)^2.\]
\mfigure{-1in}{The region used in \autoref{ex_abc3} with boundaries relabeled as functions of $y$.}{fig:abc3b}{figures/figabc3b}	

The area is found by integrating the above function with respect to $y$ with the appropriate bounds. We determine these by considering the $y$-values the region occupies. It is bounded below by $y=2$, and bounded above by $y=3$. That is, both the ``top'' and ``bottom'' functions exist on the $y$ interval $[2,3]$. Thus
\begin{align*}
	\text{Total Area}
	&= \int_2^3 \big(\sqrt{3-y}+1 - (y-2)^2\big)\ dy \\
	&= \Big(-\frac23(3-y)^{3/2}+y-\frac13(y-2)^3\Big)\Big|_2^3 \\
	&= 4/3.
\end{align*}
The important thing to notice is that by integrating with respect to $y$ instead of $x$, we only had to do one integral and did not need to find the point at which to switch from one integration to another.}

This calculus--based technique of finding area can be useful even with shapes that we normally think of as ``easy.'' \autoref{ex_abc4} computes the area of a triangle. While the formula ``$\frac12\times\text{base}\times\text{height}$'' is well known, in arbitrary triangles it can be nontrivial to compute the height. Calculus makes the problem simple.

\example{ex_abc4}{Finding the area of a triangle}{Compute the area of the regions bounded by the lines
\mtable{Graphing a triangular region in \autoref{ex_abc4}.}{fig:abc4}{\begin{tikzpicture}
\begin{axis}[width=1.16\marginparwidth,
	tick label style={font=\scriptsize},axis y line=middle,axis x line=middle,
	name=myplot,axis on top,%xtick={1,2,3},
	ymin=-.1,ymax=5.5,xmin=-.1,xmax=5.5]
\draw [draw={\colorone},thick,fill={\coloronefill}]
  (axis cs:3,0)
  -- node[below left,color=black] {$y=3-x$} (axis cs:1,2)
  -- node[above left,color=black] {$y=x+1$} (axis cs:4,5)
  -- node[pos=.7,below right,color=black] {$y=5x-15$} cycle;
\end{axis}
\node [right] at (myplot.right of origin) {\scriptsize $x$};
\node [above] at (myplot.above origin) {\scriptsize $y$};
\end{tikzpicture}}
$y=3-x$, $y=x+1$ and $y=5x-15$, as shown in \autoref{fig:abc4}.}
{Recognize that there are two ``bottom'' functions to this region, causing us to use two definite integrals.
\begin{align*}
	\text{Total Area}
	&= \int_1^3\bigl((x+1)-(3-x)\bigr)\ dx + \int_3^4\bigl((x+1)-(5x-15)\bigr)\ dx \\
%	&= \int_1^3 2x-2\ dx + \int_3^4 16-4x\ dx \\
%	&= \left.x^2-2x\right|_1^3 + \left.16x-2x^2\right|_3^4 \\
%	&= (3--1) + (32-30) \\
	&= 4+2\\
	&= 6.
\end{align*}
We can also approach this by converting each function into a function of $y$. This also requires 2 integrals, so there isn't really any advantage to doing so. We do it here for demonstration purposes.

The ``top'' function is always $x=\frac y5+3$ while there are two ``bottom'' functions: $x=3-y$ and $x=y-1$. Being mindful of the proper integration bounds, we have
\begin{align*}
	\text{Total Area}
	&= \int_0^2\left(\left(\frac y5+3\right) - (3-y)\right)\ dy
	+ \int_2^5\left(\left(\frac y5+3\right) - (y-1)\right)\ dy \\
%	&= \int_0^2\left(\frac{6y}5\right)\ dy+ \int_2^5\left(-\frac{4y}5+4\right)\ dy \\
%	&= \left.\frac{3y^2}5\right|_0^2 + \left[-\frac{2y^2}5+4y\right]_2^5 \\
%	&= \frac{12}5 + \left[10-\frac{32}5\right] \\
	&= \frac{12}5 + \frac{18}5 \\
%	&= \frac{30}5 \\
	&= 6.
\end{align*}
Of course, the final answer is the same (and we see that integrating with respect to $x$ was probably easier, since it avoided fractions).}

% move this to numerical integration?
%While we have focused on producing exact answers, we are also able to make approximations using the principle of \autoref{thm:areabetweencurves}. The integrand in the theorem is a distance (``top minus bottom''); integrating this distance function gives an area. By taking discrete measurements of distance, we can approximate an area using numerical integration techniques developed in \autoref{sec:numerical_integration}. The following example demonstrates this.
%
%\example{ex_abc5}{Numerically approximating area}{To approximate the area of a lake, shown in \autoref{fig:abc5}(a),  the ``length'' of the lake is measured at 200-foot increments as shown in \autoref{fig:abc5}(b), where the lengths are given in hundreds of feet. Approximate the area of the lake.}
%{The measurements of length can be viewed as measuring ``top minus bottom'' of two functions. The exact answer is found by integrating $\ds \int_0^{12} \big(f(x)-g(x)\big)\ dx$, but of course we don't know the functions $f$ and $g$. Our discrete measurements instead allow us to approximate.
%
%\mtable{(a) A sketch of a lake, and (b) the lake with length measurements.}{fig:abc5}{\noindent\begin{tabular}{c}\myincludegraphics{figures/figabc5b}\\(a)\\ \myincludegraphics{figures/figabc5} \\ (b)\end{tabular}}
%We have the following data points:
%$$(0,0),\ (2,2.25),\ (4,5.08),\ (6,6.35),\ (8,5.21),\ (10,2.76),\ (12,0).$$
%We also have that $\dx=\frac{b-a}{n} = 2$, so Simpson's Rule gives
%\begin{align*}
%	\text{Area}
%	&\approx \frac{2}{3}
%\Big(1\cdot0+4\cdot2.25+2\cdot5.08+4\cdot6.35+2\cdot5.21+4\cdot2.76+1\cdot0\Big)\\
%	&= 44.01\overline{3} \ \text{units}^2.
%\end{align*}
%
%Since the measurements are in hundreds of feet, units${}^2 = (100\ \text{ft})^2 = 10,000\ \text{ft}^2$, giving a total area of $440,133\ \text{ft}^2$. (Since we are approximating, we'd likely say the area was about $440,000\ \text{ft}^2$, which is a little more than 10 acres.)}

In the next section we apply \autoref{idea:app_of_defint} to finding the volumes of certain solids.

\printexercises{exercises/07_01_exercises}
