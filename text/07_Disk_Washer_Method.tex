\section{Volume by Cross-Sectional Area; Disk and Washer Methods}\label{sec:disk}

The volume of a general right cylinder, as shown in \autoref{fig:cross1}, is 
\begin{center}
Area of the base $\times$ height.
\end{center}
\mfigurethree{width=75pt,3Dmenu,activate=onclick,deactivate=onclick,
3Droll=0,
3Dortho=0.005786627996712923,
3Dc2c=0.4350872337818146 0.822970449924469 0.3652651607990265,
3Dcoo=-7.312352180480957 -16.589130401611328 29.674652099609375,
3Droo=149.99999509331877,
3Dlights=Headlamp,add3Djscript=asylabels.js}{}{0in}{The volume of a general right cylinder}{fig:cross1}{figures/figcross1}
We can use this fact as the building block in finding volumes of a variety of shapes.

Given an arbitrary solid, we can \textit{approximate} its volume by cutting it into $n$  thin slices. When the slices are thin, each slice can be approximated well by a general right cylinder. Thus the volume of each slice is approximately its cross-sectional area $\times$ thickness.
% (These slices are the differential elements.)

By orienting a solid along the $x$-axis, we can let $A(x_i)$ represent the cross-sectional area
of the $i\,^\text{th}$ slice, and let $\dx$ represent the thickness of the slices (the thickness is a small change in $x$). The total volume of the solid is approximately:
	\begin{align*} \text{Volume} &\approx \sum_{i=1}^n \Big[\text{Area}\ \times\ \text{thickness}\Big] \\
			&= \sum_{i=1}^n A(x_i)\dx.
	\end{align*}
	
Recognize that this is a Riemann Sum. By taking a limit (as the thickness of the slices goes to 0) we can find the volume exactly. 
\[\text{Volume}=\lim_{n\to \infty} \sum_{i=1}^n A(x_i)\Delta x\]
with $\Delta x=\frac{b-a}{n}$ and $x_i=a+i\Delta x$. We recognize this as a definite integral.

\theorem{thm:volume_by_cross_section}{Volume By Cross-Sectional Area}
{The volume $V$ of a solid, oriented along the $x$-axis with cross-sectional area $A(x)$ from $x=a$ to $x=b$, is \index{integration!volume!cross-sectional area}
\[V = \int_a^b A(x)\ dx.\]}

\example{ex_disk0}{Finding the volume of a solid}{Find the volume of a pyramid with a square base of side length 10 in and a height of 5 in.}
{There are many ways to ``orient'' the pyramid along the $x$-axis; \autoref{fig:disk0} gives one such way, with the pointed top of the pyramid at the origin and the $x$-axis going through the center of the base.
\mfigurethree{width=\marginparwidth,3Dmenu,activate=onclick,deactivate=onclick,
3Droll=130.3044775136707,
3Dortho=0.004594136495143175,
3Dc2c=0.4693154990673065 0.5382519960403442 0.7000198364257812,
3Dcoo=60.78499984741211 0.48087620735168457 -14.641631126403809,
3Droo=150.00000157638937,
3Dlights=Headlamp,add3Djscript=asylabels.js}{width=\marginparwidth}{0in}{Orienting a pyramid along the $x$-axis in \autoref{ex_disk0}.}{fig:disk0}{figures/figcross_area1}
%\mfigure{0in}{Orienting a pyramid along the $x$-axis in \autoref{ex_disk0}.}{fig:disk0}{figures/figcross_area1}

Each cross section of the pyramid is a square.
%; this is a sample differential element.
To determine its area $A(x)$, we need to determine the side lengths of the square.

When $x=5$, the square has side length 10; when $x=0$, the square has side length 0. Since the edges of the pyramid are lines, it is easy to figure that each cross-sectional square has side length $2x$, giving $A(x) = (2x)^2=4x^2$. % 

If one were to cut a slice out of the pyramid at $x=3$, as shown in \autoref{fig:disk0a}, one would have a shape with square bottom and top with sloped sides. If the slice were thin, both the bottom and top squares would have sides lengths of about 6, and thus the cross--sectional area of the bottom and top would be about 36in$^2$. Letting $\Delta x$ represent the thickness of the slice, the volume of this slice would then be about $36\Delta x$in$^3$. 
\mfigurethree{width=\marginparwidth,3Dmenu,activate=onclick,deactivate=onclick,
3Droll=130.3044775136707,
3Dortho=0.004594136495143175,
3Dc2c=0.4693154990673065 0.5382519960403442 0.7000198364257812,
3Dcoo=60.78499984741211 0.48087620735168457 -14.641631126403809,
3Droo=150.00000157638937,
3Dlights=Headlamp,add3Djscript=asylabels.js}{width=\marginparwidth}{0in}{Cutting a slice in the pyramid in \autoref{ex_disk0} at $x=3$.}{fig:disk0a}{figures/figcross_area1a}

Cutting the pyramid into $n$ slices divides the total volume into $n$ equally--spaced smaller pieces, each with volume $(2x_i)^2\Delta x$, where $x_i$ is the approximate location of the slice along the $x$-axis and $\Delta x$ represents the thickness of each slice. One can approximate total volume of the pyramid by summing up the volumes of these slices:
$$\text{Volume } \approx \sum_{i=1}^n (2x_i)^2\Delta x.$$
Taking the limit as $n\to\infty$ gives the actual volume of the pyramid; recognizing this sum as a Riemann Sum allows us to find the exact answer using a definite integral, matching the definite integral given by \autoref{thm:volume_by_cross_section}.

We have 
\begin{align*}
	V
	&= \lim_{n\to\infty} \sum_{i=1}^n (2x_i)^2\Delta x\\
	&= \int_0^5 4x^2\ dx\\
	&= \frac43x^3\Big|_0^5 \\
	&=\frac{500}{3}\ %\text{in}^3 \approx 166.67
	\ \text{in}^3.
\end{align*}
We can check our work by consulting the general equation for the volume of a pyramid (see the back cover under ``Volume of A General Cone''): 
\begin{center}
$\frac13\times \text{area of base}\times \text{height}$.
\end{center}
Certainly, using this formula from geometry is faster than our new method, but the calculus--based method can be applied to much more than just cones.}

An important special case of \autoref{thm:volume_by_cross_section} is when the solid is a \textbf{solid of revolution}, that is, when the solid is formed by rotating a shape around an axis.

% todo Tim double check phrasing
Start with a function $y=f(x)$ from $x=a$ to $x=b$. Revolving this curve about a horizontal axis encloses a three-dimensional solid whose cross sections are disks (thin circles), perpendicular to the axis of rotation. Let $R(x)$ represent the radius of the cross-sectional disk at $x$; the area of this disk is $\pi [R(x)]^2$. Applying \autoref{thm:volume_by_cross_section} gives the Disk Method.

\keyidea{idea:disk_method}{The Disk Method}
{Let a solid be enclosed by revolving the curve $y=f(x)$ from $x=a$ to $x=b$ around a horizontal axis, and let $R(x)$ be the radius of the cross-sectional disk at $x$. The volume of the solid is
\index{integration!volume!Disk Method}\index{Disk Method}
\[V = \pi \int_a^b [R(x)]^2\ dx.\]}

\youtubeVideo{nZqOKc067Z8}{Longer Version --- Volumes using Disks/Washers}

\clearpage

\mtable{Sketching a solid in \autoref{ex_disk1}.}{fig:disk1}{%
 \begin{tabular}{c}
  \begin{tikzpicture}
   \begin{axis}[width=\marginparwidth,
     tick label style={font=\scriptsize},axis y line=middle,
     axis x line=middle,name=myplot,axis on top,
     ymin=-.2,ymax=1.2,xmin=-.1,xmax=2.2]
    \draw[draw={\colorone},fill={\coloronefill}] plot[smooth,domain=1:2]
     (axis cs:\x,1/\x) |- (axis cs:1,0) --(axis cs:2,0)--(axis cs:1,0)-- cycle;
    \draw [thick,draw={\colortwo}] (axis cs: 1.5,0) --
      node[pos=.5,right,color=black]{\scriptsize$R(x)$} (axis cs: 1.5,.667);
   \end{axis}
   \node [right] at (myplot.right of origin) {\scriptsize $x$};
   \node [above] at (myplot.above origin) {\scriptsize $y$};
  \end{tikzpicture}
  \\ (a) \\
	\myincludeasythree{width=\marginparwidth,
3Droll=126.0060482236849,
3Dortho=0.004875946324318647,
3Dc2c=0.44462037086486816 0.39410829544067383 0.8043577671051025,
3Dcoo=67.21983337402344 3.5258262157440186 -29.566415786743164,
3Droo=150.00000160608386}{width=\marginparwidth}{figures/figdisk1_3D}\\
%\myincludegraphics{figures/figdisk1} \\
%	\myincludegraphicsthree{width=110pt,3Dmenu,activate=onclick,deactivate=onclick,
%3Droll=126.0060482236849,
%3Dortho=0.004875946324318647,
%3Dc2c=0.44462037086486816 0.39410829544067383 0.8043577671051025,
%3Dcoo=67.21983337402344 3.5258262157440186 -29.566415786743164,
%3Droo=150.00000160608386,
%3Dlights=Headlamp,add3Djscript=asylabels.js}{}{figures/figdisk1}\\
%%\myincludegraphics{figures/figdisk1} \\
	(b) \\
	\myincludegraphicsthree{width=\marginparwidth,3Dmenu,activate=onclick,
	deactivate=onclick,
3Droll=126.0060482236849,
3Dortho=0.004875946324318647,
3Dc2c=0.44462037086486816 0.39410829544067383 0.8043577671051025,
3Dcoo=67.21983337402344 3.5258262157440186 -29.566415786743164,
3Droo=150.00000160608386,
3Dlights=Headlamp,add3Djscript=asylabels.js}{width=\marginparwidth}{figures/figdisk1b}\\
%\myincludegraphics{figures/figdisk1b} \\
	(c) 
\end{tabular}}

\example{ex_disk1}{Finding volume using the Disk Method}{Find the volume of the solid formed by revolving about the $x$-axis the region bounded by the curves $y=1/x$, $x=1$, $x=2$ and the $x$-axis.}
{A sketch can help us understand this problem. In \autoref{fig:disk1}(a) we have sketched the region we will be rotating.  In \autoref{fig:disk1}(b), the curve $y=1/x$ is sketched along with the sample slice, a disk, at $x$ with radius $R(x)=1/x$. In \autoref{fig:disk1}(c) the whole solid is pictured, along with the sample slice.

The volume of the sample slice shown in part (b) of the figure is approximately $\pi R(x_i)^2\Delta x$, where $R(x_i)$ is the radius of the disk shown and $\Delta x$ is the thickness of that slice. The radius $R(x_i)$ is the distance from the $x$-axis to the curve, hence $R(x_i) = 1/x_i$.

Slicing the solid into $n$ equally--spaced slices, we can approximate the total volume by adding up the approximate volume of each slice:
$$\text{Approximate volume } = \sum_{i=1}^n \pi \left(\frac1{x_i}\right)^2\Delta x.$$

Taking the limit of the above sum as $n\to\infty$ gives the actual volume; recognizing this sum as a Riemann sum allows us to evaluate the limit with a definite integral, which matches the formula given in \autoref{idea:disk_method}:

%Using \autoref{idea:disk_method}, we have 
\begin{align*}
	V &= \lim_{n\to\infty}\sum_{i=1}^n \pi \left(\frac1{x_i}\right)^2\Delta x\\
		&= \pi\int_1^2 \left(\frac1x\right)^2\ dx \\
		&= \pi\int_1^2 \frac1{x^2}\ dx \\
		&= \pi\left[-\frac1x\right]\Big|_1^2 \\
		&= \pi \left[-\frac12 - \left(-1\right)\right] \\
		&= \frac{\pi}{2}\ \text{units}^3.\eoehere
\end{align*}}

\pagebreak

While \autoref{idea:disk_method} is given in terms of functions of $x$, the principle involved can be applied to functions of $y$ when the axis of rotation is vertical, not horizontal. We demonstrate this in the next example.

\mtable{Sketching a solid in \autoref{ex_disk2}.}{fig:disk2}{%
	\begin{tabular}{c}
  \begin{tikzpicture}
   \begin{axis}[width=1.16\marginparwidth,
     tick label style={font=\scriptsize},axis y line=middle,
     axis x line=middle,name=myplot,axis on top,
     ymin=-.2,ymax=1.2,xmin=-.1,xmax=2.2]
    \draw[draw={\colorone},fill={\coloronefill}] plot[smooth,domain=1:2]
     (axis cs:\x,1/\x) -| (axis cs:0,.5) --(axis cs:0,.5)--(axis cs:0,1)-- cycle;
    \draw [thick,draw={\colortwo}] (axis cs: 0,.75) --
     node[pos=.5,below,color=black]{\scriptsize$R(y)$} (axis cs: 1.333,.75);
   \end{axis}
   \node [right] at (myplot.right of origin) {\scriptsize $x$};
   \node [above] at (myplot.above origin) {\scriptsize $y$};
  \end{tikzpicture}
  \\ (a) \\
	\myincludegraphicsthree{width=125pt,3Dmenu,activate=onclick,deactivate=onclick,
3Droll=112.9954735907269,
3Dortho=0.003950905054807663,
3Dc2c=0.5512889623641968 0.2862306833267212 0.7836788296699524,
3Dcoo=-9.51923942565918 49.542564392089844 -9.694497108459473,
3Droo=150.00000339216956,
3Dlights=Headlamp,add3Djscript=asylabels.js}{}{figures/figdisk1a}\\
%\myincludegraphics{figures/figdisk1a} \\
	(b)\\
	\myincludegraphicsthree{width=125pt,3Dmenu,activate=onclick,deactivate=onclick,
3Droll=112.9954735907269,
3Dortho=0.003950905054807663,
3Dc2c=0.5512889623641968 0.2862306833267212 0.7836788296699524,
3Dcoo=-9.51923942565918 49.542564392089844 -9.694497108459473,
3Droo=150.00000339216956,
3Dlights=Headlamp,add3Djscript=asylabels.js}{}{figures/figdisk2a}\\
%\myincludegraphics{figures/figdisk2a} \\
	(c)
	\end{tabular}}

% todo Tim double check the phrasing of the solution
\example{ex_disk2}{Finding volume using the Disk Method}{Find the volume of the solid formed by revolving about the $y$-axis the region bounded by the curves $y=1/x$, $y=1$, $y=0.5$, and the $y$-axis.}
{Since the axis of rotation is vertical, our perpendicular cross sections have thickness $\Delta y$ and radius $x=R(y)$. We need to convert the function into a function of $y$. % and convert the $x$-bounds to $y$-bounds.
Since $y=1/x$ defines the curve, we rewrite it as $x=1/y$.% The bound $x=1$ corresponds to the $y$-bound $y=1$, and the bound $x=2$ corresponds to the $y$-bound $y=1/2$. 

Thus we are rotating about the $y$-axis the region bounded by the curves $x=1/y$, $y=1/2$, $y=1$, and the $y$-axis to form a solid. The region of revolution is sketched in \autoref{fig:disk2}(a), the curve and sample sample disk are sketched in \autoref{fig:disk2}(b), and a full sketch of the solid is in \autoref{fig:disk2}(b). We integrate to find the volume:
\begin{align*}
	V &= \pi\int_{1/2}^1 \frac{1}{y^2}\ dy \\
	&= -\frac{\pi}y\Big|_{1/2}^1 \\
	&= \pi\ \text{units}^3.\eoehere
\end{align*}}

% todo create and work out an example where the same region is rotated about the x axis and y axis but the disk method (for both) shows that they have different volumes
% not the same region
%The previous two examples demonstrate how taking the same region and rotating it about two different axes will result in different solids and thus volumes.

We can also compute the volume of solids of revolution that have a hole in the center. The general principle is simple: compute the volume of the solid irrespective of the hole, then subtract the volume of the hole. If the outside radius of the solid is $R(x)$ and the inside radius (defining the hole) is $r(x)$, then the volume is 
\[
V = \pi\int_a^b [R(x)]^2 \ dx - \pi\int_a^b [r(x)]^2\ dx
= \pi\int_a^b \left([R(x)]^2-[r(x)]^2\right)\ dx.
\]

One can generate a solid of revolution with a hole in the middle by revolving a region about an axis. Consider \autoref{fig:washeridea}(a), where a region is sketched along with a dashed, horizontal axis of rotation. By rotating the region about the axis, a solid is formed. Each cross section of this solid will be a washer (a disk with a hole in the center) as sketched in \autoref{fig:washeridea}(b). The outside of the washer has radius $R(x)$, whereas the inside has radius $r(x)$. The entire solid is sketched in \autoref{fig:washeridea}(c).  This leads us to the Washer Method.

\begin{lxfigure}
%\mtable{Establishing the Washer Method.}{fig:washeridea}{%
\begin{tabular}{ccc}
 \begin{tikzpicture}
  \begin{axis}[width=1.16\marginparwidth,
     tick label style={font=\scriptsize},axis y line=middle,
     axis x line=middle,name=myplot,axis on top,
     ymin=-3.5,ymax=3.5,xmin=-.2,xmax=3.5]
    \draw[draw={\colorone},fill={\coloronefill}]
     (axis cs:1,2.5) parabola bend (axis cs:2,3) (axis cs:3,2.5) --
     (axis cs:3,2) parabola bend (axis cs:2,1) (axis cs:1,2) -- cycle;
    \draw [dashed] (axis cs:-.2,.5) -- (axis cs:3.5,.5);
    \draw [{\colortwo}] (axis cs:1.2,2.68) --
     node [pos=.6,left,color=black] {\scriptsize $R(x)$} (axis cs:1.2,.5);
    \draw [{\colortwo}] (axis cs: 2.8,1.64) --
     node [pos=.5,right,color=black] {\scriptsize $r(x)$} (axis cs: 2.8,.5);
    \filldraw (axis cs:1.2,2.68) circle (1pt) (axis cs:1.2,.5) circle (1pt)
     (axis cs: 2.8,1.64)circle (1pt)  (axis cs: 2.8,.5) circle (1pt)  ;
  \end{axis}
  \node [right] at (myplot.right of origin) {\scriptsize $x$};
  \node [above] at (myplot.above origin) {\scriptsize $y$};
 \end{tikzpicture}
%\myincludegraphicsthree{width=125pt,3Dmenu,activate=onclick,deactivate=onclick,
%3Droll=96.94265756936434,
%3Dortho=0.005309578496962786,
%3Dc2c=0.547386884689331 0.07732175290584564 0.833299994468689,
%3Dcoo=64.78053283691406 10.129053115844727 -47.566162109375,
%3Droo=149.99999789136484,
%3Dlights=Headlamp,add3Djscript=asylabels.js}{}{figures/figwasher_idea_a}
&
%\myincludegraphics{figures/figwasher_idea_a}\\
\myincludegraphicsthree{width=125pt,3Dmenu,activate=onclick,deactivate=onclick,
3Droll=96.94265756936434,
3Dortho=0.005309578496962786,
3Dc2c=0.547386884689331 0.07732175290584564 0.833299994468689,
3Dcoo=64.78053283691406 10.129053115844727 -47.566162109375,
3Droo=149.99999789136484,
3Dlights=Headlamp,add3Djscript=asylabels.js}{}{figures/figwasher_idea_c}
&
\myincludegraphicsthree{width=125pt,3Dmenu,activate=onclick,deactivate=onclick,
3Droll=96.94265756936434,
3Dortho=0.005309578496962786,
3Dc2c=0.547386884689331 0.07732175290584564 0.833299994468689,
3Dcoo=64.78053283691406 10.129053115844727 -47.566162109375,
3Droo=149.99999789136484,
3Dlights=Headlamp,add3Djscript=asylabels.js}{}{figures/figwasher_idea_b}\\
%\myincludegraphics{figures/figwasher_idea_b}\\
(a) & (b) & (c)
%\myincludegraphicsthree{width=125pt,3Dmenu,activate=onclick,deactivate=onclick,
%3Droll=96.94265756936434,
%3Dortho=0.005309578496962786,
%3Dc2c=0.547386884689331 0.07732175290584564 0.833299994468689,
%3Dcoo=64.78053283691406 10.129053115844727 -47.566162109375,
%3Droo=149.99999789136484,
%3Dlights=Headlamp,add3Djscript=asylabels.js}{}{figures/figwasher_idea_c}\\
%%\myincludegraphics{figures/figwasher_idea_c}\\
%(c)
\end{tabular}
\caption{Establishing the Washer Method.}
\label{fig:washeridea}
\end{lxfigure}

\keyidea{idea:washermethod}{The Washer Method}
{Let a region bounded by $y=f(x)$, $y=g(x)$, $x=a$ and $x=b$ be rotated about a horizontal axis that does not intersect the region, forming a solid. Each cross section at $x$ will be a washer with outside radius $R(x)$ and inside radius $r(x)$. The volume of the solid is
\index{integration!volume!Washer Method}\index{Washer Method}
$$V = \pi\int_a^b \Big([R(x)]^2-[r(x)]^2\Big)\ dx.$$}

Even though we introduced it first, the Disk Method is just a special case of the Washer Method with an inside radius of $r(x)=0$.

\example{ex_wash1}{Finding volume with the Washer Method}{Find the volume of the solid formed by rotating the region bounded by $y=x^2-2x+2$ and $y=2x-1$ about the $x$-axis.}
{A sketch of the region will help, as given in \autoref{fig:wash1}(a).
\mtable[-4.5in]{Sketching the region, a sample slice, and solid in \autoref{ex_wash1}.}{fig:wash1}{%
 \begin{tikzpicture}
  \begin{axis}[width=\marginparwidth,
     tick label style={font=\scriptsize},axis y line=middle,
     axis x line=middle,name=myplot,axis on top,
     ymin=-.2,ymax=5.5,xmin=-.2,xmax=3.5]
    \draw[draw={\colorone},fill={\coloronefill}]
     (axis cs:1,1) parabola bend (axis cs:1,1) (axis cs:3,5) -- cycle;
    \draw [{\colortwo}] (axis cs:2.5,0) --
     node [pos=.4,right,color=black] {\scriptsize $R(x)$} (axis cs:2.5,4);
    \draw [{\colortwo}] (axis cs: 1.5,0) --
     node [pos=.5,right,color=black] {\scriptsize $r(x)$} (axis cs: 1.5,1.25);
  \end{axis}
  \node [right] at (myplot.right of origin) {\scriptsize $x$};
  \node [above] at (myplot.above origin) {\scriptsize $y$};
 \end{tikzpicture}
%\myincludegraphicsthree{width=125pt,3Dmenu,activate=onclick,deactivate=onclick,
%3Droll=97.32968340849395,
%3Dortho=0.004881054162979126,
%3Dc2c=0.43117064237594604 0.05905330181121826 0.9003357887268066,
%3Dcoo=64.78053283691406 10.12905216217041 -47.566158294677734,
%3Droo=149.99999769784193,
%3Dlights=Headlamp,add3Djscript=asylabels.js}{}{figures/figwash1}
\\
%\myincludegraphics{figures/figwash1c}\\
(a)\\
\myincludeasythree{width=\marginparwidth,
3Droll=97.32968340849395,
3Dortho=0.004881054162979126,
3Dc2c=0.43117064237594604 0.05905330181121826 0.9003357887268066,
3Dcoo=64.78053283691406 10.12905216217041 -47.566158294677734,
3Droo=149.99999769784193}{width=\marginparwidth}{figures/figwash1c_3D}\\
%\myincludegraphicsthree{width=125pt,3Dmenu,activate=onclick,deactivate=onclick,
%3Droll=97.32968340849395,
%3Dortho=0.004881054162979126,
%3Dc2c=0.43117064237594604 0.05905330181121826 0.9003357887268066,
%3Dcoo=64.78053283691406 10.12905216217041 -47.566158294677734,
%3Droo=149.99999769784193,
%3Dlights=Headlamp,add3Djscript=asylabels.js}{}{figures/figwash1c}\\
%\myincludegraphics{figures/figwash1c}\\
(b)\\
\myincludegraphicsthree{width=\marginparwidth,3Dmenu,activate=onclick,deactivate=onclick,
3Droll=97.32968340849395,
3Dortho=0.004881054162979126,
3Dc2c=0.43117064237594604 0.05905330181121826 0.9003357887268066,
3Dcoo=64.78053283691406 10.12905216217041 -47.566158294677734,
3Droo=149.99999769784193,
3Dlights=Headlamp,add3Djscript=asylabels.js}{width=\marginparwidth}{figures/figwash1b}\\
%\myincludegraphics{figures/figwash1b}\\
(c)}
%
Rotating about the $x$-axis will produce cross sections in the shape of washers, as shown in \autoref{fig:wash1}(b); the complete solid is shown in part (c). The outside radius of this washer is $R(x) = 2x+1$; the inside radius is $r(x) = x^2-2x+2$. As the region is bounded from $x=1$ to $x=3$, we integrate as follows to compute the volume.
\begin{align*}
V &= \pi\int_1^3 \Big((2x-1)^2-(x^2-2x+2)^2\Big)\ dx \\
		&= \pi\int_1^3 \big(-x^4+4x^3-4x^2+4x-3\big)\ dx \\
		&= \pi\Big[-\frac{1}{5}x^5+x^4-\frac43x^3+2x^2-3x\Big]\Big|_1^3 \\
		&=\frac{104}{15}\pi %\approx 21.78
		\ \text{units}^3.\eoehere
\end{align*}	}

When rotating about a vertical axis, the outside and inside radius functions must be functions of $y$.

\mtable{Sketching the region, a sample slice, and the solid in \autoref{ex_wash_3}.}{fig:ex_sq_and_rt}{%
 \begin{tikzpicture}
  \begin{axis}[width=\marginparwidth,tick label style={font=\scriptsize},
    axis equal,axis y line=middle,axis x line=middle,name=myplot,axis on top,
    xtick=\empty,extra x ticks={1},ytick={1},ymin=-.2,ymax=1.2,xmin=-.1,xmax=1.2]
   \addplot [draw={\coloronefill},fill={\coloronefill},domain=0:1] {sqrt(x)};
   \addplot [draw={\coloronefill},fill={\coloronefill},domain=0:1] {x^2};
   \addplot [draw={\colorone},smooth,thick,domain=0:1] {x^2};
   \addplot [draw={\colorone},smooth,thick,domain=0:1,samples=100] {sqrt x};
   \draw [thick,draw={\colortwo}] (axis cs: 0,.81) --
    node[pos=.4,above,color=black]{\scriptsize $R(x)$}(axis cs:.9,.81);
   \draw [thick,draw={\colortwo}] (axis cs: 0,.5) --
    node[pos=.5,above,color=black]{\scriptsize $r(x)$} (axis cs:.25,.5);
  \end{axis}
  \node [right] at (myplot.right of origin) {\scriptsize $x$};
  \node [above] at (myplot.above origin) {\scriptsize $y$};
 \end{tikzpicture}
 \\ (a) \\
\myincludeasythree{width=\marginparwidth,
3Droll=126.0060482236849,
3Dortho=0.004875946324318647,
3Dc2c=0.44462037086486816 0.39410829544067383 0.8043577671051025,
3Dcoo=67.21983337402344 3.5258262157440186 -29.566415786743164,
3Droo=150.00000160608386}{width=\marginparwidth}{figures/figsq_rt_3D}
% \begin{tikzpicture}
%  \begin{axis}[width=1.16\marginparwidth,tick label style={font=\scriptsize},
%    axis on top,axis y line=none,axis lines=center,y dir=reverse,name=myplot,
%    xtick={1},ztick=\empty,
%    ymin=-3.2,ymax=3.2,xmin=-1.1,xmax=1.1,zmin=-.1,zmax=1.1]
%   \addplot3[domain=.35:1,y domain=90:360, samples y=36,surf, shader=flat,
%     colormap={mp2}{\colormaptwo}] ({.71*cos(y)*(x)},{.71*sin(y)*(x)},.5);
%   \addplot3[domain=.35:1,y domain=0:180, samples y=36,surf, shader=flat,
%     colormap={mp2}{\colormaptwo}] ({.71*sin(y)*(x)},{.71*cos(y)*(x)},.5);
%   \addplot3[domain=0:1,samples y=0,thick,black] (x,0,{sqrt x});
%   \addplot3[domain=0:1,samples y=0,thick,black] (x,0,{x^2});
%   \addplot3[domain=0:1,y domain=0:1,surf,shader=flat,colormap={mp2}{\colormapone},
%     opacity=.2] (x,0,{y*((sqrt x)-x^2)+x^2});
%   \addplot3[domain=0:290,samples y=0,{\colortwo},thick,smooth,]
%    ({.71*cos(x)},{.71*sin(x)},{.5});
%   \addplot3[domain=290:360,samples y=0,{\colortwo},thick,smooth,dashed]
%    ({.71*cos(x)},{.71*sin(x)},{.5});
%   \addplot3[domain=-52:360,samples y=0,{\colortwo},thick,smooth,]
%    ({.25*cos(x)},{.25*sin(x)},{.5});
%  \end{axis}
%  \node [right] at (myplot.right of origin)[shift={(-20pt,-15pt)}] {\scriptsize $x$};
%  \node [above] at (myplot.above origin) [shift={(0,-17pt)}] {\scriptsize $y$};
% \end{tikzpicture}
 \\ (b) \\
\myincludeasythree{width=\marginparwidth,
3Droll=126.0060482236849,
3Dortho=0.004875946324318647,
3Dc2c=0.44462037086486816 0.39410829544067383 0.8043577671051025,
3Dcoo=67.21983337402344 3.5258262157440186 -29.566415786743164,
3Droo=150.00000160608386}{width=\marginparwidth}{figures/figsq_rt_b_3D}
% \begin{tikzpicture}
%  \begin{axis}[width=1.16\marginparwidth,tick label style={font=\scriptsize},
%    axis on top,axis y line=none,axis lines=center,y dir=reverse,name=myplot,
%    xtick={1},ztick=\empty,ymin=-2.2,ymax=2.2,xmin=-1.1,xmax=1.1,zmin=-.1,zmax=1.1]
%   \addplot3[domain=0:1,y domain=0:360, surf,colormap={mp2}{\colormaptwo},opacity=.8,
%     faceted color=black!60,very thin,z buffer=sort,samples=15,samples y=20]
%    ({sin(y)*((sqrt x))},{cos(y)*((sqrt x))},x);
%   \addplot3[domain=0:1,y domain=0:360, surf,colormap={mp2}{\colormapplaneone},
%     opacity=.5,faceted color=black!40,samples=15,samples y=36,very thin,
%     z buffer=sort] ({sin(y)*(x^2)},{cos(y)*(x^2)},x);
%   \addplot3[domain=0:1,samples y=0,black,smooth,dashed] (x,0,{sqrt x});
%   \addplot3[domain=0:1,samples y=0,black,thick,smooth] (x,0,{x^2});
%   \addplot3[domain=0:360,samples y=0,black,thick,smooth,] ({cos(x)},{sin(x)},1);
%   \addplot3[domain=-.2:3.3,samples y=0, black, thick, smooth,dashed] (x,0,2);
%  \end{axis}
%  \node [right] at (myplot.right of origin)[shift={(-20pt,-15pt)}] {\scriptsize $x$};
%  \node [above] at (myplot.above origin) [shift={(0,-17pt)}] {\scriptsize $y$};
% \end{tikzpicture}
 \\ (c)}

\example{ex_wash_3}{Finding volume with the Washer Method}{Find the volume of the solid formed by rotating the region bounded by $y=x^2$ and $x=y^2$ about the $y$-axis.}{In \autoref{fig:ex_sq_and_rt} we have a sketch of the region (a), a sample slice (b), and the solid (c). Rotating about the $y$-axis will produce cross sections in the shape of washers, as shown in Figure (not yet created); the complete solid is shown in  part (c). Since the axis of rotation is vertical, each radius is a function of $y$. The outside radius of this washer is $R(y)=\sqrt y$ and the inside radius is $r(y)=y^2$. As the region is bounded from $y=0$ to $y=1$, we integrate as follows to compute the volume. 
\begin{align*}
V&=\pi \int_0^1 \left((\sqrt y)^2-(y^2)^2\right) \ dy\\
&=\pi \int_0^1 y-y^4 \ dy\\
&=\pi\left.\left[\frac{1}{2} y^2-\frac{1}{5} y^5\right] \right|_0^1\\
&=\frac{3\pi}{10} \text{units}^3.\eoehere
\end{align*}}

\clearpage

\mtable{Sketching the region, a sample slice, and the solid in \autoref{ex_wash2}.}{fig:wash2}{%
 \begin{tikzpicture}
  \begin{axis}[width=1.16\marginparwidth,tick label style={font=\scriptsize},
    axis y line=middle,axis x line=middle,axis lines=center,name=myplot,
    xtick={1,2},ztick={1,2,3},ymin=-.2,ymax=3.2,xmin=-.2,xmax=2.2]
    \draw[draw={\colorone},fill={\coloronefill}] (axis cs:1,1) -- (axis cs:2,1) --(axis cs:2,3)-- cycle;
    \draw [thick,draw={\colortwo}] (axis cs: 0,2.6) --
      node[pos=.4,below,color=black]{\scriptsize$r(x)$} (axis cs: 1.8,2.6);
    \draw [thick,draw={\colortwo}] (axis cs: 0,1.2) --
      node[pos=.3,above,color=black]{\scriptsize$R(x)$} (axis cs: 2,1.2);
   \end{axis}
   \node [right] at (myplot.right of origin) {\scriptsize $x$};
   \node [above] at (myplot.above origin) {\scriptsize $y$};
  \end{tikzpicture}
  \\
%\myincludegraphicsthree{width=125pt,3Dmenu,activate=onclick,deactivate=onclick,
%3Droll=123.42078064170005,
%3Dortho=0.00488104997202754,
%3Dc2c=0.41855040192604065 0.31316813826560974 0.8524911999702454,
%3Dcoo=-7.115118026733398 66.85215759277344 -16.372848510742188,
%3Droo=149.99999819286032,
%3Dlights=Headlamp,add3Djscript=asylabels.js}{}{figures/figwash2a}\\
%%\myincludegraphics{figures/figwash2a} \\
(a) \\
\myincludegraphicsthree{width=125pt,3Dmenu,activate=onclick,deactivate=onclick,
3Droll=123.42078064170005,
3Dortho=0.00488104997202754,
3Dc2c=0.41855040192604065 0.31316813826560974 0.8524911999702454,
3Dcoo=-7.115118026733398 66.85215759277344 -16.372848510742188,
3Droo=149.99999819286032,
3Dlights=Headlamp,add3Djscript=asylabels.js}{}{figures/figwash2b}\\
%\myincludegraphics{figures/figwash2b} \\
(b) \\
\myincludegraphicsthree{width=125pt,3Dmenu,activate=onclick,deactivate=onclick,
3Droll=123.42078064170005,
3Dortho=0.00488104997202754,
3Dc2c=0.41855040192604065 0.31316813826560974 0.8524911999702454,
3Dcoo=-7.115118026733398 66.85215759277344 -16.372848510742188,
3Droo=149.99999819286032,
3Dlights=Headlamp,add3Djscript=asylabels.js}{}{figures/figwash2c}\\
%\myincludegraphics{figures/figwash2c} \\
(c)}

\example{ex_wash2}{Finding volume with the Washer Method}{Find the volume of the solid formed by rotating the triangular region with vertices at $(1,1)$, $(2,1)$ and $(2,3)$ about the $y$-axis.}
{The triangular region is sketched in \autoref{fig:wash2}(a); the sample slice is sketched in (b) and the full solid is drawn in (c). They help us establish the outside and inside radii. Since the axis of rotation is vertical, each radius is a function of $y$. 

The outside radius $R(y)$ is formed by the line connecting $(2,1)$ and $(2,3)$; it is a constant function, as regardless of the $y$-value the distance from the line to the axis of rotation is 2. Thus $R(y)=2$.

The inside radius is formed by the line connecting $(1,1)$ and $(2,3)$. The equation of this line is $y=2x-1$, but we need to refer to it as a function of $y$. Solving for $x$ gives $r(y) = \frac12(y+1)$. 

We integrate over the $y$-bounds of $y=1$ to $y=3$. Thus the volume is
\begin{align*}
V 	&=	\pi\int_1^3\Big(2^2 - \big(\frac12(y+1)\big)^2\Big)\ dy \\
		&=	\pi\int_1^3\Big(-\frac14y^2-\frac12y+\frac{15}4\Big)\ dy \\
		&= 	\pi\Big[-\frac1{12}y^3-\frac14y^2+\frac{15}4y\Big]\Big|_1^3\\
		&= \frac{10}3\pi %\approx 10.47
		\ \text{units}^3.\eoehere
\end{align*}}

In the previous examples, the axis of rotation has either been the $x$ or $y$ axis. We will now consider a problem where the axis of rotation is some other horizontal line.

\clearpage

\mtable{Sketching the solid in \autoref{ex_wash_4}.}{fig:wash4}{%
 \begin{tikzpicture}
  \begin{axis}[width=\marginparwidth,axis equal,
     tick label style={font=\scriptsize},axis y line=middle,
     axis x line=middle,name=myplot,axis on top,
     ymin=-.5,ymax=2.5,xmin=-.2,xmax=1.2]
   \addplot [draw={\coloronefill},fill={\coloronefill},domain=0:1] {sqrt(x)};
   \addplot [draw={\coloronefill},fill={\coloronefill},domain=0:1] {x};
   \addplot [draw={\colorone},smooth,thick,domain=0:1] {x};
   \addplot [draw={\colorone},smooth,thick,domain=0:1,samples=100] {sqrt x};
   \draw [dashed] (axis cs:-.2,2) -- (axis cs:2.2,2);
   \draw [draw={\colortwo},thick] (axis cs:.81,2) --
    node[pos=.5,right,color=black]{\scriptsize$r(x)$} (axis cs:.81,.9);
   \draw [draw={\colortwo},thick] (axis cs:.3,2) --
    node[pos=.5,right,color=black]{\scriptsize$R(x)$} (axis cs:.3,.3);
  \end{axis}
  \node [right] at (myplot.right of origin) {\scriptsize $x$};
  \node [above] at (myplot.above origin) {\scriptsize $y$};
 \end{tikzpicture}
 \\ (a) \\
\myincludeasythree{width=\marginparwidth,
3Droll=126.0060482236849,
3Dortho=0.004875946324318647,
3Dc2c=0.44462037086486816 0.39410829544067383 0.8043577671051025,
3Dcoo=67.21983337402344 3.5258262157440186 -29.566415786743164,
3Droo=150.00000160608386}{width=\marginparwidth}{figures/figwash4_3D}
%\begin{tikzpicture}
%\begin{axis}[width=1.16\marginparwidth,tick label style={font=\scriptsize},
%			axis on top,axis y line=none,axis lines=center,y dir=reverse,name=myplot,
%			xtick={1},ztick=\empty,ymin=-3.5,ymax=3.5,
%			xmin=-1.1,xmax=2.1,zmin=-.5,zmax=4.5]
%\addplot3[domain=.5:.7,y domain=90:270, samples y=36,surf, shader=flat, colormap={mp2}{\colormaptwo}]
%	(.5,{1.28*cos(y)*(.5+x)},{1.28*sin(y)*(.5+x)+2});
%\addplot3[domain=0:1,y domain=0:1,surf,shader=flat,colormap={mp2}{\colormapone}]
%	(x,0,{y*((sqrt x)-(x))+x});
%\addplot3[domain=.5:.7,y domain=0:180, samples y=36,surf, shader=flat, colormap={mp2}{\colormaptwo}]
%	(.5,{1.28*sin(y)*(.5+x)},{1.28*cos(y)*(.5+x)+2});
%\addplot3[domain=0:1,samples y=0,black,thick,smooth,] (x,0,{sqrt x});
%\addplot3[domain=0:1,samples y=0,black,thick,smooth,] (x,0,{x});
%\draw (axis cs:.51,0,.5) --  (axis cs:.51,0,2);
%\draw (axis cs:.49,0,.7) --  (axis cs:.49,0,2);
%\addplot3[domain=0:180,samples y=0,{\colortwo},thick,smooth]
%	(.5,{1.5*sin(x)},{1.5*cos(x)+2});
%\addplot3[domain=198:360,samples y=0,{\colortwo},thick,smooth]
%	(.5,{1.5*sin(x)},{1.5*cos(x)+2});
%\addplot3[domain=0:360,samples y=0,{\colortwo},thick,smooth]
%	(.5,{1.28*cos(x)},{1.28*sin(x)+2});
%\addplot3[domain=-1.1:2.1,samples y=0, black, thick, smooth,dashed] (x,0,2);
%\end{axis}
%\node [right] at (myplot.right of origin)[shift={(-20pt,-15pt)}] {\scriptsize $x$};
%\node [above] at (myplot.above origin) [shift={(0,-17pt)}] {\scriptsize $y$};
%\end{tikzpicture}
\\ (b) \\
\myincludeasythree{width=\marginparwidth,
3Droll=126.0060482236849,
3Dortho=0.004875946324318647,
3Dc2c=0.44462037086486816 0.39410829544067383 0.8043577671051025,
3Dcoo=67.21983337402344 3.5258262157440186 -29.566415786743164,
3Droo=150.00000160608386}{width=\marginparwidth}{figures/figwash4b_3D}
%\begin{tikzpicture}
%\begin{axis}[width=1.16\marginparwidth,tick label style={font=\scriptsize},
%			axis on top,axis y line=none,axis lines=center,y dir=reverse,name=myplot,
%			xtick={1},ztick=\empty,ymin=-3.5,ymax=3.5,
%			xmin=-1.1,xmax=2.1,zmin=-.5,zmax=4.5]
%\addplot3[domain=0:1,y domain=125:360, surf,colormap={mp2}{\colormaptwo},opacity=.8,faceted color=black!60,very thin,z buffer=sort,samples=10,samples y=16] (x,{sin(y)*(2-(sqrt x))},{cos(y)*(2-(sqrt x))+2});
%\addplot3[domain=0:1,y domain=-30:140, surf,colormap={mp2}{\colormapplaneone},opacity=.7,faceted color=black!40,samples=10,samples y=36,very thin,z buffer=sort] (x,{sin(y)*(2-x)},{cos(y)*(2-x)+2});
%\addplot3[domain=0:1,samples y=0,black,smooth,dashed] (x,0,{sqrt x});
%\addplot3[domain=0:1,samples y=0,black,thick,smooth] (x,0,{x});
%\addplot3[domain=0:360,samples y=0,black,thick,smooth,] (1,{cos(x)},{sin(x)+2});
%\addplot3[domain=-85:125,samples y=0,black,thick,smooth,] (0,{2*cos(x)},{2*sin(x)+2});
%\addplot3[domain=-1.1:2.1,samples y=0, black, thick, smooth,dashed] (x,0,2);
%\end{axis}
%\node [right] at (myplot.right of origin)[shift={(-20pt,-15pt)}] {\scriptsize $x$};
%\node [above] at (myplot.above origin) [shift={(0,-17pt)}] {\scriptsize $y$};
%\end{tikzpicture}
\\ (c)}

\example{ex_wash_4}{Finding volume with the Washer Method}{Find the volume of the solid formed by rotating the region bounded by $y=\sqrt x$ and $y=x$ about $y=2$.}{%
\autoref{fig:wash4} shows the region we are rotating (a), a sample slice (b) and the full solid (c). The axis of rotation is horizontal so the radii must be functions of $x$. The radii is the distance from the axis of rotation to the curve so the outside radius of this washer is $R(x)=2-x$ and the inside radius is $r(x)=2-\sqrt x$. The region is bounded from  $x=0$ to $x=1$, thus the volume is
\begin{align*}
V&=\pi \int_0^1 \left( (2-x)^2-(2-\sqrt x)^2\right) \ dx\\
&=\pi \int_0^1 (4-4x+x^2)-(4-4\sqrt x+x)\ dx\\
&=\pi \int_0^1 x^2 -5x+4\sqrt x \ dx\\
&=\pi \left.\left[\frac13 x^3-\frac52 x^2+\frac83 x^{3/2}\right]\right|_0^1\\
&=\frac{\pi}{2} \text{units}^3.\eoehere
\end{align*}}

This section introduced a new application of the definite integral. Our default view of the definite integral is that it gives ``the area under the curve.'' However, we can establish definite integrals that represent other quantities; in this section, we computed volume.

The ultimate goal of this section is not to compute volumes of solids. That can be useful, but what is more useful is the understanding of this basic principle of integral calculus, outlined in \autoref{idea:app_of_defint}: to find the exact value of some quantity, 
\begin{itemize}
	\item we start with an approximation (in this section, slice the solid and approximate the volume of each slice), 
	\item then make the approximation better by refining our original approximation (i.e., use more slices), 
	\item	then use limits to establish a definite integral which gives the exact value.
\end{itemize}

We practice this principle in the next section where we find volumes by slicing solids in a different way.

\printexercises{exercises/07_02_exercises}
