\section{Exponential and Logarithmic Functions}\label{sec:exp_log}

In this section we will define general exponential and logarithmic functions and find their derivatives. 

\subsection{General exponential functions}

\mtable{The function $2^x$ for rational values of $x$.}{fig:two_to_x}{%
\pdftooltip{\begin{tikzpicture}
\begin{axis}[width=\marginparwidth, axis y line=middle,axis x line=middle,
tick label style={font=\scriptsize},
ymin=-1,ymax=5,xmin=-4,xmax=4,name=myplot,xtick={1},ytick={1,2}]
\addplot[draw={\colorone},densely dotted,domain=-4:4]{2^x};
\end{axis}
\node [right] at (myplot.right of origin) {\scriptsize $x$};
\node [above] at (myplot.above origin) {\scriptsize $y$};
\end{tikzpicture}}{ALT-TEXT-TO-BE-DETERMINED}}
Consider first the function $f(x)=2^x$. If $x$ is rational, then we know how to compute $2^x$. What do we mean by $2^\pi$ though? We compute this by first looking at $2^r$ for rational numbers $r$ that are very close to $\pi$, then finding a limit. In our case we might compute $2^3$, $2^{3.1}$, $2^{3.14}$, etc. We then define $2^\pi$ to be the limit of these numbers. Note that this is actually a different kind of limit than we have dealt with before since we only consider rational numbers close to $\pi$, not all real numbers close to $\pi$. We will see one way to make this more precise in \autoref{chapter:sequences_series}. Graphically, we can plot the values of $2^x$ for $x$ rational and get something like the dotted curve in \autoref{fig:two_to_x}. In order to define the remaining values, we are ``connecting the dots'' in a way that makes the function continuous.

It follows from continuity and the properties of limits that exponential functions will satisfy the familiar properties of exponents (see \autoref{sec:deriv_prereqs}).  This implies that
\mtable{The functions $2^x$ and $2^{-x}$.}{fig:two_to_minus_x}{%
\pdftooltip{\begin{tikzpicture}
\begin{axis}[width=\marginparwidth, axis y line=middle,axis x line=middle,
tick label style={font=\scriptsize},
ymin=-1,ymax=5,xmin=-4,xmax=4,name=myplot,xtick={-1,1},ytick={.5,1,2}]
\addplot[draw={\colorone},domain=-4:4]{2^x};
\addplot[draw={\colortwo},domain=-4:4]{2^(-x)};
\end{axis}
\node [right] at (myplot.right of origin) {\scriptsize $x$};
\node [above] at (myplot.above origin) {\scriptsize $y$};
\end{tikzpicture}}{ALT-TEXT-TO-BE-DETERMINED}}
\[\left(\frac12\right)^x =(2^{-1})^x=2^{-x},\]
so the graph of $g(x)=(1/2)^x$ is the reflection of $f$ across the $y$-axis, as in \autoref{fig:two_to_minus_x}.

We can go through the same process as above for any base $a>0$, though we are not usually interested in the constant function $1^x$.

\begin{keyidea}[Properties of Exponential Functions]\label{ki_exp_func_props}
For $a>0$ and $a\neq1$ the exponential function $f(x)=a^x$ satisfies:
\begin{multicols}{2}
\begin{enumerate}\lxAddClass{columns2}
 \item $a^0=1$
 \item $\ds\lim_{x\to\infty}a^x=\begin{cases}\infty&a>1\\0&a<1\end{cases}$
 \item $a^x>0$ for all $x$
 \item $\ds\lim_{x\to-\infty}a^x=\begin{cases}0&a>1\\\infty&a<1\end{cases}$
\end{enumerate}
\end{multicols}
\end{keyidea}

\subsection{Derivatives of exponential functions}

Suppose $f(x)=a^x$ for some $a>0$. We can use the rules of exponents to find the derivative of $f$:
\begin{align*}
	\fp(x)
	&=\lim_{h\to 0}\frac{f(x+h)-f(x)}{h}\\
	&=\lim_{h\to 0}\frac{a^{x+h}-a^x}{h}\\
	&=\lim_{h\to 0}\frac{a^xa^h-a^x}{h}\\
	&=\lim_{h\to 0}\frac{a^x(a^h-1)}{h}\\
	&=a^x\lim_{h\to 0}\frac{a^h-1}{h} \qquad\text{(since $a^x$ does not depend on $h$)}\\
\end{align*}
So we know that $\fp(x)=\ds a^x \lim_{h\to 0}\frac{a^h-1}{h}$, but can we say anything about that remaining limit? First we note that
\[\fp(0)=\lim_{h\to 0}\frac  {a^{0+h}-a^0}{h}=\lim_{h\to 0}\frac{a^h-1}{h},\]
so we have $\fp(x)=a^x\fp(0)$. The actual value of the limit $\ds\lim_{h\to 0}\frac{a^h-1}{h}$ depends on the base $a$, but it can be proved that it does exist. We will figure out just what this limit is later, but for now we note that the easiest differentiation formulas come from using a base $a$ that makes $\ds\lim_{h\to 0}\frac{a^h-1}{h}=1$. This base is the
% irrational ?
number $e\approx 2.71828$ and the exponential function $e^x$ is called the natural exponential function. This leads to the following result.

\begin{theorem}[Derivative of Exponential Functions]\label{thm_baby_exp_deriv}
For any base $a>0$, the exponential function $f(x)=a^x$ has derivative $\fp(x)=a^x\fp(0)$. The natural exponential function $g(x)=e^x$ has derivative $g\primeskip'(x)=e^x$.
\end{theorem}

\youtubeVideo{U3PyUcEd7IU}{Derivatives of Exponential Functions}

%\begin{example}[Finding Exponential Derivatives]\label{ex_exp_derivs}
%Find derivatives of the following functions.
%\[\text{1. }f(x)=e^{x^2}\qquad\text{2. }g(t)=t e^t\]
%\solution
%\begin{enumerate}
%\item We note that this is a composition of functions where the \emph{inside} function is $x^2$. Hence we can apply the Chain Rule to see that: \[\fp(x)=\left(e^{x^2}\right) (2x)\]
%\item In this case we apply the Product Rule for derivatives to find: \[g\primeskip'(t)=e^t+te^t=e^t(t+1).\]
%\end{enumerate}
%\end{example}

\subsection{General logarithmic functions}

% todo Tim move this to 7.0 once that's written
Before reviewing general logarithmic functions, we'll first remind ourselves of the laws of logarithms.

\begin{keyidea}[Properties of Logarithms]\label{ki_log_laws}
For $a,x,y>0$ and $a\ne1$, we have\\
\begin{anywhereenum}
\renewcommand{\arraystretch}{1.4}
\tagpdfsetup{table/tagging=presentation}
\begin{tabular}{ll}
\item$\log_a(xy)=\log_ax+\log_ay$ &
\item$\log_a\dfrac xy=\log_ax-\log_ay$ \\
\item$\log_xy=\dfrac{\log_ay}{\log_ax}$, when $x\ne1$\qquad\null &
\item$\log_ax^y=y\log_ax$ \\
\item$\log_a1=0$ &
\item$\log_aa=1$
\end{tabular}
\end{anywhereenum}
\end{keyidea}

\mtable{The functions $y=a^x$ and $y=\log_a x$ for $a>1$.}{fig_exp_log}{%
\pdftooltip{\begin{tikzpicture}
\begin{axis}[width=\marginparwidth,name=myplot,axis equal,
tick label style={font=\scriptsize},
axis y line=middle,axis x line=middle,ymin=-2,ymax=4,xmin=-2,xmax=4,
xtick={1,2.718},xticklabels={1,$e$},
ytick={1,2.718},yticklabels={1,$e$}]
\addplot[draw={\colorone},domain=-4:1.4]{e^x};
\addplot[draw={\colortwo},domain=.1:4]{ln(x)};
\addplot[dashed,domain=-4:4]{x};
\end{axis}
\node [right] at (myplot.right of origin) {\scriptsize $x$};
\node [above] at (myplot.above origin) {\scriptsize $y$};
\end{tikzpicture}}{ALT-TEXT-TO-BE-DETERMINED}}
Let us consider the function $f(x)=a^x$ where $a\neq1$. We know that $\fp(x)=\fp(0)a^x$, where $\fp(0)$ is a constant that depends on the base $a$. Since $a^x>0$ for all $x$, this implies that $\fp(x)$ is either always positive or always negative, depending on the sign of $\fp(0)$. This in turn implies that $f$ is strictly monotonic, so $f$ is one-to-one. We can now say that $f$ has an inverse. We call this inverse the logarithm with base $a$, denoted $f^{-1}(x)=\log_ax$. When $a=e$, this is the natural logarithm function $\ln x$. So we can say that $y=\log_a x$ if and only if $a^y=x$. Since the range of the exponential function is the set of positive real numbers, the domain of the logarithm function is also the set of positive real numbers. Reflecting the graph of $y=a^x$ across the line $y=x$ we find that (for $a>1$) the graph of the logarithm looks like \autoref{fig_exp_log}.

\begin{keyidea}[Properties of Logarithmic Functions]\label{ki_log_func_props}
For $a>0$ and $a\ne1$ the logarithmic function $f(x)=\log_a x$ satisfies:
\begin{enumerate}
\item The domain of $f(x)=log_a x$ is $(0,\infty)$ and the range is $(-\infty,\infty)$.
\item $y=\log_a x$ if and only if $a^y=x$.
\item $\ds\lim_{x\to\infty}\log_a x=
\begin{cases}\infty&\text{ if $a>1$}\\ -\infty&\text{ if $a<1$}\end{cases}$
\item $\ds\lim_{x\to0^+}\log_a x=\begin{cases}-\infty&\text{ if $a>1$}\\\infty&\text{ if $a<1$}\end{cases}$
\end{enumerate}
\end{keyidea}

Using the inverse of the natural exponential function, we can determine what the value of $\fp(0)$ is in the formula $(a^x)'=\fp(0)a^x$. To do so, we note that $a=e^{\ln a}$ since the exponential and logarithm functions are inverses. Hence we can write:
\[a^x=\left(e^{\ln a}\right)^x=e^{x\ln a}\]
Now since $\ln a$ is a constant, we can use the Chain Rule to see that:
\[\frac d{\dd x} a^x=\frac d{\dd x} e^{x\ln a} =e^{x\ln a}(\ln a) =a^x\ln a\]
Comparing this to our previous result, we can restate our theorem:

\begin{theorem}[Derivative of Exponential Functions]\label{thm_exp_deriv}
For any base $a>0$, the exponential function $f(x)=a^x$ has derivative $\fp(x)=a^x\ln a$. The natural exponential function $g(x)=e^x$ has derivative $g\primeskip'(x)=e^x$.
\end{theorem}

\subsection{Change of base}

In the previous computation, we found it convenient to rewrite the general exponential function in terms of the natural exponential function. A related formula allows us to rewrite the general logarithmic function in terms of the natural logarithm.  To see how this works, suppose that $y=\log_ax$, then we have:
\begin{align*}
a^y&=x \\
\ln(a^y)&=\ln x\\
y\ln a&=\ln x\\
y&=\frac{\ln x}{\ln a}\\
\log_a x&=\frac{\ln x}{\ln a}.
\end{align*}
This change of base formula allows us to use facts about the natural logarithm to derive facts about the general logarithm.

\subsection{Derivatives of logarithmic functions}

Since the natural logarithm function is the inverse of the natural exponential function, we can use the formula $(f^{-1}(x))'=\dfrac1{\fp(f^{-1}(x))}$ to find the derivative of $y=\ln x$. We know that $\dfrac\dd{\dd x}e^x=e^x$, so we get:
\[\frac\dd{\dd x}\ln x=\frac1{e^y}=\frac1{e^{\ln x}}=\frac1x.\]
Now we
%In \autoref{sec:deriv_inverse_function}, \autoref{ex_deriv_lnx}, we found that $\frac{\dd}{\dd x}\ln x=\frac1x$. We
can apply the change of base formula to find the derivative of a general logarithmic function:
\[\frac{\dd}{\dd x}\log_ax=\frac{\dd}{\dd x}\left(\frac{\ln x}{\ln a}\right) =\frac 1{\ln a}\left(\frac{\dd}{\dd x}\ln x\right)=\frac 1{x\ln a}.\]

\begin{example}[Finding Derivatives of Logs and Exponentials]\label{ex_find_d_exp_log}
Find derivatives of the following functions.
\[
 \text{1.}\quad f(x)=x3^{4x-7}\qquad
 \text{2.}\quad g(x)=2^{x^2}\qquad
 \text{3.}\quad h(x)=\frac x{\log_5x}
\]
\solution
\begin{enumerate}
\item We apply both the Product and Chain Rules:
\[\fp(x)=3^{4x-7}+x\left(3^{4x-7}\ln 3\right)(4)=(1+4x\ln 3)3^{4x-7}\]
\item We apply the Chain Rule:
\[g\primeskip'(x)=2^{x^2}(\ln 2)(2x)=2^{x^2+1}x\ln2.\]
\item Applying the Quotient Rule:
\[h\primeskip'(x)=\frac{\log_5x-x\left(\frac1{x\ln 5}\right)}{(\log_5x)^2}=\frac{(\log_5x)(\ln 5)-1}{(\log_5x)^2\ln 5}\]
\end{enumerate}
\end{example}

\begin{example}[The Derivative of the Natural Log]\label{ex_d_ln_abs_x}
Find the derivative of the function $y=\ln\abs x$.
\solution
We can rewrite our function as
\[y=\begin{cases} \ln x & \text{if $x>0$}\\ \ln(-x) & \text{if $x<0$}\\ \end{cases}\]
Applying the Chain Rule, we see that $\frac{\dd y}{\dd x}=\frac 1x$ for $x>0$, and $\frac{\dd y}{\dd x}=\frac{-1}{-x}=\frac 1x$ for $x<0$. Hence we have
\[\frac{\dd}{\dd x}\ln\abs x=\frac 1x \quad\text{ for $x\neq0$.}\]
\end{example}

\subsection{Antiderivatives}

Combining these new results, we arrive at the following theorem:

\begin{theorem}[Derivatives and Antiderivatives of Exponentials and Logarithms]\label{thm_int_exp_log}
Given a base $a>0$ and $a\neq 1$, the following hold:\\[-1.5\baselineskip]
\begin{multicols}{2}
\begin{enumerate}\lxAddClass{columns2}
\item $\dfrac\dd{\dd x}e^x=e^x$
\item $\dfrac\dd{\dd x}a^x=a^x\ln a$
\item $\dfrac\dd{\dd x}\ln x=\dfrac1x$
\item $\dfrac\dd{\dd x}\log_a x=\dfrac1{x\ln a}$
\item $\ds\int e^x\dd x=e^x+C$
\item $\ds\int a^x\dd x=\frac{a^x}{\ln a}+C$
\item $\ds\int \frac{\dd x}{x}=\ln\abs x+C$
\item[]
\end{enumerate}
\end{multicols}
\end{theorem}

\begin{example}[Finding Antiderivatives]\label{ex_exp_log_anti}
Find the following antiderivatives.
\[
 \text{1. }\int 3^x\dd x\qquad
 \text{2. }\int x^2 e^{x^3}\dd x\qquad
 \text{3. }\int \frac{x\dd x}{x^2+1}
\]
\solution
\begin{enumerate}
\item Applying our theorem,
\[\int 3^x\dd x=\frac{3^x}{\ln 3}+C\]
\item We use the substitution $u=x^3$, $\dd u=3x^2\dd x$:
\begin{align*}
\int x^2e^{x^3}\dd x &=\frac13 \int e^u\dd u\\
&=\frac 13 e^u+C\\
&=\frac 13 e^{x^3}+C\\
\end{align*}
\item Using the substitution $u=x^2+1$, $\dd u=2x\dd x$:
\begin{align*}
\int \frac{x\dd x}{x^2+1}&=\frac 12 \int \frac{\dd u}{u}\\
&=\frac 12 \ln\abs u+C\\
&=\frac 12 \ln\abs{x^2+1}+C\\
&=\frac 12 \ln(x^2+1)+C
\end{align*}
\end{enumerate}
\end{example}

Note that we do not yet have an antiderivative for the function $f(x)=\ln x$. We remedy this in \autoref{sec:IBP} with \autoref{ex_ibp5}.
% although see Example 2.4.4

%with the following example.
%
%example: Compute $\int \ln x\dd x$.
%
%\solution While this does not look like a product of functions, we find integration by parts useful. In particular we use $u=\ln x$, $\dd u=\dd x/x$, $\dd v=\dd x$, and $v=x$:
%\begin{align*}
%\int \ln x\dd x &=x\ln x-\int (x)\left(\frac{\dd x}x\right)\\
%&=x\ln x-\int \dd x\\
%&=x\ln x-x+C\\
%\end{align*}

\subsection{Logarithmic Differentiation}

\mtable{A plot of $y=x^x$.}{fig:logdiffa}{\pdftooltip{\begin{tikzpicture}
\begin{axis}[width=1.16\marginparwidth,tick label style={font=\scriptsize},
axis y line=middle,axis x line=middle,name=myplot,
ymin=-.1,ymax=4.2,xmin=-.1,xmax=2.2]
\addplot[draw={\colorone},smooth,thick,domain=.01:2] {exp(x*ln(x))};
\draw [fill=white,draw=black,thick] (axis cs:0,1) circle (1.5pt);
\end{axis}
\node [right] at (myplot.right of origin) {\scriptsize $x$};
\node [above] at (myplot.above origin) {\scriptsize $y$};
\end{tikzpicture}}{ALT-TEXT-TO-BE-DETERMINED}}

Consider the function $y=x^x$; it is graphed in \autoref{fig:logdiffa}. It is well-defined for $x>0$ and we might be interested in finding equations of lines tangent and normal to its graph. How do we take its derivative?\index{logarithmic differentiation}

The function is not a power function: it has a ``power'' of $x$, not a constant. It is not an exponential function: it has a ``base'' of $x$, not a constant. 

A differentiation technique known as \emph{logarithmic differentiation} becomes useful here. The basic principle is this: take the natural log of both sides of an equation $y=f(x)$, then use implicit differentiation to find $y\primeskip'$. We demonstrate this in the following example.

\begin{example}[Using Logarithmic Differentiation]\label{ex_implicit10}
Given $y=x^x$, use logarithmic differentiation to find $y\primeskip'$.
\solution
As suggested above, we start by taking the natural log of both sides then applying implicit differentiation.
\begin{align*}
y &= x^x \\
\ln (y) &= \ln (x^x) && \text{\small (apply logarithm rule)}\\
\ln (y) &= x\ln x && \text{\small (now use implicit differentiation)}\\
\frac{\dd}{\dd x}\Bigl(\ln (y)\Bigr) &= \frac{\dd}{\dd x}\Bigl(x\ln x\Bigr) \\
\frac{y\primeskip'}{y} &= \ln x + x\cdot\frac1x\\
\frac{y\primeskip'}{y} &= \ln x + 1\\
y\primeskip'
&= y\bigl(\ln x+1\bigr) && \text{\small (substitute $y=x^x$)}\\
y\primeskip' &= x^x\bigl(\ln x+1\bigr).
\end{align*} 

\mtable{A graph of $y=x^x$ and its tangent line at $x=1.5$.}{fig:implicit10}{\pdftooltip{\begin{tikzpicture}
\begin{axis}[width=1.16\marginparwidth,tick label style={font=\scriptsize},
axis y line=middle,axis x line=middle,name=myplot,
ymin=-.1,ymax=4.2,xmin=-.1,xmax=2.2]
\addplot[draw={\colorone},smooth,thick,domain=.01:2] {exp(x*ln(x))};
\addplot [draw={\colortwo},thick,domain=.9:2] {2.582*(x-1.5)+1.837};
\draw [fill=white,draw=black,thick] (axis cs:0,1) circle (1.5pt);
\end{axis}
\node [right] at (myplot.right of origin) {\scriptsize $x$};
\node [above] at (myplot.above origin) {\scriptsize $y$};
\end{tikzpicture}}{ALT-TEXT-TO-BE-DETERMINED}}

To ``test'' our answer, let's use it to find the equation of the tangent line at $x=1.5$. The point on the graph our tangent line must pass through is $(1.5, 1.5^{1.5}) \approx (1.5, 1.837)$. Using the equation for $y\primeskip'$, we find the slope as
\[y\primeskip' = 1.5^{1.5}\bigl(\ln 1.5+1\bigr) \approx 1.837(1.405) \approx 2.582.\]
Thus the equation of the tangent line is $y = 2.582(x-1.5)+1.837$. \autoref{fig:implicit10} graphs $y=x^x$ along with this tangent line.
\end{example}

\printexercises{exercises/07-exp-log-exercises}

% todo write another logarithmic differentiation example
