\section{The Shell Method}\label{sec:shell_method}

Often a given problem can be solved in more than one way. A particular method may be chosen out of convenience, personal preference, or perhaps necessity. Ultimately, it is good to have options.

The previous section introduced the Disk and Washer Methods, which computed the volume of solids of revolution by integrating the cross-sectional area of the solid. This section develops another method of computing volume, the \textbf{Shell Method.} Instead of slicing the solid perpendicular to the axis of rotation creating cross-sections, we now slice it parallel to the axis of rotation, creating ``shells.''

\mtable[-.5in]{Introducing the Shell Method.}{fig:shell_intro}{%
  \begin{tikzpicture}
   \begin{axis}[width=1.16\marginparwidth,
     tick label style={font=\scriptsize},axis y line=middle,
     axis x line=middle,name=myplot,axis on top,
     ymin=-.2,ymax=1.2,xmin=-.2,xmax=1.2]
    \draw[draw={\colorone},fill={\coloronefill}] plot[smooth,domain=0:1]
     (axis cs:\x,{1/(\x*\x+1)}) |-(axis cs:1,0)
      --(axis cs:1,0)--(axis cs:0,0)
      -- cycle;
    \draw [thick,draw={\colortwo}] (axis cs: .75,0) -- (axis cs:.75,.64);
    \draw (axis cs:.8,.85) node[above] {\scriptsize$y=\frac1{1+x^2}$};
   \end{axis}
   \node [right] at (myplot.right of origin) {\scriptsize $x$};
   \node [above] at (myplot.above origin) {\scriptsize $y$};
  \end{tikzpicture}
\\(a)\smallskip\\
\myincludeasythree{width=\marginparwidth,
3Droll=134.9923368693739,
3Dortho=0.00484071671962738,
3Dc2c=0.3376615047454834 0.4032796025276184 0.8504999876022339,
3Dcoo=3.760643243789673 66.28701782226562 -12.64914608001709,
3Droo=149.99997892897957}{width=\marginparwidth}{figures/figshell_intro_d_3D}
\\(b)\smallskip\\
\myincludeasythree{width=\marginparwidth,
3Droll=134.9923368693739,
3Dortho=0.00484071671962738,
3Dc2c=0.3376615047454834 0.4032796025276184 0.8504999876022339,
3Dcoo=3.760643243789673 66.28701782226562 -12.64914608001709,
3Droo=149.99997892897957}{width=\marginparwidth}{figures/figshell_intro_a_3D}
\\(c)}

Consider \autoref{fig:shell_intro}, where the region shown in (a) is rotated around the $y$-axis forming the solid shown in (c). A small slice of the region is drawn in (a), \emph{parallel} to the axis of rotation. When the region is rotated, this thin slice forms a \textbf{cylindrical shell}, as pictured in part (b) of the figure. The previous section approximated a solid with lots of thin disks (or washers); we now approximate a solid with many thin cylindrical shells.

To compute the volume of one shell, first consider the paper label on a soup can with radius $r$ and height $h$. What is the area of this label? A simple way of determining this is to cut the label and lay it out flat, forming a rectangle with height $h$ and length $2\pi r$. Thus the area is $A = 2\pi rh$; see \autoref{fig:soupcan}(a).

Do a similar process with a cylindrical shell, with height $h$, thickness $\Delta x$, and approximate radius $r$. Cutting the shell and laying it flat forms a rectangular solid with length $2\pi r$, height $h$ and depth $\Delta x$. Thus the volume is $V \approx 2\pi rh\Delta x$; see \autoref{fig:soupcan}(b). (We say ``approximately'' since our radius was an approximation.)

By breaking the solid into $n$ cylindrical shells, we can approximate the volume of the solid as
\[V \approx \sum_{i=1}^n 2\pi r_ih_i\Delta x,\]
where $r_i$, $h_i$ and $\Delta x$ are the radius, height and thickness of the $i\,^\text{th}$ shell, respectively. 

This is a Riemann Sum. Taking a limit as the thickness of the shells approaches 0 leads to a definite integral.
\begin{align*}
V&=\lim_{n\to \infty} \sum_{i=1}^n 2\pi r_i h_i \Delta x\\
&=2\pi \int_a^b r(x)h(x)\dd x
\end{align*}

\begin{figure}[!hbt]
\flushinner{%
\begin{tabular}{cc}
% todo Tim turn these into asymptote files
\begin{tikzpicture}[scale=.67]
 \draw [draw={\colorone},thick,left color=\colorone!60,right color=\colorone!20]
  (0,0) ellipse [x radius=1.5cm, y radius=1cm];
\begin{scope}[xscale=1.5]
 \draw [draw={\colorone},thick,left color=\colorone!60,right color=\colorone!20]
 (-1,0) -- (-1,-3.5) arc (180:360:1) -- (1,0) arc (360:180:1);
 \draw [draw={\colorone},fill=white](-1,-.2) -- node [pos=.5,left,color=black] {\scriptsize $h$} (-1,-3.3) arc (180:360:1) -- (1,-.2) arc (360:180:1);
\end{scope}
 \draw [draw={\colorone},dashed] (1.06,-.91) -- node [pos=.5, above,rotate=90,color=black] {\scriptsize cut here} (1.06,-4.01);
 \draw (0,0) -- node [pos=.5,above] {\scriptsize $r$} (1.5,0);
 \draw [->,>=latex,ultra thick] (2,-2)--(3,-2);
\draw[draw={\colorone}] (3.5,-.5) -- node [pos=.5,above,color=black] {\scriptsize $2\pi r$} (8.5,-.5) -- node [pos=.5,right,color=black] {\scriptsize $h$} (8.5,-3.6)--(3.5,-3.6) -- cycle;
\draw (6,-2.05) node {\scriptsize $A = 2\pi r h$};
\end{tikzpicture}
&
\begin{tikzpicture}[scale=.67]
\draw [draw={\colorone},thick,left color=\colorone!60,right color=\colorone!20]
 (0,0) ellipse [x radius=1.5cm,y radius=1cm];
\begin{scope}[xscale=1.5]
\draw [draw={\colorone},thick,left color=\colorone!60,right color=\colorone!20]
 (-1,0) -- (-1,-3.5)  arc (180:360:1) -- (1,0) arc (360:180:1);
\end{scope}
\draw [draw={\colorone},thick,left color=\colorone!20,right color=\colorone!60]
 (0,0) ellipse [x radius=1.2cm,y radius=.8cm];
\draw (-1.5,-1.75) node [left,color=black] {\scriptsize $h$};
\draw [draw={\colorone},dashed] (.8475,-.565) --(1.065,-.71) -- node [pos=.5, above,rotate=90,color=black] {\scriptsize cut here} (1.065,-4.2);
\draw (0,0) -- node [pos=.5,above] {\scriptsize $r$} (1.35,0);
\draw [->,>=stealth](-1.8,.3) node [above] {\scriptsize $\Delta x$}-- (-1.3,0);
\draw [->,>=latex,ultra thick] (2,-2)--(3,-2);
\draw [draw={\colorone},thick,left color=\colorone!60,right color=\colorone!20]
 (3.5,-.3) -- (3.7,-.1) --node [pos=.5,above,color=black] {\scriptsize $2\pi r$} (8.7,-.1) -- (8.5,-.3);
\draw [draw={\colorone},thick,left color=\colorone!60,right color=\colorone!20]
 (8.5,-.3) -- (8.7,-.1) --node [pos=.5,right,color=black] {\scriptsize $h$} (8.7,-3.6) -- (8.5,-3.8);
\draw [draw={\colorone},thick,left color=\colorone!60,right color=\colorone!20]
 (3.5,-.3) --  (8.5,-.3) --  (8.5,-3.8)--(3.5,-3.8) -- cycle;
\draw [>=stealth,->] (4,0) node [above] {\scriptsize $\Delta x$} --(3.8,-.2);
\draw (6,-2.05) node {\scriptsize $V\approx 2\pi r h\Delta x$};
\end{tikzpicture}
\\
(a) & (b)
\end{tabular}}
\caption{Determining the volume of a thin cylindrical shell.%
\label{fig:soupcan}}
% otherwise, `Package nag Warning: \label in float, but not after \caption'
\end{figure}

\bigskip

\begin{keyidea}[The Shell Method]\label{idea:shell_method}
Let a solid be formed by revolving a region $R$, bounded by $x=a$ and $x=b$, around a vertical axis. Let $r(x)$ represent the distance from the axis of rotation to $x$ (i.e., the radius of a sample shell) and let $h(x)$ represent the height of the solid at $x$ (i.e., the height of the shell). The volume of the solid is 
\index{integration!volume!Shell Method}\index{Shell Method}
\[V = 2\pi\int_a^b r(x)h(x)\dd x.\]
\end{keyidea}

\textbf{Special Cases:}
	\begin{enumerate}
	\item		When the region $R$ is bounded above by $y=f(x)$ and below by $y=g(x)$, then $h(x) = f(x)-g(x)$.
	\item		When the axis of rotation is the $y$-axis (i.e., $x=0$) then $r(x) = x$.
	\end{enumerate}

\youtubeVideo{V6nTsxumjgU}{Volumes of Revolution --- Cylindrical Shells}

Let's practice using the Shell Method.

\begin{example}[Finding volume using the Shell Method]\label{ex_shell1}
Find the volume of the solid formed by rotating the region bounded by $y=0$, $y=1/(1+x^2)$, $x=0$ and $x=1$ about the $y$-axis.
%
\mtable[-2\baselineskip]{Graphing a region in \autoref{ex_shell1}.}{fig:shell1}{\begin{tikzpicture}
\begin{axis}[width=1.16\marginparwidth,tick label style={font=\scriptsize},
axis y line=middle,axis x line=middle,name=myplot,axis on top,xtick={1},
extra x ticks={.4},extra x tick labels={$x$},ytick={1},
ymin=-.2,ymax=1.25,xmin=-.15,xmax=1.1,]
\addplot [draw={\coloronefill},thick,smooth,domain=0:1,fill={\coloronefill}] plot {(1/(1+x^2))} -- (axis cs:1,0) -- (axis cs:0,0) -- cycle;
\addplot [draw={\colorone},thick,smooth,domain=0:1] plot {(1/(1+x^2))};
\draw [thick,draw={\colortwo}] (axis cs:.4,0) -- (axis cs:.4,.86);
\draw (axis cs:.4,.45) node [left] {\scriptsize$ h(x)\left\{\rule{0pt}{33pt}\right.$}
      (axis cs:.2,-.12) node {\scriptsize $\underbrace{\makebox[35pt]{}}_{r(x)}$}
      (axis cs: .7,1) node {\scriptsize $\ds y=\frac{1}{1+x^2}$};
\end{axis}
\node [right] at (myplot.right of origin) {\scriptsize $x$};
\node [above] at (myplot.above origin) {\scriptsize $y$};
\end{tikzpicture}}
%
\solution
This is the region used to introduce the Shell Method in \autoref{fig:shell_intro}, but is sketched again in \autoref{fig:shell1} for closer reference. A line is drawn in the region parallel to the axis of rotation representing a shell that will be carved out as the region is rotated about the $y$-axis.

The distance this line is from the axis of rotation determines $r(x)$; as the distance from $x$ to the $y$-axis is $x$, we have $r(x)=x$. The height of this line determines $h(x)$; the top of the line is at $y=1/(1+x^2)$, whereas the bottom of the line is at $y=0$. Thus $h(x) = 1/(1+x^2)-0 = 1/(1+x^2)$. The region is bounded from $x=0$ to $x=1$, so the volume is 
\begin{align*}
	V
	&= 2\pi\int_0^1 \frac{x}{1+x^2}\dd x.
%\end{align*}
\intertext{This requires substitution. Let $u=1+x^2$, so $\dd u = 2x\dd x$. We also change the bounds: $u(0) = 1$ and $u(1) = 2$. Thus we have:}
%\begin{align*}
	&= \pi\int_1^2 \frac{1}{u}\dd u \\
	&= \pi\ln u\Big|_1^2\\
	&= \pi\ln 2 %\approx 2.178
	\ \text{units}^3.
\end{align*}
Note: in order to find this volume using the Disk Method, two integrals would be needed to account for the regions above and below $y=1/2$.
\end{example}

With the Shell Method, nothing special needs to be accounted for to compute the volume of a solid that has a hole in the middle, as demonstrated next.

\mtable[-3in]{Graphing a region in \autoref{ex_shell2}.}{fig:shell2}{%
\begin{tikzpicture}
\begin{axis}[width=1.16\marginparwidth,tick label style={font=\scriptsize},
axis y line=middle,axis x line=middle,name=myplot,axis on top,xtick={1,2,3},
extra x ticks={.4},extra x tick labels={$x$},ytick={1,2,3},
ymin=-.2,ymax=3.25,xmin=-.15,xmax=3.2,]
\draw [draw={\colorone},thick,smooth,,fill={\coloronefill}] (axis cs:0,1) -- node [black,rotate=57,pos=.5,above]  {\scriptsize $y=2x+1$} (axis cs:1,3) --  (axis cs: 1,1) -- cycle;
\draw [dashed] (axis cs:3,0) -- (axis cs:3,3.2);
\draw [thick,draw={\colortwo}] (axis cs: .4,1) -- (axis cs:.4,1.8);
\draw(axis cs:1,1.4)node [right] {\scriptsize$ \left.\rule{0pt}{14pt}\right\} h(x).$}
     (axis cs:1.7,.7) node { $\underbrace{\makebox[100pt]{}}_{r(x)}$};
\end{axis}
\node [right] at (myplot.right of origin) {\scriptsize $x$};
\node [above] at (myplot.above origin) {\scriptsize $y$};
\end{tikzpicture}
\\(a)\smallskip\\
\myincludeasythree{width=\marginparwidth,
3Droll=138.26595091591264,
3Dortho=0.0048407199792563915,
3Dc2c=0.16831636428833008 0.19470007717609406 0.966313362121582,
3Dcoo=74.36175537109375 71.21556854248047 -40.660457611083984,
3Droo=149.9999886313645}{width=\marginparwidth}{figures/figshell2b_3D}
\\(b)\smallskip\\
\myincludeasythree{width=\marginparwidth,
3Droll=138.26595091591264,
3Dortho=0.0048407199792563915,
3Dc2c=0.16831636428833008 0.19470007717609406 0.966313362121582,
3Dcoo=74.36175537109375 71.21556854248047 -40.660457611083984,
3Droo=149.9999886313645}{width=\marginparwidth}{figures/figshell2c_3D}
\\(c)}

\begin{example}[Finding volume using the Shell Method]\label{ex_shell2}
Find the volume of the solid formed by rotating the triangular region determined by the points $(0,1)$, $(1,1)$ and $(1,3)$ about the line $x=3$.
\solution
The region is sketched in \autoref{fig:shell2}(a) along with a line within the region parallel to the axis of rotation. In part (b) of the figure, we see a sample shell, and in part (c) the whole solid is shown.

The height of the sample shell is the distance from $y=1$ to $y=2x+1$, the line that connects the points $(0,1)$ and $(1,3)$. Thus $h(x) = 2x+1-1 = 2x$. The radius of the sample shell is the distance from $x$ to $x=3$; that is, it is $r(x)=3-x$. The $x$-bounds of the region are $x=0$ to $x=1$, giving
\begin{align*}
	V
	&= 2\pi\int_0^1 (3-x)(2x)\dd x \\
	&= 2\pi\int_0^1 \bigl(6x-2x^2\bigr)\dd x \\
	&= 2\pi\left.\left(3x^2-\frac23x^3\right)\right|_0^1\\
	&= \frac{14}{3}\pi%\approx 14.66
	\ \text{units}^3.
\end{align*}
\end{example}

When revolving a region around a horizontal axis, we must consider the radius and height functions in terms of $y$, not $x$.

\mtable[-1in]{Graphing a region in \autoref{ex_shell3}.}{fig:shell3}{%
\begin{tikzpicture}
\begin{axis}[width=1.16\marginparwidth,tick label style={font=\scriptsize},
axis y line=middle,axis x line=middle,name=myplot,axis on top,xtick={1,2,3},
ytick={1,2,3},extra y ticks={1.8},extra y tick labels={$y$},
ymin=-.2,ymax=3.25,xmin=-.15,xmax=1.5,]
\draw [draw={\colorone},thick,smooth,,fill={\coloronefill}] (axis cs:0,1) -- node [black,rotate=41,pos=.5,above]  {\scriptsize $x=\frac12y-\frac12$} (axis cs:1,3) --  (axis cs: 1,1) -- cycle;
\draw [thick,draw={\colortwo}] (axis cs: .4,1.8) -- node [black,pos=.5,below] {$\underbrace{\makebox[42pt]{}}_{h(y)}$} (axis cs:1,1.8);
\draw (axis cs:1,.9) node [right] {\scriptsize$ \left.\rule{0pt}{29pt}\right\} r(y)$};
\end{axis}
\node [right] at (myplot.right of origin) {\scriptsize $x$};
\node [above] at (myplot.above origin) {\scriptsize $y$};
\end{tikzpicture}
\\(a) \\
\myincludeasythree{width=\marginparwidth,
3Droll=97.25039174241927,
3Dortho=0.0048407199792563915,
3Dc2c=0.34401965141296387 0.04830114170908928 0.9377192854881287,
3Dcoo=71.46640014648438 9.081968307495117 -11.759183883666992,
3Droo=149.99998743176678}{width=\marginparwidth}{figures/figshell3b_3D}
\\(b)\\
\myincludeasythree{width=\marginparwidth,
3Droll=97.25039174241927,
3Dortho=0.0048407199792563915,
3Dc2c=0.34401965141296387 0.04830114170908928 0.9377192854881287,
3Dcoo=71.46640014648438 9.081968307495117 -11.759183883666992,
3Droo=149.99998743176678}{width=\marginparwidth}{figures/figshell3c_3D}
\\(c)}

\begin{example}[Finding volume using the Shell Method]\label{ex_shell3}
Find the volume of the solid formed by rotating the region given in \autoref{ex_shell2} about the $x$-axis.
\solution
The region is sketched in \autoref{fig:shell3}(a). In part (b) of the figure the sample shell is drawn, and the solid is sketched in (c). (Note that the triangular region looks ``short and wide'' here, whereas in the previous example the same region looked ``tall and narrow.'' This is because the bounds on the graphs are different.)

The height of the sample shell is an $x$-distance, between $x=\frac12y-\frac12$ and $x=1$. Thus $h(y) = 1-(\frac12y-\frac12) = -\frac12y+\frac32.$ The radius is the distance from $y$ to the $x$-axis, so $r(y) =y$. The $y$ bounds of the region are $y=1$ and $y=3$, leading to the integral
\begin{align*}
V &= 2\pi\int_1^3\left[y\left(-\frac12y+\frac32\right)\right]\dd y \\
	&= 2\pi\int_1^3\left[-\frac12y^2+\frac32y\right]\dd y \\
	&= 2\pi\left.\left[-\frac16y^3+\frac34y^2\right]\right|_1^3 \\
	&= 2\pi\left[\frac94-\frac7{12}\right]\\
	&=	\frac{10}{3}\pi %\approx 10.472
	\ \text{units}^3.
\end{align*}
\end{example}

The following example shows how there are times when it does not matter which method you choose to evaluate the volume of a solid. In \autoref{ex_wash_4} we found the volume of the solid formed by rotating the region bounded by $y=\sqrt x$ and $y=x$ about $y=2$. We will now demonstrate how to find the volume with the shell method. Note that your answer should be the same whichever method you choose.

\begin{example}[Using the shell method instead of the washer method]\label{ex_shell_wash_eq}
Find the volume of the solid formed by rotating the region bounded by $y=\sqrt x$ and $y=x$ about $y=2$ using the Shell Method.
\solution
Since our shells are parallel to the axis of rotation, we must consider the radius and height functions in terms of $y$. The radius of a sample shell will be $r(y)=2-y$ and the height of a sample shell will be $h(y)=y-y^2$. The $y$ bounds for the region will be $y=0$ to $y=1$ resulting in the integral
\begin{align*}
V&=2\pi \int_0^1 (2-y)(y-y^2)\dd y\\
&=2\pi\int_0^1 y^3-3y^2+2y\dd y\\
&=\frac{\pi}{2} \text{units}^3.
\end{align*}
\end{example}

At the beginning of this section it was stated that ``it is good to have options.'' The next example finds the volume of a solid rather easily with the Shell Method, but using the Washer Method would be quite a chore.

\mtable[-3in]{Graphing a region in \autoref{ex_shell4}.}{fig:shell4}{%
\begin{tikzpicture}
\begin{axis}[width=\marginparwidth,tick label style={font=\scriptsize},
			axis y line=middle,axis x line=middle,name=myplot,axis on top,
			xtick=\empty,extra x ticks={1,2,3},ytick={1,2},
			ymin=-.2,ymax=2.5,xmin=-.1,xmax=3.5]
\addplot [draw={\coloronefill},fill={\coloronefill},domain=0:2] {3*x-x^2};
\addplot [draw={\colorone},smooth,thick,domain=0:3] {3*x-x^2};
\addplot[draw={\colorone},smooth,thick,domain=0:3] {x};
\draw [thick,draw={\colortwo}] (axis cs: 1,1) -- (axis cs:1,2);
\draw (axis cs: .9,1.5) node[right]{\scriptsize$ \left.\rule{0pt}{22pt}\right\} h(x)$};
\draw (axis cs:.5,1.1) node [below] {$\underbrace{\makebox[35pt]{}}_{r(x)}$} ;
\end{axis}
\node [right] at (myplot.right of origin) {\scriptsize $x$};
\node [above] at (myplot.above origin) {\scriptsize $y$};
\end{tikzpicture}
\\(a)\smallskip\\
\myincludeasythree{width=\marginparwidth,
3Droll=120.03842998357472,
3Dortho=0.0036980914883315568,
3Dc2c=0.36060017347335815 0.2289222925901413 0.9041913747787476,
3Dcoo=22.334463119506836 36.29957962036133 -19.963319778442383,
3Droo=150}{width=\marginparwidth}{figures/figshellparab_3D}
\\(b)\smallskip\\
\myincludeasythree{width=\marginparwidth,
3Droll=120.03842998357472,
3Dortho=0.0036980914883315568,
3Dc2c=0.36060017347335815 0.2289222925901413 0.9041913747787476,
3Dcoo=22.334463119506836 36.29957962036133 -19.963319778442383,
3Droo=150}{width=\marginparwidth}{figures/figshellparab_b_3D}
\\(c)}

\begin{example}[Finding volume using the Shell Method]\label{ex_shell4}
Find the volume of the solid formed by rotating the region bounded by $y=3x-x^2$ and $y=x$ about the $y$-axis.
\solution
The region, a sample shell, and the resulting solid are shown in \autoref{fig:shell4}. The radius of a sample shell is $r(x)=x$; the height of a sample shell is $h(x)=(3x-x^2)-x=2x-x^2$. The $x$ bounds on the region are $x=0$ to $x=2$ leading to the integral
\begin{align*}
V&=2\pi \int_0^2 x(2x-x^2) \dd x\\
&=2\pi \int_0^2 2x^2-x^3 \dd x\\
&=2\pi \left.\left[ \frac23 x^3-\frac14 x^4 \right] \right|_0^2\\
&=\frac{8\pi}{3} \text{units}^3.
\end{align*}

Note that in order to use the Washer Method, we would need to solve $y=3x-x^2$ for $x$, requiring us to complete the square. We must evaluate two integrals as we have two different sample slices. The volume can be computed as 
\begin{align*}
V&=\pi\int_0^2\left(y^2-\left(\frac32-\sqrt{\frac94 -y}\right)^2\right)\dd y \\
&\qquad+\pi\int_2^{9/4}\left(\left(\frac32+\sqrt{\frac94 -y}\right)^2
-\left(\frac32-\sqrt{\frac94 -y}\right)^2\right)\dd y
% this next expression is incorrect, if we want to go back to including it
%\\&=\pi\int_0^2\left(y-\frac32 +\sqrt{\frac94 -y}\right)^2 \dd y + 2\pi\int_2^{9/4} \left(2\sqrt{\frac94 -y}\right)^2\dd y
\end{align*}
While this integral is not impossible to solve, using the Shell Method gave us a significantly easier way to compute the volume.
\end{example}

We finish with an example where we can use either method to find the volume.

\begin{example}[Finding volume using both methods%the Washer Method and the Shell Method% otherwise, Method goes to the next line
]\label{ex_shell_wash}
Find the volume of the solid formed by rotating the region bounded by the curves $y=x^2+3$, $y=2x+1$, $x=0$, and $x=1$ about the y-axis using both the washer method and the shell method.
\solution
We'll start with the shell method, since that turns out to be easier.  The region, a sample shell, and the resulting solid are shown in \autoref{shell_wash_fig} parts a, b, and e respectively.  The volume is
\[
V=2\pi\int_0^1 x((x^2+3)-(2x+1))\dd x
=\frac{7\pi}6\text{units}^3.
\]
%\begin{align*}
%V&=2\pi\int_0^1 x((x^2+3)-(2x+1))\dd x\\
%&=2\pi\int_0^1 x^3-2x^2+2x\dd x\\
%%&=2\pi\left[\frac14-\frac23+1\right]\\
%&=\frac{7\pi}6\text{units}^3.
%\end{align*}

With the washer method, we need to integrate with respect to $y$ because we are rotating around a vertical axis.  We also need to divide the region in two because the washers will run into different boundaries at different heights.  We have indicated the region and the two different types of washers in \autoref{shell_wash_fig} parts c and d.  The volume is
\begin{align*}
V&=\pi\int_1^3\left(\frac{y-1}2\right)^2\dd y
+\pi\int_3^4\left(1^2-\left(\sqrt{y-3}\right)^2\right)\dd y\\
%&=\pi\int_1^3\frac{y^2}4-\frac y2+\frac14\dd y+\pi\int_3^4 4-y\dd y\\
%&=\pi\left[\frac{y^3}{12}-\frac{y^2}4+\frac y4\right]_1^3+\pi\left[4y-\frac{y^2}2\right]_3^4\\
%&=\pi\left[\frac{27}{12}-\frac94+\frac34\right]
%-\pi\left[\frac1{12}-\frac14+\frac14\right]
%+\pi\left[16-8\right]-\pi\left[12-\frac92\right]\\
%&=\pi\frac34-\pi\frac1{12}+8\pi-12\pi+\pi\frac92\\
&=\pi\frac23+\pi\frac12
%\\&
=\frac{7\pi}6\text{units}^3.
\end{align*}
We are reassured to find the same answer using either method.
\end{example}

\noindent\begin{minipage}{\linewidth}\noindent%
\captionsetup{type=figure}%
\flushinner{%
%\captionsetup{type=figure}
\begin{tabular}{ccc}
\begin{tikzpicture}
\begin{axis}[width=\marginparwidth,tick label style={font=\scriptsize},
			axis y line=middle,axis x line=middle,name=myplot,axis on top,
			xtick=\empty,extra x ticks={1,2},ytick={1,2,3,4},
			ymin=-.2,ymax=4.5,xmin=-.1,xmax=1.5]%[,axis equal]
\draw[\colorone,thick](axis cs:0,1)--(axis cs:1,3)
%node [color=black,left,pos=.5] {$\underbrace{\makebox[30pt]{}}_{r(x)}$}
--(axis cs:1,4);
\node at(axis cs:.25,2){$\underbrace{\makebox[30pt]{}}_{r(x)}$};
\addplot[draw={\colorone},smooth,thick,domain=0:1] {x^2+3};
\draw[thick,draw={\colortwo}] (axis cs: .5,2) -- (axis cs:.5,3.25)%;
node[color=black,right,pos=.5]{\scriptsize$ \left.\rule{0pt}{12pt}\right\} h(x)$};
\end{axis}
\node [right] at (myplot.right of origin) {\scriptsize $x$};
\node [above] at (myplot.above origin) {\scriptsize $y$};
\end{tikzpicture}
&
\myincludeasythree{width=\marginparwidth,
3Droll=115.86380470248459,
3Dortho=0.003982423339039087,
3Dc2c=0.35041913390159607 0.18568213284015656 0.9180024862289429,
3Dcoo=5.665955543518066 74.65992736816406 -21.359724044799805,
3Droo=150}{width=\marginparwidth}{figures/shellwash}
&
\multirow{12.6}{*}{% see https://tex.stackexchange.com/a/364935/107497
\myincludeasythree{width=\marginparwidth,
3Droll=124.28706719451111,
3Dortho=0.003981335088610649,
3Dc2c=0.3404673635959625 0.24756169319152832 0.9070805907249451,
3Dcoo=-5.416632652282715 90.77053833007812 -22.686864852905273,
3Droo=150}{width=\marginparwidth}{figures/shellwash_b}%
}
\smallskip\\(a)&(b)\smallskip\\
\begin{tikzpicture}
\begin{axis}[width=\marginparwidth,tick label style={font=\scriptsize},
			axis y line=middle,axis x line=middle,name=myplot,axis on top,
			xtick=\empty,extra x ticks={1,2},ytick={1,2,3,4},
			ymin=-.2,ymax=4.5,xmin=-.1,xmax=1.5]%[,axis equal]
\draw[\colorone,thick](axis cs:0,1)--(axis cs:1,3)
%node [color=black,left,pos=.5] {$\underbrace{\makebox[25pt]{}}_{r(x)}$}
--(axis cs:1,4);
\addplot[draw={\colorone},smooth,thick,domain=0:1] {x^2+3};
\draw[thick,draw={\colortwo}] (axis cs:0,3.64)--(axis cs: .8,3.64)
node[color=black,above,pos=.5]{\scriptsize$r(x)$};
\draw[thick,draw={\colortwo}] (axis cs: 0,2) -- (axis cs:.5,2)%;
node[color=black,above,pos=.5]{\scriptsize$R(x)$};
\end{axis}
\node [right] at (myplot.right of origin) {\scriptsize $x$};
\node [above] at (myplot.above origin) {\scriptsize $y$};
\end{tikzpicture}
&
\myincludeasythree{width=\marginparwidth,
3Droll=120.0383890024397,
3Dortho=0.004286274313926697,
3Dc2c=0.36060017347335815 0.2289222776889801 0.9041913747787476,
3Dcoo=9.441178321838379 68.38707733154297 -22.945234298706055,
3Droo=150}{width=\marginparwidth}{figures/shellwash_c}
\smallskip\\(c)&(d)&(e)
\end{tabular}%
}% closing flushinner
\captionof{figure}{Graphing the region in \autoref{ex_shell_wash}.}
\label{shell_wash_fig}
\end{minipage}%
\bigskip

%We end this section with a table summarizing the usage of the Washer and Shell Methods.
%
%\begin{keyidea}[Summary of the Washer and Shell Methods]\label{idea:shell_and_washer}
%Let a region $R$ be given with $x$-bounds $x=a$ and $x=b$ and $y$-bounds $y=c$ and $y=d$.\smallskip\\
%\begin{tabular}{cccc}
% 		& Washer Method & & Shell Method \\
% Horizontal Axis  & $\ds \pi\int_a^b \bigl(R(x)^2-r(x)^2\bigr)\dd x$ & & $\ds 2\pi\int_c^d r(y)h(y)\dd y$ \\ \\
% Vertical Axis &  $\ds\pi \int_c^d\bigl(R(y)^2-r(y)^2\bigr)\dd y$ & & $\ds 2\pi\int_a^b r(x)h(x)\dd x$
% \end{tabular}
%\index{integration!volume!Washer Method}\index{Washer Method}\index{integration!volume!Shell Method}\index{Shell Method}
%\end{keyidea}

As in the previous section, the real goal of this section is not to be able to compute volumes of certain solids. Rather, it is to be able to solve a problem by first approximating, then using limits to refine the approximation to give the exact value. In this section, we approximate the volume of a solid by cutting it into thin cylindrical shells. By summing up the volumes of each shell, we get an approximation of the volume. By taking a limit as the number of equally spaced shells goes to infinity, our summation can be evaluated as a definite integral, giving the exact value.

%We use this same principle again in the next section, where we find the length of curves in the plane.

\printexercises{exercises/07-03-exercises}
