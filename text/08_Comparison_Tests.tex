\section{Comparison Tests}\label{sec:comp_tests}

In this section we will be comparing a given series with series that we know either converge or diverge.

\theorem{thm:series_direct_compare}{Direct Comparison Test}
{Let $\{a_n\}$ and $\{b_n\}$ be positive sequences where $a_n\leq b_n$ for all $n\geq N$,
for some $N\geq 1$. 
\index{series!Direct Comparison Test}\index{Direct Comparison Test!for series}\index{convergence!Direct Comparison Test}\index{divergence!Direct Comparison Test}
\begin{enumerate}
	\item If $\ds \sum_{n=1}^\infty b_n$ converges, then $\ds \sum_{n=1}^\infty a_n$ converges.
	\item	If $\ds \sum_{n=1}^\infty a_n$ diverges, then $\ds \sum_{n=1}^\infty b_n$ diverges.
\end{enumerate}}

\mnote{\textbf{Note:} A sequence $\{a_n\}$ is a \textbf{positive sequence} if $a_n>0$ for all $n$.\\ 

Because of \autoref{thm:series_behavior}, any theorem that relies on a positive sequence still holds true when $a_n>0$ for all but a finite number of values of $n$.
\index{sequence!positive}}

\begin{proof}
First consider the partial sums of each series.
\[S_n=\sum_{i=1}^n a_i \quad \text{and} \quad T_n=\sum_{i=1}^n b_i\]
Since both series have positive terms we know that
\[S_n\leq S_n+a_{n+1}=\sum_{i=1}^n a_i+a_{n+1}=\sum_{i=1}^{n+1} a_i=S_{n+1}\]
and
\[T_n\leq T_n+b_{n+1}=\sum_{i=1}^n b_i+b_{n+1}=\sum_{i=1}^{n+1} b_i=T_{n+1}\]
Therefore, both of the sequences of partial sums,$\{S_n\}$ and $\{T_n\}$, are increasing.

For $n\ge N$, We're now going to split each series into two parts:
\begin{align*}
S&=\sum_{i=1}^{N-1}a_i & T&=\sum_{i=1}^{N-1}b_i \\
\bar S_n&=\sum_{i=N}^n a_i & \bar T_n&=\sum_{i=N}^n b_i.
\end{align*}
This means that $S_n=S+\bar S_n$ and $T_n=T+\bar T_n$.
Also, because $a_n\leq b_n$ for all $n\geq N$, we must have $\bar S_n\leq\bar T_n$.

For the first part of the theorem, assume that $\ds \sum_{n=1}^\infty b_n$ converges. Since $b_n\geq 0$ we know that 
\[T_n=\sum_{i=1}^n b_i\leq \sum_{i=1}^\infty b_i\]
From above we know that $\bar S_n\leq\bar T_n$ for all $n\geq N$ so we also have
\[S_n=S+\bar S_n\le S+\bar T_n=S+T_n-T=S-T+\sum_{i=1}^\infty b_i\]
Because $\ds \sum_{i=1}^\infty b_i$ converges it must have a finite value and $\{S_n\}$ is bounded above. We also showed that $\{S_n\}$ is increasing so by \autoref{thm:monotonic_converge} we know $\{S_n\}$ converges and so $\ds \sum_{n=1}^\infty a_n$ converges.

For the second part, assume that $\ds\sum_{n=1}^\infty a_n$ diverges. Because $a_n\geq 0$ we must have $\ds\lim_{n\to\infty} S_n=\infty$. We also know that for all $n\ge N$, $\bar S_n\leq\bar T_n$.  This means that
\begin{multline*}
\lim_{n\to \infty} T_n=\lim_{n\to \infty}(T+\bar T_n)\\
\ge T+\lim_{n\to \infty}\bar S_n=T+\lim_{n\to \infty}(S_n-S)
=T-S+\lim_{n\to \infty}S_n=\infty.
\end{multline*}
Therefore, $\{T_n\}$ is a divergent sequence and so $\ds \sum_{i=1}^\infty b_n$ diverges.
\end{proof}

\youtubeVideo{LAHKu3B-1zE}{Direct Comparison Test / Limit Comparison Test for Series --- Basic Info}

\example{ex_dct1}{Applying the Direct Comparison Test}{Determine the convergence of $\ds\sum_{n=1}^\infty \frac1{3^n+n^2}$.}
{This series is neither a geometric or $p$-series, but seems related. We predict it will converge, so we look for a series with larger terms that converges. (Note too that the Integral Test seems difficult to apply here.)

Since $3^n < 3^n+n^2$, $\ds \frac1{3^n}> \frac1{3^n+n^2}$ for all $n\geq1$. The series $\ds\sum_{n=1}^\infty \frac{1}{3^n}$ is a convergent geometric series; by \autoref{thm:series_direct_compare}, $\ds \sum_{n=1}^\infty \frac1{3^n+n^2}$ converges.}

\example{ex_dct2}{Applying the Direct Comparison Test}{Determine the convergence of $\ds\sum_{n=2}^\infty \frac{n^3}{n^4-1}$.}
{We know the Harmonic Series $\ds\sum_{n=1}^\infty \frac1n$ diverges, and it seems that the given series is closely related to it, hence we predict it will diverge. 

We have $\ds \frac{n^3}{n^4-1}>\frac{n^3}{n^4}=\frac{1}{n}$ for all $n\geq 1$. 

The Harmonic Series, $\ds\sum_{n=1}^\infty \frac{1}{n}$,  diverges, so we conclude that $\ds\sum_{n=1}^\infty \frac{n^3}{n^4-1}$ diverges as well.}

The concept of direct comparison is powerful and often relatively easy to apply. Practice helps one develop the necessary intuition to quickly pick a proper series with which to compare. However, it is easy to construct a series for which it is difficult to apply the Direct Comparison Test. 

Consider $\ds\sum_{n=1}^\infty \frac{n^3}{n^4+1}$. It is very similar to the divergent series given in \autoref{ex_dct2}. We suspect that it also diverges, as $\ds \frac 1n \approx \frac{n^3}{n^4+1}$ for large $n$. However, the inequality that we naturally want to use ``goes the wrong way'': since  $\ds \frac{n^3}{n^4+1}<\frac{n^3}{n^4}=\frac{1}{n}$ for all $n\geq 1$. The given series has terms \textit{less than} the terms of a divergent series, and we cannot conclude anything from this.

Fortunately, we can apply another test to the given series to determine its convergence.

\clearpage

\subsection{Limit Comparison Test}

\theorem{thm:series_limit_compare}{Limit Comparison Test}
{Let $\{a_n\}$ and $\{b_n\}$ be positive sequences.
\index{series!Limit Comparison Test}\index{Limit Comparison Test!for series}\index{convergence!Limit Comparison Test}\index{divergence!Limit Comparison Test}
\begin{enumerate}
	\item If $\ds\lim_{n\to\infty} \frac{a_n}{b_n} = L$, where $L$ is a positive real number, then $\ds \sum_{n=1}^\infty a_n$ and $\ds \sum_{n=1}^\infty b_n$ either both converge or both diverge.
	\item	If $\ds\lim_{n\to\infty} \frac{a_n}{b_n} = 0$, then if $\ds \sum_{n=1}^\infty b_n$ converges, then so does $\ds \sum_{n=1}^\infty a_n$.
	\item	If $\ds\lim_{n\to\infty} \frac{a_n}{b_n} = \infty$, then if $\ds \sum_{n=1}^\infty b_n$ diverges, then so does $\ds \sum_{n=1}^\infty a_n$.
\end{enumerate}}

\begin{proof}
\begin{enumerate}
\item We have $0<L<\infty$ so we can find two positive numbers, $m$ and $M$ such that $m<L<M$. Because $\ds L=\lim_{n\to\infty} \frac{a_n}{b_n}$ we know that for large enough $n$ the quotient $\frac{a_n}{b_n}$ must be close to $L$. So there must be a positive integer $N$ such that if $n>N$ we also have $\ds m<\frac{a_n}{b_n}<M$. Multiply by $b_n$ and we have $mb_n<a_n<Mb_n$ for $n>N$. If $\ds \sum_{n=1}^\infty b_n$ diverges, then so does $\ds \sum_{n=1}^\infty mb_n$. Also since $mb_n<a_n$ for sufficiently large $n$, by the Comparison Test $\ds \sum_{n=1}^\infty a_n$ also diverges. \\
Similarly, if $\ds \sum_{n=1}^\infty b_n$ converges, then so does $\ds \sum_{n=1}^\infty Mb_n$. Since $a_n<Mb_n$ for sufficiently large $n$, by the Comparison Test $\ds \sum_{n=1}^\infty a_n$ also converges.
\item Since $\ds \lim_{n\to\infty}\frac{a_n}{b_n}=0$, there is a number $N>0$ such that 
\begin{align*}
\abs{\frac{a_n}{b_n}-0}&<1\text{ for all }n>N \\
a_n&<b_n\text{ since }a_n\text{ and }b_n\text{ are positive}
\end{align*}
Now since $\ds \sum_{n=1}^\infty b_n$ converges, $\ds \sum_{n=1}^\infty a_n$ converges by the Comparison Test.
\item Since $\ds \lim_{n\to\infty}\frac{a_n}{b_n}=\infty$, there is a number $N>0$ such that 
\begin{align*}
\frac{a_n}{b_n}&>1\text{ for all }n>N \\
a_n&>b_n\text{ for all }n>N 
\end{align*}
Now since $\ds \sum_{n=1}^\infty b_n$ diverges, $\ds \sum_{n=1}^\infty a_n$ diverges by the Comparison Test.\qedhere
\end{enumerate}
\end{proof}


\autoref{thm:series_limit_compare} is most useful when the convergence of the series from $\{b_n\}$ is known and we are trying to determine the convergence of the series from $\{a_n\}$. 

We use the Limit Comparison Test in the next example to examine the series $\ds\sum_{n=1}^\infty \frac{n^3}{n^4+1}$ which motivated this new test.\\

\example{ex_lct1}{Applying the Limit Comparison Test}{Determine the convergence of $\ds\sum_{n=1}^\infty \frac{n^3}{n^4+1}$ using the Limit Comparison Test.}
{We compare the terms of $\ds\sum_{n=1}^\infty \frac{n^3}{n^4+1}$ to the terms of the Harmonic Sequence $\ds\sum_{n=1}^\infty \frac1{n}$:
\begin{align*}
	\lim_{n\to\infty}\frac{{n^3}/(n^4+1)}{1/n}
	&= \lim_{n\to\infty} \frac{n^4}{n^4+1}= \lim_{n\to\infty} \frac{1}{1+1/n^4} \\
	&= 1.
\end{align*}
Since the Harmonic Series diverges, we conclude that $\ds\sum_{n=1}^\infty \frac{n^3}{n^4+1}$ diverges as well.}


\example{ex_lct2}{Applying the Limit Comparison Test}{Determine the convergence of $\ds\sum_{n=1}^\infty \frac1{3^n-n^2}$}
{This series is similar to the one in \autoref{ex_dct1}, but now we are considering ``$3^n-n^2$'' instead of ``$3^n+n^2$.'' This difference makes applying the Direct Comparison Test difficult.

Instead, we use the Limit Comparison Test and compare with the series $\ds\sum_{n=1}^\infty \frac1{3^n}$:
\begin{align*}
	\lim_{n\to\infty}\frac{1/(3^n-n^2)}{1/3^n}
	&= \lim_{n\to\infty}\frac{3^n}{3^n-n^2} \\
	& \LHequals \lim_{n\to\infty} \frac{\ln 3 \cdot 3^n}{\ln 3 \cdot 3^n -2n}\\
	& \LHequals \lim_{n\to\infty} \frac {(\ln 3)^2 3^n}{(\ln 3)^2 3^n -2}\\
	& \LHequals \lim_{n\to\infty} \frac {(\ln 3)^3 3^n}{(\ln 3)^3 3^n}=1.
\end{align*}
We know $\ds\sum_{n=1}^\infty \frac1{3^n}$ is a convergent geometric series, hence $\ds\sum_{n=1}^\infty \frac1{3^n-n^2}$ converges as well.}

As mentioned before, practice helps one develop the intuition to quickly choose a series with which to compare. A general rule of thumb is to pick a series based on the dominant term in the expression of $\{a_n\}$. It is also helpful to note that factorials dominate increasing exponentials, which dominate algebraic functions (e.g., polynomials), which dominate logarithms. In the previous example, the dominant term of $\ds\frac{1}{3^n-n^2}$ was $3^n$, so we compared the series to $\ds \sum_{n=1}^\infty \frac1{3^n}$. It is hard to apply the Limit Comparison Test to series containing factorials, though, as we have not learned how to apply L'H\^opital's Rule to $n!$.

\example{ex_lct3}{Applying the Limit Comparison Test}{Determine the convergence of $\ds\sum_{n=1}^\infty \frac{\sqrt{n}+3}{n^2-n+1}$.}
{We na\"ively attempt to apply the rule of thumb given above and note that the dominant term in the expression of the series is $1/n^2$. Knowing that $\ds \sum_{n=1}^\infty \frac1{n^2}$ converges, we attempt to apply the Limit Comparison Test:
\begin{align*}
	\lim_{n\to\infty}\frac{(\sqrt{n}+3)/(n^2-n+1)}{1/n^2}
	&=\lim_{n\to\infty}\frac{n^2(\sqrt n+3)}{n^2-n+1}\\
	&= \infty \quad \text{(Apply L'H\^opital's Rule)}.
\end{align*}

\autoref{thm:series_limit_compare} part (3) only applies when $\ds\sum_{n=1}^\infty b_n$ diverges; in our case, it converges. Ultimately, our test has not revealed anything about the convergence of our series.

The problem is that we chose a poor series with which to compare. Since the numerator and denominator of the terms of the series are both algebraic functions, we should have compared our series  to the dominant term of the numerator divided by the dominant term of the denominator.

The dominant term of the numerator is $n^{1/2}$ and the dominant term of the denominator is $n^2$. Thus we should compare the terms of the given series to $n^{1/2}/n^2 = 1/n^{3/2}$:
\begin{align*}
\lim_{n\to\infty}\frac{(\sqrt{n}+3)/(n^2-n+1)}{1/n^{3/2}} &= \lim_{n\to \infty} \frac{n^{3/2}(\sqrt n+3)}{n^2-n+1} \\
		&= 1\quad \text{(Apply L'H\^opital's Rule)}.
\end{align*}
Since the  $p$-series $\ds\sum_{n=1}^\infty \frac1{n^{3/2}}$ converges, we conclude that $\ds\sum_{n=1}^\infty \frac{\sqrt{n}+3}{n^2-n+1}$ converges as well.}

The tests we have encountered so far have required that we analyze series from \emph{positive} sequences (the absolute value of the ratio and the root tests of \autoref{sec:ratio_root_tests} will convert the sequence into a positive sequence). The next section relaxes this restriction by  considering \emph{alternating series}, where the underlying sequence has terms that alternate between being positive and negative.

\printexercises{exercises/08_09_exercises}
