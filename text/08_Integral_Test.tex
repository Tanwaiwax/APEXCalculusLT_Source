\section{The Integral Test}\label{sec:int_tests}

Knowing whether or not a series converges is very important, especially when we discuss Power Series in \autoref{sec:power_series}. \autoref{thm:geom_series} gives criteria for when Geometric series converge and \autoref{thm:series_nth_term} gives a quick test to determine if a series diverges. There are many important series whose convergence cannot be determined by these theorems, though, so we introduce a set of tests that allow us to handle a broad range of series. We start with the Integral Test.


\subsection{Integral Test}

We stated in \autoref{sec:sequences} that a sequence $\{a_n\}$ is a function $a(n)$ whose domain is $\mathbb{N}$, the set of natural numbers. If we can extend $a(n)$ to have the domain of $\mathbb{R}$, the real numbers, and it is both positive and decreasing on $[1,\infty)$, then the convergence of $\ds \sum_{n=1}^\infty a_n$ is the same as $\ds\int_1^\infty a(x)\ dx$. 

\begin{theorem}[Integral Test]\label{thm:integral_test}
Let a sequence $\{a_n\}$ be defined by $a_n=a(n)$, where $a(n)$ is continuous, positive, and decreasing on $[1,\infty)$. Then $\ds \sum_{n=1}^\infty a_n$ converges, if, and only if, $\ds\int_1^\infty a(x)\ dx$ converges. In other words:
\begin{enumerate}
\item If $\ds \int_1^\infty a(x)\ dx$ is convergent, then $\ds \sum_{n=1}^\infty a_n$ is convergent.
\item If $\ds \int_1^\infty a(x)\ dx$ is divergent, then $\ds \sum_{n=1}^\infty a_n$ is divergent.
\end{enumerate}
\index{series!Integral Test}\index{Integral Test}\index{convergence!Integral Test}\index{divergence!Integral Test}
\end{theorem}

\mnote{\textbf{Note:} \autoref{thm:integral_test} does not state that the integral and the summation have the same value.}

Note that it is not necessary to start the series or the integral at $n=1$. We may use any interval $[n,\infty]$ on which $a(n)$ is continuous, positive and decreasing. Also the sequence $\{a_n\}$ does not have to be strictly decreasing. It must be \emph{ultimately decreasing} which means it is decreasing for all $n$ larger than some number $N$.\bigskip



We can demonstrate the truth of the Integral Test with two simple graphs. In \autoref{fig:integral_test}(a), the height of each rectangle is $a(n)=a_n$ for $n=1,2,\dotsc$, and clearly the rectangles enclose more area than the area under $y=a(x)$. Therefore we can conclude that
\begin{equation}
\int_1^\infty a(x)\ dx < \sum_{n=1}^\infty a_n.\label{eq:integral_testa}
\end{equation}
\mtable{Illustrating the truth of the Integral Test.}{fig:integral_test}{%
\begin{tikzpicture}
\begin{axis}[width=1.16\marginparwidth,tick label style={font=\scriptsize},
axis y line=middle,axis x line=middle,name=myplot,xtick={1,2,3,4,5},
ytick={1,2},ymin=-.1,ymax=2.5,xmin=-.5,xmax=5.6]
\addplot [very thick,draw={\colorone}, smooth,domain=-.5:5.6,samples=20] ({x},{2/(x+1)});
\draw [very thick,draw={\colortwo}]
	(axis cs: 1,0)--(axis cs:1,1)--(axis cs: 2,1)--(axis cs: 2,0)
	(axis cs: 2,0)--(axis cs:2,.667)--(axis cs: 3,.667)--(axis cs: 3,0)
	(axis cs: 3,0)--(axis cs:3,.5)--(axis cs: 4,.5)--(axis cs: 4,0)
	(axis cs: 4,0)--(axis cs:4,.4)--(axis cs: 5,.4)--(axis cs: 5,0);
\draw (axis cs: .95,1.75) node {\scriptsize $y=a(x)$};
\end{axis}
\node [right] at (myplot.right of origin) {\scriptsize $x$};
\node [above] at (myplot.above origin) {\scriptsize $y$};
\end{tikzpicture}
\\(a)\\
\begin{tikzpicture}
\begin{axis}[width=1.16\marginparwidth,tick label style={font=\scriptsize},
axis y line=middle,axis x line=middle,name=myplot,xtick={1,2,3,4,5},
ytick={1,2},ymin=-.1,ymax=2.5,xmin=-.5,xmax=5.6]
\addplot [very thick,draw={\colorone}, smooth,domain=-.5:5.6,samples=20] ({x},{2/(x+1)});
\draw [very thick,draw={\colortwo}] 
	(axis cs: 1,0) -- (axis cs:1,.667) -- (axis cs: 2,.667) -- (axis cs: 2,0)
	(axis cs: 2,0) -- (axis cs:2,.5) -- (axis cs: 3,.5) -- (axis cs: 3,0)
	(axis cs: 3,0) -- (axis cs:3,.4) -- (axis cs: 4,.4) -- (axis cs: 4,0)
	(axis cs: 4,0) -- (axis cs:4,.333) -- (axis cs: 5,.333) -- (axis cs: 5,0);
\draw (axis cs: .95,1.75) node {\scriptsize $y=a(x)$};
\end{axis}
\node [right] at (myplot.right of origin) {\scriptsize $x$};
\node [above] at (myplot.above origin) {\scriptsize $y$};
\end{tikzpicture}
\\(b)}
In \autoref{fig:integral_test}(b), we draw rectangles under $y=a(x)$ with the Right-Hand rule, starting with $n=2$. This time, the area of the rectangles is less than the area under $y=a(x)$, so $\ds\sum_{n=2}^\infty a_n < \int_1^\infty a(x)\ dx$. Note how this summation starts with $n=2$; adding $a_1$ to both sides lets us rewrite the summation starting with $n=1$:
\begin{equation}\sum_{n=1}^\infty a_n < a_1 +\int_1^\infty a(x)\ dx.\label{eq:integral_testb}
\end{equation} 


Combining Equations \eqref{eq:integral_testa} and \eqref{eq:integral_testb}, we have
\begin{equation}\sum_{n=1}^\infty a_n< a_1 +\int_1^\infty a(x)\ dx < a_1 + \sum_{n=1}^\infty a_n.\label{eq:integral_testc}\end{equation}
From \autoeqref{eq:integral_testc} we can make the following two statements:
\begin{enumerate}
	\item If $\ds \sum_{n=1}^\infty a_n$ diverges, so does $\ds\int_1^\infty a(x)\ dx$ \quad (because $\ds \sum_{n=1}^\infty a_n < a_1 +\int_1^\infty a(x)\ dx$)
	\item	If $\ds \sum_{n=1}^\infty a_n$ converges, so does $\ds\int_1^\infty a(x)\ dx$ \quad (because $\ds \ds \int_1^\infty a(x)\ dx < \sum_{n=1}^\infty a_n$.)
\end{enumerate}
Therefore the series and integral either both converge or both diverge. \autoref{thm:series_behavior} allows us to extend this theorem to series where $a(n)$ is positive and decreasing on $[b,\infty)$ for some $b>1$.

\youtubeVideo{ObiRjUFHJHo}{Integral Test for Series: Why It Works}

\begin{example}[Using the Integral Test]\label{ex_itest1}
Determine the convergence of $\ds\sum_{n=1}^\infty \frac{\ln n}{n^2}$.  (The terms of the sequence $\{a_n\} = \{(\ln n)/n^2\}$ and the n$^{\text{th}}$ partial sums are given in \autoref{fig:itest1}.)
\solution
\autoref{fig:itest1}
\mtable{Plotting the sequence and series in \autoref{ex_itest1}.}{fig:itest1}{\begin{tikzpicture}
\begin{axis}[width=1.16\marginparwidth,tick label style={font=\scriptsize},
axis y line=middle,axis x line=middle,name=myplot,axis on top,xtick={2,4,...,20},
ymin=-.05,ymax=.85,xmin=-1,xmax=20.9]
\addplot [only marks,draw={\colortwo},mark size={1.5pt}] coordinates{(1.,0)(2.,0.1733)(3.,0.2954)(4.,0.382)(5.,0.4464)(6.,0.4961)(7.,0.5359)(8.,0.5684)(9.,0.5955)(10.,0.6185)(11.,0.6383)(12.,0.6556)(13.,0.6708)(14.,0.6842)(15.,0.6963)(16.,0.7071)(17.,0.7169)(18.,0.7258)(19.,0.734)(20.,0.7415)};
\addplot [only marks,draw={\colorone},mark size={1.5pt},domain=1:10,samples=10]coordinates {(1.,0)(2.,0.1733)(3.,0.1221)(4.,0.08664)(5.,0.06438)(6.,0.04977)(7.,0.03971)(8.,0.03249)(9.,0.02713)(10.,0.02303)(11.,0.01982)(12.,0.01726)(13.,0.01518)(14.,0.01346)(15.,0.01204)(16.,0.01083)(17.,0.009804)(18.,0.008921)(19.,0.008156)(20.,0.007489)};
\end{axis}
\node [right] at (myplot.right of origin) {\scriptsize $n$};
\node [above] at (myplot.above origin) {\scriptsize $y$};
\node[shift={(0,-20pt)},draw] at (myplot.south) {
 \begin{tikzpicture} 
  \fill [\colorone] (0,0) circle (1.5pt) node [right,black] {\scriptsize $a_n$};
  \fill [\colortwo] (1,0) circle (1.5pt) node [right, black] {\scriptsize $S_n$};
 \end{tikzpicture}};
\end{tikzpicture}} 
implies that $a(n) = (\ln n)/n^2$ is positive and decreasing on $[2,\infty)$. We can determine this analytically, too. We know $a(n)$ is positive as both $\ln n$ and $n^2$ are positive on $[2,\infty)$. To determine that $a(n)$ is decreasing, consider $a'(n) = (1-2\ln n)/n^3$, which is negative for $n\geq 2$. Since $a'(n)$ is negative, $a(n)$ is decreasing.

Applying the Integral Test, we test the convergence of $\ds \int_1^\infty \frac{\ln x}{x^2}\ dx$. Integrating this improper integral requires the use of Integration by Parts, with $u = \ln x$ and $dv = 1/x^2\ dx$. 
\begin{align*}
	\int_1^\infty \frac{\ln x}{x^2}\ dx
	&= \lim_{t\to\infty} \int_1^t \frac{\ln x}{x^2}\ dx\\
	&= \lim_{t\to\infty}\biggl( -\frac1x\ln x\Big|_1^t + \int_1^t\frac1{x^2}\ dx \biggr)\\
	&= \lim_{t\to\infty} \biggl( \Bigl[-\frac1x\ln x -\frac 1x\Bigr]_1^t\biggr)\\
	&= \lim_{t\to\infty}\biggl(1-\frac1t-\frac{\ln t}{t}\biggr).\quad \text{Apply L'Hôpital's Rule:}\\
	&= 1-0- \lim_{t\to\infty} \frac1t\\
	&=1
\end{align*}
Since $\ds \int_1^\infty \frac{\ln x}{x^2}\ dx$ converges, so does $\ds \sum_{n=1}^\infty \frac{\ln n}{n^2}$.
\end{example}

\subsection{\texorpdfstring{$p$}{p}-Series}

Another important type of series is the \emph{p-series}.

\begin{definition}[$p$-Series, General $p$-Series]\label{def:pseries}
\index{series!p@$p$-series}\index{p@$p$-series}\mbox{}\\[-2\baselineskip]
\begin{enumerate}
\item	A \textbf{$p$-series} is a series of the form
\qquad$\ds\sum_{n=1}^\infty \frac{1}{n^p}$.
\item	A \textbf{general $p$-series} is a series of the form 
\qquad$\ds\sum_{n=1}^\infty \frac{1}{(an+b)^p}$,\medskip\\
where $a>0$, $b$ is a real number, and $an+b\neq 0$ for all $n$.
\end{enumerate}
\end{definition}

Like geometric series, one of the nice things about $p$-series is that they have easy to determine convergence properties.

\begin{theorem}[Convergence of General $p$-Series]\label{thm:pseries}
Assume $a$ and $b$ are real numbers and  $an+b\neq 0$ for all $n$.\\
A general $p$-series $\ds\sum_{n=1}^\infty \frac{1}{(an+b)^p}$ will converge if, and only if, $p>1$.
\index{series!p@$p$-series}\index{p@$p$-series}
\index{convergence!of p@of $p$-series}\index{divergence!of p@of $p$-series}
\end{theorem}

\begin{proof}
Consider the integral $\ds\int_1^\infty \frac1{(ax+b)^p}\ dx$; assuming $p\neq 1$,
\begin{align*}
	\int_1^\infty \frac1{(ax+b)^p}\ dx
	&= \lim_{t\to\infty} \int_1^t \frac1{(ax+b)^p}\ dx \\
	&= \lim_{t\to\infty} \frac{1}{a(1-p)}(ax+b)^{1-p}\Big|_1^t\\
	&= \lim_{t\to\infty} \frac{1}{a(1-p)}\bigl((at+b)^{1-p}-(a+b)^{1-p}\bigr).
\end{align*}
This limit converges if and only if, $p>1$. It is easy to show that the integral also diverges in the case of $p=1$. (This result is similar to the work preceding \autoref{idea:impint1}.)

Therefore $\ds \sum_{n=1}^\infty \frac 1{(an+b)^p}$ converges if, and only if, $p>1$.
\end{proof}

\begin{example}[Determining convergence of series]\label{ex_series6}
Determine the convergence of the following series.\\
\begin{multicols}{3}
\begin{enumerate}
\item		$\ds\sum_{n=1}^\infty \frac{1}{n}$
\item		$\ds\sum_{n=1}^\infty \frac{1}{n^2}$
\item		$\ds\sum_{n=1}^\infty \frac{1}{\sqrt{n}}$
\item		$\ds\sum_{n=1}^\infty \frac{(-1)^n}{n}$
\item		$\ds\sum_{n=11}^\infty \frac{1}{(\frac12n-5)^3}$
\item		$\ds\sum_{n=1}^\infty \frac{1}{2^n}$
\end{enumerate}
\end{multicols}
\solution
\begin{enumerate}
\item	This is a $p$-series with $p=1$. By \autoref{thm:pseries}, this series diverges.

This series is a famous series, called the \emph{Harmonic Series}, so named because of its relationship to \emph{harmonics} in the study of music and sound. 

\item	This is a $p$-series with $p=2$. By \autoref{thm:pseries}, it converges. Note that the theorem does not give a formula by which we can determine \emph{what} the series converges to; we just know it converges. A famous, unexpected result is that this series converges to $\ds{\pi^2}/{6}$.

\item	This is a $p$-series with $p=1/2$; the theorem states that it diverges.

\item	This is not a $p$-series; the definition does not allow for alternating signs. Therefore we cannot apply \autoref{thm:pseries}. We will consider this series again in \autoref{sec:alt_series}.
(Another famous result states that this series, the \emph{Alternating Harmonic Series}, converges to $-\ln 2$.)

\item	This is a general $p$-series with $p=3$, therefore it converges.

\item	This is not a $p$-series, but a geometric series with $r=1/2$. It converges.
\end{enumerate}
\end{example}

In the next section we consider two more convergence tests, both comparison tests. That is, we determine the convergence of one series by  comparing it to another series with known convergence. 

\printexercises{exercises/08_03_exercises}
