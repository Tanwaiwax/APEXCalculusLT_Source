
\section{Sequences}\label{sec:sequences}

We commonly refer to a set of events that occur one after the other as a \emph{sequence} of events. In mathematics, we use the word \emph{sequence} to refer to an ordered set of numbers, i.e., a set of numbers that ``occur one after the other.''

For instance, the numbers $2, 4, 6, 8, \dotsc$, form a sequence. The order is important; the first number is 2, the second is 4, etc. It seems natural to seek a formula that describes a given sequence, and often this can be done. For instance, the sequence above could be described by the function $a(n) = 2n$, for the values of $n = 1, 2, \dotsc$ (it could also be described by $n^4-10 n^3+35 n^2-48n+24$, to give one of infinitely many other options). To find the 10$^\text{th}$ term in the sequence, we would compute $a(10)$. This leads us to the following, formal definition of a sequence.

\begin{definition}[Sequence]\label{def:sequence}
%
\mnote{\textbf{Notation:} We use $\mathbb{N}$ to describe the set of natural numbers, that is, the integers $1, 2, 3, \dotsc$}
%
A \textbf{sequence} is a function $a(n)$ whose domain is $\mathbb{N}$. The \textbf{range} of a sequence is the set of all distinct values of $a(n)$.
\index{sequences!definition}\bigskip

The \textbf{terms} of a sequence are the values $a(1), a(2), \dotsc$, which are usually denoted with subscripts as $a_1, a_2, \dotsc$.\bigskip

A sequence $a(n)$ is often denoted as $\{a_n\}$.
\end{definition}

\youtubeVideo{9K1xx6wfN-U}{Sequences --- Examples showing convergence or divergence}

\mnote[-1.2in]{\textbf{Factorial:} The expression $3!$ refers to the number $3\cdot2\cdot1 = 6$.
\index{factorial}\bigskip\\
In general, $n! = n\cdot (n-1)\cdot(n-2)\dotsm 2\cdot1$, where $n$ is a natural number.\bigskip\\
We define $0! = 1$. While this does not immediately make sense, it makes many mathematical formulas work properly.}

\mtable[.7in]{Plotting sequences in \autoref{ex_seq1}.}{fig:seq1b}{%
\begin{tikzpicture}
\begin{axis}[width=\marginparwidth,tick label style={font=\scriptsize},
axis y line=middle,axis x line=middle,name=myplot,axis on top,xtick={1,2,3,4},
ytick={1,2,3,4,5},ymin=-.1,ymax=5.5,xmin=-.1,xmax=4.5]
\addplot [only marks,draw={\colorone},fill={\colorone},mark size={1.75pt}] coordinates {(1,3)(2,4.5)(3,4.5)(4,3.375)};
\draw (axis cs:2.5,1.5) node {\scriptsize $\ds a_n = \frac{3^n}{n!}$};
\end{axis}
\node [right] at (myplot.right of origin) {\scriptsize $n$};
\node [above] at (myplot.above origin) {\scriptsize $y$};
\end{tikzpicture}
\\(a)\\
\begin{tikzpicture}
\begin{axis}[width=\marginparwidth,tick label style={font=\scriptsize},
axis y line=middle,axis x line=middle,name=myplot,axis on top,xtick={1,2,3,4},
ytick={1,2,3,4,5},ymin=-.1,ymax=5.5,xmin=-.1,xmax=4.5]
\addplot [only marks,draw={\colorone},fill={\colorone},mark size={1.75pt},domain=1:4] {4+(-1)^x};
\draw (axis cs:2,1) node {\scriptsize $\ds a_n = 4+(-1)^n$};
\end{axis}
\node [right] at (myplot.right of origin) {\scriptsize $n$};
\node [above] at (myplot.above origin) {\scriptsize $y$};
\end{tikzpicture}
\\(b)\\
\begin{tikzpicture}
\begin{axis}[width=\marginparwidth,tick label style={font=\scriptsize},
axis y line=middle,axis x line=middle,name=myplot,axis on top,xtick={1,2,3,4,5},
ytick={-1},extra y ticks={.5,.25},extra y tick labels={$1/2$,$1/4$},
ymin=-1.1,ymax=0.6,xmin=-.1,xmax=5.5]
\addplot [only marks,draw={\colorone},fill={\colorone},mark size={1.75pt}] coordinates{(1,-1)(2,-.25)(3,.111)(4,0.0625)(5,-.04)};
\draw (axis cs:3,-.5) node {\scriptsize $\ds a_n = \frac{(-1)^{n(n+1)/2}}{n^2}$};
\end{axis}
\node [right] at (myplot.right of origin) {\scriptsize $n$};
\node [above] at (myplot.above origin) {\scriptsize $y$};
\end{tikzpicture}
\\(c)}

\begin{example}[Listing terms of a sequence]\label{ex_seq1}
List the first four terms of the following sequences.
\[
 \text{1. }\{a_n\} = \left\{\frac{3^n}{n!}\right\}\qquad
 \text{2. }\{a_n\} = \{4+(-1)^n\}\qquad
 \text{3. }\{a_n\} = \left\{\frac{(-1)^{n(n+1)/2}}{n^2}\right\}
\]
\solution
\begin{enumerate}
\item		$\ds a_1=\frac{3^1}{1!} = 3$;\qquad	$\ds a_2= \frac{3^2}{2!} = \frac92$;\qquad $\ds a_3 = \frac{3^3}{3!} = \frac92$; \qquad $\ds a_4 = \frac{3^4}{4!} = \frac{27}8$

We can plot the terms of a sequence with a scatter plot. The ``$x$''-axis is used for the values of $n$, and the values of the terms are plotted on the $y$-axis. To visualize this sequence, see \autoref{fig:seq1b}(a).

\item		$a_1= 4+(-1)^1 = 3$;\qquad $a_2 = 4+(-1)^2 = 5$; 

\noindent $a_3=4+(-1)^3 = 3$; \qquad $a_4 = 4+(-1)^4 = 5$.\\
Note that the range of this sequence is finite, consisting of only the values 3 and 5. This sequence is plotted in \autoref{fig:seq1b}(b).

\item		$\ds a_1= \frac{(-1)^{1(2)/2}}{1^2} = -1$; \qquad $\ds a_2 = \frac{(-1)^{2(3)/2}}{2^2} =-\frac14$

\noindent $\ds a_3 = \frac{(-1)^{3(4)/2}}{3^2} = \frac19$ \qquad $\ds a_4 = \frac{(-1)^{4(5)/2}}{4^2} = \frac1{16}$; 

\noindent $\ds a_5 = \frac{(-1)^{5(6)/2}}{5^2}=-\frac1{25}$.

\noindent We gave one extra term to begin to show the pattern of signs is ``$-$, $-$, $+$, $+$, $-$, $-$, \dots'', due to the fact that the exponent of $-1$ is a special quadratic. This sequence is plotted in \autoref{fig:seq1b}(c).
\end{enumerate}
\end{example}

\begin{example}[Determining a formula for a sequence]\label{ex_seq2}
Find the $n^\text{th}$ term of the following sequences, i.e., find a function that describes each of the given sequences.
\begin{multicols}{2}
\begin{enumerate}
\item		$2, 5, 8, 11, 14, \dotsc$
\item		$2, -5, 10, -17, 26, -37, \dotsc$
\item		$1, 1, 2, 6, 24, 120, 720, \dotsc$
\item		$\ds \frac52, \frac52, \frac{15}8, \frac54, \frac{25}{32}, \dotsc$
\end{enumerate}
\end{multicols}
\solution
We should first note that there is never exactly one function that describes a finite set of numbers as a sequence. There are many sequences that start with 2, then 5, as our first example does. We are looking for a simple formula that describes the terms given, knowing there is possibly more than one answer.
\begin{enumerate}
\item		Note how each term is 3 more than the previous one. This implies a linear function would be appropriate: $a(n) = a_n = 3n + b$ for some appropriate value of $b$. As we want $a_1=2$, we set $b=-1$. Thus $a_n = 3n-1$.

\item		First notice how the sign changes from term to term. This is most commonly accomplished by multiplying the terms by either $(-1)^n$ or $(-1)^{n+1}$. Using $(-1)^n$ multiplies the odd terms by $(-1)$; using $(-1)^{n+1}$ multiplies the even terms by $(-1)$. As this sequence has negative even terms, we will multiply by $(-1)^{n+1}$. 

After this, we might feel a bit stuck as to how to proceed. At this point, we are just looking for a pattern of some sort: what do the numbers 2, 5, 10, 17, etc., have in common? There are many correct answers, but the one that we'll use here is that each is one more than a perfect square. That is, $2=1^2+1$, $5=2^2+1$, $10=3^2+1$, etc. Thus our formula is $a_n= (-1)^{n+1}(n^2+1)$.

\item		One who is familiar with the factorial function will readily recognize these numbers. They are $0!$, $1!$, $2!$, $3!$, etc. Since our sequences start with $n=1$, we cannot write $a_n = n!$, for this misses the $0!$ term. Instead, we shift by 1, and write $a_n = (n-1)!$.

\item		This one may appear difficult, especially as the first two terms are the same, but a little ``sleuthing'' will help. Notice how the terms in the numerator are always multiples of 5, and the terms in the denominator are always powers of 2. Does something as simple as $a_n = \frac{5n}{2^n}$ work?

When $n=1$, we see that we indeed get $5/2$ as desired. When $n=2$, we get $10/4 = 5/2$. Further checking shows that this formula indeed matches the other terms of the sequence.
\end{enumerate}
\end{example}

A common mathematical endeavor is to create a new mathematical object (for instance, a sequence) and then apply previously known mathematics to the new object. We do so here. The fundamental concept of calculus is the limit, so we will investigate what it means to find the limit of a sequence.

%{\tcbset{grow to right by=8em}}
\begin{definition}[Limit of a Sequence, Convergent, Divergent]\label{def:seq_limit}
Let $\{a_n\}$ be a sequence and let $L$ be a real number. Given any $\epsilon>0$, if an $m$ can be found such that $\abs{a_n-L}<\epsilon$ for all $n>m$, then we say the \textbf{limit of $\{a_n\}$, as $n$ approaches infinity, is $L$}, denoted \[\lim_{n\to\infty}a_n = L.\]

If $\ds\lim_{n\to\infty} a_n$ exists, we say the sequence \textbf{converges}; otherwise, the sequence \textbf{diverges}.\index{limit!of sequence}\index{sequences!limit}\index{convergence!of sequence}\index{divergence!of sequence}\index{sequences!convergent}\index{sequences!divergent}
\end{definition}

This definition states, informally, that if the limit of a sequence is $L$, then if you go far enough out along the sequence, all subsequent terms will be \emph{really close} to $L$. Of course, the terms ``far enough'' and ``really close'' are subjective terms, but hopefully the intent is clear.

This definition is reminiscent of the $\epsilon$-$\delta$ proofs of \autoref{chapter:limits}. In that chapter we developed other tools to evaluate limits apart from the formal definition; we do so here as well.

%{\tcbset{grow to right by=3em}}
\begin{theorem}[Limit of a Sequence]\label{thm:seq_limit}
Let $\{a_n\}$ be a sequence and let $f(x)$ be a function whose domain contains the positive real numbers where $f(n) = a_n$ for all $n$ in $\mathbb{N}$.\bigskip

If $\ds \lim_{x\to\infty} f(x) = L$, then $\ds\lim_{n\to\infty} a_n = L$.
%\begin{enumerate}
%\item		If $\ds \lim_{x\to\infty} f(x) = L$, then $\ds\lim_{n\to\infty} a_n = L$.
%\item		If $\ds \lim_{x\to\infty} f(x)$ does not exist, then $\{a_n\}$ diverges.
%\end{enumerate}
\end{theorem}

\autoref{thm:seq_limit} allows us, in certain cases, to apply the tools developed in \autoref{chapter:limits} to limits of sequences. Note two things \emph{not} stated by the theorem:
	\begin{enumerate}
		\item If $\ds \lim_{x\to\infty}f(x)$ does not exist, we cannot conclude that $\ds\lim_{n\to\infty} a_n$ does not exist. It may, or may not, exist. For instance, we can define a sequence $\{a_n\} = \{\cos(2\pi n)\}$. Let $f(x) = \cos (2\pi x)$. Since the cosine function oscillates over the real numbers, the limit $\ds \lim_{x\to\infty}f(x)$ does not exist. 
		
		However, for every positive integer $n$, $\cos(2\pi n) = 1$, so $\ds \lim_{n\to\infty} a_n = 1$.
		
		%For every positive integer $n$, $\cos(2\pi n) = 1$, so $\ds \lim_{n\to\infty} a_n = 1$. 
		%
		%It is natural to set $f(x) = \cos (2\pi x)$. Since the cosine function oscillates over the real numbers, the limit $\ds \lim_{x\to\infty}f(x)$ does not exist.
		\item	If we cannot find a function $f(x)$ whose domain contains the positive real numbers where $f(n) = a_n$ for all $n$ in $\mathbb{N}$, we cannot conclude $\ds\lim_{n\to\infty} a_n$ does not exist. It may, or may not, exist.
	\end{enumerate}

%When we considered limits before, the domain of the function was an interval of real numbers. Now, as we consider limits, the domain is restricted to $\mathbb{N}$, the natural numbers. \autoref{thm:seq_limit} states that if we can extend a function whose domain is $\mathbb{N}$ to a

%\autoref{thm:seq_limit} states that this restriction of the domain does not affect the outcome of the limit and whatever tools we developed in \autoref{chapter:limits} to evaluate limits can be applied here as well.\\

%Considering again \autoref{ex_seq3}, we can now state when $\{a_n\} = \{\frac1n\}$, $\ds \lim_{n\to\infty} a_n = 0$.\\

\begin{example}[Determining convergence/divergence of a sequence]\label{ex_seq4}
Determine the convergence or divergence of the following sequences.
%
\mtable{Scatter plots of the sequences in \autoref{ex_seq4}.}{fig:seq4}{%
 \begin{tikzpicture}
  \begin{axis}[width=\marginparwidth,tick label style={font=\scriptsize},
    axis y line=middle,axis x line=middle,name=myplot,axis on top,%
    xtick={20,40,60,80,100},ymin=-11,ymax=11,xmin=-.1,xmax=110]
   \addplot [only marks,draw={\colorone},mark size={.75pt},domain=1:100,samples=21]
    {(3*x^2 - 2*x + 1)/(x^2 - 1000)};
   \draw (axis cs:60,-8) node {\scriptsize $a_n=\dfrac{3n^2-2n+1}{n^2-1000}$};
  \end{axis}
  \node [right] at (myplot.right of origin) {\scriptsize $n$};
  \node [above] at (myplot.above origin) {\scriptsize $y$};
 \end{tikzpicture}
\\(a)\\
\begin{tikzpicture}
\begin{axis}[width=\marginparwidth,tick label style={font=\scriptsize},
axis y line=middle,axis x line=middle,name=myplot,axis on top,
xtick={20,40,60,80,100},ymin=-1.1,ymax=1.1,xmin=-.1,xmax=110]
\addplot [only marks,draw={\colorone},mark size={.75pt},domain=1:100,samples=100]
 {cos(deg(x))};
\end{axis}
\node [right] at (myplot.right of origin) {\scriptsize $n$};
\node [above] at (myplot.above origin) {\scriptsize $y$};
\node [] at (myplot.north) {\scriptsize $\ds a_n = \cos n$};
\end{tikzpicture}
\\(b)\\
\begin{tikzpicture}
\begin{axis}[width=\marginparwidth,tick label style={font=\scriptsize},
axis y line=middle,axis x line=middle,name=myplot,axis on top,
ymin=-1.1,ymax=1.1,xmin=-.1,xmax=22]
\addplot [only marks,draw={\colorone},mark size={.75pt},domain=1:20,samples=20] {((-1)^x)/x};
\draw (axis cs:10,-.7) node {\scriptsize $\ds a_n = \frac{(-1)^n}{n}$};
\end{axis}
\node [right] at (myplot.right of origin) {\scriptsize $n$};
\node [above] at (myplot.above origin) {\scriptsize $y$};
\end{tikzpicture}
\\(c)}
%
\[
 \text{1. }\{a_n\} = \left\{\frac{3n^2-2n+1}{n^2-1000}\right\}\qquad
 \text{2. }\{a_n\} = \{\cos n \}\qquad
 \text{3. }\{a_n\} = \left\{\frac{(-1)^n}{n}\right\}
\]
\solution
\begin{enumerate}
\item		Using \autoref{thm:lim_rational_fn_at_infty}, we can state that $\ds\lim_{x\to\infty} \frac{3x^2-2x+1}{x^2-1000} = 3$. (We could have also directly applied L'Hôpital's Rule.) Thus the sequence $\{a_n\}$ converges, and its limit is 3. A scatter plot of every 5 values of $a_n$ is given in \autoref{fig:seq4} (a). The values of $a_n$ vary widely near $n=30$, ranging from about $-73$ to $125$, but as $n$ grows, the values approach 3.

\item		The limit $\ds\lim_{x\to\infty}\cos x$ does not exist, as $\cos x$ oscillates (and takes on every value in $[-1,1]$ infinitely many times). This means that we cannot apply \autoref{thm:seq_limit}. 

The fact that the cosine function oscillates strongly hints that $\cos n$, when $n$ is restricted to $\mathbb{N}$, will also oscillate. \autoref{fig:seq4} (b), where the sequence is plotted, shows that this is true. Because only discrete values of cosine are plotted, it does not bear strong resemblance to the familiar cosine wave.

Based on the graph, we suspect that $\ds \lim_{n\to\infty} a_n$ does not exist, but we have not decisively proven it yet.

%conclude that the sequence $\{\cos n\}$ diverges. (And in this particular case, since the domain is restricted to $\mathbb{N}$, no value of $\cos n$ is repeated!) This sequence is plotted in \autoref{fig:seq4} (b); because only discrete values of cosine are plotted, it does not bear strong resemblance to the familiar cosine wave.

\item		We cannot actually apply \autoref{thm:seq_limit} here, as the function $f(x) = (-1)^x/x$ is not well defined. (What does $(-1)^{\sqrt{2}}$ mean? In actuality, there is an answer, but it involves \emph{complex analysis}, beyond the scope of this text.) So for now we say that we cannot determine the limit. (But we will be able to very soon.) By looking at the plot in \autoref{fig:seq4} (c), we would like to conclude that the sequence converges to 0. That is true, but at this point we are unable to decisively say so.
\end{enumerate}
\end{example}

It seems that  %$\ds \left\{\frac{(-1)^n}{n}\right\}$ 
$\{(-1)^n/n\}$ converges to 0 but we lack the formal tool to prove it. The following theorem gives us that tool.

\begin{theorem}[Absolute Value Theorem]\label{thm:abs_val_seq}
Let $\{a_n\}$ be a sequence. If $\ds \lim_{n\to\infty}\abs{a_n}= 0$, then $\ds \lim_{n\to\infty} a_n = 0$\index{Absolute Value Theorem}\index{limit!Absolute Value Theorem}\index{sequence!Absolute Value Theorem}
\end{theorem}

% this proof is exercise 45, but copying the proof is not the worst learning outcome
\begin{proof}
We know $-\abs{a_n}\leq a_n\leq\abs{a_n}$ and $\ds \lim_{n\to \infty} (-\abs{a_n})=-\lim_{n\to\infty}\abs{a_n}=0$. Thus by the Squeeze Theorem $\ds \lim_{n\to\infty} a_n =0$.
\end{proof}

\begin{example}[Determining the convergence / divergence of a sequence]\label{ex_seq5}
Determine the convergence or divergence of the following sequences.
\[
 \text{1. }\{a_n\} = \left\{\frac{(-1)^n}{n}\right\}\qquad
 \text{2. }\{a_n\} = \left\{\frac{(-1)^n(n+1)}{n}\right\}
\]
\solution
\begin{enumerate}
\item		This appeared in \autoref{ex_seq4}. We want to apply \autoref{thm:abs_val_seq}, so consider the limit of $\{\abs{a_n}\}$:
\begin{align*}
\lim_{n\to\infty}\abs{a_n}&= \lim_{n\to\infty} \abs{\frac{(-1)^n}{n}} \\
					&= \lim_{n\to\infty} \frac1n \\
					&= 0.
\end{align*}
Since this limit is 0, we can apply \autoref{thm:abs_val_seq} and state that $\ds\lim_{n\to\infty} a_n=0$.

\item Because of the alternating nature of this sequence (i.e., every other term is multiplied by $-1$), we cannot simply look at the limit $\ds \lim_{x\to\infty} \frac{(-1)^x(x+1)}{x}$. We can try to apply the techniques of \autoref{thm:abs_val_seq}:
\begin{align*}
	\lim_{n\to\infty}\abs{a_n}
	&= \lim_{n\to\infty} \abs{\frac{(-1)^n(n+1)}{n}} \\
	&= \lim_{n\to\infty} \frac{n+1}{n}\\
	&= 1.
\end{align*}
%
\mtable{A plot of a sequence in \autoref{ex_seq5}, part 2.}{fig:seq5}{\begin{tikzpicture}
\begin{axis}[width=\marginparwidth,tick label style={font=\scriptsize},
axis y line=middle,axis x line=middle,name=myplot,axis on top,ytick={-1,-2,1,2},
ymin=-2.5,ymax=2.5,xmin=-.1,xmax=22]
\addplot [only marks,draw={\colorone},mark size={1.5pt},domain=1:20,samples=20]
 {((-1)^x*(x+1))/x};
\draw (axis cs:15,1.9) node {\scriptsize $\ds a_n = \frac{(-1)^n(n+1)}{n}$};
\end{axis}
\node [right] at (myplot.right of origin) {\scriptsize $n$};
\node [above] at (myplot.above origin) {\scriptsize $y$};
\end{tikzpicture}}%
%
We have concluded that when we ignore the sign, the sequence approaches 1. This means we cannot apply \autoref{thm:abs_val_seq}; it states the the limit must be 0 in order to conclude anything.

Since we know that the signs of the terms alternate \emph{and} we know that the limit of $\abs{a_n}$ is 1, we know that as $n$ approaches infinity, the terms will alternate between values close to 1 and $-1$, meaning the sequence diverges. A plot of this sequence is given in \autoref{fig:seq5}.
\end{enumerate}
\end{example}

We continue our study of the limits of sequences by considering some of the properties of these limits.

\begin{theorem}[Properties of the Limits of Sequences]\label{thm:seq_properties}
Let $\{a_n\}$ and $\{b_n\}$ be sequences such that $\ds \lim_{n\to\infty} a_n = L$ and $\ds \lim_{n\to\infty} b_n = K$, where $L$ and $K$ are real numbers, and let $c$ be a real number.\index{sequences!limit properties}
\begin{multicols}{2}
\begin{enumerate}
\item		$\ds \lim_{n\to\infty} (a_n\pm b_n) = L\pm K$
\item		$\ds \lim_{n\to\infty} (a_n\cdot b_n) = L\cdot K$
\item		$\ds \lim_{n\to\infty} (a_n/b_n) = L/K$, $K\neq 0$
\item		$\ds \lim_{n\to\infty} (c\cdot a_n) = c\cdot L$
\end{enumerate}
\end{multicols}
\end{theorem}

\begin{example}[Applying properties of limits of sequences]\label{ex_seq6}
Let the following limits be given:
\begin{itemize}
\item	 	$\ds \lim_{n\to\infty} a_n = 0$;
\item		$\ds \lim_{n\to\infty} b_n = e$; and
\item	  $\ds \lim_{n\to\infty} c_n = 5$.
\end{itemize}
Evaluate the following limits.
\[
 \text{1.}\quad\lim_{n\to\infty} (a_n+b_n)\qquad
 \text{2.}\quad\lim_{n\to\infty} (b_n\cdot c_n)\qquad
 \text{3.}\quad\lim_{n\to\infty} (1000\cdot a_n)
\]
\solution
We will use \autoref{thm:seq_properties} to answer each of these.
\begin{enumerate} 
\item		Since $\ds \lim_{n\to\infty} a_n = 0$ and $\ds \lim_{n\to\infty} b_n = e$, we conclude that $\ds \lim_{n\to\infty} (a_n+b_n) = 0+e = e.$ So even though we are adding something to each term of the sequence $b_n$, we are adding something so small that the final limit is the same as before.

\item		Since $\ds \lim_{n\to\infty} b_n = e$ and $\ds \lim_{n\to\infty} c_n = 5$, we conclude that $\ds \lim_{n\to\infty} (b_n\cdot c_n) = e\cdot 5 = 5e.$

\item		Since $\ds \lim_{n\to\infty} a_n = 0$, we have $\ds \lim_{n\to\infty} 1000a_n =1000\cdot 0 = 0$. It does not matter that we multiply each term by 1000; the sequence still approaches 0. (It just takes longer to get close to 0.)
\end{enumerate}
\end{example}

% todo write a transition paragraph to geometric sequences

\begin{definition}[Geometric Sequence]\label{def:geom_seq}
For a constant $r$, the sequence $\{r^n\}$ is known as a \textbf{geometric sequence}.\index{geometric sequence}
\end{definition}

\begin{theorem}[Convergence of Geometric Sequences]\label{thm:geom_seq}
The sequence $\{r^n\}$ is convergent if $-1<r\leq 1$ and divergent for all other values of $r$.  Furthermore,
\[
\lim_{n\to \infty} r^n=\begin{cases} 
0&  -1<r<1\\
1& r=1
\end{cases}\]
\end{theorem}

\begin{proof}
We can see from \autoref{ki_exp_func_props} and by letting $a=r$ that
\[
\lim_{n\to \infty} r^n =
\begin{cases}
\infty &  r>1\\
0 & 0<r<1.
\end{cases}
\]
We also know that $\ds  \lim_{x\to \infty} 1^n=1$ and $\ds  \lim_{x\to \infty} 0^n=0$. If $-1<r<0$, we know $0<\abs r<1$ so $\ds \lim_{x\to \infty}\abs{r^n}=\lim_{x\to \infty}\abs r^n=0$ and thus by \autoref{thm:abs_val_seq},$\ds \lim_{x\to \infty} r^n=0$. If $r\le-1$, $\ds \lim_{x\to \infty} r^n$ does not exist. Therefore, the sequence $\{ r^n\}$ is convergent if $-1<r\leq 1$ and divergent for all other values of $r$.
\end{proof}

% todo add a geometric sequence example

% todo use more than just color to distinguish a_n from S_n here throughout chapter

There is more to learn about sequences than just their limits. We will also study their range and the relationships terms have with the terms that follow. We start with some definitions describing properties of the range.

\begin{definition}[Bounded and Unbounded Sequences]\label{def:bounded}
A sequence $\{a_n\}$ is said to be \textbf{bounded} if there exist real numbers $m$ and $M$ such that $m < a_n < M$ for all $n$ in $\mathbb{N}$.\bigskip

A sequence $\{a_n\}$ is said to be \textbf{unbounded} if it is not bounded.\bigskip

A sequence $\{a_n\}$ is said to be \textbf{bounded above} if there exists an $M$ such that $a_n < M$ for all $n$ in $\mathbb{N}$; it is \textbf{bounded below} if there exists an $m$ such that $m<a_n$ for all $n$ in $\mathbb{N}$.
\index{sequences!boundedness}\index{bounded sequence}\index{unbounded sequence}
\end{definition}

It follows from this definition that an unbounded sequence may be bounded above or bounded below; a sequence that is both bounded above and below is simply a bounded sequence.

\begin{example}[Determining boundedness of sequences]\label{ex_seq3}
Determine the boundedness of the following sequences.
\[
 \text{1.}\quad\{a_n\} = \left\{\frac1n\right\}\qquad\qquad
 \text{2.}\quad\{a_n\} = \{2^n\}
\]
%
\mtable{A plot of $\{a_n\} = \{1/n\}$ and $\{a_n\} = \{2^n\}$ from \autoref{ex_seq3}.}{fig:seq3}{%
\begin{tikzpicture}
\begin{axis}[width=\marginparwidth,tick label style={font=\scriptsize},
axis y line=middle,axis x line=middle,name=myplot,axis on top,xtick={1,...,10},
ytick={1},extra y ticks={.5,.25,.1},extra y tick labels={$1/2$,$1/4$,$1/10$},
ymin=-.1,ymax=1.1,xmin=-.1,xmax=11]
\addplot [only marks,draw={\colorone},mark size={1.75pt},domain=1:10,samples=10]  {1/x};
\draw (axis cs:6,.75) node {\scriptsize $\ds a_n = \frac{1}{n}$};
\end{axis}
\node [right] at (myplot.right of origin) {\scriptsize $n$};
\node [above] at (myplot.above origin) {\scriptsize $y$};
\end{tikzpicture}
\\(a)\\[10pt]
\begin{tikzpicture}
\begin{axis}[width=\marginparwidth,tick label style={font=\scriptsize},
axis y line=middle,axis x line=middle,name=myplot,axis on top,
ymin=-.1,ymax=275,xmin=-.1,xmax=9]
\addplot [only marks,draw={\colorone},mark size={1.75pt},domain=1:8,samples=8]  {2^x};
\draw (axis cs:4,150) node {\scriptsize $\ds a_n = 2^{n}$};
\end{axis}
\node [right] at (myplot.right of origin) {\scriptsize $n$};
\node [above] at (myplot.above origin) {\scriptsize $y$};
\end{tikzpicture}
\\(b)}
\solution
\begin{enumerate}
\item	The terms of this sequence are always positive but are decreasing, so we have $0<a_n<2$ for all $n$. Thus this sequence is bounded. \autoref{fig:seq3}(a) illustrates this.

\item	The terms of this sequence obviously grow without bound. However, it is also true that these terms are all positive, meaning $0<a_n$. Thus we can say the sequence is unbounded, but also bounded below. \autoref{fig:seq3}(b) illustrates this.
\end{enumerate}
\end{example}

The previous example produces some interesting concepts. First, we can recognize that the sequence $\ds\left\{1/n\right\}$ converges to 0. This says, informally, that ``most'' of the terms of the sequence are ``really close'' to 0. This implies that the sequence is bounded, using the following logic. First, ``most'' terms are near 0, so we could find some sort of bound on these terms (using \autoref{def:seq_limit}, the bound is $\epsilon$). That leaves a ``few'' terms that are not near 0 (i.e., a \emph{finite} number of terms). A finite list of numbers is always bounded. 

This logic suggests that if a sequence converges, it must be bounded. This is indeed true, as stated by the following theorem.

\begin{theorem}[Convergent Sequences are Bounded]\label{thm:converge_bounded}
Let $\ds \left\{a_n\right\}$ be a convergent sequence. Then $\{a_n\}$ is bounded.
\index{bounded sequence!convergence}\index{convergence!of sequence}\index{sequences!convergent}
\end{theorem}

\mnote[\baselineskip]{\textbf{Note:} Keep in mind what \autoref{thm:converge_bounded} does \emph{not} say. It does not say that bounded sequences must converge, nor does it say that if a sequence does not converge, it is not bounded.}

In \autoref{ex_LHR4} part 1, we found that $\ds \lim_{x\to \infty} (1+1/x)^x=e$. If we consider the sequence $\ds \{b_n\}=\{(1+1/n)^n\}$, we see that $\ds \lim_{n\to \infty}b_n=e$. Even though it may be difficult to intuitively grasp the behavior of this sequence, we know immediately that it is bounded.

Another interesting concept to come out of \autoref{ex_seq3} again involves the sequence $\{1/n\}$. We stated, without proof, that the terms of the sequence were decreasing. That is, that $a_{n+1} < a_n$ for all $n$. (This is easy to show. Clearly $n < n+1$. Taking reciprocals flips the inequality: $1/n > 1/(n+1)$. This is the same as $a_n > a_{n+1}$.) Sequences that either steadily increase or decrease are important, so we give this property a name.

\begin{definition}[Monotonic Sequences]\label{def:monotonic}
\mbox{}\\[-2\baselineskip]\index{sequences!monotonic}\index{monotonic sequence}\begin{enumerate}
\item		A sequence $\{a_n\}$ is \textbf{monotonically increasing} if $a_n \leq a_{n+1}$ for all $n$, i.e.,
\[a_1 \leq a_2 \leq a_3 \leq \dotsb \leq a_n \leq a_{n+1} \dotsb\]
 \item	A sequence $\{a_n\}$ is \textbf{monotonically decreasing} if $a_n \geq a_{n+1}$ for all $n$, i.e.,
\[a_1 \geq a_2 \geq a_3 \geq \dotsb\geq a_n \geq a_{n+1} \dotsb\]
 \item	A sequence is \textbf{monotonic} if it is monotonically increasing or monotonically decreasing.
 \end{enumerate}
\end{definition}

\mnote{\textbf{Note:} It is sometimes useful to call a monotonically increasing sequence \emph{strictly increasing} if $a_n < a_{n+1}$ for all $n$; i.e, we remove the possibility that subsequent terms are equal.\bigskip\\
A similar statement holds for \emph{strictly decreasing.}}

\begin{example}[Determining monotonicity]\label{ex_seq7}
Determine the monotonicity of the following sequences.
\begin{multicols}{2}
\begin{enumerate}
\item $\ds \{a_n\} = \left\{\frac{n+1}n\right\}$
\item	$\ds \{a_n\} = \left\{\frac{n^2+1}{n+1}\right\}$	
\item $\ds \{a_n\} = \left\{\frac{n^2-9}{n^2-10n+26}\right\}$
\item	$\ds \{a_n\} = \left\{\frac{n^2}{n!}\right\}$	
\end{enumerate}
\end{multicols}
\solution
In each of the following, we will examine $a_{n+1}-a_n$. If $a_{n+1}-a_n \ge0$, we conclude that $a_n\le a_{n+1}$ and hence the sequence is increasing. If $a_{n+1}-a_n\le0$, we conclude that $a_n\ge a_{n+1}$ and the sequence is decreasing. Of course, a sequence need not be monotonic and perhaps neither of the above will apply.

We also give a scatter plot of each sequence. These are useful as they suggest a pattern of monotonicity, but analytic work should be done to confirm a graphical trend.

\begin{enumerate}
\item \hfill$\begin{aligned}[t]
a_{n+1}-a_n &= \frac{n+2}{n+1} - \frac{n+1}{n} \\		
	&= \frac{(n+2)(n)-(n+1)^2}{(n+1)n} \\
	&=	\frac{-1}{n(n+1)} \\
	&<0 \quad\text{ for all $n$.}
\end{aligned}$\hfill\null\\
%
\mtable[-7\baselineskip]{A plot of $\{a_n\}=\{\frac{n+1}n\}$ in \autoref{ex_seq7}(a).}{fig:seq7a}{\begin{tikzpicture}
\begin{axis}[width=1.16\marginparwidth,tick label style={font=\scriptsize},
axis y line=middle,axis x line=middle,name=myplot,axis on top,ytick={-1,-2,1,2},
ymin=-.1,ymax=2.5,xmin=-1,xmax=11]
\addplot [only marks,draw={\colorone},mark size={1.5pt},domain=1:10,samples=10] {(x+1)/x};
\draw (axis cs:6,1.9) node {\scriptsize $\ds a_n = \frac{n+1}{n}$};
\end{axis}
\node [right] at (myplot.right of origin) {\scriptsize $n$};
\node [above] at (myplot.above origin) {\scriptsize $y$};
\end{tikzpicture}}
%
Since $a_{n+1}-a_n<0$ for all $n$, we conclude that the sequence is decreasing.

\item
%
\mtable[5\baselineskip]{A plot of $\{a_n\}=\{\frac{n^2+1}{n+1}\}$ in \autoref{ex_seq7}(b).}{fig:seq7b}{\begin{tikzpicture}
\begin{axis}[width=1.16\marginparwidth,tick label style={font=\scriptsize},
axis y line=middle,axis x line=middle,name=myplot,axis on top,
ymin=-1,ymax=11,xmin=-1,xmax=11]
\addplot [only marks,draw={\colorone},mark size={1.5pt},domain=1:10,samples=10]
 {(x^2+1)/(x+1)};
\draw (axis cs:7.5,1.9) node {\scriptsize $\ds a_n = \frac{n^2+1}{n+1}$};
\end{axis}
\node [right] at (myplot.right of origin) {\scriptsize $n$};
\node [above] at (myplot.above origin) {\scriptsize $y$};
\end{tikzpicture}}
%
\hfill$\begin{aligned}[t]
	a_{n+1}-a_n &= \frac{(n+1)^2+1}{n+2} - \frac{n^2+1}{n+1} \\		
	&= \frac{\bigl((n+1)^2+1\bigr)(n+1)- (n^2+1)(n+2)}{(n+1)(n+2)}\\
	&=	\frac{n(n+3)}{(n+1)(n+2)} \\
	&> 0 \quad \text{ for all $n$.}
\end{aligned}$\hfill\null

Since $a_{n+1}-a_n>0$ for all $n$, we conclude the sequence is increasing.

\item%
%
\mtable[10\baselineskip]{A plot of $\{a_n\}=\{\frac{n^2-9}{n^2-10n+26}\}$ in \autoref{ex_seq7}(c).}{fig:seq7c}{\begin{tikzpicture}
\begin{axis}[width=\marginparwidth,tick label style={font=\scriptsize},
axis y line=middle,axis x line=middle,name=myplot,axis on top,
ymin=-1,ymax=16,xmin=-1,xmax=11]
\addplot [only marks,draw={\colorone},mark size={1.5pt},domain=1:10,samples=10]
 {(x^2-9)/(x^2-10*x+26)};
\draw (axis cs:3.4,11) node {\scriptsize $\ds a_n = \frac{n^2-9}{n^2-10n+26}$};
\end{axis}
\node [right] at (myplot.right of origin) {\scriptsize $n$};
\node [above] at (myplot.above origin) {\scriptsize $y$};
\end{tikzpicture}}%
%
	We can clearly see in \autoref{fig:seq7c}, where the sequence is plotted, that it is not monotonic. However, it does seem that after the first 4 terms it is decreasing. To understand why, perform the same analysis as done before:
\begin{align*}
	a_{n+1}-a_n &= \frac{(n+1)^2-9}{(n+1)^2-10(n+1)+26} - \frac{n^2-9}{n^2-10n+26} \\	
	&= \frac{n^2+2n-8}{n^2-8n+17}-\frac{n^2-9}{n^2-10n+26}\\
	&= \frac{(n^2+2n-8)(n^2-10n+26)-(n^2-9)(n^2-8n+17)}{(n^2-8n+17)(n^2-10n+26)}\\
	&= \frac{-10n^2+60n-55}{(n^2-8n+17)(n^2-10n+26)}.
\end{align*}

We want to know when this is greater than, or less than, 0. The denominator is always positive, therefore we are only concerned with the numerator. Using the quadratic formula, we can determine that $-10n^2+60n-55=0$ when $n\approx 1.13, 4.87$. So for $n<1.13$, the sequence is decreasing. Since we are only dealing with the natural numbers, this means that $a_1 > a_2$.

Between $1.13$ and $4.87$, i.e., for $n=2$, 3 and 4, we have that $a_{n+1}>a_n$ and the sequence is increasing. (That is, when $n=2$, 3 and 4, the numerator $-10n^2+60n-55$ from the fraction above is $>0$.)

When $n> 4.87$, i.e, for $n\geq 5$, we have that $-10n^2+60n-55<0$, hence $a_{n+1}-a_n<0$, so the sequence is decreasing.

In short, the sequence is simply not monotonic. However, it is useful to note that for $n\geq 5$, the sequence is monotonically decreasing. 

\item
%
\mtable{A plot of $\{a_n\}=\{n^2/n!\}$ in \autoref{ex_seq7}(d).}{fig:seq7d}{\begin{tikzpicture}
\begin{axis}[width=1.16\marginparwidth,tick label style={font=\scriptsize},
axis y line=middle,axis x line=middle,name=myplot,axis on top,
ymin=-.1,ymax=2.1,xmin=-1,xmax=11]
\addplot [only marks,draw={\colorone},mark size={1.5pt},domain=1:10,samples=10]
 {(x^2)/(factorial(x))};
\draw (axis cs:7,1.25) node {\scriptsize $\ds a_n = \frac{n^2}{n!}$};
\end{axis}
\node [right] at (myplot.right of origin) {\scriptsize $n$};
\node [above] at (myplot.above origin) {\scriptsize $y$};
\end{tikzpicture}}
%
	Again, the plot in \autoref{fig:seq7d} shows that the sequence is not monotonic, but it suggests that it is monotonically decreasing after the first term. Instead of looking at $a_{n+1}-a_n$, this time we'll look at $a_n/a_{n+1}$:
\begin{align*}
	\frac{a_n}{a_{n+1}} &= \frac{n^2}{n!}\frac{(n+1)!}{(n+1)^2} \\
	&= \frac{n^2}{n+1} \\
	&= n-1+\frac1{n+1}
\end{align*}
When $n=1$, the above expression is $<1$; for $n\geq 2$, the above expression is $>1$. Thus this sequence is not monotonic, but it is monotonically decreasing after the first term.
\end{enumerate}
\end{example}

Knowing that a sequence is monotonic can be useful. In particular, if we know that a sequence is bounded and monotonic, we can conclude it converges. Consider, for example, a sequence that is monotonically decreasing and is bounded below. We know the sequence is always getting smaller, but that there is a bound to how small it can become. This is enough to prove that the sequence will converge, as stated in the following theorem.

\begin{theorem}[Bounded Monotonic Sequences are Convergent]\label{thm:monotonic_converge}
Let $\{a_n\}$ be a bounded, monotonic sequence. Then $\{a_n\}$ converges; i.e., $\ds \lim_{n \to\infty}a_n$ exists.
\index{sequences!convergent}\index{convergence!of monotonic sequences}
\end{theorem}

Consider once again the sequence $\{a_n\} = \{1/n\}$. It is easy to show it is monotonically decreasing and that it is always positive (i.e., bounded below by 0). Therefore we can conclude by \autoref{thm:monotonic_converge} that the sequence converges. We already knew this by other means, but in the following section this theorem will become very useful.

Convergence of a sequence does not depend on the first $N$ terms of a sequence.  For example, we could adapt the sequence of the previous paragraph to be
\[
 1,\ 10,\ 100,\ 1000,\ \frac15,\ \frac16,\ \frac17,\ \frac18,\ \frac19,\ \frac1{10},
 \dotsc
\]
Because we only changed three of the first 4 terms, we have not affected whether the sequence converges or diverges.

Sequences are a great source of mathematical inquiry. The On-Line Encyclopedia of Integer Sequences (\url{http://oeis.org}) contains thousands of sequences and their formulae. (As of this writing, there are 257,537 sequences in the database.) Perusing this database quickly demonstrates that a single sequence can represent several different ``real life'' phenomena. 

Interesting as this is, our interest actually lies elsewhere. We are more interested in the \emph{sum} of a sequence. That is, given a sequence $\{a_n\}$, we are very interested in $a_1+a_2+a_3+\dotsb$. Of course, one might immediately counter with ``Doesn't this just add up to `infinity'?'' Many times, yes, but there are many important cases where the answer is no. This is the topic of \emph{series}, which we begin to investigate in the next section.

\printexercises{exercises/08-01-exercises}
