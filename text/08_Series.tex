\section{Infinite Series}\label{sec:series}

Given the sequence $\{a_n\} = \{1/2^n\} = 1/2,\ 1/4,\ 1/8,\ \ldots$, consider the following sums:

$$\begin{array}{ccccc}
a_1				&=& 1/2					&=& 1/2\\
a_1+a_2			&=& 1/2+1/4				&=& 3/4\\
a_1+a_2+a_3		&=& 1/2+1/4+1/8			&=& 7/8\\
a_1+a_2+a_3+a_4	&=& 1/2+1/4+1/8+1/16		&=& 15/16
\end{array}$$
Later, we will be able to show that
$$a_1+a_2+a_3+\dotsb+a_n = \frac{2^n-1}{2^n} = 1-\frac{1}{2^n}.$$
Let $S_n$ be the sum of the first $n$ terms of the sequence $\{1/2^n\}$. From the above, we see that $S_1=1/2$, $S_2 = 3/4$, and that $S_n = 1-1/2^n$. 

Now consider the following limit: $\ds \lim_{n\to\infty}S_n = \lim_{n\to\infty}\big(1-1/2^n\big) = 1$. This limit can be interpreted as saying something amazing: \emph{the sum of \emph{all} the terms of the sequence $\{1/2^n\}$ is 1.} 


This example illustrates some interesting concepts that we explore in this section. We begin this exploration with some definitions.

% make sure the next stays on the opening page

%\setboxwidth{50pt}
\definition{def:series}{Infinite Series, $n^\text{th}$ Partial Sums, Convergence, Divergence}
{Let $\{a_n\}$ be a sequence.
\index{series!definition}\index{series!partial sums}\index{series!convergent}\index{series!divergent}\index{convergence!of series}\index{divergence!of series}
\begin{enumerate}
\item		The sum $\ds \sum_{n=1}^\infty a_n$ is an \textbf{infinite series} (or, simply \textbf{series}).
\item		Let $\ds S_n = \sum_{i=1}^n a_i$\,; the sequence $\{S_n\}$ is the sequence of \textbf{$n^\text{th}$ partial sums} of $\{a_n\}$.
\item		If the sequence $\{S_n\}$ converges to $L$, we say the series $\ds \sum_{n=1}^\infty a_n$ \textbf{converges} to $L$, and we write $\ds \sum_{n=1}^\infty a_n = L$.
\item		If the sequence $\{S_n\}$ diverges, the series $\ds \sum_{n=1}^\infty a_n$ \textbf{diverges}.
\end{enumerate}}

Using our new terminology, we can state that the series $\ds \sum_{n=1}^\infty 1/2^n$ converges, and $\ds \sum_{n=1}^\infty 1/2^n = 1.$

\youtubeVideo{cyoiIBs7kIg}{Finding a Formula for a Partial Sum of a Telescoping Series}

We will explore a variety of series in this section. We start with two series that diverge, showing how we might discern divergence.

\example{ex_series1}{Showing series diverge}{\mbox{}\\[-2\baselineskip]
\begin{enumerate}
\item		Let $\{a_n\} = \{n^2\}$. Show $\ds \sum_{n=1}^\infty a_n$ diverges.
\item		Let $\{b_n\} = \{(-1)^{n+1}\}$. Show $\ds \sum_{n=1}^\infty b_n$ diverges.
\end{enumerate}}
{\begin{enumerate}
\item	Consider $S_n$, the $n^\text{th}$ partial sum.
\mfigure{0in}{Scatter plots relating to the series of \autoref{ex_series1} part 1.}{fig:series1a}{figures/figseries1a}
\begin{align*}
	S_n &= a_1+a_2+a_3+\dotsb+a_n \\		
	&= 1^2+2^2+3^2\dotsb+ n^2 \\
	&= \frac{n(n+1)(2n+1)}{6}. \qquad\text{by \autoref{thm:summation}}
\end{align*}
Since $\ds \lim_{n\to\infty}S_n = \infty$, we conclude that the series $\ds \sum_{n=1}^\infty n^2$ diverges. It is instructive to write $\ds \sum_{n=1}^\infty n^2=\infty$ for this tells us \emph{how} the series diverges: it grows without bound.

A scatter plot of the sequences $\{a_n\}$ and $\{S_n\}$ is given in \autoref{fig:series1a}. The terms of $\{a_n\}$ are growing, so the terms of the partial sums $\{S_n\}$ are growing even faster, illustrating that the series diverges.

\item	The sequence $\{b_n\}$ starts with 1, $-1$, 1, $-1$, $\ldots$. Consider some of the partial sums $S_n$ of $\{b_n\}$:
\begin{align*}
S_1 &= 1\\
S_2 &= 0\\
S_3 &= 1\\
S_4 &= 0
\end{align*}
\mfigure{0in}{Scatter plots relating to the series of \autoref{ex_series1} part 2.}{fig:series1b}{figures/figseries1b}
This pattern repeats; we find that
$S_n = \begin{cases}
1  & n\ \text{ is odd}\\
0  & n\ \text{ is even}
\end{cases}$.
As $\{S_n\}$ oscillates, repeating 1, 0, 1, 0, $\ldots$, we conclude that $\ds\lim_{n\to\infty}S_n$ does not exist, hence $\ds\sum_{n=1}^\infty (-1)^{n+1}$ diverges.		

A scatter plot of the sequence $\{b_n\}$ and the partial sums $\{S_n\}$ is given in \autoref{fig:series1b}. When $n$ is odd, $b_n = S_n$ so the marks for $b_n$ are drawn oversized to show they coincide.\eoehere
\end{enumerate}}

While it is important to recognize when a series diverges, we are generally more interested in the series that  converge. In this section we will demonstrate a few general techniques for determining convergence; later sections will delve deeper into this topic.

\subsection*{Geometric Series}

One important type of series is a \emph{geometric series}.

\definition{def:geom_series}{Geometric Series}
{A \textbf{geometric series} is a series of the form 
$$\sum_{n=0}^\infty ar^n = a+ar+ar^2+ar^3+\dotsb+ar^n+\dotsb$$
Note that the index starts at $n=0$, if the index starts at $n=1$ we have $\ds \sum_{n=1}^\infty ar^{n-1}$.%
\index{series!geometric}\index{geometric series}
}

We started this section with a geometric series, although we dropped the first term of $1$. One reason geometric series are important is that they have nice convergence properties.

% todo Tim the statement has $S_n=\sum_0^n$.  The proof has $S_n=\sum_0^{n-1}$
\theorem{thm:geom_series}{Convergence of Geometric Series}
{Consider the geometric series $\ds \sum_{n=0}^\infty ar^n$.
\index{series!geometric}\index{geometric series}\index{convergence!of geometric series}\index{divergence!of geometric series}
\begin{enumerate}
\item	If $r\neq1$, the $n^\text{th}$ partial sum is: $\ds S_n = \frac{a(1-r\,^{n+1})}{1-r}$.
\item	The series converges if, and only if, $\abs r<1$. When $\abs r<1$, 
\[\sum_{n=0}^\infty ar^n = \frac{a}{1-r}.\]
\end{enumerate}}

\begin{proof}
If $r=1$, then $S_n=a+a+a+\dotsb+a=na$. Since $\lim_{n\to \infty} S_n=\pm \infty$, the geometric series diverges.\\
If $r\neq 1$, we have
$$S_n=a+ar+ar^2+\dotsb+ar^{n-1}.$$
Multiply each term by $r$ and we have 
$$rS_n=ar+ar^2+ar^3\dotsb+ar^n.$$
Subtract these two equations and solve for $S_n$.
\begin{align*}
S_n-rS_n &=a-ar^n \\
S_n &=\frac{a(1-r^n)}{1-r}\\
\end{align*}
From \autoref{thm:geom_seq}, we know that if $-1<r<1$, then $\ds \lim_{n\to \infty} r^n=0$ so
$$\lim_{n\to \infty} S_n=\lim_{n\to \infty}=\frac{a(1-r^n)}{1-r}=\frac{a}{1-r}- \frac{a}{1-r}\lim_{n\to \infty}r^n=\frac{a}{1-r}.$$
So when $|r|<1$ the geometric series converges and its sum is $\ds \frac{a}{1-r}$.

If either $r\leq -1$ or $r>1$, the sequence $\{r^n\}$ is divergent by \autoref{thm:geom_seq}. Thus $\ds \lim_{n\to \infty} S_n$ does not exist, so the geometric series diverges if $r\leq -1$ or $r>1$.
\end{proof}

According to \autoref{thm:geom_series}, the series 
$$\ds\sum_{n=0}^\infty \frac{1}{2^n} =\sum_{n=0}^\infty \left(\frac 12\right)^2= 1+\frac12+\frac14+\dotsb$$ converges as $r=1/2$, and $\ds \sum_{n=0}^\infty \frac{1}{2^n} = \frac{1}{1-1/2} = 2.$ This concurs with our introductory example; while there we got a sum of 1, we skipped the first term of 1.\\

\mtable{Scatter plots relating to the series in \autoref{ex_series2}.}{fig:series2}{%
\myincludegraphics{figures/figseries2a}
\smallskip\\(a)\bigskip\\
\myincludegraphics{figures/figseries2b}
\smallskip\\(b)\bigskip\\
\myincludegraphics{figures/figseries2c}
\smallskip\\(c)}

\example{ex_series2}{Exploring geometric series}{Check the convergence of the following series. If the series converges, find its sum.
\[
 \text{1. }\sum_{n=2}^\infty \left(\frac34\right)^n\qquad
 \text{2. }\sum_{n=0}^\infty \left(\frac{-1}{2}\right)^n\qquad
 \text{3. }\sum_{n=0}^\infty 3^n
\]}
{\begin{enumerate}
\item		Since $r=3/4<1$, this series converges. By \autoref{thm:geom_series}, we have that
$$\sum_{n=0}^\infty \left(\frac34\right)^n = \frac{1}{1-3/4} = 4.$$ However, note the subscript of the summation in the given series: we are to start with $n=2$. Therefore we subtract off the first two terms, giving:
$$\sum_{n=2}^\infty \left(\frac34\right)^n = 4 - 1 - \frac34 = \frac94.$$
This is illustrated in \autoref{fig:series2}(a).

\item	Since $\abs r = 1/2 < 1$, this series converges, and by \autoref{thm:geom_series},
$$\sum_{n=0}^\infty \left(\frac{-1}{2}\right)^n = \frac{1}{1-(-1/2)} = \frac23.$$
The partial sums of this series are plotted in \autoref{fig:series2}(b). Note how the partial sums are not purely increasing as some of the terms of the sequence $\{(-1/2)^n\}$ are negative.

\item		Since $r>1$, the series diverges. (This makes ``common sense''; we expect the sum $$1+3+9+27 + 81+243+\dotsb$$ to diverge.) This is illustrated in \autoref{fig:series2}(c).\eoehere
\end{enumerate}}

Later sections will provide tests by which we can determine whether or not a given series converges. This, in general, is much easier than determining \emph{what} a given series converges to. There are many cases, though, where the sum can be determined.

\example{ex_series3}{Telescoping series}{Evaluate the sum $\ds \sum_{n=1}^\infty \left(\frac1n-\frac1{n+1}\right)$.
\index{series!telescoping}\index{telescoping series}}
{It will help to write down some of the first few partial sums of this series.
\begin{align*}
S_1 &=	\frac11-\frac12 & & = 1-\frac12\\
S_2 &=	\left(\frac11-\frac12\right) + \left(\frac12-\frac13\right) & & = 1-\frac13\\
S_3 &=	\left(\frac11-\frac12\right) + \left(\frac12-\frac13\right)+\left(\frac13-\frac14\right) & &= 1-\frac14\\
S_4 &=	\left(\frac11-\frac12\right) + \left(\frac12-\frac13\right)+\left(\frac13-\frac14\right) +\left(\frac14-\frac15\right)& &= 1-\frac15
\end{align*}
\mfigure{0in}{Scatter plots relating to the series of \autoref{ex_series3}.}{fig:series3}{figures/figseries3}
Note how most of the terms in each partial sum subtract out. In general, we see that $\ds S_n = 1-\frac{1}{n+1}$. The sequence $\{S_n\}$ converges,  as $\ds \lim_{n\to\infty}S_n = \lim_{n\to\infty}\left(1-\frac1{n+1}\right) = 1$, and so we conclude that $\ds \sum_{n=1}^\infty \left(\frac1n-\frac1{n+1}\right) = 1$. Partial sums of the series are plotted in \autoref{fig:series3}.}

The series in \autoref{ex_series3} is an example of a \sword{telescoping series}. Informally, a telescoping series is one in which the partial sums reduce to just a finite number of terms. The partial sum $S_n$ did not contain $n$ terms, but rather just two: 1 and $1/(n+1)$.\index{series!telescoping}\index{telescoping series}

When possible, seek a way to write an explicit formula for the $n^\text{th}$ partial sum $S_n$. This makes evaluating the limit $\ds\lim_{n\to\infty} S_n$ much more approachable. We do so in the next example.

%\noindent\textbf{Note on notation:} Most of the series we encounter will start with $n=1$. For ease of notation, we will often write $\sum a_n$ instead of writing $\ds\sum_{n=1}^\infty a_n$.\\


\example{ex_series4}{Evaluating series}{Evaluate each of the following infinite series.
\[
 \text{1.}\quad\sum_{n=1}^\infty \frac{2}{n^2+2n}\qquad\qquad
 \text{2.}\quad\sum_{n=1}^\infty \ln\left(\frac{n+1}{n}\right)
\]}
{\begin{enumerate}
\item		We can decompose the fraction $2/(n^2+2n)$ as
$$\frac2{n^2+2n} = \frac1n-\frac1{n+2}.$$
(See \autoref{sec:partial_fraction}, Partial Fraction Decomposition, to recall how  this is done, if necessary.)

Expressing the terms of $\{S_n\}$ is now more instructive:\\
\flushinnerequ{\footnotesize
\begin{align*}
S_1 &= 1-\frac13 &&= 1-\frac13\\
S_2 &= \left(1-\frac13\right) + \left(\frac12-\frac14\right) &&= 1+\frac12-\frac13-\frac14\\
S_3 &= \left(1-\frac13\right) + \left(\frac12-\frac14\right) + \left(\frac13-\frac15\right)
&&= 1+\frac12-\frac14-\frac15\\
S_4 &= \left(1-\frac13\right) + \left(\frac12-\frac14\right) + \left(\frac13-\frac15\right) + \left(\frac14-\frac16\right)
&&= 1+\frac12-\frac15-\frac16\\
S_5 &= \left(1-\frac13\right) + \left(\frac12-\frac14\right) + \left(\frac13-\frac15\right) + \left(\frac14-\frac16\right) + \left(\frac15-\frac17\right)
&&= 1+\frac12-\frac16-\frac17
\end{align*}}\\
\mfigure{0in}{Scatter plots relating to the series of \autoref{ex_series4} part 1.}{fig:series4a}{figures/figseries4a}
We again have a telescoping series. In each partial sum, most of the terms pair up to add to zero and we obtain the formula $\ds S_n = 1+\frac12-\frac1{n+1}-\frac1{n+2}.$ Taking limits allows us to determine the convergence of the series:\\
\flushinnerequ{%
$$\lim_{n\to\infty}S_n = \lim_{n\to\infty} \left(1+\frac12-\frac1{n+1}-\frac1{n+2}\right) = \frac32,\quad \text{so } \sum_{n=1}^\infty \frac1{n^2+2n} = \frac32.$$}
This is illustrated in \autoref{fig:series4a}.

\item	We begin by writing the first few partial sums of the series:
\begin{align*}
S_1 &= \ln\left(2\right) \\
S_2 &= \ln\left(2\right)+\ln\left(\frac32\right) \\
S_3 &= \ln\left(2\right)+\ln\left(\frac32\right)+\ln\left(\frac43\right) \\
S_4 &= \ln\left(2\right)+\ln\left(\frac32\right)+\ln\left(\frac43\right)
+\ln\left(\frac54\right) 
\end{align*}
At first, this does not seem helpful, but recall the logarithmic identity: $\ln x+\ln y = \ln (xy).$ Applying this to $S_4$ gives:
$$S_4 = \ln\left(2\right)+\ln\left(\frac32\right)+\ln\left(\frac43\right)
+\ln\left(\frac54\right)
=\ln\left(\frac21\cdot\frac32\cdot\frac43\cdot\frac54\right)
=\ln\left(5\right).$$
We must generalize this for $S_n$.\\
\flushinnerequ{%
$$S_n=\ln\left(2\right)+\ln\left(\frac32\right)+\cdots +\ln \left(\frac{n+1}{n}\right)=\ln\left(\frac21\cdot\frac32 \cdots  \frac{n}{n-1}\cdot \frac{n+1}{n}\right)=\ln ( n+1)$$}

%\clearpage

\mfigure{0in}{Scatter plots relating to the series of \autoref{ex_series4} part 2.}{fig:series4b}{figures/figseries4b}
We can conclude that $\{S_n\} = \big\{\ln (n+1)\big\}$. This sequence  does not converge, as $\ds \lim_{n\to\infty}S_n=\infty$. Therefore  $\ds\sum_{n=1}^\infty  \ln\left(\frac{n+1}{n}\right)=\infty$; the series diverges. Note in \autoref{fig:series4b} how the sequence of partial sums grows slowly; after 100 terms, it is not yet over 5. Graphically we may be fooled into thinking the series converges, but our analysis above shows that it does not.\eoehere
\end{enumerate}}

We are learning about a new mathematical object, the series. As done before, we apply ``old'' mathematics to this new topic.

\theorem{thm:series_prop}{Properties of Infinite Series}
{Suppose that \quad$\ds \sum_{n=1}^\infty a_n$\quad and \quad $\ds\sum_{n=1}^\infty b_n$ are convergent series, and that \quad$\ds \sum_{n=1}^\infty a_n = L$,\quad  $\ds\sum_{n=1}^\infty b_n = K$, and $c$ is a constant.
\index{series!properties}\index{Sum/Difference Rule!of series}
\index{Constant Multiple Rule!of series}
\begin{enumerate}
\item  Constant Multiple Rule: $\ds\sum_{n=1}^\infty c\cdot a_n = c\cdot\sum_{n=1}^\infty a_n = c\cdot L.$
\item		Sum/Difference Rule: $\ds\sum_{n=1}^\infty \big(a_n\pm b_n\big) = \sum_{n=1}^\infty a_n \pm \sum_{n=1}^\infty b_n = L \pm K.$
\end{enumerate}}

Before using this theorem, we will consider the harmonic series $\ds \sum_{n\to \infty} \frac{1}{n}$.

\example{ex_harm_series}{Divergence of the Harmonic Series}{Show that the harmonic series $\ds \sum_{n\to \infty} \frac{1}{n}$ diverges.}{We will use a proof by contradiction here. Suppose the harmonic series converges to $S$. That is $$S=1+\frac12+ \frac13 +\frac14+\frac15+\frac16+\frac17+\frac18+\cdots$$
We then have
\begin{align*}
S &\geq 1+\frac12+\frac14+\frac14+\frac16+\frac16+\frac18+\frac18+\cdots\\
&=1+\frac12+\frac12\phantom{+\frac14}+\frac13\phantom{+\frac16}
+\frac14\phantom{+\frac18}+\cdots\\
&=\frac12+S
\end{align*}
This gives us $S\geq \frac12+S$ which can never be true, thus our assumption that the harmonic series converges must be false. Therefore, the harmonic series diverges.}

%Before using this theorem, we provide a few ``famous'' series.
%
%\setboxwidth{20pt}
%\keyidea{idea:famous_series}{Important Series}
%{\begin{enumerate}
%\item	\parbox{90pt}{$\ds\sum_{n=0}^\infty \frac1{n!} = e$. } (Note that the index starts with $n=0$.)
%\item	$\ds\sum_{n=1}^\infty \frac1{n^2} = \frac{\pi^2}{6}$.
%\item	$\ds\sum_{n=1}^\infty \frac{(-1)^{n+1}}{n^2} = \frac{\pi^2}{12}$.
%\item	$\ds\sum_{n=0}^\infty \frac{(-1)^{n}}{2n+1} = \frac{\pi}{4}$.
%\item	\parbox{90pt}{$\ds\sum_{n=1}^\infty \frac{1}{n} $ \quad diverges.} (This is called the \emph{Harmonic Series}.)\index{Harmonic Series}
%\item	\parbox{90pt}{$\ds\sum_{n=1}^\infty \frac{(-1)^{n+1}}{n} = \ln 2$.} (This is called the \emph{Alternating Harmonic Series}.)\index{Alternating Harmonic Series}
%\end{enumerate}}
%
%\example{ex_series5}{Evaluating series}{Evaluate the given series.
%
%\[
%\text{1.}\quad\sum_{n=1}^\infty \frac{(-1)^{n+1}\big(n^2-n\big)}{n^3}\qquad
%\text{2.}\quad\sum_{n=1}^\infty \frac{1000}{n!}\qquad
%\text{3.}\quad\frac1{16}+\frac1{25}+\frac1{36}+\frac1{49}+\dotsb
%\]}
%{\begin{enumerate}
%\item	We start by using algebra to break the series apart:
%\begin{align*}
%\sum_{n=1}^\infty \frac{(-1)^{n+1}\big(n^2-n\big)}{n^3} &= \sum_{n=1}^\infty\left(\frac{(-1)^{n+1}n^2}{n^3}-\frac{(-1)^{n+1}n}{n^3}\right) \\
%						&= \sum_{n=1}^\infty\frac{(-1)^{n+1}}{n}-\sum_{n=1}^\infty\frac{(-1)^{n+1}}{n^2} \\
%						&= \ln(2) - \frac{\pi^2}{12}	\approx	-0.1293.
%\end{align*}
%
%This is illustrated in \autoref{fig:series5}(a).
%%\mfigure{0in}{Scatter plots relating to the series of \autoref{ex_series5} part 1.}{fig:series5a}{figures/figseries5a}
%
%\item		This looks very similar to the series that involves $e$ in \autoref{idea:famous_series}. Note, however, that the series given in this example starts with $n=1$ and not $n=0$. The first term of the series in the Key Idea is $1/0! = 1$, so we will subtract this from our result below:
%\begin{align*}
%		\sum_{n=1}^\infty \frac{1000}{n!} &= 1000\cdot\sum_{n=1}^\infty \frac{1}{n!} \\
%							&= 1000\cdot (e-1) \approx  1718.28.
%\end{align*}
%This is illustrated in \autoref{fig:series5}(b). The graph shows how this particular series converges very rapidly.
%%\mfigure{0in}{Scatter plots relating to the series of \autoref{ex_series5} part 2.}{fig:series5b}{figures/figseries5b}
%\mtable{Scatter plots relating to the series in \autoref{ex_series5}.}{fig:series5}{%
%\begin{tabular}{c}
%\myincludegraphics{figures/figseries5a}\\[10pt]
%(a)\\[15pt]
%\myincludegraphics{figures/figseries5b}\\[10pt]
%(b)
%\end{tabular}
%}
%
%\item		The denominators in each term are perfect squares; we are adding $\ds \sum_{n=4}^\infty \frac{1}{n^2}$ (note  we start with $n=4$, not $n=1$). This series will converge. Using the formula from \autoref{idea:famous_series}, we have the following:
%\begin{align*}
%\sum_{n=1}^\infty \frac1{n^2} &= \sum_{n=1}^3 \frac1{n^2} +\sum_{n=4}^\infty \frac1{n^2} \\
%\sum_{n=1}^\infty \frac1{n^2} - \sum_{n=1}^3 \frac1{n^2} &=\sum_{n=4}^\infty \frac1{n^2} \\
%\frac{\pi^2}{6} - \left(\frac11+\frac14+\frac19\right) &= \sum_{n=4}^\infty \frac1{n^2} \\
%\frac{\pi^2}{6} - \frac{49}{36} &= \sum_{n=4}^\infty \frac1{n^2} \\
%0.2838&\approx \sum_{n=4}^\infty \frac1{n^2} 
%\end{align*}
%\end{enumerate}}

It may take a while before one is comfortable with this statement, whose truth lies at the heart of the study of infinite series: \emph{it is possible that the sum of an infinite list of nonzero numbers is finite.} We have seen this repeatedly in this section, yet it still may ``take some getting used to.''

As one contemplates the behavior of series, a few facts become clear. 
\begin{enumerate}
\item		In order to add an infinite list of nonzero numbers and get a finite result, ``most'' of those numbers must be ``very near'' 0. 
\item		If a series diverges, it means that the sum of an infinite list of numbers is not finite (it may approach $\pm \infty$ or it may oscillate), and:
\begin{enumerate}
	\item	The series will still diverge if the first term is removed.
	\item	The series will still diverge if the first 10 terms are removed.
	\item	The series will still diverge if the first $1{,}000{,}000$ terms are removed.
	\item	The series will still diverge if any finite number of terms from anywhere in the series are removed.
\end{enumerate}
\end{enumerate}

These concepts are very important and lie at the heart of the next two theorems.

\theorem{thm:series_conv}{Convergence of Sequence}{If the series $\ds \sum_{n\to\infty}a_n$ converges, then $\lim_{n\to\infty}a_n=0$.}

\begin{proof}
Let $S_n=a_1+a_2+\cdots+a_n$. We have 
\begin{align*}
S_n&=a_1+a_2+\cdots+a_{n-1}+a_n\\
S_n&=S_{n-1}+a_n\\
a_n&=S_n-S_{n-1}
\end{align*}
Since  $\ds \sum_{n\to\infty}a_n$ converges, the sequence $\{ S_n\}$ converges.  Let $\ds \lim_{n\to\infty} S_n=S$. As $n\to \infty$, $n-1$ also goes to $\infty$, so $\ds \lim_{n\to\infty} S_{n-1}=S$. We now have
\begin{align*}
\lim_{n\to\infty} a_n&= \lim_{n\to\infty}(S_n-S_{n-1})\\
&= \lim_{n\to\infty}S_n - \lim_{n\to\infty} S_{n-1}\\
&=S-S=0\qedhere
\end{align*}
\end{proof}

\theorem{thm:series_nth_term}{Test for Divergence}{If $ \ds \lim_{n\to\infty} a_n$ does not exist or $\ds  \lim_{n\to\infty}a_n\neq0$, then the series $\ds \sum_{n=1}^\infty a_n$ diverges.}

The Test for Divergence follows from \autoref{thm:series_conv}. If the series does not diverge, it must converge and therefore $\ds  \lim_{n\to\infty}a_n=0$.

Note that the two statements in Theorems \ref{thm:series_conv} and \ref{thm:series_nth_term} are really the same. In order to converge, the limit of the terms of the sequence must approach 0; if they do not, the series will not converge. 

Looking back, we can apply this theorem to the series in \autoref{ex_series1}. In that example, we had $\{a_n\} = \{n^2\}$ and $\{b_n\} = \{(-1)^{n+1}\}$.
$$\lim_{n\to\infty} a_n=\lim_{n\to\infty} n^2=\infty$$
and
$$\lim_{n\to\infty} b_n=\lim_{n\to\infty}(-1)^{n+1}\text{ which does not exist.}$$
Thus by the Test for Divergence, both series will diverge.

\textbf{Important!} This theorem \emph{does not state} that if $\ds \lim_{n\to\infty} a_n = 0$ then $\ds \sum_{n=1}^\infty  a_n $ converges. The standard example of this is the Harmonic Series, as given in \autoref{ex_harm_series}. The Harmonic Sequence, $\{1/n\}$, converges to 0; the Harmonic Series, $\ds \sum_{n=1}^\infty 1/n$, diverges.

\theorem{thm:series_behavior}{Infinite Nature of Series}
{The convergence or divergence remains unchanged by the insertion or deletion of any finite number of terms. That is:
\begin{enumerate}
	\item	A divergent series will remain divergent with the insertion or deletion of any finite number of terms.
	\item	A convergent series will remain convergent with the insertion or deletion of any finite number of terms. (Of course, the \emph{sum} will likely change.)
\end{enumerate}}

In other words, when we are only interested in the convergence or divergence of a series, it is safe to ignore the first few billion terms.

\example*{ex_trunc_harm}{Removing Terms from the Harmonic Series}{Consider once more the Harmonic Series $\ds\sum_{n=1}^\infty\frac1n$\vspace{-.8\baselineskip} which diverges; that is, the partial sums $S_N=\ds\sum_{n=1}^N\frac1n$ grow (very, very slowly) without bound. One might think that by removing the ``large'' terms of the sequence that perhaps the series will converge. This is simply not the case. For instance, the sum of the first 10 million terms of the Harmonic Series is about 16.7. Removing the first 10 million terms from the Harmonic Series changes the partial sums,  effectively subtracting 16.7 from the sum. However, a sequence that is growing without bound will still grow without bound when 16.7 is subtracted from it. 

The equation below illustrates this. Even though we have subtracted off the first 10 million terms, this only subtracts a constant off of an expression that is still growing to infinity. Therefore, the modified series is still growing to infinity.
\begin{multline*}
 \sum_{n=10,000,001}^\infty\frac1n
 =\lim_{N\to\infty}\sum_{n=10,000,001}^N\frac1n
 =\lim_{N\to\infty}\sum_{n=1}^N\frac1n-\sum_{n=1}^{10,000,001}\frac1n \\
 =\lim_{N\to\infty}\sum_{n=1}^N\frac1n-16.7
 =\infty.\eoehere
\end{multline*}}

This section introduced us to series and defined a few special types of series whose convergence properties are well known. We know when a geometric series converges or diverges. Most series that we encounter are not one of these types, but we are still interested in knowing whether or not they converge. The next three sections introduce tests that help us determine whether or not a given series converges. 

\printexercises{exercises/08_02_exercises}
