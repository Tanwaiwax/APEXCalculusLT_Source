\section{Taylor Polynomials}\label{sec:taylor_poly}

Consider a function $y=f(x)$ and a point $\bigl(c,f(c)\bigr)$. The derivative, $\fp(c)$, gives the instantaneous rate of change of $f$ at $x=c$. Of all lines that pass through the point $\bigl(c,f(c)\bigr)$, the line that best approximates $f$ at this point is the tangent line; that is, the line whose slope (rate of change) is $\fp(c)$.

\mtable[-\baselineskip]{Plotting $y=f(x)$ and a table of derivatives of $f$ evaluated at 0.}{fig:taypolyintroa}{\pdftooltip{\begin{tikzpicture}
\begin{axis}[width=1.16\marginparwidth,tick label style={font=\scriptsize},
axis y line=middle,axis x line=middle,name=myplot,axis on top,
ymin=-5.5,ymax=8.5,xmin=-4.2,xmax=4.4]
\addplot [draw={\colorone},domain=-4:4,smooth,thick,samples=50] {exp(x)*sin(deg(x))*cos(deg(x))+2};
\addplot [draw={\colortwo},domain=-4:4,thick] {x+2};
\draw (axis cs:-3.,3) node {\scriptsize $y=f(x)$};
\draw (axis cs:-2.75,-2.2) node {\scriptsize $y=p_1(x)$};
\end{axis}
\node [right] at (myplot.right of origin) {\scriptsize $x$};
\node [above] at (myplot.above origin) {\scriptsize $y$};
\end{tikzpicture}}{ALT-TEXT-TO-BE-DETERMINED}
\begin{align*}
f(0)&=2 & \fp''(0)&=-1\\
\fp(0)&=1 & f\,^{(4)}(0)&=-12 \\
\fpp(0)&=2 & f\,^{(5)}(0)&=-19\vspace{-.5\baselineskip}
\end{align*}}

In \autoref{fig:taypolyintroa}, we see a function $y=f(x)$ graphed. The table below the graph shows that $f(0)=2$ and $\fp(0) = 1$; therefore, the tangent line to $f$ at $x=0$ is $p_1(x) = 1(x-0)+2 = x+2$. The tangent line is also given in the figure. Note that ``near'' $x=0$, $p_1(x) \approx f(x)$; that is, the tangent line approximates $f$ well.

One shortcoming of this approximation is that the tangent line only matches the slope of $f$; it does not, for instance, match the concavity of $f$. We can find a polynomial, $p_2(x)$, that does match the concavity without much difficulty, though. The table in \autoref{fig:taypolyintroa} gives the following information:
\[f(0) = 2 \qquad \fp(0) = 1\qquad \fp'(0) = 2.\]
Therefore, we want our polynomial $p_2(x)$ to have these same properties. That is, we need
\[p_2(0) = 2 \qquad p_2'(0) = 1 \qquad p_2''(0) = 2.\]

This is simply an initial-value problem. We can solve this using the techniques first described in \autoref{sec:antider}. To keep $p_2(x)$ as simple as possible, we'll assume that not only  $p_2''(0)=2$, but that $p_2''(x)=2$. That is, the second derivative of $p_2$ is  constant.

\mtable{Plotting $f$, $p_2$, and $p_4$.}{fig:taypolyintrob}{\pdftooltip{\begin{tikzpicture}
\begin{axis}[width=1.16\marginparwidth,tick label style={font=\scriptsize},
axis y line=middle,axis x line=middle,name=myplot,axis on top,
ymin=-5.5,ymax=8.5,xmin=-4.2,xmax=4.4]
\addplot [draw={\colorone},domain=-4:4,smooth,thick,samples=50]
 {exp(x)*sin(deg(x))*cos(deg(x))+2};
\addplot [draw={\colortwo},domain=-4:4,thick,smooth] {x^2+x+2};
\addplot [draw={\colortwo!40},domain=-4:4,thick,smooth] {-x^4/2-x^3/6+x^2+x+2};
\draw (axis cs:-3.,3.5) node {\scriptsize $y=p_2(x)$};
\draw (axis cs:-3.1,-2.5) node {\scriptsize $y=p_4(x)$};
\end{axis}
\node [right] at (myplot.right of origin) {\scriptsize $x$};
\node [above] at (myplot.above origin) {\scriptsize $y$};
\end{tikzpicture}}{ALT-TEXT-TO-BE-DETERMINED}}

If $p_2''(x) = 2$, then $p_2'(x) = 2x+C$ for some constant $C$. Since we have determined that $p_2'(0) = 1$, we find that $C=1$ and so $p_2'(x) = 2x+1$. Finally, we can compute $p_2(x) = x^2+x+C$. Using our initial values, we know $p_2(0) = 2$ so $C=2.$ We conclude that $p_2(x) = x^2+x+2.$ This function is plotted with $f$ in \autoref{fig:taypolyintrob}.

We can repeat this approximation process by creating polynomials of higher degree that match more of the derivatives of $f$ at $x=0$. In general, a polynomial of degree $n$ can be created to match the first $n$ derivatives of $f$. \autoref{fig:taypolyintrob} also shows $p_4(x)= -x^4/2-x^3/6+x^2+x+2$, whose first four derivatives at 0 match those of $f$. (Using the table in \autoref{fig:taypolyintroa}, start with $p_4^{(4)}(x)=-12$ and solve the related initial-value problem.)

\mtable[2\baselineskip]{Plotting $f$ and $p_{13}$.}{fig:taypolyintroc}{\pdftooltip{\begin{tikzpicture}
\begin{axis}[width=1.16\marginparwidth,tick label style={font=\scriptsize},
axis y line=middle,axis x line=middle,name=myplot,axis on top,
ymin=-5.5,ymax=8.5,xmin=-4.2,xmax=4.4]
\addplot [draw={\colorone},domain=-4:4,smooth,thick,samples=50] {exp(x)*sin(deg(x))*cos(deg(x))+2};
\addplot [draw={\colortwo},domain=-4:4,thick,smooth] coordinates {(-3.52,-6.364)(-3.36,-2.334)(-3.2,-0.1706)(-3.04,0.9569)(-2.88,1.528)(-2.72,1.809)(-2.56,1.944)(-2.4,2.009)(-2.24,2.038)(-2.08,2.048)(-1.92,2.046)(-1.76,2.031)(-1.6,2.006)(-1.44,1.969)(-1.28,1.924)(-1.12,1.872)(-0.96,1.82)(-0.8,1.775)(-0.64,1.747)(-0.48,1.747)(-0.32,1.783)(-0.16,1.866)(0,2.)(0.16,2.185)(0.32,2.411)(0.48,2.662)(0.64,2.908)(0.8,3.112)(0.96,3.227)(1.12,3.202)(1.28,2.988)(1.44,2.546)(1.6,1.855)(1.76,0.9264)(1.92,-0.1926)(2.08,-1.406)(2.24,-2.565)(2.4,-3.474)(2.56,-3.891)(2.72,-3.542)(2.88,-2.133)(3.04,0.6461)(3.2,5.139)(3.36,11.79)};
% the degree 13 polynomial
\draw (axis cs:-2,-2.75) node {\scriptsize $y=p_{13}(x)$};
\end{axis}
\node [right] at (myplot.right of origin) {\scriptsize $x$};
\node [above] at (myplot.above origin) {\scriptsize $y$};
\end{tikzpicture}}{ALT-TEXT-TO-BE-DETERMINED}}

As we use more and more derivatives, our polynomial approximation to $f$ gets better and better. In this example, the interval on which the approximation is ``good'' gets bigger and bigger. \autoref{fig:taypolyintroc} shows $p_{13}(x)$; we can visually affirm that this polynomial approximates $f$ very well on $[-2,3]$. The polynomial $p_{13}(x)$ is fairly complicated:
\[\scriptstyle
\frac{16901x^{13}}{6227020800}+\frac{13x^{12}}{1209600}-\frac{1321x^{11}}{39916800}-\frac{779x^{10}}{1814400}-\frac{359x^9}{362880}+\frac{x^8}{240}+\frac{139x^7}{5040}+\frac{11 x^6}{360}-\frac{19x^5}{120}-\frac{x^4}{2}-\frac{x^3}{6}+x^2+x+2.
\]

The polynomials we have created are examples of \emph{Taylor polynomials}, named after the British mathematician Brook Taylor who made important discoveries about such functions. While we created the above Taylor polynomials by solving initial-value problems, it can be shown that Taylor polynomials follow a general pattern that makes their formation much more direct. This is described in the following definition.

{
\tcbset{grow to right by=10em}
\begin{definition}[Taylor Polynomial, Maclaurin Polynomial]\label{def:taypoly}
Let $f$ be a function whose first $n$ derivatives exist at $x=c$.
\index{Taylor Polynomial!definition}\index{Maclaurin Polynomial!definition} \index{Maclaurin Polynomial|seealso{Taylor Polynomial}}
\begin{enumerate}
	\item	The \textbf{Taylor polynomial of degree $n$ of $f$ at $x=c$} is 
	\begin{align*}
	p_n(x)
	&= f(c) + \fp(c)(x-c) + \frac{\fpp(c)}{2!}(x-c)^2+\frac{\fp''(c)}{3!}(x-c)^3+\dotsb+\frac{f\,^{(n)}(c)}{n!}(x-c)^n \\
	&=\sum_{k=0}^n\frac{f\,^{(k)}(c)}{k!}(x-c)^k.
	\end{align*}
	\item	A special case of the Taylor polynomial is the Maclaurin polynomial, where $c=0$. That is, the \textbf{Maclaurin polynomial of degree $n$ of $f$} is 
	\begin{align*}
	p_n(x)
	&= f(0) + \fp(0)x + \frac{\fpp(0)}{2!}x^2+\frac{\fp''(0)}{3!}x^3+\dotsb+\frac{f\,^{(n)}(0)}{n!}x^n \\
	&=\sum_{k=0}^n\frac{f\,^{(k)}(0)}{k!}x^k.
	\end{align*}
\end{enumerate}
\end{definition}
}

\mnote{\textbf{Note:} The summations in this definition use the convention that $x^0=1$ even when $x=0$ and that $f^{(0)}=f$.  They also use the definition that $0!=1$.}
Generally, we order the terms of a polynomial to have decreasing degrees, and that is how we began this section.  This definition, and the rest of this chapter, reverses this order to reflect the greater importance of the lower degree terms in the polynomials that we will be finding.

\youtubeVideo{UINFWG0ErSA}{Taylor Polynomial to Approximate a Function, Ex 3}

We will practice creating Taylor and Maclaurin polynomials in the following examples.

\begin{example}[Finding and using Maclaurin polynomials]\label{ex_taypoly1}
\mbox{}\\[-2\baselineskip]\parbox[t]{\linewidth}{%
\begin{enumerate}
	\item	Find the $n^\text{th}$ Maclaurin polynomial for $f(x) = e^x$.
	\item	Use $p_5(x)$ to approximate the value of $e$.
\end{enumerate}}\vspace{0pt}
\solution
\mtable{The derivatives of $f(x)=e^x$ evaluated at $x=0$.}{fig:taypoly1a}{%
\begin{align*}
% todo Tim latexml https://github.com/brucemiller/LaTeXML/issues/1763
f(x)&=e^x \quad\Rightarrow & f(0)&=1 \\
\fp(x)&=e^x \quad\Rightarrow & \fp(0)&=1 \\
\fpp(x)&=e^x \quad\Rightarrow & \fpp(0)&=1 \\
\vdots && \vdots & \\
f\,^{(n)}(x)&=e^x \quad\Rightarrow & f\,^{(n)}(0)&=1
%f(x)&=e^x & \Rightarrow && f(0)&=1 \\
%\fp(x)&=e^x & \Rightarrow && \fp(0)&=1 \\
%\fpp(x)&=e^x & \Rightarrow && \fpp(0)&=1 \\
%\vdots &&&& \vdots & \\
%f\,^{(n)}(x)&=e^x & \Rightarrow && f\,^{(n)}(0)&=1
\end{align*}}
\begin{enumerate}
\item We start with creating a table of the derivatives of $e^x$ evaluated at $x=0$. In this particular case, this is relatively simple, as shown in \autoref{fig:taypoly1a}. By the definition of the Maclaurin polynomial, we have 
\[
	p_n(x)
	=\sum_{k=0}^n\frac{f\,^{(k)}(0)}{k!}x^k
	=\sum_{k=0}^n\frac1{k!}x^k.
\]

\item	Using our answer from part 1, we have
\[p_5(x) = 1+x+\frac{1}{2}x^2+\frac{1}{6}x^3 + \frac{1}{24}x^4 + \frac{1}{120}x^5.\]
To approximate the value of $e$, note that $e = e^1 = f(1) \approx p_5(1).$ It is very straightforward to evaluate $p_5(1)$:
%
\mtable{A plot of $f(x)=e^x$ and its 5$^\text{th}$ degree Maclaurin polynomial $p_5(x)$.}{fig:taypoly1b}{\pdftooltip{\begin{tikzpicture}
\begin{axis}[width=1.16\marginparwidth,tick label style={font=\scriptsize},
axis y line=middle,axis x line=middle,name=myplot,axis on top,
ymin=-3,ymax=11,xmin=-3.75,xmax=2.9]
\addplot [draw={\colorone},domain=-3.5:2.5,smooth,thick,samples=50] {exp(x)};
\addplot [draw={\colortwo},domain=-4:4,smooth,thick] {1+x+x^2/2+x^3/6+x^4/24+x^5/120};
\draw (axis cs:-3.,-2) node {\scriptsize $y=p_5(x)$};
\end{axis}
\node [right] at (myplot.right of origin) {\scriptsize $x$};
\node [above] at (myplot.above origin) {\scriptsize $y$};
\end{tikzpicture}}{ALT-TEXT-TO-BE-DETERMINED}}
%
\[p_5(1) = 1+1+\frac12+\frac16+\frac1{24}+\frac1{120} = \frac{163}{60} \approx 2.71667.\]
This is an error of about $0.0016$, or $0.06\%$ of the true value.

A plot of $f(x)=e^x$ and $p_5(x)$ is given in \autoref{fig:taypoly1b}.
\end{enumerate}
\end{example}

\begin{example}[Finding and using Taylor polynomials]\label{ex_taypoly2}
\mbox{}\\[-2\baselineskip]\parbox[t]{\linewidth}{\begin{enumerate}
	\item	Find the $n^\text{th}$ Taylor polynomial of $y=\ln x$ at $x=1$.
	\item	Use $p_6(x)$ to approximate the value of $\ln 1.5$.
	\item	Use $p_6(x)$ to approximate the value of $\ln 2$. 
\end{enumerate}}\vspace{0pt}
\solution
\begin{enumerate}
\item	We begin by creating a table of derivatives of $\ln x$ evaluated at $x=1$. While this is not as straightforward as it was in the previous example, a pattern does emerge, as shown in \autoref{fig:taypoly2a}.
\mtable[-5\baselineskip]{Derivatives of $\ln x$ evaluated at $x=1$.}{fig:taypoly2a}{%
\begin{align*}
% todo Tim latexml https://github.com/brucemiller/LaTeXML/issues/1763
f(x) &= \makebox[3em]{$\ln x$} \quad \Rightarrow & f(1)&=\phantom-0\\
\fp(x) &= \makebox[3em]{$\phantom-1/x$} \quad \Rightarrow & \fp(1) &= \phantom-1\\
\fp'(x) &= \makebox[3em]{$-1/x^2$} \quad \Rightarrow & \fp'(1) &= -1\\
\fp''(x) &= \makebox[3em]{$\phantom-2/x^3$} \quad \Rightarrow & \fp''(1) &= \phantom-2\\
f\,^{(4)}(x) &= \makebox[3em]{$-6/x^4$} \quad \Rightarrow & f\,^{(4)}(1) &= -6\\
%f\,^{(5)}(x) &= \makebox[3em]{$24/x^5$} \quad\Rightarrow  &f\,^{(5)}(1) &= 24\\
\vdots && \vdots & \\
f\,^{(n)}(x) &= \makebox[3em]{}\quad \Rightarrow & f\,^{(n)}(1) &= \smallskip\\
\makebox[1pt]{$\dfrac{(-1)^{n+1}(n-1)!}{x^n}$} && \makebox[1pt]{$(-1)^{n+1}(n-1)!$} &
%f(x) &= \ln x & \Rightarrow && f(1)&=\phantom-0\\
%\fp(x) &= \phantom-1/x & \Rightarrow && \fp(1) &= \phantom-1\\
%\fp'(x) &= -1/x^2 & \Rightarrow && \fp'(1) &= -1\\
%\fp''(x) &= \phantom-2/x^3 & \Rightarrow && \fp''(1) &= \phantom-2\\
%f\,^{(4)}(x) &= -6/x^4 & \Rightarrow && f\,^{(4)}(1) &= -6\\
%\vdots &&&& \vdots & \\
%f\,^{(n)}(x) &= & \Rightarrow && f\,^{(n)}(1) &= \smallskip\\
%\makebox[1pt]{$\dfrac{(-1)^{n+1}(n-1)!}{x^n}$} &&&& \makebox[1pt]{$(-1)^{n+1}(n-1)!$} &
\end{align*}}

Using \autoref{def:taypoly}, we have
\[
	p_n(x)
	= \sum_{k=0}^n\frac{f\,^{(k)}(c)}{k!}(x-c)^k
	= \sum_{k=1}^n\frac{(-1)^{k+1}}k(x-1)^k.
\]

\item	We can compute $p_6(x)$ using our work above:
\[
p_6(x)
 = (x-1)-\frac12(x-1)^2+\frac13(x-1)^3-\frac14(x-1)^4+\frac15(x-1)^5-\frac16(x-1)^6.
\]
Since $p_6(x)$ approximates $\ln x$ well near $x=1$, we approximate $\ln 1.5 \approx p_6(1.5)$:
\begin{align*}
	p_6(1.5)
	&= (1.5-1)-\frac12(1.5-1)^2+\frac13(1.5-1)^3 \\
	&\qquad\qquad{}-\frac14(1.5-1)^4+\frac15(1.5-1)^5-\frac16(1.5-1)^6\\
	&=\frac{259}{640}\\
	&\approx 0.404688.
\end{align*}
%
\mtable{A plot of $y=\ln x$ and its 6$^\text{th}$ degree Taylor polynomial at $x=1$.}{fig:taypoly2b}{\pdftooltip{\begin{tikzpicture}
\begin{axis}[width=1.16\marginparwidth,tick label style={font=\scriptsize},
axis y line=middle,axis x line=middle,name=myplot,axis on top,
ymin=-4.5,ymax=2.4,xmin=-.5,xmax=3.2]
\addplot [draw={\colorone},domain=0.01:3,smooth,thick,samples=50] {ln(x)};
\addplot [draw={\colortwo},domain=-.5:3.2,smooth,thick]
 {(x-1)-(x-1)^2/2+(x-1)^3/3-(x-1)^4/4+(x-1)^5/5-(x-1)^6/6};
\draw (axis cs:2.5,1.65) node {\scriptsize $y=\ln x$};
\draw (axis cs:2.2,-2.75) node {\scriptsize $y=p_{6}(x)$};
\end{axis}
\node [right] at (myplot.right of origin) {\scriptsize $x$};
\node [above] at (myplot.above origin) {\scriptsize $y$};
\end{tikzpicture}}{ALT-TEXT-TO-BE-DETERMINED}}
%
This is a good approximation as a calculator shows that $\ln 1.5 \approx 0.4055.$  %This is an error of $0.0008$, or $0.2\%$. % this is in next example
\autoref{fig:taypoly2b} plots $y=\ln x$ with $y=p_6(x)$. We can see that $\ln 1.5\approx p_6(1.5)$.

\item	
We approximate $\ln 2$ with $ p_6(2)$:
\begin{align*}
p_6(2) &= (2-1)-\frac12(2-1)^2+\frac13(2-1)^3 \\
			&\qquad\qquad{}-\frac14(2-1)^4+\frac15(2-1)^5-\frac16(2-1)^6\\
			&=	1-\frac12+\frac13-\frac14+\frac15-\frac16 \\
			&= \frac{37}{60}\\ 
			&\approx 0.616667.
\end{align*}
This approximation is not terribly impressive: a hand held calculator shows that $\ln 2 \approx 0.693147.$
% This is an error of $0.08$, or $11\%$. % this is in next example
The graph in \autoref{fig:taypoly2b} shows that $p_6(x)$ provides less accurate approximations of $\ln x$ as $x$ gets close to 0 or 2. 

% todo use 20th degree polynomial instead of coordinates in Figure 9.9.8 fig:taypoly2c
\mtable[-.5in]{A plot of $y=\ln x$ and its 20$^\text{th}$ degree Taylor polynomial at $x=1$.}{fig:taypoly2c}{\pdftooltip{\begin{tikzpicture}
\begin{axis}[width=1.16\marginparwidth,tick label style={font=\scriptsize},
axis y line=middle,axis x line=middle,name=myplot,axis on top,
ymin=-4.5,ymax=2.4,xmin=-.5,xmax=3.2]
\addplot [draw={\colorone},domain=0.01:3,smooth,thick,samples=50] {ln(x)};
\addplot [draw={\colortwo},smooth,thick] coordinates {
 (-0.108,-7.567)(-0.052,-4.958)(0.004,-3.519)(0.06,-2.671)(0.116,-2.13)
 (0.172,-1.756)(0.228,-1.478)(0.284,-1.259)(0.34,-1.079)(0.396,-0.9263)
 (0.452,-0.7941)(0.508,-0.6773)(0.564,-0.5727)(0.62,-0.478)(0.676,-0.3916)
 (0.732,-0.312)(0.788,-0.2383)(0.844,-0.1696)(0.9,-0.1054)(0.956,-0.045)
 (1.012,0.01193)(1.068,0.06579)(1.124,0.1169)(1.18,0.1655)(1.236,0.2119)
 (1.292,0.2562)(1.348,0.2986)(1.404,0.3393)(1.46,0.3784)(1.516,0.4161)
 (1.572,0.4523)(1.628,0.4874)(1.684,0.5212)(1.74,0.5538)(1.796,0.5853)
 (1.852,0.6154)(1.908,0.6427)(1.964,0.6635)(2.02,0.6665)(2.076,0.621)
 (2.132,0.4476)(2.188,-0.04877)(2.244,-1.326)(2.3,-4.421)};
% the 20th degree Taylor approximation
\draw (axis cs:2.5,1.65) node {\scriptsize $y=\ln x$};
\draw (axis cs:1.7,-2.75) node {\scriptsize $y=p_{20}(x)$};
\end{axis}
\node [right] at (myplot.right of origin) {\scriptsize $x$};
\node [above] at (myplot.above origin) {\scriptsize $y$};
\end{tikzpicture}}{ALT-TEXT-TO-BE-DETERMINED}}

Surprisingly enough, even the 20$^\text{th}$ degree Taylor polynomial fails to approximate $\ln x$ for $x>2$, as shown in \autoref{fig:taypoly2c}. We'll soon discuss why this is.
\end{enumerate}
\end{example}

Taylor polynomials are used to approximate functions $f(x)$ in mainly two situations:
\begin{enumerate}
	\item	When $f(x)$ is known, but perhaps ``hard'' to compute directly. For instance, we can define $y=\cos x$ as either the ratio of sides of a right triangle (``adjacent over hypotenuse'') or with the unit circle. However, neither of these provides a convenient way of computing $\cos 2$. A Taylor polynomial of sufficiently high degree can provide a reasonable method of computing such values using only operations usually hard-wired into a computer ($+$, $-$, $\times$ and $\div$).
	
	\item	When $f(x)$ is not known, but information about its derivatives is known. This occurs more often than one might think, especially in the study of differential equations.
\end{enumerate}

\mnote{\textbf{Note:} Even though Taylor polynomials \emph{could} be used in calculators and computers to calculate values of trigonometric functions, in practice they generally aren't. Other more efficient and accurate methods have been developed, such as the CORDIC algorithm.}
	
In both situations, a critical piece of information to have is ``How good is my approximation?'' If we use a Taylor polynomial to compute $\cos 2$, how do we know how accurate the approximation is? 

We had the same problem with Numerical Integration. \autoref{thm:numerical_error} provided bounds on the error when using, say, Simpson's Rule to approximate a definite integral. These bounds allowed us to determine that, for example, using $10$ subintervals provided an approximation within $\pm .01$ of the exact value. The following theorem gives similar bounds for Taylor (and hence Maclaurin) polynomials.

{\tcbset{grow to right by=.5em}
\begin{theorem}[Taylor's Theorem]\label{thm:taylorthm}
\mbox{}\\[-2\baselineskip]\index{Taylor Polynomial!Taylor's Theorem}\index{Taylor's Theorem}
\begin{enumerate}
	\item	Let $f$ be a function whose $(n+1)^{\text{th}}$ derivative exists on an open interval $I$ and let $c$ be in $I$. Then, for each $x$ in $I$, there exists $z_x$ between $x$ and $c$ such that\vspace{-.3\baselineskip}
\[R_n(x) = f(x) - \sum_{k=0}^n\frac{f\,^{(k)}(c)}{k!}(x-c)^k = \frac{f\,^{(n+1)}(z_x)}{(n+1)!}(x-c)^{n+1}.\]
%\[f(x) = \sum_{k=0}^n\frac{f\,^{(k)}(c)}{k!}(x-c)^k+R_n(x),\]
%where $\ds R_n(x) = \frac{f\,^{(n+1)}(z_x)}{(n+1)!}(x-c)^{n+1}.$
	\item	$\abs{R_n(x)}\leq\dfrac{\max_z\abs{\,f\,^{(n+1)}(z)}}{(n+1)!}\abs{x-c}^{n+1}$, where $z$ is between $x$ and $c$.
\end{enumerate}
\end{theorem}}

The first part of Taylor's Theorem states that $f(x) = p_n(x) + R_n(x)$, where $p_n(x)$ is the $n^\text{th}$ order Taylor polynomial and $R_n(x)$ is the remainder, or error, in the Taylor approximation. The second part gives bounds on how big that error can be. If the $(n+1)^\text{th}$ derivative is large, the error may be large; if $x$ is far from $c$, the error may also be large. However, the $(n+1)!$ term in the denominator tends to ensure that the error gets smaller as $n$ increases.

The following example computes error estimates for the approximations of $\ln 1.5$ and $\ln 2$ made in \autoref{ex_taypoly2}.

\begin{example}[Finding error bounds of a Taylor polynomial]\label{ex_taypoly3}
Use \autoref{thm:taylorthm} to find error bounds when approximating $\ln 1.5$ and $\ln 2$ with $p_6(x)$, the Taylor polynomial of degree 6 of $f(x)=\ln x$ at $x=1$, as calculated in \autoref{ex_taypoly2}.
\solution
\begin{enumerate}
\item	We start with the approximation of $\ln 1.5$ with $p_6(1.5)$.
% Taylor's Theorem references an open interval $I$ that contains both $x$ and $c$. The smaller the interval we use the better; it will give us a more accurate (and smaller) approximation of the error. We let $I = (0.9,1.6)$, as this interval contains both $c=1$ and $x=1.5$. 
%
Taylor's Theorem references $\max\abs{f\,^{(n+1)}(z)}$. In our situation, this is asking ``How big can the $7^\text{th}$ derivative of $y=\ln x$ be on the interval $[1,1.5]$?'' The seventh derivative is $y = 6!/x^7$. The largest absolute value it attains on $I$ is 720. Thus we can bound the error as:
\begin{align*}
	\abs{R_6(1.5)}
	&\leq \frac{\max\abs{f\,^{(7)}(z)}}{7!}\abs{1.5-1}^7\\
	&\leq \frac{720}{5040}\cdot\frac1{2^7}\\
	&\approx 0.001.
\end{align*}
We computed $p_6(1.5) = 0.404688$; using a calculator, we find $\ln 1.5 \approx 0.405465$, so the actual error is about $0.000778$ (or $0.2\%$), which is less than our bound of $0.001$. This affirms Taylor's Theorem; the theorem states that our approximation would be within about one thousandth of the actual value, whereas the approximation was actually closer.

	\item	%We again find an interval $I$ that contains both $c=1$ and $x=2$; we choose $I = (0.9,2.1)$.
	The maximum value of the seventh derivative of $f$ on $[1,2]$ %this interval
	is again 720 (as the largest values come at $x=1$). Thus 
\begin{align*}
	\abs{R_6(2)}
	&\leq \frac{\max\abs{f\,^{(7)}(z)}}{7!}\abs{2-1}^7\\
	&\leq \frac{720}{5040}\cdot1^7\\
	&\approx0.15.
\end{align*}
This bound is not as nearly as good as before. Using the degree 6 Taylor polynomial at $x =1$ will bring us within 0.15 of the correct answer. As $p_6(2)\approx 0.61667$, our error estimate guarantees that the actual value of $\ln 2$ is somewhere between $0.46$ and $0.76$. These bounds are not particularly useful.

In reality, our approximation was only off by about $0.07$ (or $11\%$). However, we are approximating ostensibly because we do not know the real answer. In order to be assured that we have a good approximation, we would have to resort to using a polynomial of higher degree.
\end{enumerate}
\end{example}

We practice again. This time, we use Taylor's theorem to find $n$ that guarantees our approximation is within a certain amount.

\begin{example}[Finding sufficiently accurate Taylor polynomials]\label{ex_taypoly4}
Find $n$ such that the $n^\text{th}$ Taylor polynomial of $f(x)=\cos x$ at $x=0$ approximates $\cos 2$ to within $0.001$ of the actual answer. What is $p_n(2)$?
\solution
Following Taylor's theorem, we need bounds on the size of the derivatives of $f(x)=\cos x$. In the case of this trigonometric function, this is easy. All derivatives of cosine are $\pm \sin x$ or $\pm \cos x$. In all cases, these functions are never greater than 1 in absolute value. We want the error to be less than $0.001$. To find the appropriate $n$, consider the following inequalities:
\begin{align*}
\frac{\max\abs{f\,^{(n+1)}(z)}}{(n+1)!}\abs{2-0}^{n+1} &\leq 0.001 \\
\frac1{(n+1)!}\cdot2^{n+1} &\leq 0.001
\end{align*}
We find an $n$ that satisfies this last inequality with trial-and-error. When $n=8$, we have $\ds \frac{2^{8+1}}{(8+1)!} \approx 0.0014$; when $n=9$, we have $\ds \frac{2^{9+1}}{(9+1)!} \approx 0.000282 <0.001$. Thus we want to approximate $\cos 2$ with $p_9(2)$.\bigskip

\mtable[-2\baselineskip]{A table of the derivatives of $f(x)=\cos x$ evaluated at $x=0$.}{fig:taypoly4a}{\begin{align*}
f(x) &= \phantom-\cos x & \Rightarrow && f(0) &= \phantom-1\\
\fp(x) &= -\sin x & \Rightarrow && \fp(0) &= \phantom-0\\
\fp'(x) &= -\cos x & \Rightarrow && \fp'(0) &= -1\\
\fp''(x) &= \phantom-\sin x & \Rightarrow && \fp''(0) &= \phantom-0\\
f\,^{(4)}(x) &= \phantom-\cos x & \Rightarrow && f\,^{(4)}(0) &= \phantom-1\\
f\,^{(5)}(x) &= -\sin x & \Rightarrow && f\,^{(5)}(0) &= \phantom-0\\
f\,^{(6)}(x) &= -\cos x & \Rightarrow && f\,^{(6)}(0) &= -1\\
f\,^{(7)}(x) &= \phantom-\sin x & \Rightarrow && f\,^{(7)}(0) &= \phantom-0\\
f\,^{(8)}(x) &= \phantom-\cos x & \Rightarrow && f\,^{(8)}(0) &= \phantom-1\\
f\,^{(9)}(x) &= -\sin x & \Rightarrow && f\,^{(9)}(0) &= \phantom-0
\end{align*}}

We now set out to compute $p_9(x)$. We again need a table of the derivatives of $f(x)=\cos x$ evaluated at $x=0$. A table of these values is given in \autoref{fig:taypoly4a}. Notice how the derivatives, evaluated at $x=0$, follow a certain pattern. All the odd powers of $x$ in the Taylor polynomial will disappear as their coefficient is 0. While our error bounds state that we need $p_9(x)$, our work shows that this will be the same as $p_8(x)$.

Since we are forming our polynomial at $x=0$, we are creating a Maclaurin polynomial, and:\vspace{-.3\baselineskip}
\[
	p_8(x) = \sum_{k=0}^8\frac{f\,^{(k)}(0)}{k!}x^k
	=  1-\frac{1}{2!}x^2+\frac{1}{4!}x^4-\frac{1}{6!}x^6+\frac{1}{8!}x^8
\]

We finally approximate $\cos 2$:
\[\cos 2 \approx p_8(2) = -\frac{131}{315} \approx -0.41587.\]
%
% todo Tim unnest tikzpicture
\mtable{A graph of $f(x)= \cos x$ and its degree 8 Maclaurin polynomial.}{fig:taypoly4b}{\pdftooltip{\begin{tikzpicture}
\begin{axis}[width=1.16\marginparwidth,tick label style={font=\scriptsize},
axis y line=middle,axis x line=middle,name=myplot,axis on top,
xtick={-5,-4,-3,-2,-1,1,2,3,4,5},ymin=-1.1,ymax=1.5,xmin=-5.5,xmax=5.5]
\addplot [draw={\colorone},domain=-5:5,smooth,thick,samples=50] {cos(deg(x))};
\addplot [draw={\colortwo},domain=-4:4,smooth,thick] {1-x^2/2+x^4/24-x^6/720+x^8/40320};
\draw (axis cs:-3.,1) node {\scriptsize $y=p_8(x)$};
\end{axis}
\node [right] at (myplot.right of origin) {\scriptsize $x$};
\node [above] at (myplot.above origin) {\scriptsize $y$};
\node [below] at (myplot.below origin) {
 \begin{tikzpicture}
  \draw[thick,draw={\colorone}] (0,0)--(10pt,0)
   node [right,black] {\scriptsize $f(x)= \cos x$};
 \end{tikzpicture}};
\end{tikzpicture}}{ALT-TEXT-TO-BE-DETERMINED}}
%
Our error bound guarantees that this approximation is within $0.001$ of the correct answer. Technology shows us that our approximation is actually within about $0.0003$ (or $0.07\%$) of the correct answer.

\autoref{fig:taypoly4b} shows a graph of $y=p_8(x)$ and $y=\cos x$. Note how well the two functions agree on about $(-\pi,\pi)$.
\end{example}

\begin{example}[Finding and using Taylor polynomials]\label{ex_taypoly5}
\mbox{}\\[-2\baselineskip]\parbox[t]{\linewidth}{\begin{enumerate}
	\item	Find the degree 4 Taylor polynomial, $p_4(x)$, for $f(x)=\sqrt{x}$ at $x=4.$
	\item	Use $p_4(x)$ to approximate $\sqrt{3}$.
	\item	Find bounds on the error when approximating $\sqrt{3}$ with $p_4(3)$.
\end{enumerate}}\vspace{0pt}
\solution
\begin{enumerate}
	\item	We begin by evaluating the derivatives of $f$ at $x=4$. This is done in \autoref{fig:taypoly5a}. These values allow us to form the Taylor polynomial $p_4(x)$:
	%
\mtable[-2\baselineskip]{A table of the derivatives of $f(x)=\sqrt{x}$ evaluated at $x=4$.}{fig:taypoly5a}{%
\begin{align*}
f(x) &= \sqrt{x} & \Rightarrow && f(4) &= 2\\
\ds\fp(x) &= \frac{1}{2\sqrt{x}} & \Rightarrow && \ds\fp(4) &= \frac{1}{4}\\
\ds\fp'(x) &= \frac{-1}{4x^{3/2}} & \Rightarrow && \ds\fp'(4) &= \frac{-1}{32}\\
\ds\fp''(x) &= \frac3{8x^{5/2}} & \Rightarrow && \ds\fp''(4) &= \frac{3}{256}\\
\ds f\,^{(4)}(x) &= \frac{-15}{16x^{7/2}} & \Rightarrow
&& \ds f\,^{(4)}(4) &= \frac{-15}{2048}\vspace{-.5\baselineskip}
\end{align*}}
%
\begin{multline*}
p_4(x) = \\
2 + \frac14(x-4) +\frac{-1/32}{2!}(x-4)^2+\frac{3/256}{3!}(x-4)^3+\frac{-15/2048}{4!}(x-4)^4.
\end{multline*}

	\item	As $p_4(x) \approx \sqrt{x}$ near $x=4$, we approximate $\sqrt{3}$ with $p_4(3) = 1.73212$.

	\item	%To find a bound on the error, we need an open interval that contains $x=3$ and $x=4$. We set $I = (2.9,4.1)$.
	The largest value the fifth derivative of $f(x)=\sqrt{x}$ takes on $[3,4]$ is when $x=3$, at about $0.0234$. Thus
\[\abs{R_4(3)} \leq \frac{0.0234}{5!}\abs{3-4}^5 \approx 0.00019.\]
%
\mtable[\baselineskip]{A graph of $f(x)=\sqrt{x}$ and its degree 4 Taylor polynomial at $x=4$.}{fig:taypoly5b}{\pdftooltip{\begin{tikzpicture}
\begin{axis}[width=1.16\marginparwidth,tick label style={font=\scriptsize},
axis y line=middle,axis x line=middle,name=myplot,axis on top,
ymin=-.1,ymax=3.5,xmin=-1,xmax=11]
\addplot [draw={\colorone},domain=0:10,smooth,thick,samples=50] {sqrt(x)};
\addplot [draw={\colortwo},smooth,thick,domain=0:10]
 {2+(x-4)/4-(x-4)^2/64+3*(x-4)^3/1536-15*(x-4)^4/49152};
\draw (axis cs:8.,1) node {\begin{tikzpicture}
  \draw[thick,draw={\colorone}](0,0)--(10pt,0)node[right,black]{\scriptsize$y=\sqrt x$};
  \draw[draw={\colortwo},thick](0,-10pt)--(10pt,-10pt)
   node[right,black]{\scriptsize$y=p_4(x)$};
 \end{tikzpicture}};
\end{axis}
\node [right] at (myplot.right of origin) {\scriptsize $x$};
\node [above] at (myplot.above origin) {\scriptsize $y$};
\end{tikzpicture}}{ALT-TEXT-TO-BE-DETERMINED}}%
%
This shows our approximation is accurate to at least the first 2 places after the decimal. It turns out that our approximation has an error of $0.00007$, or $0.004\%$. A graph of $f(x)=\sqrt x$ and $p_4(x)$ is given in \autoref{fig:taypoly5b}. Note how the two functions are nearly indistinguishable on $(2,7)$.
\end{enumerate}
\end{example}

%Our final example gives a brief introduction to using Taylor polynomials to solve differential equations.
%
%\begin{example}[Approximating an unknown function]\label{ex_taypoly6}
%A function $y=f(x)$ is unknown save for the following two facts.
%\begin{enumerate}
%\item		$y(0) = f(0) = 1$, and
%\item		$y\primeskip'= y^2$
%\end{enumerate}
%(This second fact says that amazingly, the derivative of the function is actually the function squared!)
%
%Find the degree 3 Maclaurin polynomial $p_3(x)$ of $y=f(x)$.
%\solution
%One might initially think that not enough information is given to find $p_3(x)$. However, note how the second fact above actually lets us know what $y\primeskip'(0)$ is:
%\[y\primeskip' = y^2 \Rightarrow y\primeskip'(0) = y^2(0).\]
%Since $y(0) = 1$, we conclude that $y\primeskip'(0) = 1$.
%
%Now we find information about $y\primeskip''$. Starting with $y\primeskip'=y^2$, take derivatives of both sides, \emph{with respect to $x$}. That means we must use implicit differentiation.
%\begin{align*}
%y\primeskip' &= y^2\\
%\frac{\dd}{\dd x}\bigl(y\primeskip'\bigr) &= \frac{\dd}{\dd x}\bigl(y^2\bigr)\\
%y\primeskip'' &= 2y\cdot y\primeskip'.
%\intertext{Now evaluate both sides at $x=0$:}
%y\primeskip''(0) &= 2y(0)\cdot y\primeskip'(0)\\
%y\primeskip''(0) &= 2
%\end{align*}
%We repeat this once more to find $y\primeskip'''(0)$. We again use implicit differentiation; this time the Product Rule is also required.
%\begin{align*}
%\frac{\dd}{\dd x}\bigl(y\primeskip''\bigr) &= \frac{\dd}{\dd x} \bigl(2yy\primeskip'\bigr)\\
%y\primeskip''' &= 2y\primeskip'\cdot y\primeskip' + 2y\cdot y\primeskip''.
%\intertext{Now evaluate both sides at $x=0$:}
%y\primeskip'''(0) &= 2y\primeskip'(0)^2 + 2y(0)y\primeskip''(0)\\
%y\primeskip'''(0) &=	2+4=6
%\end{align*}
%In summary, we have:
%\[y(0) = 1 \qquad y\primeskip'(0) = 1  \qquad y\primeskip''(0) = 2 \qquad y\primeskip'''(0) = 6.\]
%We can now form $p_3(x)$:
%\begin{align*}
%p_3(x) &= 1 + x + \frac{2}{2!}x^2 + \frac{6}{3!}x^3\\
%				&= 1+x+x^2+x^3.
%\end{align*}
%\mtable{A graph of $y=-1/(x-1)$ and $y=p_3(x)$ from \autoref{ex_taypoly6}.}{fig:taypoly6}{\pdftooltip{\begin{tikzpicture}
%\begin{axis}[width=1.16\marginparwidth,tick label style={font=\scriptsize},
%axis y line=middle,axis x line=middle,name=myplot,axis on top,
%ymin=-.1,ymax=3.5,xmin=-1.1,xmax=1.1]
%\addplot [draw={\colorone},domain=-1:.7,smooth,thick,samples=50] {-1/(x-1)};
%\addplot [draw={\colortwo},smooth,thick] coordinates {(-1.,0)(-0.96,0.07686)(-0.92,0.1477)(-0.88,0.2129)(-0.84,0.2729)(-0.8,
%0.328)(-0.76,0.3786)(-0.72,0.4252)(-0.68,0.468)(-0.64,0.5075)(-0.6,0.
%544)(-0.56,0.578)(-0.52,0.6098)(-0.48,0.6398)(-0.44,0.6684)(-0.4,0.
%696)(-0.36,0.7229)(-0.32,0.7496)(-0.28,0.7764)(-0.24,0.8038)(-0.2,0.
%832)(-0.16,0.8615)(-0.12,0.8927)(-0.08,0.9259)(-0.04,0.9615)(0,1.)(0.
%04,1.042)(0.08,1.087)(0.12,1.136)(0.16,1.19)(0.2,1.248)(0.24,1.311)(0.
%28,1.38)(0.32,1.455)(0.36,1.536)(0.4,1.624)(0.44,1.719)(0.48,1.821)(0.
%52,1.931)(0.56,2.049)(0.6,2.176)(0.64,2.312)(0.68,2.457)(0.72,2.612)(
%0.76,2.777)(0.8,2.952)(0.84,3.138)(0.88,3.336)(0.92,3.545)(0.96,3.766)(1.,4.)};
%\draw (axis cs:.75,.9) node {\begin{tikzpicture}
%	\draw[thick,draw={\colorone}] (0,0)--(10pt,0) node [right,black] {\scriptsize $\displaystyle y= \frac{1}{1-x}$};
%	\draw[draw={\colortwo},thick] (0,-15pt)--(10pt,-15pt) node [right,black] {\scriptsize $y= p_3(x)$};
%\end{tikzpicture}};
%\end{axis}
%\node [right] at (myplot.right of origin) {\scriptsize $x$};
%\node [above] at (myplot.above origin) {\scriptsize $y$};
%\end{tikzpicture}}{ALT-TEXT-TO-BE-DETERMINED}}
%It turns out that the differential equation we started with, $y\primeskip'=y^2$, where $y(0)=1$, can be solved without too much difficulty: $\ds y = \frac{1}{1-x}$. \autoref{fig:taypoly6} shows this function plotted with $p_3(x)$. Note how similar they are near $x=0$.
%\end{example}
%
%It is beyond the scope of this text to pursue error analysis when using Taylor polynomials to approximate solutions to differential equations. This topic is often broached in introductory Differential Equations courses and usually covered in depth in Numerical Analysis courses. Such an analysis is very important; one needs to know how good their approximation is. We explored this example simply to demonstrate the usefulness of Taylor polynomials.

\bigskip

Most of this chapter has been devoted to the study of infinite series. This section has stepped aside from this study, focusing instead on finite summation of terms. In the next section, we will combine power series and Taylor polynomials into \textbf{Taylor Series}, where we represent a function with an infinite series.

\printexercises{exercises/08-07-exercises}

% todo for exercises 21-24, reach back to lin approx / Newton’s method and redo the appropriate problems with Taylor series
