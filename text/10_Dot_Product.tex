\section{The Dot Product}\label{sec:dot_product}

The previous section introduced vectors and described how to add them together and how to multiply them by scalars. This section introduces \emph{a} multiplication on vectors called the \textbf{dot product}.

\begin{definition}[Dot Product]\label{def:dot_product}%
\index{dot product!definition}\index{vectors!dot product}
\mbox{}\\[-2\baselineskip]\parbox[t]{\linewidth}{\begin{enumerate}
	\item Let $\vec u =\bracket{u_1,u_2}$ and $\vec v = \bracket{v_1,v_2}$  in $\mathbb{R}^2$. The \textbf{dot product} of $\vec u$ and $\vec v$, denoted $\vec u\cdot\vec v$, is 
	\[\vec u\cdot\vec v = u_1v_1+u_2v_2.\]
	\item Let $\vec u = \bracket{u_1,u_2,u_3}$ and $\vec v = \bracket{v_1,v_2,v_3}$  in $\mathbb{R}^3$. The \textbf{dot product} of $\vec u$ and $\vec v$, denoted $\vec u\cdot \vec v$, is 
	\[\vec u\cdot\vec v = u_1v_1+u_2v_2+u_3v_3.\]
\end{enumerate}}
\end{definition}

Note how this product of vectors returns a \emph{scalar}, not another vector.

We practice evaluating a dot product in the following example, then we will discuss why this product is useful.

\begin{example}[Evaluating dot products]\label{ex_dotp1}%
\mbox{}\\[-2\baselineskip]\parbox[t]{\linewidth}{%
\begin{enumerate}
	\item Let $\vec u=\bracket{1,2}$, $\vec v=\bracket{3,-1}$ in $\mathbb{R}^2$. Find $\vec u\cdot\vec v$.
	\item Let $\vec x = \bracket{2,-2,5}$ and $\vec y = \bracket{-1, 0, 3}$ in $\mathbb{R}^3$. Find $\vec x\cdot\vec y$.
\end{enumerate}}\vspace{0pt}
\solution
\begin{enumerate}
	\item Using \autoref{def:dot_product}, we have
	\[\vec u\cdot\vec v = 1(3)+2(-1) = 1.\]
	\item	Using the definition, we have
	\[\vec x\cdot\vec y = 2(-1)  -2(0) + 5(3) = 13.\]
\end{enumerate}
\end{example}

The dot product, as shown by the preceding example, is very simple to evaluate. It is only the sum of products. While the definition gives no hint as to why we would care about this operation, there is an amazing connection between the dot product and angles formed by the vectors. Before stating this connection, we give a theorem stating some of the properties of the dot product.

\begin{theorem}[Properties of the Dot Product]\label{thm:dot_product_properties}%
Let $\vec u$, $\vec v$ and $\vec w$ be vectors in $\mathbb{R}^2$ or $\mathbb{R}^3$ and let $c$ be a scalar.\index{dot product!properties}\index{vectors!dot product}
\begin{enumerate}
	\item \parbox{150pt}{$\vec u\cdot\vec v = \vec v\cdot\vec u$}{Commutative Property}
	\item \parbox{150pt}{$\vec u\cdot(\vec v+\vec w) = \vec u\cdot\vec v + \vec u\cdot\vec w$}{Distributive Property}
	\item	$c(\vec u\cdot\vec v) = (c\vec u)\cdot \vec v = \vec u \cdot (c\vec v)$
	\item	$\vec 0\cdot\vec v = 0$
	\item	$\vec v\cdot\vec v=\norm{\vec v}^2 $
\end{enumerate}
\end{theorem}

The last statement of the theorem makes a handy connection between the magnitude of a vector and the dot product with itself. Our definition and theorem give properties of the dot product, but we are still likely wondering ``What does the dot product \emph{mean}?'' It is helpful to understand that the dot product of a vector with itself is connected to its magnitude.

The next theorem extends this understanding by connecting the dot product to magnitudes and angles. Given vectors $\vec u$ and $\vec v$ in the plane, an angle $\theta$ is clearly formed when $\vec u$ and $\vec v$ are drawn with the same initial point as illustrated in \autoref{fig:dotpangle}(a). (We always take $\theta$ to be the angle in $[0,\pi]$ as two angles are actually created.) 

\mtable{Illustrating the angle formed by two vectors with the same initial point.}{fig:dotpangle}{%
\begin{tikzpicture}[alt={Vectors u, v share a tail; interior angle θ is sketched between the two arrow shafts.},>=stealth,scale=1.25]
	\draw [rotate=-15,->] (0,0) -- (2,0) node (A) [right] {\scriptsize $\vec u$};
	\draw [rotate=35,->] (0,0) -- (2.5,0) node (B) [right] {\scriptsize $\vec v$};
	\draw [rotate=-15,->] (.75,0) arc (0:50:.75);
	\draw [rotate=10] (.9,0) node {\scriptsize $\theta$};
\end{tikzpicture}
\\(a)\\[15pt]
\myincludeasythree{
3Droll=0,
3Dortho=0.0045,
3Dc2c=.54 .61 .58,
3Dcoo=0 0 40,
3Droo=170}{\marginparwidth}{3-D axes; u, v lie in a shaded plane through origin; angle θ marked between them.}{figures/figdotpangle3D_3D}
\\(b)}

The same is also true of 2 vectors in space: given $\vec u$ and $\vec v$ in $\mathbb{R}^3$ with the same initial point, there is a plane that contains both $\vec u$ and $\vec v$. (When $\vec u$ and $\vec v$ are co-linear, there are infinite planes that contain both vectors.) In that plane, we can again find an angle $\theta$ between them (and again, $0\leq \theta\leq \pi$). This is illustrated in \autoref{fig:dotpangle}(b).

The following theorem connects this angle $\theta$ to the dot product of $\vec u$ and $\vec v$.

\begin{theorem}[The Dot Product and Angles]\label{thm:dot_product}%
Let $\vec u$ and $\vec v$ be vectors in $\mathbb{R}^2$ or $\mathbb{R}^3$. Then 
\[\vec u\cdot\vec v = \norm{\vec u}\,\norm{\vec v} \cos\theta,\]
where $\theta$, $0\leq\theta\leq \pi$, is the angle between $\vec u$ and $\vec v$.
\index{dot product!properties}\index{vectors!dot product}
\end{theorem}

When $\theta$ is an acute angle (i.e., $0\leq \theta <\pi/2$), $\cos \theta$ is positive; when $\theta = \pi/2$, $\cos \theta = 0$; when $\theta$ is an obtuse angle ($\pi/2<\theta \leq \pi$), $\cos \theta$ is negative. Thus the sign of the dot product gives a general indication of the angle between the vectors, illustrated in \autoref{fig:dotpsign}.

\noindent\begin{minipage}[t]{\linewidth}\noindent%
\captionsetup{type=figure}%
\centering
\begin{tikzpicture}[alt={Panels: (1) acute θ between u and v, u·v > 0; (2) right angle θ = π⁄2, u·v = 0; (3) obtuse θ, u·v < 0.},>=stealth,thick,scale=1.25]
	\begin{scope}
		\draw[->] (0,0) -- (1,0) node [pos=.5,below]
		 {\scriptsize $\vec u\cdot \vec v >0$} node [right] {\scriptsize $\vec u$};
		\draw[->] (0,0) -- (.5,.866)node [above right] {\scriptsize $\vec v$};
		\draw[->] (.3,0) arc (0:60:.3);
		\draw[rotate=30] (.45,0) node {\scriptsize $\theta$};
	\end{scope}
	\begin{scope}[shift={(2.4cm,0)}]
		\draw[->] (0,0) -- (1,0)node [pos=.5,below]
		 {\scriptsize $\vec u\cdot \vec v =0$}node [right] {\scriptsize $\vec u$};
		\draw[->] (0,0) -- (0,1)node [above ] {\scriptsize $\vec v$};
		\draw[->] (.3,0) arc (0:90:.3);
		\draw[rotate=22.5] (.7,0) node {\scriptsize $\theta=\pi/2$};
	\end{scope}
	\begin{scope}[shift={(5.2cm,0)}]
		\draw[->] (0,0) -- (1,0)node [pos=.5,below]
		 {\scriptsize $\vec u\cdot \vec v <0$}node [right] {\scriptsize $\vec u$};
		\draw[->] (0,0) -- (-.707,.707)node [above left] {\scriptsize $\vec v$};
		\draw[->] (.3,0) arc (0:135:.3);
		\draw[rotate=67.5] (.45,0) node {\scriptsize $\theta$};
	\end{scope}
\end{tikzpicture}
\caption{Illustrating the relationship between the angle between vectors and the sign of their dot product.}
\label{fig:dotpsign}
\end{minipage}

We \emph{can} use \autoref{thm:dot_product} to compute the dot product, but generally this theorem is used to find the angle between known vectors (since the dot product is generally easy to compute). To this end, we rewrite the theorem's equation as\vspace{-.3\baselineskip}
\[\cos \theta = \frac{\vec u\cdot\vec v}{\norm{\vec u}\norm{\vec v}} \quad \Leftrightarrow \quad \theta = \cos^{-1}\left(\frac{\vec u\cdot\vec v}{\norm{\vec u}\norm{\vec v}}\right).\]

\youtubeVideo{98C7iv8OcnI}{Vectors: The Dot Product}

We practice using this theorem in the following example.

\mtable{Vectors used in \autoref{ex_dotp2}.}{fig:dotp2}{\begin{tikzpicture}[alt={On xy grid, rays u(3,1), v(–2,6), w(–4,3); arcs label alpha, beta, theta between the directions.},>=stealth]
\begin{axis}[width=1.16\marginparwidth,tick label style={font=\scriptsize},
axis y line=middle,axis x line=middle,name=myplot,axis on top,minor x tick num=4,
minor y tick num=1,ymin=-1.8,ymax=6.99,xmin=-5.5,xmax=5.5]
\draw [thick,->] (axis cs:0,0) --(axis cs: 3,1) node [above] {\scriptsize$\vec u$};
\draw [thick,->] (axis cs:0,0) --(axis cs: -2,6) node [above] {\scriptsize$\vec v$};
\draw [thick,->] (axis cs:0,0) --(axis cs: -4,3) node [above] {\scriptsize$\vec w$};
\draw [->] (axis cs: .949,.316) arc (18.4:108.4:12pt);
\draw [] (axis cs:0.680986, 1.33651) node {\scriptsize $\alpha$};
\draw [->] (axis cs: -0.474342, 1.42302) arc (108.4:155.1:12pt);
\draw [] (axis cs:-1.13929, 1.58257) node {\scriptsize $\beta$};
\draw [->] (axis cs: 2.3725, 0.79) arc (18.4:143.1:30pt);
\draw [] (axis cs:1.5, 2.59808) node {\scriptsize $\theta$};
\end{axis}
\node [right] at (myplot.right of origin) {\scriptsize $x$};
\node [above] at (myplot.above origin) {\scriptsize $y$};
\end{tikzpicture}}

\begin{example}[Using the dot product to find angles]\label{ex_dotp2}%
Let $\vec u = \bracket{3,1}$, $\vec v = \bracket{-2,6}$ and $\vec w = \bracket{-4,3}$, as shown in \autoref{fig:dotp2}. Find the angles $\alpha$, $\beta$ and $\theta$.
\solution
We start by computing the magnitude of each vector.
\[
\norm{\vec u} = \sqrt{10};\quad \norm{\vec v} = 2\sqrt{10};\quad \norm{\vec w} = 5.
\]
We now apply \autoref{thm:dot_product} to find the angles.
\begin{align*}
	\alpha &= \cos^{-1}\left(\frac{\vec u\cdot\vec v}{(\sqrt{10})(2\sqrt{10})}\right) \\
	&= \cos^{-1}(0) = \frac{\pi}2 = 90^\circ.\displaybreak[0]\\[10pt]
	\beta &= \cos^{-1}\left(\frac{\vec v\cdot\vec w}{(2\sqrt{10})(5)}\right) \\
	&= \cos^{-1}\left(\frac{26}{10\sqrt{10}}\right) \\
	&\approx 0.6055 \approx 34.7^\circ.\displaybreak[0]\\[10pt]
	\theta &= \cos^{-1}\left(\frac{\vec u\cdot\vec w}{(\sqrt{10})(5)}\right) \\
	&= \cos^{-1}\left(\frac{-9}{5\sqrt{10}}\right) \\
	&\approx 2.1763 \approx 124.7^\circ
\end{align*}
\end{example}

We see from our computation that $\alpha + \beta = \theta$, as indicated by \autoref{fig:dotp2}. While we knew this should be the case, it is nice to see that this non-intuitive formula indeed returns the results we expected.

We do a similar example next in the context of vectors in space.

\mtable{Vectors used in \autoref{ex_dotp3}.}{fig:dotp3}{%
\myincludeasythree{
3Droll=0,
3Dortho=0.0045,
3Dc2c=.89 .4 .23,
3Dcoo=10 50 46,
3Droo=200}{\marginparwidth}{Axes in space with u(3,1,0), v(2,–5,–3), w(–1,4,4); dashed boxes show xyz components.}{figures/figdotp3_3D}}

\begin{example}[Using the dot product to find angles]\label{ex_dotp3}%
Let $\vec u = \bracket{1,1,1}$, $\vec v = \bracket{-1,3,-2}$ and $\vec w = \bracket{-5,1,4}$, as illustrated in \autoref{fig:dotp3}. Find the angle between each pair of vectors.
\solution
\begin{enumerate}
	\item Between $\vec u$ and $\vec v$:\vspace{-\baselineskip}
	\begin{align*}
		\theta &= \cos^{-1}\left(\frac{\vec u\cdot\vec v}{\norm{\vec u}\norm{\vec v}}\right)\\
		&= \cos^{-1}\left(\frac{0}{\sqrt{3}\sqrt{14}}\right)\\
		&= \frac{\pi}2.
	\end{align*}
	\item	Between $\vec u$ and $\vec w$:\vspace{-\baselineskip}
	\begin{align*}
		\theta &= \cos^{-1}\left(\frac{\vec u\cdot\vec w}{\norm{\vec u}\norm{\vec w}}\right)\\
		&= \cos^{-1}\left(\frac{0}{\sqrt{3}\sqrt{42}}\right)\\
		&= \frac{\pi}2.
	\end{align*}
	\item	Between $\vec v$ and $\vec w$:\vspace{-\baselineskip}
	\begin{align*}
		\theta &= \cos^{-1}\left(\frac{\vec v\cdot\vec w}{\norm{\vec v}\norm{\vec w}}\right)\\
		&= \cos^{-1}\left(\frac{0}{\sqrt{14}\sqrt{42}}\right)\\
		&= \frac{\pi}2.
	\end{align*}
\end{enumerate}
While our work shows that each angle is $\pi/2$, i.e.,  $90^\circ$, none of these angles looks to be a right angle in \autoref{fig:dotp3}. Such is the case when drawing three-dimensional objects on the page.
\end{example}

All three angles between these vectors was $\pi/2$, or $90^\circ$. We know from geometry and everyday life that $90^\circ$ angles are ``nice'' for a variety of reasons, so it should seem significant that these angles are all $\pi/2$. Notice the common feature in each calculation (and also the calculation of $\alpha$ in \autoref{ex_dotp2}): the dot products of each pair of angles was 0. We use this as a basis for a definition of the term \textbf{orthogonal}, which is essentially synonymous to \emph{perpendicular}.

\begin{definition}[Orthogonal]\label{def:orthogonal}%
Vectors $\vec u$ and $\vec v$ are \textbf{orthogonal} if their dot product is 0.
\index{orthogonal}\index{vectors!orthogonal}
\end{definition}

\mnote{\textbf{Note:} The term \emph{perpendicular} originally referred to lines. As mathematics progressed, the concept of ``being at right angles to'' was applied to other objects, such as vectors and planes, and the term \emph{orthogonal} was introduced. It is especially used when discussing objects that are hard, or impossible, to visualize: two vectors in 5-dimensional space are orthogonal if their dot product is 0. It is not wrong to say they are \emph{perpendicular}, but common convention gives preference to the word \emph{orthogonal}.}

\begin{example}[Finding orthogonal vectors]\label{ex_dotp8}%
Let $\vec u = \bracket{3,5}$ and $\vec v = \bracket{1,2,3}$. 
\begin{enumerate}
	\item Find two vectors in $\mathbb{R}^2$ that are orthogonal to $\vec u$.
	\item	Find two non-parallel vectors in $\mathbb{R}^3$ that are orthogonal to $\vec v$.
\end{enumerate}
\solution
\begin{enumerate}
	\item Recall that a line perpendicular to a line with slope $m$ has slope $-1/m$, the ``opposite reciprocal slope.'' We can think of the slope of $\vec u$ as $5/3$, its ``rise over run.'' A vector orthogonal to $\vec u$ will have slope $-3/5$. There are many such choices, though all parallel:
	\[\bracket{-5,3}\quad \text{or} \quad\bracket{5,-3}\quad \text{or} \quad \bracket{-10,6}\quad \text{or} \quad \bracket{15,-9},\text{etc.}\]
	\item		There are infinite directions in space orthogonal to any given direction, so there are an infinite number of non-parallel vectors orthogonal to $\vec v$. Since there are so many, we have great leeway in finding some.
	
	One way is to arbitrarily pick values for the first two components, leaving the third unknown. For instance, let $\vec v_1 = \bracket{2,7,z}$. If $\vec v_1$ is to be orthogonal to $\vec v$, then $\vec v_1\cdot\vec v = 0$, so 
	\[2+14+3z=0 \quad \Rightarrow \quad z = \frac{-16}{3}.\]
	So $\vec v_1 = \bracket{2, 7, -16/3}$ is orthogonal to $\vec v$. We can apply a similar technique by leaving the first or second component unknown.
	
	Another method of finding a vector orthogonal to $\vec v$ mirrors what we did in part 1. Let $\vec v_2 = \bracket{-2,1,0}$. Here we switched the first two components of $\vec v$, changing the sign of one of them (similar to the ``opposite reciprocal'' concept before). Letting the third component be 0 effectively ignores the third component of $\vec v$, and it is easy to see that 
	\[\vec v_2\cdot\vec v = \bracket{-2,1,0}\cdot\bracket{1,2,3}= 0.\]
	Clearly $\vec v_1$ and $\vec v_2$ are not parallel.
\end{enumerate}
\end{example}

\mtable{Developing the construction of the \emph{orthogonal projection}.}{fig:dotpproj}{%
\begin{tikzpicture}[alt={Triangle from origin: u to apex, dashed height meets v at right angle to build projection.},>=stealth]
	\draw [thick,->] (0,0) -- (4,2) node [right] {\scriptsize $\vec v$};
	\draw [thick,->] (0,0) -- (1,3) node [above ] {\scriptsize $\vec u$};
	\draw [dotted,thick] (1,3) -- (2,1);
	\draw (1.82111, 0.910557) -- ({1.73167, 1.08944})--(1.91056, 1.17889);
	\draw (.4,.2) arc (26.5:71.6:.45);
	\draw [rotate=49] (.6,0) node {\scriptsize $\theta$};
\end{tikzpicture}
\\(a)\\[15pt]
\begin{tikzpicture}[alt={Triangle from origin: height labelled z, foot labelled w; grey v emphasises projection line.},>=stealth]
	\draw [thick,->] (2,1) -- (4,2) node [right] {\scriptsize $\vec v$};
	\draw [thick,->] (0,0) -- (1,3) node [above ] {\scriptsize $\vec u$};
	\draw [thick,->,gray] (0,0) -- (2,1) node [below,black] {\scriptsize $\vec w$};
	\draw [<-,thick] (1,3) -- (2,1) node [right,pos=.5] {\scriptsize $\vec z$};
	\draw (1.82111, 0.910557) -- ({1.73167, 1.08944})--(1.91056, 1.17889);
	\draw (.4,.2) arc (26.5:71.6:.45);
	\draw [rotate=49] (.6,0) node {\scriptsize $\theta$};
\end{tikzpicture}
\\(b)}

An important construction is illustrated in \autoref{fig:dotpproj}, where vectors $\vec u$ and $\vec v$ are sketched. In part (a), a dotted line is drawn from the tip of $\vec u$ to the line containing $\vec v$, where the dotted line is orthogonal to $\vec v$. In part (b), the dotted line is replaced with the vector $\vec z$ and  $\vec w$ is formed, parallel to $\vec v$. It is clear by the diagram that $\vec u = \vec w+\vec z$. What is important about this construction is this: $\vec u$ is \emph{decomposed} as the sum of two vectors, one of which is parallel to $\vec v$ and one that is perpendicular to $\vec v$. It is hard to overstate the importance of this construction (as we'll see in upcoming examples). 

The vectors $\vec w$, $\vec z$ and $\vec u$ as shown in \autoref{fig:dotpproj} (b) form a right triangle, where the angle between $\vec v$ and $\vec u$ is labeled $\theta$. We can find $\vec w$ in terms of $\vec v$ and $\vec u$.

Using trigonometry, we can state that 
\begin{equation}
\norm{\vec w} = \norm{\vec u}\cos \theta. \label{eq:proj1}%
\end{equation}

We also know that $\vec w$ is parallel to to $\vec v$\,; that is, the direction of $\vec w$ is the direction of $\vec v$, described by the unit vector $\frac{1}{\norm{\vec v}}\vec v$. The vector $\vec w$ is the vector in the direction $\frac{1}{\norm{\vec v}}\vec v$ with magnitude $\norm{\vec u}\cos \theta$:
\begin{align*}
\vec w &= \Bigl(\norm{\vec u}\cos\theta \Bigr)\frac{1}{\norm{\vec v}}\vec v.
\intertext{Replace $\cos\theta$ using \autoref{thm:dot_product}:}
			&= \left(\norm{\vec u}\frac{\vec u\cdot\vec v}{\norm{\vec u}\norm{\vec v}}\right)\frac{1}{\norm{\vec v}} \vec v\\ 
			&= \frac{\vec u\cdot\vec v}{\norm{\vec v}^2}\vec v.
			\intertext{Now apply \autoref{thm:dot_product_properties}.}
			&= \frac{\vec u\cdot\vec v}{\vec v\cdot\vec v}\vec v.
\end{align*}

Since this construction is so important, it is given a special name.

\begin{definition}[Orthogonal Projection]\label{def:orthogonal_projection}%
Let $\vec u$ and $\vec v$ be given, where $\vec v\neq\vec 0$. The \textbf{orthogonal projection of $\vec u$ onto $\vec v$}, denoted $\proj uv$, is 
\index{orthogonal projection}\index{vectors!orthogonal projection}
\[\proj uv = \frac{\vec u\cdot\vec v}{\vec v\cdot\vec v}\vec v.\]
\end{definition}

\mtable[-1in]{Graphing the vectors used in \autoref{ex_dotp4}.}{fig:dotp4}{%
\begin{tikzpicture}[alt={2-D plot of u, v and proj_v; right-angle tick at foot; axes graduated.},>=stealth]
\begin{axis}[width=1.16\marginparwidth,tick label style={font=\scriptsize},
axis y line=middle,axis x line=middle,name=myplot,axis on top,xtick={-3,-2,-1,1,2,3},
ytick={1,2,-1,-2},ymin=-2.8,ymax=2.8,xmin=-2.9,xmax=3.9]
\draw [thick,->] (axis cs:0,0) --(axis cs: -2,1) node [above] {\scriptsize$\vec u$};
\draw [thick,->] (axis cs:0,0) --(axis cs: 3,1) node [above] {\scriptsize$\vec v$};
\draw [thick,->,gray] (axis cs:0,0) --(axis cs: -1.5,-.5)
 node [below,black] {\scriptsize$\text{proj}_{\, \vec v}\, \vec u$};
\draw [dotted,thick] (axis cs: -2,1) -- (axis cs:-1.5,-.5);
\end{axis}
\node [right] at (myplot.right of origin) {\scriptsize $x$};
\node [above] at (myplot.above origin) {\scriptsize $y$};
\end{tikzpicture}
\\(a)\\[15pt]
\myincludeasythree{
3Droll=0,
3Dortho=0.0046,
3Dc2c=.9 .12 .42,
3Dcoo=0 50 30,
3Droo=250}{.8\marginparwidth}{3-D view: blue w, red proj_v w, black x; dashed cubes outline component lengths.}{figures/figdotp4b_3D}
\\(b)
\iftoggle{in_threeD}{}{%
\\[15pt]
\myincludegraphics[alt={3-D view: blue w, red proj_v w, black x; dashed cubes outline component lengths.},width=.8\marginparwidth]{figures/figdotp4c_3D} % myinc g 3D
\\(c)}% end iftoggle
}% end mtable

\begin{example}[Computing the orthogonal projection]\label{ex_dotp4}%
\mbox{}\\[-2\baselineskip]\parbox[t]{\linewidth}{%
\begin{enumerate}
	\item Let $\vec u=\bracket{-2,1}$ and $\vec v=\bracket{3,1}$. Find $\proj uv$, and sketch all three vectors with initial points at the origin.
	\item	Let $\vec w =\bracket{2,1,3}$ and $\vec x =\bracket{1,1,1}$. Find $\proj wx$, and sketch all three vectors with initial points at the origin.
\end{enumerate}}\vspace{0pt}
\solution
\begin{enumerate}
	\item Applying \autoref{def:orthogonal_projection}, we have
	\begin{align*}
		\proj uv
		&= \frac{\vec u\cdot\vec v}{\vec v\cdot\vec v}\vec v \\
		&= \frac{-5}{10}\bracket{3,1}\\
		&=\bracket{-\frac32,-\frac12}.
	\end{align*}
	Vectors $\vec u$, $\vec v$ and $\proj uv$ are sketched in \autoref{fig:dotp4}(a). Note how the projection is parallel to $\vec v$; that is, it lies on the same line through the origin as $\vec v$, although it points in the opposite direction. That is because the angle between $\vec u$ and $\vec v$ is obtuse (i.e., greater than $90^\circ$).
	
	\item	Apply the definition:
	\begin{align*}
		\proj wx
		&= \frac{\vec w\cdot\vec x}{\vec x\cdot\vec x}\vec x \\
		&= \frac{6}{3}\bracket{1,1,1}\\
		&=\bracket{2,2,2}.
	\end{align*}
	\iftoggle{in_threeD}{%
	 These vectors are sketched in \autoref{fig:dotp4}(b).%
	}
\end{example}

Consider \autoref{fig:dotpprojc} where the concept of the orthogonal projection is again illustrated. It is clear that
\begin{equation}
\vec u = \proj uv + \vec z.
\label{eq:orthogproj}
\end{equation}
As we know what $\vec u$ and $\proj uv$ are, we can solve for $\vec z$ and state that
\[\vec z = \vec u - \proj uv.\]%
%
\mtable[-7\baselineskip]{Illustrating the orthogonal projection.}{fig:dotpprojc}{\begin{tikzpicture}[alt={Vectors u, v with right-angle foot; z fills gap so u = proj_v u + z.},>=stealth]
 \draw [thick,->] (2,1) -- (4,2) node [below] {\scriptsize $\vec v$};
 \draw [thick,->] (0,0) -- (1,3) node [above ] {\scriptsize $\vec u$};
 \draw [thick,->,gray] (0,0) -- (2,1)
  node [pos=.6,below,black] {\scriptsize $\text{proj}_{\, \vec v}\, \vec u$};
 \draw [<-,thick] (1,3) -- (2,1) node [right,pos=.5] {\scriptsize $\vec z$};
 \draw (1.82111, 0.910557) -- ({1.73167, 1.08944})--(1.91056, 1.17889);
\end{tikzpicture}}%
%
This leads us to rewrite \autoeqref{eq:orthogproj} in a seemingly silly way:
\[\vec u = \proj uv + (\vec u - \proj uv).\]
This is not nonsense, as pointed out in the following Key Idea. (Notation note: the expression ``$\parallel \vec y$\,'' means ``is parallel to $\vec y$.'' We can use this notation to state ``$\vec x\parallel\vec y$\,'' which means ``$\vec x$ is parallel to $\vec y$.'' The expression ``$\perp \vec y$\,'' means ``is orthogonal to $\vec y$,'' and is used similarly.)

\begin{keyidea}[Orthogonal Decomposition of Vectors]\label{idea:orthog_proj}%
Let $\vec u$ and $\vec v$ be given. Then $\vec u$ can be written as the sum of two vectors, one of which is parallel to $\vec v$, and one of which is orthogonal to $\vec v$:
\index{orthogonal decomposition of vectors}\index{orthogonal!decomposition}\index{vectors!orthogonal decomposition}
\[\vec u = \underbrace{\proj uv}_{\parallel\ \vec v}\ +\  (\underbrace{\vec u-\proj uv}_{\perp\ \vec v}).\]
\end{keyidea}

We illustrate the use of this equality in the following example.

\begin{example}[Orthogonal decomposition of vectors]\label{ex_dotp5}%
\mbox{}\\[-2\baselineskip]\parbox[t]{\linewidth}{%
\begin{enumerate}
	\item Let $\vec u =\bracket{-2,1}$ and $\vec v =\bracket{3,1}$ as in \autoref{ex_dotp4}. Decompose $\vec u$ as the sum of a vector parallel to $\vec v$ and a vector orthogonal to $\vec v$.
	\item	Let $\vec w =\bracket{2,1,3}$ and $\vec x  =\bracket{1,1,1}$ as in \autoref{ex_dotp4}. Decompose $\vec w$ as the sum of a vector parallel to $\vec x$ and a vector orthogonal to $\vec x$.
\end{enumerate}}\vspace{0pt}
\solution
\begin{enumerate}
	\item In \autoref{ex_dotp4}, we found that $\proj uv =\bracket{-1.5,-0.5}$. Let
	\[\vec z = \vec u - \proj uv =\bracket{-2,1}-\bracket{-1.5,-0.5}=\bracket{-0.5, 1.5}.\]
	Is $\vec z$ orthogonal to $\vec v$\,? (I.e, is $\vec z \perp\vec v$\ ?) We check for orthogonality with the dot product:
	\[\vec z\cdot\vec v =\bracket{-0.5,1.5}\cdot\bracket{3,1}=0.\]
	Since the dot product is 0, we know $\vec z \perp \vec v$. Thus:
	\begin{align*}
	\vec u &= \proj uv\ +\ (\vec u - \proj uv) \\
	\bracket{-2,1}&= \underbrace{\bracket{-1.5,-0.5}}_{\parallel\ \vec v}\ +\ \underbrace{\bracket{-0.5,1.5}}_{\perp \ \vec v}.
	\end{align*}
	
	\item	We found in \autoref{ex_dotp4} that $\proj wx =\bracket{2,2,2}$. Applying the \autoref{idea:orthog_proj}, we have:
	\[\vec z = \vec w - \proj wx =\bracket{2,1,3}-\bracket{2,2,2}=\bracket{0,-1,1}.\] We check to see if $\vec z \perp \vec x$:
	\[\vec z\cdot\vec x =\bracket{0,-1,1}\cdot\bracket{1,1,1}= 0.\]
	Since the dot product is 0, we know the two vectors are orthogonal.
	We now write $\vec w$ as the sum of two vectors, one parallel and one orthogonal to $\vec x$:
	\begin{align*}
	\vec w &= \proj wx\ +\ (\vec w - \proj wx) \\
	\bracket{2,1,3}&= \underbrace{\bracket{2,2,2}}_{\parallel\ \vec x}\ +\ \underbrace{\bracket{0,-1,1}}_{\perp \ \vec x}
	\end{align*}
\end{enumerate}
\end{example}

We give an example of where this decomposition is useful.

\begin{example}[Orthogonally decomposing a force vector]\label{ex_dotp6}%
Consider \autoref{fig:dotp6}(a), showing a box weighing 50 lb on a ramp that rises 5 ft over a span of 20 ft. Find the components of force, and their magnitudes, acting on the box (as sketched in part (b) of the figure):
%
\mtable[-3\baselineskip]{Sketching the ramp and box in \autoref{ex_dotp6}. Note: \emph{The vectors are not drawn to scale.}}{fig:dotp6}{%
\begin{tikzpicture}[alt={20 by 5 ramp; weight g acts vertically on box resting on the incline.},>=stealth,scale=.2]
	\draw [thick,->] (20,5) -- node [right,pos=.5] {\scriptsize $5$} (20,0)
	 -- node [below,pos=.6] {\scriptsize 20}(0,0)
	 -- (20,5) node [above] {\scriptsize $\vec r$};
	\draw [thick] (10,2.5) -- (9.25,5.5) -- (12.25,6.25) -- (13,3.25);
	\draw [thick,->] (11.125,4.375) -- (11.125,-3.625)
	 node [below] {\scriptsize $\vec g$};
\end{tikzpicture}
\\(a)\\[15pt]
\begin{tikzpicture}[alt={20 by 5 ramp; g split into grey proj_r g along ramp plus orthogonal residual z.},>=stealth,scale=.2]
	\draw [thick,->] (20,5) -- node [right,pos=.5] {\scriptsize $5$} (20,0)
	 -- node [below,pos=.6] {\scriptsize 20}(0,0)
	 -- (20,5) node [above] {\scriptsize $\vec r$};
	\draw [thick] (10,2.5) -- (9.25,5.5) -- (12.25,6.25) -- (13,3.25);
	\draw [thick,->] (11.125,4.375) -- (11.125,-3.625)
	 node [below] {\scriptsize $\vec g$};
	\draw [gray,thick,->] (11.125,4.375) -- (13,-3.15)
	 node [right,pos=.4,black] {\scriptsize $\vec z$};
	\draw [gray,thick,->] (13,-3.15) -- (11.125,-3.625)
	 node [shift={(5pt,-5pt)} ,black,pos=0] {\scriptsize $\proj gr$};
\end{tikzpicture}
\\(b)}
%
\begin{enumerate}
	\item in the direction of the ramp, and
	\item	orthogonal to the ramp.
\end{enumerate}
\solution
As the ramp rises 5 ft over a horizontal distance of 20 ft, we can represent the direction of the ramp with the vector $\vec r=\bracket{20,5}$. Gravity pulls down with a force of 50 lb, which we represent with $\vec g =\bracket{0,-50}$. 
\begin{enumerate}
	\item To find the force of gravity in the direction of the ramp, we compute the projection $\proj gr$:\vspace{-.5\baselineskip}
	\begin{align*}
	\proj gr &= \frac{\vec g\cdot\vec r}{\vec r\cdot\vec r}\vec r\\
	&=  \frac{-250}{425}\bracket{20,5}\\
	&=\bracket{-\frac{200}{17},-\frac{50}{17}}%\approx\bracket{-11.76,-2.94}
	.
	\end{align*}
	The magnitude of $\proj gr$ is $\norm{\proj gr} = 50/\sqrt{17} \approx 12.13\text{ lb}$. Though the box weighs 50 lb, a force of about 12 lb is enough to keep the box from sliding down the ramp.
	
	\item		To find the component $\vec z$ of gravity orthogonal to the ramp, we use \autoref{idea:orthog_proj}.\vspace{-.5\baselineskip}
	\begin{align*}
	\vec z &= \vec g - \proj gr \\
	&=\bracket{\frac{200}{17},-\frac{800}{17}}%\approx\bracket{11.76,-47.06}
	.
	\end{align*}
	The magnitude of this force is $\norm{\vec z}=200/\sqrt{17}%\approx 48.51
	$ lb. In physics and engineering, knowing this force is important when computing things like static frictional force. (For instance, we could easily compute if the static frictional force alone was enough to keep the box from sliding down the ramp.)
\end{enumerate}
\end{example}

\subsection{Application to Work}

In physics, the application of a force $F$ to move an object in a straight line a distance $d$ produces \emph{work}; the amount of work $W$ is $W=Fd$, (where $F$ is in the direction of travel). The orthogonal projection allows us to compute work when the force is not in the direction of travel.

Consider \autoref{fig:dotpwork}, where a force $\vec F$ is being applied to an object moving in the direction of $\vec d$. (The distance the object travels is the magnitude of $\vec d$.) The work done is the amount of force in the direction of $\vec d$, $\norm{\proj Fd}$, times $\vnorm d$:

\mtable{Finding work when the force and direction of travel are given as vectors.}{fig:dotpwork}{\begin{tikzpicture}[alt={Box pulled by force F; horizontal travel d; grey proj_d F shows effective work.},>=stealth,scale=.8]
 \draw [thick] (0,0) rectangle (2,2);
 \draw [thick,->] (0,-.25) -- (5,-.25) node [right] {\scriptsize $\vec d$};
 \draw [thick,->] (1,1) -- (3,2) node [shift={(5pt,2.5pt)}] {\scriptsize $\vec F$};
 \draw [thick,gray,->] (1,1) -- (3,1) node [right,black] {\scriptsize $\proj Fd$};
\end{tikzpicture}}

\begin{align*}
\norm{\proj Fd}\cdot\vnorm d
 &= \norm{\frac{\vec F\cdot\vec d}{\vec d\cdot\vec d}\vec d}\cdot \vnorm d \\
		&= \abs{\frac{\vec F\cdot\vec d}{\vnorm d^2}}\cdot \vnorm d\cdot\vnorm d\\
		&= \frac{\abs{\vec F\cdot\vec d}}{\vnorm d^2}\vnorm d^2\\
		&= \abs{\vec F\cdot\vec d}.
\end{align*}

The expression $\vec F\cdot\vec d$ will be positive if the angle between $\vec F$ and $\vec d$ is acute; when the angle is obtuse (hence $\vec F\cdot\vec d$ is negative), the force is causing motion in the opposite direction of $\vec d$, resulting in ``negative work.'' We want to capture this sign, so we drop the absolute value and find that $W = \vec F\cdot\vec d$.

\begin{definition}[Work]\label{def:work}%
Let $\vec F$ be a constant force that moves an object in a straight line from point $P$ to point $Q$. Let $\vec d = \vv{PQ}$. The \textbf{work} $W$ done by $\vec F$ along $\vec d$ is $W = \vec F\cdot\vec d$.\index{work}
\end{definition}

\mtable{Computing work when sliding a box up a ramp in \autoref{ex_dotp7}.}{fig:dotp7}{\begin{tikzpicture}[alt={Force F 30 deg above ramp of rise 3 run 15; dashed component along slope marked.},>=stealth,scale=.3]
 \draw [thick] (0,0) -- node [below,pos=.5] {\scriptsize 15} (15,0)
  -- node [right, pos=.5] {\scriptsize 3} (15,3) -- cycle;
 \draw [thick] (5,1) -- (4.55,3.25) -- (6.8,3.7) -- (7.25,1.45);
 \begin{scope}[,shift={(5.9,2.35)}]
  \draw [thick,rotate=30,->] (0,0) -- (5,0) node [right] {\scriptsize $\vec F$};
  \draw [thick,dashed] (-2,0) -- (4,0);
  \draw (2,0) arc (0:30:2);
  \draw [rotate=15] (3,0) node {\scriptsize $30^\circ$};
 \end{scope}
\end{tikzpicture}}

\begin{example}[Computing work]\label{ex_dotp7}%
A man slides a box along a ramp that rises 3 ft over a distance of 15 ft by applying 50 lb of force as shown in \autoref{fig:dotp7}. Compute the work done.
\solution
The figure indicates that the force applied makes a $30^\circ$ angle with the horizontal, so $\vec F = 50\bracket{\cos 30^\circ,\sin 30^\circ}\bracket{25\sqrt3,25}%\approx\bracket{43.3,25}
$. The ramp is represented by $\vec d  =\bracket{15,3}$. The work done is simply
\[\vec F\cdot\vec d = \bracket{25\sqrt3,25}\cdot\bracket{15,3}=375\sqrt3+75%\approx 724.5
\text{ ft-lb}.\]

Note how we did not actually compute the distance the object traveled, nor the magnitude of the force in the direction of travel; this is all inherently computed by the dot product.
\end{example}

The dot product is a powerful way of evaluating computations that depend on angles without actually using angles. The next section explores another product on vectors, the \emph{cross product.} Once again, angles play an important role, though in a much different way.

\printexercises{exercises/10-03-exercises}
