\section{The Calculus of Motion}\label{sec:vvf_motion}

A common use of vector--valued functions is to describe the motion of an object in the plane or in space. A \textbf{position function} $\vec r(t)$ gives the position of an object at \textbf{time} $t$. This section explores how derivatives and integrals are used to study the motion described by such a function.

\definition{def:vvf_motion}{Velocity, Speed and Acceleration}
{Let $\vec r(t)$ be a position function in $\mathbb{R}^2$ or $\mathbb{R}^3$.
\index{velocity}\index{speed}\index{acceleration}\index{vector--valued function!describing motion}
\begin{enumerate}
	\item \textbf{Velocity}, denoted $\vec v(t)$, is the instantaneous rate of position change; that is, $\vec v(t) = \vrp(t)$.
	\item	\textbf{Speed} is the magnitude of velocity, $\norm{\vec v(t)}$.
	\item	\textbf{Acceleration}, denoted $\vec a(t)$, is the instantaneous rate of velocity change; that is, $\vec a(t) = \vec v\,'(t) = \vrp'(t)$.
\end{enumerate}}

\youtubeVideo{gD2R4Jqw6dQ}{Example of Position, Velocity and Acceleration in Three Space}

\example{ex_motion1}{Finding velocity and acceleration}{An object is moving with position function $\vec r(t) =\bracket{t^2-t,t^2+t}$, $-3\leq t\leq 3$, where distances are measured in feet and time is measured in seconds.
\begin{enumerate}
	\item Find \vvt\  and \vat.
	\item	Sketch \vrt; plot $\vec v(-1)$, $\vec a(-1)$, $\vec v(1)$ and $\vec a(1)$, each with their initial point at their corresponding point on the graph of $\vrt$.
	\item	When is the object's speed minimized?
\end{enumerate}}
{\begin{enumerate}
	\item Taking derivatives, we find
	\[\vvt = \vrp(t) =\bracket{2t-1,2t+1}\quad \text{and} \quad \vat = \vrp'(t) =\bracket{2,2}.\]
	Note that acceleration is constant.

	\item		$\vec v(-1) =\bracket{-3,-1}$,\ \ $\vec a(-1) =\bracket{2,2}$; \quad $\vec v(1) =\bracket{1,3}$,\ \ $\vec a(1) =\bracket{2,2}$. These are plotted with \vrt\ in \autoref{fig:motion1}(a).
	
\mtable{Graphing the position, velocity and acceleration of an object in \autoref{ex_motion1}.}{fig:motion1}{%
\begin{tikzpicture}[>=stealth]
\begin{axis}[width=1.16\marginparwidth,tick label style={font=\scriptsize},
axis y line=middle,axis x line=middle,name=myplot,
ymin=-1.9,ymax=13,xmin=-1.9,xmax=13]
\addplot [thick,draw={\colorone}, smooth,domain=-3:3,samples=20] ({x^2-x},{x^2+x});
\filldraw (axis cs:2,0) circle (1.5pt);
\filldraw (axis cs:0,2) circle (1.5pt);
\draw [thick,->,draw={\colortwo}] (axis cs:2,0)--(axis cs:-1,-1);
\draw [thick,->,draw={\colortwo}] (axis cs:0,2)--(axis cs:1,5);
\draw [thick,->] (axis cs:2,0)--(axis cs:4,2);
\draw [thick,->] (axis cs:0,2)--(axis cs:2,4);
\draw [thick,->,draw={\colorone}] (axis cs:8.75,3.75) -- (axis cs: 8.69,3.71);
\end{axis}
\node [right] at (myplot.right of origin) {\scriptsize $x$};
\node [above] at (myplot.above origin) {\scriptsize $y$};
\end{tikzpicture}
\\(a)\\[10pt]
\begin{tikzpicture}[>=stealth]
\begin{axis}[width=1.16\marginparwidth,tick label style={font=\scriptsize},
axis y line=middle,axis x line=middle,name=myplot,
ymin=-1.9,ymax=13,xmin=-1.9,xmax=13]
\addplot [thick,draw={\colorone}, smooth,domain=-3:3,samples=20] ({x^2-x},{x^2+x});
\filldraw (axis cs:6,2) circle (1.5pt);
\filldraw (axis cs:2,0) circle (1.5pt);
\filldraw (axis cs:0,0) circle (1.5pt);
\filldraw (axis cs:0,2) circle (1.5pt);
\filldraw (axis cs:2,6) circle (1.5pt);
\filldraw (axis cs:12,6) circle (1.5pt);
\filldraw (axis cs:6,12) circle (1.5pt);
\draw [thick,->,draw={\colortwo}] (axis cs:6,2)--(axis cs:1,-1);
\draw [thick,->,draw={\colortwo}] (axis cs:2,0)--(axis cs:-1,-1);
\draw [thick,->,draw={\colortwo}] (axis cs:0,0)--(axis cs:-1,1);
\draw [thick,->,draw={\colortwo}] (axis cs:0,2)--(axis cs:1,5);
\draw [thick,->,draw={\colortwo}] (axis cs:2,6)--(axis cs:5,11);
\draw [thick,->] (axis cs:6,2)--(axis cs:8,4);
\draw [thick,->] (axis cs:2,0)--(axis cs:4,2);
\draw [thick,->] (axis cs:0,0)--(axis cs:2,2);
\draw [thick,->] (axis cs:0,2)--(axis cs:2,4);
\draw [thick,->] (axis cs:2,6)--(axis cs:4,8);
\draw [thick,->,draw={\colorone}] (axis cs:8.75,3.75) -- (axis cs: 8.69,3.71);
\end{axis}
\node [right] at (myplot.right of origin) {\scriptsize $x$};
\node [above] at (myplot.above origin) {\scriptsize $y$};
\end{tikzpicture}
\\(b)}

	We can think of acceleration as ``pulling'' the velocity vector in a certain direction. At $t=-1$, the velocity vector points down and to the left; at $t=1$, the velocity vector has been pulled in the $\bracket{2,2}$ direction and is now pointing up and to the right. In \autoref{fig:motion1}(b) we plot more velocity/acceleration vectors, making more clear the effect acceleration has on velocity.
	
	Since $\vat$ is constant in this example, as $t$ grows large \vvt\ becomes almost parallel to \vat. For instance, when $t=10$, $\vec v(10) =\bracket{19,21}$, which is nearly parallel to $\bracket{2,2}$.
	
	\item		The object's speed is given by 
	\[\norm{\vvt} = \sqrt{(2t-1)^2+(2t+1)^2} =\sqrt{8t^2+2}.\] To find the minimal speed, we could apply calculus techniques (such as set the derivative equal to 0 and solve for $t$, etc.) but we can find it by inspection. Inside the square root we have a quadratic which is minimized when $t=0$. Thus the speed is minimized at $t=0$, with a speed of $\sqrt{2}$ ft/s.
	
	The graph in \autoref{fig:motion1}(b) also implies speed is minimized here. The filled dots on the graph are located at integer values of $t$ between $-3$ and 3. Dots that are far apart imply the object traveled a far distance in 1 second, indicating high speed; dots that are close together imply the object did not travel far in 1 second, indicating a low speed. The dots are closest together near $t=0$, implying the speed is minimized near that value.\eoehere
\end{enumerate}}

\example{ex_motion2}{Analyzing Motion}{Two objects follow an identical path at different rates on $[-1,1]$. The position function for Object 1 is $\vec r_1(t) =\bracket{t, t^2}$; the position function for Object 2 is $\vec r_2(t) =\bracket{t^3, t^6}$, where distances are measured in feet and time is measured in seconds. Compare the velocity, speed and acceleration of the two objects on the path.}
{We begin by computing the velocity and acceleration function for each object:
\begin{align*}
\vec v_1(t) &=\bracket{1,2t}& \vec v_2(t) &=\bracket{3t^2,6t^5}\\
\vec a_1(t) &=\bracket{0,2}& \vec a_2(t) &=\bracket{6t,30t^4}
\end{align*}
We immediately see that Object 1 has constant acceleration, whereas Object 2 does not. 

At $t=-1$, we have $\vec v_1(-1) =\bracket{1,-2}$ and $\vec v_2(-1) =\bracket{3,-6}$; the velocity of Object 2 is three times that of Object 1 and so it follows that the speed of Object 2 is three times that of Object 1 ($3\sqrt{5}$ ft/s compared to $\sqrt{5}$ ft/s.)

\mtable{Plotting velocity and acceleration vectors for Object 1 in \autoref{ex_motion2}.}{fig:motion2a}{\begin{tikzpicture}[>=stealth]
\begin{axis}[width=1.16\marginparwidth,tick label style={font=\scriptsize},
axis y line=middle,axis x line=middle,name=myplot,
ymin=-1.1,ymax=3.1,xmin=-2.1,xmax=2.1]
\addplot [thick,draw={\colorone}, smooth,domain=-1:1,samples=20] ({x},{x^2});
\filldraw (axis cs:-1.,1.) circle (1.5pt);
\filldraw (axis cs:-0.5,0.25) circle (1.5pt);
\filldraw (axis cs:0.,0.) circle (1.5pt);
\filldraw (axis cs:0.5,0.25) circle (1.5pt);
\filldraw (axis cs:1.,1.) circle (1.5pt);
\draw [thick,->,draw={\colortwo}] (axis cs:-1.,1.)--(axis cs:0.,-1.);
\draw [thick,->,draw={\colortwo}] (axis cs:-0.5,0.25)--(axis cs:0.5,-0.75);
\draw [thick,->,draw={\colortwo}] (axis cs:0.,0.)--(axis cs:1.,0.);
\draw [thick,->,draw={\colortwo}] (axis cs:0.5,0.25)--(axis cs:1.5,1.25);
\draw [thick,->,draw={\colortwo}] (axis cs:1.,1.)--(axis cs:2.,3.);
\draw [thick,->] (axis cs:-1.,1.)--(axis cs:-1.,3.);
\draw [thick,->] (axis cs:-0.5,0.25)--(axis cs:-0.5,2.25);
\draw [thick,->] (axis cs:0.,0.)--(axis cs:0.,2.);
\draw [thick,->] (axis cs:0.5,0.25)--(axis cs:0.5,2.25);
\draw [thick,->] (axis cs:1.,1.)--(axis cs:1.,3.);
\draw [thick,->,draw={\colorone}] (axis cs:-.9,.81) -- (axis cs: -.89,.7921);
\end{axis}
\node [right] at (myplot.right of origin) {\scriptsize $x$};
\node [above] at (myplot.above origin) {\scriptsize $y$};
\end{tikzpicture}}

At $t=0$, the velocity of Object 1 is $\vec v(1) =\bracket{1,0}$ and the velocity of Object 2 is $\vec 0$. This tells us that Object 2 comes to a complete stop at $t=0$. 

In \autoref{fig:motion2a}, we see the velocity and acceleration vectors for Object 1 plotted for $t=-1, -1/2, 0, 1/2$ and $t=1$. Note again how the constant acceleration vector seems to ``pull'' the velocity vector from pointing down, right to up, right. We could plot the analogous picture for Object 2, but the velocity and acceleration vectors are rather large ($\vec a_2(-1) =\bracket{-6,30}$). 

Instead, we simply plot the locations of Object 1 and 2  on intervals of $1/5^{\text{th}}$ of a second, shown in \autoref{fig:motion2b}(a) and (b). Note how the $x$-values of Object 1 increase at a steady rate. This is because the $x$-component of $\vec a(t)$ is 0; there is no acceleration in the $x$-component. The dots are not evenly spaced; the object is moving faster near $t=-1$ and $t=1$ than near $t=0$.
\mtable{Comparing the positions of Objects 1 and 2 in \autoref{ex_motion2}.}{fig:motion2b}{%
\begin{tikzpicture}[>=stealth]
\begin{axis}[width=1.16\marginparwidth,tick label style={font=\scriptsize},
axis y line=middle,axis x line=middle,name=myplot,
ymin=-.1,ymax=1.1,xmin=-1.1,xmax=1.1]
\addplot [thick,draw={\colorone}, smooth,domain=-1:1,samples=20] {x^2};
\foreach \x in {-1,-.8,...,1} {
 \edef\temp{\noexpand\filldraw(axis cs:\x,\x*\x)circle(1.5pt);}\temp
}
\draw [thick,->,draw={\colorone}] (axis cs:-.9,.81) -- (axis cs: -.89,.7921);
\draw (axis cs: .5,.6) node {\scriptsize $\vec r_1(t)$};
\end{axis}
\node [right] at (myplot.right of origin) {\scriptsize $x$};
\node [above] at (myplot.above origin) {\scriptsize $y$};
\end{tikzpicture}
\\(a)\\[10pt]
\begin{tikzpicture}[>=stealth]
\begin{axis}[width=1.16\marginparwidth,tick label style={font=\scriptsize},
axis y line=middle,axis x line=middle,name=myplot,
ymin=-.1,ymax=1.1,xmin=-1.1,xmax=1.1]
\addplot [thick,draw={\colorone}, smooth,domain=-1:1,samples=20] {x^2};
\foreach \x in {-1,-.8,...,1} {
 \edef\temp{\noexpand\filldraw(axis cs:\x^3,{(\x)^6})circle(1.5pt);}\temp
}
\draw [thick,->,draw={\colorone}] (axis cs:-.9,.81) -- (axis cs: -.89,.7921);
\draw (axis cs: .5,.6) node {\scriptsize $\vec r_2(t)$};
\end{axis}
\node [right] at (myplot.right of origin) {\scriptsize $x$};
\node [above] at (myplot.above origin) {\scriptsize $y$};
\end{tikzpicture}
\\(b)}

In part (b) of the Figure, we see the points plotted for Object 2. Note the large change in position from $t=-1$ to $t=-0.8$; the object starts moving very quickly. However, it slows considerably at it approaches the origin, and comes to a complete stop at $t=0$. While it looks like there are 3 points near the origin, there are in reality 5 points there.

Since the objects begin and end at the same location, they have the same displacement. Since they begin and end at the same time, with the same displacement, they have the same average rate of change (i.e, they have the same average velocity). Since they follow the same path, they have the same distance traveled. Even though these three measurements are the same, the objects obviously travel the path in very different ways.}

\example{ex_motion3}{Analyzing the motion of a whirling ball on a string}{A young boy whirls a ball, attached to a string, above his head in a counter-clockwise circle. The ball follows a circular path and makes 2 revolutions per second. The string has length 2ft.
\begin{enumerate}
	\item Find the position function $\vec r(t)$ that describes this situation.
	\item	Find the acceleration of the ball and give a physical interpretation of it.
	\item	A tree stands 10ft in front of the boy. At what $t$-values should the boy release the string so that the ball hits the tree?
\end{enumerate}}
{\begin{enumerate}
	\item The ball whirls in a  circle. Since the string is 2ft long, the radius of the circle is 2. The position function $\vrt =\bracket{2\cos t, 2\sin t}$ describes a circle with radius 2, centered at the origin, but makes a full revolution every $2\pi$ seconds, not two revolutions per second. We modify the period of the trigonometric functions to be 1/2 by multiplying $t$ by $4\pi$. The final position function is thus \[\vrt =\bracket{2\cos (4\pi t), 2\sin (4\pi t)}.\]
	(Plot this for $0\leq t\leq 1/2$ to verify that one revolution is made in 1/2 a second.)
	
	\item		To find $\vat$, we differentiate $\vrt$ twice.
	\begin{align*}
	\vvt = \vrp(t) &=\bracket{-8\pi \sin (4\pi t), 8\pi \cos (4\pi t)}\\
	\vat =\vrp'(t) &=\bracket{-32\pi^2 \cos (4\pi t), -32\pi^2 \sin (4\pi t)}\\
				&= -32\pi^2\bracket{\cos (4\pi t), \sin (4\pi t)}.
	\end{align*}
	Note how $\vat$ is parallel to \vrt, but has a different magnitude and points in the opposite direction. Why is this?
	
	Recall the classic physics equation, ``Force $=$ mass $\times$ acceleration.'' A force acting on a mass induces acceleration (i.e., the mass moves); acceleration acting on a mass induces a force (gravity gives our mass a \emph{weight}). Thus force and acceleration are closely related. A moving ball ``wants'' to travel in a straight line. Why does the ball in our example move in a circle? It is attached to the boy's hand by a string. The string applies a force to the ball, affecting it's motion: the string \emph{accelerates} the ball. This is not acceleration in the sense of ``it travels faster;'' rather, this acceleration is changing the velocity of the ball. In what direction is this force/acceleration being applied? In the direction of the string, towards the boy's hand.
	
		The magnitude of the acceleration is related to the speed at which the ball is traveling. A ball whirling quickly is rapidly changing direction/velocity. When velocity is changing rapidly, the acceleration must be ``large.''
		
	\item		When the boy releases the string, the string no longer applies a force to the ball, meaning acceleration is $\vec 0$ and the ball can now move in a straight line in the direction of $\vec v(t)$. 
	
	Let $t=t_0$ be the time when the boy lets go of the string. The ball will be at $\vec r(t_0)$, traveling in the direction of $\vec v(t_0)$. We want to find $t_0$ so that this line contains the point $(0,10)$ (since the tree is 10ft directly in front of the boy).
	
\mtable{Modeling the flight of a ball in \autoref{ex_motion3}.}{fig:motion3}{\begin{tikzpicture}[>=latex,scale=.75]
\begin{scope}[scale=.5,shift={(0,10)}]
\ifthenelse{\boolean{inColor}}
{\draw[fill=brown,ultra thick] }
{\draw[ultra thick,fill=white]}
   (.75,-1)  ..  controls (.5,.5)  and   (.5,3)    .. (0.5,4) 
   -- (-0.5,4)  ..  controls (-.5,3) and (-.5,.5)     .. (-.75,-1) ;
\ifthenelse{\boolean{inColor}}
{\draw[ultra thick,fill=green!50!black] }
{\draw[ultra thick,fill=white]}
      (0,10) .. controls  (0,8)     and   (1,7)    .. (1.5,7) 
             ..  controls (1,7)     and   (1,7)    .. (0.5,7.25) 
             ..  controls (1.5,5)   and   (2.5,4)  .. (3,4)
             ..  controls (2,4)     and   (1.25,4) .. (1,4.5)
             ..  controls (2,2)     and   (3.5,2)  .. (4,2)
             ..  controls (1,1)     and   (-1,1)   .. (-4,2) 
             ..  controls (-3.5,2)  and   (-2,2)   .. (-1,4.5)
             ..  controls (-1.25,4) and   (-2,4)   .. (-3,4) 
             ..  controls (-2.5,4)  and   (-1.5,5) .. (-0.5,7.25) 
             ..  controls  (-1,7)   and   (-1,7)   .. (-1.5,7)
             ..  controls  (-1,7)   and   (0,8)    .. (0,10)
              ;
\end{scope}	
\begin{scope}[scale=.5]
\coordinate (ball) at (215:2cm);
\coordinate (release) at (11.5:2cm);
\draw [rotate=11.5] (0,0) -- (1.9,0) node [above,sloped,pos=.5] {\scriptsize 2ft};
\draw [thick,dashed] (0,0) circle (2cm);
\draw [thick,->] (ball) arc (215:225:2cm);
\draw (0,10) circle (2pt);
\draw[dotted,thick] (release) -- (0,10) node [sloped, pos=.5, above] {\scriptsize $\bracket{0,10}- \vec r(t_0)$};
\filldraw[draw=black,fill=white,thick] (release) circle (.1);
\end{scope}
\end{tikzpicture}}

	There are many ways to find this time value. We choose one that is relatively simple computationally. As shown in \autoref{fig:motion3}, the vector from the release point to the tree is $\bracket{0,10}- \vec r(t_0)$. This line segment is tangent to the circle, which means it is also perpendicular to $\vec r(t_0)$ itself, so their dot product is 0.
	
	\begin{align*}
	\vec r(t_0) \cdot \big(\bracket{0,10}- \vec r(t_0)\big) &=0\\
	\bracket{2\cos (4\pi t_0), 2\sin (4\pi t_0)}\cdot\bracket{-2\cos(4\pi t_0),10-2\sin (4\pi t_0)}&=0\\
	-4\cos^2(4\pi t_0) + 20\sin (4\pi t_0)-4\sin^2(4\pi t_0) &= 0\\
	20\sin (4\pi t_0) - 4 &=0\\
	\sin (4\pi t_0) &=1/5\\
	4\pi t_0 &= \sin^{-1}(1/5)\\
	4\pi t_0 &\approx 0.2 + 2\pi n\\%,
	%\intertext{where $n$ is an integer. Solving for $t_0$ we have:}
	t_0 &\approx 0.016 + n/2
	\end{align*}
	where $n$ is an integer.
%	This is a wonderful formula.
	This means that every 1/2 second after $t=0.016$s the boy can release the string (since the ball makes 2 revolutions per second, he has two chances each second to release the ball).\eoehere
\end{enumerate}}

\example{ex_motion4}{Analyzing motion in space}{An object moves in a spiral with position function $\vrt =\bracket{\cos t, \sin t, t}$, where distances are measured in meters and time is in minutes. Describe the object's speed and acceleration at time $t$.}
{With $\vrt =\bracket{\cos t,\sin t, t}$, we have:
\begin{align*}
\vvt &=\bracket{-\sin t, \cos t, 1}\quad \text{and} \\
\vat &=\bracket{-\cos t, -\sin t, 0}.
\end{align*}

The speed of the object is $\norm{\vvt} = \sqrt{(-\sin t)^2+\cos^2t+1} = \sqrt{2}$m/min; it moves at a constant speed. Note that the object does not accelerate in the $z$-direction, but rather moves up at a constant rate of 1m/min.}

The objects in Examples \ref{ex_motion3} and \ref{ex_motion4} traveled at a constant speed. That is, $\norm{\vvt} = c$ for some constant $c$. Recall \autoref{thm:vects_of_constant_length}, which states that if a vector--valued function \vrt\ has constant length, then \vrt\ is perpendicular to its derivative: $\vrt\cdot\vrp(t) = 0$. In these examples, the velocity function has constant length, therefore we can conclude that the velocity is perpendicular to the acceleration: $\vvt\cdot\vat = 0$. A quick check verifies this.\index{vector--valued function!of constant length}

There is an intuitive understanding of this. If acceleration is parallel to velocity, then it is only affecting the object's speed; it does not change the direction of travel. (For example, consider a dropped stone. Acceleration and velocity are parallel -- straight down -- and the direction of velocity never changes, though speed does increase.) If acceleration is not perpendicular to velocity, then there is some acceleration in the direction of travel, influencing the speed. If speed is constant, then acceleration must be orthogonal to velocity, as it then only affects direction, and not speed.

\keyidea{idea:constant_speed}{Objects With Constant Speed}
{If an object moves with constant speed, then its velocity and acceleration vectors are orthogonal. That is, $\vvt\cdot\vat=0$.
\index{vector--valued function!of constant length}}


\subsection{Projectile Motion}

An important application of vector--valued position functions is \emph{projectile motion}: the motion of objects under only the influence of gravity. We will measure time in seconds, and distances will either be in meters or feet. We will show that we can completely describe the path of such an object knowing its initial position and initial velocity (i.e., where it \emph{is} and where it \emph{is going.})
\index{vector--valued function!projectile motion}\index{projectile motion}

Suppose an object has initial position $\vec r(0) =\bracket{x_0,y_0}$ and initial velocity $\vec v(0) =\bracket{v_x,v_y}$. It is customary to rewrite $\vec v(0)$ in terms of its speed $v_0$ and direction $\vec u$, where $\vec u$ is a unit vector. Recall all unit vectors in $\mathbb{R}^2$ can be written as $\bracket{\cos \theta,\sin \theta}$, where $\theta$ is an angle measure counter--clockwise from the $x$-axis. (We refer to $\theta$ as the \textbf{angle of elevation.}\index{angle of elevation}) Thus $\vec v(0) = v_0\bracket{\cos \theta,\sin \theta}.$ 

Since the acceleration of the object is known, namely $\vat =\bracket{0,-g}$, where $g$ is the gravitational constant, we can find $\vrt$ knowing our two initial conditions. We first find $\vvt$:
\mnote{\textbf{Note:} In this text we use $g=32$ft/s$^2$ when using Imperial units, and $g=9.8$m/s$^2$ when using SI units.}
\begin{align*}
\vec v(t) &= \int \vat \ dt\\
\vvt &= \int\bracket{0,-g}\ dt\\
\vvt &=\bracket{0,-gt}+ \vec C.
\end{align*}
Knowing $\vec v(0) = v_0\bracket{\cos \theta,\sin \theta}$, we have $\vec C = v_0\bracket{\cos\theta,\sin\theta}$ and so
\[\vec v(t) =\bracket{v_0\cos \theta, -gt+v_0\sin\theta}.\]
We integrate once more to find $\vrt$:
\begin{align*}
\vrt &= \int \vvt\ dt \\
\vrt &= \int\bracket{v_0\cos \theta, -gt+v_0\sin\theta}\ dt\\
\vrt &=\bracket{\big(v_0\cos \theta\big)t, -\frac12gt^2+\big(v_0\sin\theta\big)t}+ \vec C.
\end{align*}
Knowing $\vec r(0) =\bracket{x_0,y_0}$, we conclude $\vec C =\bracket{x_0,y_0}$ and
\begin{equation}
\vrt
=\bracket{\big(v_0\cos\theta\big)t+x_0\ ,-\frac12gt^2+\big(v_0\sin\theta\big)t+y_0}.
\label{idea:projectile}
\end{equation}
 %\\
%\vrt &=\bracket{0,-\frac12g}t^2 + v_0\bracket{\cos\theta,\sin \theta}t +\bracket{x_0,y_0}.

%\keyidea{idea:projectile}{Projectile Motion}
%{The position function of a projectile propelled from an initial position of $\vec r_0=\bracket{x_0,y_0}$, with initial speed $v_0$, with angle of elevation $\theta$ and neglecting all accelerations but gravity is 
%\index{vector--valued function!projectile motion}\index{projectile motion}
%\[\vrt =\bracket{\big(v_0\cos \theta\big)t+x_0\ , -\frac12gt^2+\big(v_0\sin\theta\big)t+y_0\ }.\]
%Letting $\vec v_0 = v_0\bracket{\cos \theta,\sin \theta}$,\ \ $\vrt$ can be written as
%\[\vrt =\bracket{0,-\frac12gt^2}+ \vec v_0t+\vec r_0.\]}

We demonstrate how to use this position function in the next two examples.

\example{ex_motion5}{Projectile Motion}{Sydney shoots her Red Ryder\textregistered\ bb gun across level ground from an elevation of 4ft, where the barrel of the gun makes a $5^\circ$ angle with the horizontal. Find how far the bb travels before landing, assuming the bb is fired at the advertised rate of 350ft/s and ignoring air resistance.}
{A direct application of \autoeqref{idea:projectile} gives
\begin{align*}
\vrt &=\bracket{(350\cos 5^\circ)t, -16t^2 + (350\sin 5^\circ)t + 4}\\
&\approx\bracket{346.67t, -16t^2+30.50t+4},
\end{align*}
where we set her initial position to be $\bracket{0,4}$.
We need to find \emph{when} the bb lands, then we can find \emph{where}. We accomplish this by setting the $y$-component equal to 0 and solving for $t$:
\begin{align*}
-16t^2+30.50t+4 &= 0 \\
t &= \frac{-30.50 \pm \sqrt{30.50^2-4(-16)(4)}}{-32}\\
t &\approx 2.03s.
\end{align*}
(We discarded a negative solution that resulted from our quadratic equation.) 

We have found that the bb lands 2.03s after firing; with $t=2.03$, we find the $x$-component of our position function is $346.67(2.03) = 703.74$ft. The bb lands about 704 feet away.}

\example{ex_motion61}{Projectile Motion}{Alex holds his sister's bb gun at a height of 3ft and wants to shoot a target that is 6ft above the ground, 25ft away. At what angle should he hold the gun to hit his target? (We still assume the muzzle velocity is 350ft/s.)}
{The position function for the path of Alex's bb is
\[\vrt =\bracket{(350\cos \theta)t, -16t^2+(350\sin\theta)t+3}.\]
We need to find $\theta$ so that $\vrt =\bracket{25,6}$ for some value of $t$. That is, we want to find $\theta$ and $t$ such that 
\[(350\cos\theta)t = 25 \quad \text{and}\quad -16t^2+(350\sin\theta)t+3 = 6.\]
This is not trivial (though not ``hard''). We start by solving each equation for $\cos\theta$ and $\sin \theta$, respectively.
\[\cos\theta = \frac{25}{350t} \quad \text{and} \quad \sin\theta = \frac{3+16t^2}{350t}.\]
Using the Pythagorean Identity $\cos^2\theta+\sin^2\theta=1$, we have
\[\left(\frac{25}{350t}\right)^2 + \left(\frac{3+16t^2}{350t}\right)^2 =1\]
Multiply both sides by $(350t)^2$:
\begin{align*}
25^2 + (3+16t^2)^2 &=350^2t^2\\
256t^4-122,404t^2+634 &=0.
\end{align*}
This is a quadratic \emph{in} $t^2$. That is, we can apply the quadratic formula  to find $t^2$, then solve for $t$ itself.
\begin{align*}
t^2 &= \frac{122,404\pm\sqrt{122,404^2-4(256)(634)}}{512}\\
t^2 &= 0.0052,\ 478.135\\
t &=  \pm 0.072,\ \pm 21.866
\end{align*}
Clearly the negative $t$ values do not fit our context, so we have $t=0.072$ and $t=21.866$. Using $\cos \theta = 25/(350 t)$, we can solve for $\theta$:
\begin{align*}
\theta &= \cos^{-1}\left(\frac{25}{350\cdot 0.072}\right)\quad \text{and}\quad \cos^{-1}\left(\frac{25}{350\cdot 21.866}\right)\\
\theta &= 7.03^\circ \quad \text{and} \quad 89.8^\circ.
\end{align*}
Alex has two choices of angle. He can hold the rifle at an angle of about $7^\circ$ with the horizontal and hit his target $0.07$s after firing, or he can hold his rifle almost straight up, with an angle of $89.8^\circ$, where he'll hit his target about 22s later. The first option is clearly the option he should choose.}

\subsection{Distance Traveled}

Consider a driver who sets her cruise--control to 60mph, and travels at this speed for an hour. We can ask:
\begin{enumerate}
	\item How far did the driver travel?
	\item	How far from her starting position is the driver?
\end{enumerate} 
The first is easy to answer: she traveled 60 miles. The second is impossible to answer with the given information. We do not know if she traveled in a straight line, on an oval racetrack, or along a slowly--winding highway.

This highlights an important fact: to compute distance traveled, we need only to know the speed, given by $\norm{\vvt}$.

\theorem{thm:distance_traveled}{Distance Traveled}
{Let $\vvt$ be a velocity function for a moving object. The distance traveled by the object on $[a,b]$ is:
\index{distance!traveled}\index{vector--valued function!distance traveled}\index{integration!distance traveled}
\[\text{distance traveled} = \int_a^b \norm{\vvt}\ dt.\]}

Note that this is just a restatement of \autoref{thm:vvf_arc_length}: arc length is the same as distance traveled, just viewed in a different context.

%Given a position function $\vrt =\bracket{f(t),g(t)}$, \autoref{thm:distance_traveled}
 %states that the distance traveled is
%\[\int_a^b \sqrt{\big(\fp(t)\big)^2+\big(g'(t)\big)^2}\ dt.\]
%Comparing this to \autoref{idea:arc_length_parametric} we see that \emph{distance traveled} measures the same thing as \emph{arc length}, just in different contexts.\\

\example{ex_motion6}{Distance Traveled, Displacement, and Average Speed}{A particle moves in space with position function $\vrt =\bracket{t,t^2,\sin (\pi t)}$ on $[-2,2]$, where $t$ is measured in seconds and distances are in meters. Find:
\begin{enumerate}
	\item The distance traveled by the particle on $[-2,2]$.
	\item	The displacement of the particle on $[-2,2]$.
	\item	The particle's average speed.
\end{enumerate}}
{\begin{enumerate}
	\item We use \autoref{thm:distance_traveled} to establish the integral:
	%%
	%% Using a boolean to give different graphics for 2D and 3D versions.
	\ifthenelse{\boolean{in_threeD}}{% if in three D
	\mtable{The path of the particle in \autoref{ex_motion6}.}{fig:motion6}{
	\myincludeasythree{width=\marginparwidth,
3Droll=-1.3988262638349023,
3Dortho=0.004399999976158142,
3Dc2c=0.5026575922966003 0.7320348024368286 0.4598483145236969,
3Dcoo=-14.488340377807617 25.037630081176758 5.939126491546631,
3Droo=150.0000062372992}{width=\marginparwidth}{figures/figmotion6_3D}}
	}%ends in 3D
	{% now in 2D	
	\mtable{The path of the particle, from two perspectives, in \autoref{ex_motion6}.}{fig:motion6}{%
	\myincludegraphics[width=\marginparwidth]{figures/figmotion6}\\[-20pt]
	(a)\\[20pt]
	% todo Tim make sure these three figures match
	\myincludegraphics[width=\marginparwidth]{figures/figmotion6_3D}\\
	(b)
	}% ends \mtable
	}% ends 2D, also ends all of if-then-else
	\begin{align*}
	\text{distance traveled}
	&= \int_{-2}^2 \norm{\vvt}\ dt \\
	&= \int_{-2}^2 \sqrt{1+(2t)^2+ \pi^2\cos^2(\pi t)}\ dt.
	\end{align*}
	This cannot be solved in terms of elementary functions so we turn to numerical integration, finding the distance to be 12.88 m.
		
	\item	The displacement is the vector
	\[\vec r(2)-\vec r(-2) =\bracket{2,4,0}-\bracket{-2,4,0}=\bracket{4,0,0}.\]
	That is, the particle ends with an $x$-value increased by 4 and with $y$- and $z$-values the same (see \autoref{fig:motion6}).
	
	\item	We found above that the particle traveled 12.88 m over 4 seconds. We can compute average speed by dividing: 12.88/4 = 3.22 m/s. 
	
	We should also consider \autoref{def:av_val} of \autoref{sec:FTC}, which says that the average value of a function $f$ on $[a,b]$ is $\frac{1}{b-a}\int_a^b f(x)\ dx$. In our context, the average value of the speed is
	\[
	\text{average speed} = \frac{1}{2-(-2)}\int_{-2}^2 \norm{\vvt}\ dt \approx \frac14 12.88 = 3.22\text{m/s}.
	\]
	Note how the physical context of a particle traveling gives meaning to a more abstract concept learned earlier.\eoehere
\end{enumerate}}

In \autoref{def:av_val} of \autoref{chapter:integration} we defined the average value of a function $f(x)$ on $[a,b]$ to be
\[\frac{1}{b-a}\int_a^bf(x)\ dx.\]
Note how in \autoref{ex_motion6} we computed the average speed as
\[\frac{\text{distance traveled}}{\text{travel time}} = \frac1{2-(-2)}\int_{-2}^2\norm{\vvt}\ dt;\]
that is, we just found the average value of $\norm{\vvt}$ on $[-2,2]$.

Likewise, given position function $\vrt$, the average velocity on $[a,b]$ is
\[\frac{\text{displacement}}{\text{travel time}} = \frac1{b-a}\int_a^b \vec{r}\,'(t)\ dt = \frac{\vec r(b)-\vec r(a)}{b-a};\]
that is, it is the average value of $\vec r\,'(t)$, or $\vvt$, on $[a,b]$.\\

\keyidea{idea:average_speed_velocity}{Average Speed, Average Velocity}
{Let $\vec r(t)$ be a continuous position function on an open interval $I$ containing $a<b$. \\

The \sword{average speed} is:
\[\frac{\text{distance traveled}}{\text{travel time}} = \frac{\int_a^b \norm{\vvt}\ dt}{b-a} = \frac1{b-a}\int_a^b\norm{\vvt}\ dt.\]

The \sword{average velocity} is:
\[\frac{\text{displacement}}{\text{travel time}} = \frac{\int_a^b \vec{r}\,'(t)\ dt}{b-a} = \frac1{b-a}\int_a^b\vec{r}\,'(t)\ dt.\]}

The next two sections investigate more properties of the graphs of vector--valued functions and we'll apply these new ideas to what we just learned about motion.

\printexercises{exercises/11_03_exercises}
