\section[Volume Between Surfaces and Triple Integration]{\!Volume Between Surfaces and Triple Integration}\label{sec:triple_int}

We learned in \autoref{sec:double_int_volume} how to compute the signed volume $V$ under a surface $z=f(x,y)$ over a region $R$: $V = \iint_R f(x,y)\dd A$. It follows naturally that if $f(x,y)\geq g(x,y)$ on $R$, then the \textbf{volume between $f(x,y)$ and $g(x,y)$ on $R$} is 
\[
V = \iint_R f(x,y)\dd A - \iint_R g(x,y)\dd A = \iint_R \bigl(f(x,y)-g(x,y)\bigr)\dd A.
\]

\begin{theorem}[Volume Between Surfaces]\label{thm:volume_between_surfaces}
Let $f$ and $g$ be continuous functions on a closed, bounded region $R$, where $f(x,y)\geq g(x,y)$ for all $(x,y)$ in $R$. The volume $V$ between $f$ and $g$ over $R$ is
\index{volume}
\[V =\iint_R \bigl(f(x,y)-g(x,y)\bigr)\dd A.\]
\end{theorem}

\begin{example}[Finding volume between surfaces]\label{ex_trip1}
Find the volume of the space region bounded by the planes $z=3x+y-4$ and $z=8-3x-2y$ where $x,y>0$. In \autoref{fig:trip1}(a) the planes are drawn; in (b), only the defined region is given.
\mtable{Finding the volume between the planes given in \autoref{ex_trip1}.}{fig:trip1}{%
\myincludeasythree{width=\marginparwidth,
3Droll=0,
3Dortho=0.005003686994314194,
3Dc2c=0.5144106149673462 -0.7084314823150635 0.48322510719299316,
3Dcoo=69.411376953125 50.58059310913086 38.15925598144531,
3Droo=149.99999640034895}{width=\marginparwidth}{figures/figtrip1_3D}
\\(a)\\
\myincludeasythree{width=\marginparwidth,
3Droll=0,
3Dortho=0.005003686994314194,
3Dc2c=0.5144106149673462 -0.7084314823150635 0.48322510719299316,
3Dcoo=69.411376953125 50.58059310913086 38.15925598144531,
3Droo=149.99999640034895}{width=\marginparwidth}{figures/figtrip1b_3D}
\\(b)}
\solution
We need to determine the region $R$ over which we will integrate. To do so, we need to determine where the planes intersect. They have common $z$-values when $3x+y-4=8-3x-2y$. Applying a little algebra, we have:
\begin{align*}
3x+y-4 &= 8-3x-2y\\
6x+3y &=12\\
2x+y &=4
\end{align*}
The planes intersect along the line $2x+y=4$. Therefore the region $R$ is bounded by $x=0$, $y=0$, and $y=4-2x$; we can convert these bounds to integration bounds of $0\leq x\leq 2$, $0\leq y\leq 4-2x$. Thus
\begin{align*}
V &= \iint_R \bigl(8-3x-2y-(3x+y-4)\bigr)\dd A \\
	&= \int_0^2\int_0^{4-2x} \bigl(12-6x-3y\bigr)\dd y\dd x\\
	&= 16\text{units}^3.
\end{align*}
The volume between the surfaces is $16$ cubic units.
\end{example}

In the preceding example, we found the volume by evaluating the integral
\[\ds \int_0^2\int_0^{4-2x} \bigl(8-3x-2y-(3x+y-4)\bigr)\dd y\dd x.\]
Note how we can rewrite the integrand as an integral, much as we did in \autoref{sec:iterated_integrals}:
\[8-3x-2y-(3x+y-4) = \int_{3x+y-4}^{8-3x-2y}\dd z.\]
Thus we can rewrite the double integral that finds volume as
\[\int_0^2\int_0^{4-2x} \bigl(8-3x-2y-(3x+y-4)\bigr)\dd y\dd x = \int_0^2\int_0^{4-2x}\left(\int_{3x+y-4}^{8-3x-2y}\dd z\right)\dd y\dd x.\]

This no longer looks like a ``double integral,'' but more like a ``triple integral.'' Just as our first introduction to double integrals was in the context of finding the area of a plane region, our introduction into triple integrals will be in the context of finding the volume of a space region.

\mtable{Approximating the volume of a region $D$ in space.}{fig:tripintro}{%
\myincludeasythree{width=\marginparwidth,
3Droll=0,
3Dortho=0.004937310703098774,
3Dc2c=0.6666666865348816 0.6666666865348816 0.3333333432674408,
3Dcoo=-0.0000029802322387695312 -0.0000029802322387695312 -0.0000014901161193847656,
3Droo=150}{width=\marginparwidth}{figures/figtripintro_3D}
\\(a)\\[15pt]
\myincludeasythree{width=\marginparwidth,
3Droll=0,
3Dortho=0.004000000189989805,
3Dc2c=0.6065065264701843 0.7266713380813599 0.32264310121536255,
3Dcoo=-17.34011459350586 28.072246551513672 -20.15855598449707,
3Droo=200.0000009711906}{width=\marginparwidth}{figures/figtripintroa_3D}
\\(b)}

To formally find the volume of a closed, bounded region $D$ in space, such as the one shown in \autoref{fig:tripintro}(a), we start with an approximation. Break $D$ into $n$ rectangular solids; the solids near the boundary of $D$ may possibly not include portions of $D$ and/or include extra space. In \autoref{fig:tripintro}(b), we zoom in on a portion of the boundary of $D$ to show a rectangular solid that contains space not in $D$; as this is an approximation of the volume, this is acceptable and this error will be reduced as we shrink the size of our solids.

The volume $\Delta V_i$ of the $i^\text{\,th}$ solid $D_i$ is $\Delta V_i = \Delta x_i\Delta y_i\Delta z_i$, where $\Delta x_i$, $\Delta y_i$ and $\Delta z_i$ give the dimensions of the rectangular solid in the $x$, $y$ and $z$ directions, respectively. By summing up the volumes of all $n$ solids, we get an approximation of the volume $V$ of $D$:
\[V \approx \sum_{i=1}^n \Delta V_i = \sum_{i=1}^n \Delta x_i\Delta y_i\Delta z_i.\]

Let $\norm{\Delta D}$ represent the length of the longest diagonal of rectangular solids in the subdivision of $D$. As $\norm{\Delta D}\to 0$, the volume of each solid goes to 0, as do each of $\Delta x_i$, $\Delta y_i$ and $\Delta z_i$, for all $i$. Our calculus experience tells us that taking a limit as $\norm{\Delta D}\to 0$ turns our approximation of $V$ into an exact calculation of $V$. Before we state this result in a theorem, we use a definition to define some terms.

{
\tcbset{grow to right by=9em}
\begin{definition}[Triple Integrals, Iterated Integration (Part I)]\label{def:triple_integral}
Let $D$ be a closed, bounded region in space. Let $a$ and $b$ be real numbers, let $g_1(x)$ and $g_2(x)$ be continuous functions of $x$, and let $f_1(x,y)$ and $f_2(x,y)$ be continuous functions of $x$ and $y$.
\index{integration!triple}\index{triple integral}\index{iterated integration}
\begin{enumerate}
	\item	The volume $V$ of $D$ is denoted by a \textbf{triple integral},
	$\ds V = \iiint_D\dd V$.
	
	\item The iterated integral $\ds \int_a^b\int_{g_1(x)}^{g_2(x)}\int_{f_1(x,y)}^{f_2(x,y)} \dd z\dd y\dd x$ is evaluated as 
	\[\int_a^b\int_{g_1(x)}^{g_2(x)}\int_{f_1(x,y)}^{f_2(x,y)} \dd z\dd y\dd x=\int_a^b\int_{g_1(x)}^{g_2(x)}\left(\int_{f_1(x,y)}^{f_2(x,y)} \dd z\right)\dd y\dd x.\]
	Evaluating the above iterated integral is \textbf{triple integration.}
\end{enumerate}
\end{definition}
}

Our informal understanding of the notation $\iiint_D\dd V$ is ``sum up lots of little volumes over $D$,'' analogous to our understanding of $\iint_R\dd A$ and $\iint_R\dd m$. We now state the major theorem of this section.

{
\tcbset{grow to right by=9em}
\begin{theorem}[Triple Integration (Part I)]\label{thm:triple_integration}
Let $D$ be a closed, bounded region in space and let $\Delta D$ be any subdivision of $D$ into $n$ rectangular solids, where the  $i^\text{\,th}$ subregion $D_i$ has dimensions $\Delta x_i\times\Delta y_i\times\Delta z_i$ and volume $\Delta V_i$.
\index{integration!triple}\index{triple integral}\index{iterated integration}
\begin{enumerate}
	\item	The volume $V$ of $D$ is
	\[V = \iiint_D\dd V = \lim_{\norm{\Delta D}\to0} \sum_{i=1}^n \Delta V_i = \lim_{\norm{\Delta D}\to0} \sum_{i=1}^n \Delta x_i\Delta y_i\Delta z_i.\]

	\item	If $D$ is defined as the region bounded by the planes $x=a$ and $x=b$, the cylinders $y=g_1(x)$ and $y=g_2(x)$, and the surfaces $z=f_1(x,y)$ and $z=f_2(x,y)$, where $a<b$, $g_1(x)\leq g_2(x)$ and $f_1(x,y)\leq f_2(x,y)$ on $D$, then
	\[\iiint_D \dd V = \int_a^b\int_{g_1(x)}^{g_2(x)}\int_{f_1(x,y)}^{f_2(x,y)} \dd z\dd y\dd x.\]

	\item	$V$ can be determined using iterated integration with other orders of integration (there are 6 total), as long as $D$ is defined by the region enclosed by a pair of planes, a pair of cylinders, and a pair of surfaces.
\end{enumerate}
\end{theorem}
}

We evaluated the area of a plane region $R$ by iterated integration, where the bounds were ``from curve to curve, then from point to point.'' \autoref{thm:triple_integration} allows us to find the volume of a space region with an iterated integral with bounds ``from surface to surface, then from curve to curve, then from point to point.'' In the iterated integral 
\[\int_a^b\int_{g_1(x)}^{g_2(x)}\int_{f_1(x,y)}^{f_2(x,y)} \dd z\dd y\dd x,\]
the bounds $a\leq x\leq b$ and $g_1(x)\leq y\leq g_2(x)$ define a region $R$ in the $x$-$y$ plane over which the region $D$ exists in space. However, these bounds are also defining surfaces in space; $x=a$ is a plane and $y=g_1(x)$ is a cylinder. The combination of these 6 surfaces enclose, and define, $D$.

\youtubeVideo{zFy-OpajEtA}{Triple Integrals}

Examples will help us understand triple integration, including integrating with various orders of integration.

\mtable{The region $D$ used in \autoref{ex_trip2} in (a); in (b), the region found by collapsing $D$ onto the $x$-$y$ plane.}{fig:trip2}{%
\myincludeasythree{width=\marginparwidth,
3Droll=0,
3Dortho=0.004707579035311937,
3Dc2c=0.4086907207965851 0.6385185718536377 0.6521241664886475,
3Dcoo=3.675152540206909 -7.818903923034668 16.216196060180664,
3Droo=150.00000119319247}{width=\marginparwidth}{figures/figtrip2_3D}
\\(a)\\[10pt]
\myincludeasythree{width=\marginparwidth,
3Droll=0,
3Dortho=0.004707579035311937,
3Dc2c=0.4086907207965851 0.6385185718536377 0.6521241664886475,
3Dcoo=3.675152540206909 -7.818903923034668 16.216196060180664,
3Droo=150.00000119319247}{width=\marginparwidth}{figures/figtrip2b_3D}
\\(b)}

\begin{example}[Finding the volume of a space region with triple integration]\label{ex_trip2}
Find the volume of the space region in the 1\textsuperscript{st} octant bounded by the plane $z=2-y/3-2x/3$, shown in \autoref{fig:trip2}(a), using the order of integration $\dd z\dd y\dd x$. Set up the triple integrals that give the volume in the other 5 orders of integration.
\solution
Starting with the order of integration $\dd z\dd y\dd x$, we need to first find bounds on $z$. The region $D$ is bounded below by the plane $z=0$ (because we are restricted to the first octant) and above by $z=2-y/3-2x/3$; $0\leq z\leq 2-y/3-2x/3$.

To find the bounds on $y$ and $x$, we ``collapse'' the region onto the $x$-$y$ plane, giving the triangle shown in \autoref{fig:trip2}(b). (We know the equation of the line $y=6-2x$ in two ways. First, by setting $z=0$, we have $0 = 2-y/3-2x/3 \Rightarrow y=6-2x$. Secondly, we know this is going to be a straight line between the points $(3,0)$ and $(0,6)$ in the $x$-$y$ plane.)

We define that region $R$, in the integration order of $\dd y\dd x$, with bounds $0\leq y\leq 6-2x$ and $0\leq x\leq 3$. Thus the volume $V$ of the region $D$ is:
\begin{align*}
	V &= \iiint_D \dd V\\
	&= \int_0^3\int_0^{6-2x}\int_0^{2-\frac 13y-\frac 23x}\dd z\dd y\dd x \\
	&= \int_0^3\int_0^{6-2x}\left(\int_0^{2-\frac 13y-\frac 23x}\dd z\right)\dd y\dd x \\
	&=\int_0^3\int_0^{6-2x}z\Big|_0^{2-\frac 13y-\frac 23x}\dd y\dd x \\
	&= \int_0^3\int_0^{6-2x}\left(2-\frac 13y-\frac 23x\right)\dd y\dd x.
	\intertext{From this step on, we are evaluating a double integral as done many times before. We skip these steps and give the final volume,}
	&= 6\ \text{units}^3.		
\end{align*}
\noindent The order $\dd z\dd x\dd y$:\bigskip

Now consider the volume using the order of integration $\dd z\dd x\dd y$. The bounds on $z$ are the same as before, $0\leq z\leq 2-y/3-2x/3$. Collapsing the space region on the $x$-$y$ plane as shown in \autoref{fig:trip2}(b), we now describe this triangle with the order of integration $\dd x\dd y$. This gives bounds $0\leq x\leq 3-y/2$ and $0\leq y\leq 6$. Thus the volume is given by the triple integral
\[V = \int_0^6\int_0^{3-\frac12y}\int_0^{2-\frac13y-\frac23x}\dd z\dd x\dd y.\]

\noindent The order $\dd x\dd y\dd z$:\bigskip

Following our ``surface to surface\dotso'' strategy, we need to determine the $x$-\emph{surfaces} that bound our space region. To do so, approach the region ``from behind,'' in the direction of increasing $x$. The first surface we hit as we enter the region is the $y$-$z$ plane, defined by $x=0$. We come out of the region at the plane $z=2-y/3-2x/3$; solving for $x$, we have $x= 3-y/2-3z/2$. Thus the bounds on $x$ are: $0\leq x\leq 3-y/2-3z/2$.

We now need to collapse the space region onto the $y$-$z$ plane, as we see in \autoref{fig:trip2b}(a). (Again, we find the equation of the line $z=2-y/3$ by setting $x=0$ in the equation $x=3-y/2-3z/2$.) We need to find bounds on this region with the order $\dd y\dd z$. The \emph{curves} that bound $y$ are $y=0$ and $y=6-3z$; the \emph{points} that bound $z$ are 0 and 2. Thus the triple integral giving volume is:
\[
 \begin{gathered}
  0\leq x\leq 3-y/2-3z/2\\
  0\leq y\leq 6-3z\\
  0\leq z\leq 2
 \end{gathered}
 \qquad\Rightarrow\qquad
 \int_0^2\int_0^{6-3z}\int_0^{3-y/2-3z/2}\dd x\dd y\dd z.
\]

\mtable{The region $D$ in \autoref{ex_trip2} is collapsed onto the $y$-$z$ plane in (a); in (b), the region is collapsed onto the $x$-$z$ plane.}{fig:trip2b}{%
\myincludeasythree{width=\marginparwidth,
3Droll=0,
3Dortho=0.004707579035311937,
3Dc2c=0.4086907207965851 0.6385185718536377 0.6521241664886475,
3Dcoo=3.675152540206909 -7.818903923034668 16.216196060180664,
3Droo=150.00000119319247}{width=\marginparwidth}{figures/figtrip2c_3D}
\\(a)\\[15pt]
\myincludeasythree{width=\marginparwidth,
3Droll=0,
3Dortho=0.004707579035311937,
3Dc2c=0.4086907207965851 0.6385185718536377 0.6521241664886475,
3Dcoo=3.675152540206909 -7.818903923034668 16.216196060180664,
3Droo=150.00000119319247}{width=\marginparwidth}{figures/figtrip2d_3D}
\\(b)}

\noindent The order $\dd x\dd z\dd y$:\bigskip

The $x$-bounds are the same as the order above. We now consider the triangle in \autoref{fig:trip2b}(a) and describe it with the order $\dd z\dd y$: $0\leq z\leq 2-y/3$ and $0\leq y\leq 6$. Thus the volume is given by:
\[
 \begin{gathered}
  0\leq x\leq 3-y/2-3z/2\\
  0\leq z\leq 2-y/3\\
  0\leq y\leq 6
 \end{gathered}
 \qquad\Rightarrow\qquad
 \int_0^6\int_0^{2-y/3}\int_0^{3-y/2-3z/2}\dd x\dd z\dd y.
\]

\noindent The order $\dd y\dd z\dd x$:\bigskip

We now need to determine the $y$-surfaces that determine our region. Approaching the space region from ``behind'' and moving in the direction of increasing $y$, we first enter the region at $y=0$, and exit along the plane $z= 2-y/3-2x/3$. Solving for $y$, this plane has equation $y = 6-2x-3z$. Thus $y$ has bounds $0\leq y\leq 6-2x-3z$. 

Now collapse the region onto the $x$-$z$ plane, as shown in \autoref{fig:trip2b}(b). The curves bounding this triangle are $z=0$ and $z=2-2x/3$; $x$ is bounded by the points $x=0$ to $x=3$. Thus the triple integral giving volume is: 
\[
 \begin{gathered}
  0\leq y\leq 6-2x-3z\\
  0\leq z\leq 2-2x/3\\
  0\leq x\leq 3
 \end{gathered}
 \qquad\Rightarrow\qquad
 \int_0^3\int_0^{2-2x/3}\int_0^{6-2x-3z}\dd y\dd z\dd x.
\]

\noindent The order $\dd y\dd x\dd z$:\bigskip

The $y$-bounds are the same as in the order above. We now determine the bounds of the triangle in \autoref{fig:trip2b}(b) using the order $\dd y\dd x\dd z$. We see $x$ is bounded by $x=0$ and $x=3-3z/2$; $z$ is bounded between $z=0$ and $z=2$. This leads to the triple integral:
\[
 \begin{gathered}
  0\leq y\leq 6-2x-3z\\
  0\leq x\leq 3-3z/2\\
  0\leq z\leq 2
 \end{gathered}
 \qquad\Rightarrow\qquad
 \int_0^2\int_0^{3-3z/2}\int_0^{6-2x-3z}\dd y\dd x\dd z.
\]

This problem was long, but hopefully useful, demonstrating how to determine bounds with every order of integration to describe the region $D$. In practice, we only need 1, but being able to do them all gives us flexibility to choose the order that suits us best.
\end{example}

In the previous example, we collapsed the surface into the $x$-$y$, $x$-$z$, and $y$-$z$ planes as we determined the ``curve to curve, point to point'' bounds of integration. Since the surface was a triangular portion of a plane, this collapsing, or \emph{projecting}, was simple: the \emph{projection} of a straight line in space onto a coordinate plane is a line.

The following example shows us how to do this when dealing with more complicated surfaces and curves.

\mtable{Finding the projections of the curve of intersection in \autoref{ex_trip2b}.}{fig:trip2bb}{%
\myincludeasythree{width=\marginparwidth,
3Droll=0,
3Dortho=0.004707579966634512,
3Dc2c=0.8939384818077087 0.2635158896446228 0.3625374734401703,
3Dcoo=6.587856292724609 50.29287338256836 11.16304874420166,
3Droo=149.9999888496757}{width=\marginparwidth}{figures/fig3d_proj_3D}
\\(a)\\
\myincludeasythree{width=\marginparwidth,
3Droll=0,
3Dortho=0.004707579966634512,
3Dc2c=0.8939384818077087 0.2635158896446228 0.3625374734401703,
3Dcoo=6.587856292724609 50.29287338256836 11.16304874420166,
3Droo=149.9999888496757}{width=\marginparwidth}{figures/fig3d_projb_3D}
\\(b)}

\begin{example}[Finding the projection of a curve in space onto the coordinate planes]\label{ex_trip2b}
Consider the surfaces $z=3-x^2-y^2$ and $z=2y$, as shown in \autoref{fig:trip2bb}(a). The curve of their intersection is shown, along with the projection of this curve into the coordinate planes, shown dashed. Find the equations of the projections into the coordinate planes.
\solution
The two surfaces are $z=3-x^2-y^2$ and $z=2y$. To find where they intersect, it is natural to set them equal to each other: $3-x^2-y^2=2y$. This is an implicit function of $x$ and $y$ that gives all points $(x,y)$ in the $x$-$y$ plane where the $z$ values of the two surfaces are equal. 

We can rewrite this implicit function by completing the square:
\[3-x^2-y^2=2y \quad \Rightarrow \quad y^2+2y+x^2=3\quad \Rightarrow \quad (y+1)^2+x^2=4.\]
Thus in the $x$-$y$ plane the projection of the intersection is a circle with radius 2, centered at $(0,-1)$.

To project onto the $x$-$z$ plane, we do a similar procedure: find the $x$ and $z$ values where the $y$ values on the surface are the same. We start by solving the equation of each surface for $y$. In this particular case, it works well to actually solve for $y^2$:
\[
\begin{aligned}
z&=3-x^2-y^2\\
z&=2y
\end{aligned}
\qquad\Rightarrow\qquad
\begin{aligned}
y^2&=3-x^2-z\\
y^2&=z^2/4
\end{aligned}
\]

Thus we have (after again completing the square):
\[3-x^2-z = z^2/4 \quad \Rightarrow\quad \frac{(z+2)^2}{16}+\frac{x^2}4=1,\]
and ellipse centered at $(0,-2)$ in the $x$-$z$ plane with a major axis of length 8 and a minor axis of length 4. 

Finally, to project the curve of intersection into the $y$-$z$ plane, we solve equation for $x$. Since $z=2y$ is a cylinder that lacks the variable $x$, it becomes our equation of the projection in the $y$-$z$ plane.

All three projections are shown in \autoref{fig:trip2bb}(b).
\end{example}

\mtable{The region $D$ in \autoref{ex_trip3} is shown in (a); in (b), it is collapsed onto the $x$-$y$ plane.}{fig:trip3}{%
\myincludeasythree{width=\marginparwidth,
3Droll=0,
3Dortho=0.004735102877020836,
3Dc2c=0.8340418338775635 0.34454345703125 0.43088746070861816,
3Dcoo=-6.1953253746032715 -30.81180191040039 48.79259490966797,
3Droo=149.99999973988238}{width=\marginparwidth}{figures/figtrip3_3D}
\\(a)\\
\myincludeasythree{width=\marginparwidth,
3Droll=0,
3Dortho=0.004735102877020836,
3Dc2c=0.8340418338775635 0.34454345703125 0.43088746070861816,
3Dcoo=-6.1953253746032715 -30.81180191040039 48.79259490966797,
3Droo=149.99999973988238}{width=\marginparwidth}{figures/figtrip3b_3D}
\\(b)}

\begin{example}[Finding the volume of a space region with triple integration]\label{ex_trip3}
Set up the triple integrals that find the volume of the space region $D$ where $x^2+y^2\le1$ and $0\le z\le-y$, as shown in \autoref{fig:trip3}(a), with the orders of integration $\dd z\dd y\dd x$, $\dd y\dd x\dd z$ and $\dd x\dd z\dd y$.
\solution
The order $\dd z\dd y\dd x$:\bigskip

The region $D$ is bounded below by the plane $z=0$ and above by the plane $z=-y$. The cylinder $x^2+y^2=1$ does not offer any bounds in the $z$-direction, as that surface is parallel to the $z$-axis. Thus $0\leq z\leq -y$.

Collapsing the region into the $x$-$y$ plane, we get part of the region bounded by the circle with equation $x^2+y^2=1$ as shown in \autoref{fig:trip3}(b). As a function of $x$, this half circle has equation $y=-\sqrt{1-x^2}$. Thus $y$ is bounded below by $-\sqrt{1-x^2}$ and above by $y=0$: $-\sqrt{1-x^2}\leq y\leq 0$. The $x$ bounds of the half circle are $-1\leq x\leq 1$. All together, the bounds of integration and triple integral are  as follows:
\[
 \begin{gathered}
  0\leq z\leq -y\\
  -\sqrt{1-x^2}\leq y\leq 0\\
  -1\leq x\leq 1
 \end{gathered}
 \qquad\Rightarrow\qquad
 \int_{-1}^1\int_{-\sqrt{1-x^2}}^{0}\int_0^{-y}\dd z\dd y\dd x.
\]
We evaluate this triple integral:
\begin{align*}
	\int_{-1}^1\int_{-\sqrt{1-x^2}}^{0}\int_0^{-y}\dd z\dd y\dd x
	&= \int_{-1}^1\int_{-\sqrt{1-x^2}}^{0}\bigl(-y\bigr)\dd y\dd x\\
	&=\int_{-1}^1\bigl(-\frac12y^2\bigr)\Big|_{-\sqrt{1-x^2}}^{0}\dd x\\
	&= \int_{-1}^1 \frac12\bigl(1-x^2\bigr)\dd x\\
	&= \left.\left(\frac12\left(x-\frac13x^3\right)\right)\right|_{-1}^1\\
	&= \frac23\text{units}^3.
\end{align*}

\noindent With the order $\dd y\dd x\dd z$:\bigskip

The region is bounded ``below'' in the $y$-direction by the surface $x^2+y^2=1 \Rightarrow y=-\sqrt{1-x^2}$ and ``above'' by the surface $y=-z$. Thus the $y$ bounds are $-\sqrt{1-x^2}\leq y\leq -z$.

\mtable{The region $D$ in \autoref{ex_trip3} is shown collapsed onto the $x$-$z$ plane in (a); in (b), it is collapsed onto the $y$-$z$ plane.}{fig:trip3b}{%
\myincludeasythree{width=\marginparwidth,
3Droll=0,
3Dortho=0.004735102877020836,
3Dc2c=0.8340418338775635 0.34454345703125 0.43088746070861816,
3Dcoo=-6.1953253746032715 -30.81180191040039 48.79259490966797,
3Droo=149.99999973988238}{width=\marginparwidth}{figures/figtrip3c_3D}
\\(a)\\
\myincludeasythree{width=\marginparwidth,
3Droll=0,
3Dortho=0.004735102877020836,
3Dc2c=0.8340418338775635 0.34454345703125 0.43088746070861816,
3Dcoo=-6.1953253746032715 -30.81180191040039 48.79259490966797,
3Droo=149.99999973988238}{width=\marginparwidth}{figures/figtrip3d_3D}
\\(b)}

Collapsing the region onto the $x$-$z$ plane gives the region that is shown in \autoref{fig:trip3b}(a); this half disk is bounded by a circle with equation $x^2+z^2=1$. (We find this curve by solving each surface for $y^2$, then setting them equal to each other. We have $y^2=1-x^2$ and $y=-z\Rightarrow y^2=z^2$. Thus $x^2+z^2=1$.) It is bounded below by $x=-\sqrt{1-z^2}$ and above by $x=\sqrt{1-z^2}$, where $z$ is bounded by $0\leq z\leq 1$. All together, we have:
\[
 \begin{gathered}
  -\sqrt{1-x^2}\leq y\leq -z\\
  -\sqrt{1-z^2}\leq x\leq \sqrt{1-z^2}\\
  0\leq z\leq 1
 \end{gathered}
 \qquad\Rightarrow\qquad
 \int_{0}^1\int_{-\sqrt{1-z^2}}^{\sqrt{1-z^2}}\int_{-\sqrt{1-x^2}}^{-z}\dd y\dd x\dd z.
\]

\noindent With the order $\dd x\dd z\dd y$:\bigskip

$D$ is bounded below by the surface $x=-\sqrt{1-y^2}$ and above by $\sqrt{1-y^2}$. We then collapse the region onto the $y$-$z$ plane and get the triangle shown in \autoref{fig:trip3b}(b). (The hypotenuse is the line $z=-y$, just as the plane.) Thus $z$ is bounded by $0\leq z\leq -y$ and $y$ is bounded by $-1\leq y\leq 0$. This gives:
\[
 \begin{gathered}
  -\sqrt{1-y^2}\leq x\leq \sqrt{1-y^2}\\
  0\leq z\leq -y\\
  -1\leq y\leq 0
 \end{gathered}
 \qquad\Rightarrow\qquad
 \int_{-1}^0\int_{0}^{-y}\int_{-\sqrt{1-y^2}}^{\sqrt{1-y^2}}\dd x\dd z\dd y.
\]
\end{example}

The following theorem states two things that should make ``common sense'' to us. First, using the triple integral to find volume of a region $D$ should always return a positive number; we are computing \emph{volume} here, not \emph{signed volume}. Secondly, to compute the volume of a ``complicated'' region, we could break it up into subregions and compute the volumes of each subregion separately, summing them later to find the total volume.

\begin{theorem}[Properties of Triple Integrals]\label{thm:triple_int_prop}
Let $D$ be a closed, bounded region in space, and let $D_1$ and $D_2$ be non-overlapping regions such that $D=D_1\bigcup D_2$.
\index{triple integral!properties}\index{iterated integration!properties}
\begin{enumerate}
	\item $\ds \iiint_D \dd V \geq 0$
	\item	$\ds \iiint_D\dd V = \iiint_{D_1}\dd V + \iiint_{D_2}\dd V.$
\end{enumerate}
\end{theorem}

We use this latter property in the next example.

\begin{example}[Finding the volume of a space region with triple integration]\label{ex_trip4}
Find the volume of the space region $D$ bounded by the coordinate planes, $z=1-x/2$ and $z=1-y/4$, as shown in \autoref{fig:trip4}(a). Set up the triple integrals that find the volume of $D$ in all 6 orders of integration.
\mtable[-3\baselineskip]{The region $D$ in \autoref{ex_trip4} is shown in (a); in (b), it is collapsed onto the $x$-$y$ plane.}{fig:trip4}{%
\myincludeasythree{width=\marginparwidth,
3Droll=0,
3Dortho=0.004984850063920021,
3Dc2c=0.6418777108192444 0.6134042739868164 0.46013936400413513,
3Dcoo=-9.54626750946045 -8.89417552947998 36.88103103637695,
3Droo=150.0000021860597}{width=\marginparwidth}{figures/figtrip4_3D}
\\(a)\\
\myincludeasythree{width=\marginparwidth,
3Droll=0,
3Dortho=0.004984850063920021,
3Dc2c=0.6418777108192444 0.6134042739868164 0.46013936400413513,
3Dcoo=-9.54626750946045 -8.89417552947998 36.88103103637695,
3Droo=150.0000021860597}{width=\marginparwidth}{figures/figtrip4d_3D}
\\(b)}
\solution
Following the bounds-determining strategy of ``surface to surface, curve to curve, and point to point,''  we can see that the most difficult orders of integration are the two in which we integrate with respect to $z$ first, for there are two ``upper'' surfaces that bound $D$ in the $z$-direction. So we start by noting that we have 
\[0\leq z\leq 1-\frac12x \quad\text{and}\quad 0\leq z\leq 1-\frac14y.\]
We now collapse the region $D$ onto the $x$-$y$ axis, as shown in \autoref{fig:trip4}(b). The boundary of $D$, the line from $(0,0,1)$ to $(2,4,0)$, is shown in part (b) of the figure as a dashed line; it has equation $y=2x$. (We can recognize this in two ways: one, in collapsing the line from $(0,0,1)$ to $(2,4,0)$ onto the $x$-$y$ plane, we simply ignore the $z$-values, meaning the line now goes from $(0,0)$ to $(2,4)$. Secondly, the two surfaces meet where $z=1-x/2$ is equal to $z=1-y/4$: thus $1-x/2=1-y/4 \Rightarrow y=2x.$)

We use the second property of \autoref{thm:triple_int_prop} to state that 
\[\iiint_D \dd V = \iiint_{D_1}\dd V + \iiint_{D_2}\dd V,\]
where $D_1$ and $D_2$ are the space regions above the plane regions $R_1$ and $R_2$, respectively. Thus we can say
\[
\iiint_D\dd V
= \iint_{R_1}\left(\int_0^{1-x/2}\dd z\right)\dd A + \iint_{R_2}\left(\int_0^{1-y/4}\dd z\right)\dd A.
\]
All that is left is to determine bounds of $R_1$ and $R_2$, depending on whether we are integrating with order $\dd x\dd y$ or $\dd y\dd x$. We give the final integrals here, leaving it to the reader to confirm these results.\bigskip

\noindent $\dd z\dd y\dd x$:\\[-2\baselineskip]\parbox[t]{\linewidth}{%
\begin{gather*}
 \begin{gathered}
  0\leq z\leq 1-x/2\\
  0\leq y\leq 2x\\
  0\leq x\leq 2
 \end{gathered}
 \qquad\qquad
 \begin{gathered}
  0\leq z\leq 1-y/4\\
  2x\leq y\leq 4\\
  0\leq x\leq 2
 \end{gathered} \\
 \iiint_D\dd V = \int_0^2\int_0^{2x}\int_0^{1-x/2}\dd z\dd y\dd x +\int_0^2\int_{2x}^4\int_0^{1-y/4}\dd z\dd y\dd x
\end{gather*}}

\noindent $\dd z\dd x\dd y$:\\[-2\baselineskip]\parbox[t]{\linewidth}{%
\begin{gather*}
 \begin{gathered}
  0\leq z\leq 1-x/2\\
  y/2\leq x\leq 2\\
  0\leq y\leq 4
 \end{gathered}
 \qquad\qquad
 \begin{gathered}
  0\leq z\leq 1-y/4\\
  0\leq x\leq y/2\\
  0\leq y\leq 4
 \end{gathered} \\
 \iiint_D\dd V = \int_0^4\int_{y/2}^{2}\int_0^{1-x/2}\dd z\dd x\dd y +\int_0^4\int_{0}^{y/2}\int_0^{1-y/4}\dd z\dd x\dd y
\end{gather*}}

\mtable{The region $D$ in \autoref{ex_trip4} is shown collapsed onto the $x$-$z$ plane in (a); in (b), it is collapsed onto the $y$-$z$ plane.}{fig:trip4b}{%
\myincludeasythree{width=\marginparwidth,
3Droll=0,
3Dortho=0.004984850063920021,
3Dc2c=0.6418777108192444 0.6134042739868164 0.46013936400413513,
3Dcoo=-9.54626750946045 -8.89417552947998 36.88103103637695,
3Droo=150.0000021860597}{width=\marginparwidth}{figures/figtrip4b_3D}
\\(a)\\[10pt]
\myincludeasythree{width=\marginparwidth,
3Droll=0,
3Dortho=0.004984850063920021,
3Dc2c=0.6418777108192444 0.6134042739868164 0.46013936400413513,
3Dcoo=-9.54626750946045 -8.89417552947998 36.88103103637695,
3Droo=150.0000021860597}{width=\marginparwidth}{figures/figtrip4c_3D}
\\(b)}

The remaining four orders of integration do not require a sum of triple integrals. In \autoref{fig:trip4b} we show $D$ collapsed onto the other two coordinate planes. Using these graphs, we give the final orders of integration here, again leaving it to the reader to confirm these results.

\noindent $\dd y\dd x\dd z$:\\[-\baselineskip]\parbox[t]{\linewidth}{%
\[
 \begin{gathered}
  0\leq y\leq 4-4z\\
  0\leq x\leq 2-2z\\
  0\leq z\leq 1
 \end{gathered} 
 \qquad\Rightarrow\qquad
 \int_0^1\int_{0}^{2-2z}\int_0^{4-4z}\dd y\dd x\dd z 
\]}
\noindent $\dd y\dd z\dd x$:\\[-\baselineskip]\parbox[t]{\linewidth}{%
\[
 \begin{gathered}
  0\leq y\leq 4-4z\\
  0\leq z\leq 1-x/2\\
  0\leq x\leq 2
 \end{gathered}
 \qquad\Rightarrow\qquad
 \int_0^2\int_{0}^{1-x/2}\int_0^{4-4z}\dd y\dd x\dd z 
\]}
\noindent $\dd x\dd y\dd z$:\\[-\baselineskip]\parbox[t]{\linewidth}{%
\[
 \begin{gathered}
  0\leq x\leq 2-2z\\
  0\leq y\leq 4-4z\\
  0\leq z\leq 1
 \end{gathered}
 \qquad\Rightarrow\qquad
 \int_0^1\int_{0}^{4-4z}\int_0^{2-2z}\dd x\dd y\dd z 
\]}
\noindent $\dd x\dd z\dd y$:\\[-\baselineskip]\parbox[t]{\linewidth}{%
\[
 \begin{gathered}
  0\leq x\leq 2-2z\\
  0\leq z\leq 1-y/4\\
  0\leq y\leq 4
 \end{gathered}
 \qquad\Rightarrow\qquad
 \int_0^4\int_{0}^{1-y/4}\int_0^{2-2z}\dd x\dd z\dd y
\]}
\end{example}

We give one more example of finding the volume of a space region.

\begin{example}[Finding the volume of a space region]\label{ex_trip5}
Set up a triple integral that gives the volume of the space region $D$ bounded by $z= 2x^2+2$ and $z=6-2x^2-y^2$. These surfaces are plotted in \autoref{fig:trip5}(a) and (b), respectively; the region $D$ is shown in part (c) of the figure.

\noindent\begin{minipage}[t]{\linewidth}\noindent%
\captionsetup{type=figure}%
\flushinner{%
\begin{tabular}{ccc}
\myincludeasythree{width=.8\marginparwidth,
3Droll=0,
3Dortho=0.004895491059869528,
3Dc2c=0.5800048112869263 0.5719658732414246 0.5800426006317139,
3Dcoo=-9.46509075164795 -11.93244457244873 49.59981155395508,
3Droo=150.00000228824675}{width=.8\marginparwidth}{figures/figtrip5b_3D}
&
\myincludeasythree{width=.8\marginparwidth,
3Droll=0,
3Dortho=0.004895491059869528,
3Dc2c=0.5800048112869263 0.5719658732414246 0.5800426006317139,
3Dcoo=-9.46509075164795 -11.93244457244873 49.59981155395508,
3Droo=150.00000228824675}{width=.8\marginparwidth}{figures/figtrip5c_3D}
&
\myincludeasythree{width=.8\marginparwidth,
3Droll=0,
3Dortho=0.004895491059869528,
3Dc2c=0.5800048112869263 0.5719658732414246 0.5800426006317139,
3Dcoo=-9.46509075164795 -11.93244457244873 49.59981155395508,
3Droo=150.00000228824675}{width=.8\marginparwidth}{figures/figtrip5_3D}
\\
(a)&(b)&(c)
\end{tabular}}
\caption{The region $D$ is bounded by the surfaces shown in (a) and (b); $D$ is shown in (c).}
\label{fig:trip5}
\end{minipage}
\solution
The main point of this example is this: integrating with respect to $z$ first is rather straightforward; integrating with respect to $x$ first is not.\bigskip

\noindent The order $\dd z\dd y\dd x$:\bigskip

The bounds on $z$ are clearly $2x^2+2\leq z\leq 6-2x^2-y^2$. Collapsing $D$ onto the $x$-$y$ plane gives the ellipse shown in \autoref{fig:trip5}(c). The equation of this ellipse is found by setting the two surfaces equal to each other: 
\[2x^2+2 = 6-2x^2-y^2\quad \Rightarrow\quad 4x^2+y^2=4\quad \Rightarrow\quad x^2+\frac{y^2}4=1.\]

We can describe this ellipse with the bounds 
\[-\sqrt{4-4x^2} \leq y\leq \sqrt{4-4x^2}\quad \text{and}\quad -1\leq x\leq 1.\]
Thus we find volume as
\[
	\begin{gathered}
		2x^2+2\leq z\leq 6-2x^2-y^2\\
		-\sqrt{4-4x^2}\leq y\leq \sqrt{4-4x^2}\\
		-1\leq x\leq 1
	\end{gathered} 
	\qquad\Rightarrow\qquad
	\ds\int_{-1}^1\int_{-\sqrt{4-4x^2}}^{\sqrt{4-4x^2}}\int_{2x^2+2}^{6-2x^2-y^2}\dd z\dd y\dd x 
\]

\noindent The order $\dd y\dd z\dd x$:\bigskip

Integrating with respect to $y$ is not too difficult. Since the surface $z=2x^2+2$ is a cylinder whose directrix is the $y$-axis, it does not create a border for $y$. The paraboloid $z=6-2x^2-y^2$ does; solving for $y$, we get the bounds 
\[-\sqrt{6-2x^2-z}\leq y\leq \sqrt{6-2x^2-z}.\]
Collapsing $D$ onto the $x$-$z$ axes gives the region shown in \autoref{fig:trip5b}(a); the lower curve is from the cylinder, with equation $z=2x^2+2$. The upper curve is from the paraboloid; with $y=0$, the curve is $z=6-2x^2$. Thus bounds on $z$ are $2x^2+2\leq z\leq 6-2x^2$; the bounds on $x$ are $-1\leq x\leq 1$. Thus we have:
\[
	\begin{gathered}
		-\sqrt{6-2x^2-z}\leq y\leq \sqrt{6-2x^2-z}\\
		2x^2+2\leq z\leq 6-2x^2\\
		-1\leq x\leq 1
	\end{gathered}
	\Rightarrow\quad
	\int_{-1}^1\int_{2x^2+2}^{6-2x^2}\int_{-\sqrt{6-2x^2-z}}^{\sqrt{6-2x^2-z}}\dd y\dd z\dd x.
\]

\mtable{The region $D$ in \autoref{ex_trip5} is collapsed onto the $x$-$z$ plane in (a); in (b), it is collapsed onto the $y$-$z$ plane.}{fig:trip5b}{%
\myincludeasythree{width=\marginparwidth,
3Droll=0,
3Dortho=0.004895491059869528,
3Dc2c=0.5800048112869263 0.5719658732414246 0.5800426006317139,
3Dcoo=-9.46509075164795 -11.93244457244873 49.59981155395508,
3Droo=150.00000228824675}{width=\marginparwidth}{figures/figtrip5e_3D}
\\(a)\\[10pt]
\myincludeasythree{width=\marginparwidth,
3Droll=0,
3Dortho=0.004895491059869528,
3Dc2c=0.5800048112869263 0.5719658732414246 0.5800426006317139,
3Dcoo=-9.46509075164795 -11.93244457244873 49.59981155395508,
3Droo=150.00000228824675}{width=\marginparwidth}{figures/figtrip5d_3D}
\\(b)}

\noindent The order $\dd x\dd z\dd y$:\bigskip

This order takes more effort as $D$ must be split into two subregions. The two surfaces create two sets of upper/lower bounds in terms of $x$; the cylinder creates bounds
\[-\sqrt{z/2-1}\leq x\leq \sqrt{z/2-1}\]
for region $D_1$  and the paraboloid creates bounds
\[-\sqrt{3-y^2/2-z^2/2}\leq x\leq \sqrt{3-y^2/2-z^2/2}\]
for region $D_2$.

Collapsing $D$ onto the $y$-$z$ axes gives the regions shown in \autoref{fig:trip5b}(b). We find the equation of the curve $z=4-y^2/2$ by noting that the equation of the ellipse seen in \autoref{fig:trip5}(c) has equation 
\[x^2+y^2/4=1 \quad \Rightarrow \quad x = \sqrt{1-y^2/4}.\]  
Substitute this expression for $x$ in either surface equation, $z=6-2x^2-y^2$ or $z=2x^2+2$. In both cases, we find
\[z=4-\frac12y^2.\]
Region $R_1$, corresponding to $D_1$, has bounds
\[2\leq z\leq 4-y^2/2,\quad -2\leq y\leq 2\]
and region $R_2$, corresponding to $D_2$, has bounds
\[4-y^2/2\leq z\leq 6-y^2,\quad -2\leq y\leq 2.\]
Thus the volume of $D$ is given by:
\[\int_{-2}^2\int_2^{4-y^2/2}\int_{-\sqrt{z/2-1}}^{\sqrt{z/2-1}}\dd x\dd z\dd y +\int_{-2}^2\int_{4-y^2/2}^{6-y^2}\int_{-\sqrt{3-y^2/2-z^2/2}}^{\sqrt{3-y^2/2-z^2/2}}\dd x\dd z\dd y.\]
\end{example}

If all one wanted to do in \autoref{ex_trip5} was find the volume of the region $D$, one would have likely stopped at the first integration setup (with order $\dd z\dd y\dd x$) and computed the volume from there. However, we included the other two methods (1) to show that it could be done, ``messy'' or not, and (2) because sometimes we ``have'' to use a less desirable order of integration in order to actually integrate.

\subsection{Triple Integration and Functions of Three Variables}

There are uses for triple integration beyond merely finding volume, just as there are uses for integration beyond ``area under the curve.'' These uses start with understanding how to integrate functions of three variables, which is effectively no different than integrating functions of two variables. This leads us to a definition, followed by an example.

{\tcbset{grow to right by=5em}
\begin{definition}[Iterated Integration, (Part II)]\label{def:triple_integral_2}
Let $D$ be a closed, bounded region in space, over which $g_1(x)$, $g_2(x)$, $f_1(x,y)$, $f_2(x,y)$ and $h(x,y,z)$ are all continuous, and let $a$ and $b$ be real numbers.\bigskip

The \textbf{iterated integral} $\ds \int_a^b\int_{g_1(x)}^{g_2(x)}\int_{f_1(x,y)}^{f_2(x,y)} h(x,y,z)\dd z\dd y\dd x$ is evaluated as
\index{integration!triple}\index{triple integral}\index{iterated integration}
\small
\[
\int_a^b\int_{g_1(x)}^{g_2(x)}\int_{f_1(x,y)}^{f_2(x,y)} h(x,y,z)\dd z\dd y\dd x
= \int_a^b\int_{g_1(x)}^{g_2(x)}\left(\int_{f_1(x,y)}^{f_2(x,y)} h(x,y,z)\dd z\right)\dd y\dd x.
\]
\end{definition}}

\begin{example}[Evaluating a triple integral of a function of three variables]\label{ex_trip6}
Evaluate $\ds \int_0^1\int_{x^2}^x\int_{x^2-y}^{2x+3y} \bigl(xy+2xz\bigr)\dd z\dd y\dd x$.
\solution
We evaluate this integral according to \autoref{def:triple_integral_2}.\bigskip

$\ds \int_0^1\int_{x^2}^x\int_{x^2-y}^{2x+3y} \bigl(xy+2xz\bigr)\dd z\dd y\dd x $
\begin{align*}
			&=	\int_0^1\int_{x^2}^x\left(\int_{x^2-y}^{2x+3y} \bigl(xy+2xz\bigr)\dd z\right)\dd y\dd x\\
			&= \int_0^1\int_{x^2}^x\left(\bigl(xyz+ xz^2\bigr)\Big|_{x^2-y}^{2x+3y}\right)\dd y\dd x\\
			&= \int_0^1\int_{x^2}^x\Biggl(xy(2x+3y)+x(2x+3y)^2-\Bigl(xy(x^2-y)+x(x^2-y)^2\Bigr)\Biggr)\dd y\dd x\\
			&=\int_0^1\int_{x^2}^x\Bigl(-x^5+x^3y+4x^3+14x^2y+12xy^2\Bigr)\dd y\dd x.
			\intertext{We continue as we have in the past, showing fewer steps.}
			&= \int_0^1\Biggl(-\frac72x^7-8x^6-\frac72x^5+15x^4\Biggr)\dd x\\
			&= \frac{281}{336}%\approx 0.836
			.
\end{align*}
\end{example}

We now know \emph{how} to evaluate a triple integral of a function of three variables; we do not yet understand what it \emph{means}. We build up this understanding in a way very similar to how we have understood integration and double integration.

Let $h(x,y,z)$ be a continuous function of three variables, defined over some space region $D$. We can partition $D$ into $n$ rectangular-solid subregions, each with dimensions $\Delta x_i\times\Delta y_i\times\Delta z_i$. Let $(x_i,y_i,z_i)$ be some point in the $i^{\,\text{th}}$ subregion, and consider the product $h(x_i,y_i,z_i)\Delta x_i\Delta y_i\Delta z_i$. It is the product of a function value (that's the $h(x_i,y_i,z_i)$ part) and a small volume $\Delta V_i$ (that's the $\Delta x_i\Delta y_i\Delta z_i$ part). One of the simplest understanding of this type of product is when $h$ describes the density of an object, for then $h\times\text{volume}=\text{mass}$.

We can sum up all $n$ products over $D$. Again letting $\norm{\Delta D}$ represent the length of the longest diagonal of the $n$ rectangular solids in the partition, we can take the limit of the sums of products as $\norm{\Delta D}\to 0$. That is, we can find
\[S = \lim_{\norm{\Delta D}\to 0} \sum_{i=1}^n h(x_i,y_i,z_i)\Delta V_i=\lim_{\norm{\Delta D}\to 0} \sum_{i=1}^n h(x_i,y_i,z_i)\Delta x_i\Delta y_i\Delta z_i.\]

While this limit has lots of interpretations depending on the function $h$, in the case where $h$ describes density, $S$ is the total mass of the object described by the region $D$.

We now use the above limit to define the \textbf{triple integral}, give a theorem that relates triple integrals to iterated iteration, followed by the application of triple integrals to find the centers of mass of solid objects.

\mnote{\textbf{Note:} In the marginal note%
\ifbool{latexml}{%
 \ at \autoref{thm:double_int}%
}{%
 \ on \autopageref{note:doubleint}%
}%
, we showed how the summation of rectangles over a region $R$ in the plane could be viewed as a double sum, leading to the double integral. Likewise, we can view the sum $\ds \sum_{i=1}^nh(x_i,y_i,z_i)\Delta x_i\Delta y_i\Delta z_i$ as a triple sum, \[\sum_{k=1}^p\sum_{j=1}^n\sum_{i=1}^mh(x_i,y_j,z_k)\Delta x_i\Delta y_j\Delta z_k,\]
which we evaluate as
\[\sum_{k=1}^p\left(\sum_{j=1}^n\left(\sum_{i=1}^m h(x_i,y_j,z_k)\Delta x_i\right)\Delta y_j\right)\Delta z_k.\]
Here we fix a $k$ value, which establishes the $z$-height of the rectangular solids on one ``level'' of all the rectangular solids in the space region $D$. The inner double summation adds up all the volumes of the rectangular solids on this level, while the outer summation adds up the volumes of each level.\bigskip\\
This triple summation understanding leads to the $\iiint_D$ notation of the triple integral, as well as the method of evaluation shown in \autoref{thm:triple_integration2}.}

\begin{definition}[Triple Integral]\label{def:triple_integral_3}
Let $w=h(x,y,z)$ be a continuous function over a closed, bounded space region $D$, and let $\Delta D$ be any partition of $D$ into $n$ rectangular solids with volume $\Delta V_i$. The \textbf{triple integral of $h$ over $D$} is
\index{integration!triple}\index{triple integral}\index{iterated integration}
\[
\iiint_Dh(x,y,z)\dd V = \lim_{\norm{\Delta D}\to 0}\sum_{i=1}^n h(x_i,y_i,z_i)\Delta V_i.
\]
\end{definition}

The following theorem assures us that the above limit exists for continuous functions $h$ and gives us a method of evaluating the limit.

\begin{theorem}[Triple Integration (Part II)]\label{thm:triple_integration2}
Let $w=h(x,y,z)$ be a continuous function over a closed, bounded space region $D$, and let $\Delta D$ be any partition of $D$ into $n$ rectangular solids with volume $V_i$.
\index{integration!triple}\index{triple integral}\index{iterated integration}

\begin{enumerate}
\item		The limit $\ds \lim_{\norm{\Delta D}\to 0}\sum_{i=1}^n h(x_i,y_i,z_i)\Delta V_i$ exists.

\item		If $D$ is defined as the region bounded by the planes $x=a$ and $x=b$, the cylinders $y=g_1(x)$ and $y=g_2(x)$, and the surfaces $z=f_1(x,y)$ and $z=f_2(x,y)$, where $a<b$, $g_1(x)\leq g_2(x)$ and $f_1(x,y)\leq f_2(x,y)$ on $D$, then
	\[\iiint_D h(x,y,z)\dd V = \int_a^b\int_{g_1(x)}^{g_2(x)}\int_{f_1(x,y)}^{f_2(x,y)} h(x,y,z)\dd z\dd y\dd x.\]
\end{enumerate}
\end{theorem}

We now apply triple integration to find the centers of mass of solid objects.

\subsection{Mass and Center of Mass}

One may wish to review \autoref{sec:center_of_mass} for a reminder of the relevant terms and concepts. 

\begin{definition}[Mass, Center of Mass of Solids]\label{def:mass_3d}
Let a solid be represented by a region $D$ in space with variable density function $\delta(x,y,z)$. 
\index{moment}\index{center of mass}\index{mass}
\begin{enumerate}
	\item The \textbf{mass} of the object is $\ds M= \iiint_D \dd m=\iiint_D \delta(x,y,z)\dd V$.
	\item	The \textbf{moment about the $x$-$y$ plane} is $\ds M_{xy}=\iiint_D z\delta(x,y,z)\dd V$.
	\item	The \textbf{moment about the $x$-$z$ plane} is $\ds M_{xz}=\iiint_D y\delta(x,y,z)\dd V$.
	\item	The \textbf{moment about the $y$-$z$ plane} is $\ds M_{yz}=\iiint_D x\delta(x,y,z)\dd V$.
	\item The \textbf{center of mass} of the object is
	\[\bigl(\overline{x},\overline{y},\overline{z}\bigr) = \left(\frac{M_{yz}}M,\frac{M_{xz}}M,\frac{M_{xy}}M\right).\]
\end{enumerate}
\end{definition}

\begin{example}[Finding the center of mass of a solid]\label{ex_trip7}
Find the mass and center of mass of the solid represented by the space region bounded by the coordinate planes and $z=2-y/3-2x/3$, shown in \autoref{fig:trip7}, with constant density $\delta(x,y,z)=3$g/cm$^3$. (Note: this space region was used in \autoref{ex_trip2}.)

\mtable{Finding the center of mass of this solid in \autoref{ex_trip7}.}{fig:trip7}{%
\myincludeasythree{width=\marginparwidth,
3Droll=0,
3Dortho=0.004707579035311937,
3Dc2c=0.4086907207965851 0.6385185718536377 0.6521241664886475,
3Dcoo=3.675152540206909 -7.818903923034668 16.216196060180664,
3Droo=150.00000119319247}{width=\marginparwidth}{figures/figtrip2_3D}}
\solution
We apply \autoref{def:mass_3d}. In \autoref{ex_trip2}, we found bounds for the order of integration $\dd z\dd y\dd x$ to be $0\leq z\leq 2-y/3-2x/3$, $0\leq y\leq 6-2x$ and $0\leq x\leq 3$. We find the mass of the object: 
\begin{align*}
M &= \iiint_D \delta(x,y,z)\dd V \\
  &= \int_0^3\int_0^{6-2x}\int_0^{2-y/3-2x/3} \bigl(3\bigr)\dd z\dd y\dd x\\
	&= 3\int_0^3\int_0^{6-2x}\int_0^{2-y/3-2x/3} \dd z\dd y\dd x\\
	&= 3(6) = 18\text{g}.
\end{align*}
The evaluation of the triple integral is done in \autoref{ex_trip2}, so we skipped those steps above. Note how the mass of an object with constant density is simply ``density$\times$volume.''

We now find the moments about the planes.
\begin{align*}
M_{xy} &= \iiint_D 3z\dd V \\
			&= \int_0^3\int_0^{6-2x}\int_0^{2-y/3-2x/3} \bigl(3z\bigr)\dd z\dd y\dd x\\
			%&= \int_0^3\int_0^{6-2x} \frac16\bigl(2x+y-6\bigr)^2\dd y\dd x \\
			&= \int_0^3\int_0^{6-2x} \frac32\bigl(2-y/3-2x/3\bigr)^2\dd y\dd x \\
			&= \int_0^3 -\frac49\bigl(x-3\bigr)^3\dd x\\
			&= 9.
\end{align*}

We omit the steps of integrating to find the other moments.
\begin{align*}
M_{yz} &= \iiint_D 3x\dd V = \frac{27}2.\\
M_{xz} &= \iiint_D 3y\dd V = 27.
\end{align*}
The center of mass is
\[\bigl(\overline{x},\overline{y},\overline{z}\bigr) = \left(\frac{27/2}{18},\frac{27}{18},\frac{9}{18}\right) = \bigl(0.75,1.5,0.5\bigr).\]
\end{example}


\begin{example}[Finding the center of mass of a solid]\label{ex_trip8}
Find the center of mass of the solid represented by the region bounded by the planes $z=0$ and $z=-y$ and the cylinder $x^2+y^2=1$, shown in \autoref{fig:trip8}, with density function $\delta(x,y,z) = 10+x^2+5y-5z$. (Note: this space region was used in \autoref{ex_trip3}.)

\mtable{Finding the center of mass of this solid in \autoref{ex_trip8}.}{fig:trip8}{%
\myincludeasythree{width=\marginparwidth,
3Droll=0,
3Dortho=0.004735102877020836,
3Dc2c=0.8340418338775635 0.34454345703125 0.43088746070861816,
3Dcoo=-6.1953253746032715 -30.81180191040039 48.79259490966797,
3Droo=149.99999973988238}{width=\marginparwidth}{figures/figtrip3_3D}}
\solution
As we start, consider the density function. It is symmetric about the $y$-$z$ plane, and the farther one moves from this plane, the denser the object is. The symmetry indicates that $\overline x$ should be 0. 

As one moves away from the origin in the $y$ or $z$ directions, the object becomes less dense, though there is more volume in these regions.  

Though none of the integrals needed to compute the center of mass are particularly hard, they do require a number of steps. We emphasize here the importance of knowing how to set up the proper integrals; in complex situations we can appeal to technology for a good approximation, if not the exact answer. We use the order of integration $\dd z\dd y\dd x$, using the bounds found in \autoref{ex_trip3}. (As these are the same for all four triple integrals, we explicitly show the bounds only for $M$.)

\begin{align*}
	M
	&= \iiint_D \bigl(10+x^2+5y-5z\bigr)\dd V \\
	&= \int_{-1}^1\int_{-\sqrt{1-x^2}}^0\int_0^{-y} \bigl(10+x^2+5y-5z\bigr)\dd V\\
	&= \frac{64}5-\frac{15\pi}{16} \approx 3.855.\\
	M_{yz}
	&= \iiint_D x\bigl(10+x^2+5y-5z\bigr)\dd V = 0.\\
	M_{xz}
	&= \iiint_D y\bigl(10+x^2+5y-5z\bigr)\dd V = 2-\frac{61\pi}{48}\approx -1.99.\\
	M_{xy}
	&= \iiint_D z\bigl(10+x^2+5y-5z\bigr)\dd V = \frac{61\pi}{96}-\frac{10}9\approx 0.885.
\end{align*}
Note how $M_{yz}=0$, as expected. The center of mass is
\[
\bigl(\overline{x},\overline{y},\overline{z}\bigr)
\approx \left(0,\frac{-1.99}{3.855},\frac{0.885}{3.855}\right) \approx \bigl(0,-0.516, 0.230\bigr).
\]
\end{example}

As stated before, there are many uses for triple integration beyond finding volume. When $h(x,y,z)$ describes a rate of change function over some space region $D$, then $\ds \iiint_D h(x,y,z)\dd V$ gives the total change over $D$. Our one specific example of this was computing mass; a density function is simply a ``rate of mass change per volume'' function. Integrating density gives total mass.

While knowing \emph{how to integrate}  is important, it is arguably much more important to know \emph{how to set up} integrals. It takes skill to create a formula that describes a desired quantity; modern technology is very useful in evaluating these formulas quickly and accurately.

\printexercises{exercises/13-06-exercises}
