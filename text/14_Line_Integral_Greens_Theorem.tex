\section{Flow, Flux, Green's Theorem and the Divergence Theorem}\label{sec:greensthm}

\subsection{Flow and Flux}

\mtable{Illustrating the principles of flow and flux.}{fig:flowfluxintro}{\begin{tikzpicture}
\begin{axis}[width=1.16\marginparwidth,height=1.16\marginparwidth,
tick label style={font=\scriptsize},
axis y line=middle,axis x line=middle,name=myplot,axis on top,axis equal,
xtick={1},ytick={1},
ymin=-.2,ymax=1.2,xmin=-.2,xmax=1.2]
\foreach \x in {-.2,0,...,1.11} {
    \foreach \y in {-.1,.1,...,1.11} {
        \edef\temp{\noexpand\draw[->,lightgray](axis cs:{\x},{\y})--(axis cs:{\x+.15},{\y});}
        \temp
    }
}
\draw[thick,draw={\colorone},->](axis cs:.2,.2)--(axis cs:.2,.6);
\draw[thick,draw={\colorone}](axis cs:.2,.6)--(axis cs:.2,1)node[pos=.5,left]{$C_2$};
\draw[thick,draw={\colorone},->,dashed](axis cs:.2,1)--(axis cs:.6,.6)node[below,pos=1]{$C_3$};
\draw[thick,draw={\colorone},dashed](axis cs:.6,.6)--(axis cs:1,.2);
\draw[thick,draw={\colorone},->](axis cs:.2,.2)--(axis cs:.6,.2);
\draw[thick,draw={\colorone}](axis cs:.6,.2)--(axis cs:1,.2)node[pos=.5,below]{$C_1$};
\end{axis}
\node [right] at (myplot.right of origin) {\scriptsize $x$};
\node [above] at (myplot.above origin) {\scriptsize $y$};
\end{tikzpicture}}

Line integrals over vector fields have the natural interpretation of computing work when $\vec F$ represents a force field. It is also common to use vector fields to represent velocities. In these cases, the line integral $\int_C \vec F\cdot\dd\vec r$ is said to represent \textbf{flow}.\index{flow}\index{flux}\index{circulation}

Let the vector field $\vec F=\bracket{1,0}$ represent the velocity of water as it moves across a smooth surface, depicted in \autoref{fig:flowfluxintro}. A line integral over $C$ will compute ``how much water is moving \emph{along} the path $C$.'' 

In the figure, ``all'' of the water above $C_1$ is moving along that curve, whereas ``none'' of the water above $C_2$ is moving along that curve (the curve and the flow of water are at right angles to each other). Because $C_3$ has nonzero horizontal and vertical components, ``some'' of the water above that curve is moving along the curve.

When $C$ is a closed curve, we call flow \textbf{circulation}, represented by $\oint_C \vec F\cdot\dd\vec r$.

% This section on unit could be returned ...
%The units of flow require understanding. If $\vec F$ has units feet/second and $C$ is measured in feet, then $\vec F\cdot\dd\vec r$ has units ft/s$\cdot$ft = ft$^2$/s, an ``area of water per second.'' We generally do not measure quantity of water by area, but rather by volume (surface area \emph{covered} by water is an entirely different matter). We resolve this by recognizing that inherently our water flowing across the surface has \emph{depth}; by accounting for this extra dimension, the units become ft$^3$/s.

The ``opposite'' of flow is \textbf{flux}, a measure of ``how much water is moving \emph{across} the path $C$.'' If a curve represents a filter in flowing water, flux measures how much water will pass through the filter. Considering again \autoref{fig:flowfluxintro}, we see that a screen along $C_1$ will not filter any water as no water passes across that curve. Because of the nature of this field,  $C_2$ and $C_3$ each filter the same amount of water per second. 

The terms ``flow'' and ``flux'' are used apart from velocity fields, too. Flow is measured by $\int_C \vec F\cdot\dd\vec r$, which is the same as $\int_C \vec F\cdot\vec T\dd s$ by \autoref{def:line_integral2}. That is, flow is a summation of the amount of $\vec F$ that is \emph{tangent} to the curve $C$. %is the measure of ``how much of the field is tangent to (or, in line with) the curve.'' 

By contrast, flux is a summation of the amount of $\vec F$ that is \emph{orthogonal} to the direction of travel. To capture this orthogonal amount of $\vec F$, we use $\int_C \vec F \cdot \vec n\dd s$ to measure flux, where $\vec n$ is a unit vector orthogonal to the curve $C$. %is the measure of ``how much of the field is orthogonal to (or, across) the curve.'' 
(Later, we'll measure flux across surfaces, too. %Whenever we want to measure the amount of something across a curve or surface, we will be measuring flux. 
For example, in physics it is useful to measure the amount of a magnetic field that passes through a surface.)

%The terms ``flow'' and ``flux'' are used apart from velocity fields, too. Flow is the measure of ``how much of the field is tangent to (or, in line with) the curve.'' Flux is the measure of ``how much of the field is orthogonal to (or, across) the curve.'' Later, we'll measure flux across surfaces, too. %Whenever we want to measure the amount of something across a curve or surface, we will be measuring flux. 
%For example, in physics it is useful to measure the amount of a magentic field that passes across a surface: this is a measure of flux.

%In the plane, flow is measured by $\int_C \vec F\cdot\dd\vec r$, which is the same as $\int_C \vec F\cdot\vec T\dd s$ by \autoref{def:line_integral2}. That is, flow is a summation of the amount of $\vec F$ that is \emph{parallel} to the direction of travel $\vec T$. 
%
%By contrast, flux is a summation of the amount of $\vec F$ that is \emph{orthogonal} to the direction of travel. To capture this orthogonal amount of $\vec F$, we use $\int_C \vec F \cdot \vec n\dd s$ to measure flux, where $\vec n$ is a vector orthogonal to the direction of travel. 

How is $\vec n$ determined? We'll later see that if $C$ is a closed curve, we'll want $\vec n$ to point to the outside of the curve (measuring how much is ``going out''). We'll also adopt the convention that closed curves should be traversed counterclockwise. 

\mtable{Determining ``counterclockwise'' is not always simple without a good definition.}{fig:fluxcounterclockwise}{\begin{tikzpicture}
\begin{axis}[width=1.16\marginparwidth,height=1.16\marginparwidth,
tick label style={font=\scriptsize},
axis y line=middle,axis x line=middle,name=myplot,axis on top,axis equal,
xtick={-1,1},ytick={-1,1},
ymin=-1.2,ymax=1.2,xmin=-1.2,xmax=1.2]
\addplot [smooth,thick, draw={\colorone},samples=40,domain=0:pi] ({cos(deg(x))},{sin(deg(x))});
\addplot [smooth,thick, draw={\colorone},samples=40,domain=pi:2*pi] ({cos(deg(x))},{-sin(deg(3*(x-pi)))/3})node(dot)[pos=.5]{};
\draw[fill=black](dot)circle(2pt);
\draw(dot)node[above right]{$A$};
\end{axis}
\node [right] at (myplot.right of origin) {\scriptsize $x$};
\node [above] at (myplot.above origin) {\scriptsize $y$};
\end{tikzpicture}}

(If $C$ is a complicated closed curve, it can be difficult to determine what ``counterclockwise'' means. Consider \autoref{fig:fluxcounterclockwise}. Seeing the curve as a whole, we know which way ``counterclockwise'' is. If we zoom in on point $A$, one might incorrectly choose to traverse the path in the wrong direction. So we offer this definition: \emph{a closed curve is being traversed counterclockwise if the outside is to the right of the path and the inside is to the left.})

When a curve $C$ is traversed counterclockwise by $\vrt =\bracket{f(t), g(t)}$, we rotate $\vec T$ clockwise 90$^\circ$  to obtain $\vec n$:
\[
\vec T = \frac{\bracket{\fp(t),g'(t)}}{\norm{\vrp(t)}} \quad \Rightarrow \quad \vec n = \frac{\bracket{g'(t),-\fp(t)}}{\norm{\vrp(t)}}.
\]

Letting $\vec F =\bracket{M,N}$, we calculate flux as:
\begin{align*}
\int_C \vec F\cdot \vec n\dd s &= \int_C \vec F\cdot \frac{\bracket{g'(t),-\fp(t)}}{\norm{\vrp(t)}} \norm{\vrp(t)}\dd t \\
				&= \int_C\bracket{M,N}\cdot\bracket{g'(t),-\fp(t)}\dd t \\
				&= \int_C \Bigl(M\,g'(t) - N\,\fp(t)\Bigr)\dd t\\
				&= \int_C M\,g'(t)\dd t - \int_C N\,\fp(t)\dd t.\\
				\intertext{As the $x$ and $y$ components of $\vrt$ are $f(t)$ and $g(t)$ respectively, the differentials of $x$ and $y$ are $\dd x = \fp(t)\dd t$ and $\dd y=g'(t)\dd t$. We can then write the above integrals as:}
				&= \int_C M\dd y - \int_C N\dd x.\\
				\intertext{This is often written as one integral (not incorrectly, though somewhat confusingly, as this one integral has two ``$\dd$\,'s''):}
				&=\int_CM\dd y -N\dd x.
\end{align*}
We summarize the above in the following definition.

\begin{definition}[Flow, Flux]\label{def:flowflux}
Let $\vec F=\bracket{M,N}$ be a vector field with continuous components defined on a smooth curve $C$, parameterized by $\vrt =\bracket{f(t),g(t)}$, let $\vec T$ be the unit tangent vector of $\vrt$, and let $\vec n$ be the clockwise 90$^\circ$degree rotation of $\vec T$.\index{flow}\index{flux}
\begin{itemize}
	\item The \textbf{flow} of $\vec F$ along $C$ is
\[\int_C \vec F\cdot\vec T\dd s=\int_C \vec F\cdot\dd\vec r.\]
	\item The \textbf{flux} of $\vec F$ across $C$ is
\[
\int_C \vec F\cdot \vec n\dd s =  \int_C M\dd y -N\dd x = \int_C\Bigl(M\,g'(t) - N\,\fp(t)\Bigr)\dd t.
\]
\end{itemize}
\end{definition}

This definition of flow also holds for curves in space, though it does not make sense to measure ``flux across a curve'' in space.

Measuring flow is essentially the same as finding work performed by a force as done in the previous examples. Therefore we practice finding only flux in the following example.

\mtable{Illustrating the curves and vector fields in \autoref{ex_flux1}. In (a) the vector field is $\vec F_1$, and in (b) the vector field is $\vec F_2$.}{fig:flux1}{\begin{tikzpicture}
\begin{axis}[width=1.16\marginparwidth,height=1.16\marginparwidth,
tick label style={font=\scriptsize},
axis y line=middle,axis x line=middle,name=myplot,axis on top,axis equal,
xtick={1},ytick={1},
ymin=-.2,ymax=1.2,xmin=-.2,xmax=1.2]
\foreach \x in {-.1,0,...,1.11} {
    \foreach \y in {-.1,0,...,1.11} {
		\edef\vx{\y}
		\edef\vy{1-(\x)}
        \edef\temp{\noexpand\draw[->,lightgray](axis cs:{\x-(\vx)/30},{\y-(\vy)/30})--(axis cs:{\x+(\vx)/30},{\y+(\vy)/30});}
        \temp
    }
}
\addplot [smooth,thick, draw={\colorone},samples=40,domain=0:.5,->] ({1-x},{x})
node[pos=1,below left]{$C_1$};
\addplot [smooth,thick, draw={\colorone},samples=40,domain=.5:1] ({1-x},{x});
\addplot [smooth,thick, draw={\colortwo},samples=40,domain=0:pi/4,->] ({cos(deg(x))},{sin(deg(x))})node[pos=1,above right]{$C_2$};
\addplot [smooth,thick, draw={\colortwo},samples=40,domain=pi/4:pi/2] ({cos(deg(x))},{sin(deg(x))});
\end{axis}
\node [right] at (myplot.right of origin) {\scriptsize $x$};
\node [above] at (myplot.above origin) {\scriptsize $y$};
\end{tikzpicture}
\\(a)\\
\begin{tikzpicture}
\begin{axis}[width=1.16\marginparwidth,height=1.16\marginparwidth,
tick label style={font=\scriptsize},
axis y line=middle,axis x line=middle,name=myplot,axis on top,axis equal,
xtick={1},ytick={1},
ymin=-.2,ymax=1.2,xmin=-.2,xmax=1.2]
\foreach \x in {-.1,0,...,1.11} {
    \foreach \y in {-.1,0,...,1.11} {
		\edef\vx{(-\x)}
		\edef\vy{(2*\y-(\x))}
        \edef\temp{\noexpand\draw[->,lightgray](axis cs:{\x-(\vx)/30},{\y-(\vy)/30})--(axis cs:{\x+(\vx)/30},{\y+(\vy)/30});}
        \temp
    }
}
\addplot [smooth,thick, draw={\colorone},samples=40,domain=0:.5,->] ({1-x},{x})
node[pos=1,below left]{$C_1$};
\addplot [smooth,thick, draw={\colorone},samples=40,domain=.5:1] ({1-x},{x});
\addplot [smooth,thick, draw={\colortwo},samples=40,domain=0:pi/4,->] ({cos(deg(x))},{sin(deg(x))})node[pos=1,above right]{$C_2$};
\addplot [smooth,thick, draw={\colortwo},samples=40,domain=pi/4:pi/2] ({cos(deg(x))},{sin(deg(x))});
\end{axis}
\node [right] at (myplot.right of origin) {\scriptsize $x$};
\node [above] at (myplot.above origin) {\scriptsize $y$};
\end{tikzpicture}
\\(b)}

\begin{example}[Finding flux across curves in the plane]\label{ex_flux1}
Curves $C_1$ and $C_2$ each start at $(1,0)$ and end at $(0,1)$, where $C_1$ follows the line $y=1-x$ and $C_2$ follows the unit circle, as shown in \autoref{fig:flux1}. Find the flux across both curves for the vector fields $\vec F_1 =\bracket{y,-x+1}$ and $\vec F_2 =\bracket{-x,2y-x}$.
\solution
We begin by finding parameterizations of $C_1$ and $C_2$. As done in \autoref{ex_livf3}, parameterize $C_1$ by creating the line that starts at $(1,0)$ and moves in the $\bracket{-1,1}$ direction: $\vec r_1(t) =\bracket{1,0}+t\bracket{-1,1}=\bracket{1-t,t}$, for $0\leq t\leq 1$. We parameterize $C_2$ with the familiar $\vec r_2(t) =\bracket{\cos t,\sin t}$ on $0\leq t\leq \pi/2$. For reference later, we give each function and its derivative below:
\begin{gather*}
\vec r_1(t) =\bracket{1-t,t},\quad\vrp_1(t)=\bracket{-1,1}.\\
\vec r_2(t) =\bracket{\cos t,\sin t}, \quad \vrp_2(t) =\bracket{-\sin t,\cos t}.
\end{gather*}

When $\vec F = \vec F_1 =\bracket{y,-x+1}$ (as shown in \autoref{fig:flux1}(a)), over $C_1$ we have $M = y =t$ and $N = -x+1 = -(1-t)+1 = t$. Using \autoref{def:flowflux}, we compute the flux:

\begin{align*}
\int_{C_1} \vec F\cdot \vec n\dd s & = \int_{C_1} \Bigl(M\,g'(t) - N\,\fp(t)\Bigr)\dd t\\
			&= \int_0^1 \Bigl( t(1) - t(-1)\Bigr)\dd t \\
			&= \int_0^1 2t\dd t\\
			&= 1.
\end{align*}
Over $C_2$, we have $M = y = \sin t$ and $N = -x+1 = 1-\cos t$. Thus the flux across $C_2$ is:
\begin{align*}
	\int_{C_1} \vec F\cdot \vec n\dd s
	& = \int_{C_1} \Bigl(M\,g'(t) - N\,\fp(t)\Bigr)\dd t\\
	&= \int_0^{\pi/2}\Bigl((\sin t)(\cos t) - (1-\cos t)(-\sin t)\Bigr)\dd t\\
	&= \int_0^{\pi/2} \sin t\dd t\\
	&=1.
\end{align*}
Notice how the flux was the same across both curves. This won't hold true when we change the vector field.

When $\vec F=\vec F_2=\bracket{-x,2y-x}$ (as shown in \autoref{fig:flux1}(b)), over $C_1$ we have $M=-x=t-1$ and $N=2y-x=2t-(1-t)=3t-1$. Computing the flux across $C_1$:
\begin{align*}
	\int_{C_1} \vec F\cdot \vec n\dd s
	& = \int_{C_1} \Bigl(M\,g'(t) - N\,\fp(t)\Bigr)\dd t\\
	&= \int_0^1 \Bigl( (t-1)(1) - (3t-1)(-1)\Bigr)\dd t \\
	&= \int_0^1 (4t-2)\dd t\\
	&= 0.
\end{align*}
Over $C_2$, we have $M = -x = -\cos t$ and $N = 2y-x = 2\sin t-\cos t$. Thus the flux across $C_2$ is:

\begin{align*}
	\int_{C_1} \vec F\cdot \vec n\dd s
	& = \int_{C_1} \Bigl(M\,g'(t) - N\,\fp(t)\Bigr)\dd t\\
	&= \int_0^{\pi/2}\Bigl((-\cos t)(\cos t) - (2\sin t-\cos t)(-\sin t)\Bigr)\dd t\\
	&= \int_0^{\pi/2} \bigl(2\sin^2 t-\sin t\cos t-\cos^2t\bigr)\dd t\\
	&=\pi/4 - 1/2\approx 0.285.
\end{align*}
We analyze the results of this example below.
\end{example}

In \autoref{ex_flux1}, we saw that the flux across the two curves was the same when the vector field was $\vec F_1 =\bracket{y, -x+1}$. This is not a coincidence. We show why they are equal in \autoref{ex_div2}. In short, the reason is this: the divergence of $\vec F_1$ is 0, and when $\divv \vec F = 0$, the flux across any two paths with common beginning and ending points will be the same.

We also saw in the example that the flux across $C_1$ was 0 when the field was $\vec F_2 =\bracket{-x, 2y-x}$. Flux measures ``how much'' of the field crosses the path from left to right (following the conventions established before). Positive flux means most of the field is crossing from left to right; negative flux means most of the field is crossing from right to left; zero flux means the same amount crosses from each side. When we consider \autoref{fig:flux1}(b), it seems plausible that the same amount of $\vec F_2$ was crossing $C_1$ from left to right as from right to left.

%In short, the flux across $C_2$ was different than the flux across $C_1$ because $\divv \vec F_2 \neq 0$.
%
%In the next section, we'll investigate the connection between flux and divergence, as well as the connection between flow and curl.

\subsection{Green's Theorem}

There is an important connection between the circulation around a closed region $R$ and the curl of the vector field inside of $R$, as well as a connection between the flux across the boundary of $R$ and the divergence of the field inside $R$. These connections are described by Green's Theorem and the Divergence Theorem, respectively. We'll explore each in turn.\index{Green's Theorem}

Green's Theorem states ``the counterclockwise circulation around a closed region $R$ is equal to the sum of the curls over $R$.''

\begin{theorem}[Green's Theorem]\label{thm:greens}
Let $R$ be a closed, bounded region of the plane whose boundary $C$ is composed of finitely many smooth curves, let $\vec r(t)$ be a counterclockwise parameterization of $C$, and let $\vec F =\bracket{M,N}$ where $N_x$ and $M_y$ are continuous over $R$. Then\index{Green's Theorem}
\[\oint_C \vec F\cdot\dd\vec r = \iint_R\curl \vec F\dd A.\]
\end{theorem}

\youtubeVideo{a_zdFvYXX_c}{Green's Theorem}

We'll explore Green's Theorem through an example.

\begin{example}[Confirming Green's Theorem]\label{ex_green1}
%
\mtable{The vector field and planar region used in \autoref{ex_green1}.}{fig:green1}{\begin{tikzpicture}
\begin{axis}[width=1.16\marginparwidth,height=1.16\marginparwidth,
tick label style={font=\scriptsize},
axis y line=middle,axis x line=middle,name=myplot,axis on top,axis equal,
xtick={-1,1},ytick={1,2},
ymin=-.2,ymax=2.2,xmin=-1.2,xmax=1.2]
\foreach \x in {-1.1,-.9,...,1.11} {
    \foreach \y in {-.1,.1,...,2.11} {
		\edef\vx{((-\y)/50)}
		\edef\vy{(((\x)*(\x)+1)/50)}
        \edef\temp{\noexpand\draw[->,lightgray](axis cs:{\x-\vx},{\y-\vy})--(axis cs:{\x+\vx},{\y+\vy});}
        \temp
    }
}
\draw[thick,draw={\colorone},->](axis cs:0,0)--(axis cs:1,0)--(axis cs:.5,1);
\draw[thick,draw={\colorone},->](axis cs:.5,1)--(axis cs:0,2)--(axis cs:-.5,1);
\draw[thick,draw={\colorone},->](axis cs:-.5,1)--(axis cs:-1,0)--(axis cs:0,0);
\draw(axis cs:0,.5)node[right]{$R$};
\end{axis}
\node [right] at (myplot.right of origin) {\scriptsize $x$};
\node [above] at (myplot.above origin) {\scriptsize $y$};
\end{tikzpicture}}
%
Let $\vec F =\bracket{-y,x^2+1}$ and let $R$ be the region of the plane bounded by the triangle with vertices $(-1,0)$, $(1,0)$ and $(0,2)$, shown in \autoref{fig:green1}. Verify Green's Theorem; that is, find the circulation of $\vec F$ around the boundary of $R$ and show that is equal to the double integral of $\curl \vec F$ over $R$.
\solution
The curve $C$ that bounds $R$ is composed of 3 lines. While we need to traverse the boundary of $R$ in a counterclockwise fashion, we may start anywhere we choose. We arbitrarily choose to start at $(-1,0)$, move to $(1,0)$, etc., with each line parameterized by $\vec r_1(t)$, $\vec r_2(t)$ and $\vec r_3(t)$, respectively.

We leave it to the reader to confirm that the following parameterizations of the three lines are accurate:

\begin{align*}
\vec r_1(t) &=\bracket{2t-1,0}, && \text{for }0\leq t\leq 1,\quad\text{with }\vrp_1(t) =\bracket{ 2,0},\\
\vec r_2(t) &=\bracket{1-t,2t}, && \text{for }0\leq t\leq 1,\quad\text{with }\vrp_2(t) =\bracket{ -1,2}, and\\
\vec r_3(t) &=\bracket{-t,2-2t}, && \text{for }0\leq t\leq 1,\quad\text{with } \vrp_3(t) =\bracket{ -1,-2}.
\end{align*}

The circulation around $C$ is found by summing the flow along each of the sides of the triangle. We again leave it to the reader to confirm the following computations:
\begin{align*}
\int_{C_1}\vec F\cdot\dd\vec r_1 &= \int_0^1\bracket{0,(2t-1)^2+1}\cdot\bracket{2,0}\dd t = 0,\\
\int_{C_2}\vec F\cdot\dd\vec r_2 &= \int_0^1\bracket{-2t,(1-t)^2+1}\cdot\bracket{-1,2}\dd t = 11/3, \text{and}\\
\int_{C_3}\vec F\cdot\dd\vec r_3 &= \int_0^1\bracket{2t-2,t^2+1}\cdot\bracket{-1,-2}\dd t = -5/3.
\end{align*}
The circulation is the sum of the flows: $2$.

We confirm Green's Theorem by computing $\iint_R \curl \vec F\dd A$. We find $\curl \vec F = 2x+1$. The region $R$ is bounded by the lines $y = 2x+2$, $y=-2x+2$ and $y=0$. Integrating with the order $\dd x\dd y$ is most straightforward, leading to
\[\int_0^2\int_{y/2-1}^{1-y/2} (2x+1)\dd x\dd y = \int_0^2 (2-y)\dd y = 2,\]
which matches our previous measurement of circulation.
\end{example}

\begin{example}[Using Green's Theorem]\label{ex_green2}
%
\mtable{The vector field and planar region used in \autoref{ex_green2}.}{fig:green2}{\begin{tikzpicture}
\begin{axis}[width=1.16\marginparwidth,height=1.16\marginparwidth,
tick label style={font=\scriptsize},
axis y line=middle,axis x line=middle,name=myplot,axis on top,axis equal,
xtick={-2,2},ytick={2,2},
ymin=-2.4,ymax=2.4,xmin=-2.4,xmax=2.4]
\foreach \x in {-2.2,-1.8,...,2.21} {
    \foreach \y in {-2.2,-1.8,...,2.21} {
		\edef\vx{(sin(deg(\x))/8)}
		\edef\vy{(cos(deg(\y))/8)}
        \edef\temp{\noexpand\draw[->,lightgray](axis cs:{\x-\vx},{\y-\vy})--(axis cs:{\x+\vx},{\y+\vy});}
        \temp
    }
}
\draw(axis cs:0,.5)node[right]{$R$};
\addplot [smooth,thick, draw={\colorone},samples=100,domain=.1-pi:.1+pi,->] ({2*cos(deg(x))+cos(10*deg(x))/10},{2*sin(deg(x))+sin(10*deg(x))/10});
\end{axis}
\node [right] at (myplot.right of origin) {\scriptsize $x$};
\node [above] at (myplot.above origin) {\scriptsize $y$};
\end{tikzpicture}}
%
Let $\vec F =\bracket{\sin x,\cos y}$ and let $R$ be the region enclosed by the curve $C$ parameterized by $\vec r(t) =\bracket{2\cos t+ \frac1{10}\cos(10t),2\sin t+\frac1{10}\sin(10t)}$ on $0\leq t\leq 2\pi$, as shown in \autoref{fig:green2}. Find the circulation around $C$.
\solution
Computing the circulation using the line integral looks difficult, as the integrand will include terms like ``$\sin\bigl(2\cos t + \frac1{10}\cos(10t)\bigr)$.'' 

Green's Theorem states that $\oint_C\vec F\cdot\dd\vec r = \iint_R \curl\vec F\dd A$; since $\curl \vec F = 0$ in this example, the double integral is simply 0 and hence the circulation is 0.

Since $\curl \vec F = 0$, we can conclude that the circulation is 0 in two ways. One method is to employ Green's Theorem as done above. The second way is to recognize that $\vec F$ is a conservative field, hence there is a function $z=f(x,y)$ wherein $\vec F = \nabla f$. Let $A$ be any point on the curve $C$; since $C$ is closed, we can say that $C$ ``begins'' and ``ends'' at $A$. By the Fundamental Theorem of Line Integrals, $\oint_C \vec F\dd\vec r = f(A)-f(A) = 0$.
\end{example}

One can use Green's Theorem to find the area of an enclosed region by integrating along its boundary. Let $C$ be a closed curve, enclosing the region $R$, parameterized by $\vec r(t) =\bracket{f(t),g(t)}$. We know the area of $R$ is computed by the double integral $\iint_R \dd A$, where the integrand is $1$. By creating a field $\vec F$ where $\curl \vec F =1$, we can employ Green's Theorem to compute the area of $R$ as $\oint_C \vec F\cdot\dd\vec r$. 

One is free to choose any field $\vec F$ to use as long as $\curl\vec F = 1$. Common choices are $\vec F =\bracket{0,x}$, $\vec F =\bracket{-y,0}$ and $\vec F =\bracket{-y/2,x/2}$. We demonstrate this below.

%Note: take this sentence out and make as exercise Choosing $\vec F =\bracket{-y,0}$ has special significance. This field leads to the line integral $\oint_C\bracket{0,-f(t)}\cdot\bracket{\fp(t),\gp(t)}\dd t$.

\begin{example}[Using Green's Theorem to find area]\label{ex_green3}
Let $C$ be the closed curve parameterized by $\vrt =\bracket{t-t^3,t^2}$ on $-1\leq t\leq 1$, enclosing the region $R$, as shown in \autoref{fig:green3}. Find the area of $R$.%
\mtable{The region $R$, whose area is found in \autoref{ex_green3}.}{fig:green3}{\begin{tikzpicture}
\begin{axis}[width=1.16\marginparwidth,height=1.16\marginparwidth,
tick label style={font=\scriptsize},
axis y line=middle,axis x line=middle,name=myplot,axis on top,axis equal,
xtick={-1,1},ytick={1},
ymin=-.6,ymax=1.6,xmin=-1.1,xmax=1.1]
\draw(axis cs:0,.5)node[right]{$R$};
\addplot [smooth,thick, draw={\colorone},samples=100,domain=-1:1] ({x-x*x*x},{x*x});
\end{axis}
\node [right] at (myplot.right of origin) {\scriptsize $x$};
\node [above] at (myplot.above origin) {\scriptsize $y$};
\end{tikzpicture}}
\solution
We can choose any field $\vec F$, as long as $\curl \vec F = 1$. We choose $\vec F =\bracket{-y,0}$. We also confirm (left to the reader) that $\vrt$ traverses the region $R$ in a counterclockwise fashion. Thus
\begin{align*}
	\text{Area of $R$}
	&= \iint_R\dd A \\
	&= \oint_C \vec F\cdot\dd\vec r\\
	&= \int_{-1}^1\bracket{-t^2,0}\cdot\bracket{1-3t^2,2t}\dd t\\
	&= \int_{-1}^1 (-t^2)(1-3t^2)\dd t \\
	&= \frac8{15}.
\end{align*}
\end{example}

\subsection{The Divergence Theorem}

Green's Theorem makes a connection between the circulation around a closed region $R$ and the sum of the curls over $R$. The Divergence Theorem makes a somewhat ``opposite'' connection: the total flux across the boundary of $R$ is equal to the sum of the divergences over $R$. 

\begin{theorem}[The Divergence Theorem (in the plane)]\label{thm:divergence1}
Let $R$ be a closed, bounded region of the plane whose boundary $C$ is composed of finitely many smooth curves, let $\vec r(t)$ be a counterclockwise parameterization of $C$, and let $\vec F =\bracket{M,N}$ where $M_x$ and $N_y$ are continuous over $R$. Then\index{Divergence Theorem!in the plane}
\[\oint_C \vec F\cdot \vec n\dd s = \iint_R\divv \vec F\dd A.\]
\end{theorem}

\mtable{The region $R$ used in \autoref{ex_div1}.}{fig:div1}{\begin{tikzpicture}
\begin{axis}[width=1.16\marginparwidth,height=1.16\marginparwidth,
tick label style={font=\scriptsize},
axis y line=middle,axis x line=middle,name=myplot,axis on top,axis equal,
xtick={-2,2},ytick={-2,2},
ymin=-2.4,ymax=2.4,xmin=-2.4,xmax=2.4]
\foreach \x in {-2.2,-1.8,...,2.21} {
    \foreach \y in {-2.2,-1.8,...,2.21} {
		\edef\vx{((\x-\y)/24)}
		\edef\vy{((\x+\y)/24)}
        \edef\temp{\noexpand\draw[->,lightgray](axis cs:{\x-\vx},{\y-\vy})--(axis cs:{\x+\vx},{\y+\vy});}
        \temp
    }
}
\draw(axis cs:0,.5)node[right]{$R$};
\addplot [smooth,thick, draw={\colorone},samples=100,domain=.1-pi:.1+pi,->] ({2*cos(deg(x))},{2*sin(deg(x))});
\end{axis}
\node [right] at (myplot.right of origin) {\scriptsize $x$};
\node [above] at (myplot.above origin) {\scriptsize $y$};
\end{tikzpicture}}

\begin{example}[Confirming the Divergence Theorem]\label{ex_div1}
Let $\vec F =\bracket{x-y,x+y}$, let $C$ be the circle of radius 2 centered at the origin and define $R$ to be the interior of that circle, as shown in \autoref{fig:div1}. Verify the Divergence Theorem; that is, find the flux across $C$ and show it is equal to the double integral of $\divv \vec F$ over $R$.
\solution
We parameterize the circle in the usual way, with\\
$\vrt =\bracket{2\cos t,2\sin t}$, $0\leq t\leq 2\pi$. The flux across $C$ is
\begin{align*}
	\oint_C \vec F\cdot \vec n\dd s
	&= \oint_C\bigl(M\gp(t)-N\fp(t)\bigr)\dd t\\
	&= \int_0^{2\pi} \bigl((2\cos t-2\sin t)(2\cos t) - (2\cos t+2\sin t)(-2\sin t)\bigr)\dd t\\
	&= \int_0^{2\pi} 4\dd t = 8\pi.
\end{align*}
We compute the divergence of $\vec F$ as $\divv \vec F = M_x+N_y = 2$. Since the divergence is constant, we can compute the following double integral easily:
\[
\iint_R \divv \vec F\dd A = \iint_R 2\dd A = 2\iint_R\dd A
= 2(\text{area of $R$}) = 8\pi,
\]
which matches our previous result.
\end{example}

\begin{example}[Flux when $\divv \vec F = 0$]\label{ex_div2}
Let $\vec F$ be any field where $\divv \vec F = 0$, and let $C_1$ and $C_2$ be any two nonintersecting paths, except that each begin at point $A$ and end at point $B$ (see \autoref{fig:div2}). Show why the flux across $C_1$ and $C_2$ is the same.
\solution
%
\mtable{As used in \autoref{ex_div2}, the vector field has a divergence of 0 and the two paths only intersect at their initial and terminal points.}{fig:div2}{\begin{tikzpicture}
\begin{axis}[width=1.16\marginparwidth,height=1.16\marginparwidth,
tick label style={font=\scriptsize},axis line style={draw=none},
name=myplot,axis on top,axis equal,xtick=\empty,ytick=\empty,tick style={draw=none},
ymin=-.1,ymax=1.1,xmin=-.1,xmax=1.1]
\foreach \x in {-.1,0,...,1.11} {
    \foreach \y in {-.1,0,...,1.11} {
		\edef\vx{(cos(deg(3*pi*\y))/30)}
		\edef\vy{(sin(deg(3*pi*\x))/30)}
        \edef\temp{\noexpand\draw[->,lightgray](axis cs:{\x-\vx},{\y-\vy})--(axis cs:{\x+\vx},{\y+\vy});}
        \temp
    }
}
\addplot [smooth,thick, draw={\colortwo},samples=40,domain=0:pi/4,->] ({cos(deg(x))+.05*sin(deg(10*x))},{sin(deg(x))+.05*sin(deg(10*x))})node[pos=1,above right]{$C_1$};
\addplot [smooth,thick, draw={\colortwo},samples=40,domain=pi/4:pi/2] ({cos(deg(x))+.05*sin(deg(10*x))},{sin(deg(x))+.05*sin(deg(10*x))})node[pos=1,left]{$B$};
\addplot [smooth,thick, draw={\colorone},samples=40,domain=-1:-.4,->] ({-x},{(1+x)^2})node[pos=0,below]{$A$}node[pos=1,above right]{$C_2$};
\addplot [smooth,thick, draw={\colorone},samples=40,domain=-.4:0] ({-x},{(1+x)^2});
\end{axis}
\end{tikzpicture}}%
%
By referencing \autoref{fig:div2}, we see we can make a closed path $C$ that combines $C_1$ with $C_2$, where $C_2$ is traversed with its opposite orientation. We label the enclosed region $R$. Since $\divv \vec F = 0$, the Divergence Theorem states that
\[\oint_C \vec F\cdot \vec n\dd s = \iint_R \divv \vec F\dd A = \iint_R 0\dd A = 0.\]
Using the properties and notation given in \autoref{thm:line_int_properties_vector}, consider:
\begin{align*}
	0
	&= \oint_C \vec F\cdot \vec n\dd s \\
	&= \int_{C_1} \vec F\cdot \vec n\dd s +\int_{C_2^*} \vec F\cdot \vec n\dd s
\intertext{(where $C_2^*$ is the path $C_2$ traversed with opposite orientation)}
	&= \int_{C_1} \vec F\cdot \vec n\dd s -\int_{C_2} \vec F\cdot \vec n\dd s.\\
	\int_{C_2} \vec F\cdot \vec n\dd s
	&= \int_{C_1} \vec F\cdot \vec n\dd s.
\end{align*}
Thus the flux across each path is equal.
\end{example}

In this section, we have investigated flow and flux, quantities that measure interactions between a vector field and a planar curve. We can also measure flow along spatial curves, though as mentioned before, it does not make sense to measure flux across spatial curves.

It does, however, make sense to measure the amount of a vector field that passes across a surface in space --- i.e, the flux across a surface. We will study this, though in the next section we first  learn about a more powerful way to describe surfaces than using functions of the form $z=f(x,y)$.

%Regular S, math $S$, and mathcal $\mathcal{S}$. $\mathcal{U}$
%\begin{tikzpicture}[x={(.55ex,0)},y={(0,.55ex)}]
	%\draw[smooth, thin] (0.9667,0.545) -- (0.8273,0.6541) -- (0.6696,0.7624) -- (0.5,0.866) -- (0.3254,0.961) -- (0.1528,1.043) -- (-0.01133,1.11) -- (-0.1607,1.156) -- (-0.2899,1.181) -- (-0.3945,1.181) -- (-0.471,1.155) -- (-0.5173,1.103) -- (-0.5326,1.025) -- (-0.5173,0.9226) -- (-0.4734,0.7971) -- (-0.4038,0.6515) -- (-0.3127,0.4894) -- (-0.2051,0.3147) -- (-0.08684,0.1318) -- (0.03595,-0.05447) -- (0.1569,-0.2394) -- (0.2696,-0.418) -- (0.3682,-0.5859) -- (0.4473,-0.7388) -- (0.5024,-0.873) -- (0.5299,-0.9855) -- (0.5274,-1.074) -- (0.494,-1.137) -- (0.4299,-1.173) -- (0.3366,-1.184) -- (0.2172,-1.169) -- (0.07558,-1.132) -- (-0.08312,-1.073) -- (-0.253,-0.9971) -- (-0.4276,-0.9068) -- (-0.6001,-0.8063) -- (-0.7635,-0.6994) -- (-0.9113,-0.5902);
	%\draw[thin] (-1.18431, -0.959643)--(-0.638322, -0.220787)
								%(0.693671, 0.175596)--(1.23965, 0.914452);
%\end{tikzpicture}
%

\printexercises{exercises/14_04_exercises}
