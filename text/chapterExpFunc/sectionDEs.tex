\Section{Differential Equations}

% MAKE THIS USE TRANSCENDENTAL FUNCTIONS

It's possible to think of antidifferentiation as a type of
\emph{differential equation}.  The problem
\[\frac\dy\dx=f(x)\]
is a differential equation with \emph{general solution}
$y=\int f(x)\dx=F(x)+C$.
If we also specify an \emph{initial condition} $y(x_0)=y_0$, then we can figure
out the value of $C$ that makes this a \emph{particular solution}.
(A differential equation with initial conditions is sometimes called an
\emph{initial value problem}.)

To give another example,
\[x^2y''+3xy'-8y=0\]
% y=x^r y'=rx^{r-1} y''=r(r-1)x^{r-2}
% r(r-1)+3r-8=0
% r^2+2r-8=0
% (r+4)(r-2)=0
is also a differential equation.  It turns out that every solution to
this differential equation (the general solution) can be written in the form
$y=C_1x^2+C_2x^{-4}$, for some constants $C_1$ and $C_2$.
It's not too hard to show that this is a solution:
\[y'=2C_1x-4C_2x^{-5}\qquad\text{and}\qquad y''=2C_1+20C_2x^{-6}\]
so that
\begin{align*}
 x^2y''+3xy'-8y
 &=x^2(2C_1+20C_2x^{-6})+3x(2C_1x-4C_2x^{-5})-8(C_1x^2+C_2x^{-4})\\
 &=C_1(2x^2+6x^2-8x^2)+C_2(20x^{-4}-12x^{-4}-8x^{-4})=0.
\end{align*}
If we also have initial conditions such as $y(1)=5$ and $y'(1)=-2$ to give an
initial value problem, then we can
show that $C_1=3$ and $C_2=2$, so that the particular solution is
$y=3x^2+2x^{-4}$.

That's all that it takes to show that you've found a solution to a differential
equation.  The difficult part (and the entire point of a course like 266) is to
figure out what that general solution should be in the first place.
