\section{Exponential Functions and Their Derivatives}

Exponentials functions are functions of the form $f(x)=b^x$. Here
the positive constant $b$ is called the \textbf{base}. We restrict
ourselves to positive bases since if the base where negative, the
definition of some powers of the base, such as $(-1)^{\frac12}$, would involve
imaginary numbers, and we want to stick with real numbers in Calculus. 

%\begin{wrapfigure}{r}{0pt}
%\includegraphics{images/7_1_exponential}
%\end{wrapfigure}
You should recognize the characteristic shape of the graph of
exponential functions $f(x)=b^x$. They all have a \emph{swoopy} sort of
look that you probably picture when you hear someone use the phrase \emph{such and such is growing exponentially} (for example, population is
growing exponentially, or the national debt is growing exponentially).

Make sure that you are familiar with the laws of exponents presented here:
\begin{align*}
b^0&=1 & (b^x)^y&=b^{xy} & b^{1/n}&=\sqrt[n]{b}\\
b^x b^y&=b^{x+y} & \frac{b^x}{b^y}&=b^{x-y} & b^{-x}&=\frac1{b^x}.
\end{align*}
If you can't work comfortably with these rules, this chapter will be quite
difficult.
The same can be said of logarithmic function, which we will see shortly.

Actually, the \emph{swoopy up} picture is correct only if the constant $b$
is greater than $1$. Of course, if $b=1$, then $b^x=1$ for all $x$, and
so the graph is just a horizontal line, and if $0<b<1$, then the graph
swoops down, approaching the positive $x$-axis as a horizontal
asymptote.

Now for some calculus of the exponential functions. What is the
slope of the tangent line to the graph of the exponential function
$f(x)=b^x$ at some point $(x,b^x)$ on the curve? The answer, of course, using the
methods we learned in Calculus I, 
is
\[f'(x)=\lim_{h\to0}\frac{f(x+h)-f(x)}h=\lim_{h\to0}\frac{b^{x+h}-b^x}h.\]

Just to practice a bit, let's compute the slope of the tangent line at
$x=0$. That would be given by
\[f'(0)=\lim_{h\to0}\frac{f(0+h)-f(0)}h=\lim_{h\to0}\frac{b^h-1}h.\]

Well, that certainly looks like a tough limit to compute. So it seems hard
even to find the slope of the tangent line at $x=0$. How about the derivative in
general? In other words, suppose we wanted to compute
\[f'(x)=\lim_{h\to0}\frac{f(x+h)-f(x)}h=\lim_{h\to0}\frac{b^{x+h}-b^x}h.\]

To make a little more progress with this limit, we will make use of
a property of exponents: $b^{t+u}=b^tb^u$. This, and other of the
familiar properties you are used to for manipulating exponents can all
be proved as theorems based on the definition of $b^x$ as a limit. In
any case, using the property mentioned above, we find
\[
 f'(x)=\lim_{h\to0}\frac{b^{x+h}-b^x}h
 =\lim_{h\to0}\frac{b^xb^h-b^x}h
 =\lim_{h\to0}b^x\frac{b^h-1}h
 =b^x\lim_{h\to0}\frac{b^h-1}h.
\]
Since the limit on the right is the formula for $f'(0)$, 
we can write that, for every $x$, $f'(x)=f'(0)b^x$, and
so to compute $f'(x)$, all we really need to do is determine the
value of $f'(0)$.

Unfortunately, for most choices of $b$, the value of the limit
\[\lim_{h\to0}\frac{b^h-1}h\]
is a number that is difficult to determine, and not very neat anyway. For example,
when $b=10$ the value of the limit turns out to be (after a lot of work)
$2.302585093\cdots$. Not a very nice number, and certainly a number that you don't
want to have to memorize to use in computations. That's why the exponential
function $y=10^x$ is not very popular in mathematics courses.
(We'll find out in the next section that the limit is $\ln b$.)

However, as demonstrated in the text on page 346, it seems reasonable that
there is a 
choice for the base $a$ somewhere between $2$ and $3$ for which
$f'(0)=1$. This number is denoted by $e$ and it is called the
base for the \textbf{natural exponential function}. It is an irrational
number; its approximate value is $2.718281828459\cdots$, but just call it $e$ for
short, and remember that it is about $2.7$. If the phrase
\emph{exponential function} is used with no further explanation, it is
safe to assume that $f(x)=e^x$ is the one that is meant.

The important thing to keep in mind is that since $f'(0) = 1$ when
$f(x)=e^x$, the computations above 
show
\[f'(x)=\left(e^x\right)'=e^xf'(0)=e^x\cdot1=e^x.\]
In other words, we get the important formula
\[\left(e^x\right)'=e^x.\]
If someone refers to ``the'' exponential function, this is the one that they
mean (and this is the reason why this base is the best choice).
If someone only refers to some exponential function, they mean $b^x$ for some
$b$ (although we can use logarithms to rewrite this as $e^{(\ln b)x}$).

Every time a new differentiation formula is derived, a free integration formula comes along for the 
ride:
\[\int e^{x} dx=e^x+C.\]
And of course, we can immediately write down more general versions of those formulas:
\begin{align}
\left(e^{mx+b}\right)&=m e^{mx+b}\\
\int e^{mx+b}dx&=\frac1m e^{mx+b}+C.
\end{align}
%\equDesc{Derivative and integral of the natural exponential function}

Exponential functions grow extremely quickly.
The function $1.1^x$ is eventually bigger than $x^{100}$, once we take $x$
large enough (we need $x\ge9624$ in this case).

\subsection*{An Optional Side Note}

The value of $b^x$ when $x$ is a rational number (fraction) is
reasonably obvious. For example $3^{\frac35}$, means cube $3$ to get
$27$, and now compute the $5$-th root of $27$, which we can do as
accurately as we wish by a variety of algebraic techniques. However, it
is not at all clear how we would ever compute $3^{\sqrt2}$. We
\textbf{define} $b^x$, when $x$ is irrational, by the following rule:
\[b^x=\lim_{r\to x}b^x,\qquad r\text{ rational.}\]
So we base (pun intended!) the definition of $b^x$ for irrational $x$ on the sorts
of powers we understand, namely \textbf{rational} powers of $b$.
