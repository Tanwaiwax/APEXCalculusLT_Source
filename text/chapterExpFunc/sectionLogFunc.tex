\section{Logarithmic Functions}

The \textbf{logarithm base b}, written as $\log_b$, is defined to
be the function inverse of $f(x)=b^x$. In other words, if you are asked
to compute $f(x) = b^x$, you need answer the question \emph{what do I get
when I raise the number $b$ to the power $x$}. On the other hand, if
you are asked to compute $g(x)=\log_b x$, you need to answer the
question \emph{to what power do I need to raise $b$ in order to get the result $x$}.
You can see this is just the inverse question of the first one. For
example, $\log_3 81 = 4$, since we need to raise $3$ to the $4^\text{th}$
power to get the result $81$.

The most important thing for you to keep in mind is that the
equations
\[a=b^c\qquad\text{and}\qquad c=\log_b a\]
say exactly the same thing. When you see one of these two equations, you
can replace it with the other. Remember that; it is a very useful
tool.

Just as $e$ is used as the standard base for the exponential
function, logarithms base $e$ are the most frequently seen logarithms.
Logarithms base $e$ are called \textbf{natural logarithms}, and instead of
writing $\log_e x$, the usual symbol is $\ln x$.  That  is pronounced 
\emph{ell-en of $x$}

Historically, logarithms were important tools for carrying out
complicated calculations. They are not used for that purpose much any
more, but they have many other important applications in science and
mathematics today. There are a number of properties of logarithms that
made them useful for computational problems. Essentially, logarithms
convert a messy multiplication problem to a less messy addition problem.
As an equation, this is expressed as
\[\log(st) = \log s + \log t.\]
In other words, \emph{the log of a product is the sum of the logs}.
The particular base used makes no difference, so I left it unspecified.
Two related properties of the logarithm functions are
\[
 \log\left(\frac st\right) = \log s - \log t,
 \qquad\text{and}\qquad
 \log \left(s^t\right) = t\log s.
\]
All three of these facts are easy to prove using the corresponding
properties of the exponential functions mentioned in Section 7.1. You
should see if you can work out the verifications.

You'll notice that your calculator can directly compute only
logarithms base $10$ (usually a key labeled $\log$), and base $e$
(usually a key labeled $\ln$). You can convert logarithms in any base to
natural logarithms, and, since calculus is much neater using the natural
base, you should always do so. The conversion is easy. Suppose we
have to deal with the number
\[a = \log_b x.\]
Then we know that $b^a=x$.
Taking natural logs of both sides of that last equation, we find
$\ln b^a=\ln x$,
and making use of one of the properties of logs, we
get $a\ln b = \ln x$,
or $a=\dfrac{\ln x}{\ln b}.$
In other words, we
can replace 
\[\log_b x\qquad\text{with}\qquad\frac{\ln x}{\ln b}.\]
