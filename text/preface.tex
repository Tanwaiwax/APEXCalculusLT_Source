\prefacegeometry
\chapter*{Preface}
\addcontentsline{toc}{chapter}{Preface}
\pagestyle{plain} % doesn't take?
\thispagestyle{empty}

\subsection{A Note on Using this Text}

Thank you for reading this short preface. Allow us to share a few key points about the text so that you may better understand what you will find beyond this page.

This text comprises a three-volume series on Calculus.
The first part covers material taught in many ``Calculus 1'' courses: limits, derivatives, and the basics of integration, found in Chapters~\ref{chapter:limits} through
\iftoggle{bsc}{\ref{chapter:integration}}{\ref{chapter:app_of_int}}.
The second text covers material often taught in ``Calculus 2'': integration and its applications, along with an introduction to sequences, series and Taylor Polynomials, found in
Chapters~\iftoggle{bsc}{\ref{chapter:app_of_int}}{\ref{chapter:diff_conc}}
through \ref{chapter:planar_curves}. The third text covers topics common in ``Calculus 3'' or ``Multivariable Calculus'': parametric equations, polar coordinates, vector-valued functions, and functions of more than one variable, found in Chapters \ref{chapter:vectors} through \ref{chapter:vector_calc}. All three are available separately for free.

Printing the entire text as one volume makes for a large, heavy, cumbersome book. One can certainly only print the pages they currently need, but some prefer to have a nice, bound copy of the text. Therefore this text has been split into these three manageable parts, each of which can be purchased separately.

A result of this splitting is that sometimes material is referenced that is not contained in the present text. The context should make it clear whether the ``missing'' material comes before or after the current portion. Downloading the appropriate pdf, or the entire \emph{\apex\ Calculus LT} pdf, will give access to these topics.

\subsection{For Students: How to Read this Text}

Mathematics textbooks have a reputation for being hard to read. High-level mathematical writing often seeks to say much with few words, and this style often seeps into texts of lower-level topics. This book was written with the goal of being easier to read than many other calculus textbooks, without becoming too verbose. 

Each chapter and section starts with an introduction of the coming material, hopefully setting the stage for ``why you should care,'' and ends with a look ahead to see how the just-learned material helps address future problems. Additionally, each chapter includes a section zero, which provides a basic review and practice problems of pre-calculus skills. Since this content is a pre-requisite for calculus, reviewing and mastering these skills are considered your responsibility. This means that it is your responsibility to seek assistance outside of class from your instructor, a math resource center or other math tutoring available on-campus.  A solid understanding of these skills is essential to your success in solving calculus problems.

\emph{Please read the text;} it is written to explain the concepts of Calculus. There are numerous examples to demonstrate the meaning of definitions, the truth of theorems, and the application of mathematical techniques. When you encounter a sentence you don't understand, read it again. If it still doesn't make sense, read on anyway, as sometimes confusing sentences are explained by later sentences.

\emph{You don't have to read every equation.} The examples generally show ``all'' the steps needed to solve a problem. Sometimes reading through each step is helpful; sometimes it is confusing. When the steps are illustrating a new technique, one probably should follow each step closely to learn the new technique. When the steps are showing the mathematics needed to find a number to be used later, one can usually skip ahead and see how that number is being used, instead of getting bogged down in reading how the number was found.

\emph{Some proofs have been delayed until later (or omitted completely).} In mathematics, \emph{proving} something is always true is extremely important, and entails much more than testing to see if it works twice. However, students often are confused by the details of a proof, or become concerned that they should have been able to construct this proof on their own. To alleviate this potential problem, we do not include the more difficult proofs in the text. The interested reader is highly encouraged to find other proofs online or from their instructor. In most cases, one is very capable of understanding what a theorem \emph{means} and \emph{how to apply it} without knowing fully \emph{why} it is true.

\emph{Work through the examples.}  The best way to learn mathematics is to do it.  Reading about it (or watching someone else do it) is a poor substitute.  For this reason, every page has a place for \emph{you} to put \emph{your} notes so that \emph{you} can work out the examples.  That being said, sometimes it is useful to watch someone work through an example.  For this reason, this text also provides links to online videos where someone is working through a similar problem.  If you want even more videos, these are generally chosen from
\iflatexml\begin{itemize}\else\begin{itemize}[nosep]\fi
\item Khan Academy: \url{https://www.khanacademy.org/}
\item Math Doctor Bob: \url{http://www.mathdoctorbob.org/}
\item Just Math Tutorials: \url{http://patrickjmt.com/} (unfortunately, they're not well organized)
\end{itemize}
Some other sites you may want to consider are
\iflatexml\begin{itemize}\else\begin{itemize}[nosep]\fi
\item Larry Green's Calculus Videos: \url{http://www.ltcconline.net/greenl/courses/105/videos/VideoIndex.htm}
\item Mathispower4u: \url{http://www.mathispower4u.com/}
\item Yay Math: \url{http://www.yaymath.org/} (for prerequisite material)
\end{itemize}
All of these sites are completely free (although some will ask you to donate).  Here's a sample one:

\youtubeVideo{ILNfpJTZLxk}{Practical Advice for Those Taking College Calculus}


\subsection{Thanks from Greg Hartman}

There are many people who deserve recognition for the important role they have played in the development of this text. First, I thank Michelle for her support and encouragement, even as this ``project from work'' occupied my time and attention at home. Many thanks to Troy Siemers, whose most important contributions extend far beyond the sections he wrote or the 227 figures he coded in Asymptote for 3D interaction.  He provided incredible support, advice and encouragement for which I am very grateful. My thanks to Brian Heinold and Dimplekumar Chalishajar for their contributions and to Jennifer Bowen for reading through so much material and providing great feedback early on. Thanks to Troy, Lee Dewald, Dan Joseph, Meagan Herald, Bill Lowe, John David, Vonda Walsh, Geoff Cox, Jessica Libertini and other faculty of VMI who have given me numerous suggestions and corrections based on their experience with teaching from the text. (Special thanks to Troy, Lee \& Dan for their patience in teaching Calc III while I was still writing the Calc III material.) Thanks to Randy Cone for encouraging his tutors of VMI's Open Math Lab to read through the text and check the solutions, and thanks to the tutors for spending their time doing so. A very special thanks to Kristi Brown and Paul Janiczek who took this opportunity far above \& beyond what I expected, meticulously checking every solution and carefully reading every example. Their comments have been extraordinarily helpful. I am also thankful for the support provided by Wane Schneiter, who as my Dean provided me with extra time to work on this project. I am blessed to have so many people give of their time to make this book better.

\subsection{\apex\ --- Affordable Print and Electronic teXts}

\apex\ is a consortium of authors  who collaborate to produce high-quality, low-cost textbooks. The current textbook-writing paradigm is facing a potential revolution as desktop publishing and electronic formats increase in popularity. However, writing a good textbook is no easy task, as the time requirements alone are substantial. It takes countless hours of work to produce text, write examples and exercises, edit and publish. Through collaboration, however, the cost to any individual can be lessened, allowing us to create texts that we freely distribute electronically and sell in printed form for an incredibly low cost. Having said that, nothing is entirely free; someone always bears some cost. This text ``cost'' the authors of this book their time, and that was not enough. \emph{\apex\ Calculus} would not exist had not the Virginia Military Institute, through a generous Jackson-Hope grant, given the lead author significant time away from teaching so he could focus on this text.

Each text is available as a free .pdf, protected by a Creative Commons Attribution --- Noncommercial 4.0 copyright. That means you can give the .pdf to anyone you like, print it in any form you like, and even edit the original content and redistribute it. If you do the latter, you must clearly reference this work and you cannot sell your edited work for money.

We encourage others to adapt this work to fit their own needs. One might add sections that are ``missing'' or remove sections that your students won't need. The source files can be found at \url{https://github.com/APEXCalculus}.

You can learn more at \texttt{\href{http://www.vmi.edu/APEX}{www.vmi.edu/APEX}}.

~\hfill Greg Hartman

\subsection{Creating \apex~LT}

Starting with the source at \url{https://github.com/APEXCalculus},
faculty at the University of North Dakota made several substantial changes to create \apex\ Late Transcendentals.  The most obvious change was to rearrange the text to delay proving the derivative of transcendental functions until Calculus 2.  UND added Sections \ref{sec:inv_funcs} and \ref{sec:exp_log}, adapted several sections from other resources, created the prerequisite sections, included links to videos and Geogebra, and added several examples and exercises.  In the end, every section had some changes (some more substantial than others), resulting in a document that is about 10\% longer. The source files can now be found at\iflatexml\ \else\\\fi
\url{https://github.com/teepeemm/APEXCalculusLT_Source}.
%Jerry Metzger provided many of the links to the videos.

Extra thanks are due
to Michael Corral for allowing us to use portions of his Vector Calculus, available at \texttt{\href{http://www.mecmath.net/}{www.mecmath.net/}}
(specifically, %\ref{sec:other_systems},
\autoref{sec:lagrange} and the Jacobian in \autoref{sec:cylindrical_spherical})
and
to Paul Dawkins for allowing us to use portions of his online math notes from \texttt{\href{http://tutorial.math.lamar.edu/}{tutorial.math.lamar.edu/}} (specifically, Sections \ref{sec:int_techniques} and \ref{sec:series_techniques}, as well as ``Area with Parametric Equations'' in \autoref{sec:par_calc}).
The work on Calculus III was partially supported by the NDUS OER Initiative.


\subsection{Electronic Resources}

A distinctive feature of \apex\ is interactive, 3D graphics in the .pdf version. Nearly all graphs of objects in space can be rotated, shifted, and zoomed in/out so the reader can better understand the object illustrated. 

Currently, the only pdf viewers that support these 3D graphics for computers are Adobe Reader \& Acrobat. To activate the interactive mode, click on the image. Once activated, one can click/drag to rotate the object and use the scroll wheel on a mouse to zoom in/out. (A great way to investigate an image is to first zoom in on the page of the pdf viewer so the graphic itself takes up much of the screen, then zoom inside the graphic itself.) A CTRL-click/drag pans the object left/right or up/down. By right-clicking on the graph one can access a menu of other options, such as changing the lighting scheme or perspective. One can also revert the graph back to its default view. If you wish to deactivate the interactivity, one can right-click and choose the ``Disable Content'' option.

\newcommand{\threedurl}{https://sites.und.edu/timothy.prescott/apex/prc/}

\iflatexml\else
\noindent
\begin{minipage}[t]{.74\linewidth}%
\setlength{\parindent}{\saveparindent}
\indent
\fi
The situation is more interesting for tablets and smartphones.  The 3D graphics files have been arrayed at \url{\threedurl}.  At the bottom of the page are links to Android and iOS apps that can display the interactive files.
\iflatexml\else
The QR code to the right will take you to that page.
\end{minipage}
\quad
\begin{minipage}[t]{2cm}%
\vspace{-.5\baselineskip}\qrcode{\threedurl}%
\end{minipage}
\fi

\iflatexml\else
Additionally, a web version of the book is available at \url{https://sites.und.edu/timothy.prescott/apex/web/}.  While we have striven to make the pdf accessible for non-print formats, html is far better in this regard.
\fi


\restoregeometry

\iffalse

The rest of this file is commented out.

Michael Corral's permission for Vector Calc:

Delivered-To: teepeemm+und@gmail.com
Received: by 10.12.137.196 with SMTP id 4csp206633qvs;
        Tue, 11 Oct 2016 21:50:21 -0700 (PDT)
X-Received: by 10.28.208.204 with SMTP id h195mr902815wmg.25.1476247821239;
        Tue, 11 Oct 2016 21:50:21 -0700 (PDT)
Return-Path: <timothy.prescott.und+caf_=teepeemm+und=gmail.com@gmail.com>
Received: from mail-wm0-f46.google.com (mail-wm0-f46.google.com. [74.125.82.46])
        by mx.google.com with ESMTPS id iu2si7937925wjb.79.2016.10.11.21.50.20
        for <teepeemm+und@gmail.com>
        (version=TLS1_2 cipher=ECDHE-RSA-AES128-GCM-SHA256 bits=128/128);
        Tue, 11 Oct 2016 21:50:21 -0700 (PDT)
Received-SPF: pass (google.com: domain of timothy.prescott.und+caf_=teepeemm+und=gmail.com@gmail.com designates 74.125.82.46 as permitted sender) client-ip=74.125.82.46;
Authentication-Results: mx.google.com;
       spf=pass (google.com: domain of timothy.prescott.und+caf_=teepeemm+und=gmail.com@gmail.com designates 74.125.82.46 as permitted sender) smtp.mailfrom=timothy.prescott.und+caf_=teepeemm+und=gmail.com@gmail.com
Received: by mail-wm0-f46.google.com with SMTP id c78so5170585wme.1
        for <teepeemm+und@gmail.com>; Tue, 11 Oct 2016 21:50:20 -0700 (PDT)
X-Google-DKIM-Signature: v=1; a=rsa-sha256; c=relaxed/relaxed;
        d=1e100.net; s=20130820;
        h=x-original-authentication-results:x-gm-message-state:delivered-to
         :date:from:to:subject:in-reply-to:message-id:references:user-agent
         :mime-version;
        bh=4QZTTpi4S0EQNJbS8tptvA7JD1EZlV+X7XZHmQ8AkGM=;
        b=gnrdxJXm28XRjisuz3lkoHQGazVvxlHaU9qMt5eZbcjAVXe3KQXotfrPD/BI7qlcBv
         326C0dy43JJCte8GViUowUtK760ErySLnTcijMjJ5mALRmfugtcqjk+jYXN457iLlWIz
         P/dyAn1gAUp8dU7n3nNc0J1wlMBKNAobVHEg/pDKMPBLC76Rx6A0vkcDrU/VxjtaNoJf
         Pm/rzcGxcSRxVkWdRTFtw7lJYTae7BV3gHd2ptASaSglpGvZCyE/Q0wRCGfZZ9WdRtz2
         cBmqwpId+BSIxAzhqicY+PcVj8b4/Dt1JIg6KRamlb72cKizaAREBTUsTYuMRp58QRn8
         GIwg==
X-Original-Authentication-Results: mx.google.com;
       spf=neutral (google.com: 173.247.247.235 is neither permitted nor denied by best guess record for domain of mcorral@mecmath.net) smtp.mailfrom=mcorral@mecmath.net
X-Gm-Message-State: AA6/9RmAB4vzE4sB8058XycdVu6s2+1v2SYnRZQBET7qioZ9UKk72GLVyunHUnk/NPDHWdnr+q+aEOQG//nqrkZ3aEE3/QM=
X-Received: by 10.194.157.193 with SMTP id wo1mr8038181wjb.22.1476247820810;
        Tue, 11 Oct 2016 21:50:20 -0700 (PDT)
X-Forwarded-To: teepeemm+und@gmail.com
X-Forwarded-For: timothy.prescott.und@gmail.com teepeemm+und@gmail.com
Delivered-To: timothy.prescott.und@gmail.com
Received: by 10.80.183.175 with SMTP id h44csp359244ede;
        Tue, 11 Oct 2016 21:50:19 -0700 (PDT)
X-Received: by 10.99.110.78 with SMTP id j75mr6174565pgc.2.1476247819627;
        Tue, 11 Oct 2016 21:50:19 -0700 (PDT)
Return-Path: <mcorral@mecmath.net>
Received: from biz104.inmotionhosting.com (biz104.inmotionhosting.com. [173.247.247.235])
        by mx.google.com with ESMTPS id t5si3802730pgb.173.2016.10.11.21.50.18
        for <timothy.prescott.und@gmail.com>
        (version=TLS1 cipher=AES128-SHA bits=128/128);
        Tue, 11 Oct 2016 21:50:19 -0700 (PDT)
Received-SPF: neutral (google.com: 173.247.247.235 is neither permitted nor denied by best guess record for domain of mcorral@mecmath.net) client-ip=173.247.247.235;
Received: from c-73-191-129-108.hsd1.mi.comcast.net ([73.191.129.108]:35732 helo=banana.sluggo.net) by biz104.inmotionhosting.com with esmtpa (Exim 4.87) (envelope-from <mcorral@mecmath.net>) id 1buBUe-0006uW-AF for timothy.prescott.und@gmail.com; Tue, 11 Oct 2016 21:50:17 -0700
Date: Wed, 12 Oct 2016 00:56:58 -0400 (EDT)
From: Michael Corral <mcorral@mecmath.net>
X-X-Sender: mcorral@banana.sluggo.net
To: Timothy Prescott <timothy.prescott.und@gmail.com>
Subject: Re: using Vector Calculus with a different open license
In-Reply-To: <CAJv=1OksH0MhTddy9Yu4=BoHo8733_oCjHU8aCUbjv2ZxdY=Rg@mail.gmail.com>
Message-ID: <alpine.LFD.2.20.1610120051340.7844@banana.sluggo.net>
References: <CAJv=1OksH0MhTddy9Yu4=BoHo8733_oCjHU8aCUbjv2ZxdY=Rg@mail.gmail.com>
User-Agent: Alpine 2.20 (LFD 67 2015-01-07)
X-Mailer: Alpine 2.02 <Fedora 15 x86_64>
MIME-Version: 1.0
Content-Type: multipart/mixed; BOUNDARY="-1463747071-1550951339-1476248223=:7844"
X-OutGoing-Spam-Status: No, score=-1.0
X-AntiAbuse: This header was added to track abuse, please include it with any abuse report
X-AntiAbuse: Primary Hostname - biz104.inmotionhosting.com
X-AntiAbuse: Original Domain - gmail.com
X-AntiAbuse: Originator/Caller UID/GID - [47 12] / [47 12]
X-AntiAbuse: Sender Address Domain - mecmath.net
X-Get-Message-Sender-Via: biz104.inmotionhosting.com: authenticated_id: mcorral@mecmath.net
X-Authenticated-Sender: biz104.inmotionhosting.com: mcorral@mecmath.net
X-Source: 
X-Source-Args: 
X-Source-Dir: 

---1463747071-1550951339-1476248223=:7844
Content-Type: text/plain; charset=UTF-8; format=flowed
Content-Transfer-Encoding: 8BIT

Hello Professor Prescott,

Sorry for the delay in replying, my mailbox gets bombarded with
spam and it often takes me a while to sift through it all.

Yes, you have my permission to use those sections from my Vector
Calculus book with the same license as the Apex book.

I'm glad you found that material useful.

Thanks,

Michael Corral
Schoolcraft College

On 09/28/16, Timothy Prescott wrote:
> Professor Corral,
> 
> The University of North Dakota is in the process of adopting Apex Calculus
> for our entire calculus sequence.  Unfortunately for us, it does not
> currently have a chapter on Line and Surface Integrals (it is also missing a
> section on Lagrange multipliers, and the change of variables formula for
> multiple integration).  We were wondering if we would be able to use those
> sections from your Vector Calculus book (along with some of the exercises as
> well, probably).
> 
> I realize that this is currently allowed under the book?s GNU Free
> Documentation License.  But Apex Calculus uses a CC-BY-NC 4.0 license. 
> Would you be willing to allow us to use your material with the same license
> as Apex?
> 
> Thank you for your time,
> 
> Tim Prescott
> Associate Professor of Mathematics
> University of North Dakota
> 
>
---1463747071-1550951339-1476248223=:7844--


Paul Dawkin's permission for Paul's Online Math Notes:

Delivered-To: teepeemm+und@gmail.com
Received: by 10.12.133.134 with SMTP id o6csp1283228qva;
        Tue, 8 Nov 2016 05:18:53 -0800 (PST)
X-Received: by 10.194.2.198 with SMTP id 6mr12218551wjw.51.1478611133415;
        Tue, 08 Nov 2016 05:18:53 -0800 (PST)
Return-Path: <timothy.prescott.und+caf_=teepeemm+und=gmail.com@gmail.com>
Received: from mail-wm0-f41.google.com (mail-wm0-f41.google.com. [74.125.82.41])
        by mx.google.com with ESMTPS id mc8si35239494wjb.127.2016.11.08.05.18.53
        for <teepeemm+und@gmail.com>
        (version=TLS1_2 cipher=ECDHE-RSA-AES128-GCM-SHA256 bits=128/128);
        Tue, 08 Nov 2016 05:18:53 -0800 (PST)
Received-SPF: pass (google.com: domain of timothy.prescott.und+caf_=teepeemm+und=gmail.com@gmail.com designates 74.125.82.41 as permitted sender) client-ip=74.125.82.41;
Authentication-Results: mx.google.com;
       spf=pass (google.com: domain of timothy.prescott.und+caf_=teepeemm+und=gmail.com@gmail.com designates 74.125.82.41 as permitted sender) smtp.mailfrom=timothy.prescott.und+caf_=teepeemm+und=gmail.com@gmail.com
Authentication-Results: spf=none (sender IP is ) smtp.mailfrom=michele.iiams@email.und.edu;
Received: by mail-wm0-f41.google.com with SMTP id p190so242528778wmp.1
        for <teepeemm+und@gmail.com>; Tue, 08 Nov 2016 05:18:53 -0800 (PST)
X-Google-DKIM-Signature: v=1; a=rsa-sha256; c=relaxed/relaxed;
        d=1e100.net; s=20130820;
        h=x-original-authentication-results:x-gm-message-state:delivered-to
         :from:to:subject:thread-topic:thread-index:date:message-id
         :references:in-reply-to:accept-language:content-language
         :spamdiagnosticoutput:mime-version;
        bh=4kSirSYfDGZUsnfr1Vb8STyHxn3qnhDe1dcvINCiK5U=;
        b=lj0ITCnwrXmaSy8uXbLHT2ObUU1tvmsYhcrtXalC2PCKUwkyC07i0mjJSMoOJ0XHvz
         pqvpI18JYRD2Mhkf8/3pdmgvDvWrXHoIOiCmGmWnQ0aK1boZ8ZfCcijzJspIeG/e2Qu+
         l4+71bcS/5PTUEXKVcLUv8ektKTg/uIxB5iZHK7q+lR0wt40S7RnNCzjqXOfbsfHw/qu
         ewAkPNyDBC4r927dTcR6GEnZut2p2wAGZ1fowHIpPc85Wx42wyfkMq4h4SaJyMuf5aOM
         2P+wLTCKq3++o6lXM02wPflYzGnjuzDGG/QZt/L3crUaJMw2oG7I7C36J6OChYb0G1dG
         TgPw==
X-Original-Authentication-Results: mx.google.com;
       spf=pass (google.com: domain of michele.iiams@email.und.edu designates 104.47.38.126 as permitted sender) smtp.mailfrom=michele.iiams@email.und.edu
X-Gm-Message-State: ABUngve/ExvFC7TJh3/a4vpVWfb5VK1kUMFTgWHSzA56pyoE2plfGvOxaR8562yczUM5tOcXMNjq4z9f4MORNhJESDIiLN8=
X-Received: by 10.28.135.207 with SMTP id j198mr14293705wmd.109.1478611132870;
        Tue, 08 Nov 2016 05:18:52 -0800 (PST)
X-Forwarded-To: teepeemm+und@gmail.com
X-Forwarded-For: timothy.prescott.und@gmail.com teepeemm+und@gmail.com
Delivered-To: timothy.prescott.und@gmail.com
Received: by 10.80.183.137 with SMTP id h9csp859157ede;
        Tue, 8 Nov 2016 05:18:52 -0800 (PST)
X-Received: by 10.107.195.206 with SMTP id t197mr12580417iof.221.1478611132036;
        Tue, 08 Nov 2016 05:18:52 -0800 (PST)
Return-Path: <michele.iiams@email.und.edu>
Received: from NAM02-BL2-obe.outbound.protection.outlook.com (mail-bl2nam02on0126.outbound.protection.outlook.com. [104.47.38.126])
        by mx.google.com with ESMTPS id r6si9819679ith.85.2016.11.08.05.18.51
        for <timothy.prescott.und@gmail.com>
        (version=TLS1_2 cipher=ECDHE-RSA-AES128-SHA bits=128/128);
        Tue, 08 Nov 2016 05:18:51 -0800 (PST)
Received-SPF: pass (google.com: domain of michele.iiams@email.und.edu designates 104.47.38.126 as permitted sender) client-ip=104.47.38.126;
Received: from MWHPR08MB2974.namprd08.prod.outlook.com (10.173.240.140) by MWHPR08MB2782.namprd08.prod.outlook.com (10.173.239.12) with Microsoft SMTP Server (version=TLS1_2, cipher=TLS_ECDHE_RSA_WITH_AES_256_CBC_SHA384_P384) id 15.1.707.6; Tue, 8 Nov 2016 13:18:48 +0000
Received: from MWHPR08MB2974.namprd08.prod.outlook.com ([10.173.240.140]) by MWHPR08MB2974.namprd08.prod.outlook.com ([10.173.240.140]) with mapi id 15.01.0707.006; Tue, 8 Nov 2016 13:18:47 +0000
From: "Iiams, Michele" <michele.iiams@email.und.edu>
To: "Prescott, Timothy" <timothy.prescott@email.und.edu>
Subject: Fwd: Permission to use another section of your Calculus II notes
Thread-Topic: Permission to use another section of your Calculus II notes
Thread-Index: AQHSOVab9+h2/DCJmECwdGJyzg58G6DPEU4AgAABS4I=
Date: Tue, 8 Nov 2016 13:18:47 +0000
Message-ID: <ek6uw3t7hsit4k5e9q11wp9b.1478611124428@email.android.com>
References: <MWHPR08MB29747D6C8920390167ED8DF3AFA60@MWHPR08MB2974.namprd08.prod.outlook.com>,<CALDs=E=oujWn2etJAwwtJrWWT1y+Jqb5nzK-aCO1fGvadOiDqA@mail.gmail.com>
In-Reply-To: <CALDs=E=oujWn2etJAwwtJrWWT1y+Jqb5nzK-aCO1fGvadOiDqA@mail.gmail.com>
Accept-Language: en-US
Content-Language: en-US
X-MS-Has-Attach: 
X-MS-Exchange-Inbox-Rules-Loop: timothy.prescott@email.und.edu
X-MS-TNEF-Correlator: 
x-ms-exchange-messagesentrepresentingtype: 1
x-originating-ip: [2600:1014:b047:6b3a:38a4:65f8:a92d:c144]
x-ms-office365-filtering-correlation-id: 4d08d5b9-fa36-4096-40ed-08d407d9c605
x-microsoft-exchange-diagnostics: 1;MWHPR08MB2782;24:ig3Mns3v7+IIIVzP1b8sDE6nD0GME60ic+KXHmQY90CKzkm56Yay4XRxdd7saKmQYgdSN8i02ep0saS12JuJnPWlcvZp1sQ046yX0WdMqY4=
x-microsoft-antispam: UriScan:;BCL:0;PCL:0;RULEID:;SRVR:MWHPR08MB2782;
x-exchange-antispam-report-test: UriScan:(34617014829592);
x-exchange-antispam-report-cfa-test: BCL:0;PCL:0;RULEID:(9101524098)(601004)(2401047)(8121501046)(3002001)(10201501046);SRVR:MWHPR08MB2782;BCL:0;PCL:0;RULEID:;SRVR:MWHPR08MB2782;
x-forefront-antispam-report: SFV:SKI;SFS:;DIR:INB;SFP:;SCL:-1;SRVR:MWHPR08MB2782;H:MWHPR08MB2974.namprd08.prod.outlook.com;FPR:;SPF:None;LANG:en;SFV:NSPM;SFS:(10019020)(7916002)(377454003)(24454002)(2473002)(199003)(189002)(86362001)(122556002)(88552002)(2950100002)(102836003)(101416001)(54356999)(3280700002)(99286002)(42882006)(105586002)(77096005)(68736007)(8936002)(110136003)(75432002)(8676002)(9686002)(586003)(7906003)(6116002)(6636002)(2900100001)(33646002)(63666004)(7846002)(107886002)(2171001)(95246002)(450100001)(81156014)(7736002)(89836001)(81166006)(87936001)(97736004)(5660300001)(3660700001)(50986999)(76176999)(189998001)(106356001)(106116001)(89122001)(92566002)(51650200001);DIR:OUT;SFP:1102;SCL:1;SRVR:MWHPR08MB2782;H:MWHPR08MB2974.namprd08.prod.outlook.com;FPR:;SPF:None;PTR:InfoNoRecords;A:1;MX:1;LANG:en;
spamdiagnosticoutput: 1:0
x-forefront-prvs: 01208B1E18
received-spf: None (protection.outlook.com: email.und.edu does not designate permitted sender hosts)
Content-Type: multipart/alternative; boundary="_000_ek6uw3t7hsit4k5e9q11wp9b1478611124428emailandroidcom_"
MIME-Version: 1.0
X-OriginatorOrg: email.und.edu
X-MS-Exchange-CrossTenant-originalarrivaltime: 08 Nov 2016 13:18:47.7012 (UTC)
X-MS-Exchange-CrossTenant-fromentityheader: Hosted
X-MS-Exchange-CrossTenant-id: ec37a091-b9a6-47e5-98d0-903d4a419203
X-MS-Exchange-Transport-CrossTenantHeadersStamped: MWHPR08MB2782

--_000_ek6uw3t7hsit4k5e9q11wp9b1478611124428emailandroidcom_
Content-Type: text/plain; charset="us-ascii"
Content-Transfer-Encoding: quoted-printable





Sent from my Verizon, Samsung Galaxy smartphone


-------- Original message --------
From: Paul Dawkins <pdawkins@gmail.com>
Date: 11/8/16 7:14 AM (GMT-06:00)
To: "Iiams, Michele" <michele.iiams@email.und.edu>
Subject: Re: Permission to use another section of your Calculus II notes

You are welcome to do that!

Paul.

On Mon, Nov 7, 2016 at 6:25 PM, Iiams, Michele <michele.iiams@email.und.edu=
<mailto:michele.iiams@email.und.edu>> wrote:

Paul,


This is Michele Iiams from the University of North Dakota. We are in the fi=
nal phase of revising  the OER text for Calculus II and have discovered one=
 more missing piece. We need a section on Areas of Parametric Curves and Cy=
cloids. I am writing to ask permission to use your work at  http://tutorial=
.math.lamar.edu/Classes/CalcII/ParaArea.aspx to fill this need.


We are grateful for your previous consent to use integration and series sec=
tions. Once our work is complete we will send you a link to the finished pr=
oduct.


Michele


Michele Iiams
Mathematics Department
University of North Dakota

"Never trust atoms. They make up everything." AmericInn sign in Grand Forks=
, ND


--_000_ek6uw3t7hsit4k5e9q11wp9b1478611124428emailandroidcom_
Content-Type: text/html; charset="us-ascii"
Content-Transfer-Encoding: quoted-printable

<html>
<head>
<meta http-equiv=3D"Content-Type" content=3D"text/html; charset=3Dus-ascii"=
>
<meta content=3D"text/html; charset=3Dutf-8">
</head>
<body>
<div><br>
</div>
<div><br>
</div>
<div><br>
</div>
<div><br>
</div>
<div id=3D"composer_signature">
<div dir=3D"auto" style=3D"font-size:85%; color:#575757">Sent from my Veriz=
on, Samsung Galaxy smartphone</div>
</div>
<div><br>
</div>
<div><br>
</div>
<div>-------- Original message --------</div>
<div>From: Paul Dawkins &lt;pdawkins@gmail.com&gt; </div>
<div>Date: 11/8/16 7:14 AM (GMT-06:00) </div>
<div>To: &quot;Iiams, Michele&quot; &lt;michele.iiams@email.und.edu&gt; </d=
iv>
<div>Subject: Re: Permission to use another section of your Calculus II not=
es </div>
<div><br>
</div>
<div>
<div dir=3D"ltr">You are welcome to do that!
<div><br>
</div>
<div>Paul.</div>
</div>
<div class=3D"gmail_extra"><br>
<div class=3D"gmail_quote">On Mon, Nov 7, 2016 at 6:25 PM, Iiams, Michele <=
span dir=3D"ltr">
&lt;<a href=3D"mailto:michele.iiams@email.und.edu" target=3D"_blank">michel=
e.iiams@email.und.edu</a>&gt;</span> wrote:<br>
<blockquote class=3D"gmail_quote" style=3D"margin:0 0 0 .8ex; border-left:1=
px #ccc solid; padding-left:1ex">
<div dir=3D"ltr">
<div id=3D"m_8033978807228229300divtagdefaultwrapper" dir=3D"ltr" style=3D"=
font-size:12pt; color:#000000; font-family:Calibri,Arial,Helvetica,sans-ser=
if">
<p>Paul,</p>
<p><br>
</p>
<p>This is Michele Iiams from the University of North Dakota.&nbsp;We are i=
n the final phase of revising&nbsp;&nbsp;the OER text for Calculus II and h=
ave discovered one&nbsp;more missing piece. We need a section on Areas of P=
arametric Curves and Cycloids. I am writing to ask permission
 to use your work at&nbsp;&nbsp;<a href=3D"http://tutorial.math.lamar.edu/C=
lasses/CalcII/ParaArea.aspx" class=3D"m_8033978807228229300OWAAutoLink" id=
=3D"m_8033978807228229300LPlnk499613" target=3D"_blank">http://tutorial.mat=
h.<wbr>lamar.edu/Classes/CalcII/<wbr>ParaArea.aspx</a>&nbsp;to
 fill this need.</p>
<br>
<p></p>
<h2 style=3D"margin:12pt 0in 3pt; text-indent:0in; font-size:14pt; font-fam=
ily:Arial,sans-serif; font-style:italic; border:none; padding:0in">
<a name=3D"m_8033978807228229300__Toc170864981"><span style=3D"font-size:12=
pt; font-family:Cambria,serif"></span></a></h2>
We are grateful for your previous&nbsp;consent to use integration and serie=
s sections.&nbsp;Once our work is complete we&nbsp;will send you a link to =
the finished product.
<p></p>
<p><br>
</p>
<p>Michele</p>
<p><br>
</p>
<div id=3D"m_8033978807228229300Signature">
<div id=3D"m_8033978807228229300divtagdefaultwrapper" style=3D"font-size:12=
pt; color:#000000; background-color:#ffffff; font-family:Calibri,Arial,Helv=
etica,sans-serif">
<div><font face=3D"Tahoma" size=3D"2">Michele Iiams</font></div>
<div><font face=3D"tahoma" size=3D"2">Mathematics Department</font></div>
<div><font face=3D"tahoma" size=3D"2">University of North Dakota</font></di=
v>
<div><font face=3D"tahoma" size=3D"2"><br>
</font></div>
<div><font face=3D"tahoma" size=3D"2"><em>&quot;Never trust atoms. They mak=
e up everything.&quot; AmericInn sign in Grand Forks, ND</em></font></div>
</div>
</div>
</div>
</div>
</blockquote>
</div>
<br>
</div>
</div>
</body>
</html>

--_000_ek6uw3t7hsit4k5e9q11wp9b1478611124428emailandroidcom_--

\fi
